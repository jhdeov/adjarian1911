\part{Introductory chapters by Adjarian}







\chapter{Introduction by Adjarian}

\section{History of Armenian dialectology}
\begin{adjarianpage}\label{page:1}\end{adjarianpage}% should be 1

The first person who was occupied in Armenian dialectology was the Dutch Armenologist Schröder (\armenian{Շրէօտէր}). In his work (\citealt{Shroder-1711-TheasauresArmenian}: \textit{Thesaurus linguae armenicae}), he provides a singular article where he talks about various Armenian dialects in the Caucasus, and he provides a succinct sample of the Agulis and Julfa dialects. After him comes Jacques Chahan de Cirbied (\armenian{Շահան Ջրպետ}), an Armenologist from Evdokia. In his extensive grammar (\citealt{Cirbied-1823-GrammarArmenian}: \textit{Grammaire de langue arménienne}), he dedicates a part to Armenian dialects, about which he provides more information than his predecessor; and he provides a general sketch for a few of these dialects. Third place goes to the doctor Gevorg Akhverdian (\armenian{Գէորգ Ախվերդեան}; \translatorHD{SEA: /ɡevoɾkʰ ɑχveɾdjɑn/ or /ɑχveɾtʰjɑn/}). He had such a special love for Armenian dialectology that he published the provincial songs of Sayat Nova in the Tbilisi dialect \citep{SayatNovaAdjarian}, providing an exhaustive introduction, where he studies the Tbilisi dialect with a very skilled and specialist hand. Akhverdian intended to study other dialects as well, but he died too soon, squashing his beautiful prospects. 

In 1866, the grammar of the Viennese monk Arsen Aydinian (\armenian{Հ. Ա. Այտընեան}; \translatorHD{SWA: /ɑɾsen ɑjdinjɑn/}) was published \citep{Aydnian-1867-GrammarArmenian}, wherein the author gave the first general classification of the Armenian dialects, though in very uncertain terms. Aydinian recognized four dialectal branches. 




\begin{adjarianpage}\label{page:2}\end{adjarianpage}% should be 2

\begin{enumerate}[noitemsep]
 \item Russia, Persia, and India
\item Turkish Armenia and Mesopotamia
\item Asia Minor
\item Austria-Hungary (Transylvania, Artyal or \armenian{Առտեալ})

\end{enumerate}

The author talks about each of the four branches, provides a general description for each, focusing especially on the morphology. He does not consider the phonology. 

In the same year, Kerovbe Patkanian (\armenian{Պատկանեան}; \translatorHD{SEA: /kʰeɾovbe pɑtkɑnjɑn/}) published a grammar in German on the Agulis dialect \citep{Patkanoff-1866-Agulis}. This was followed by the Armenologist Petermann's study on the Tbilisi dialect \citep{Petermann-1867-Tiflis}.

In 1869, Patkanian published his Russian work \citep{Patkanian-1869-RussianDialects}\footnote{\translatorHD{In the original source, Adjarian used the pre-revolution writing system and called the book: Изслҍдованіе о діалектахъ армянскаго языка.}}, where he gives a succinct description of eight Armenian dialects, based on a handful of written sources. This was followed in 1878 by the same author who published \citeauthor{Patkanian-1875-RussianDialects}.\footnote{\translatorHD{In the original source, Adjarian used the pre-revolution writing system and called the book: Матеріалы для изученія армянскихъ нарҍчій.}} in two volumes: the first volume on New Nakhichevan, and the second on Mush. Both works are extensive.

After Patkanian, there was no work on Armenian dialectology for a long time, until in 1883 when the Agulis linguist S. Sargsiants (\armenian{Ս. Սարգսեանց}; \translatorHD{SEA: /sɑɾkʰəsjɑnt͡s/}) published his detailed study on the Agulis dialect \citet{Sargiants-1883-Agulis}. His work exceeds all previous works, both in its extensiveness and its scientific correctness. In 1886, the Polish Armenologist Jan Hanusz (\armenian{Յովհ. Հանուշ}) started a study on the Polish-Armenian dialect. He published two volumes, wherein he studies the lexicon and morphology of Polish-Armenian.\footnote{\translatorHD{Adjarian is referring to a collection by Hanusz, available at the \href{https://polona.pl/search/?query=Jan_Hanusz&filters=creator:\%22Hanusz,\_Jan\_(1858--1887)\%22,public:1,hasTextContent:0}{Polono digital library}. }} Because of his death, the remaining parts of his study of the dialect were unfinished. 

After Hanusz, the Russian Armenologist Alexander Thomson (Томсонъ; \translatorHD{modern Russian: Александр Томсон}) published a study on the Akhaltskha dialect in 1887 \citep{Thomson-1887-Karin} and the Tbilisi dialect in 1890 \citep{Thomson-1890-Tiflis}. 

Starting in 1896, Armenian dialectology reached a new momentum, and the number of studies grew day by day. In the same year, the \textit{Azgagrakan Handes} (\citeauthor{AzgagrakanHandes}) was established, under the editorship of Lalayan (\armenian{Լալայեան}; \translatorHD{SEA: /lɑlɑjɑn/}). To this day, the journal continues to provide... 

 

\begin{adjarianpage}\label{page:3}\end{adjarianpage}% should be 3

... many samples on the provincial or vernacular language. Then in 1896, the Armenologist Melik S. Davit-Bek (\armenian{Մէլիք Ս. Դաւիթ-Բէգ}; \translatorHD{SWA: /melikʰ tʰɑvitʰ pʰekʰ/, SEA /dɑvitʰ beɡ/}) published a succinct study on the Marash dialect, first in Armenian (see \armenian{ՀԱ} 1986) and then in French.\footnote{\translatorHD{Adjarian spells this person's name in various ways. I couldn't find this person's work on Marash. The closest record I found was from a \href{https://hy.wikisource.org/wiki/\armenian{Էջ}:\armenian{Հայկական}_\armenian{Սովետական}_\armenian{Հանրագիտարան}_(Soviet_Armenian_Encyclopedia)_7.djvu/298}{Soviet encyclopedia}, but sadly the encyclopedia page doesn't clarify what \armenian{ՀԱ} stands for (probably \citeauthor{HandesAmsorya} or another old Armenian periodical). } } In 1897, Levon Mserian (\armenian{Լ. Մսերեան}; \translatorHD{SWA/SEA: /levon məseɾjɑn/, also called Mseriants \armenian{Մսերեանց} /məsəɾjɑnt͡sʰ/}) published a detailed study on the Mush dialect.\footnote{\translatorHD{It seems that Adjarian is referring to \citep{Mserianz-1899-Mush}. }} In 1898, 1899, and 1901,\footnote{\translatorHD{Adjarian is likely referring to this series: \citealt{Mseriants-1897-Part1,Mseriants-1901-Part2}.}} he published his dialectological series on the dialects of Aslanbeg, Suceava, Karabakh, and Van; the latter is in German. Then in 1898, the \citeauthor{Byurakn} ethnographic periodical is established in Istanbul, which continued to exist for three years (1898-1900), providing very many samples on the Armenian dialects of various provinces, many of which were unheard of by then. If Lalayan's Ethnographic Journal had taken Byurakn's trajectory, then Armenian dialectology would have currently been in an envious position. The existence of Byurakn sadly did not last long, showing that we Armenians still do not have the capacity to revive scientific journals. 

In 1899, H. Kazandjian (\armenian{Յ}. \citealt{KazandjianBook}) published ``\textit{The provincial language of Evdokia}'' (\armenian{Եւդոկիոյ Հայոց գաւառաբարբառը}) in the journal \textit{Handes Amsorya} (\citeauthor{HandesAmsorya}). Starting in 1900, the same journal starts publishing the extensive study of the Arapgir dialect by Melik S. Davit-Bek (\armenian{Մէլիք Ս. Դաւիթ-բէկ}). But unfortunately after a few years, the study halts. 

Special attention should be given to the editorship of the \textit{Eminian Azgagrakan Joghovatsou} (\citeauthor{Eminian}), that was established in the Lazarian Institute in Moscow (\armenian{Մոսկուայի Լազարեան Ճեմարան}) by the will of deceased and skilled Armenologist M. Emin (\armenian{Մ. Էմին}). To this day, the journal has provided the most respectable volumes of all the dialectological works published so far. By now, it has published 7 books, of which five are completely dedicated to Armenian dialects. 

Among the less known workers in Armenian dialectology is H. Nazariants (\armenian{Յ. Նազարեանց}; \translatorHD{SEA/SWA: /nɑzɑɾjɑnt͡sʰ/}). In the journal \textit{Ports} (year 5, number 2, page 150-164, \armenian{Փորձ հանդէս}; \translatorHD{SEA: /pʰoɾt͡sʰ hɑndes/}), he published an article called ``\textit{About the Armenian dialects}'' (\armenian{Հայոց բարբառների մասին}).\footnote{\translatorHD{I think this journal is the one listed on \href{https://hy.wikipedia.org/wiki/\armenian{Փորձ}_(\armenian{ամսագիր})}{Wikipedia}. Page 90 of this archived \href{https://arar.sci.am/dlibra/publication/286700/edition/263242/content}{index} lists the article by Nazariants.}} He proposed five sections, and called on Armenian folklorists to translate them to the dialect of each village and send them to him.

\begin{adjarianpage}\label{page:4}\end{adjarianpage}% should be 4

The man of course will of course be waiting for a long time for a hopeless endeavor. 

Let us remember lastly the German Armenologist Karst, who in his grammar on Cilician (\citealt{Karst-1901-MiddleArmenain}: \textit{Historische Grammatik des Kilikische-Armenischen}) talks about the Middle period, specifically on Cilician Armenian. But whenever he explains the various words and forms, he always compares the data against the current Armenian dialects. 

\section{Disadvantages of dialectological studies}

As can be seen, Armenian dialectology is still not rich. The number of foundational and complete studies is not high. Of the published sources, some are lacking in their phonology, some in their morphology. And when it comes to the transcription of words, they generally lack scientific accuracy. For example, see what Melik S. Davit-Bek (\armenian{Մէլիք Ս. Դաւիթ-Բէգ}) says in his Arabkir grammar (\citeauthor{DavitBekArabgir}: \armenian{ՀԱ} 1901, page 39): 

\begin{quote}
 By the term `usual pronunciation', we mean the Armenian pronunciation, whether it's from Yerevan, Tbilisi, Karabakh, or Van, Mush, Karin, Diyarbakır, or the Arapgir pronunciation. That is, we don't accept the so-called Western and Eastern pronunciations. Having studied a large number of the dialects of provinces of Van, Mush, Karin, Kharberd, Sebastia, and Diyarbakır, we have seen that there is no reason to accept such a decisive abyss. The main reason is that the actual populace is the only one that is entrusted with the provincial dialects and the pronunciation, whether in the Araratian provinces or in Lesser Armenia; there is one and only one pronunciation.''\footnote{\translatorHD{It's hard to translate the last few sentences because I couldn't make sense of the original; it was too convoluted.}}
\end{quote}


And this person is a linguist.

S. Sarsgian (\armenian{Ս. Սարգսեան}) was also a linguist, whose dialect of Agulis is a choice work. But see what he also says (Part B, page 111):

\begin{quote}
``In the Agulis dialect, the sound /k/ <\armenian{կ}> is pronounced as hard (\armenian{կոշտ}) in nearly all positions, whether before a vowel or after. But if before the /k/ <\armenian{կ}> there is /i/ <\armenian{ի}> or /e/ <\armenian{է}> or such an /ɑ/ <\armenian{ա}> (\armenian{Ցղ. է})\footnote{\translatorHD{I don't understand the abbreviation that he uses.}}, such that the literary form of the word uses /i/ <\armenian{ի}>, then that /k/ <\armenian{կ}> is pronounced as soft (\armenian{կակուղ}) /kʲ/ <\armenian{կյ}>. '' 
\end{quote}


But sometimes it happens that a person is confused on how to read such a form. 

\begin{adjarianpage}\label{page:5}\end{adjarianpage}% should be 5

For example on page 39, line 6, it's written /menɑk/ <\armenian{մենակ}> `alone'. On the contrary, the literary form of this word has two options: /minak/ <\armenian{մինակ}> or /menɑk/ <\armenian{մենակ}>. So how should we read this word: /mænæk/ <\armenian{մա̈նա̈կ}> or /mænækʲ/ <\armenian{մա̈նա̈կյ}>? Let's then be grateful for the French and English orthographies. 

It is apparent that in Armenian dialectological studies, even the best are considered incomplete and deficient



\section{Program for dialectological studies}

In order to have a perfect study in Armenian dialectology, it should contain the following components, besides having brief geographical and statistical information on the studied dialect

\textit{Component A} - Phonetics (\armenian{Ձայնախօսութիւն}, German: \textit{Phonetik}): This section establishes the sound system of the dialect, meaning what sounds are found in the dialect, the way these sounds are articulated, their uses and number, their origins from either Armenian or from other sounds.

\textit{Component B} - Phonology (\armenian{Ձայնաբանութիւն}, German: \textit{Lautlehre}): This section provides the rules for sounds in the dialect. One by one, it goes through the Armenian vowels, diphthongs, and consonants; these are compared against the dialect. It establishes what Armenian letters or sounds underwent what sound changes in this dialect. Because phonology is the most important branch of linguistics, it is thus necessary that this chapter is detailed, correct, and extensive. Each Armenian sound must be examined on its own, meaning word-initially, word-medially, or word-finally, whether alone or next to a vowel or consonant. Furthermore, the provided examples must be complete, so that we can decide well the strength of the rule and the number of exceptions. 

\textit{Component C} - Morphology (\armenian{Ձեւախօսութիւն}, German: \textit{Morphologie}): This is the grammar by the conventional sense of the word. Or more accurately, this is called the etymology of the grammar. In this section, it is necessary to give a detailed examination on the dialect's declensions, conjugations, pronouns, their form alternations, and so on. 

\textit{Component D} - Syntax (\armenian{Համաձայնութիւն}\footnote{\translatorHD{The Armenian term literally means `agreement'. But Adjarian is using it in the sense of `general syntax'.}}): This is... 

\begin{adjarianpage}\label{page:6}\end{adjarianpage}% should be 6


... an inseparable part of a grammar, which is necessary for every language. But because our dialects have not deviated from the usual agreement of literary language, it's not important to focus heavily on this part. 

At the end of every dialectal study, there must be an extensive text sample of the dialect. The text sample shows the syntactic and agreement rules of the dialect, and the use of the rules for the above components. It would be good if the manuscript used conversational data. With this, we can see how a verb is used in different tenses, numbers, and persons. 

Such are the required components for a dialectal study, so that a study is considered complete. It is also necessary to examine the circumstances with which the work can be scientifically established as correct and complete. 

\section{Scientific alphabet}\label{sec:IntroAdjarian:scientificAlphabet}

In a scientific field, the first thing that we need is a scientific alphabet. This is an alphabet that we can use to show nuances of all the sounds of the studied dialect. For this goal, European linguistics have devised many and diverse letters for different purposes, such as symbols for lengthening or shortening, open or closed pronunciation, monophthong or diphthong, stressed or unstressed, simple or nasal, voiced or voiceless, aspirated or unaspirated, and so on. it would be good of course if dialectology took these European symbols. But our nationalist zeal, the poverty of our publishers, and primarily the untrained eye don't allow the use of this desired point. Thus, we need a scientific alphabet that uses the Armenian letters.

Our focus is of course about a scientific alphabet, and it has no link at all with literary and current language orthographies. 

For a scientific alphabet, the required conditions are as follows:
\begin{enumerate}
 \item Each sound must be symbolized by only one symbol
\item Every symbol must have only one sound. 
\end{enumerate}


For example, the /t͡ʃʰ/ <\armenian{չ}> sound is a single sound. Therefore, writing it with two or more letters (Eng. <ch>, French <tch>, ... 


\begin{adjarianpage}\label{page:7}\end{adjarianpage}% should be 7


... German <tsch>) is against the first condition. The sound /o/ <\armenian{օ}> is a single sound too. Writing this sound as <o>, <au>, or <eau> is against the first condition as well. As another example, consider the Armenian letter <\armenian{յ}>. Word-initially, this letter pronounced as /h/ <\armenian{հ}>, medially as /j/, and word-finally it's unspoken. This violates the second condition. The scientific alphabet requires that /t͡ʃʰ/ <\armenian{չ}> is written as one symbol (such as in Armenian or in the European scientific alphabet <c> character), the sound /o/ <\armenian{օ}> as just /o/ <\armenian{օ}>, and the <\armenian{յ}> letter has only one pronunciation (for us /j/), and so on. 

It speaks for itself that an un-read letter should not exist. 

Now, the 38 letters of Armenian cannot symbolize all dialectal sounds. This is especially because these letters have different pronunciations based on their position in the word, in both the Eastern and Western pronunciation systems. Thus, we need to establish once and for all what sounds they symbolize. This decision must be decisive and stable, used in all places and books. 

It is scientifically established that among the two literary languages, it is the Eastern form where the Armenian letters preserve the pronunciations of the fifth century, and the pronunciations agree with the transliteration or transcription of foreign words in Old Armenian. Because of this, our plosive letters should be based on the Eastern pronunciation (Table \ref{tab:intro:stopsaffr}).\footnote{\translatorHD{I suspect that in his transcriptions, Adjarian forgot to distinguish the two affricate series, and to distinguish aspiration. It could also have been a typo.}} 

\begin{table}[H]
 \centering
 \caption{Adjarian's transcriptions for Armenian sounds that vary between Standard Western and Standard Eastern Armenian}\label{tab:intro:stopsaffr} 
 \begin{tabular}{|lll|lll|lll| }
 \hline \multicolumn{3}{|l|}{Armenian letter} & \multicolumn{3}{l|}{Adjarian's transcription} & \multicolumn{3}{l|}{IPA letter} 
 \\ \hline 
 \armenian{բ} & \armenian{պ} & \armenian{փ} & b & p & p' & b & p & pʰ
 \\
 \armenian{գ} & \armenian{կ} & \armenian{ք} & g & k & k' & ɡ & k & kʰ
 \\
 \armenian{դ} & \armenian{տ} & \armenian{թ} & d & t & t' & d & t & tʰ
\\
\armenian{ձ} & \armenian{ծ} & \armenian{ց} & j & c & c & d͡z & t͡s & 
 t͡sʰ \\
\armenian{ջ} &\armenian{ճ} &\armenian{չ} & j & c& c & d͡ʒ & t͡ʃ & 
 t͡ʃʰ
\\ \hline 
 \end{tabular}
 
\end{table}

Among these, the first column is in the location of the second column for Western. The second column's sounds do not exist in Western.\footnote{\translatorHD{To clarify, he means that the letters from the first column are voiceless aspirates in Western Armenian; and that Western Armenian doesn't have voiceless unaspirates.}} This is hard to explain why. We should emphasize that the Western reader is not deceived by the equivalent transcribed /p, k, t/ sounds and analogous sounds. These European sounds are pronounced stronger than in Armenian, such that the Western Armenian perceives these sounds as /pʰ, kʰ, tʰ, t͡sʰ, t͡ʃʰ/ <\armenian{փ, ք, թ, ց, չ}>. However, it is not only strength... 

\begin{adjarianpage}\label{page:8}\end{adjarianpage}% should be 8

... that is perceived, but it is also the absence of voicing (the voicelessness of the sound), what is the equivalent to both the Armenian and the European. 

The following letters in Table \ref{tab:intro:othersound} are pronounced the same in Eastern and Western. Thus we don't need to give them special attention. 

\begin{table}[H]
 \centering
 \caption{Adjarian's transcriptions for Armenian sounds that don't vary between Standard Western and Standard Eastern Armenian}\label{tab:intro:othersound}
 \begin{tabular}{|lll ll|lllll|lllll| }
 \hline \multicolumn{5}{|l|}{Armenian letter} & \multicolumn{5}{l|}{Adjarian's transcription} & \multicolumn{5}{l|}{HD: IPA letter} 
 \\ \hline 
 \armenian{ա} & \armenian{է}& \armenian{ը}& \armenian{ի}& \armenian{օ} & 
 a & e & ə & i& o & 
 ɑ & e & ə & i& o 
\\
\armenian{ֆ} & \armenian{վ}& \armenian{ս}& \armenian{զ}& \armenian{շ} & 
f & v & s & z & ṡ & 
f & v& s& z& ʃ
\\
\armenian{ժ} & \armenian{խ} & \armenian{ղ} & \armenian{հ} & &
ż & x & ġ & h & & 
ʒ & χ & ʁ & h & 
\\ 
\armenian{լ} & \armenian{մ}& \armenian{ն}& \armenian{ր} & \armenian{ռ} & 
l & m & n & r & ṙ & 
l & m & n & ɾ & r
\\ \hline 
 \end{tabular}
 
\end{table}

However for the letters <\armenian{ե, յ, ո, ւ}>, they have a complicated situation. The letter <\armenian{ե}> is pronounced as /je/ word-initially, as /e/ word-medially, and it is not found word-finally. But because /je/ is a mixture of sounds, we cannot use one symbol to symbolize it. Similarly, the sound /e/ is already symbolized by the letter <\armenian{է}.> We don't need to also represent /e/ by <\armenian{ե}>. Meaning that the letter <\armenian{ե}> is additional. Thus we should transcribe such vowels as in Table \ref{tab:frontMid}. 

\begin{table}[H]
\centering
\caption{Transcribing front mid vowels}\label{tab:frontMid}
\begin{tabular}{|lll|l l| }
\hline 
& Trad. ortho. & Ref. ortho. & \multicolumn{2}{l|}{Transcription (SEA)}
\\\hline 
`yesterday' (dialectal) & \armenian{երէկ} & \armenian{երեկ} & \armenian{յէրէկ} & /jeɾek/ \\
`yesterday' (standard) & \armenian{էրէկ} & \armenian{էրեկ} & \armenian{էրէկ} & /eɾek/ 
\\ \hline 
\end{tabular}
\end{table}


As we said, the letter <\armenian{յ}> is pronounced as /h/ word-initially, as /j/ word-medially, and as un-pronounced word-finally. Such diversity is contrary to a scientific alphabet. For the sound /h/, we already have the letter <\armenian{հ}>. We don't need to use the letter <\armenian{յ}> for the same sound. Second, if some letter is unpronounced, then it's unneeded. Once these two situations are removed, the <\armenian{յ}> letter ends up having only one sound /j/. And thus we read this letter in this way in both the beginning, middle, and end of words (Table \ref{tab:intr:scientificAlphabet:glideTranscribe}). The use of the letter <\armenian{յ}> for only this sound was also the situation in the fifth century. 


\begin{table}[H]
\centering
\caption{Transcribing the glide /j/}\label{tab:intr:scientificAlphabet:glideTranscribe}
\begin{tabular}{|lll|l l| }
\hline 
& Trad. ortho. & Ref. ortho. & \multicolumn{2}{l|}{Transcription (SEA)}
\\\hline 
N/A & & & \armenian{յիս} & /jis/ \\
`I' & \armenian{ես} & \armenian{ես} &\armenian{յէս} & /jes/ \\
`Armenian' & \armenian{հայ} & \armenian{հայ}& \armenian{հայ} & /hɑj/ \\
N/A& && \armenian{ալայ} & /ɑlɑj/
\\ \hline 
\end{tabular}
\end{table}


The letter <\armenian{ո}> is pronounced as /vo/ word-initially, /o/ word-medially, and is not found word-finally. Because /vo/ is a doubled sound, we shouldn't write it with one letter. For the sound /o/, we already have the letter <\armenian{օ}>, ... 
\begin{adjarianpage}\label{page:9}\end{adjarianpage}% should be 9

... thus the letter <\armenian{ո}> is excessive. The following words are written as in Table \ref{tab:intr:scientificAlphabet:vo}. 



\begin{table}[H]
\centering
\caption{Transcribing front mid vowels}\label{tab:intr:scientificAlphabet:vo}
\begin{tabular}{|lll|l l| }
\hline 
& Trad. ortho. & Ref. ortho. & \multicolumn{2}{l|}{Transcription (SEA)}
\\\hline 
N/A & & & \armenian{օսկի} & /oski/ \\
`gold' & \armenian{ոսկի} & \armenian{ոսկի} &\armenian{վօսկի} & /voski/ \\
\hline 
\end{tabular}
\end{table}


The letter <\armenian{ւ}> is read rightly as /v/. But the difference is that this this letter cannot start a word. We can't thus transcribe the word <\armenian{վրայ}> /vɾɑ/ `on' as <\armenian{ւրա}>. The use of the letter <\armenian{ւ}> is unneeded or excessive for the following reasons. We can't have one sound correspond to two symbols. We can't write the letter <\armenian{ւ}> word-initially. And the letter <\armenian{ւ}> is also used in the diphthongs \armenian{ու, իւ} /u, ʏ/.\footnote{\translatorHD{It is more accurate to use the term `digraph' instead of `diphthong' here, but the term `diphthong' is more faithful to Adjarian's original word \armenian{երկբարբառ} `diphthong'. }} Thus, we must write as in Table \ref{tab:intr:scientificAlphabet:v}, and not with the traditional orthography. 



\begin{table}[H]
\centering
\caption{Transcribing the letter <\armenian{ւ}> }\label{tab:intr:scientificAlphabet:v}
\begin{tabular}{|lll|l l| }
\hline 
& Trad. ortho. & Ref. ortho. & \multicolumn{2}{l|}{Transcription (SEA)}
\\\hline 
`bird (CA); chicken (SEA)' & \armenian{հաւ} &\armenian{հավ} & \armenian{հավ} & /hɑv/ \\
& & & not <\armenian{հաւ}> & \\
`king' & \armenian{թագաւոր} & \armenian{թագավոր} &\armenian{թաքավօր} & /tʰɑkʰɑvoɾ/ \\
& & & not <\armenian{թագաւոր}> & \\
`pain stone' & \armenian{ցաւաքար} & \armenian{ցավաքար} &\armenian{ցավաքար} & /t͡sʰɑvɑkʰɑɾ/ \\
& & & not <\armenian{ցաւագար}> & \\
\hline 
\end{tabular}
\end{table}




In this we establish the the scientific value of the Armenian alphabet.

However, our dialects have sounds that the Armenian alphabet cannot explain, and for these sounds we need to create new symbols. 

We creating new symbols, we must consider two circumstances:
\begin{enumerate}
\item Publication appropriateness, meaning we should create simple letters that aren't far off from aesthetics and which are appropriate to the style of Armenian drawing. 
\item The created letters should by themselves remind us what the sound is. In other words, we shouldn't create entirely novel forms, but the form should have some symbol or other formal marking that distinguishes it.
\end{enumerate}


Within Armenian dialects, the most commonly found sounds are the following.\footnote{\translatorHD{Adjarian does not use any special diacritics to denote diphthongs like the symbol ̰ -- such markings are our own. }}

/æ/: This sound is between /ɑ/ <\armenian{ա}> and /e/ <\armenian{է}>, such as in the Karabakh word /bɑn/ <\armenian{բան}> `thing'. This sound is transcribed by Sargsian (\armenian{Սարգսեան}) as an <\armenian{ա}> with two dots above it, by S. Melik Davit-Bek (\armenian{Ս. Մէլիք-Դաւիթ բէգ}) as <\armenian{ա}> with a circle on top. Both of these strategies are inappropriate. First, the fewer such markings are used, the better. Second, using this strategy makes it necessary to create a new letter. Third, experience has shown these these symbols are hard to keep on our letters; and because of their thinness, they break quickly. Fourth, when it's necessary to add stress on the sound, we end up using two or three markings next to each other. Because of these reasons, I consider the most appropriate strategy is to use a rotated <\armenian{ա}>... 

\begin{adjarianpage}\label{page:10}\end{adjarianpage}% should be 10

... This symbol was first thought of and used by the Protestant missionaries. The appropriateness of this letter is that doesn't have markings, we don't need to create a new letter, and we can add on it stress symbols.\footnote{\translatorHD{Ironically, it seems that Armenian dialectology prefers to mark the fronted vowel /æ/ using <\armenian{ա}> with two dots <\armenian{ա}̈>. This is the system that Adjarian argued against. And unfortunately, Adjarian's upside-down <\armenian{ա}> symbol was only recently given a Unicode symbol <\armeniang{\armenian{ՠ}}>, and a person has to actively download the right font so that they can even display this letter. Because of these issues, I have chosen to write Adjarian's upside-down <\armeniang{\armenian{ՠ}}> as <\armenian{ա}̈>, while I follow the IPA in using /æ/.
}}

/i̯e/: This sound is used in Mush, Van, Karin. We can consider this sound as a fast pronunciation of the a sequence /ie/ <\armenian{իէ}>. This word is found for example in the words /mi̯eɾ/ <\armenian{մեր}> `our', /d͡zi̯eɾ/ <\armenian{ձեր}> `you.{\gen}.{\pl}'.\footnote{\translatorHD{Adjarian doesn't state the relevant dialect. }} For this sound, we think it's appropriate to use the letter <\armenian{ե}>, because this sound's real source is represented and we won't need to invent a new sound.

/u̯o/: This sound is used in the same provinces. It is a fast pronunciation of a sequence /uo/ <\armenian{ուօ}>, such as in /su̯oχ/ \armenian{սոխ} `onion', /ɡu̯oʁ/ \armenian{գող} `thief'.\footnote{\translatorHD{Adjarian doesn't state the relevant dialect. }} We represent this sound with the letter <\armenian{ո}> for the same reasons above. 

/bʰ, ɡʰ, dʰ, d͡zʰ, d͡ʒʰ/:\footnote{\translatorHD{Adjarian uses the European-based transcriptions <bh, gh, dh, jh, jh>. }} These sounds are found in many Armenian dialects as we shall later see more extensively. These are aspirated forms of the sounds /b, ɡ, d, d͡zʰ, d͡ʒʰ/ <\armenian{բ,դ,գ,ձ,ջ}>. To represent these sounds, the most appropriate wait is to add a reverse apostrophe: <\armenian{բ}‘, \armenian{դ}‘, \armenian{գ}‘, \armenian{ձ}‘, \armenian{ջ}‘>. 
 
/ɦ/:\footnote{\translatorHD{Adjarian uses an apostrophe-like symbol <\armenian{՚}>. }} This is a long glottal sound\footnote{\translatorHD{Adjarian uses the word \armenian{հագագ} which dictionaries translate as `uvular', but the definition of this word is more in line with a glottal articulation.}} The Armenians of Karin, Mush, Alashkert, and other places use this sound for the word-initial letter <\armenian{յ}>, such as in the pronunciation of the name /hɑkopʰ/ <\armenian{Յակոբ}> or /hɑɾutʰjun/ <\armenian{Յարութիւն}>. Because this sound is a type of /h/ <\armenian{յ}> sound, it is appropriate to use the symbol <\armenian{յ}̵> (the letter <\armenian{յ}> with a line through it). Although this is a new symbol, it does need new molding because it looks like the Latin letter <f> but upside-down.

The only sound that we must inevitable mold is the small <\armenian{յ}> symbol. This has a wide use. It will be used to form the diphthongs /ɑi̯, ei̯, ii̯, oi̯/ <\armenian{այ, էյ, իյ, օյ}>, and to form palatalized sounds /ɡʲ, kʲ, kʰʲ, hʲ/ <\armenian{գյ, կյ, քյ, հյ}>.\footnote{\translatorHD{Adjarian implies that he wants this symbol <\armenian{յ}> to be a subscript. But the printed editions don't show a subscript form. It's possible that reprints of his work couldn't do a subscript <\armenian{յ}>. I don't do any subscript notation because I ultimately don't know what exact sounds he wants. }}

In a few dialects, the semivowel /w/ is found. Austrian Armenians have the diphthongs /ɑu̯, ou̯, eu̯/ and the triphthong /i̯eu̯/.\footnote{\translatorHD{Adjarian incorrectly calls /i͜e͜u/ a diphthong, perhaps because he didn't know of an Armenian translation for the word `triphthong'.}} To represent all of these, we must use the letter <\armenian{ւ}>, such that <\armenian{ւա}> = /wɑ/ or /u̯ɑ/, <\armenian{աւ}> = /ɑu̯/, <\armenian{օւ}> = /ou̯/, <\armenian{էւ}> = /eu̯/, <\armenian{եւ}> = /i̯eu̯/, and so on. 

The least inappropriate forms are <\armenian{էօ, իւ, ու}> /œ, ʏ, u/.\footnote{\translatorHD{Adjarian uses umlauted symbols <ö, ü>. But I use the conventional IPA symbols instead.}} For these sounds, we could have proposed united <\armenian{էօ, իւ, ու}> forms, ... 


\begin{adjarianpage}\label{page:11}\end{adjarianpage}% should be 11

... But because these would be impractical, we are forced to continue the old style for now. 

Besides the above words, there are a few other rare sounds that we will see later. 

\section{Methods of studying the dialects}

There are four manners to study or investigate a dialect:
\begin{enumerate}
 \item The investigator is a local and thus knows the dialect as their mother tongue and then studies it. 
\item The investigator is a foreigner and studies the dialect within the dialect's location. 
\item The investigator studies the dialect but in a foreign location (not the dialect's location) by working with a person or persons who speak that dialect as a mother tongue. 
\item The investigator does their research from written sources. 

\end{enumerate}

The first manner is the most desired manner. The second is almost as good, the third less good, and the fourth doesn't need anything, especially if the writer doesn't know about scientific orthographies.\footnote{\translatorHD{The original Armenian is <\armenian{իսկ չորրորդը բանի մը պէտք չէ, եթէ մանաւանդ գրողի գիտական ուղղագրութեան տեղեակ չէ}>. The first clause can mean either a compliment ``the fourth one doesn't need anything' where the phrase <\armenian{բանի մը}> `thing-{\gen} {\indf}'' is the direct object of the sentence. But the clause can also be an insult ``the fourth one is not needed by anything'' where <\armenian{պէտք}> `need' is the direct object. The following clause implies negativity, but it's unclear in total.}} 

However in every case, it's also necessary that the investigator is familiar with linguistic science and is experienced. 

How should we conduct a study of a dialect?

The most primary thing is the dictionary. Every dialect consists of the following three elements:
\begin{enumerate}
 \item Native words: These are words which descend from Old Armenian, such as <\armenian{ջուր}> /d͡ʒuɾ/ `water', <\armenian{հաց}> /hat͡sʰ/ `bread', <\armenian{գինի}> /ɡini/ `wine>.
\item Provincial words: These are Armenian words that are absent from Classical Armenian, and are often newly formed words. For example <\armenian{ականակոյր}> /ɑkɑnɑkujɾ/ `very dark darkness', <\armenian{քաջքոտ}> /kʰɑd͡ʒkʰod/ `possessed by a demon', <\armenian{հրուկ}> /həɾuɡ/ `piece of soap', and so on. 
\item Foreign words: These are words that were borrowed from many other languages, such as <\armenian{սամավար}> /samavaɾ/ `samovar', <\armenian{յօրղան}> /joɾɣan/ `blanket'.
\end{enumerate}

To study a dialect, the first group is the most important. There is a fixed line to find the phonetic laws. \footnote{\translatorHD{It seems Adjarian is using some idiom <\armenian{հաստատուն եզր}>, literally `fixed or stable line/edge/border'.}} 



\begin{adjarianpage}\label{page:12}\end{adjarianpage}% should be 12

The investigator must design a complete collection of these words. And to do this, the only way is to take an Armenian dictionary; and against each word, find the dialectal form, alongside its declension or conjugation system. In other words, we must design a dialectal Armenian dictionary. From the Armenian words, tens of thousands are lost in the dialect, so this work doesn't seem terrifying. However, we confess that this is nevertheless hard and burdensome. However, this is the only way. And the investigator will be comforted in knowing that when they are organizing the phonetics, phonology, and morphology, they will produce a complete work. This is because the investigator will be able to show us all the phonetic rules, all their examples and explanations, and also all the grammatical rules, their exceptions, and so on. 

Moving on from this general glance and program for dialectology, let us move on to detailing the present work. 

\section{Structure of the present work}

The goal of this present work is the general classification of Armenian dialects. We set the number of all the Armenian dialects as 31\footnote{\translatorHD{It's not completely clear if this number 31 is about the dialects that are studied in this book (which is 31), or the total number of dialects which Adjarian acknowledges as existing.}}, some of which have sub-dialects. We also considered it important to provide a sample text for every dialect and sub-dialect, to show the linguistic state of the dialect in practice. The samples that I personally collected are in the scientific orthography. As for the samples that I took from other sources, they generally don't have scientific accuracy. About this topic, I provide a note below each sample. 

Given the situation, such that many of the 31 dialects are still unfamiliar to science, or when they are only available from an insignificant manuscript, such a work is still premature. However, for this issue I have benefited from my own original works. 

In 1892, I started doing dialectological research for the first time. I organized first a succinct grammar of the Istanbul dialect. In 1898, I published... 
\begin{adjarianpage}\label{page:13}\end{adjarianpage}% should be 13


... a small volume on the Aslanbeg dialect, by working with a friend from Aslanbeg, Mr. Aleksan Nalpandian (\armenian{Ալէքսան Նալպանդեան}; \translatorHD{SEA: /ɑlekʰsɑn nɑlpɑndjɑn/, SWA /nɑlbɑntʰjɑn/}). This manuscript, as a first attempt, had its weaknesses when compared against the above points of our program. However, in the \citeauthor{Byurakn} newspaper (1900, page 609-613), a certain Mushegh Varg (\armenian{Մուշեղ Վարգ}; \translatorHD{SEA: /muʃeʁ vɑɾɡ/ or /vɑɾkʰ/}), criticized that work, and found errors from page to page. I found it unnecessary to respond to his uncivil behavior, not only because I found his improper style unbecoming, but also he confuses phonetics with phonology, doesn't know the what's an open /e/ <\armenian{է}> or a closed /e/ <\armenian{է}>, and he seemed devoid of linguistic understanding. On this, I received a paper from Aslanbeg that said that a group of men were preparing to publicly condemn my study. However, being scared of their influential position, they are obliged to be sufficed with the same letter. 

After studying Aslanbeg, I started publishing a study on the Suceava dialect in the Venetian newspaper \citeauthor{Bazmaveb} (\translatorHD{SEA: /bɑzmɑvep/, SWA: /pʰɑzmɑveb/}). I had prepared that study by working with a priest from Suceava named Father Karapet Kaynayian (\armenian{Տէր Կարապետ Կայայեան}; \translatorHD{SEA: /teɾ kɑɾɑpet kɑjɑjɑn/, SWA: /deɾ ɡɑɾɑbed ɡɑjɑjɑn/}). However, the numerous typographical errors and the lack of printing caused the cancellation of my publication, and the work was left half-done.

My third study was on the Karabakh dialect, which I prepared with archimandrite Rev. Khachik Dadian (\armenian{Արժ. Խաչիկ Վրդ. Դադեան}; \translatorHD{SEA: /χɑt͡ʃʰik dɑdjɑn/, SWA: /tʰɑtʰjɑn/}), the honorable deacon of M. Babayian who was a deacon of Yesa, now archimandrite Zaven (\armenian{Շնորհ. Եսայի սակ. Մ. Բաբայեան (այժմ Զաւէն վրդ}.); \translatorHD{SEA: /zɑven/ and /bɑbɑjɑn/, SWA: /pʰɑpʰɑjɑn/}), and Mr. Avetis Ter Harutyoun (\armenian{Պր. Աւետիս Տէր Յարութիւն}; \translatorHD{SEA: /ɑvetis teɾ hɑɾutʰjun/, SWA: /ɑvedis deɾ hɑɾutʰʏn/}). My work was organized based on the program that we set up above. 

Besides these, I also have many other unpublished studies. These studies are on the dialects of Agulis, Zeytun, Tbilisi, Kharberd, Karin, Hamshen, Maragha, Mush, New Nakhichevan, Vozim, Istanbul, Rodosto, Van, and Tigranakert. I have collected other information on many other dialects whether in person during my travels (Istanbul, Adapazar, Samsun, Trabzon, Baberd, Karin, Paris, Tbilisi, Etchmiadzin, Yerevan, Dilijan, Shushi, \armenian{Թաւրիզ}, Baku, Batumi, New Bayazet, ... 


\begin{adjarianpage}\label{page:14}\end{adjarianpage}% should be 14

... New Nakhichevan, Rodosto) or through emigres. 

For a while, I've had the idea of creating a complete map of Armenian dialects, where every village would be categorized into a dialectal group. The French have just completed a linguistic atlas of French, which took them years to make. The whole thing forms a volume of 1750 maps, such that each word is marked in terms of what form is taken in every corner of France. We won't see such a grand undertaking even in our dreams. But it's possible to create a simple linguistic map.

With this goal, in July 1907 I started traveling. At the same time, I visited 31 Armenian villages in the New Bayazet province, except for the city where I stayed for a year. I decided the position of every village within the dialectological classification. And from each village I took a sample, as we shall see in my work. For the subsequent years, I set my mind to continue and complete my travels, as much as my life and abilities would permit. 

And thus these investigations happened, which allowed me to create the present volume, whose goal, as we said above, is the classification of Armenian dialects, their attested spread, their borders, their general characteristics, a general sketch of their their phonetics, phonology, and morphology, and their characteristic borders with which a dialect differs from other dialects. Alongside my writings, there is a linguistic map of Armenian. There, I have marked only those cities and villages where Armenians exist. The language and dialect of those areas are decided or marked with colors and borders. We confess that there are many things missing that we need to fill, there many uncertain points that we might verify, and there are many errors to fix. Our book shows above all else what are the parts that need further study and where the attention of ethnographers should go. We expect in the future the completion of my work. 


\section{Differences between Old Armenian and New Armenian}

Before we go through my main work, I think that is important... 



\begin{adjarianpage}\label{page:15}\end{adjarianpage}% should be 15

... that we explain in this introduction those differences that distinguish New Armenian from Old Armenian. Because these differences are common across almost all our dialects, then by discussing these differences, we save ourselves from the extra work, and we won't need to repeat the same points for each dialect. 


The various differences between Old Armenian and New Armenian can be divided into four types: 
\begin{enumerate}
 \item phonetic differences
\item lexical differences 
\item phonological differences
\item syntactic differences
\end{enumerate}


\subsection{Phonetic differences}\label{sec:IntroAdjarian:Differences:Phonetic}


\translatorHD{In this section, Adjarian first lists the various segments from Classical Armenian. He then describes how these segments underwent alternations to the modern dialects. However, Adjarian does not actually explain the phonological system of Classical Armenian. I explain Classical Armenian in \S\ref{sec:HossepIntro:phonotransc:CA}.}

\subsubsection{Segment inventory of Classical Armenian}
Old Armenian had the following 46 sounds. 
It had 7 vowels (Table \ref{tab:classicalVowel}). 

\begin{table}[H]
 \centering
 \caption{Monophthong vowels of Classical Armenian}
 \label{tab:classicalVowel}
 \begin{tabular}{|l|lllllll|}
 \hline 
 Orthography & \armenian{ա} & \armenian{ե} & \armenian{է} & \armenian{ը}& \armenian{ի} & \armenian{ո} & \armenian{ու}\\
 HMB transliteration & a & e & ē & ə & i & o & u \\
 IPA transcription & ɑ & e & ē & ə & i & o & u 
 \\ \hline
 \end{tabular}
\end{table}




 Old Armenian had 9 diphthongs (Table \ref{tab:classicalDiphthong}). 


\begin{table}[H]
 \centering
 \caption{Diphthong vowels of Classical Armenian}
 \label{tab:classicalDiphthong}
 \begin{tabular}{|l|lllllllll|}
 \hline 
 Orthography & \armenian{այ} & \armenian{աւ} & \armenian{եա} & \armenian{եւ} & \armenian{եայ}& \armenian{եաւ}& \armenian{իւ} & \armenian{ոյ} & \armenian{ուա}\\
 HMB transliteration & ay & aw & ea & ew & eay& eaw& iw & oy & ua\\
 IPA transcription & ɑi̯ & ɑu̯ &e̯ɑ & eu̯ & e̯ɑi̯& e̯ɑu̯ & iu̯ & oi̯ & u̯ɑ
 \\ \hline
 \end{tabular}
\end{table}



Old Armenian had 30 consonants (Table \ref{tab:classicalConsonant}). 

\begin{table}[H]
 \centering
 \caption{Consonants of Classical Armenian}
 \label{tab:classicalConsonant}
 \begin{tabular}{|l|lllllllll|}
 \hline 
 Orthography & \armenian{բ} &\armenian{պ}& \armenian{փ} &\armenian{դ}& \armenian{տ} &\armenian{թ}& \armenian{գ}& \armenian{կ}& \armenian{ք} \\
 HMB transliteration & b &p& pʿ &d& t &tʿ& g& k& kʿ \\
 IPA transcription & b &p& pʰ &d& t &tʰ& ɡ& k& kʰ \\
 \hline 
 Orthography &\armenian{ձ}& \armenian{ծ}& \armenian{ց} &\armenian{ջ}& \armenian{ճ}& \armenian{չ} & & & \\
 HMB transliteration &j &c &cʿ& ǰ &č &čʿ & & & \\
 IPA transcription & d͡z & t͡s & t͡sʰ & d͡ʒ & t͡ʃ & t͡ʃʰ & & & \\
 \hline 
Orthography & \armenian{վ} & \armenian{ս}& \armenian{զ}& \armenian{շ}& \armenian{ժ}& \armenian{խ} & \armenian{հ} & & \\
HMB transliteration & v & s& z& š& ž& x & h & & \\
IPA transcription& v & s& z& ʃ& ʒ& χ & h & & 
\\ 
 \hline
 Orthography & \armenian{մ} & \armenian{ն} & \armenian{ր}& \armenian{ռ}& \armenian{լ}& \armenian{ղ} & \armenian{ւ} & \armenian{յ} & \\
 HMB transliteration & m & n & r & ṙ&l & ł & w & y & \\
 IPA transcription & m & n & ɾ & r& l & ł & w & j& 
\\ \hline 
 \end{tabular}
\end{table}




\subsubsection{Sound changes from Classical to Modern Armenian}
In this phonetic system, New Armenian has introduced the following changes. 
 

\paragraph{Front mid vowels}\label{sec:IntroAdjarian:differences:phonetic:change:midvowelfront}

Old Armenian differentiated the Classical sounds /e,ē/ <\armenian{ե,է}> whose difference is however unclear to us. New Armenian has removed one of these two sounds, such that in many dialects (and also in the literary languages), these two sounds are rendered into one sound which we represent as /e/ <\armenian{է}>. Some of the dialects (such as Karin, Mush, Van, Suceava, etc.), differentiate the two types of /e/ <\armenian{է}> sounds in stressed syllables. They changed the Classical Armenian /ē/ <\armenian{է}> to /e/ <\armenian{է}>, while the Classical /e/ <\armenian{ե}> becomes a diphthong. In unstressed syllables, both Classical /e, ē/ <\armenian{ե,է}> became /e/ <\armenian{է}>. Like some other dialects, the literary languages distinguish the reflexes of word-initial CA /e,ē/ <\armenian{ե,է}>, such that the reflex of CA /e/ <\armenian{ե}> is pronounced /je/ <\armenian{յէ}>, while the reflex of initial CA /ē/ <\armenian{է}> is pronounced as /e/ <\armenian{է}>. 





\begin{adjarianpage}\label{page:16}\end{adjarianpage}% should be 16

\paragraph{New vowels}

In the vowel series, some dialects have a added a few new sounds, the chief among them are /æ/ <\armenian{ա}̈>, /œ/ <\armenian{էօ}>, /ʏ/ <\armenian{իւ}>. The literary language has not accepted these sounds. But the sounds /œ, ʏ/ <\armenian{էօ, իւ}> are used often in foreign words or in names.\footnote{\translatorHD{In my experience, such front round vowels only appear in borrowings for Standard Western, not Standard Eastern. Thus why Table \ref{tab:frontRoundLiterary} uses SWA instead of SEA. }} The sound /æ/ <\armenian{ա}̈> is not used in the literary languages.

\begin{table}[H]
 \centering
 \caption{Front round vowels in borrowings in literary Armenian (Standard Western Armenian)}
 \label{tab:frontRoundLiterary}
 \begin{tabular}{|lll|ll|}
 \hline & &Trad. Ortho. & \multicolumn{2}{l|}{Transcription (SWA)} \\
 `Young Turk' & (from French \textit{Jeune Turc} or & \armenian{Ժէօն Թիւրք} & \armenian{Ժէօն Թիւրք} & /ʒœn tʰʏɾkʰ/\\
 & Turkish \textit{Jön Türk}) & && \\
 `Eugène Sue'& (French name) & \armenian{Էօժէն Սիւ} & \armenian{Էօժէն Սիւ} & /œʒen sʏ/
\\ \hline 
 \end{tabular}
\end{table}

\paragraph{Loss of Classical diphthongs}\label{sec:AdjarianIntro:difference:soundChange:DiphthongLoss}

New Armenian in contrast does not have diphthongs. The rich usage of diphthongs in Classical Armenian has wholly dissolved, becoming either vowels or vowel+consonant sequences. There are only a few dialects which have created new diphthongs. The literary language in contrast has preserved the form of the Old Armenian diphthongs, but they have been given a pronunciation which sometimes corresponds to Classical Armenian, sometimes to the present dialects, and sometimes to neither. The following is a summary of their form changes: 

\begin{table}[H]
 \centering
 \caption{Summary of diachronic changes from Classical Armenian diphthongs}
 \label{tab:diphthongDiachrony}
 
\begin{tabular}{|ll|l l|l l| }
\hline 
\multicolumn{2}{|l|}{Classical Armenian} & \multicolumn{2}{l|}{Dialects} & \multicolumn{2}{l|}{Literary language} \\
\hline 
\armenian{այ}& ɑi̯ & ɑ, e & \armenian{ա, է} & ɑj & \armenian{այ} \\
 \armenian{աւ}& ɑu̯ & o, œ & \armenian{օ, էօ}& o & \armenian{օ} \\
\armenian{եա}&e̯ɑ & e, i & \armenian{է, ի} & jɑ & \armenian{յա} \\
\armenian{եւ}&eu̯ & ev, iv & \armenian{էվ, իվ} & ev & \armenian{էվ} \\
 \armenian{եայ}&e̯ɑi̯ & – & - & jɑ & \armenian{յա} \\
\armenian{եաւ}& e̯ɑu̯ & ev, iv & \armenian{էվ, իվ} & ev & \armenian{էվ} \\
 \armenian{իւ}&iu̯ & u, ʏ & \armenian{ու, իւ} & ʏ, ju, ji & \armenian{իւ, յու, յի} \\
\armenian{ոյ}&oi̯ & u, ʏ & \armenian{ու, իւ} & uj & \armenian{ույ} \\
\armenian{ուա}&u̯ɑ &vɑ & \armenian{վա}&vɑ/ & \armenian{վա} \\\hline
\end{tabular}
\end{table}

\paragraph{Change from CA /ɑu̯/ to MA /o/}


Because the Classical diphthong /ɑu̯/ <\armenian{աւ}> became modern /o/ <\armenian{օ}>, the modern language created two types of vowels /o/ <\armenian{օ}>. One is /o/ <\armenian{օ}> from Classical /o/ <\armenian{ո}>, the other is /o/ <\armenian{օ}> from Classical /ɑu̯/ <\armenian{աւ}>. Of the dialects that distinguish the reflexes of CA /e,ē/ <\armenian{ե,է}>, they have also created a diphthong <\armenian{ո}> (read as /u̯o/ <\armenian{ուօ}>); in stressed syllables, they distinguish Classical /o/ that became modern /u̯o/ (\armenian{ո}>\armenian{ո}) from Classical /ɑu̯/ that became modern /o/ (\armenian{աւ}>\armenian{օ}). The literary language does not know of this distinction. For the literary language, the letters <\armenian{ո,օ}> have the same pronunciation /o/, and the diphthongal pronunciation of <\armenian{ո}> as /u͜o/ does not exist. The literary languages distinguishes only the word-initial letters <\armenian{ո,օ}> (just as for <\armenian{ե,է}>) with the former pronounced as /vo/ <\armenian{վօ}>, and the latter as /o/ <\armenian{օ}>. 




\begin{adjarianpage}\label{page:17}\end{adjarianpage}% should be 17

\paragraph{Laryngeal features of consonants}
Old Armenian distinguishes three degrees of plosive consonants: voiced (\armenian{թրթռուն}), voiceless unaspirated (\armenian{խուլ}), and voicelessaspirated (\armenian{թաւ}). The voiceless aspirated series is preserved almost everywhere. But the voiced and voiceless unaspirated series have changed or exchanged places in many dialects. We will later see the details throughout my work, when each dialect is discussed in turn. Some of the dialects have introduced an entirely new series of plosives, which we can call voiced aspirates (\armenian{շնչաւոր թրթռուն}). These are the sounds /bʰ, ɡʰ, dʰ, d͡zʰ, d͡ʒʰ/ <\armenian{բՙ, գՙ, դՙ, ձՙ, ջՙ}>, which are represented in the European system as <bh, gh, dh> and so on. They originate from the Classical sounds /b, ɡ, d, d͡z, d͡ʒ/ <\armenian{բ, գ, դ, ձ, ջ}>. The literary Eastern language has in general preserved the old pronunciation of consonants. But the literary Western language has changed the voiced plosives into voiceless aspirates, while the voiceless unaspirated were changed to voiced. Cf. my phonetic tables in \citet{Adjarian-1899-ArmenianExplosives}. 

\paragraph{Changes for CA /j/ and CA /ɬ/}\label{sec:AdjarianIntro:difference:soundChange:VelarGlide}

For the other consonants, the most changes have happened to the CA /j, ɬ/ <\armenian{յ, ղ}> whose pronunciations have entirely changed. The letter <\armenian{յ}> was pronounced as CA /j/ everywhere in the old language. But word-initially, almost every dialect has deleted this letter; some have turned it into /ɦ/ <\armenian{յ}̵> ; while the literary languages have turned it to /h/ <\armenian{հ}>. The letter <\armenian{ղ}> in the old language was some type of thick /l/ <\armenian{լ}>. \translatorHD{It was a velar lateral /ɬ/ in Classical Armenian.} But in all the dialects and in literary languages, this sound acquired its familiar guttural (\armenian{կոկորդային}) pronunciation without exception. \translatorHD{It became a dorsal fricative, such as the SEA and SWA /ʁ/.}

\paragraph{The sound /f/}

Old Armenian did not have the /f/ <\armenian{ֆ}> sound. The new dialects have created this sound, whether by borrowing foreign words or by sound changes (\armenian{ձայնաշրջութեամբ}) in native words. The literary language uses this sound only in transcribing foreign words. 

\paragraph{Syncope of word medial CA /ɑ/}

In many of our dialects, especially the ones which are known as being in the Western branch, the reflex of the Classical sound /ɑ/ <\armenian{ա}> of polysyllabic words is deleted when it is not in the initial or final syllable. This sound change appears quite simply in the declension of words (Table \ref{tab:syncopeDataDecl}).\footnote{\translatorHD{For this section on syncope, Adjarian doesn't specify which modern variety of Armenian he's talking about. I assume he meant Standard Western Armenian. Note that in SWA, post-fricative stops deaspirate and there is obstruent voicing assimilation; thus a more correct transcription of /kʰɑʁkʰ-i/ is [kʰɑχk-i]. I don't modify Adjarian's original transcriptions.}}


\begin{table}[H]
 \centering
 \caption{Syncope of word medial CA /ɑ/ in declension} 
 \label{tab:syncopeDataDecl}
 \begin{tabular}{|ll| ll ll | }
 \hline & & `mouth'& & `city'& \\
 Classical & &beɾɑn & \armenian{բերան} &kʰɑɬɑkʰ & \armenian{քաղաք} 
 \\
 SWA& (citation) & pʰeɾɑn & \armenian{բերան} & kʰɑʁɑkʰ & \armenian{քաղաք} 
 \\
 & Genitive & pʰeɾn-i & \armenian{բերնի} & kʰɑχkʰ-i & \armenian{քաղքի} 
 \\
 & Intrumental & pʰeɾn-ov & \armenian{բերնով} &kʰɑχkʰ-ov & \armenian{քաղքով}
 \\ \hline 
 \end{tabular}
\end{table}

Such is the case also for the words in Table \ref{tab:syncopeDataOther}.


\begin{table}[H]
 \centering
 \caption{Syncope of word medial CA /ɑ/ in other words} 
 \label{tab:syncopeDataOther}
 \begin{tabular}{| l | ll ll | }
 \hline & `to waste' & & `wedding' & \\
 Classical & hɑtɑnil & \armenian{հատանիլ} & hɑrsɑnikʰ& \armenian{հարսանիք}
 \\
 SWA& hɑdnil & \armenian{հատնիլ} & hɑɾsnikʰ & \armenian{հարսնիք}

 \\ \hline 
 \end{tabular}
\end{table}

Thanks to this, it often happens that two... 



\begin{adjarianpage}\label{page:18}\end{adjarianpage}% should be 18

... ... consonants become adjacent and this causes new sound changes to happen (Table \ref{tab:SyncopeFeedSoundChange}).

\begin{table}[H]
 \centering
 \caption{Medial syncope of CA /ɑ/ feeds other sound changes}
 \label{tab:SyncopeFeedSoundChange}
 \begin{tabular}{| l | ll | ll | ll|}
 \hline &Classical & & SWA && Other \\
 `to pass' & ɑnt͡sʰɑnel & \armenian{անցանել} & ɑnt͡sʰnil & \armenian{անցնիլ} & ɑsnil & \armenian{ասնիլ}
 \\
 `to recognize' & t͡ʃɑnɑt͡ʃʰel & \armenian{ճանաչել} &d͡ʒɑnt͡ʃʰnɑl & \armenian{ճանչնալ} & t͡ʃɑʃnɑl & \armenian{ճաշնալ}
 \\
 `to button' & *kot͡ʃɑkel & *\armenian{կոճակել} &ɡod͡ʒɡel & \armenian{կոճկել} & koʒkel & \armenian{կոժկէլ} \\ \hline
 \end{tabular}
 
\end{table}

\paragraph{Rhotic metathesis}
Some words show the movement or metathesis of the Classical sound /ɾ/ <\armenian{ր}>, which is constant across all the dialects (Table \ref{tab:metathesis}).\footnote{\translatorHD{This seems like an overgeneralization. The modern word for `bridge' lacks metathesis is SWA /ɡɑmuɾt͡ʃ/ and SEA /kɑmuɾd͡ʒ/ <\armenian{կամուրջ}>. }} For this rule, \citet[241ff]{Grammont-Saussure}.


 
 \begin{table}[H]
 \centering
 \caption{Diachronic rhotic metathesis}
 \label{tab:metathesis}
\begin{tabular}{|l|ll|lll|}
 \hline &Classical& &Modern & & \\
 \hline `bridge' & kɑmuɾd͡ʒ & \armenian{կամուրջ} & kɑɾmund͡ʒ & \armenian{կարմունջ} & unspecified dialect 
 \\
 `carpet' & kɑpeɾt & \armenian{կապերտ} & kɑɾpet & \armenian{կարպետ} & unspecified dialect 
\\
 unclear gloss & pʰipʰeɾd & \armenian{փիփերդ} & pʰiɾpʰet & \armenian{փիրփէտ} & Karabakh 
\\
 `clean' & *sesuɾb & *\armenian{սեսուրբ} & seɾsupʰ & \armenian{սէրսուփ} & Van 
\\\hline 

 \end{tabular}
 \end{table}

\paragraph{Nasal epenthesis}

 In many places, after a word's final syllable, the nasal /n/ <\armenian{ն}> is inserted between a vowel and consonant (Table \ref{tab:nasalEpenthisis}).


 \begin{table}[H]
 \centering
 \caption{Diachronic nasal epenthesis }
 \label{tab:nasalEpenthisis}
\begin{tabular}{|l|ll|ll|}
 \hline &\multicolumn{2}{l|}{Classical Armenian}& \multicolumn{2}{l|}{Unspecified modern variety} \\
 \hline `we' & mekʰ & \armenian{մեք} & menkʰ & \armenian{մենք}
 \\
 `green' & kɑnɑt͡ʃʰ & \armenian{կանաչ} & kɑnɑnt͡ʃʰ & \armenian{կանանչ} 
\\
 `bridge' & kɑmuɾd͡ʒ & \armenian{կամուրջ} & kɑɾmund͡ʒ & \armenian{կարմունջ} 
\\
 `recognition' & t͡ʃɑnɑt͡ʃʰ & \armenian{ճանաչ} & t͡ʃɑnɑnt͡ʃʰ & \armenian{ճանանչ} 
\\
 `they' & *ɑnokʰ & *\armenian{անոք} & nokʰɑ > ɑnonkʰ & \armenian{նոքա} > \armenian{անոնք} 
 \\\hline 

 \end{tabular}
 \end{table} 
 
 
 
 
 In these words, the insertion of the nasal /n/ <\armenian{ն}> is due to the influence of the preceding syllable's nasal /m,n/ <\armenian{մ,ն}>. In verbs, the 1{\pl} imperfective and perfective forms also show this insertion, via analogy to present verbs. 
 
 
 For example, present 1{\pl} of `to eat' is SWA /ɡute-nkʰ/ <\armenian{կ՚ուտենք}>, past imperfective 1{\pl} is /ɡutei-nkʰ/ <\armenian{կ՚ուտէինք}>, and past perfective 1{\pl} is /ɡeɾa-nkʰ/ <\armenian{կերանք}>.\footnote{\translatorHD{I segment out the 1{\pl} suffix /nkʰ/. What Adjarian means is that in Classical Armenian, the 1{\pl} suffix was /mkʰ/ for the present, but just /k/ʰ without a nasal for the past \citep[31,49]{Thomson-1989-IntroClassicalArmenian}. He argues the nasal spread via analogy.}} As for the word CA /kʰitʰ/ <\armenian{քիթ}> `nose' which became MA /kʰintʰ/ <\armenian{քինթ}>, and and similar words, the insertion of the nasal /n/ is due to some unknown phonetic rule. 

\paragraph{Verbs of `to say'}


Against the Classical words /ɑsel/ <\armenian{ասել}> `to say' and /ɑnel/ <\armenian{անել}> `to do', we often find in the new dialects words like /ɑsel/ <\armenian{ասել}> `to say' and /ɑnel/ <\armenian{անել}> `to do' (in the Eastern branch), while the Western branch has /əsel/ <\armenian{ըսել}> `to say' and /ənel/ <\armenian{ընել}> `to do'. And with this way, they have entered the literary language. 

\subsection{Lexical differences}
 
The lexicon of the new language has changed a lot. The largest portion of the words from Old Armenian have either been lost in the new dialects or have gained new meanings. The collection and study of this latter group of words is important for advancing the study of the history of their meanings.\footnote{\translatorHD{I think he means diachronic semantics or semantic change.}} The new dialects have also created many new words which are known under the name of provincial (\armenian{գաւառական}) words, and they do not exist in Classical Armenian. In my extensive provincial dictionary (unpublished), the number of these words is 30,000. The two literary languages have also created many new words, which are also absent from Classical Armenian. For example, SEA /ʃokʰenɑv/ \armenian{շոգենաւ} `steamboat' and SEA /herɑχos/ `telephone'. The complete collection of these words is still lacking.


\begin{adjarianpage}\label{page:19}\end{adjarianpage}% should be 19

Those words that are common in both Classical Armenian and the new dialects often underwent certain sound changes which are not easy to explain with conventional phonetic laws. In many places we find words that have changed so much that it's sufficiently hard to recognize their original form. For example, the Moks province has the word /χa/ <\armenian{խա}> instead of CA /het/ <\armenian{հետ}> `with, together'. The Zeytun dialect has the word /bɑjob/ <\armenian{բայոբ}> instead of Classical /pɑrɑ̯u/ <\armenian{պառաւ}> `old woman'. The Hamshen dialect has the word /onluχkʰ/ <\armenian{օնլուխք}> instead of Classical /ɑnɑnuχ/ <\armenian{անանուխ}> `mint'. The number of such words is not large. 

Our dialects also have many foreign words which are borrowed from neighboring langauges. The quality and quantity of these borrowings distinguishes the dialects based on their position. Among the lender languages (\translatorHD{the languages which provide borrowings}), first place goes to Turkish which with its various branches (Ottoman, Azerbaijani Turkish, Tatar) has had a tremendous influence on our dialects without exception. The number of words in the Istanbul dialect that were borrowed from Turkish is 4200. For the dialects in Armenia proper, they have around only a half of this number. See \citep{Adjarian-1902-TUrkishWordsArmenian}. 

After Turkish, we have the languages of Kurdish, Georgian, Russia, and Italian. 

For words borrowed from Kurdish, the number of these words is still uncertain. These words are found in the dialects of Mush, Van, and Tigranakert. The words borrowed from Georgian are found in the dialects of Tbilisi and Artvin. The number of words borrowed from Russia is 600 in my (unpublished) collection, and they are found in in all the Russian-Armenian (\armenian{ռուսահայ}) dialects. In the New Nakhichevan dialect, these words reach the thousands. Italian borrowings are found only in the Istanbul dialect, and sometimes in the neighboring areas. There are also borrowings from Romanian, Polish, and Hungarian; these are found only in the Austrian-Hungarian dialect. 

The literary language does not have these lexical differences. The orthography of Old Armenian is restored almost everywhere (there are very few exceptions... 



\begin{adjarianpage}\label{page:20}\end{adjarianpage}% should be 20

... such as the words for `other': /ɑl, el/ <\armenian{ալ, էլ}> instead of /ɑjl/ <\armenian{այլ}>.) The provincial (\armenian{գաւառական}) words are in general left in the popular language (\armenian{ռամիկ լեզու}), and recently there is only a hope that they will enter the literary language. Foreign words are by principle excluded in our two literary languages. It is only Eastern Armenian where European scientific borrowings have some visibility.\footnote{Ter-Ghazarian \citep{DerGhazarian-DictionaryBorrowed} has collected these scientific borrowings. Their number is 1500 in that work.}

In this way, we can say that Old Armenian and the new literary languages do not have lexical differences. Our lexicon is entirely Classical, and it is significantly different from the real people's colloquial vernacular. This is why the ordinary popular call the literary language Classical Armenian. 


\subsection{Morphological differences}
In both the dialects and literary languages, there is a large number of morphological differences. The goal of these differences is linguistic simplification. Through the laws of analogy (\armenian{հանգիտութեան}), the most usual and regular forms of the language have generalized, while secondary forms and exceptions have been erased.

%\translatorHD{In what follows, the subsection divisioning in this section is my own. }

\subsubsection{Declension}
 

The declension of Classical Armenian, whose extreme complexity have by and large turned into a cause of difficulty, have been rendered into perfect simplicity in Modern Armenian (\armenian{աշխարհաբար}). Of the many stems of Old Armenian, only one has been kept. The singular genitive-dative takes the suffix /-i/ <\armenian{ի}> and the ablative takes /-e/ <\armenian{է}> (These were unique to Classical Armenian /i/-stems <\armenian{ի}> and /ɑ/-stems <\armenian{ա}>). The instrumental takes /-ov/ <\armenian{ով}>, which was unique to the Classical /o/-stem <\armenian{ո}>. The plural has an entirely new construction. Classical Armenian formed its plurals with the suffixes /-kʰ, -t͡sʰ, -s/ <\armenian{ք, ց, ս}>, which vary based on the stem and declension. In contrast, New Armenian has two new plural suffixes which in all circumstances stay the same. These are /-eɾ/ <\armenian{եր}> for monosyllabic words, /-neɾ/ <\armenian{ներ}> for polysyllabic words. (For an explanation of these forms, see \citet[169]{Karst-1901-MiddleArmenain}, ... 



\begin{adjarianpage}\label{page:21}\end{adjarianpage}% should be 21

... Pedersen, KZ 39, 456ff)\footnote{\translatorHD{I couldn't find out what was this exact reference by Pederson.}} The singular case markers are simply attached after these plural markers, without changing forms. It is only genitive-dative case suffix that takes the form /-u/ <\armenian{ու}> in the plural, whereas this suffix is restricted to a very small number of words in the singular. 

And this is the method of declension for the largest number of the new dialects and for the Western literary language. In a few other dialects and in the literary Eastern language, there are a few small differences. In these dialects, the ablative is formed by the new suffix /-it͡sʰ/ <-\armenian{ից}>. The plural genitive-dative case is formed the same way as in the singular, with the suffix /-i/ <\armenian{ի}>. And subsequently, there is more analogy than in the former dialects. 

It should be mentioned also that Classical prepositions /i-, j-, z-/ <\armenian{ի, յ, զ}> which were attached to various case declensions in Classical Armenian (accusative, ablative, locative (\armenian{ներգոյական}), prepositional (\armenian{նախդրի}), narrative, circumlative) have been lost in the new language. In a few dialects and in the literary Eastern language, the locative is formed with the suffix /-um/ <-\armenian{ում}>. 


Table \ref{tab:declWAWEA} is a table of the declensions for the literary languages. 


\begin{table}[H]
\caption{Declension system (plural + case) for Standard Western and Eastern Armenian}\label{tab:declWAWEA}
\begin{tabular}{ |l|ll|ll|ll|ll|}
 \hline & \multicolumn{4}{l|}{The Western language} & \multicolumn{4}{l|}{The Eastern language} \\
 \hline 
 & \multicolumn{2}{l|}{Singular} & \multicolumn{2}{l|}{Plural} & \multicolumn{2}{l|}{Singular} & \multicolumn{2}{l|}{Plural} \\
 \hline 
{\nom} & – & & /-eɾ/ & <-\armenian{եր}> & – & & /-eɾ/ & <-\armenian{եր}> \\
& & & /-neɾ/ & <-\armenian{ներ}> & && /-neɾ/ & <-\armenian{ներ}>
\\
{\gen}/{\dat} & /-i/ & <-\armenian{ի}> & /-eɾ-u/ & <-\armenian{երու}> & /-i/ & <-\armenian{ի}> & /-eɾ-u/ & <-\armenian{երի}> \\
& & & /-neɾ-u/ & <-\armenian{ներու}> & && /-neɾ-u/ & <-\armenian{ներու}>
\\
{\acc} & \multicolumn{2}{l|}{(like {\nom})} & \multicolumn{2}{l|}{(like {\nom})} &\multicolumn{2}{l|}{(like {\nom} or {\dat})} &\multicolumn{2}{l|}{(like {\nom} or {\dat})} \\
{\abl} & /-e/ & <-\armenian{է}> & /-eɾ-e/ & <-\armenian{երէ}> & /-it͡sʰ/ & <-\armenian{ից}> & /-eɾ-it͡sʰ/ & <-\armenian{երից}> \\ 
& & & /-neɾ-e/ & <-\armenian{ներէ}> & && /-neɾ-it͡sʰ/ & <-\armenian{ներից}>
\\

{\ins} & /-ov/ & <-\armenian{ով}> & /-eɾ-ov/ & <-\armenian{երով}> & /-ov/ & <-\armenian{ով}> & /-eɾ-ov/ & <-\armenian{երով}> \\
& & & /-neɾ-ov/ & <-\armenian{ներով}> & && /-neɾ-ov/ & <-\armenian{ներով}>
\\ 
{\locgloss} & \multicolumn{2}{l|}{(doesn't have it)} & \multicolumn{2}{l|}{(doesn't have it)} & /-um/ & <-\armenian{ում}> & /-eɾ-um/ & <-\armenian{երում}> \\
& & & & & && /-neɾ-um/ & <-\armenian{ներում}>
\\\hline
\end{tabular}
\end{table}

\subsubsection{Definite article}

Old Armenian had a definite article /-n/ <\armenian{ն}>, but it did not have a general and regular usage. In the new language, phonetic developments created two forms: /-n/ <\armenian{ն}> which was specialized for vowel-final words, and /-ə/ <\armenian{ը}> for consonant-final words. Besides this, the language developed general and complete uses for the article, in the same way as do the new European languages (French, English, German, and so on). 




\begin{adjarianpage}\label{page:22}\end{adjarianpage}% should be 22

\subsubsection{Pronoun declension}
A few of the Old Armenian pronouns have been lost in the new language. Others have kept their old form. However, because the ablative, instrumental, and locative cases have distanced themselves from their previous state, these cases are formed in the way that nouns are, with suffixes: {\abl} /-e/ <-\armenian{է}>, {\abl} /-it͡sʰ/ <-\armenian{ից}>, {\ins} /-ov/ <-\armenian{ով}>, {\locgloss} /-um/ <-\armenian{ում}>. These suffixes are added not to the nominative form, but to the dative form.

\subsubsection{Adpositions}
All the prepositions have become postpositions. There are no prepositions in the new language.\footnote{\translatorHD{This is an incorrect overgeneralization. Modern Standard Armenian does have a handful of prepositions like SEA /ɑrɑnt͡sʰ/ <\armenian{առանց}> `without'. }}


\subsubsection{Verb conjugations}
The morphological changes in verb conjugation are much larger. First and foremost, the fourth conjugation class (CA /-um/ <-\armenian{ում}>) has been erased, and New Armenian recognizes only three conjugations. Of the six verb forms from Old Armenian (present \armenian{ներկայ}, imperfective \armenian{անկատար}, perfective \armenian{կատարեալ}, Future <\armenian{ապառնի}>, imperative \armenian{հրամայական}, and subjunctive \armenian{ստորադասական}), only the perfective and imperative keep their old construction. The present and the imperfect have received three new constructions, which we will talk about later. The future has a composite shape and it is formed also in three new ways, in various dialects: with the formative /kə/ <\armenian{կը}>, with the formative /piti/ <\armenian{պիտի}>, or by combing the future participle (\armenian{դերբայ}) with the copular verb (\armenian{էական բայ}). The indicative present of Old Armenian has become the the present subjunctive. 

In Classical Armenian, the formation of the passive was very complicated; and sometimes by creating simple verbs (\armenian{հասարակ բայեր}), the meanings can get confusing. In place of these complications, New Armenian developed a very simple form /-vil/ <-\armenian{ուիլ}>, by which all passive verbs form one conjugation class.

In Classical Armenian, the negative (\armenian{բացասական}) had a very simple construction. And it should be thought that at least this construction has been free from general metamorphoses. But because the conjugation of verbs has entirely changed in its form, thus it is natural that the negative would follow these changes. 

The causative (\armenian{անցողական}) formative in Classical Armenian was /-et͡sʰut͡sʰɑnel/ <-\armenian{եցուցանել}>; because of its great length, it has shortened and become modern /-t͡sʰnel/ <\armenian{ցնել}>, /-t͡sʰut͡sʰel/ <\armenian{ցուցել}>, and so on. 

Let us also mention that New Armenian has created many new complex tenses, which did not exist in the old language.

\begin{adjarianpage}\label{page:23}\end{adjarianpage}% should be 23

\subsection{Syntactic differences}

In terms of syntax (\armenian{համաձայնական կողմէ}), the New Armenian dialects significantly differ from Classical Armenian. In many other cases, the literary language restored many things according to the old language; but in this case, the literary language completely follows the dialects; and the literary language rarely but sometimes diverges from the dialects, and that divergence is for literary higher registers (\armenian{բարձր սեռերու}).%\footnote{\translatorHD{Subsection divisioning in this section is my own.}}

\subsubsection{Word order of verbs}

In Old Armenian, the verb was generally placed at the beginning of the sentence or before its arguments. In contrast, New Armenian works by putting the verb all the way at the end. Consider the following examples (\ref{sent:syntaxWordOrder}).%\footnote{\translatorHD{The glossing and segmentation is my own. I use a simplified segmentation for illustration. }}

\begin{exe}
\ex \label{sent:syntaxWordOrder}
\begin{xlist}
 
\ex \begin{xlist}
 \ex Classical Armenian \glll V S ~ ~ O ~ \\
 emut noi̯ eu̯ oɾdi-kʰ noɾɑ i tɑpɑn-ən \\
 entered Noah and son-{\pl} his to ark-{\defgloss} \\
 \trans `Noah and his sons entered the ark.'\\
\armenian{Եմուտ Նոյ եւ որդիք նորա ի տապանն։}
 \ex Modern Standard Western Armenian \glll S ~ ~ O V \\
 noj jev ɑnoɾ voɾtʰi-neɾ-ə dɑbɑn mədɑn \\ 
 Noah and his son-{\pl}-{\defgloss} ark entered \\
 \trans `Noah and his sons entered the ark.'\\
 \armenian{Նոյ եւ անոր որդիները տապան մտան։}
 \end{xlist}

\ex \begin{xlist}
 \ex Classical Armenian \glll O V ~ O ~ \\
 zint͡ʃʰ ɑɾɑɾit͡sʰ vɑsən oɾdʰwoi̯ imoi̯ \\
 what do for son my \\
 \trans `What should I do for my boy?'\\
\armenian{Զի՞նչ արարից վասն որդւոյ իմոյ։}
 \ex Modern Standard Western Armenian \glll O ~ O V \\
 dəʁus hɑmɑɾ int͡ʃʰ ənem \\ 
 boy for what do \\
 \trans `What should I do for my boy?'\\
 \armenian{Տղուս համար ի՞նչ ընեմ։}
 \end{xlist}
\ex \begin{xlist}
 \ex Classical Armenian \glll V Voc O ~ ~ \\
 luɾ oɾde̯ɑk χəɾɑtu hɑu̯ɾ kʰo \\
 listen son advice father.{\gen} your \\
 \trans `What should I do for my boy?'\\
\armenian{Լո՛ւր, որդեակ, խրատու հօր քո։}
 \ex Modern Standard Western Armenian \glll Voc ~ O ~ V \\
 dəʁɑs hoɾətʰ χəɾɑdə mədiɡ əɾe \\ 
 boy father.{\gen} advice listen do \\
 \trans `My boy, listen to your father's advice!'\\
 \armenian{Տղաս, հօրդ խրատը մտիկ ըրէ։}
 \end{xlist}
\end{xlist}
\end{exe}

\subsubsection{Word order of genitive possessors}

In Old Armenian, the modifier word (\armenian{յատկացուցիչը}) was placed after the modified word (\armenian{յատկացեալը}). In New Armenian, the exact opposite occurs: the modifier is placed before the modified (\ref{sent:genPoss}). 

\begin{exe}
 
\ex \label{sent:genPoss} \begin{xlist}
 \ex Classical Armenian \glll N Poss \\
 zeɬbɑi̯ɾən jovhɑnnu \\
 brother John.{\gen} \\
 \trans `John's brother'\\
\armenian{Զեղբայրն Յովհաննու։}
 \ex Modern Standard Western Armenian \glll Poss N \\
 ohɑnnesin ɑχpɑɾə \\ 
 John.{\gen} brother \\
 \trans `John's brother!'\\
 \armenian{Օհաննէսին ախբարը}
 \end{xlist}
 \end{exe}

\subsubsection{Word order of adjectives and nouns}

In Old Armenian, adjectives could be placed either before or after the noun. When the adjective is after the noun, the adjective agrees with the noun in number and case. When the adjective is before the noun, the adjective usually does not agree. Because the latter is the simplest structure, thus New Armenian always places its adjectives before the noun. 

\subsubsection{Word order of demonstrative and possessive pronouns}

The demonstrative and possessive pronouns\footnote{\translatorHD{His Armenian term is more literally translated as `adjective', but the word `pronoun' is more technically correct}} (\armenian{ցուցական եւ ստացական ածականները}), unlike the former (\translatorHD{meaning unlike adjectives)}, are usually placed after the noun and agree with the noun. In New Armenian, the opposite occurs: they are placed before the noun and don't agree (\ref{sent:demPoss}). 


\begin{exe}
 
\ex \label{sent:demPoss} \begin{xlist}
 
 \ex Classical Armenian \glll N Poss, N Dem, N Poss\\ 
 tun im, ɑi̯ɾəs ɑi̯s, hɑu̯ɾ imum \\
 house my, man this, father my \\
 \trans `my house, this man, my father' \\
\armenian{տուն իմ, այրս այս, հօր իմում}
 \ex Modern Standard Western Armenian \glll Poss N, Dem N, Poss N \\
 im dunəs, ɑjs mɑɾtʰə, im hoɾəs \\ 
 my house, this man, my father \\
 \trans `my house, this man, my father' \\
 \armenian{իմ տունս, այս մարդը, իմ հօրս}
 \end{xlist}
 \end{exe} 

\subsubsection{Word order of adpositions}

In Old Armenian, prepositions (\armenian{նախադրութիւններ}) were unconditionally placed before the noun. In the new language, the word `preposition' has no such meaning, because there are postpositions (\armenian{յետադրութիւն}). For example (\ref{sent:Prep}). 


\begin{exe}
 
\ex \label{sent:Prep} \begin{xlist}
 
 \ex Classical Armenian \glll P N ~, P N ~ \\ 
 ɑrɑd͡ʒi hɑu̯ɾ imoi̯, ənd seɬɑnov kʰov \\ 
 front father my, under table your \\
 \trans `in front of my father, under your table' \\ 
\armenian{առաջի հօր իմոյ, ընդ սեղանով քով}
 \ex Modern Standard Western Armenian \glll N P, N P \\
 hoɾəs ɑɾt͡ʃev, seʁɑnitʰ dɑɡə \\ 
 father.my front, table.your under \\
 \trans `in front of my father, under your table' \\ 
 \armenian{հօրս առջեւ, սեղանիդ տակը}
 \end{xlist}
 \end{exe} 



By individually taking these differences, they perhaps don't seem severe to us. But when we consider them entirely, and we compare the word order (\armenian{շարադասութիւն}) of the Modern Armenian sentence... 



\begin{adjarianpage}\label{page:24}\end{adjarianpage}% should be 24

... to Classical Armenian, we shall be surprised by this great divergence that divides the two languages. 

And truly, while Old Armenian has free word order (\armenian{ազատ շարադասութեան}) like the syntax of old Indo-European languages, its analytical word order (\armenian{վերլուծական շարադասութեան}) completely follows the new European languages, such as French word order. In contrast, New Armenian is free from this syntactic freedom, and its words are placed in a stable order, just like in Turkish syntactic style, and unlike the European one. 

Here are two sentences from Old and New Armenian, compared against French and Turkish.%\footnote{\translatorHD{For the Turkish examples, Adjarian wrote them in the Armenian script. }}

\begin{exe}
\ex \begin{xlist}

\ex `I saw the bird that sings on the tree.'\begin{xlist}
\ex Classical Armenian \gll 
 tesi əztʰərt͡ʃʰunən oɾ eɾɡēɾ i veɾɑi̯ t͡sɑroi̯n \\
saw bird that sing to on tree \\
\trans \armenian{Տեսի զթռչունն որ երգէր ի վերայ ծառոյն։}
\ex French \gll 
 J'ai vu l'oiseau qui chantait sur l'arbre \\
 I saw bird that sing on tree \\
\ex Modern Standard Western Armenian \gll 
 d͡zɑɾin vəɾɑ jeɾkʰoʁ tʰəɾt͡ʃʰunə desɑ \\
tree on singing bird saw \\
\trans \armenian{Ծառին վրայ երգող թռչունը տեսայ}
\ex Turkish \glll
Ağacın üstünde öten kuşu gördüm \\
tree on singing bird saw \\
\armenian{Աղաջըն իւսթիւնդէ էօթէօն քուշու գէօրդիւմ} \\
\end{xlist}
\ex `The sheets of the books of Leon, my neighbor's son' 
\begin{xlist}
 \ex Classical Armenian \gll
 tʰeɾtʰək ɡəɾot͡sʰ le{wo}ni oɾdwoi̯ dəɾɑt͡sʰwoi̯ imoi̯\\
 sheets books.{\gen} Leon.{\gen} son.{\gen} neighbor.{\gen} my \\
 \trans \armenian{Թերթք գրոց Լեւոնի՝ որդւոյ դրացւոյ իմոյ}
 \ex French \gll
 Les feuilles des livres de Leon fils de mon voisin \\
 the sheets of books of Leon son of my neighbor \\
 \ex Modern Standard Western Armenian \gll 
 tʰəɾɑt͡sijis dəʁun levonin kʰəɾkʰəɾun tʰeɾtʰeɾə \\
 neighbor.{\gen} son.{\gen} Leon.{\gen} books.{\gen} sheets
 \\ \trans `\armenian{Դրացիիս տղուն Լեւոնին գրքերուն թերթերը}'
\ex (Ottoman) Turkish \glll 
Komşumun oğlu Levon'un kitaplarının yaprakları \\
 neighbor.{\gen} son.{\gen} Leon.{\gen} books.{\gen} sheets\\
\armenian{Քօնշումուն օղլու Լէօվոնըն քիթաբլարընըն յափրաքլարը} \\
\end{xlist}
\end{xlist}
\end{exe}




Everything is done in this way, such that you would think that New Armenian syntax is based on the Turkish template. On this investigatable issue, see Pedersen, KZ 32,472. 




\chapter{Armenian residences}


\begin{adjarianpage}\label{page:25}\end{adjarianpage}% should be 25

As we know, the homeland of Armenians, Armenia, is divided today in the middle of three states. The largest portion is in the hand of the Ottomans; 7 out of 15 provinces from Old Armenia:\begin{itemize}
 \item Upper Armenia (\armenian{Բարձր Հայք})
 \item Fourth Armenia or Sophene (\armenian{Չորրորդ Հայք})
 \item Aghdznik or Arzanene (\armenian{Աղձնիք})
 \item Turuberan (\armenian{Տուրուբերան})
 \item Mokk' or Moxoene (\armenian{Մոկք})
 \item Korchayk or Corduene (\armenian{Կորճայք})
 \item Vaspurakan (\armenian{Վասպուրական})
\end{itemize}

A smaller portion is in the hands of the Russians: 
\begin{itemize}
 \item Artsakh (\armenian{Արցախ})
 \item Syunik (\armenian{Սիւնիք})
 \item Utik (\armenian{Ուտի})
 \item Gugark (\armenian{Գուգարք})
 \item Tayk (\armenian{Տայք})
 \item Ayrarat (\armenian{Այրարատ})
\end{itemize}

And the smallest part is in the hands of the Persians:\begin{itemize}
 \item Paytakaran (\armenian{Փայտակարան})
 \item Parskahayk or Persarmenia or Nor Shirakan (\armenian{Պարսկահայք})
\end{itemize} 

The largest portion of Armenians today are still found in their homeland. But outside of their homeland, Armenians have spread into many other countries in the following manner. 

%\translatorHD{In this chapter, the section divisioning is my own.}

\section{The northern migration line}
Armenian title: \armenian{Հիւսիսային գաղթնական գիծ}
\subsection{Georgia}

The city with the most Armenians is Tbilisi and its surrounding areas. But the Armenians are also scattered in other cities in Georgia, such as in the state of Tbilisi in Gori, Signagi, Telavi, Dusheti, Tianeti. In the Kutaisi province: Kutaisi, Poti, the two villages of Shorapani. In the Lechkhumi province, the village of Lailashi; in the Racha province: Oni village, Batumi, Artvin, Ardanuç, Şavşat, Sokhumi; in the Chornomorets province: Novorossiysk, Anapa, and the shores of the entire Black sea. The Armenian populace in this region is around 200,000. 
\subsection{Aghvank or Caucasian Albania}
Aghvank: The native population of this country was previouslu Armenian, while later a portion are Muslim. In that way, today the native element of the country is Armenian or Turkish. The cities where Armenians live are Baku, Shamakhi (with 23 villages), Geokchay with 20 villages, Nukha (42 villages), Zagatala... 


\begin{adjarianpage}\label{page:26}\end{adjarianpage}% should be 26

... 12 villages, Agdash (6 villages), Quba (Khachmaz and Kilvar villages), and finally Darband. The entire Armenian population of Aghvank is around 150,000 people. 

\subsection{North Caucasus}

Here, the Armenians represent a mixture of migrants that came from different places. They live in the Dagestan area: Petrovsk, Temir-Khan-Shura, Chiri-Yurt, Ilkarti. In the Terek area: Kizlyar, Mozdok, Vladikavkaz. In the state of Stavropol: Stavropol, Machar or Budyonnovsk. In the Kuban area: Armavir, Yekaterinodar, Batalbashu, Yeysk, Caucasus, Labin, Maykop, Temryuk. In total, 28,835 people. 

\subsection{Tatarstan (from the Volga to the ocean)}

In this area, the Armenians are chiefly in the city of Astrakhan. But in recent years, they have spread also to farther places: Tsaritsyn, Saratov, Samara, Syzran, Simbirsky, Penza, Balashov, Uribeno, Durovka, Kamyshin, Krasnovodsk, Jibil, Chakichlar, Qızıl Arvad, Ashgabat, Artəgh, Kakhka, Dulak, Merv, Charjoy, Petro-Aleksandrovsk, Samarkand, Bukhara, Ziadqin, Chernaevo, Golodnaya Steppe, Kattaqurqan, Jizzakh, Khujand, New Margelan, Kokand, Andijan, Osh, Namangan, Tashkent, Arəs, Turkistan city, Petrovsk, and many Siberian stations. The entire number of Armenians in Tatarstan is 16,000.

\subsection{Crimea}

At its time, this place had a large Armenian population. But because of migrations in 1779, many people were scattered. Today, the Armenian-populated cities in this peninsula are Theodosia, Kerch, Alushta, Yalta, Sevastopol, Yevpatoriya, Perekop, Or or Armiansk, Simferopol, Bakhchisaray, Karasubazar, and Old Crimea. The migrants of Crimea are established in New Nakhichevan and its 5 villages, which they built. From here, they also spread to Rostov, Melitopol, Berdiansk, Azov, Novocherkassk, Nogaisk, Dnipro, Taganrog, Yekaterinoslav, and other places. The number of Armenians in this area is 35,000. 

\subsection{Russia}

Russia: Here, the Armenians are very few. The entire number is less than a thousand. A large portion are students... 
\begin{adjarianpage}\label{page:27}\end{adjarianpage}% should be 27

... and soldiers. The number of native and established people is small; they are found mainly in Moscow, Saint Petersburg, Kharkiv, Voronezh, and so on. 

\subsection{Poland}

At its time, Poland had a large Armenian populace, both in its Austrian and Russian parts. In the Russian part, there are no longer any Armenians. As for the Austrian part, the main Armenian-populated location is Guti or Guter. Also, a very few number of Armenians is found in Lemberg and elsewhere. The Armenians of Guti are around 100 houses.

\subsection{Romania}

The Armenian-populated cities are Focșani, Bucharest, Botoșani, Iași, Trkvokna, Galați, Brăila, Bacău, Roman, Constanța, Sulina, Tulcea, Babadag, Pitești, Jurjevo, Ploiești, and so on. The Armenian migrants consist of two certain groups. The old migrants or natives, and the new migrants who came from various corners of Ottoman Turkey after the massacres of the Ottoman Armenians. The total number of both groups is 14,000, of which 4000 people are the new migrants. 

\subsection{Bessarabia}

Here, very few Armenians are found in Chișinău, Akkerman, Khotyn, Balti, Bender, Ismail, and Hîncești, with whom we should include the Armenians of the Cherson province (Odesa and Grigoriopol).

\subsection{Austro-Hungary}

This is Bukovina, Transylvania, Hungary, and Austria proper. The Armenians of Bukovina primarily reside in the cities of Suceava, Chernovitsa, and Seret. The Armenians of Transylvania primarily live in the cities of Gherla or Armenopolis, Dumbrăveni or Elisabethopolis, Gheorgheni, Sibviz, Brașov or Kronstadt. Few numbers of Armenians are scattered also in the various corners of Hungary, until Beshta and Vienna. The total number of Armenians in this region is 15,000. 

\section{The southeastern migration line}

Armenian title: \armenian{Հարաւային-արեւելեան գաղթնական գիծ}

\subsection{Assyria}

There are Armenians only in Mosul, Kirkuk, Baghdad, Basra, and Sughushuk. The total is 1400 people.

\subsection{Persia or Iran}


The Armenians of this country are divided into two separate dioceses (\armenian{թեմերու}). Azerbaijani Persian (Atropatene) and Persia proper. The Azerbaijan diocese has not only Khoy, Maku, Salmast, Urmia, and Karabakh, which are provinces of Armenia proper, ... 

\begin{adjarianpage}\label{page:28}\end{adjarianpage}% should be 28

... but also Tabriz, Mujumbar, Maragha, Kurdistan, and Ardabil. In Persia proper, the Armenian-populated cities are New Julfa (with its 80 villages), Tehran (with its 6 villages), Qazvin, Rasht, Anzali, Hamadan, Sheverin, Shiraz, Bushehr, and so on. The entire number of Persian Armenians is 66,000 of which 25,000 belong to Persian Armenia. 

\subsection{India}

It now has 700 Armenian residents who live in the cities of Kolkata, Madras, Bombay, and Dhaka. 

\subsection{Birmania or Myanmar}

The total number is 252 Armenians, of which 193 people live in Rangoon.

\subsection{Island of Java}

There are 170 Armenians, who live in Batavia (Jakarta), Surabaya, Singapore, Semarang, and so on. 

\section{The southwestern migration line}

Armenian title: \armenian{Հարաւային-արեւմտեան գաղթնական գիծ}


\subsection{Cilicia}

This has been Armenianized since the time of the Rubenid (\armenian{Ռուբինեան}) kingdom. Now, the main Armenian-populated cities are Sis, Hadjin, Zeytun, Adana, Tarsus, Mersin, Misis, Marash. They have a total of 190,000 Armenians.

\subsection{Cyprus}

It has now 562 Armenian residents, who are found mostly in the capital Nicosia. The others lives in Larnaca, Limassol, Paphos, Sourp Magar, Famagusta, and so on. 

\subsection{Syria and Lebanon}

The northern part, as bordering Cilicia, has a quite a lot of Armenians. But as we go south, the number of Armenians decreases. The total number of Armenians is 36,000 people, who live in the cities of Ayntap, Antioch, Aleppo, Beirut, Sham (Damascus), and Latakia. Ayntap has 6 villages, Antioch has 18 villages, Aleppo has 12 villages. Among these, the following villages are well-known. In Antioch, Svedia, Kesab, Aramo, and in Aleppo, Kilis, Belan, and Jisr al-Shughur.
Palestine. It has 730 Armenians who live in Jerusalem, Jaffa, Bethlehem, and Ramla. 

\subsection{Egypt}

It has around 10,000 Armenians who live primarily in Alexandria and Cairo. 

\subsection{Other}

There are few Armenians who also live in Tripoli, Ethiopia, Kaplandia, and Transvaal. 


\begin{adjarianpage}\label{page:29}\end{adjarianpage}% should be 29


\section{The western migration line}

Armenian title: \armenian{Արեւմտեան գաղթնական գիծ}

\subsection{Asia Minor or Anatolia}

This extends from the western borders of Armenia until the Archipelago (\armenian{Արշիպեղտգոս}) and Marmara. It includes also \armenian{Փոքր-Հայքը}, which is a heavily Armenian-populated area. The main Armenian-populated cities in Asia Minor are, from east to west, Urfa, Malatya, Divriği, Akn, Arapgir, Şebinkarahisar, Gürün, Darende, Hisn-Mansur or Adıyaman, Trabzon, Gümüşhane, Giresun, Ordu, Sebastia, Evdokia, Amasia, Merzifon, Samsun, Kayseri, Yozgat, Ankara, Konya, Kastamonu, Kütahya, Afyonkarahisar, İzmir (Smyrna), Aydın, Manisa, Bursa, Bilecik, Balıkesir, Bandırma, Nicomedia, and Adapazar. The statistics of the area are still uncertain.

\subsection{Istanbul}

Taking together the villages that are on the two shores of the Bosporus, there are 45 districts and 180,000 Armenians. Before the massacres, there were 250,000 Armenians, of which 60,000 were migrants. Because these people were deprived of their lands, the number of Istanbul Armenians significantly dropped. Now, it is rising again. 

\subsection{Greece}

It has 200 Armenians who live primarily in Athens.

\subsection{Crete}


It has about the same number of Armenians as in Kandiye. 

\subsection{European Turkey}

The Armenian-populated cities are Adrianopolis, Rodosto, Malkara, Silivri, Çorlu, Gallipoli, and Thessaloniki.

\subsection{Bulgaria}

It has 15000 Armenians who live in the following cities: Varna, Ruse-Shumla, Silistra, Sofia, Trnova, Razgrad, Vidin, Dobrich, Teleorman, Filibe, Burgas, Tatar Pazardzhik, Sliven, Yampil, Eski Zagra, Haskovo, Aytos, Karnobat, and Straldzha. 

\subsection{France}

There are 1000 Armenians who live in Paris and Marseille, and a portion in Nancy, Montpellier, and so on.

\subsection{England}

Here, there aren't as many Armenians as in France. The Armenian-populated cities are London and Manchester. 

\subsection{The United States}

It has over 40,000 Armenians, who live primarily in Worcester, New York, Providence, Fresno, Boston, and many other cities.

\begin{adjarianpage}\label{page:30}\end{adjarianpage}% should be 30

\subsection{Other}

A few number of Armenians are also found in Italy, Switzerland, Belgium, Holland, and Germany, where there are still no migrant communities. And the resident Armenians there are only temporary immigrants.

\section{Summary}
The migrant community of Armenians is more than 1 million.

\chapter{Armenians who speak foreign languages}\label{chapter:NonSpeaking}

Although Armenian is the most widely spoken language among Armernians, there are many Armenians who have forgotten Armenians; because of the influence of the dominant languages, they have adopted foreign languages. The foreign-speaking Armenians are primarily found outside the borders of Armenia and Lesser Armenia, in various foreign countries. However, even in the extremities of Armenia, there are places where Armenian has been replaced by foreign languages. But in contrast, not all migrant Armenians have forgotten Armenian. There are many places like New Julfa, Astrakhan, Smyrna, Nicomedia, Istanbul, Suceava, and others where the Armenians speak more pure Armenian dialects than some Armenians do in Armenia proper. 

However we should emphasize the circumstances such that anywhere where there is an Armenian (even if in Armenia proper), if the Armenian doesn't lose their mother tongue, then they know at least two languages: Armenian with either Turkish, Kurdish, Persian, or Russian. It is the female sex which is weaker in this regard and generally more loyal to her mother tongue, than the male sex. This bilingualism of Armenians is caused by the foreign populations that coexist with the Armenians and that have an almost equal number of people as the Armenians.This bilingualism has had a significant effect on the Armenian language

The foreign languages that have been adopted by the Armenians are the following. 

%\translatorHD{Section divisioning is my own. }

\section{Turkish}

Turkish, with its two major dialects: Western Turkish or Ottoman, and Eastern Turkish or Azerbaijani. This language is spread across the following. 

\subsection{Western Asia Minor}

Almost all of Western Asia Minor, starting from around Kastamonu until Zile, south until Kayseri, and from southeast of Kayseri onto Sis and Ayntap until the Euphrates. From west of these borders until the beaches of Marmara, of the archipelago, and of ... 

\begin{adjarianpage}\label{page:31}\end{adjarianpage}% should be 31

... the Mediterranean, all the Armenians speak Turkish. Exceptions are only the Armenians in the regions of Istanbul, Nicomedia, and Smyrna, as well a number of the villages in Ankara and Aydın, specifically Yenikent), Nallıhan, Sivrihisar, Ödemiş, and Burdur. I have heard that a few of the villages in Yozgat are also Armenian-speaking, but their names aren't clear to me. 

\subsection{Asia Minor}
In Niksar, at the northeast side of Evdokia, there's an islet of Turkish-speaking Armenians due the beast-like barbarity of the many resident Turks.
\subsection{Island of Cyprus}

The old Armenian migrant community is Turkish-speaking, but the new migrant community is Armenian-speaking. 
\subsection{European Turkey, Bulgaria, and Eastern Rumelia}
Another region of Turkish-speaking Armenians is also European Turkey, Bulgaria, and Eastern Rumelia, starting from the other side of the Marmara. Exceptions are only Rodosto and Malkara. The other Armenian-populated cities, such as Gallipoli, Silivri, Çorlu, Ereğli, Çatalca, Adrianopolis, Dimetoka, Gyumyurdjina, and Dedeağaç, are Turkish-speaking. The old migrant communities of Bulgaria and Eastern Rumelia are entirely Turkish-speaking; but after the last Ottoman-Armenian massacres, the presence of a large number of asylum-seeking Armenians caused the restoration of the forgotten Armenian language, of course only in those cities where a large number of them were relocated, such as in Filibe, Burgas, Varna, Trnova, Ruse. The other cities remain Turkish-speaking, such as Silistra, Razgrad, Shumla, Sliven, Aytos, Karnobat, Yampil, Eski Zagra, and Haskovo. 
\subsection{Roamania}
Romanian Armenian-populated cities that were previously settlements from Bulgaria, such as Babadag, Tulcea, Sulina. Here as well, the Armenians who fled the massacres have restored the Armenian language, such as in Galați, Ibraila, and Constanța. 

\subsection{Bessarabia}

Bessarabia is Turkish-speaking because it was previously part of Romania. Such as Ismail, Balti, Bender, Chișinău, Akkerman. Similarly the Armenian migrants of Bessarabia are Turkish-speaking, such as Grigoriopol, Odesa, and Cherson.

\subsection{Lazistan}
In the eastern side of Trabzon, there are Armenians found in Lazistan, who are scattered among the Turks and the Laz.

\subsection{West of Akhalkalak}

On the western side of Akhalkalak, there are four villages which are Bavra, ... 



\begin{adjarianpage}\label{page:32}\end{adjarianpage}% should be 32

... Khulgumo, Kartikami , and Turs ; they are Turkish-speaking.

\subsection{Olti}
In the region of Olti, 45 verst away from Olti, there is the Turkish-speaking village of Kalkos (25 houses).

\subsection{Urmia}
On the northern banks of Lake Urmia, especially in Sovushpulagh and Miandoab, or in one words Persian Kurdistan, the small Armenian community is Turkish-speaking. 

\subsection{Summary}

As can be seen, the Turkish-speaking Armenians form a significant number. But thankfully, this number decreases day by day. In all the major cities of Anatolia, such as in Bursa, Kayseri, and Yozgat, the new generation has become Armenian-speaking thanks to schools and because of immigration to Istanbul. A large portion of the Armenians in Cyprus, Eastern Rumelia, and Bulgaria have become Armenian-speaking thanks to the new migrants. The Ottoman government in its time used violent means or force to erase the Armenian language and to make Turkish be the dominant language (such as can be said for how the Pashas in Anatolia killed the language of Armenian-speaking Armenians), but currently it has no intention or ability of doing such methods.\footnote{\translatorHD{It is quite sad that Adjarian's optimism was soon disproven by the Armenian Genocide.}} In Bessarabia, instead of Turkish, Russian is now wide-spread. The entire population already knows Russian, and we only need a short amount of time before Turkish is completely lost. 

\section{Georgian}

This language is spoken by almost all the Georgian-Armenians. Exceptions are Tbilisi and the cities on the shortest of the Black Sea, such as Batumi, Poti, Sokhumi, and so on. The Armenians are Georgian-speaking in Sighnag, Telavi, Gori, Kutaisi, and the neighoring areas. Two of the villages of Akhalkalak are also Georgian-speaking: Vargavi and Khizabavra . The Armenians of Vladikavkaz are also Georgian-speaking, because a large portion of them have emigrated from Georgia. 

\section{Persian}

It is spoken in a very small border, between Mədrəsə (close to Shamakhi) and Kilvar (close to Quba) and in the villages of Khachmaz. Vardapet Makar Barkhudariants (\armenian{Մակար վրդ. Բարխուդարեանց}; \translatorHD{SEA: /mɑkɑɾ bɑɾχudɑɾjɑnt͡sʰ/, SWA: /mɑɡɑɾ pʰɑɾχutʰɑɾjɑnt͡sʰ/}) and bishop Mesrob Smbatian (\armenian{Մեսրոպ եպս. Սմբատեան}; \translatorHD{SEA: /mesɾop səmbɑtjɑn/, SWA: /mesɾob səmpʰɑtʰjɑn/}) have said in their topographies that the language of these villages is called Lahij (\armenian{լահճերէն}) and Tat (\armenian{թաթերէն}). But we shouldn't be confused by these names, because this language is a very clear and easy-to-understand dialect of Persian.


\begin{adjarianpage}\label{page:33}\end{adjarianpage}% should be 33



\section{Circassian}

Circassian (\armenian{չերքէզերէն}) is spoken only in Armavir, where there is an Armenian-populated village in the Kuban region. The Armenians of Armavir migrated from Circassia and founded the found in 1830.


 
\section{Kurdish}

In the Northern Armenia, Kurdish is a widely-spread language. But it has become the mother tongue in a small border. That is the villages of Hizan, the provinces of Ğarzan and Shirvan in Paghesh province (\armenian{կուստկալութիւն}); in the Tigranakert province (\armenian{կուստկալութիւն}), Meyafarikîn or Silvan, Bisheriye, Bohdan; Samsat (formerly Samosata) in Cilicia. The total number is over 100 villages. 

\section{Arabic}\label{sec:Langauges:Arabic}

It has become the mother tongue of the Armenians in Syria, Palestine, Mesopotamia, and Assyria. The Armenians in Aleppo, Damascus, Beirut, Mardin, Mosul, Kirkuk, and also Siirt in Armenia are Arabic-speaking. 

\section{Romanian}

This has become the mother tongue of the majority of Romanian-Armenian migrant community, and part of the Armenians in Bukovina. There are Turkish-speaking or Armenian-speaking Armenians only in the eastern seashores of Romania until Galați; some of these people are Armenian-speaking thanks to the recent Armenian migrants. 

\section{Polish}

This is spread almost everywhere among the Polish-Armenians, except for Guter which is Armenian-speaking. The Armenians of Poland can be considered already nationally lost.

\section{Hungarian}

It is spoken as a mother tongue among the entirety of Armenians in Hungary and Transylvania. Except for the cities of Szamosújvár or Armenopolis and Gherla or Elisabethopolis, which are Armenian-speaking.


\section{English}

This is spoken in the Indian-Armenian migrant communities, whereas the Armenians of England are still a recent settlement so they're Armenian-speaking. 

\section{Summary}
The extent and borders of these languages are all accurately represented in the map that is placed at the end of this book. 


\begin{adjarianpage}\label{page:34}\end{adjarianpage}% should be 34


\chapter{The three branches of Armenian dialects}\label{chapter:Branches}

\section{Overview}\label{sec:ThreeBranch:overview}



In general among us, the Armenian dialects are divided into two branches: Eastern or Russian-Armenian dialects, and Western or Ottoman-Armenian dialects. For me, these terms are incorrect and inappropriate, even though they are accepted and used everywhere. Calling the dialects Eastern or Western is wrong because many of the dialects that are called such are found at longitudinally equal degrees, yet when we compare them to each other, they don't fall either West or East. For example, the Van dialect and the Bayazit sub-dialect are both found longitudinally at the 44th degree, but the former is called Western while the latter Eastern. There are more surprising cases. For example, Artvin is much more west than Akhalkalak and Alexandropol (Gyumri); but based on the above division, Artvin is called Eastern, while Akhalkalak and Alexandropol are considered Western provinces. 

The names ``Russian-Armenian dialects'' and ``Ottoman-Armenian dialects'' are strange and in reality completely inappropriate. Many of the Armenians in Russia speak the Ottoman-Armenian dialects; just as there are Armenians in Ottoman Turkey that speak Russian-Armenian dialects. For example, in Russian, the Ottoman-Armenian dialects are spoken in the Akhalkalak, Akhaltskha, Alexandropol, Kars, and the villages of New Bayazet. Similarly in Ottoman Turkey, Russian-Armenian dialects are spoken in Bayazit, Burdur, Ödemiş. Besides that, the migrant communities of Persian-Armenians, Bulgarian-Armenians, Romanian-Armenians, Egyptian-Armenians, and American-Armenians are ignored; and they are inappropriately called Russian-Armenians or Ottoman-Armenians. 

I propose here new terms which not only remove the aforementioned inconveniences, but they also have the benefit of incorporating the primary characteristics of the dialects that they describe. These terms are:


\begin{adjarianpage}\label{page:35}\end{adjarianpage}% should be 35

\begin{itemize}
 \item /um/ <\armenian{ում}> branch: With this name, we mean all the dialects that are called Eastern or Russian-Armenian.
 \item /kə/ <\armenian{կը}> branch: With this name, we mean all the dialects that are called Western or Ottoman-Armenian.

\end{itemize}

For the dialects of the /um/ <\armenian{ում}> branch, the locative case suffix (as well as the present and imperfective tenses) are made with the formative /-um/ <\armenian{ում}>. This is the main characteristic of these dialects; thus we give them this name. As for the dialects of the /kə/ \armenian{կը} branch, they do not have a locative case, they don't have a formative /-um/ <\armenian{ում}>, and the present and imperfective tenses are formed with the formative /kə/ \armenian{կը}. This is their primary characteristic, and thus they get this name. But besides these two, there is also a third branch which has dialects that have neither the /um/ <\armenian{ում}> nor /kə/ \armenian{կը} particles. They form the present and imperfective tenses using either the infinitive or some invisible means, and in combination with the /em/ <\armenian{եմ}> copular verb. Among our dialects, this branch is not generally distinguished and is it appended to the /um/ <\armenian{ում}> branch. We propose using the name /el/ <\armenian{ել}> branch. 

There is no confusion in our division, and the new terminology applies only to the dialects, and they don't have anything to do with the literary languages. For them, the term Eastern and Western, or Russian-Armenian and Ottoman-Armenian are still appropriate names, because the former language is centered in Tbilisi while the latter in Istanbul. 


\begin{adjarianpage}\label{page:36}\end{adjarianpage}% should be 36

\section{Terminology}
\translatorHD{This was originally a footnote on page 36. But it is quite important and stands out. So I made it its own section. }

Against the European word ``dialecte'', we use the terms \armenian{բարբառ} `dialect', \armenian{գաւառաբարբար} `provincial dialect', and \armenian{գաւառական} `vernacular, provincial'. Because every word in the scientific language must be certain, we have thus decided to use these words. The word \armenian{գաւառաբարբառ} `provincial dialect' is alien and is an incorrect word. It is alien because of its length; and because it already contains the word \armenian{բարբառ} `dialect', it doesn't add anything. It is incorrect because a dialect has no connection to a province, and the dialect could be spoken not in the entire province but merely in a single village or city. For example, the Agulis dialect is not spoken in an entire province, but only in a small circle of villages. Similarly, the Istanbul dialect does not encompass an entire province, but only the city of Istanbul. Thus, it is preferable to use the word \armenian{բարբառ} `dialect'; it is shorter and more normal. 

A dialect can have some secondary branches that are slightly different from it; these are referred to by the European word ``sous-dialecte''. In this place, we use the Armenian word \armenian{ենթաբարբառ} `subdialect'. 

Subdialects also contain many groups, which are called in French ``parler''. For this, we use the word \armenian{գաւառական} `vernacular'. We also use this term in those situations where we cannot with certainty assign the spoken language of some place to a rank. We also use the term when we are enumerating dialects, subdialects, and vernaculars. In other words, the word \armenian{գաւառական} `vernacular' also has the general meaning of a non-literary language. 


\section{Excluded communities}\label{sec:Branches:excluded}

\translatorHD{This was originally a note on page \ref{page:293}. I moved it here because it's more relevant here. }

The Armenian settlements of Bulgaria, Rumania, Greece, France, England, Egypt, and America are newly formed, and are a mixture of Armenians from diverse places. They don't have a proper dialect, so they are not part of our present work. 


\part{The /um/ <\armenian{ում}> branch}

The /um/ <\armenian{ում}> branch has 7 dialects:

\begin{enumerate}
 \item Dialect of Yerevan
\item Dialect of Tbilisi
\item Dialect of Karabakh
\item Dialect of Shamakhi
\item Dialect of Astrakhan
\item Dialect of Julfa
\item Dialect of Agulis
 
\end{enumerate}




\chapter{Yerevan}



\begin{adjarianpage}\label{page:37}\end{adjarianpage}% should be 37

\section{Background}
The Yerevan dialect is spoken in the city of Yerevan and in the surrounding provinces, especially in the provinces of Yerevan, Etchmiadzin, and New Bayazet. It spreads from the south side to Tabriz, the capital of Azerbaijan, from the west Kaghzvan, from the southwest side it enters Ottoman Turkey and it reaches until Bayazit, from the northern and southern sides it gets mixed with the Karin and Karabakh dialects, which demarcate its two borders. On its north sides, the Yerevan dialect forms two islets; one of them is in the province of Borchaly (Shulaver, Shamshadin, Lori, and the surrounding areas), and the second is in Avlabari district of Tbilisi, which is a migrant settlement of Yerevan. 


Besides the main dialect, the Yerevan dialect has three subdialects, which are the following:

\begin{itemize}
 \item Bayazit subdialect: This in Ottoman Armenia. Its one settlement is the city of New Bayazet, the shores of Lake Sevan, with 10 surrounding Armenian-populated villages. These are Ordaklu, Noraduz, Gyshlag, Pashakendi, Kösemehmet, Kulali, Kyarimkend, Dalikardash, Lanjaghbyur, and Bashkend. This entire region speaks the same dialect, as in Bayazit. 
\item Astabad subdialect: This is spoken near Old Julfa in the village of Astabad and its surrounding area.
\item Tabriz subdialect: In Atropatene, the Armenian settlement in Tabriz has two districts: Ghala and Lilava. The people of Lilava are much larger and they have recently migrated from Karabakh; they speak the Karabakh subdialect. As for the people of Ghala, they form less than half the Armenian population of Tabriz, and they are considered natives, and they speak the Tabriz dialect.
\end{itemize}



\begin{adjarianpage}\label{page:38}\end{adjarianpage}% should be 38

The Yerevan dialect is very pure and it is very close to the literary language. And if we consider only the /um/ <\armenian{ում}> dialects, then it is the purest. And thus, it is because of its pureness and its extensive size that it serves as a base for the formation of the Russian-Armenian literary language. 

\section{Phonology}

\subsection{Segment inventory}
The phonetic system of the Yerevan dialect has the following sounds in Tables \ref{tab:Yerevan:vowels} and \ref{tab:classicalConsonant}. 



\begin{table}[H]
 \centering
 \caption{Vowels of the Yerevan dialect}
 \label{tab:Yerevan:vowels}
 \begin{tabular}{|lll|}
\hline /i/ <\armenian{ի}> & &/u/ <\armenian{ու}> \\
/e/ <\armenian{է}> & /ə/ <\armenian{ը}> & /o/ <\armenian{օ}> \\
& & /ɑ/ <\armenian{ա}> 
\\
 \hline 
 % Orthography & \armenian{ա} & \armenian{է} & \armenian{ը}& \armenian{ի}& \armenian{օ}& \armenian{ու} \\
 % IPA transcription & ɑ & e & ə & i & o & u 
 % \\ \hline
 \end{tabular}
\end{table}

\begin{table}[H]
 \centering
 \caption{Consonants of the Yerevan dialect}
 \label{tab:Yerevan:Consonant}
 \begin{tabular}{|l|lll|llll|lll|}
 \hline 
 & \multicolumn{3}{l|}{Labial}& \multicolumn{4}{l|}{Coronal}& \multicolumn{3}{l|}{Dorsal/back}\\
 Stops& /b/ & /p/ & /pʰ/ & /d/ & /t/ & /tʰ/& & /ɡ/ & /k/ & /kʰ/ 
 \\
 & <\armenian{բ}> &<\armenian{պ}>& <\armenian{փ}> &<\armenian{դ}>& <\armenian{տ}> &<\armenian{թ}>&& <\armenian{գ}>& <\armenian{կ}>& <\armenian{ք}>\\
 \hline 
 Affricates & && & /d͡z/ & /t͡s/ & /t͡sʰ/ & && & \\
 & && &<\armenian{ձ}>& <\armenian{ծ}>& <\armenian{ց}> & & & & \\
 & && & /d͡ʒ/ & /t͡ʃ/ & / t͡ʃʰ/ && & & \\
 & & & &<\armenian{ջ}>& <\armenian{ճ}>& <\armenian{չ}> & & && \\
 \hline 
 Fricatives& /f/ & /v/ & &/s/& /z/& /ʃ/& /ʒ/& /χ/ & /ʁ/ & /h/ \\
 & <\armenian{ֆ}> & <\armenian{վ}>& & <\armenian{ս}>& <\armenian{զ}>& <\armenian{շ}>& <\armenian{ժ}>& <\armenian{խ}> & <\armenian{ղ}> & <\armenian{հ}>
\\ \hline 
 Sonorants & /m/ & /n/& & /ɾ/ & /r/& /l/ & /j/ && & \\
& <\armenian{մ}> & <\armenian{ն}> && <\armenian{ր}>& <\armenian{ռ}>& <\armenian{լ}>& <\armenian{յ}> && & 
\\ \hline 
 \end{tabular}
\end{table}

Like other Armenian dialects, the Yerevan dialect does not have diphthongs. The diphthongs of Old Armenian have become either simple vowels (Table \ref{tab:Yerevan:DiphthongLoss}a) or have turned into a consonant-vowel sequence (Table \ref{tab:Yerevan:DiphthongLoss}b). 


\begin{table}[H]
 \centering
 \caption{Loss of Classical diphthongs in Yerevan Armenian}
 \label{tab:Yerevan:DiphthongLoss}
 \begin{tabular}{|l| ll|ll| ll|}
 \hline & \multicolumn{2}{l|}{Classical Armenian} &\multicolumn{2}{l|}{> Yerevan} & \multicolumn{2}{l|}{cf. SEA} \\ 
a. `father' & hɑi̯ɾ & \armenian{հայր}& heɾ & \armenian{հէր} & hɑjɾ & \armenian{հայր} \\ 
 b. `God' & ɑstu̯ɑt͡s & \armenian{Աստուած} & ɑstvɑd͡z & \armenian{Աստվաձ} & ɑstvɑt͡s & \armenian{Աստված}
\\ \hline 
 \end{tabular}
\end{table}

From this list, it appears that the Yerevan dialect has almost completely preserved the rich phonetic system of Old Armenian. Among the vowel, the Classical vowels /e-ē/ <\armenian{ե-է}> and /o-ɑu̯/ <\armenian{ո-օ}> have merged with each other; in the modern dialect, they are both pronounced as /e/ <\armenian{է}> and /o/ <\armenian{օ}>. The sounds /œ/ <\armenian{էօ}> and /ʏ/ <\armenian{իւ}> that are found in other dialects, do not exist here. Among the consonants, the only sound that was lost is Classical /w/ <\armenian{ւ}>; but it has gained the sound /f/ <\armenian{ֆ}>, about which see more below.

\subsection{Sound changes}

 



Among the sound changes that happened in the Yerevan dialect, the following are noticeable. 


\subsubsection{Monophthong vowels}

The Classical Armenian /e/ <\armenian{ե}> has become /je/ <\armenian{յէ}> word-initially in monosyllables. But at the beginning of polysyllabic words, it has become /e/ <\armenian{է}> among all words. Various other dialects have /i̯e/ <\armenian{ե}> and the literary language has word-initial /je/ in polysyllabic words; but these don't happen here. Examples in Table \ref{tab:Yerevan:SoundChange:Vowel:E}. 

\begin{table}[H]
 \centering
 \caption{Sound changes from CA /e/ <\armenian{ե}> in the Yerevan dialect}
 \label{tab:Yerevan:SoundChange:Vowel:E}
 \begin{tabular}{|l|ll|ll|ll|}
 \hline & \multicolumn{2}{l|}{Classical Armenian}& \multicolumn{2}{l|}{> Yerevan }& \multicolumn{2}{l|}{cf. SEA }
 \\
 `I'& es & \armenian{ես} & jes & \armenian{յէս} & jes & \armenian{ես} \\
 `he has come' & eke̯ɑl ē & \armenian{եկեալ է} & ekel ɑ & \armenian{էկէլ ա} & jekel e & \armenian{եկել է} \\
 `to go' & eɾtʰɑl & \armenian{երթալ} & etʰɑl & \armenian{էթալ} & jeɾtʰɑl & \armenian{երթալ} \\
 `to cook' & epʰel & \armenian{եփել} & epʰel & \armenian{էփէլ} & jepʰel & \armenian{եփել} \\
 `dream' & eɾɑz & \armenian{երազ} & eɾɑz & \armenian{էրազ} & jeɾɑz & \armenian{երազ} \\
 `big' &met͡s & \armenian{մեծ} & met͡s& \armenian{մէծ} &met͡s & \armenian{մեծ} \\
 `grave' & ɡeɾezmɑn & \armenian{գերեզման} & ɡeɾezmɑn & \armenian{գէրէզման} & ɡeɾezmɑn & \armenian{գերեզման}
 \\ \hline
 \end{tabular}
 
\end{table}

 
The Classical Armenian /o/ <\armenian{ո}>, unlike the former and like the literary language, has becomes /vo/ <\armenian{վօ}> word-initially in both monosyllabic and polysyllabic words; while it comes /o/ <\armenian{օ}> word-medially. Examples in Table \ref{tab:Yerevan:SoundChange:Vowel:O}. 

 

%\translatorHD{Note that oftentimes, Adjarian would provide a reflex that is actually SEA instead of CA. For example for `lentil', Adjarian gave the source word <\armenian{ոսպ}>. But the actual Classical form is <\armenian{ոսպն}>. The form <\armenian{ոսպ}> is modern. Such discrepancies are quite common in Adjarian's manuscript. {tab:Yerevan:SoundChange:Vowel:O} }

\begin{table}[H]
 \centering
 \caption{Sound changes from CA /o/ <\armenian{ո}> in the Yerevan dialect}
 \label{tab:Yerevan:SoundChange:Vowel:O}
 \begin{tabular}{|l|ll|ll|ll|}
 \hline & \multicolumn{2}{l|}{Classical Armenian}& \multicolumn{2}{l|}{> Yerevan }& \multicolumn{2}{l|}{cf. SEA }
 \\
 `lentil' & ospən & \armenian{ոսպն} & vosp & \armenian{վօսպ} & vosp & \armenian{ոսպ} \\
 `gold' & oski & \armenian{ոսկի} & voski & \armenian{վօսկի} & voski& \armenian{ոսկի} \\
 `foot' & ot-(ə?)kʰ (-{\pl}) & \armenian{ոտք} & votkʰ & \armenian{վօտք} & votkʰ & \armenian{ոտք} \\
 `to massacre' & kotoɾel & \armenian{կոտորել} & kotoɾel & \armenian{կօտօրէլ} & kotoɾel & \armenian{կոտորել} \\
 `to forget' & morɑnɑl & \armenian{մոռանալ} & morɑnɑl & \armenian{մօռանալ} & morɑnɑl & \armenian{մոռանալ} 
 \\ \hline
 \end{tabular}
 
\end{table}

 
\begin{adjarianpage}\label{page:39}\end{adjarianpage}% should be 39

\subsubsection{Diphthongs}

\paragraph{Classical Armenian /ɑi̯/ <\armenian{այ}>}

Classical Armenian /ɑi̯/ <\armenian{այ}> became Yerevan /e/ <\armenian{է}> as in Table \ref{tab:Yerevan:SoundChange:Diphthong:Aj:Medial}. 

\begin{table}[H]
 \centering
 \caption{Sound changes from medial CA /ɑi̯/ <\armenian{այ}> in the Yerevan dialect}
 \label{tab:Yerevan:SoundChange:Diphthong:Aj:Medial}
 \begin{tabular}{|l|ll|ll|ll|}
 \hline & \multicolumn{2}{l|}{Classical Armenian}& \multicolumn{2}{l|}{> Yerevan }& \multicolumn{2}{l|}{cf. SEA }
 \\
 `father' & hɑi̯ɾ & \armenian{հայր} & heɾ & \armenian{հէր} & hɑjɾ & \armenian{հայր} \\
 `mother' & mɑi̯ɾ & \armenian{մայր} & meɾ & \armenian{մէր} & mɑjɾ & \armenian{մայր} \\
`wagon' & sɑi̯l & \armenian{սայլ} & sel & \armenian{սէլ} & sɑjl & \armenian{սայլ} \\
`edge' & t͡sɑi̯ɾ & \armenian{ծայր} & t͡seɾ & \armenian{ծէր} &t͡sɑjɾ & \armenian{ծայր} \\
 `wood' & pʰɑi̯t & \armenian{փայտ} & pʰet & \armenian{փէտ} &pʰɑjt & \armenian{փայտ} 
\\ \hline
 \end{tabular}
 
\end{table}




Word-finally, Classical /ɑi̯/ <\armenian{այ}> has become Yerevan /ɑ/ <\armenian{ա}> as in Table \ref{tab:Yerevan:SoundChange:Diphthong:Aj:Final}. 


\begin{table}[H]
 \centering
 \caption{Sound changes from final CA /ɑi̯/ <\armenian{այ}> in the Yerevan dialect}
 \label{tab:Yerevan:SoundChange:Diphthong:Aj:Final}
 \begin{tabular}{|l|ll|ll|ll|}
 \hline & \multicolumn{2}{l|}{Classical Armenian}& \multicolumn{2}{l|}{> Yerevan }& \multicolumn{2}{l|}{cf. SEA }
 \\
 `bridegroom' & pʰesɑi̯ & \armenian{փեսայ} & pʰesɑ & \armenian{փէսա} & pʰesɑ & \armenian{փեսա} \\
 `child' & eɾɑχɑi̯ & \armenian{երախայ} & eɾeχɑ & \armenian{էրէխա} & jeɾeχɑ & \armenian{երեխա} \\
`(male?) child' & təɬɑi̯ & \armenian{տղայ} & təʁɑ & \armenian{տըղա}& təʁɑ & \armenian{տղա}

\\ \hline
 \end{tabular}
 
\end{table}


But, when the word has the article /n/ <\armenian{ն}> or the plural marker /kʰ/ <\armenian{ք}>, then the reflex of CA /ɑi̯/ <\armenian{այ}> becomes word-medial and turns into MA /e/ <\armenian{է}> (Table \ref{tab:Yerevan:SoundChange:Diphthong:Aj:Suffixed}).\footnote{\translatorHD{Note that the article /n/ was a distal marker in Classical Armenian, while it is a definite marker in SEA. It is possible that Adjarian implies that this marker is also a definite marker in Yerevan. }}


\begin{table}[H]
 \centering
 \caption{Sound changes from CA /ɑi̯/ <\armenian{այ}> in the Yerevan dialect when there's a suffix}
 \label{tab:Yerevan:SoundChange:Diphthong:Aj:Suffixed}
 \begin{tabular}{|ll|ll|ll|ll|}
 \hline & & \multicolumn{2}{l|}{Classical Armenian}& \multicolumn{2}{l|}{> Yerevan }& \multicolumn{2}{l|}{cf. SEA }
 \\
 + article & `bridegroom' & pʰesɑi̯-n & \armenian{փեսայն} & pʰese-n & \armenian{փէսէն} & pʰesɑ-n & \armenian{փեսան} \\
 & `child' & eɾɑχɑi̯-n & \armenian{երախայն} & eɾeχe-n & \armenian{էրէխէն} & jeɾeχɑ-n & \armenian{երեխան} \\
& `(male?) child' & təɬɑi̯-n & \armenian{տղայն} & təʁe-n & \armenian{տըղէն}& təʁɑ-n & \armenian{տղան}
\\
+ plural & `bridegroom' & pʰesɑi̯-kʰ & \armenian{փեսայք} & pʰese-kʰ & \armenian{փէսէք} & NA & \\
 & `child' & eɾɑχɑi̯-kʰ & \armenian{երախայք} & eɾeχe-kʰ & \armenian{էրէխէք} & NA & \\
& `(male?) child' & təɬɑi̯-kʰ & \armenian{տղայք} & təʁe-kʰ & \armenian{տըղէք}& təʁɑ-kʰ & \armenian{տղաք}
\\ \hline
 \end{tabular}
 
\end{table}


\paragraph{Classical Armenian /oi̯/ <\armenian{ոյ}>}


CLassical Armenian /oi̯/ <\armenian{ոյ}> became Yerevan /i/ <\armenian{ի}> (Table \ref{tab:Yerevan:SoundChange:Diphthong:Oj}).\footnote{\translatorHD{Adjarian translates /ɑkɑnɑkiɾ/ <\armenian{ականակիր}> as a `eye-blinding darkness, very dark night'. }}


\begin{table}[H]
 \centering
 \caption{Sound changes from CA /oi̯/ <\armenian{ոյ}> in the Yerevan dialect}
 \label{tab:Yerevan:SoundChange:Diphthong:Oj}
 \begin{tabular}{|l|ll|ll|ll|}
 \hline & \multicolumn{2}{l|}{Classical Armenian}& \multicolumn{2}{l|}{> Yerevan }& \multicolumn{2}{l|}{cf. SEA }
 \\
 `light' & loi̯s & \armenian{լոյս} & lis & \armenian{լիս} & lujs & \armenian{լույս} \\
 `sister' & kʰoi̯ɾ & \armenian{քոյր} & kʰiɾ & \armenian{քիր} & kʰujɾ & \armenian{քույր} \\
`conservation' & zəɾoi̯t͡sʰ & \armenian{զրոյց} & zɾit͡sʰ & \armenian{զրից} & zəɾujt͡sʰ & \armenian{զրույց} \\
`dark night' & *ɑkɑnɑkoi̯ɾ & *\armenian{ականակոյր} & ɑkɑnɑkiɾ & \armenian{ականակիր} & ɑkɑnɑkujɾ & \armenian{ականակույր}
\\ \hline
 \end{tabular}
 
\end{table}




The same occurs also in suffixation, for the form /u/ <\armenian{ու}> that originates from the Classical diphthong /oi̯/ <\armenian{ոյ}> (Table \ref{tab:Yerevan:SoundChange:Diphthong:Oj:Derived}). %\translatorHD{I provide the morphologically-related forms with CA /oi̯/; I don't how they behave in Yerevan. }


\begin{table}[H]
 \centering
 \caption{Sound changes from CA /u/ <\armenian{ու}> that is synchronically related to CA /oi̯/ <\armenian{ոյ}> from Classical to the Yerevan dialect}
 \label{tab:Yerevan:SoundChange:Diphthong:Oj:Derived}
 \begin{tabular}{|l|ll|ll|ll|}
 \hline & \multicolumn{2}{l|}{Classical Armenian}& \multicolumn{2}{l|}{> Yerevan }& \multicolumn{2}{l|}{cf. SEA }
 \\
 `to go blind' & ckuɾɑnɑl & \armenian{կուրանալ} & kiɾɑnɑl & \armenian{կիրանալ} & kuɾɑnɑl & \armenian{կուրանալ} \\
 cf. `blind' & koi̯ɾ & \armenian{կոյր} & & &kujɾ & \armenian{կույր} 
 \\
 `to amass' & kutel & \armenian{կուտել} & kitel & \armenian{կիտէլ} & kutel & \armenian{կուտել} 
 \\
 cf. `heap' & koi̯t & \armenian{կոյտ} & & &kujt & \armenian{կույտ} 
 
\\ \hline
 \end{tabular}
 
\end{table}



\paragraph{Classical Armenian /iu̯/ <\armenian{իւ}>}



Classical Armenian /iu̯/ <\armenian{իւ}> becomes Yerevan /i/ <\armenian{ի}> (Table \ref{tab:Yerevan:SoundChange:Diphthong:Iw}).


\begin{table}[H]
 \centering
 \caption{Sound changes from CA /iu̯/ <\armenian{իւ}> in the Yerevan dialect}
 \label{tab:Yerevan:SoundChange:Diphthong:Iw}
 \begin{tabular}{|l|ll|ll|ll|}
 \hline & \multicolumn{2}{l|}{Classical Armenian}& \multicolumn{2}{l|}{> Yerevan }& \multicolumn{2}{l|}{cf. SEA }
 \\ 
`hundred' & hɑɾiu̯ɾ & \armenian{հարիւր} & hɑɾiɾ & \armenian{հարիր} & hɑɾjuɾ & \armenian{հարյուր} \\
`snow' & d͡ziu̯n & \armenian{ձիւն} & d͡zin & \armenian{ձին} & d͡zjun & \armenian{ձյուն} \\
`column' & siu̯n & \armenian{սիւն} & sin & \armenian{սին} & sjun & \armenian{սյուն} \\
`blood' & ɑɾiu̯n & \armenian{արիւն} & ɑɾin & \armenian{արին} & ɑɾjun & \armenian{արյուն} \\
`flour' & ɑliu̯ɾ & \armenian{ալիւր} & ɑliɾ & \armenian{ալիր} & ɑljuɾ & \armenian{ալյուր} 
\\ \hline
 \end{tabular}
 
\end{table}

\subsubsection{Consonants}

The sound changes for consonants are the following. 

\paragraph{Stops and affricates}

Let us first discuss the Old Armenian the three-way series CA /b p pʰ ɡ k kʰ/ <\armenian{բ պ փ գ կ ք}> and so on. In the New Armenian dialects, these sounds have undergo many types of changes. If we accept that in Old Armenian the sounds CA /b ɡ d d͡z d͡ʒ/ <\armenian{բ գ դ ձ ջ}> were voiced, just as the letters <b, g, d> of modern French(but not German), then we must accept that they have been preserved in very few places. One such place is the Yerevan dialect. 


The Classical sounds /p t k t͡s t͡ʃ/ <\armenian{պ կ տ ծ ճ}> have undergone many changes. Many of the dialects in the /kə/ <\armenian{կը}> branch have changed these sounds into voiced consonants; while in the Tbilisi dialect, they are accompanied with a glottal closure (\armenian{կոկորդի սեղմումով}), similar to Georgian voiceless consonants.\footnote{\translatorHD{I think he means they're ejectives.}} But in the Yerevan dialect, there is no such closure and they are pronounced as simply and purely as the French sounds <p, k, t> (unlike German), with equal voicelessness, but with less strength. 

The Classical sounds /pʰ kʰ tʰ t͡sʰ t͡ʃʰ/ <\armenian{փ ք թ ց չ}> have a single pronunciation across all the dialects, and thus they don't need their own description.\footnote{\translatorHD{He means they are pronounced /pʰ kʰ tʰ t͡sʰ t͡ʃʰ/.}}

\begin{adjarianpage}\label{page:40}\end{adjarianpage}% should be 40

\paragraph{Other consonants}
 For the other consonants, the noticeable changes are the following. 

\subparagraph{Classical Armenian /h/ <\armenian{հ}>}

Word-initially, except before CA /o/ <\armenian{ո}>, this sound has become /f/ <\armenian{ֆ}> (Table \ref{tab:Yerevan:SoundChange:Consonant:H}).\footnote{\translatorHD{For the word `calf', Adjarian provides an Classical ancestor /hoɾtʰ/ <\armenian{հորթ}>. But the most prescriptive Classical form is /oɾtʰ/ <\armenian{որթ}>. I changed his example for accuracy. The /h/ was diachronically added on the path from Classical to SEA; this /h/ must likewise been epenthesized on the path from Classical to Yerevan and then became a /f/: /$\emptyset$/ > /h/ > /f/. Similarly, Adjarian provides a reconstructed */hoɾs/ <\armenian{հորս}> for `prey', but this likely developed from attested CA /oɾs/ <\armenian{որս}> with an epenthetic /h/ that became /f/. }}



\begin{table}[H]
 \centering
 \caption{Sound changes from CA /h/ <\armenian{հ}> in the Yerevan dialect}
 \label{tab:Yerevan:SoundChange:Consonant:H}
 \begin{tabular}{|l|ll|ll|ll|}
 \hline & \multicolumn{2}{l|}{Classical Armenian}& \multicolumn{2}{l|}{> Yerevan }& \multicolumn{2}{l|}{cf. SEA }
 \\ 
 `soul' & hoɡi & \armenian{հոգի} & fokʰi & \armenian{ֆօքի} & hokʰi & \armenian{հոգի} \\
 `earth' & hoɬ & \armenian{հող} & foʁ & \armenian{ֆօղ} & hoʁ & \armenian{հող} \\
 `smell' & hot & \armenian{հոտ} & fot & \armenian{ֆօտ} & hot & \armenian{հոտ} \\
 `smell' & oɾtʰ & \armenian{որթ} & foɾtʰ & \armenian{ֆօրթ} & hoɾt & \armenian{հորթ} \\
 `prey' & oɾs & \armenian{որս} & foɾs & \armenian{ֆօրս} & voɾs & \armenian{որս} 
\\ \hline 
 \end{tabular}
 
\end{table}



The sound /f/ <\armenian{ֆ}> is generally a foreign sound in the other dialects and it only found there in foreign words. But in contrast in the Yerevan dialect, it seems that the /f/ <\armenian{ֆ}> sound is an internal and native sound that arouse from natural sound changes. 

\subparagraph{Classical Armenian \armenian{խ} /χ/}

Word-initially, before the sound /ʁ/ <\armenian{ղ}>, this sound becomes /h/ <\armenian{հ}> by rule of dissimilation (Table \ref{tab:Yerevan:SoundChange:Consonant:h}). This situation does not appear in the other dialects. 



\begin{table}[H]
 \centering
 \caption{Sound changes from CA /h/ <\armenian{հ}> in the Yerevan dialect}
 \label{tab:Yerevan:SoundChange:Consonant:h}
 \begin{tabular}{|l|ll|ll|ll|}
 \hline & \multicolumn{2}{l|}{Classical Armenian}& \multicolumn{2}{l|}{> Yerevan }& \multicolumn{2}{l|}{cf. SEA }
 \\ 
 `game' & χɑɬ & \armenian{խաղ} & hɑʁ & \armenian{հաղ} & χɑʁ & \armenian{խաղ} \\
 `to play' & χɑɬɑl & \armenian{խաղալ} & hɑʁɑl & \armenian{հաղալ} & χɑʁɑl & \armenian{խաղալ} \\
 `grape' & χɑɬoɬ & \armenian{խաղող} & hɑʁoʁ & \armenian{հաղօղ} & χɑʁoʁ & \armenian{խաղող} 
\\ \hline 
 \end{tabular}
 
\end{table}


\subparagraph{Classical Armenian /ɬ/ <\armenian{ղ}>}



Word-finally in some words, it is lost (Table \ref{tab:Yerevan:SoundChange:Consonant:Gh}). However, the word `place' CA /teɬ/ <\armenian{տեղ}> on its own did not undergo this rule. \footnote{\translatorHD{For `yonder', the SEA form shows regressive devoicing of /ɑjd-teʁ/ to /ɑjt-teʁ/. We don't know if CA also had regressive devoicing. For the question word `where', stress is variable in SEA. }}



\begin{table}[H]
 \centering
 \caption{Sound changes from CA /ɬ/ <\armenian{ղ}> in the Yerevan dialect}
 \label{tab:Yerevan:SoundChange:Consonant:Gh}
 \begin{tabular}{|l|ll|ll|ll|}
 \hline & \multicolumn{2}{l|}{Classical Armenian}& \multicolumn{2}{l|}{> Yerevan }& \multicolumn{2}{l|}{cf. SEA }
 \\ 
 `here' (= `this-place')& ɑi̯s-teɬ & \armenian{այստեղ}& əste & \armenian{ըստէ}& ɑjs-teʁ & \armenian{այստեղ} \\
 `there' (= `that-place') & ɑi̯d-teɬ & \armenian{այդտեղ}& əste & \armenian{ըտէ}& ɑjt-teʁ & \armenian{այդտեղ} \\
 `yonder' (= `that-place') & ɑi̯n-teɬ & \armenian{այնտեղ}& ənde & \armenian{ընդէ}& ɑjn-teʁ & \armenian{այնտեղ} \\
 `where' (= `which-place') & oɾ-teɬ & \armenian{որտեղ}& v\'oɾte & \armenian{վօ՛րտէ}& voɾ-teʁ & \armenian{որտեղ} 
 \\
 cf. `place' & teɬ & \armenian{տեղ}& & & teʁ & \armenian{տեղ} 
\\ \hline 
 \end{tabular}
 
\end{table}

\subparagraph{Classical Armenian /t/ <\armenian{տ}>}




Before CA /n/ <\armenian{ն}>, this sound became /n/ <\armenian{ն}> through a rule of assimilation (Table \ref{tab:Yerevan:SoundChange:Consonant:tn}). Even the Russian loanword /ponnot͡sʰ/ <\armenian{պօննօց}> from /podnos/ <подносъ> `tray' (\translatorHD{modern Standard Russian: /pɐdnos/ <поднос>}). The /tn/>/nn/ sound change rule is more general in the Karabakh and Kharberd dialects. 


\begin{table}[H]
 \centering
 \caption{Sound changes from CA /tn/ <\armenian{տն}> in the Yerevan dialect}
 \label{tab:Yerevan:SoundChange:Consonant:tn}
 \begin{tabular}{|l|ll|ll|ll|}
 \hline & \multicolumn{2}{l|}{Classical Armenian}& \multicolumn{2}{l|}{> Yerevan }& \multicolumn{2}{l|}{cf. SEA }
 \\ 
 `ground' & ɡetnin & \armenian{գետնին}& ɡennin & \armenian{գէննին}& ɡetnin & \armenian{գետնին} \\
 `with-{\defgloss}' & hetən & \armenian{հետն}& henna & \armenian{հէննա}& hetə & \armenian{հետը} \\
 `after' & jetən & \armenian{յետն}& jenna & \armenian{յէննա}& jet(ə)n & \armenian{ետն} \\
 `from after' & jetnut͡sʰ & \armenian{յետնուց}& jennut͡sʰ & \armenian{յէննուց}& jetit͡sʰ & \armenian{ետից} \\
 `thimble' (Tabriz) & mɑtnot͡sʰ & \armenian{մատնոց}& mɑnnot͡sʰ & \armenian{մաննօց}& mɑtnot͡sʰ & \armenian{մատնոց} \\
 `ring' (Tabriz) & mɑtɑni & \armenian{մատանի}& mɑnnik & \armenian{մաննիկ}& mɑtɑni & \armenian{մատանի} 
\\ \hline 
 \end{tabular}
 
\end{table}


\subparagraph{Classical Armenian /ɾ/ <\armenian{ր}>}

The sound is deleted before sibilants (\armenian{շչական}) as in Table \ref{tab:Yerevan:SoundChange:Consonant:r}. But this is a general phenomenon across almost all the dialects. Besides these, we also have the word /etʰal/ `to go'. 



\begin{table}[H]
 \centering
 \caption{Sound changes from CA /ɾ/ <\armenian{ր}> in the Yerevan dialect}
 \label{tab:Yerevan:SoundChange:Consonant:r}
 \begin{tabular}{|l|ll|ll|ll|}
 \hline & \multicolumn{2}{l|}{Classical Armenian}& \multicolumn{2}{l|}{> Yerevan }& \multicolumn{2}{l|}{cf. SEA }
 \\ 
 `earthquake' & ʃɑɾʒ & \armenian{շարժ}& ʒɑʒ & \armenian{ժաժ}& ʃɑɾʒ & \armenian{շարժ} \\
 `outside' & i duɾs & \armenian{ի դուրս}& dus & \armenian{դուս}& duɾs & \armenian{դուրս} \\
 `inside' & i neɾkʰəs & \armenian{ի ներքս}& nes & \armenian{նէս}& neɾs & \armenian{ներս} \\
 `to boil' & χɑɾʃel & \armenian{խարշել}& χɑʃel & \armenian{խաշէլ}& χɑɾʃel (dated), χɑʃel & \armenian{խարշել, խաշել} \\
 `cheap' & ɑɾʒɑn & \armenian{արժան}& eʒɑn & \armenian{էժան}& ɑɾʒɑn & \armenian{արժան} \\
 `to go' & eɾtʰɑl & \armenian{երթալ}& etʰɑl & \armenian{էթալ}& jeɾtʰɑl & \armenian{արժան} 
\\ \hline 
 \end{tabular}
 
\end{table}

 


\subsection{Stress}\label{sec:Yerevan:phono:stress}

In terms of stress, the Yerevan dialect has a major innovation. In Old Armenian and without exception in all the dialects of the /kə/ <\armenian{կը}> branch, stress is on the final syllable. But in the Yerevan dialect, stress is on the penultimate syllable. This form of stress also exists to a greater extent in the dialects of Karabakh, Agulis, and Tbilisi, and it appears that... 

\begin{adjarianpage}\label{page:41}\end{adjarianpage}% should be 41

... it is widespread across the entire /um/ <um> branch. In other places \citep[185]{Adjarian-1901-Kharabagh}, we have shown that this method of stress manifested because of the influence of the Caucasian languages; and thus originating from the north, it gradually spread to the south. 

\section{Morphology}

\subsection{Noun inflection or declension}

The Yerevan dialect has seven cases. The genitive is formed with the formative /-i/ <\armenian{ի}>, and it has the characteristic that it cannot take any article (\ref{ex:Yerevan:Morpho:Noun:Gen}); it differs in this way from the dative \ref{ex:Yerevan:Morpho:Noun:Dat}). 

\begin{exe}
 \ex Yerevan\begin{xlist}
 \ex \gll kɑɾɑpet-i ɡiɾkʰ-ə \\
 Karapet-{\gen} book-{\defgloss} \\
 \trans `the book of Karapet' \label{ex:Yerevan:Morpho:Noun:Gen}\\
 \armenian{Կարապետի գիրքը}
 \ex \gll kɑɾɑpet-i-n tvi \\
 Karapet-{\dat}-{\defgloss} gave \\
 \trans `I gave it to Karapet.' \label{ex:Yerevan:Morpho:Noun:Dat}
 \\ \armenian{Կարապետին տվի} 
 \end{xlist}
\end{exe}
 

As in all the other dialects of the /um/ <\armenian{ում}> branch, the accusative case distinguishes between animate and inanimate objects (\armenian{շնշաւոր եւ անշնաւոր առարկաներ}). The accusative case of animate objects has the form of the dative case (\ref{ex:Yerevan:Morpho:Noun:DOM:Dat}); while inanimate objects have the form of the nominative (\ref{ex:Yerevan:Morpho:Noun:Dom:Acc}). 

\begin{exe}
 \ex Yerevan\begin{xlist}
 \ex \gll kɑtv-i-n əspɑnet͡sʰ\\
 cat-{\dat}-{\defgloss} killed \\
 \trans `He killed the cat.'
 \label{ex:Yerevan:Morpho:Noun:DOM:Dat}\\
 \armenian{կատվին ըսպանէց} 
 \ex \gll ɡiɾkʰ-ə tuɾ \\
 book-{\defgloss} give \\
 \trans `Give the book!.'
 \label{ex:Yerevan:Morpho:Noun:Dom:Acc}
 \\ 
 \armenian{գիրքը տուր} 
 \end{xlist}
\end{exe}

 


The other cases are special markers: ablative /-it͡sʰ/ <\armenian{ից}>, instrumental /-ov/ <\armenian{ով}>, and locative /um/ <\armenian{ում}>. 

The plural is formed with either the formative /-eɾ/ <\armenian{էր}> or /-neɾ/ <\armenian{նէր}>; the former is for monosyllabic words, while the latter for polysyllabic words. 

In the plural, the genitive case stays /-i/ <\armenian{ի}> (Table \ref{tab:Yerevan:Morpho:Noun:PlGen}); this is unlike many other dialects where the case marker is /-i/ <\armenian{ի}> in the singular, but /-u/ in the plural <\armenian{ու}>.


\begin{table}[H]
 \centering
 \caption{Genitive marking in the Yerevan dialect}
 \label{tab:Yerevan:Morpho:Noun:PlGen}
 \begin{tabular}{|l|ll|ll|}
 \hline & \multicolumn{2}{l|}{Yerevan }& \multicolumn{2}{l|}{cf. SEA }
 \\ 
 bread-{\pl}-{\dat} & hɑt͡sʰ-eɾ-i & \armenian{հացէրի} & hɑt͡sʰ-eɾ-i & \armenian{հացերի}\\
 house-{\pl}-{\dat} & tn-eɾ-i & \armenian{տնէրի} & tən-eɾ-i & \armenian{տների}\\
 bread-{\pl}-{\dat}-{\defgloss} & hɑt͡sʰ-eɾ-i-n & \armenian{հացէրին} & hɑt͡sʰ-eɾ-i-n & \armenian{հացերին}\\
 house-{\pl}-{\dat}-{\defgloss} & tn-eɾ-i-n & \armenian{տնէրին} & tən-eɾ-i-n & \armenian{տներին} 
\\ \hline 
 \end{tabular}
 
\end{table}

\subsection{Pronoun inflection or declension}

For pronouns, note the following declensions in Table \ref{tab:Yerevan:morpho:pronoun}. 

\begin{table}[H]
 \centering
 \caption{Demonstrative pronouns in the Yerevan dialect} \label{tab:Yerevan:morpho:pronoun}

 \begin{tabular}{|l|lll|lll|}
 \hline & \multicolumn{3}{c|}{Singular} & \multicolumn{3}{c|}{Plural} \\
 & proximal & medial & distal & proximal & medial & distal \\
 & `this' & `that' & `yonder' & `these' & `those' & `those yonder' \\
 \hline 
 {\nom} & es &ed& en &estonkʰ& etonkʰ& endonkʰ
 \\
 & & & & əstonkʰ& ətonkʰ& əndonkʰ
 \\
 & \armenian{էս} & \armenian{էդ} & \armenian{էն} & \armenian{էստօնք} & \armenian{էտօնք} & \armenian{էնդօնք}\\
 & & & & \armenian{ըստօնք} & \armenian{ըտօնք} & \armenian{ընդօնք}\\
\hline {\gen} & estuɾ &etuɾ &enduɾ &estont͡sʰ &etont͡sʰ & endont͡sʰ \\
 & əstuɾ & ətuɾ & ənduɾ & əstont͡sʰ & ətont͡sʰ& əndont͡sʰ \\
& \armenian{էստուր} & \armenian{էտուր} & \armenian{էնդուր} &\armenian{էստօնց} &\armenian{էտօնց} & \armenian{էնդօնց} 
\\ 
& \armenian{ըստուր}& \armenian{ըտուր}& \armenian{ընդուր} & \armenian{ըստօնց} & \armenian{ըտօնց} & \armenian{ընդօնց}
\\
\hline 
{\abl} & est-ut͡sʰ &et-ut͡sʰ & end-ut͡sʰ & estont͡sʰ-it͡sʰ & etont͡sʰ-it͡sʰ & endont͡sʰ-it͡sʰ \\
 & əst-ut͡sʰ & ət-ut͡sʰ& ənd-ut͡sʰ & əstont͡sʰ-it͡sʰ & ətont͡sʰ-it͡sʰ & əndont͡sʰ-it͡sʰ \\
& \armenian{էստուց} &\armenian{էտուց} &\armenian{էնդուց} & \armenian{էստօնցից} & \armenian{էտօնցից} &\armenian{էնդօնցից} \\
& \armenian{ըստուց} & \armenian{ըտուց} & \armenian{ընդուց}& \armenian{ըստօնցից} & \armenian{ըտօնցից} & \armenian{ընդօնցից} \\
\hline {\ins} & est-ov & et-ov & end-ov & estont͡sʰ-ov &etont͡sʰ-ov & endont͡sʰ-ov 
\\
 & əst-ov& ət-ov& ənd-ov& əstont͡sʰ-ov & ətont͡sʰ-ov& əndont͡sʰ-ov
 \\
 & \armenian{էստօվ} & \armenian{էտօվ} & \armenian{էնդօվ} &\armenian{էստօնցօվ} & \armenian{էտօնցօվ} & \armenian{էնդօնցօվ} 
\\
& \armenian{ըստօվ} & \armenian{ըտօվ} & \armenian{ընդօվ} & \armenian{ըստօնցօվ} & \armenian{ըտօնցօվ} & \armenian{ընդօնցօվ}
\\ \hline
 \end{tabular}
\end{table}

For some of the pronominal forms, the sound /i/ <\armenian{ի}> or /e/ <\armenian{է}> becomes /e/ <\armenian{ը}> when next to the conjunction /el/ <\armenian{էլ}> `also'.


\begin{table}[H]
 \centering
 \caption{Replacement of /i,e/ with /ə/ in cliticized pronouns in the Yerevan dialect}
 \label{tab:Yerevan:Morpho:pronoun:SchwaChange}
 \begin{tabular}{|l|lll|ll|}
 \hline & &\multicolumn{2}{l|}{Yerevan }& \multicolumn{2}{l|}{cf. SEA }
 \\ 
 me-also `also me' & & əs el & \armenian{ըս էլ} & jes el & \armenian{ես էլ}\\
 & instead of & jes el & \armenian{յէս էլ} & & \\
 one-also `also one' & & mək el & \armenian{մըկ էլ}& mek el & \armenian{մեկ էլ}\\
 & instead of & mek el & \armenian{մէկ էլ} & & \\
 one-also `also one' & & mən el & \armenian{մըն էլ}& min el & \armenian{մին էլ}\\
 & instead of & min el & \armenian{մին էլ} & & \\
 me.{\dat}-also `for me also' & & ənd͡z el & \armenian{ընձ էլ}& ind͡z el & \armenian{ինձ էլ}\\
 & instead of & ind͡z el & \armenian{ինձ էլ} & & \\
 me.{\dat}-also `for me also' & & mənkʰ el & \armenian{մընք էլ}& meŋkʰ el & \armenian{մենք էլ}\\
 & instead of & menkʰ el & \armenian{մէնք էլ} & & 
 \\ \hline 
 \end{tabular}
 
\end{table}

\subsection{Verb inflection or conjugation}

\subsubsection{General paradigms for the reflex of the E-Class}\label{sec:Yerevan:morpho:verb:paradigm}
Verbs are subject to the basic changes. First, two of the four conjugation classes are lost. The CA /-il/ <\armenian{իլ}> and CA /-ul/ <\armenian{ուլ}> suffixes have become /-el/ <\armenian{ել}>, and are thus conjugated... 



\begin{adjarianpage}\label{page:42}\end{adjarianpage}% should be 42

... as the first conjugation class. The Old Armenian present has turned into the composite form /-um/ <\armenian{ում}>, while the formative /kə/ <\armenian{կը}> is used in the future. As an example, we show the conjugation of the verb /siɾem/ <\armenian{սիրեմ}> `I like'. 

{\paradigmExplanation}
\paragraph{Indicative present and past imperfective}

\translatorHD{The indicative present in SEA is formed via periphrasis (Table \ref{tab:Yerevan:morpho:verb:paradigm:presentIndc}). The verb is in a converb form called the imperfective converb with the suffix /-um/. Tense and agreement is on an inflected auxiliary. The Yerevan dialect seems to follow the same system, though the suffix /-um/ can optionally reduce to just /-əm/. Note how we don't know if Yerevan also had nasal place assimilation in the 1{\pl} suffix. }


\begin{table}[H]
 \centering
 \caption{Indicative present <\armenian{ներկայ}> of the verb `to like' in the Yerevan dialect}
 \label{tab:Yerevan:morpho:verb:paradigm:presentIndc}
 \begin{tabular}{|l|ll|ll|ll|}
\hline & \multicolumn{4}{l|}{Yerevan (form I and form II)} & \multicolumn{2}{l|}{cf. SEA} \\
1SG & siɾ-um e-m & \armenian{սիրում էմ} & siɾ-əm e-m & \armenian{սիրըմ էմ} & siɾ-um e-m &\armenian{սիրում եմ} \\
2SG & siɾ-um e-s & \armenian{սիրում էս} & siɾ-əm e-s & \armenian{սիրըմ էս} & siɾ-um e-s &\armenian{սիրում ես} \\
3SG & siɾ-um ɑ & \armenian{սիրում ա} & siɾ-əm ɑ & \armenian{սիրըմ ա} & siɾ-um e &\armenian{սիրում է} \\
1PL & siɾ-um e-nkʰ & \armenian{սիրում էնք} & siɾ-əm e-nkʰ & \armenian{սիրըմ էնք} & siɾ-um e-ŋk &\armenian{սիրում ենք} \\
2PL & siɾ-um e-kʰ & \armenian{սիրում էք} & siɾ-əm e-kʰ & \armenian{սիրըմ էք} & siɾ-um e-kʰ &\armenian{սիրում եք} \\
3PL& siɾ-um e-n & \armenian{սիրում էն} & siɾ-əm e-n & \armenian{սիրըմ էն} & siɾ-um e-n &\armenian{սիրում են} \\
& \multicolumn{2}{l|}{$\sqrt{}$-{\impfcvb} {\aux}-{\agr}}& \multicolumn{2}{l|}{$\sqrt{}$-{\impfcvb} {\aux}-{\agr}}& \multicolumn{2}{l|}{$\sqrt{}$-{\impfcvb} {\aux}-{\agr}}\\
\hline 
\end{tabular}
\end{table}

\translatorHD{The indicative past imperfective uses the same imperfective converb as in the present (Table \ref{tab:Yerevan:morpho:verb:paradigm:pastImpfIndc}). The difference is that auxiliary is now in the past tense. In SEA, the auxiliary has a constant shape /e/. Outside of the 3SG, when the past suffix /i/ is added, a glide is epenthesized. But in Yerevan, it seems that this auxiliary morph /e/ is deleted before the past suffix /i/. All zero morphs are of my own. Modern Tehrani Iranian Armenian behaves similarly \citep[ch. 6.2]{DolatianEtAl-prep-IranianGrammar}.}



\begin{table}[H]
 \centering
 \caption{Indicative past imperfective <\armenian{անկատար}> of the verb `to like' in the Yerevan dialect}
 \label{tab:Yerevan:morpho:verb:paradigm:pastImpfIndc}
 \begin{tabular}{|l|ll|ll|ll|l|}
\hline & \multicolumn{4}{l|}{Yerevan (form I and form II)} & \multicolumn{2}{l|}{cf. SEA} \\
1SG & siɾ-um $\emptyset$-i-$\emptyset$ & \armenian{սիրում ի} & siɾ-əm $\emptyset$-i-$\emptyset$ & \armenian{սիրըմ ի} & siɾ-um ej-i-$\emptyset$ &\armenian{սիրում էի} \\
2SG& siɾ-um $\emptyset$-i-ɾ & \armenian{սիրում իր} & siɾ-əm $\emptyset$-i-ɾ & \armenian{սիրըմ իր} & siɾ-um ej-i-ɾ &\armenian{սիրում էիր} \\
3SG& siɾ-um e-$\emptyset$-ɾ & \armenian{սիրում էք} & siɾ-əm e-$\emptyset$-ɾ & \armenian{սիրըմ էր} & siɾ-um e-$\emptyset$-ɾ &\armenian{սիրում էր} \\
1PL& siɾ-um $\emptyset$-i-nkʰ & \armenian{սիրում ինք} & siɾ-əm $\emptyset$-i-nkʰ & \armenian{սիրըմ ինք} & siɾ-um ej-i-ŋkʰ &\armenian{սիրում էինք} \\
2PL& siɾ-um $\emptyset$-i-kʰ & \armenian{սիրում իք} & siɾ-əm $\emptyset$-i-kʰ & \armenian{սիրըմ իք} & siɾ-um ej-i-kʰ &\armenian{սիրում էիք} \\
3PL& siɾ-um $\emptyset$-i-n & \armenian{սիրում ին} & siɾ-əm $\emptyset$-i-n & \armenian{սիրըմ ին} & siɾ-um ej-i-n &\armenian{սիրում էին} \\
& \multicolumn{2}{l|}{$\sqrt{}$-{\impfcvb} {\aux}-{\pst}-{\agr}} & \multicolumn{2}{l|}{$\sqrt{}$-{\impfcvb} {\aux}-{\pst}-{\agr}} & \multicolumn{2}{l|}{$\sqrt{}$-{\impfcvb} {\aux}-{\pst}-{\agr}}\\
\hline 
\end{tabular}
\end{table}
\paragraph{Present perfect and past perfect}

\translatorHD{The present perfect (Table \ref{tab:Yerevan:morpho:verb:paradigm:presentPerfect}) and past perfect (Table \ref{tab:Yerevan:morpho:verb:paradigm:pastPerfect}) in SEA are formed with periphrasis. The verb is in the form of the perfective converb with the suffix /-el/. The present tense auxiliary is added for the present perfect, while the past auxiliary for the past perfect. The Yerevan dialect seems to use the same strategy. }

\begin{table}[H]
 \centering
 \caption{Present perfect <\armenian{յարակատար}> of the verb `to like' in the Yerevan dialect}
 \label{tab:Yerevan:morpho:verb:paradigm:presentPerfect}
 \begin{tabular}{|l|ll|ll|}
\hline & \multicolumn{2}{l|}{Yerevan} & \multicolumn{2}{l|}{cf. SEA} \\
1SG &siɾ-el e-m & \armenian{սիրէլ էմ} & siɾ-el e-m &\armenian{սիրել եմ} \\
2SG &siɾ-el e-s & \armenian{սիրէլ էս} & siɾ-el e-s &\armenian{սիրել ես} \\
3SG &siɾ-el ɑ & \armenian{սիրէլ ա} & siɾ-el e &\armenian{սիրել է} \\
1PL&siɾ-el e-nkʰ & \armenian{սիրէլ էնք} & siɾ-el e-ŋk &\armenian{սիրել ենք} \\
2PL&siɾ-el e-kʰ & \armenian{սիրէլ էք}& siɾ-el e-kʰ &\armenian{սիրել եք} \\
3PL &siɾ-el e-n & \armenian{սիրէլ էն} & siɾ-el e-n &\armenian{սիրել են} \\
& \multicolumn{2}{l|}{$\sqrt{}$-{\perfcvb} {\aux}-{\agr}}& \multicolumn{2}{l|}{$\sqrt{}$-{\perfcvb} {\aux}-{\agr}}\\ 

\hline 
\end{tabular}
\end{table}


\begin{table}[H]
 \centering
 \caption{Past perfect <\armenian{գերակատար}> of the verb `to like' in the Yerevan dialect}
 \label{tab:Yerevan:morpho:verb:paradigm:pastPerfect}
 \begin{tabular}{|l|ll|ll| }
\hline & \multicolumn{2}{l|}{Yerevan} & \multicolumn{2}{l|}{cf. SEA} \\
1SG &siɾ-el $\emptyset$-i-$\emptyset$ & \armenian{սիրէլ ի} & siɾ-el ej-i-$\emptyset$ &\armenian{սիրել էի} \\
2SG &siɾ-el $\emptyset$-i-ɾ & \armenian{սիրէլ իր} & siɾ-el ej-i-ɾ &\armenian{սիրել էիր} \\
3SG &siɾ-el e-$\emptyset$-ɾ & \armenian{սիրէլ էր} & siɾ-el e-$\emptyset$-ɾ &\armenian{սիրել էր} \\
1PL&siɾ-el $\emptyset$-i-nkʰ & \armenian{սիրէլ ինք} & siɾ-el ej-i-ŋkʰ &\armenian{սիրել էինք} \\
2PL&siɾ-el $\emptyset$-i-kʰ & \armenian{սիրէլ իք}& siɾ-el ej-i-kʰ &\armenian{սիրել էիք} \\
3PL &siɾ-el $\emptyset$-i-n & \armenian{սիրէլ ին} & siɾ-el ej-i-n &\armenian{սիրել էին} \\ 
& \multicolumn{2}{l|}{$\sqrt{}$-{\perfcvb} {\aux}-{\pst}-{\agr}}& \multicolumn{2}{l|}{$\sqrt{}$-{\perfcvb} {\aux}-{\pst}-{\agr}}\\ 

\hline 
\end{tabular}
\end{table}

\paragraph{Past perfective or aorist}

\translatorHD{The past perfective (Table \ref{tab:Yerevan:morpho:verb:paradigm:pastperfectiveAorist}) is also called the aorist. In SEA for /siɾ-e-l/ `to like', the past perfective is formed by taking the root and theme vowel, adding the aorist or perfective suffix /-t͡sʰ-/, and then adding the past suffix /-i/ and the appropriate agreement suffixes. The 3SG uses covert tense and agreement suffixes. The Yerevan dialect behaves the same. }


\begin{table}[H]
 \centering
 \caption{Past perfective or aorist <\armenian{կատարեալ}> of the verb `to like' in the Yerevan dialect}
 \label{tab:Yerevan:morpho:verb:paradigm:pastperfectiveAorist}
 \begin{tabular}{|l|ll|ll|}
\hline & \multicolumn{2}{l|}{Yerevan} & \multicolumn{2}{l|}{cf. SEA} \\
1SG & siɾ-e-t͡sʰ-i-$\emptyset$ & \armenian{սիրէցի} & siɾ-e-t͡sʰ-i-$\emptyset$ & \armenian{սիրեցի} \\
2SG& siɾ-e-t͡sʰ-i-ɾ & \armenian{սիրէցիր} & siɾ-e-t͡sʰ-i-ɾ &\armenian{սիրեցիր} \\
3SG & siɾ-e-t͡sʰ-$\emptyset$-$\emptyset$ & \armenian{սիրէց} &siɾ-e-t͡sʰ-$\emptyset$-$\emptyset$ &\armenian{սիրեց} \\
1PL & siɾ-e-t͡sʰ-i-nkʰ & \armenian{սիրէցինք} & siɾ-e-t͡sʰ-i-ŋkʰ &\armenian{սիրեցինք} \\
2PL & siɾ-e-t͡sʰ-i-kʰ & \armenian{սիրէցիք} &siɾ-e-t͡sʰ-i-kʰ &\armenian{սիրեցիք}\\
3PL & siɾ-e-t͡sʰ-i-n & \armenian{սիրէցին} & siɾ-e-t͡sʰ-i-n &\armenian{սիրեցին} \\
& \multicolumn{2}{l|}{$\sqrt{}$-{\thgloss}-{\aor}-{\pst}-{\agr}}& \multicolumn{2}{l|}{$\sqrt{}$-{\thgloss}-{\aor}-{\pst}-{\agr}}\\ 

\hline 
\end{tabular}
\end{table}
\paragraph{Subjunctive present and past imperfective } 

\translatorHD{In SEA, the subjunctive present (Table \ref{tab:Yerevan:morpho:verb:paradigm:subjPresent}) is formed by adding agreement suffixes after the theme vowel /e/. These are the same agreement suffixes that are added onto the present auxiliary in the indicative present. For a verb like `to like', the 3SG involves changing the theme vowel /e/ to /i/ in the 3SG. The Yerevan dialect behaves the same.} 


\begin{table}[H]
 \centering
 \caption{Subjunctive present <\armenian{ստորադասական ներկայ}> of the verb `to like' in the Yerevan dialect}
 \label{tab:Yerevan:morpho:verb:paradigm:subjPresent}
 \begin{tabular}{|l|ll|ll|}
\hline & \multicolumn{2}{l|}{Yerevan} & \multicolumn{2}{l|}{cf. SEA} \\
1SG & siɾ-e-m & \armenian{սիրէմ} & siɾ-e-m & \armenian{սիրեմ} \\
2SG & siɾ-e-s & \armenian{սիրէս} & siɾ-e-s & \armenian{սիրես} \\
3SG & siɾ-i-$\emptyset$ & \armenian{սիրի} & siɾ-i-$\emptyset$ & \armenian{սիրի} \\
1PL & siɾ-e-nkʰ & \armenian{սիրէնք} & siɾ-e-ŋkʰ & \armenian{սիրենք} \\
2PL & siɾ-e-kʰ & \armenian{սիրէք} & siɾ-e-kʰ & \armenian{սիրեք} \\
3PL & siɾ-e-n & \armenian{սիրէն} & siɾ-e-n & \armenian{սիրեն} \\
& \multicolumn{2}{l|}{$\sqrt{}$-{\thgloss}-{\agr}}& \multicolumn{2}{l|}{$\sqrt{}$-{\thgloss}-{\agr}}\\ 

\hline 
\end{tabular}
\end{table}

\translatorHD{In SEA, the subjunctive past imperfective (Table \ref{tab:Yerevan:morpho:verb:paradigm:subjPast}) is formed by adding the past suffix /i/ and agreement suffixes after the theme vowel. In Yerevan, the theme vowel /e/ is deleted before the past suffix /i/. Modern Tehrani Iranian Armenian behaves the same \citep[ch. 6.4.2]{DolatianEtAl-prep-IranianGrammar}. }



\begin{table}[H]
 \centering
 \caption{Subjunctive past <\armenian{ստորադասական անցեալ}> of the verb `to like' in the Yerevan dialect}
 \label{tab:Yerevan:morpho:verb:paradigm:subjPast}
 \begin{tabular}{|l|ll|ll|}
\hline & \multicolumn{2}{l|}{Yerevan} & \multicolumn{2}{l|}{cf. SEA} \\
1SG & siɾ-$\emptyset$-i-$\emptyset$ & \armenian{սիրի} & siɾ-ej-i-$\emptyset$ & \armenian{սիրեի} \\
2SG & siɾ-$\emptyset$-i-ɾ & \armenian{սիրիր} & siɾ-ej-i-ɾ & \armenian{սիրեիր} \\
3SG & siɾ-e-$\emptyset$-ɾ & \armenian{սիրէր} & siɾ-e-$\emptyset$-ɾ & \armenian{սիրեր} \\
1PL & siɾ-$\emptyset$-i-nkʰ & \armenian{սիրինք} & siɾ-ej-i-ŋkʰ & \armenian{սիրեինք} \\
2PL & siɾ-$\emptyset$-i-kʰ & \armenian{սիրիք} & siɾ-ej-i-kʰ & \armenian{սիրեիք} \\
3PL & siɾ-$\emptyset$-i-n & \armenian{սիրին} & siɾ-ej-i-n & \armenian{սիրեին} \\
& \multicolumn{2}{l|}{$\sqrt{}$-{\thgloss}-{\pst}-{\agr}}& \multicolumn{2}{l|}{$\sqrt{}$-{\thgloss}-{\pst}-{\agr}}\\ 

\hline 
\end{tabular}
\end{table}


 
\paragraph{Tenses built from the subjunctive: Future and debitive}
 
 
 \translatorHD{In Yerevan, many other tenses seem to be built off of the subjunctive (Table \ref{tab:Yerevan:morpho:verb:paradigm:complexSubjunctive}). The future and future perfect are built by adding the prefix /kə/ before the subjunctive present and subjunctive past. The debitive and debitive perfect are formed also by adding the proclitic /pti/ before the appropriate subjunctive form. SEA behaves essentially the same and I don't provide its paradigm. }
 

\begin{table}[H]
 \centering
 \caption{Forms that are built from the subjunctive forms for the verb `to like' in the Yerevan dialect}
 \label{tab:Yerevan:morpho:verb:paradigm:complexSubjunctive}
 \begin{tabular}{|l|ll|ll|}
\hline & 
\multicolumn{2}{l|}{Future <\armenian{ապառնի}>} & \multicolumn{2}{l|}{Future perfect <\armenian{անցեալ ապառնի}> } \\
1SG & kə siɾ-e-m & \armenian{կը սիրեմ} & kə siɾ-$\emptyset$-i-$\emptyset$ & \armenian{կը սիրի} \\
2SG & kə siɾ-e-s & \armenian{կը սիրէս}& kə siɾ-$\emptyset$-i-ɾ & \armenian{կը սիրիր} \\
3SG & kə siɾ-i-$\emptyset$ & \armenian{կը սիրի} & kə siɾ-e-$\emptyset$-ɾ & \armenian{կը սիրէր} \\
1PL & kə siɾ-e-nkʰ & \armenian{կի սիրէնք}& kə siɾ-$\emptyset$-i-nkʰ & \armenian{կը սիրինք} \\
2PL & kə siɾ-e-kʰ & \armenian{կը սիրէնք} & kə siɾ-$\emptyset$-i-kʰ & \armenian{կը սիրիք} \\
3PL & kə siɾ-e-n & \armenian{կը սիրէն} & kə siɾ-$\emptyset$-i-n & \armenian{կը սիրին} 
\\
& \multicolumn{2}{l|}{{\fut} $\sqrt{}$-{\thgloss}-{\agr}}& \multicolumn{2}{l|}{{\fut} $\sqrt{}$-{\thgloss}-{\pst}-{\agr}}
\\ \hline 
& \multicolumn{2}{l|}{Debitive \armenian{պարտաւորական ներկայ} } & \multicolumn{2}{l|}{Debitive perfect \armenian{պարտաւորական անցեալ} } \\

1SG & pti siɾ-e-m & \armenian{պտի սիրէմ} &pti siɾ-$\emptyset$-i-$\emptyset$ & \armenian{պտի սիրի} \\
2SG & pti siɾ-e-s & \armenian{պտի սիրէս} &pti siɾ-$\emptyset$-i-ɾ & \armenian{պտի սիրիր} \\
3SG & pti siɾ-i-$\emptyset$ & \armenian{պտի սիրի} &pti siɾ-e-$\emptyset$-ɾ & \armenian{պտի սիրէր}\\
1PL & pti siɾ-e-nkʰ & \armenian{պտի սիրէնք} &pti siɾ-$\emptyset$-i-nkʰ & \armenian{պտի սիրինք} \\
2PL & pti siɾ-e-kʰ & \armenian{պտի սիրէք} &pti siɾ-$\emptyset$-i-kʰ & \armenian{պտի սիրիք} \\
3PL & pti siɾ-e-n & \armenian{պտի սիրէն} &pti siɾ-$\emptyset$-i-n & \armenian{պտի սիրին}
\\
& \multicolumn{2}{l|}{{\deb} $\sqrt{}$-{\thgloss}-{\agr}}& \multicolumn{2}{l|}{{\deb} $\sqrt{}$-{\thgloss}-{\pst}-{\agr}}
\\\hline \end{tabular}
\end{table}

\translatorHD{For the debitive forms, an alternative strategy in Yerevan (Table \ref{tab:Yerevan:morpho:verb:paradigm:debitiveOther}) is to keep the verb in a constant shape with the same /-il/. Adjarian doesn't state if this form is a participle or not. Then, the debitive morph /pti/ is placed after the verb. This morph then takes the tense and agreement suffixes, thus agremeemt morphology is mobile. I don't gloss some of the involved morphs because it's not clear to me what they designate. }

\begin{table}[H]
 \centering
 \caption{Alternative forms for the debitive for the verb `to like' in the Yerevan dialect}
 \label{tab:Yerevan:morpho:verb:paradigm:debitiveOther}
 \begin{tabular}{|l|ll|ll|}
\hline 
& \multicolumn{2}{l|}{Debitive \armenian{պարտաւորական} } & \multicolumn{2}{l|}{Debitive perfect \armenian{անցեալ պարտաւորական} } \\
1SG & siɾ-il pt-i-m & \armenian{սիրիլ պտիմ} & siɾ-il pt-i-$\emptyset$ & \armenian{սիրիլ պտի} \\
2SG & siɾ-il pt-i-s & \armenian{սիրիլ պտիս} & siɾ-il pt-i-ɾ & \armenian{սիրիլ պտիր} \\
3SG & siɾ-il pt-i-$\emptyset$ & \armenian{սիրիլ պտի} & siɾ-il pt-e-ɾ & \armenian{սիրիլ պտէր} \\
1PL & siɾ-il pt-i-nkʰ & \armenian{սիրիլ պտինք} & siɾ-il pt-i-nkʰ & \armenian{սիրիլ պտինք} \\
2PL & siɾ-il pt-i-kʰ & \armenian{սիրիլ պտիք} & siɾ-il pt-i-kʰ & \armenian{սիրիլ պտիք} \\
3PL & siɾ-il pt-i-n & \armenian{սիրիլ պտին} & siɾ-il pt-i-n & \armenian{սիրիլ պտին} \\
& \multicolumn{2}{l|}{$\sqrt{}$-? {\deb}-?-{\agr}}& \multicolumn{2}{l|}{$\sqrt{}$-? {\deb}-?-{\agr}}
\\\hline \end{tabular}
\end{table}


\paragraph{Imperative and prohibitive}

\translatorHD{For the imperative 2SG, SEA adds the morph /-iɾ/ after the root for a verb like `to like' (Table \ref{tab:Yerevan:morpho:verb:paradigm:Imp}). For the 2PL, archaic SEA adds the sequence /-e-t͡sʰ-ekʰ/ after the root such that /-e-t͡sʰ/ forms the aorist stem, while /-ekʰ/ is the agreement marker. More modern registers of SEA instead just add the sequence /-ekʰ/ directly after the root. Yerevan uses similar strategies: the 2SG marker is either /-i/ or /-ɑ/. }


\begin{table}[H]
 \centering
 \caption{Imperative forms <\armenian{հրամայական}> for the verb `to like' in the Yerevan dialect}
 \label{tab:Yerevan:morpho:verb:paradigm:Imp}
 \begin{tabular}{|l|ll|ll|l|}
\hline & \multicolumn{2}{l|}{Yerevan} & \multicolumn{2}{l|}{cf. SEA} & \\
2SG & s\'iɾ-i & \armenian{սի՛րի} & siɾ-\'iɾ & \armenian{սիրի՛ր} & $\sqrt{}$-{\imp}.2{\sg}
\\
& s\'iɾ-ɑ & \armenian{սի՛րա} & & & $\sqrt{}$-{\imp}.2{\sg}
\\
2PL& siɾ-e-t͡sʰ-ekʰ& \armenian{սիրէցէք} & siɾ-e-t͡sʰ-ekʰ& \armenian{սիրեցեք} & $\sqrt{}$-{\thgloss}-{\aor}-{\imp}.2{\pl}
\\
& siɾ-ekʰ&\armenian{սիրէք} & siɾ-ekʰ&\armenian{սիրեք}& $\sqrt{}$-{\imp}.2{\pl}
 
\\\hline \end{tabular}
\end{table}

\translatorHD{For the prohibitive or negative imperative (Table \ref{tab:Yerevan:morpho:verb:paradigm:Proh}), SEA simply adds the prohibitive formative /mi/ before the imperative form. For the 2SG, Yerevan can do the same, and it also an alternative strategies of keeping the verb in a non-finite form with /-il/. For the 2PL, the agreement marker /-ekʰ/ jumps to the prohibitive marker, thus agreement morphology is mobile. } 


\begin{table}[H]
 \centering
 \caption{Negative imperative or prohibitive forms for the verb `to like' in the Yerevan dialect}
 \label{tab:Yerevan:morpho:verb:paradigm:Proh}
 \begin{tabular}{|l|lll|lll|}
\hline & \multicolumn{3}{l|}{Yerevan} & \multicolumn{3}{l|}{cf. SEA} \\
2SG & m\'i siɾ-i & \armenian{մի՛ սիրի} & {\proh} $\sqrt{}$-{\imp}.2{\sg} & m\'i siɾ-iɾ & \armenian{մի՛ սիրիր} & {\proh} $\sqrt{}$-{\imp}.2{\sg} \\
 & m\'i siɾ-ɑ & \armenian{մի՛ սիրա} & {\proh} $\sqrt{}$-{\imp}.2{\sg} & && \\
& mi siɾ-il & \armenian{մի սիրիլ} & {\proh} $\sqrt{}$-? & && \\
 
2PL & m-\'ekʰ siɾ-il& \armenian{մէ՛ք սիրիլ} & {\proh}-{\imp}.2{\pl} $\sqrt{}$-? & m\'i siɾ-ekʰ& \armenian{մի՛ սիրեք} & {\proh} $\sqrt{}$-{\imp}.2{\pl} \\
 & s\'iɾ-il m-ekʰ & \armenian{սի՛րիլ մէք} & $\sqrt{}$-? {\proh}-{\imp}.2{\pl} & &
& 
\\\hline \end{tabular}
\end{table}

\paragraph{Non-finite forms}

\translatorHD{Finally, Adjarian lists the following non-finite forms of this verb (participles or converbs) in Table \ref{tab:Yerevan:morpho:verb:paradigm:participle}. I give SEA forms for just some of them because it's unclear to me what these Yerevan participles mean. Note that the past participle is also called the perfective converb. } 

\begin{table}[H]
 \centering
 \caption{Participles or converbs <\armenian{դերբայներ}> for the verb `to like' in the Yerevan dialect}
 \label{tab:Yerevan:morpho:verb:paradigm:participle}
 \begin{tabular}{|ll|lll|lll|}
\hline & & \multicolumn{3}{l|}{Yerevan} & \multicolumn{3}{l|}{cf. SEA} \\
 Infinitive & \armenian{անորոշ} & siɾ-e-l & \armenian{սիրէլ} & $\sqrt{}$-{\thgloss}-{\infgloss} & siɾ-e-l & \armenian{սիրել} & $\sqrt{}$-{\thgloss}-{\infgloss} \\
 Present & \armenian{ներկայ} & siɾ-e-l-on & \armenian{սիրէլոն} & $\sqrt{}$-{\thgloss}-{\infgloss}-? & & & \\
 Past & \armenian{անցեալ} & siɾ-el & \armenian{սիրէլ} & $\sqrt{}$-{\perfcvb} & siɾ-el & \armenian{սիրել} & $\sqrt{}$-{\perfcvb} \\
& & siɾ-e & \armenian{սիրէ}& $\sqrt{}$-{\perfcvb} & & 
& 
\\\hline \end{tabular}
\end{table}

\begin{adjarianpage}\label{page:43}\end{adjarianpage}% should be 43

\subsubsection{Other conjugation classes}

 The conjugation class of CA /-il/ <\armenian{իլ}> is also inflected this way (Table \ref{tab:Yerevan:morpho:verb:xosum}). 
 
\begin{table}[H]
 \centering
 \caption{Partial paradigm of the CA /-il/ <\armenian{իլ}> conjugation class for `to speak' in the Yerevan dialect} 
 \label{tab:Yerevan:morpho:verb:xosum}
 \begin{tabular}{|l|ll|ll|l|}
 \hline & \multicolumn{2}{l|}{Yerevan }& \multicolumn{2}{l|}{cf. SEA }
 &\\ 
 {\infgloss} & & & χos-e-l & \armenian{խոսել} & $\sqrt{}$-{\thgloss}-{\infgloss}\\
 {\prs} 1{\sg} & χos-um e-m & \armenian{խօսում էմ} &χos-um e-m & \armenian{խոսում եմ} & $\sqrt{}$-{\impfcvb} {\aux}-1{\sg} \\
 {\pst} {\impf} 1{\sg} & χos-um $\emptyset$-i-$\emptyset$- & \armenian{խօսում ի} & χos-um ej-i-$\emptyset$ & \armenian{խոսում էի}& $\sqrt{}$-{\impfcvb} {\aux}-{\pst}-1{\sg} \\
{\pst} {\perf} 1{\sg} & χos-e-t͡sʰ-i-$\emptyset$& \armenian{խօսէցի} & χos-e-t͡sʰ-i-$\emptyset$ & \armenian{խոսեցի} & $\sqrt{}$-{\thgloss}-{\aor}-{\pst}-1{\sg} \\
{\imp} 2{\sg} & χ\'os-i & \armenian{խօ՛սի} &χos-\'iɾ & \armenian{խոսի՛ր} & $\sqrt{}$-{\imp}.2{\sg}\\
& χ\'os-ɑ & \armenian{խօ՛սա} & && 
\\ \hline 
 \end{tabular}

\end{table}
 
 As for the CA /ɑl/ <\armenian{ալ}> conjugation class, it keeps the style of Old Armenian in the perfective and elsewhere 
 (Table \ref{tab:Yerevan:morpho:verb:al}). 
 
\begin{table}[H]
 \centering
 \caption{Partial paradigm of the CA /-ɑl/ <\armenian{ալ}> conjugation class for `to cough' in the Yerevan dialect}
 \label{tab:Yerevan:morpho:verb:al}
 \begin{tabular}{|l|ll|ll|l|}
 \hline & \multicolumn{2}{l|}{Yerevan }& \multicolumn{2}{l|}{cf. SEA }
 & \\ 
 {\infgloss} & & & hɑz-ɑ-l & \armenian{հազալ}& $\sqrt{}$-{\thgloss}-{\infgloss} \\
 {\prs} 1{\sg} & hɑz-əm e-m & \armenian{հազըմ էմ} &hɑz-um e-m & \armenian{հազում եմ}& $\sqrt{}$-{\impfcvb} {\aux}-1{\sg} \\
 {\pst} {\impf} 1{\sg} & hɑz-əm $\emptyset$-i-$\emptyset$&\armenian{հազըմ ի} & hɑz-um ej-i & \armenian{հազում էի}& $\sqrt{}$-{\impfcvb} {\aux}-{\pst}-1{\sg} \\
{\pst} {\perf} 1{\sg} & hɑz-ɑ-t͡sʰ-i-$\emptyset$& \armenian{հազացի} & hɑz-ɑ-t͡sʰ-i-$\emptyset$ & \armenian{հազեցի} & $\sqrt{}$-{\thgloss}-{\aor}-{\pst}-1{\sg} \\
{\prs} {\perf} 1{\sg} & hɑz-ɑ-t͡sʰ-el e-m & \armenian{հազացէլ էմ} & hɑz-ɑ-t͡sʰ-el e-m & \armenian{հազացել եմ} & $\sqrt{}$-{\thgloss}-{\aor}-{\perfcvb} {\aux}-1{\sg} \\
{\imp} 2{\sg} & hɑz-\'ɑ & \armenian{հազա} &hɑz-\'ɑ & \armenian{հազա՛} & $\sqrt{}$-{\thgloss} \\
 \hline 
 \end{tabular}

\end{table}
\subsubsection{Morphological details and diachronic changes}

In verbal conjugation, the following circumstances are notable. 

\paragraph{Present 3SG copula or auxiliary}\label{sec:Yerevan:morpho:verb:morphodetials:copula}


The present 3{\sg} for the verbal copula is /ɑ/ <\armenian{ա}>. And according to this, all the verbs conjugate in the third person in this same form (Table \ref{tab:Yerevan:morpho:verb:3sgA}, sentence \ref{ex:Yerevan:morpho:verb:3sgA:sent}) . 


\begin{table}[H]
 \centering
 \caption{Present 3{\sg} auxiliary as /ɑ/ <\armenian{ա}> in the Yerevan dialect}
 \label{tab:Yerevan:morpho:verb:3sgA}
 \begin{tabular}{|l|ll|ll|}
 \hline & \multicolumn{2}{l|}{Yerevan }& \multicolumn{2}{l|}{cf. SEA }
 \\ 
 `he likes' & siɾ-um ɑ & \armenian{սիրում ա} & siɾ-um e & \armenian{սիրում է} \\
 `he brings' & beɾu-m ɑ & \armenian{բէրում ա} & beɾ-um e & \armenian{բերում է} \\
 `he says' & ɑs-um ɑ & \armenian{ասում ա} & ɑs-um e & \armenian{ասում է} \\
 `he speaks' & χos-um ɑ & \armenian{խօսում ա} & χos-um e & \armenian{խոսում է} \\
 \hline & \multicolumn{2}{l|}{$\sqrt{}$-{\impfcvb} {\aux} }& \multicolumn{2}{l|}{$\sqrt{}$-{\impfcvb} {\aux}} 
 \\
\hline 
 \end{tabular}

\end{table}

\begin{exe}
 \ex Yerevan \gll zɾit͡s ɑ ɑn-um \\
 conversation {\aux} do-{\impfcvb} \\ 
 \trans `He's doing a conversation.' \label{ex:Yerevan:morpho:verb:3sgA:sent}\\
 \armenian{զրից ա անում} 

\end{exe}

\paragraph{Deletion of /e/ before past imperfective /i/}

In the imperfective, the /e/ <\armenian{է}> sound is deleted next to /i/ <\armenian{ի}> (Table \ref{tab:Yerevan:morpho:verb:other:pastEdeletion} and sentence \ref{sent:yerevan:morpho:verb:other:edeletion}).
 

\begin{table}[H]
 \centering
 \caption{Deletion of the vowel /e/ before the past suffix /i/ in the Yerevan dialect}
 \label{tab:Yerevan:morpho:verb:other:pastEdeletion}
 \begin{tabular}{|l|ll|ll|l|}
 \hline & \multicolumn{2}{l|}{Yerevan }& \multicolumn{2}{l|}{cf. SEA } & \\
 `I was liking' & siɾ-um $\emptyset$-i-$\emptyset$ & \armenian{սիրում ի} & siɾ-um ej-i-$\emptyset$ & \armenian{սիրում էի} & $\sqrt{}$-{\impfcvb} {\aux}-{\pst}-1{\sg} \\
 `you were bring' & beɾ-um $\emptyset$-i-ɾ & \armenian{բէրում իր} & beɾ-um ej-i-ɾ & \armenian{բէրում էիր} & $\sqrt{}$-{\impfcvb} {\aux}-{\pst}-2{\sg} \\
 \hline 
 \end{tabular}

\end{table}

 \begin{exe}
 \ex Yerevan\label{sent:yerevan:morpho:verb:other:edeletion} \begin{xlist}
 \ex \gll du kə siɾ-$\emptyset$-i-ɾ \\
 you.{\sg} {\fut} like-{\thgloss}-{\pst}-2{\sg} \\
 \trans `You would like it.' \\
 \armenian{դու կը սիրիր}
 \ex \gll nɑ kə siɾ-e-$\emptyset$-ɾ \\
 he {\fut} like-{\thgloss}-{\pst}-3{\sg}\\
 \trans `He would like it.' \\
 \armenian{նա կը սիրէր}
 \end{xlist}
 \end{exe}
 

\translatorHD{This was also discussed in Tables \ref{tab:Yerevan:morpho:verb:paradigm:pastImpfIndc} and \ref{tab:Yerevan:morpho:verb:paradigm:subjPast}. }

\paragraph{Debitive morphology}

In the debitive, the form CA /piti/ <\armenian{պիտի}> has shortened to /pti/ <\armenian{պտի}>, as it has in other dialects. \translatorHD{See Table \ref{tab:Yerevan:morpho:verb:paradigm:complexSubjunctive} for its paradigm.}


In the second form of the debitive, the formative /pti/ <\armenian{պտի}> is inflected, and the verb stays uninflected; whereas in the first form, ... 

\begin{adjarianpage}\label{page:44}\end{adjarianpage}% should be 44

... it is the verb which is inflected, while the /pti/ <\armenian{պտի}> doesn't change. The second form is rare in other places. It does not exist at all in the /kə/ \armenian{կը} branch. \translatorHD{For paradigms, see Table \ref{tab:Yerevan:morpho:verb:paradigm:complexSubjunctive} vs. Table \ref{tab:Yerevan:morpho:verb:paradigm:debitiveOther}. }

\paragraph{Imperative morphology}
In the second form of the imperative, there is the ending /ɑ/ <\armenian{ա}> (Table \ref{tab:Yerevan:morpho:verb:other:ImpA}). It is unique to the Etchmiadzin area. While in the dialect of Yerevan proper, the forms are different.\footnote{\translatorHD{For the verb to `fill', the more accurate segmentation in SEA is /lə-t͡sʰɾ-u/ where the /-t͡sʰ-/ is the causative suffix.}} 
\translatorHD{See also Table \ref{tab:Yerevan:morpho:verb:paradigm:Imp}. }




\begin{table}[H]
 \centering
 \caption{Use of imperative 2SG forms with final /-ɑ/ in the Yerevan dialect (Etchmiadzin area) vs. using /-i/ in Yerevan proper}
 \label{tab:Yerevan:morpho:verb:other:ImpA}
 \begin{tabular}{|l|ll|ll|ll| }
 \hline & \multicolumn{2}{l|}{Etchmiadzin }& \multicolumn{2}{l|}{Yerevan proper }& \multicolumn{2}{l|}{cf. SEA } \\
 `want!' & uz-\'ɑ & \armenian{ուզա ՛}& uz-\'i & \armenian{ուզի ՛} & uz-\'iɾ & \armenian{ուզիր}\\
 `turn on!' & vɑr-\'ɑ & \armenian{վառա՛} & vɑr-\'i & \armenian{վառի՛} & vɑr-\'iɾ & \armenian{վառիր}\\
 `fill!' & lt͡sʰɾ-\'ɑ & \armenian{լցրա՛} & lt͡sʰɾ-u & \armenian{լցրու} & lət͡sʰɾ-\'u & \armenian{լցրու}\\
 `roast!' & ɑʁɑnd͡z-\'ɑ & \armenian{աղանձա՛} && & ɑʁɑnd͡z-\'iɾ & \armenian{աղանձիր}\\
 `carry!' & ʃɑlɑk-\'ɑ & \armenian{շալակա՛} & && ʃɑlɑk-\'iɾ & \armenian{շալակիր}\\
 & \multicolumn{2}{l|}{$\sqrt{}$-{\imp}.2{\sg}} & \multicolumn{2}{l|}{$\sqrt{}$-{\imp}.2{\sg}} & \multicolumn{2}{l|}{$\sqrt{}$-{\imp}.2{\sg}} 
 \\
 \hline 
 \end{tabular}


\end{table}

\paragraph{Prohibitive morphology and mobile ordering} 

The forms /m\'ekʰ siɾil/ <\armenian{մէ՛ք սիրիլ}> `don't like' or the opposite order /s\'iɾil mekʰ/ <\armenian{սի՛րիլ մէք}> (where the plural marker of the verb has passed onto the particle) are also used in the Karabakh dialect. \translatorHD{See Table \ref{tab:Yerevan:morpho:verb:paradigm:Proh}.}

\paragraph{Present participle }

The form /-on/ <-\armenian{օն}> of the present participle (Table \ref{tab:Yerevan:morpho:verb:other:presentParticiple}). It is not used in any other locations. 



\begin{table}[H]
 \centering
 \caption{Present participle in the Yerevan dialect }
 \label{tab:Yerevan:morpho:verb:other:presentParticiple}
 \begin{tabular}{|l|ll|ll| }
 \hline & \multicolumn{2}{l|}{Yerevan }& \multicolumn{2}{l|}{cf. SEA } \\
 `with liking' & siɾ-e-l-on & \armenian{սիրէլօն} & siɾ-e-l-ov & \armenian{սիրելով} \\
 `with saying' & ɑs-e-l-on& \armenian{ասէլօն} & ɑs-e-l-ov & \armenian{ասելով} \\
 `with going' & etʰ-ɑ-l-on & \armenian{էթալօն} & jeɾtʰ-ɑ-l-ov & \armenian{երթալով} 
 \\
 & \multicolumn{2}{l|}{$\sqrt{}$-{\thgloss}-?}& \multicolumn{2}{l|}{$\sqrt{}$-{\thgloss}-{\ins}}
 \\
 \hline 
 \end{tabular}


\end{table}
\paragraph{Past participle (perfective converb) and auxiliary-induced changes }

The form of the past participle is /siɾ-el/ <\armenian{սիրէլ}> `liked' from Classical /siɾe̯ɑl/ <\armenian{սիրեալ}>.\footnote{\translatorHD{At least SEA, this non-finite form is more accurately called the perfective converb, as is done on the Eastern Armenian National Corpus. }} \translatorHD{See paradigms in Table \ref{tab:Yerevan:morpho:verb:paradigm:participle}. }


\begin{table}[H]
 \centering
 \caption{Pre-auxiliary form of the past participle (perfective converb) with /-el/ in the Yerevan dialect }
 \label{tab:Yerevan:morpho:verb:other:pastParticiplePreAux}
 \begin{tabular}{|l|ll|ll| }
 \hline & \multicolumn{2}{l|}{Yerevan }& \multicolumn{2}{l|}{cf. SEA } \\
`I have liked' & siɾ-el e-m & \armenian{սիրէլ էմ} & siɾ-el e-m & \armenian{սիրել եմ} \\
`I have brought' & beɾ-el e-m & \armenian{բէրէլ էմ} & beɾ-el e-m & \armenian{բերել եմ} 
 
 \\
 & \multicolumn{2}{l|}{$\sqrt{}$-{\perfcvb} {\aux}-1{\sg}}& \multicolumn{2}{l|}{$\sqrt{}$-{\perfcvb} {\aux}-1{\sg}}
 \\
 \hline 
 \end{tabular}


\end{table}


This form is used when the auxiliary is placed after it. But when the auxiliary is before it, the final /l/ <\armenian{լ}> is shortened to form the participle /siɾ-e/ <\armenian{սիրէ}>, /beɾ-e/ <\armenian{բէրէ}>, and so on. See (\ref{sent:yerevan:morpho:verb:other:ppastParticiplePostAux}). 

\begin{exe}
\ex Yerevan \label{sent:yerevan:morpho:verb:other:ppastParticiplePostAux}\begin{xlist}
\ex \gll jes e-m siɾ-e \\
I {\aux}-1{\sg} like-{\perfcvb} \\ 
\trans `\textbf{I} (focused) have liked (not someone else).'
\\ \armenian{յէս էմ սիրէ} 
\ex \gll en ɑ beɾ-e \\
that {\aux} bring-{\perfcvb} \\ 
\trans `He has brought that.'
\\ \armenian{էն ա բէրէ}
\ex \gll siɾt ɑ ɑɾ-e ek-e\\
heart {\aux} do-{\perfcvb} come-{\perfcvb} \\ 
\trans \translatorHD{I'm not sure what this means}
\\ \armenian{սիրտ ա արէ էկէ}
\end{xlist}
\end{exe}


\translatorHD{This shortening process is described in-depth for Tehrani Iranian Armenian as a type of phonosyntactic process in \citet[ch. 3.3]{DolatianEtAl-prep-IranianGrammar}.}

\paragraph{Irregular imperfective converbs for monosyllabic verbs }

For monosyllabic verbs, the base of the present and imperfective stem is formed with the formative /-is/ <\armenian{իս}> instead of the form /-um/ <\armenian{ում}> (Table \ref{tab:Yerevan:morpho:verb:other:imperfIrregIs}). \translatorHD{Contrast these irregular verbs with /-is/ against regular verbs with /-um/ in Table \ref{tab:Yerevan:morpho:verb:paradigm:presentIndc}. Such irregular verbs are monosyllabic in the infinitive like SEA /ɡ-ɑ-l/ `to give'. }



\begin{table}[H]
 \centering
 \caption{Irregular imperfective converbs for monosyllabic verbs with /-is/ in the Yerevan dialect }
 \label{tab:Yerevan:morpho:verb:other:imperfIrregIs}

 \begin{tabular}{|l|ll|ll| }
 \hline & \multicolumn{2}{l|}{Yerevan }& \multicolumn{2}{l|}{cf. SEA } \\
 \hline Infinitive &&& \multicolumn{2}{l|}{$\sqrt{}$-{\thgloss}-{\infgloss}} \\
 `to come' & & & ɡ-ɑ-l & \armenian{գալ} \\
 `to give' & & & t-ɑ-l & \armenian{տալ} \\
 `to cry' & & & l-ɑ-l & \armenian{լալ} \\
 \hline Present 1SG & \multicolumn{4}{l|}{$\sqrt{}$-{\thgloss}-{\infgloss}-{\impfcvb} {\aux}-1{\sg}} \\
`I come' & ɡ-ɑ-l-is e-m & \armenian{գալիս էմ} & ɡ-ɑ-l-is e-m & \armenian{գալիս եմ} \\
`I give' & t-ɑ-l-is e-m & \armenian{տալիս էմ} & t-ɑ-l-is e-m & \armenian{տալիս եմ} \\
`I cry' & l-ɑ-l-is e-m & \armenian{լալիս էմ} & l-ɑ-l-is e-m & \armenian{լալիս եմ} \\
\hline Past Impf. 1SG & \multicolumn{4}{l|}{$\sqrt{}$-{\thgloss}-{\infgloss}-{\impfcvb} {\aux}-{\pst}1{\sg} } \\
`I was coming' & ɡ-ɑ-l-is $\emptyset$-i$-\emptyset$ & \armenian{գալիս ի} & ɡ-ɑ-l-is ej-i-$\emptyset$ & \armenian{գալիս էի} \\
`I was giving' & t-ɑ-l-is $\emptyset$-i$-\emptyset$ & \armenian{տալիս ի} & t-ɑ-l-is ej-i-$\emptyset$ & \armenian{տալիս էի} \\
`I was crying' & l-ɑ-l-is $\emptyset$-i$-\emptyset$ & \armenian{լալիս ի}& l-ɑ-l-is ej-i-$\emptyset$ & \armenian{լալիս էի} \\
 \hline 
 \end{tabular}


\end{table}




But when the auxiliary verb is before it, then the final /s/ <\armenian{ս}> is deleted (\ref{sent:Yerevan:morpho:verb:other:irregImpfIs}). \translatorHD{This process is also described for Iranian Armenian as the same process for the perfective converb \citep[ch. 3.3]{DolatianEtAl-prep-IranianGrammar}.} 

\begin{exe}
\ex Yerevan \label{sent:Yerevan:morpho:verb:other:irregImpfIs} \begin{xlist}
\ex \gll jes e-m ɡ-ɑ-l-i \\
 I {\aux}-1{\sg} come-{\thgloss}-{\infgloss}-{\impfcvb} \\
 \trans `\textbf{I} am coming (as opposed to someone else).' \\
\ex \gll χi e-s l-ɑ-l-i \\
 why {\aux}-2{\sg} cry-{\thgloss}-{\infgloss}-{\impfcvb} \\
 \trans `Why are you crying?' \\
 \armenian{խի՞ էս լալի}
\ex \gll t͡ʃʰ-e-s t-ɑ-l-i \\
 {\neggloss}-{\aux}-2{\sg} give-{\thgloss}-{\infgloss}-{\impfcvb} \\
 \trans `Won't you give?' \\
 \armenian{չէ՞ս տալի} 
\end{xlist} 
\end{exe}

\paragraph{Mobile negation }


In negative forms, the negative particle can be either before or after the verb (Table \ref{tab:Yerevan:morpho:verb:other:negationMobile}). 


\begin{table}[H]
 \centering
 \caption{Mobile negation in in the Yerevan dialect }
 \label{tab:Yerevan:morpho:verb:other:negationMobile}
 \begin{tabular}{|l|ll|ll| l| }
 \hline & \multicolumn{2}{l|}{Yerevan }& \multicolumn{2}{l|}{cf. SEA }& \\
`I don't want' & t͡ʃʰ-\'e-m uz-um & \armenian{չէ՛մ ուզում} & t͡ʃʰ-\'e-m uz-um & \armenian{չեմ ուզում} & {\neggloss}-{\aux}-1{\sg} want-{\impfcvb} \\
 & \'uz-um t͡ʃʰ-e-m & \armenian{ո՛ւզում չէմ} & & & want-{\impfcvb} {\neggloss}-{\aux}-1{\sg} \\
 `I wouldn't want' & t͡ʃʰ-$\emptyset$-\'i-$\emptyset$ uz-um & \armenian{չի՛ ուզում} & t͡ʃʰ-ej-\'i-$\emptyset$ uz-um & \armenian{չէի ուզում} & {\neggloss}-{\aux}-{\pst}-1{\sg} want-{\impfcvb} \\
 & \'uz-u-m t͡ʃʰ-$\emptyset$-i-$\emptyset$ & \armenian{ո՛ւզում չի} & & & want-{\impfcvb} {\neggloss}-{\aux}-{\pst}-1{\sg} 
 \\
 ? & t͡ʃʰ-\'e-m uz-il & \armenian{չէ՛մ ուզիլ} & && want-? {\neggloss}-{\aux}-1{\sg} \\
 & \'uz-il t͡ʃʰ-e-m & \armenian{ո՛ւզիլ չէմ} & & & {\neggloss}-{\aux}-1{\sg} want-? \\
 `I didn't like' & t͡ʃ-siɾ-\'e-t͡sʰ-i-$\emptyset$ & \armenian{չսիրէ՛ցի} & t͡ʃə-siɾ-e-t͡sʰ-\'i-$\emptyset$ & \armenian{չսիրեցի} & {\neggloss}-$\sqrt{}$-{\thgloss}-{\aor}-{\pst}-1{\sg} \\
 & siɾ-\'e-t͡sʰ-i-$\emptyset$ vot͡ʃʰ & \armenian{սիրէ՛ցի վոչ} & & & $\sqrt{}$-{\thgloss}-{\aor}-{\pst}-1{\sg} no
 \\
 \hline 
 \end{tabular}


\end{table}

\section{Subdialects}
 \subsection{Bayazit}

 For the Bayazit subdialect, the main characteristics are the following.

 \translatorHD{Note that throughout his manuscript, Adjarian alternates in calling this language New Bayazet vs. just Bayazit (and the two names are spelled differently in Armenian: \armenian{Նոր-Բայազէտ} vs. \armenian{Պայազիտ}). That makes it unclear if he's always referring to the same dialect when he's mentioning such a name. }

 \subsubsection{Diphthongs}

 Whereas in the Yerevan dialect, the Classical sounds /e/ <\armenian{ե}> and /o/ <\armenian{ո}> have merged into modern /e/ <\armenian{է}> and /o/ <\armenian{օ}>, the Bayazit subdialect distinguishes these with a diphthongal pronunciation (Table \ref{tab:Yerevan:subdialect:bayazit:diphthong}). (Read these as /mienkʰ/ <\armenian{միէնք}>, /ənduont͡sʰ/ <\armenian{ընդուօնց}>. Besides these, it also includes the vowel /æ/ <\armenian{ա}̈>. 

 
\begin{table}[H]
 \centering
 \caption{Diphthongs in the Bayazit subdialect of the Yerevan dialect }
 \label{tab:Yerevan:subdialect:bayazit:diphthong}
 \begin{tabular}{|l|ll|ll| }
 \hline & \multicolumn{2}{l|}{Bayazit }& \multicolumn{2}{l|}{cf. SEA } \\
`we' & mi̯enkʰ & \armenian{մենք} & meŋkʰ & \armenian{մենք} 
\\
`from those' & əndu̯ont͡sʰ & \armenian{ընդոնց} & & 
\\
 \hline 
 \end{tabular}


\end{table}

 \subsubsection{Voiced aspirated stops and affricates}

The Classical consonants /b ɡ d d͡z d͡ʒ/ <\armenian{բ գ դ ձ ջ}> have become /bʰ ɡʰ dʰ d͡zʰ d͡ʒʰ/ <\armenian{բՙ գՙ դՙ ձՙ ջՙ}>. 

\begin{adjarianpage}\label{page:45}\end{adjarianpage}% should be 45

 \subsubsection{Changing Classical /h/ <\armenian{հ}> to /χ/ <\armenian{խ}>}
 
The Classical sound /h/ \armenian{հ} has turned into /χ/ \armenian{խ} without exception (Table \ref{tab:Yerevan:subdialect:bayazit:hkh}, sentence \ref{sent:yerevan:subdialect:bayazit:kh}). 


\begin{table}[H]
 \centering
 \caption{Changing CA /h/ <\armenian{հ}> to /χ/ <\armenian{խ}> in the Bayazit subdialect of the Yerevan dialect}
 \label{tab:Yerevan:subdialect:bayazit:hkh}
 \begin{tabular}{|l|ll|ll|ll| }
 \hline & \multicolumn{2}{l|}{Classical Armenian } & \multicolumn{2}{l|}{> Bayazit }& \multicolumn{2}{l|}{cf. SEA } \\
 `Armenian' & hɑi̯ & \armenian{հայ} & χɑj & \armenian{խայ} & hɑj & \armenian{հայ} \\
 `bread' & hɑt͡sʰ & \armenian{հաց} & χɑt͡sʰ & \armenian{խաց} & hɑt͡sʰ & \armenian{հաց} \\
 `father' & hɑi̯ɾ & \armenian{հայր} & χeɾ & \armenian{խէր} & hɑjɾ & \armenian{հայր} \\
 `to preserve' & pɑhel & \armenian{պահել} & pɑχel & \armenian{պախել} & pɑhel & \armenian{պահել} \\
 `fear' & ɑh& \armenian{ահ} & ɑχ & \armenian{ախ} & ɑh & \armenian{ահ} \\
 
 \hline 
 \end{tabular}


\end{table}

\begin{exe}
 \ex \label{sent:yerevan:subdialect:bayazit:kh}\begin{xlist}
 \ex SEA (approximates the original form of the Bayazit sentence)
 \gll im hoɾ hɑɾsɑnik-i-n hiŋɡ hɑv hɑtɑv \\
 my father.{\gen} wedding-{\dat}-{\defgloss} five chicken died \\
 \trans `Five chickens died for my father's wedding.'\\
 \armenian{իմ հօր հարսանիքին հինգ հաւ հատաւ (սատկեցաւ}) 
 \ex Bayazid subdialect (Yerevan dialect) 
 \gll im χoɾ χɑɾsnis-i-n χinɡ χɑv χɑtɑv \\
 my father.{\gen} wedding-{\dat}-{\defgloss} five chicken died \\
 \trans `Five chickens died for my father's wedding.'\\
 \armenian{իմ խօր խարսնիսին խինգ խավ խատավ}
 \end{xlist}
\end{exe}

 \subsubsection{Repetition of the definite article}

After vowel-final words, the definite article (\armenian{դիմորոշ յօդը}) is repeated. 



\begin{table}[H]
 \centering
 \caption{Repetition of the definite article after vowel-final words in the Bayazit subdialect of the Yerevan dialect}
 \label{tab:Yerevan:subdialect:bayazit:defRep}
 \begin{tabular}{|l|ll|ll|ll| }
 \hline & \multicolumn{2}{l|}{Bayazit }& \multicolumn{2}{l|}{instead of }& \multicolumn{2}{l|}{cf. SEA } \\
 `the cat' &kɑtu-nə& \armenian{կատունը} &kɑtu-n & \armenian{կատուն}&kɑtu-n & \armenian{կատուն}
 \\
 `the bee' &mi̯eʁu-nə& \armenian{մեղունը} &mi̯eʁu-n & \armenian{մեղուն}&meʁu-n & \armenian{մեղուն}
 \\
 `the horse' &d͡zʰi-nə& \armenian{ձՙինը} &d͡zi-n & \armenian{ձին}&d͡z\armenian{ՙ}i-n & \armenian{ձին}
 \\
 & $\sqrt{}$-{\defgloss} & & & & $\sqrt{}$-{\defgloss}& 
 \\
 \hline 
 \end{tabular}


\end{table}

\translatorHD{In SEA, the definite article is /-n/ after vowels, and /-ə/ after consonants. Although Adjarian describes this process as the definite article repeating, I think it's more accurate to state that Bayazit has replaced the post-vocalic allomorph /-n/ with /-nə/. See \citet{Dolatian-prep-Definite} for discussion on how the definite article's morphophonology is complicated. }



 \subsubsection{Regularization of monosyllabic verbs} 

For monosyllabic verbs, the stem of the present and imperfective uses /-um/ <\armenian{ում}> instead of /-is/ <\armenian{իս}>, thus fully regular (Table \ref{tab:Yerevan:subdialect:bayazit:imperfIrregIs}). \translatorHD{Contrast against Table \ref{tab:Yerevan:morpho:verb:other:imperfIrregIs}. }


\begin{table}[H]
 \centering
 \caption{Regularization of imperfective converbs for monosyllabic verbs with /-um/ in the Bayazit subdialect of the Yerevan dialect }
 \label{tab:Yerevan:subdialect:bayazit:imperfIrregIs}

 \begin{tabular}{|l|ll|ll| }
 \hline & \multicolumn{2}{l|}{Bayazit }& \multicolumn{2}{l|}{cf. SEA } \\
 \hline Infinitive &&& \multicolumn{2}{l|}{$\sqrt{}$-{\thgloss}-{\infgloss}} \\
 `to come' & & & ɡ-ɑ-l & \armenian{գալ} \\
 `to give' & & & t-ɑ-l & \armenian{տալ} \\
 `to cry' & & & l-ɑ-l & \armenian{լալ} \\
 \hline Present 3SG & \multicolumn{4}{l|}{$\sqrt{}$-{\thgloss}-{\infgloss}-{\impfcvb} {\aux}} \\
`He comes' & ɡʰ-ɑ-l-um ɑ & \armenian{գՙալում ա} & ɡ-ɑ-l-is e & \armenian{գալիս է} \\
`He gives' & t-ɑ-l-um ɑ & \armenian{տալում ա} & t-ɑ-l-is e & \armenian{տալիս է} \\
`He cries' & l-ɑ-l-um ɑ& \armenian{լալում ա} & l-ɑ-l-is e & \armenian{լալիս է} \\
 \hline 
 \end{tabular}


\end{table}

 \subsubsection{Repetition of the auxiliary when the auxiliary has moved} 

In those circumstances where, in the Yerevan dialect, the auxiliary verb is placed before the verb, the Bayazit subdialect places the auxiliary before and after, causing the repetition of the auxiliary (\ref{sent:yerevan:subdialect:bayazit:auxRep}). 

\begin{exe}
 \ex Bayazit subdialect of the Yerevan dialect \label{sent:yerevan:subdialect:bayazit:auxRep}
 \begin{xlist}
 \ex \gll ji̯es i̯e-m uz-um i̯e-m \\
 I {\aux}-1{\sg} want-{\impfcvb} {\aux}-1{\sg} \\
 \trans `\textbf{I} want (as opposed to someone).'\\
 \armenian{յես եմ ուզում եմ} 
 \ex \gll t͡ʃʰ-i̯e-s bʰi̯eɾ-um i̯e-s \\
 {\neggloss}-{\aux}-2{\sg} bring-{\impfcvb} {\aux}-2{\sg} \\
 \trans `Don't/won't you bring?' \\
\armenian{չե՞ս բՙերում ես} 
 \ex \gll t͡ʃʰ-i uz-um ɑ \\
 {\neggloss}-{\aux}.3{\sg} want-{\impfcvb} {\aux} \\
 \trans `He doesn't want.' \\
\armenian{չի ուզում ա} 
 \ex \gll t͡ʃʰ-i̯e-nkʰ uz-um i̯e-nkʰ \\
 {\neggloss}-{\aux}-1{\pl} want-{\impfcvb} {\aux}-1{\pl} \\
 \trans `We don't want.' \\
\armenian{չենք ուզում ենք} 
 \ex \gll t͡ʃʰ-$\emptyset$-i-ɾ uz-um $\emptyset$-i-ɾ \\
 {\neggloss}-{\aux}-{\pst}-2{\sg} want-{\impfcvb} {\aux}-{\pst}-2{\sg} \\
 \trans `You wouldn't want.' \\
\armenian{չի՞ր ուզում իր} 
 \ex \gll t͡ʃʰ-e-$\emptyset$-ɾ χos-um e-$\emptyset$-ɾ\\
 {\neggloss}-{\aux}-{\pst}-3{\sg} speak-{\impfcvb} {\aux}-{\pst}-3{\sg} \\
 \trans `You wasn't speaking.' \\
\armenian{չէր խօսում էր} 
\end{xlist}
\end{exe}


 \subsubsection{Past participle or perfective converb with /-i̯eɾ, -e/} 

The past participle ends in /-i̯eɾ/ <\armenian{եր}>, like in the dialects of the /kə/ <\armenian{կը}> branch (\ref{sent:yerevan:subdialect:bayazit:pastParticiple}).

\begin{exe}
 \ex Bayazit subdialect of the Yerevan dialect \label{sent:yerevan:subdialect:bayazit:pastParticiple}
\begin{xlist}
\ex \gll ɑs-i̯eɾ ɑ \\
say-{\perfcvb} {\aux} \\
\trans `He has said.' \\
\armenian{ասեր ա} 
\ex \gll tɑɾ-i̯eɾ ɑ \\
take-{\perfcvb} {\aux} \\
\trans `He has taken.' \\
\armenian{տարեր ա} 
\ex \gll ek-i̯eɾ ɑ \\
come-{\perfcvb} {\aux} \\
\trans `He has come.' \\
\armenian{էկեր ա} 
\end{xlist}
\end{exe}
 

But when the auxiliary moves back, the form is /-e/ \ref{sent:yerevan:subdialect:bayazit:pastParticipleShift}.

\begin{exe}
 \ex Bayazit subdialect of the Yerevan dialect \label{sent:yerevan:subdialect:bayazit:pastParticipleShift}
\begin{xlist} 
\ex \gll en ɑ tɑɾ-e ɑ \\
that {\aux} take-{\perfcvb}\\
\trans `He has taken that.' \\
\armenian{էն ա տարէ}
\ex \gll t͡ʃʰ-i̯e-m ek-e \\
{\neggloss}-{\aux}-1{\sg} come-{\perfcvb} \\
\trans `I have not come come.' \\
 \armenian{չեմ էկէ}
 \end{xlist}
\end{exe}

 \subsubsection{Past participle or perfective converb with /-uk/} 

The past participle is also formed with the formative /-uk/ <\armenian{ուկ}>, but only in passives (\armenian{կրաւորական}) and middle verbs (\armenian{չէզոք}). \translatorHD{I think he means for intransitives in general.}



\begin{table}[H]
 \centering
 \caption{Past participles with /-uk/ <\armenian{ուկ}> in the Bayazit subdialect of the Yerevan dialect}
 \label{tab:Yerevan:subdialect:bayazit:uk}
 \begin{tabular}{|l|ll|ll| l | }
 \hline & \multicolumn{2}{l|}{> Bayazit }& \multicolumn{2}{l|}{cf. SEA } & \\
 `slept' & pɑrk-uk & \armenian{պառկուկ} & pɑrk-el & \armenian{պառկել} & $\sqrt{}$-{\perfcvb} \\
 `written' & ɡʰɾ-uk & \armenian{գՙրուկ} & ɡəɾ-el & \armenian{գրել} & $\sqrt{}$-{\perfcvb}\\
 `washed' & lv-ɑ-t͡sʰ-uk & \armenian{լվացուկ} & ləv-ɑ-t͡sʰ-el & \armenian{լվացել} & $\sqrt{}$-{\thgloss}-{\aor}-{\perfcvb}\\
 `he has ploughed' & χi̯eɾk-uk & \armenian{խերկուկ ա} & heɾk-el & \armenian{հերկել} & $\sqrt{}$-{\perfcvb} {\aux}\\
 \hline 
 \end{tabular}


\end{table} 

\subsection{Astabad subdialect}

\subsubsection{Characteristics of the Astabad subdialect}
For the Astabad subdialect, the main characteristics are the following.

\paragraph{Phonetic `purity'}

The vowels and consonants are pronounced purely. 

\paragraph{Lack of /h/ > /χ/ change} 

The Classical sound /h/ <\armenian{հ}> sound has not changed to /χ/ <\armenian{խ}>. 

\paragraph{Pre-tonic vowel deletion}


 Before stress, vowels sometimes fall (Table \ref{tab:Yerevan:subdialect:Astabad:pretonic}).%\footnote{\translatorHD{For the word `gathered', Adjarian suggested that the ancestor is a word <\armenian{հաւաքուած}>. But based on this word's morphology, I doubt that it's from Classical Armenian instead of SEA. }}


\begin{table}[H]
 \centering
 \caption{Pre-tonic deletion of vowels in the Astabad subdialect of the Yerevan dialect}
 \label{tab:Yerevan:subdialect:Astabad:pretonic}
 \begin{tabular}{|l|ll|ll|ll| }
 \hline & \multicolumn{2}{l|}{Classical Armenian } & \multicolumn{2}{l|}{> Astabad }& \multicolumn{2}{l|}{cf. SEA } \\
 `evening' & eɾekoi̯ & \armenian{երեկոյ} & ɾikun & \armenian{րիկուն} & jeɾeko & \armenian{երեկո} \\
 `grave' & ɡeɾezmɑn & \armenian{գերեզման} & ɡɾezmɑn & \armenian{գրէզման} & ɡeɾezmɑn & \armenian{գերեզման} \\
 `to gather' & hɑ{wɑ}kʰel &\armenian{հաւաքել} & & & hɑvɑkʰel & \armenian{հավաքել} \\
 `gathered' & & & vɑkʰvɑt͡s & \armenian{վաքված} & hɑvɑkʰvɑt͡s & \armenian{հավաքված} \\

 \hline 
 \end{tabular}


\end{table}

\paragraph{Ablative suffixes}

The ablative is formed with both the suffixes /-e/ <\armenian{է}> and /-it͡sʰ/ <\armenian{ից}>. 

\paragraph{Present copula as /ɑ/}

The entire present conjugation of the copula verb is pronounced with the vowel /ɑ/ (

Table \ref{tab:Yerevan:subdialect:Astabad:copula}). 


\begin{table}[H]
 \centering
 \caption{Present copula with the vowel /ɑ/ in the Astabad subdialect of the Yerevan dialect }
 \label{tab:Yerevan:subdialect:Astabad:copula}
 \begin{tabular}{|l|ll|ll| }
 \hline & \multicolumn{2}{l|}{Astabad }& \multicolumn{2}{l|}{cf. SEA } \\
 1SG `I am' & ɑ-m & \armenian{ամ} & e-m & \armenian{եմ} \\
 2SG `you are' & ɑ-s & \armenian{աս} & e-s & \armenian{ես} \\
 3SG `he is' & ɑ & \armenian{ա} & e & \armenian{է}\\
 1PL `we are' & ɑ-nkʰ & \armenian{անք} & e-ŋkʰ & \armenian{ենք} \\
 2PL `you are' & ɑ-kʰ & \armenian{աք} & e-kʰ & \armenian{եք} \\
 3PL `they are' & ɑ-n & \armenian{ան} & e-n & \armenian{են} \\
 & \multicolumn{2}{l|}{{\aux}-{\agr}}& \multicolumn{2}{l|}{{\aux}-{\agr}}
\\ \hline 
 \end{tabular}


\end{table}
\paragraph{Past perfective 1SG suffix as /-m/}

he past perfective gets the ending /m/ <\armenian{մ}> (Table \ref{tab:Yerevan:subdialect:Astabad:pstm}). \translatorHD{He means this subdialect uses the suffix /-m/ for past perfective 1{\sg} marker, whereas Yerevan and SEA use a zero suffix. }


\begin{table}[H]
 \centering
 \caption{Use of past perfective 1SG marker /-m/ <\armenian{մ}> in the Astabad subdialect of the Yerevan dialect }
 \label{tab:Yerevan:subdialect:Astabad:pstm}
 \begin{tabular}{|l|ll|ll| l| }
 \hline & \multicolumn{2}{l|}{Astabad }& \multicolumn{2}{l|}{cf. SEA } & \\
 `I saw' & tes-ɑ-m & \armenian{տէսամ} & tes-ɑ-$\emptyset$ &\armenian{տեսա} & $\sqrt{}$-{\pst}-1{\sg} \\
 `I went' & ɡn-ɑ-t͡sʰ-i-m & \armenian{գնացիմ} & ɡən-ɑ-t͡sʰ-i-$\emptyset$ &\armenian{գնացի} & $\sqrt{}$-{\thgloss}-{\aor}-{\pst}-1{\sg} \\
 `I came' & ek-ɑ-m & \armenian{էկամ} & jek-ɑ-$\emptyset$ &\armenian{եկա} & $\sqrt{}$-{\pst}-1{\sg} \\

 \hline 
 \end{tabular}


\end{table} 

\subsubsection{Classification}

Based on these characteristics, we see that the Astabad dialect lies between the dialects of Yerevan, Karabakh, and Julfa. The first two characteristics belong to Yerevan, the third to Karabakh, and the last three characteristics cause Astabad to resemble... 
\begin{adjarianpage}\label{page:46}\end{adjarianpage}% should be 46

... the Julfa dialect. But we didn't want to add this dialect to Julfa, because the main characteristic of the Julfa dialect is missing in the Astabad subdialect; the main characteristic of Julfa is that the use of the present stem with /-mɑn/ <\armenian{ման}>, such as in (\ref{sent:yerevan:subdialect:bayazit:notJulfa}). While the ablative in /-e/ <\armenian{է}>, the use of /-m/ <\armenian{մ}> in the perfective, and the vowel /ɑ/ <\armenian{ա}> vowel in the copular verb are also found in the other vernaculars (Karabakh, Shamakhi, Tabriz). 


\begin{exe}
 \ex Julfa dialect \gll
 ɡn-ɑ-mɑn ɑ-m \\
 go-{\thgloss}-{\impfcvb} {\aux}-1{\sg} \\
 \trans `I go.'\label{sent:yerevan:subdialect:bayazit:notJulfa} \\
 \armenian{գնաման ամ} 
 
\end{exe}


\subsection{Tabriz subdialect}


For the Tabriz dialect, the main characteristics are the following. 


\subsubsection{Miscellaneous segmental differences} 

The Classical sound /h/ <\armenian{հ}> became /χ/ <\armenian{խ}>, as in the Bayazit subdialect. But the vowels /i̯e, u̯o/ <\armenian{ե, ո}> of the Bayazit are not found in Tabriz subdialect. In many places, we see the /æ/ <\armenian{ա}̈> vowel (Table \ref{tab:Yerevan:subdialect:Tabriz:ae}). 





\begin{table}[H]
 \centering
 \caption{Vowel /æ/ <\armenian{ա}̈> in the Tabriz subdialect of the Yerevan dialect }
 \label{tab:Yerevan:subdialect:Tabriz:ae}
 \begin{tabular}{|lll|lll| }
 \hline \multicolumn{3}{| l|}{Classical and SEA }&\multicolumn{3}{l|}{> Tabriz } \\
 
 `luck' & bɑχt & \armenian{բախտ} & `spouse' & bæχt & \armenian{բա̈խտ} 
 \\
 `old' & dɑh & \armenian{դահ} & `old animal' & tæχ & \armenian{տա̈խ} 
 
 \\

 \hline 
 \end{tabular}


\end{table} 

\subsubsection{Change of Classical /u̯ɑ/ to /ivɑ/} 


The Classical diphthong /u̯ɑ/ <\armenian{ուա}> has changed to /ivɑ/ <\armenian{իվա}> in Astabad. There are no other examples than in (Table \ref{tab:Yerevan:subdialect:tabriz:ua}). 

\begin{table}[H]
 \centering
 \caption{Change from CA /u̯ɑ/ <\armenian{ուա}> to /ivɑ/ <\armenian{իվա}> in the Astabad subdialect of the Yerevan dialect}
 \label{tab:Yerevan:subdialect:tabriz:ua}
 \begin{tabular}{|l|ll|ll|ll|}
 \hline & \multicolumn{2}{l|}{Classical Armenian}& \multicolumn{2}{l|}{> Tabriz }& \multicolumn{2}{l|}{cf. SEA }
 \\
 \armenian{՝}rope' & t͡ʃʰu̯ɑn & \armenian{չուան} & t͡ʃʰivɑn & \armenian{չիվան} & t͡ʃʰəvɑn &\armenian{չվան} \\
 `confused' & ʃu̯ɑɾ & \armenian{շուար} & ʃivɑɾ & \armenian{շիվար} & ʃəvɑɾ & \armenian{շվար} \\
 \hline
 \end{tabular}
 
\end{table}

\subsubsection{Devoicing of Classical voiced stops} 

The Classical voiced sounds /b ɡ d d͡z d͡ʒ/ <\armenian{բ գ դ ձ ջ}> have become voiceless /p k t t͡s t͡ʃ/ <\armenian{պ կ տ ծ ճ}>. This basic change brings this subdialect closer to the dialects of Urmia-Maragha and Van.


 
\subsubsection{Definite suffix allomorphy}





When the words for `horse' and `egg' take the definite suffix, the forms are as in Table \ref{tab:Yerevan:subdialect:tabriz:def}. \translatorHD{He means that the definite suffix is /-jə/ for these forms instead of /-n/. Note that suffix marks definiteness in the modern vernaculars but it's a distal deixis marker in CA. }



\begin{table}[H]
 \centering
 \caption{Use of definite suffix /-jə/ <-\armenian{յը}> in the Astabad subdialect of the Yerevan dialect}
 \label{tab:Yerevan:subdialect:tabriz:def}
 \begin{tabular}{|l|ll|ll|ll|}
 \hline & \multicolumn{2}{l|}{Classical Armenian}& \multicolumn{2}{l|}{> Tabriz }& \multicolumn{2}{l|}{cf. SEA }
 \\
 `horse' & d͡zi& \armenian{ձի} & & & d͡zi& \armenian{ձի} \\
 `the horse' & d͡zi-n& \armenian{ձի} & t͡si-jə& \armenian{ծիյը} & d͡zi-n& \armenian{ձնի} \\
 `egg' & d͡zu& \armenian{ձու} & & & d͡zu& \armenian{ձու} \\
 `the egg' & d͡zu-n& \armenian{ձի} & t͡su-jə &\armenian{ծույը} & d͡zu-n& \armenian{ձուն} 
 \\ \hline
 \end{tabular}
 
\end{table}

\subsubsection{Use of past perfective 1SG marker /-m/}


The past perfective tense of verbs is similar to the present because it uses the suffix /-m/ <\armenian{մ}>, just as in the dialects of Urmia-Maragha, Khoy, and Julfa. \translatorHD{He means this is for the 1{\sg} marker which is /-m/ in SEA for the present, but /-m/ for the past perfective in SEA. In Tabriz, the form /-m/ is used for both tenses. }\footnote{\translatorHD{For the SEA forms given for `I opened' and `I gave', these are actually more Colloquial Eastern Armenian forms. The purely standard and prescriptive versions would be /bɑt͡sʰ-e-t͡sʰ-i-$\emptyset$, təv-e-t͡sʰ-i-$\emptyset$/ <\armenian{բացեցի, տվեցի}>. I suspect that in Adjarian's time, the forms he provided were more common. } }


\begin{table}[H]
 \centering
 \caption{Use of past perfective 1{\sg} marker /-m/ <\armenian{մ}> in the Astabad subdialect of the Yerevan dialect}
 \label{tab:Yerevan:subdialect:tabriz:pastM}
 \begin{tabular}{|l|ll|ll|l|}
 \hline & \multicolumn{2}{l|}{Tabriz }& \multicolumn{2}{l|}{cf. SEA }
 & \\
 `I said' &ɑs-ɑ-m & \armenian{ասամ} & ɑs-ɑ-t͡sʰ-i-$\emptyset$ & \armenian{ասացի} & $\sqrt{}$-({\thgloss}-{\aor})-{\pst}-1{\sg}\\
 `I opened' &pɑt͡sʰ-ɑ-m & \armenian{պացամ} & bɑt͡sʰ-i-$\emptyset$ & \armenian{բացի} & $\sqrt{}$-{\pst}-1{\sg}\\
 `I gave' &təv-ɑ-m & \armenian{տըվամ} & təv-i-$\emptyset$ & \armenian{տվի} & $\sqrt{}$-{\pst}-1{\sg}\\
 `I saw' &tes-ɑ-m & \armenian{տէսամ} & tes-ɑ-$\emptyset$ & \armenian{տեսա} & $\sqrt{}$-{\pst}-1{\sg}\\
 `I said' &kn-ɑ-t͡sʰ-i-m & \armenian{կնացիմ} & ɡən-ɑ-t͡sʰ-i-$\emptyset$ & \armenian{գնացի} & $\sqrt{}$-{\thgloss}-{\aor}-{\pst}-1{\sg}\\
 \hline
 \end{tabular}
 
\end{table}

\subsubsection{Past participle or perfective converb with /-eɾ/ <\armenian{էր}> and rhotic deletion} 



The past participle is formed with the formative /-eɾ/ <\armenian{էր}> particle, just as in the Bayazit subdialect.

\begin{exe}
 \ex \begin{xlist}
 \ex Tabriz \gll 
 χɑs-eɾ ɑ. en ɑ ɑs-e \\
 reach-{\perfcvb} {\aux}. that {\aux} say-{\perfcvb} \\
 \trans `He has reached. He has said that.' \\
 
 \armenian{խասէր ա. էն ա ասէ}
 \ex SEA \gll 
 hɑs-el e. nɑ e ɑs-el \\
 reach-{\perfcvb} {\aux}. that {\aux} say-{\perfcvb} \\
 \trans `He has reached. He has said that.' \\
 \armenian{հասել է. նա է ասել}
 \end{xlist}
\end{exe}

An interesting example of the loss of the /ɾ/ <\armenian{ր}> in the Tabriz subdialect is found in the word /ɑχə/ <\armenian{ախը}> that is borrowed from Turkish (elsewhere as <\armenian{ախըր}> /ɑχəɾ/ `but').\footnote{\translatorHD{It's not clear to me what the source Turkish word is. }} 
\subsubsection{Repetition of nasals in the 3SG past perfective} 




In the 3SG past perfective of some verbs, the sound /n/ <\armenian{ն}> is repeated. But for other persons, there is no change. 


\begin{table}[H]
 \centering
 \caption{Repetition or gemination of a nasal in the 3{\sg} past perfective 1{\sg} of some verbs in the Astabad subdialect of the Yerevan dialect}
 \label{tab:Yerevan:subdialect:tabriz:gemNasal}
 \begin{tabular}{|l|ll|ll|l|}
 \hline & \multicolumn{2}{l|}{Tabriz }& \multicolumn{2}{l|}{cf. SEA }
 & \\
 `to go' & & & ɡən-ɑ-l & \armenian{գնալ} & $\sqrt{}$-{\thgloss}-{\infgloss} \\
 `he went' &kənn-ɑ-t͡sʰ-$\emptyset$-$\emptyset$ & \armenian{կըննաց}& ɡən-ɑ-t͡sʰ-$\emptyset$-$\emptyset$ & \armenian{գնաց} & $\sqrt{}$-{\thgloss}-{\aor}-{\pst}-3{\sg} \\
 `to stay' & & & mən-ɑ-l & \armenian{մնալ} & $\sqrt{}$-{\thgloss}-{\infgloss} \\
 `he stayed' &mənn-ɑ-t͡sʰ-$\emptyset$-$\emptyset$& \armenian{մըննաց} & mən-ɑ-t͡sʰ-$\emptyset$-$\emptyset$ & \armenian{մնաց} & $\sqrt{}$-{\thgloss}-{\aor}-{\pst}-3{\sg} \\
 \hline
 \end{tabular}
 
\end{table}

\section{Literature}
 
 Despite the importance of the Yerevan dialect, both it and its three subdialects have still not been studied. However, three are many publications that use this dialect, which... 

\begin{adjarianpage}\label{page:47}\end{adjarianpage}% should be 47

... can provide ample material for study. Among these publications, we note a few of the main ones.

{\litoverview}
 
\begin{itemize}
 \item Literature with the Yerevan dialect
 \begin{itemize}
 \item General Yerevan dialect:
\begin{itemize} 
\item \armenian{Խաչատուր Աբովեանի երկերը- Մոսկուա}, 1897
 ... \armenian{Տ. Նաւասարդեանցի Հայ ժողով. հէքիաթ. հաւաքածուն}
 \item \armenian{Ա. Աբեղեան} - \armenian{Սասնայ ծռեր. Ազգ. Հանդ}., \armenian{Թ. էջ} 117-143
\item \armenian{Ե. Լալայեան} - \armenian{Բորչալուի գաւառի բանաւոր գրականութիւնը. Ազգ. Հանդ}., \armenian{ԺԱ. էջ} 33-124
\end{itemize}
\item Bayazit subdialect
\begin{itemize} 
\item \armenian{Ս. Փիլոյեան} - \armenian{Կորած Մարգարիտ. Տփխիս}, 1880
\item \armenian{Տ. Նաւասարդեան} - \armenian{Հայ ժող. հէքիաթ. Ե. էջ} 79-80
\end{itemize}
\item Astabad subdialect
\begin{itemize} 
\item \armenian{Մ. Աբեղեան} - \armenian{Ոգիներ, տես Տ. Նաւասարդեանցի Հէքիաթներու հաւաքածուն, հտ. Է. էջ} 24-35
\item \armenian{Մ. Աբեղեան} - \armenian{Առածներ. անդ էջ} 76-88
\item \armenian{Գ. Շիրմազանեան – Իմ նշանածը Արազն ա. Կռունկ}, 1861, \armenian{էջ} 266-282
\end{itemize}
\item Tabriz subdialect
\begin{itemize} 
\item \armenian{Գ. Շիրմազանեան} - \armenian{Ազգային հարսանիք. Կռունկ}, 1861, \armenian{էջ} 426-440
\end{itemize}
\item Lori vernacular 
\begin{itemize} 
\item \armenian{Գ. Քուչարեանց} - \armenian{Լօռու գիւղական կեանքից պատկերներ. Փորձ, Դ}. N 4, \armenian{յաւել. էջ} 1-12
\item \armenian{Յ. Ղազարեան} - \armenian{Եադՙասա. Թիֆլիս}, 1904
\item \armenian{Տ. Նաւասարդեան} - \armenian{Հայ ժող. հէքիաթներ. Ե. էջ} 32-78
\end{itemize}
\end{itemize}
\end{itemize}

\section{Text samples}
{\sampleoverview}
 
\subsection{Yerevan dialect}
Adjarian's Source: \armenian{Տ. Նաւասարդեանցի Հայ ժող. հէք}., \armenian{Է էջ} 42

\armenian{Հարս ու կէսուր էն ըլնում. դրանք բօլ սիսէռ էն ունէնում։ Էդ հարսը շատ սիսէռակէր ա ըլնում։ Ասսու իրան օրը սիսէռը գօղանում էր, տանում թաքուն էփում ուտում։ Մի՛ն, է՛րկու, հի՛նգ, տա՛սը, օրէն մի օրը կէսուրը վարավուրդ ա անում, տէնում իրանց սիսէռի տօպրակը կէսքն ա ընկէ։ – Կա չըկա, ասում ա ինքն իրան, էս մէր հարսի բանէրն ա։}

\begin{adjarianpage}\label{page:48}\end{adjarianpage}% should be 48

\armenian{Թէ կէսուրը սատանա էր, հարսն էլ պակաս չէր նրանից. ալբիալը ֆահմում ա, վօր կէսուրը գլխի ա ընկէ։ – Ի՛նչ անէմ, ի՛նչ չանէմ, ասում ա, վօր ինձ վրա սայիշ չը տանի։}

– \armenian{Օրէն մի օրը տունն ավլէլիս մի հատ սիսէռ ա գըտնում, վէր ա ունում, տանում կէսօրը շանց տալի, ասում}.

– \armenian{Ընթիկ ունի, քնթիկ ունի, կատվի նման դնչիկ ունի, նա՛նի, յա՛րաբ էս ի՞նչ ա։}

\armenian{Կէսուրն էս վօր լսում ա, ինքն իրան ասում ա}.

– \armenian{Ֆօղն իմ գլխին, վօ՛նց ի էս խէղճ հարսի մէղքը վէր ունում նա ըսկի չի էլ իմանում սիսէռն ի՛նչ ա ի՛նչ չի. ո՛ւր մնաս թէ գօղանա}

\subsection{New Bayazet subdialect}

– \armenian{Էտ ի՞նչ ա, խա՛րսէ, խպարտցեր ես, բՙարեվ էլ չես տալում ես. յօ՛լըտ ծանդըրցեր ա։}

– \armenian{Վա՜յ Խօոէ, դՙո՞ւյ ես. չըթա̈յմնէցի։}

– \armenian{Է ի՞մալ ես, դՙո՞ր ես էթում ես, հո՞ւստ ես գՙալում ես։}

– \armenian{Գՙացեր ի խէոանցըս տուն. ընդոնց կշտից եմ գՙալում եմ։}

– \armenian{Լսա ասին խէրըտ խիվանդ ա. մկա ի՞մալ ա։}

– \armenian{Քա՛ էնի շուտուց խիվանդ ա, խօ թա̈զա̈ չէ. իմ խէրը հա̈լա̈ խէրվընէ խիվանդ ա. էնէնց պառկուկ ա։}

– \armenian{Է՜, խաս (հաց) կէրե՞ր ես. հէտա (ահաւաղիկ) Վոսկին էլ էկավ, հարի իրեքով էթանք մեր տունը խաց ուտենք։}

– \armenian{Չէ, չեմ կանա, բՙան կա, պտի էթամ Գՙիրքորին տէնամ։}

\subsection{Astabad subdialect}

Adjarian's source: Ibid., page 34. 

\armenian{Ղօրմածիկ Մարդումէնց Պօղօսը կասէր. «Մին ամառ յէս կալ իմ անում. կալս լուսահօքի Տէր-Աբրահամէնց բախչի տակին էր։ Մին քշէր, կալումը քնէլ ալ. մին քիչ էլ ցուրտ էր. կալս յէտ էր ընկէլ, աշունքանում էր. առավտադէմ վէր կացամ, տէնամ լուսնակալաս ա. լուսնակը հէնց ա թէքվէլ ա Մասսա ղօլը՝ շօղշօղում ա. քամին էլ ցուրտ ցուրտ փչում ա։ Մին էլ տէնամ՝ հրէսիկանկ մին քանի գէլ, մին սիւրու չարունք աղաքնին ան արէլ, ճըվճըվացնէլէն, բօռացնէլէն Չայի ղօլէն քշում բէրում ան, վօր տանէն դպա գէտը, գէտը լցնէն},


\begin{adjarianpage}\label{page:49}\end{adjarianpage}% should be 49

\armenian{խէխտէն յա ուտէն։ Էն չարունքը ընէնց մին ճըխճըվում ին, հաբա՜յ տալիս, վօր մարթի ջանը զարզանդում էր։ Ըտէնց ճըվճըվցնէլէն քշում ին, վօր լուսալուս էր, ժամհարը ժամէրը տըվավ, մին էլ տէնամ էլ զադ չը կա. լուսնակն էլ ասրի քամակն ա մննում»։}

\subsection{Tabriz subdialect}

Adjarian's source: Written by Ms. Satenik Melik-Babajanian (\armenian{օր. Սաթենիկ Մէլիք-Բաբաջանեան}; \translatorHD{SEA: /sɑtʰenik melikʰ bɑbɑd͡ʒɑnjɑn/}).

\armenian{Մէնք վէց քուր ինք, համա ախպէր չունինք, խէրս էլ խօքյասըբ մարթ չէր, օղօմի իրան, քյաթխօդա էր։ Մէ խէտ (մէկ անգամ) էլավ կըննաց Էչմիյածին ուխտ. վօր Աստված իրան մէ տղա տա։ Ընդէղըմը իմ խօրս մէ քանի մասունք ին տըվէ, խէլս էլ իրան ուխտը արէր էր, ընդօնք էլ վէրցէր էր ճանապար էր ընգէ։ Հա̈լա̈ Թա̈րվիզ չը խասած մէզի կրէր էր քի մէ օր առաչ էկէք Մա̈րա̈նդ. մէնք էլ մէր մէ քանի էլօվ գիւնօվ էլանք կնացինք, խէտնէրս էլ մէ խատ վօխչար տարանք. հէնց վօր խէրս էկավ սախ-սալամաթ խասավ, վօտի տակը մէ մատաղ արանք։ Էս խէտ վէրցանք էկանք Թա̈րխիզ, մէ մատաղ էլ ըստէ արանք, հհըմմէն տէրտէրնէրին խաց տվանք։}

\armenian{Սօրա (յետոյ) էտ մասունքները տրանբ մէ կութու մէչ, տարանք մէր բալախանէն, տակը մէ թա̈միզ շօր քըցանք, կութին արանք թախչէն, իրէցին էլ մէ թօզ (շղարշ) քաշանք, էս խէտ հա̈ր (ամէն մի) քշէ (գիշէր) էթըմ ինք աղօթք ինք անըմ, մում ինք վառըմ, մէր պարէկամնէրից էլ ուզօղը կալիս էր ուխտ էր անըմ։}

\armenian{Էլ ինչ ասէմ, ա՛յ բալից, էս խէտ հա̈ր շափաթ ըտու տակի շօրը տանըմ ինք լվանըմ ինք, պէրըմ ինք քըցըմ ինք տակը։}

\armenian{Ըտունց ընցավ մէ տարի. մէրս է՛լը մէ ախչիկ պէրավ, համա խէրս էլ ըսկի զադ չարավ, մասունքնէրը էլ յէտ չըղորկավ. ասավ «լավ չէ՛լնի վօր մասունքնէրը յէտ ըղըրկէմ։ Աստված ինչ վօր տալիս ա՝ թօղ տա, իրան կամքն ա»։}

\armenian{Էլը ըտուց ընցավ մէ խինգ վէզ տարի. էտ փստիկ քուր, վօր վէրէն ուխտ ին արէ, այ բալէս, վօչ էր էլէ, մէ կէծակ էր շատ էր չարութուն անըմ. մէրս էլ խօ ընդուր ըսկի չէր ուզում, աշկի կրօղն էր։ Մէ օր էնքան ծէծավ, էնքան ուշունց}

\begin{adjarianpage}\label{page:50}\end{adjarianpage}% should be 50

\armenian{տվավ էտ ախճկան, վօր լափ հալից ընկավ. խէղճ խօրս էլ լափ կուսամա̈րգ} (??? \armenian{վշտամէռ) արավ մէրս «Տու հէ՞ր կնացիր ուխտ արար, հէ՞ր էսէնց փիս ախճիկ էլավ»։ Խէրս էլ ըսկի չէր խօսըմ։ Էս խէէա մութնը ընզավ. ամառ էր, մէնք էլ հըմէնանքս հայաթըմն ինք քընըմ. մէկ էլ տէսանք բիրդան մէ զըրընգօց էկավ. էն սահաթին հըմէնանքս զարհնանք, նըստանք ասանք «բա՛, մօրս վիզանօցն էր, կօղը թալավ, վօր սօրա էթա վէրցնի»։ Խէրս էլավ վօր էթա մէր կոնշինէրին զարհնացնի, տէսավ հըմէնն էլ զարթուն էն. էն պլպլալի զադն էլ տէսէր էն։ Էս խէտ խէրս մէ քան տըղէրքի խէտ էկավ, կըննաց բալախանէն, հա̈ր տէղը ման էկան, ըսկի զար չը քթան։ Էս խէտ էկան էս թէխը (այս կողմը, ասդին}), \armenian{տէսան մասունքի կութին պացվէր ա, մէչի մասունքը չը կա։ Էս խէտ խէրս մէզի կանչավ, ասավ. «Էն պլպլալի զադը հէնց մասունքն էր վօր թռավ». մօրս ասավ. «էնքան էսօր էն ախճկան ուշուն տվար, վօր մասունքը թռավ»։}

\armenian{Ըտուց մէ քանի վախդ սօրա խէրս էտ պանը կրավ (գրեց) Էչմիյածինի տէրտէրնէրին. տէրտէրնէրն էլ ընդէղից կրան (գրեցին) քի «էտ մասունքնէրը ըտեղից թռէր էն էկէր էն ըստէ. հա̈լբա̈թ ծէր տունը լավ մաք}?\armenian{ր չէն պախէ»։ Սօրա խէրս իմացավ վօր էն օրը մէրս էնքան էն քրօչս ծէծավ ուշուն տվավ վօր, ընձի էլ էնքան ասավ հէ՞ր կնացիր ուխտ արար, հէնց էտէնց խօսկէրի համար մասունքնէրը թռան։}

\armenian{Էտ պանէրից սօրա էլ մէր տանը խէր բարաքյաթը կտրըվավ. խէրս քանի մընմ էր՝ խաբաբ էր էլնըմ. ըտուց սօրա խէղճ խէրս չօրս տարի ապրավ, հա}?\armenian{ա վա՛յ էն ապրէլուն. քյասըբացավ պանից ընգավ մըննաց տունը կուսամա̈րգ էլավ մէռավ։ Հէնց խէրս մէռավ, սէր ծօվ տունը խանխարաբ էլավ։}

\subsection{Lori vernacular }

Adjarian's footnote: See \armenian{Փորձ}, 1880 \armenian{ապրիլ, յաւել. էջ} 5-6. The writing does not have scientific accuracy. 

– \armenian{Հը՛, Համբարձում ապէր, ասում ես ամէն ուտելեղէնի}

\begin{adjarianpage}\label{page:51}\end{adjarianpage}% should be 51

\armenian{ու խմելեղէնի էժանութին ա էլի. բա'ս բանը մնաց շորէղէնի վրա. էր հալբաթ որ թանգ կըլի էլի. քու ապրանքն ա, կը թանգացնես, որ մեր հացն ու եղն էժան գնով առնուս ու քու փթած ապրանքը մեզ վրա սաղացնես։ Դուն ու քու Աստոծը, Համբարձում ջան, դրուստ չե՞մ ասում։}

– \armenian{Է՛հ, Միքէլ բիձէն ես, էլի սեւ ու սիպտակ չես հարցնում։}

– \armenian{Հը՛՛ ղնդռիկ Ասլօ}', \armenian{ի՞նչ ես վէրի ծէրին բազմվել. տօ' էդ քու տե՞ղն ա, որ բռնել ես։}

– \armenian{Տօ}', \armenian{քօո շուն, քեղանից էլ պակաս մարդ եմ, որ քու գլխին եմ նստած. էդ քու դառդակ կարկաժը չի' վեր ունում հա՞, իմ բոյին քունիցը երկար, իմ շորերը քու շորերիցն նոր, ու իմ չիփխի ճիպօտը էրկու քու չիփխի չափ։ Մի ասա տէնունք, ի՞նքս ա պակաս նստած տեղիս գօրա։}

– \armenian{Ը՞մն ինչդ ա պակաս, շատ քիչ բան ա պակաս. ծալդ ա պակաս, ծալդ. թէ բէդամաղ չես ըլիլ՝ կօկօղդ դառդակ ա, խելք չի ունիս, խելք. երկար բոյգ ու չիփխի ճիպոտը ինչներուս ա պատքը։ Ամա}', \armenian{աղպէր. դրուստն էլ ասած վէրի ծէրին նստօղի ամէն բանը պէտք ա թամամ ըլի։ Թող ամէն մարդ իրան տեղը բռնէ ոնց որ իրան պատիւն կուզէ։}

– \armenian{Ո՞նց ջոգենք պատւաւոր մարկերանցն, բիձա Միքէլ։}

– \armenian{Ես ըլիմ իմ Աստոծը, օրէնքն էն ա, ո'ւմ կնիկը սիրուն ա՝ թող նա վէրի ծէրին նստի. ում կնիկը գէշ ա, նրա տեղը դռան տակն ա։}

– \armenian{Թող ըթէնց ըլի, լաւ ասիր, բիձա ջան. էս սհաթին ասած պէտք ա գլուխ բերենք։}

– \armenian{Տօ}', \armenian{էդ սարսաղ Մխօին ի՞նչ էք լսում. դրա էդ գոնչ գլխիցն խէլօվ բան դուրս կը գա՞յ. շաշ շաշ դուս ա տալի էլի։}

– \armenian{Հը ծուռտիկ շէդօ}', \armenian{բանդ խարաբ ա հա}', \armenian{դռան տակին էջ տեղ չի ունիս։ Վեր կաց. շուտ արա' կորի'ր տեղիցդ։}

– \armenian{Հալա մի էս զնդռիկ Ասլօին ու ծուռտիկ Շէդօին քաջ տուէք իրանց տաք տեղից ներքեւ, որ մի տեղը իստակուի, յետոյ կը տեսնենք թէ ի՞նչ են վայ տալի դրանց շաշ գլխին։}





\chapter{Tbilisi}





\section{Background}

\begin{adjarianpage}\label{page:52}\end{adjarianpage}% should be 52

The Tbilisi dialect is currently spoken only in the city of Tbilisi. But it can be thought that this dialect was previously spread out across all of Georgia. Bit by bit, Georgian took over its borders. Today, the Tbilisi dialect is slowly becoming lost, being conquered from one side by the spread of Georgian and Russian, and the other side by the spread of the literary Armenian language. 

\section{Phonology}
\subsection{Segmental inventory}

The sound system of the Tbilisi dialect contains exactly those sounds that are found in the Yerevan dialect, minus the sound /f/ <\armenian{ֆ}>, and plus the sound /q/ <\armenian{ղՙ}>. 

The vowels /œ, ʏ/ <\armenian{էօ, իւ}> do not exist, as in the Yerevan dialect. 

The consonants /b ɡ d d͡z d͡ʒ/ <\armenian{բ գ դ ձ ջ}> are pronounced with a perfect voicing, and with much more voicing than in the Yerevan dialect, like the French <b, g, d> sounds. 

The Old Armenian sounds */p k t t͡s t͡ʃ/ <\armenian{պ կ տ ծ ճ}> are preserved here with their perfect voicelessness; but because of the influence of Georgian, these sounds are accompanied with a glottal closure (\armenian{կոկորդի սեղմումով}), which makes it hard to forget the Tbilisi speaker and it gives a very characteristic color to their consonants. The sounds /pʰ kʰ tʰ t͡sʰ t͡ʃʰ/ <\armenian{փ ք թ ց չ}> are completely simple and strong. 

Because the sound /f/ <\armenian{ֆ}> does not exist in this dialect, then all foreign words change this sound to \armenian{փ} /pʰ/. 


\begin{table}[H]
 \centering
 \caption{Changing borrowed /f/ to /pʰ/ in the Tbilisi dialect }
 \label{tab:tbilisi:phonology:segmet:f}
 \begin{tabular}{|l|lll|ll| }
 \hline & \multicolumn{3}{l|}{Source language } &\multicolumn{2}{l|}{> Tbilisi } \\
 
 `coffee' &Arabic & <qahwa> & \textarab{قهوة}& qɑpʰɑ & \armenian{ղՙափա} \\
 `fortune-telling' &Persian & <fâl> & \textarab{فال} & pʰɑl & \armenian{փալ} \\
 `tablecloth' &Persian & <sufra> & \textarab{سفره} & supʰɾɑ & \armenian{սուփրա} \\
 
 \hline 
 \end{tabular}


\end{table} 
 

The sound \armenian{ղՙ} /q/ is entirely a Georgian sound and it represents the Georgian letter \includegraphics[scale=0.15]{images/georgianLetter.png}.\footnote{\translatorHD{Adjarian provides a Georgian letter \includegraphics[scale=0.15]{images/georgianLetter.png} to signify the uvular stop. I don't know Georgian so I discussed the matter with David Erschler and Thomas Wier. Based on our discussion, the letter that Adjarian provided is probably a misprint because it doesn't exactly match an existing Georgian letter. It may have been a misprint of the letter for the sound /q\armenian{ʼ}/ (the letter q'ari; Mkhedruli <ყ>, Asomtavruli <Ⴗ>) or for the sound /q\textsuperscript{(}ʰ\textsuperscript{)}/ (the latter qari or khari; Mkhedruli <ჴ>). \label{footnote:georgian letter}}} It is pronounced as a strictly glottal /ʁ/ <\armenian{ղ}>, similar to the Arabic sound /q/.



\begin{table}[H]
 \centering
 \caption{Glottal /q/ <\armenian{ղՙ}> in the Tbilisi dialect }
 \label{tab:tbilisi:phonology:segment:q}
 \begin{tabular}{|lll| }
 \hline `from where' & vuɾqɑnt͡sʰ& \armenian{վուրղՙանց} \\
`from here' & esqɑnt͡sʰ& \armenian{էսղՙանց} \\
 \hline 
 \end{tabular}


\end{table} 

As can be seen, the Tbilisi has almost completely preserved the phonetic richness of Old Armenian. But... 


\begin{adjarianpage}\label{page:53}\end{adjarianpage}% should be 53

... this is due to the influence of the Georgian language, such that all of the dialect;s sounds bear the stamp of Georgian pronunciation. Especially the glottal pronunciation of the sounds /p k t t͡s t͡ʃ/ <\armenian{պ կ տ ծ ճ}> is Georgian. The sound /q/ <\armenian{ղՙ}>, as we know, is a pure borrowing from Georgian. The sound /f/ <\armenian{ֆ}>is absent in this dialect because it is also absent in Georgian. The same is for the sounds /œ, ʏ, æ/ <\armenian{էօ, իւ, ա}̈> . 



\subsection{Sound changes}
Among the phonetic changes of the Tbilisi dialect, whether limited or very general, the following are the characteristic ones for this dialect. 

\subsubsection{Monophthong vowels}
\paragraph{Classical Armenian /e/ <\armenian{ե}> }


Classical Armenian /e/ <\armenian{ե}> has become /ji/ <\armenian{յի}> when word-initial in monosyllabic words (Table \ref{tab:tbilisi:phono:change:EIniitalMono}). 


\begin{table}[H]
 \centering
 \caption{Change from word-initial monosyllabic CA /e/ <\armenian{ե}> to /ji/ <\armenian{յի}> in the Tbilisi dialect}
 \label{tab:tbilisi:phono:change:EIniitalMono}
 \begin{tabular}{|l|ll|ll|ll|}
 \hline & \multicolumn{2}{l|}{Classical Armenian}& \multicolumn{2}{l|}{> Tbilisi }& \multicolumn{2}{l|}{cf. SEA }
 \\
 \armenian{՝}I' & es & \armenian{ես} & jis & \armenian{յիս} & jes & \armenian{ես} \\
 \armenian{՝}when' & eɾb & \armenian{երբ} & jipʰ & \armenian{յիփ} & jeɾpʰ & \armenian{երբ} \\
 \armenian{՝}ox' & ezən & \armenian{եզն} & jizɾ & \armenian{յիզր} & jez & \armenian{եզ} \\
 \armenian{՝}oil' & eɬ & \armenian{եղ} & jiʁ & \armenian{յիղ} & juʁ & \armenian{յուղ} \\
 \hline
 \end{tabular}
 
\end{table}

At the beginning of polysyllabic words, this sound changed to /e/ <\armenian{է}> (Table \ref{tab:tbilisi:phono:change:EIniitalPoly}). 




\begin{table}[H]
 \centering
 \caption{Change from word-initial polysyllabic CA /e/ <\armenian{ե}> to /e/ <\armenian{է}> in the Tbilisi dialect}
 \label{tab:tbilisi:phono:change:EIniitalPoly}
 \begin{tabular}{|l|ll|ll|ll|}
 \hline & \multicolumn{2}{l|}{Classical Armenian}& \multicolumn{2}{l|}{> Tbilisi }& \multicolumn{2}{l|}{cf. SEA }
 \\
 \armenian{՝}iron' & eɾk\'ɑtʰ & \armenian{երկաթ} & \'eɾkɑtʰ & \armenian{է՛րկաթ} & jeɾk\'ɑtʰ & \armenian{երկաթ} \\
 \armenian{՝}face' & eɾ\'es & \armenian{երես} & \'eɾes & \armenian{է՛րէս} & jeɾ\'es & \armenian{երես} \\
 \armenian{՝}two' & eɾk\'u & \armenian{երկու} & \'eɾku & \armenian{է՛րկու} & jeɾk\'u & \armenian{երկու} \\
 \armenian{՝}dream' & eɾ\'ɑz & \armenian{երազ} & \'eɾɑz & \armenian{է՛րազ} & jeɾ\'ɑz & \armenian{երազ} \\
 \armenian{՝}child' & eɾɑχ\'ɑi̯ & \armenian{երախայ} & eɾ\'eχɑ &\armenian{էրէ՛խա} & jeɾeχ\'ɑ & \armenian{երեխա} \\
 \armenian{՝}sky' & eɾk\'in & \armenian{երկին} & \'eɾɡinkʰ & \armenian{է՛րգինք} & jeɾk\'iŋkʰ & \armenian{երկինք} \\
 \hline
 \end{tabular}
 
\end{table}

In the final syllable, meaning when it was stressed, the Classical /e/ <\armenian{ե}> becomes /i/ <\armenian{ի}> (Table \ref{tab:tbilisi:phono:change:EFinalStressed}).


\begin{table}[H]
 \centering
 \caption{Change from final CA /e/ <\armenian{ե}> to /i/ <\armenian{ի}> in the Tbilisi dialect}
 \label{tab:tbilisi:phono:change:EFinalStressed}
 \begin{tabular}{|l|ll|ll|ll|}
 \hline & \multicolumn{2}{l|}{Classical Armenian}& \multicolumn{2}{l|}{> Tbilisi }& \multicolumn{2}{l|}{cf. SEA }
 \\
 \armenian{՝}place' & teɬ & \armenian{տեղ} & tiʁ & \armenian{տիղ} & teʁ & \armenian{տեղ} \\
 \armenian{՝}night' & ɡiʃ\'eɾ & \armenian{գիշեր} & ɡ\'iʃiɾ & \armenian{գի՛շիր} & ɡiʃ\'eɾ & \armenian{գիշեր} \\
 \armenian{՝}you' & kʰez & \armenian{քեզ} & ɡiz & \armenian{քիզ} & kʰez & \armenian{քեզ} \\
 \armenian{՝}honey' & meɬəɾ & \armenian{մեղր} & miʁɾ & \armenian{միղր} & meʁəɾ & \armenian{մեղր} \\
 \armenian{՝}seed' & seɾmən & \armenian{սերմն} & siɾm & \armenian{սիրմ} & seɾm & \armenian{սերմ} \\
 
 \hline
 \end{tabular}
\end{table}

But in preceding syllables, meaning before the stressed syllable, the vowel is /e/ <\armenian{է}> (Table \ref{tab:tbilisi:phono:change:EFinalPreStressed}).


\begin{table}[H]
 \centering
 \caption{Change from pre-tonic CA /e/ <\armenian{ե}> to /e/ <\armenian{է}> in the Tbilisi dialect}
 \label{tab:tbilisi:phono:change:EFinalPreStressed}
 \begin{tabular}{|l|ll|ll|ll|}
 \hline & \multicolumn{2}{l|}{Classical Armenian}& \multicolumn{2}{l|}{> Tbilisi }& \multicolumn{2}{l|}{cf. SEA }
 \\
 \armenian{՝}to see' & tesɑn\'el & \armenian{տեսանել} & t\'esnil & \armenian{տէ՛սնիլ} & tesn\'el & \armenian{տեսնել} \\
 \armenian{՝}to bring' & beɾ\'el & \armenian{բերել} & b\'eɾil &\armenian{բէ՛րիլ} & beɾ\'el & \armenian{բերել} \\

 
 \hline
 \end{tabular}
\end{table}
 

\paragraph{Classical Armenian /o/ <\armenian{ո}> }


Classical Armenian /o/ <\armenian{ո}> became /vu/ <\armenian{վու}> at the beginning of both monosyllabic and polysyllabic words (Table \ref{tab:tbilisi:phono:change:oInitial}).


\begin{table}[H]
 \centering
 \caption{Change word-initial CA /o/ <\armenian{ո}> to /vu/ <\armenian{վու}> in the Tbilisi dialect}
 \label{tab:tbilisi:phono:change:oInitial}
 \begin{tabular}{|l|ll|ll|ll|}
 \hline & \multicolumn{2}{l|}{Classical Armenian}& \multicolumn{2}{l|}{> Tbilisi }& \multicolumn{2}{l|}{cf. SEA }
 \\
 \armenian{՝}orphan' & oɾb & \armenian{որբ} & vuɾpʰ & \armenian{վուրփ} & voɾpʰ & \armenian{որբ} \\
 \armenian{՝}son' & oɾd\'i & \armenian{որդի} & v\'uɾtʰi & \armenian{վո՛ւրթի} & voɾtʰ\'i & \armenian{որդի} \\
 \armenian{՝}that' & oɾ & \armenian{որ} & vuɾ & \armenian{վուր} & voɾtʰi & \armenian{որ} \\
 \armenian{՝}foot' & otən & \armenian{ոտն} & vut & \armenian{վուտ} & votkʰ & \armenian{ոտք} \\
 \armenian{՝}nothing' & ot͡ʃʰ\'int͡ʃʰ & \armenian{ոչինչ} & v\'unt͡ʃʰit͡ʃʰ & \armenian{վո՛ւնչիչ} & vot͡ʃʰ\'int͡ʃʰ & \armenian{ոչինչ} \\

 
 \hline
 \end{tabular}
\end{table}
 
 

In the final syllable, meaning under stress, the vowel becomes /u/ <\armenian{ու}> (Table \ref{tab:tbilisi:phono:change:oFinal}).


\begin{table}[H]
 \centering
 \caption{Change of final CA /o/ <\armenian{ո}> to /u/ <\armenian{ու}> in the Tbilisi dialect}
 \label{tab:tbilisi:phono:change:oFinal}
 \begin{tabular}{|l|ll|ll|ll|}
 \hline & \multicolumn{2}{l|}{Classical Armenian}& \multicolumn{2}{l|}{> Tbilisi }& \multicolumn{2}{l|}{cf. SEA }
 \\
 \armenian{՝}work' & ɡoɾt͡s & \armenian{գործ}& ɡuɾd͡z & \armenian{գուրձ} & ɡoɾt͡s & \armenian{գործ} \\
 \armenian{՝}belly' & pʰoɾ & \armenian{փոր}& pʰuɾ & \armenian{փուր} & pʰoɾ & \armenian{փոր} \\
 \armenian{՝}smell' & hot & \armenian{հոտ}& hut & \armenian{հուտ} & hot & \armenian{հոտ} \\
 \armenian{՝}four' & t͡ʃʰoɾs & \armenian{չորս}& t͡ʃʰuɾs & \armenian{չուրս} & t͡ʃʰoɾs & \armenian{չորս} \\
 \armenian{՝}bosom' & t͡sot͡sʰ & \armenian{ծոց}& t͡sut͡sʰ & \armenian{ծուց} & t͡sot͡sʰ & \armenian{ծոց} \\
 \armenian{՝}new' & noɾ & \armenian{նոր}& nuɾ & \armenian{նուր} & noɾ & \armenian{նոր} \\
 \hline
 \end{tabular}
\end{table}
 


In other syllables, it stays /o/ <\armenian{օ}> (Table \ref{tab:tbilisi:phono:change:oOther}).


\begin{table}[H]
 \centering
 \caption{Change of other positions of CA /o/ <\armenian{ո}> to /o/ <\armenian{օ}> in the Tbilisi dialect}
 \label{tab:tbilisi:phono:change:oOther}
 \begin{tabular}{|l|ll|ll|ll|}
 \hline & \multicolumn{2}{l|}{Classical Armenian}& \multicolumn{2}{l|}{> Tbilisi }& \multicolumn{2}{l|}{cf. SEA }
 \\
 \armenian{՝}barefoot' & bok\'ik & \armenian{բոկիկ}& b\'oblik & \armenian{բօ՛բլիկ} & bob\'ik & \armenian{բոբիկ} \\
 \armenian{՝}to grow moldy' & boɾbosil & \armenian{բորբոսիլ}& boɾpʰsnil & \armenian{բօրփսնիլ} & boɾbosel & \armenian{բորբոսել} \\
 \armenian{՝}to praise' & ɡov\'el & \armenian{գովել}& ɡ\'ovil & \armenian{գօ՛վիլ} & ɡov\'el & \armenian{գովել} \\
 \hline
 \end{tabular}
\end{table}


\subsubsection{Diphthongs }
\paragraph{Classical Armenian /oi̯/ <\armenian{ոյ}> to /u/ <\armenian{ու}> }

Classical Armenian /oi̯/ <\armenian{ոյ}> becomes /u/ <\armenian{ու}> (Table \ref{tab:tbilisi:phono:change:oj}).


\begin{table}[H]
 \centering
 \caption{Change of CA /oi̯/ <\armenian{ոյ}> to /u/ <\armenian{ու}> in the Tbilisi dialect}
 \label{tab:tbilisi:phono:change:oj}
 \begin{tabular}{|l|ll|ll|ll|}
 \hline & \multicolumn{2}{l|}{Classical Armenian}& \multicolumn{2}{l|}{> Tbilisi }& \multicolumn{2}{l|}{cf. SEA }
 \\
 \armenian{՝}light' & loi̯s & \armenian{լոյս}& lus & \armenian{լուս} & lujs & \armenian{լույս} \\
 \armenian{՝}sister' & kʰoi̯ɾ & \armenian{քոյր}& kʰuɾ & \armenian{քուր} & lujs & \armenian{քույր} \\
 \armenian{՝}sweet' & ɑn\'oi̯ʃ & \armenian{անոյշ}& \'ɑnuʃ & \armenian{ա՛նուշ} & ɑn\'ujʃ (dated) & \armenian{անույշ} \\
 & & & & & ɑn\'uʃ & \armenian{անուշ} \\
 \armenian{՝}color' & ɡoi̯n-kʰ (plural) & \armenian{գոյնք}& ɡunkʰ & \armenian{գունք} & ɡujn & \armenian{գույն} \\
 \hline
 \end{tabular}
\end{table}



\paragraph{Classical Armenian /iu̯/ <\armenian{իւ}> to /u/ <\armenian{ու}> }

Classical Armenian /iu̯/ <\armenian{իւ}> became /u/ <\armenian{ու}> (Table \ref{tab:tbilisi:phono:change:iu}).%\footnote{\translatorHD{For `weaved', Adjarian provides an ancestor <\armenian{հիւսած}>. But it's unclear to me if this form is attested in Classical Armenian, so as an ancestor I show the infinitive <\armenian{հիւսել}>.}}

\begin{table}[H]
 \centering
 \caption{Change of CA /iu̯/ <\armenian{իւ}> to /u/ <\armenian{ու}> in the Tbilisi dialect}
 \label{tab:tbilisi:phono:change:iu}
 \begin{tabular}{|l|ll|ll|ll|}
 \hline & \multicolumn{2}{l|}{Classical Armenian}& \multicolumn{2}{l|}{> Tbilisi }& \multicolumn{2}{l|}{cf. SEA }
 \\
 \armenian{՝}blood' & ɑɾ\'iu̯n & \armenian{արիւն}& \'ɑɾun & \armenian{ա՛րուն} & ɑɾj\'un & \armenian{արյուն} \\
 \armenian{՝}flour' & ɑl\'iu̯ɾ & \armenian{ալիւր}& \'ɑluɾ & \armenian{ա՛լուր} & ɑlj\'uɾ & \armenian{ալյուր} \\
 \armenian{՝}hundred' & hɑɾ\'iu̯ɾ & \armenian{հարիւր}& h\'ɑɾuɾ & \armenian{հա՛րուր} & hɑɾj\'uɾ & \armenian{հարյուր} \\
 \armenian{՝}to weave' & hiu̯s\'el & \armenian{հիւսել}& & & hjus\'el & \armenian{հյուսել} \\
 \armenian{՝}weaved' & & & h\'usɑt͡s & \armenian{հո՛ւսած} & hjus\'ɑt͡s & \armenian{հյուսած} \\
 \armenian{՝}guest' & hiu̯ɾ & \armenian{հիւր}& huɾ & \armenian{հուր} & hjuɾ & \armenian{հյուր} \\
 \armenian{՝}snow' & d͡ziu̯n & \armenian{ձիւն}& d͡zun & \armenian{ձուն} & d͡zjun & \armenian{ձյուն} \\
 \armenian{՝}branch' & t͡ʃiu̯ɬ & \armenian{ճիւղ}& t͡ʃuχk & \armenian{ճուխկ} & t͡ʃjuʁ & \armenian{ճյուղ} \\
 \hline
 \end{tabular}
\end{table}
\subsubsection{Consonant changes}
\paragraph{Stops and affricates }

For the three degrees of consonants, although the dialect has in general preserved the old sources, but in some places the sounds have gotten confused with each other. Let us cite some examples of these special cases (Table \ref{tab:tbilisi:phono:change:voicing}). 


\begin{table}[H]
 \centering
 \caption{Sporadic laryngeal changes of stops in the Tbilisi dialect}
 \label{tab:tbilisi:phono:change:voicing}
 \begin{tabular}{|l|ll|ll|ll|}
 \hline & \multicolumn{2}{l|}{Classical Armenian}& \multicolumn{2}{l|}{> Tbilisi }& \multicolumn{2}{l|}{cf. SEA }
 \\
 \armenian{՝}to find' & ɡətɑnel & \armenian{գտանել}& ɡtʰnil & \armenian{գթնիլ} & ɡətnel & \armenian{գտնել} \\
 `courtyard' & bɑk & \armenian{բակ}& bɑɡ & \armenian{բագ} & bɑk & \armenian{բակ} \\
 `sun' & ɑɾeɡ\'ɑkən & \armenian{արեգակն}& ɑɾ\'eɡɑɡ & \armenian{արէ՛գագ} & ɑɾeɡ\'ɑk & \armenian{արեգակ} \\
 `sky' & eɾk\'in-kʰ (plural) & \armenian{երկինք}& \'eɾɡinkʰ & \armenian{է՛րգինք} & jeɾk\'iŋkʰ & \armenian{երկինք} \\
 `land' & eɾk\'iɾ & \armenian{երկիր}& \'eɾɡiɾ & \armenian{է՛րգիր}& jeɾkiɾ & \armenian{երկ}\'\armenian{իր} \\
 `I know' & ɡit\'em & \armenian{գիտեմ}& ɡ\'idim & \armenian{գի՛դիմ}& ɡitem & \armenian{գիտ}\'\armenian{եմ} \\
 `with' & het & \armenian{հետ}& hid & \armenian{հիդ} & het& \armenian{հետ} \\
 `to respect' & met͡sɑɾ\'el & \armenian{մեծարել}& m\'ed͡zɾil & \armenian{մէ՛ձրիլ} & met͡sɑɾ\'el& \armenian{մեծարել} \\
 `ground' & ɡet\'in & \armenian{գետին}& ɡ\'edin & \armenian{գէ՛դին} & ɡet\'in& \armenian{գետին} \\
 `work' & ɡoɾt͡s & \armenian{գործ}& ɡuɾd͡z & \armenian{գուրձ} & ɡoɾt͡s& \armenian{գործ} \\
 \hline
 \end{tabular}
\end{table}


\paragraph{Post-nasal voicing}
As a general, all voiceless sounds become voiced after the nasal /n/ <\armenian{ն}> (Table \ref{tab:tbilisi:phono:change:postvoicing}). 


\begin{table}[H]
 \centering
 \caption{Post-nasal voicing in the Tbilisi dialect}
 \label{tab:tbilisi:phono:change:postvoicing}
 \begin{tabular}{|l|ll|ll|ll|}
 \hline & \multicolumn{2}{l|}{Classical Armenian}& \multicolumn{2}{l|}{> Tbilisi }& \multicolumn{2}{l|}{cf. SEA }
 \\
 \armenian{՝}to plant' & tənkel & \armenian{տնկել}& tnɡil & \armenian{տնգիլ} & təŋkel & \armenian{տնկել} \\
 \armenian{՝}ear' & ɑk\'ɑnd͡ʒ & \armenian{ականջ}& \'ɑnɡɑt͡ʃ & \armenian{ա՛նգաճ} & ɑk\'ɑnd͡ʒ & \armenian{ականջ} \\
 \armenian{՝}free, ownerless' & ɑnt\'eɾ & \armenian{անտէր}& \'ɑndeɾ & \armenian{ա՛նդէր} & ɑnt\'eɾ & \armenian{անտեր} \\
 \armenian{՝}friend' & ənkeɾ & \armenian{ընկեր}& nənɡiɾ & \armenian{նընգիր} & əŋkeɾ & \armenian{ընկեր} \\
 \armenian{՝}woman (nominative)' & & & knik & \armenian{կնիկ} & kənik & \armenian{կնիկ} \\
 \armenian{՝}woman (genitive)' & & & knɡɑ & \armenian{կնգա} & kəŋkɑn & \armenian{կնկան} \\
 \hline
 \end{tabular}
\end{table}

\paragraph{Nasal epenthesis}

For the sound sequence /ənɡ/ <\armenian{ընգ}> in the Tbilisi dialect, there is an also an initial \armenian{ն} /n/ (Table \ref{tab:tbilisi:phono:change:nasalEPen}). 


\begin{table}[H]
 \centering
 \caption{Nasal epenthesis in the Tbilisi dialect}
 \label{tab:tbilisi:phono:change:nasalEPen}
 \begin{tabular}{|l|ll|ll|ll|}
 \hline & \multicolumn{2}{l|}{Classical Armenian}& \multicolumn{2}{l|}{> Tbilisi }& \multicolumn{2}{l|}{cf. SEA }
 \\
 \armenian{՝}friend' & ənkeɾ & \armenian{ընկեր}& nənɡiɾ & \armenian{նընգիր} & əŋkeɾ & \armenian{ընկեր} \\
 \armenian{՝}walnut-tree' & ənkuzeni & \armenian{ընկուզենի}& nənɡzi & \armenian{նընգզի} & əŋkuzeni & \armenian{ընկուզենի} \\
 \armenian{՝}to fall' & ɑnkɑnil & \armenian{անկանիլ}& nənɡnil & \armenian{նընգնիլ} & əŋknel& \armenian{ընկնել} \\
 \hline
 \end{tabular}
\end{table}



\begin{adjarianpage}\label{page:54}\end{adjarianpage}% should be 54

\paragraph{Change from /h/ /χ/}

The */h/ <\armenian{հ}> sound is unchanged. But it has turned to /χ/ <\armenian{խ}> in only the words in Table \ref{tab:tbilisi:phono:change:hx}. 


\begin{table}[H]
 \centering
 \caption{Change from CA /h/ <\armenian{հ}> to /χ/ <\armenian{խ}> in the Tbilisi dialect}
 \label{tab:tbilisi:phono:change:hx}
 \begin{tabular}{|l|ll|ll|ll|}
 \hline & \multicolumn{2}{l|}{Classical Armenian}& \multicolumn{2}{l|}{> Tbilisi }& \multicolumn{2}{l|}{cf. SEA }
 \\
 \armenian{՝}earth' & hoɬ & \armenian{հող}& χuʁ & \armenian{խուղ} & hoʁ & \armenian{հող} \\
 \armenian{՝}earth' & ɑu̯ɾhnel & \armenian{աւրհնել}& oχnil & \armenian{օխնիլ} & oɾhnel & \armenian{օրհնել} \\
 \hline
 \end{tabular}
\end{table}

\subsection{Stress}

Stress has moved from the last syllable to the penultimate syllable, as in the Yerevan dialect.

\section{Morphology}

\subsection{Noun inflection or declension}

The Tbilisi dialect has 7 cases, which are in general the same as in the Yerevan dialect, in both form and composition. The following are the main differences for the Tbilisi dialect:
\begin{itemize}
 \item The ablative uses the formative /eme, emen/ <\armenian{էմէ, էմէն}> (Table \ref{tab:tbilisi:morpho:noun:abl}). 


\begin{table}[H]
 \centering
 \caption{Ablative suffix as /eme/ <\armenian{էմէ}> in the Tbilisi dialect}
 \label{tab:tbilisi:morpho:noun:abl}
 \begin{tabular}{|l|ll|ll|ll|}
 \hline & \multicolumn{2}{l|}{Tbilisi }& \multicolumn{2}{l|}{cf. Yerevan }& \multicolumn{2}{l|}{cf. SEA }
 \\
 \armenian{՝}writing' & ɡɾ-emen & \armenian{գրէմէն}& ɡɾ-it͡sʰ & \armenian{գրից} & ɡəɾ-it͡sʰ & \armenian{գրից} \\
 \armenian{՝}house' & tn-emen & \armenian{տնէմէն}& tən-it͡sʰ & \armenian{տնից} & tən-it͡sʰ & \armenian{տնից} \\
 \armenian{՝}death' & mɑh-emen & \armenian{մահէմէն}& mɑh-it͡sʰ & \armenian{մահից} & mɑh-it͡sʰ & \armenian{մահից} \\
 \hline
 \end{tabular}
\end{table}


\item The nominative plural is formed with the formatives /iɾ, niɾ/ <\armenian{իր, նիր}>. But the other cases keep the sound /e/ <\armenian{է}>, according to the phonetic rules. 
\item The plural genitive takes the formative /-u/ <\armenian{ու}>, similar to the /kə/ <\armenian{կը}> branch dialects. 
\end{itemize}

The following is the declension of the word /div/ <\armenian{դիվ}> from Classical /deu̯/ <\armenian{դեւ}> `demon'. \translatorHD{I suspect that the final /n/ in all the words in Table \ref{tab:Tbilisi:morpho:noun} is actually a separate definite suffix /-n/, but I'm not sure. }

\begin{table}[H]
\caption{Paradigm for noun inflection of the word /div/ <\armenian{դիվ}> `demon' in the Tbilisi dialect }\label{tab:Tbilisi:morpho:noun}
\centering \begin{tabular}{|l|ll|ll|}
\hline & \multicolumn{2}{l|}{Singular}& \multicolumn{2}{l|}{Plural} \\
\hline {\nom} ({\acc}) & div & \armenian{դիվ} & div-iɾ & \armenian{դիվիր} \\
{\gen} & div-i & \armenian{դիվի} & div-eɾ-u & \armenian{դիվէրու} \\
{\dat} ({\acc}) & div-i, div-in & \armenian{դիվի, դիվին} & div-eɾ-u-(n) & \armenian{դիվէրու(ն}) \\
{\abl} & div-emen & \armenian{դիվէմէն} & div-eɾ-emen & \armenian{դիվէրէմէն} \\
{\ins} & div-ov & \armenian{դիվօվ} & div-er-ov & \armenian{դիվէրօվ} \\
{\locgloss} & div-um & \armenian{դիվում} & div-er-um & \armenian{դիվէրում} 
\\ \hline 
\end{tabular}
\end{table}

\subsection{Pronoun inflection or declension}

The pronoun declensions are as follows. 

\translatorHD{Table \ref{tab:Tbilisi:morpho:pronoun:personal} is for personal pronouns. }

\begin{table}[H]
\caption{Inflection paradigm for personal pronouns in the Tbilisi dialect }\label{tab:Tbilisi:morpho:pronoun:personal}
\centering
\begin{tabular}{|l|lll|lll|}
 \hline & 1SG & 2SG & 3SG & 1PL & 2PL & 3PL \\
 & `I' & `you' & `he/she' & `we'& `you' & `they'\\\hline 
{\nom} & jis & du & nɑ & minkʰ & dukʰ & nɾɑnkʰ \\
 & \armenian{յիս} & \armenian{դու} & \armenian{նա} & \armenian{մինք} & \armenian{դուք} & \armenian{նրանք} \\
\hline {\gen} & im & kʰu & nɾɑ & miɾ & d͡ziɾ & nɾɑnt͡sʰ \\
 & \armenian{իմ} & \armenian{քու} & \armenian{նրա} & \armenian{միր} & \armenian{ձիր} & \armenian{նրանց} \\
\hline {\dat},{\acc} & ind͡zi & kʰiz & nɾɑn & miz & d͡ziz & nɾɑnt͡sʰ \\
 & \armenian{ինձի} & \armenian{քիզ} & \armenian{նրան} & \armenian{միզ} & \armenian{ձիզ} & \armenian{նրանց} \\
\hline {\abl} & ind͡z-m-en & kʰiz-m-en & nɾɑ-m-en & m-iz-m-en & d͡ziz-m-en & nɾɑnt͡sʰ-m-en \\
 & \armenian{ինձմէն} & \armenian{քիզմէն} & \armenian{նրամէն} & \armenian{միզմէն} & \armenian{ձիզմէն} & \armenian{նրանցմէն} \\
\hline {\ins} & ind͡z-m-ov & kʰiz-m-ov & nɾɑ-n-ov & m-iz-m-ov & d͡ziz-m-ov & nɾɑnt͡sʰ-ov \\
 & \armenian{ինձմօվ} & \armenian{քիզմօվ} & \armenian{նրանօվ} & \armenian{միզմօվ} & \armenian{ձիզմօվ} & \armenian{նրանցօվ} \\
\hline {\locgloss} & ind͡z-(ɑ)n-um & kʰiz-(ɑ)n-um & nrɑn-um & miz-(ɑ)n-um & d͡ziz-(ɑ)n-um & nɾɑnt͡sʰ-um \\
 & \armenian{ինձ(ա)նում} & \armenian{քիզ(ա)նում} & \armenian{նրանում} & \armenian{միզ(ա)նում} & \armenian{ձիզ(ա)նում} & \armenian{նրանցում} 
 \\ \hline 
\end{tabular}
\end{table}

\begin{adjarianpage}\label{page:55}\end{adjarianpage}% should be 55

\translatorHD{Table \ref{tab:Tbilisi:morpho:pronoun:dem} is for demonstrative pronouns. }

\begin{table}[H]
\caption{Inflection paradigm for demonstrative pronouns in the Tbilisi dialect }\label{tab:Tbilisi:morpho:pronoun:dem}
\centering \begin{tabular}{|l|lll|lll|}
\hline & \multicolumn{3}{c|}{Singular}& \multicolumn{3}{c|}{Plural}
\\ 
 & proximal & medial & distal & proximal & medial & distal \\
 & `this' & `that' & `that yonder' & `these' & `those' & `those yonder' \\ \hline
{\nom} & es & et & en & estunkʰ & etunkʰ & endunkʰ \\
 & \armenian{էս} & \armenian{էտ} & \armenian{էն} & \armenian{էստունք} & \armenian{էտունք} & \armenian{էնդունք} \\\hline
{\gen}, {\dat} & estu & etu & endu & estunt͡sʰ & etunt͡sʰ & endunt͡sʰ \\
 & \armenian{էստու} & \armenian{էտու} & \armenian{էնդու} & \armenian{էստունց} & \armenian{էտունց} & \armenian{էնդունց} \\\hline
{\abl} & estumen & etumen & endumen & estunt͡sʰmen & etunt͡sʰmen & endunt͡sʰmen \\
 & \armenian{էստումէն} & \armenian{էտումէն} & \armenian{էնդումէն} & \armenian{էստունցմէն} & \armenian{էտունցմէն} & \armenian{էնդունցմէն} \\\hline
{\ins} & estov & etov & endov & estunt͡sʰov & etunt͡sʰov & endunt͡sʰov \\
 & \armenian{էստօվ} & \armenian{էտօվ} & \armenian{էնդօվ} & \armenian{էստունցօվ} & \armenian{էտունցօվ} & \armenian{էնդունցօվ} \\\hline
{\locgloss} & estum & etum & endum & estunt͡sʰum & etunt͡sʰum & endunt͡sʰum \\
 & \armenian{էստում} & \armenian{էտում} & \armenian{էնդում} & \armenian{էստունցում} & \armenian{էտունցում} & \armenian{էնդունցում} 
 \\ \hline
\end{tabular}
\end{table}

\translatorHD{Table \ref{tab:Tbilisi:morpho:pronoun:ink} is for the logophoric third person pronoun. }

\begin{table}[H]
\caption{Inflection paradigm for demonstrative pronouns in the Tbilisi dialect }\label{tab:Tbilisi:morpho:pronoun:ink}
\centering \begin{tabular}{|l|ll|ll|}
\hline & \multicolumn{2}{l|}{3SG} & \multicolumn{2}{l|}{3PL} \\\hline 
{\nom} & inkʰə & \armenian{ինքը} & iɾɑnkʰ & \armenian{իրանք} \\
{\gen} & iɾ(ɑ) & \armenian{իր(ա}) & iɾɑnt͡sʰ & \armenian{իրանց} \\
{\dat},{\acc} & iɾɑn & \armenian{իրան} & iɾɑnt͡sʰ & \armenian{իրանց} \\
{\abl} & iɾmen & \armenian{իրմէն} & iɾɑnt͡sʰmen & \armenian{իրանցմէն} \\
{\ins} & iɾmov & \armenian{իրմօվ} & iɾɑnt͡sʰmov & \armenian{իրանցմօվ} \\
{\locgloss} & iɾɑnum & \armenian{իրանում} & iɾɑnt͡sʰum & \armenian{իրանցում} \\ \hline 
\end{tabular}
\end{table}

\subsection{Verb inflection or conjugation}

\subsubsection{Various aspects of verb inflection}
\paragraph{Sound changes for verbal vowels}
The verbs are declined in the manner of Yerevan, except for the required phonetic changes. For example, for the copula verb in the present, the Classical sounds /e, ē/ <\armenian{ե, է}> become become /i/ \armenian{ի} for all the persons (except for the third). And because of this, the stem of the verb uses the endings /-um im, -um is/ <-\armenian{ում իմ}, -\armenian{ում իս}>. 

\translatorHD{These points are illustrated later in \S\ref{section:Tbilisi:morpho:verb:paradigm:indcPresPst}.}

\paragraph{Irregular imperfective converb for monosyllabic verbs}

Like the Yerevan dialect, the monosyllabic verbs take the formative /-is/ <\armenian{իս}> (Table \ref{tab:Tbilisi:morpho:verb:imperfIrregIs}). 



\begin{table}[H]
 \centering
 \caption{Irregular imperfective converbs for monosyllabic verbs with /-is/ in the Tbilisi dialect }
 \label{tab:Tbilisi:morpho:verb:imperfIrregIs}

 \begin{tabular}{|l|ll|ll| }
 \hline & \multicolumn{2}{l|}{Tbilisi }& \multicolumn{2}{l|}{cf. SEA } \\
 \hline Infinitive &&& \multicolumn{2}{l|}{$\sqrt{}$-{\thgloss}-{\infgloss}} \\
 `to come' & & & ɡ-ɑ-l & \armenian{գալ} \\
 `to give' & & & t-ɑ-l & \armenian{տալ} \\
 `to cry' & & & l-ɑ-l & \armenian{լալ} \\
 \hline Present 1SG & \multicolumn{4}{l|}{$\sqrt{}$-{\thgloss}-{\infgloss}-{\impfcvb} {\aux}-1{\sg}} \\
`I come' & ɡ-\'ɑ-l-is e-m & \armenian{գա՛լիս էմ} & ɡ-ɑ-l-is e-m & \armenian{գալիս եմ} \\
`I give' & t-\'ɑ-l-is e-m & \armenian{տա՛լիս էմ} & t-ɑ-l-is e-m & \armenian{տալիս եմ} \\
`I cry' & l-\'ɑ-l-is e-m & \armenian{լա՛լիս էմ} & l-ɑ-l-is e-m & \armenian{լալիս եմ} \\
 \hline 
 \end{tabular} 
\end{table}

\paragraph{Lack of /e/ deletion for past imperfective}\label{sec:Tbilisi:morpho:verb:various:eImpf}


In the past imperfective, the reflex of the Classical sound /ē/ <\armenian{է}> does not shorten. The forms are pronounced as in Old Armenian: /ei, eiɾ, eɾ, einkʰ, eikʰ, en/ <\armenian{էի էիր էր էինք էիք էին}> (Table \ref{tab:Tbilisi:morpho:verb:imperf}). \translatorHD{What he means is that unlike in the Yerevan dialect, the auxiliary /e/ and theme vowel /e/ do not delete before the past suffix /-i/. The Tbilisi dialect thus patterns with SEA in this regard.} 



\begin{table}[H]
 \centering
 \caption{Past imperfective in the Tbilisi dialect }
 \label{tab:Tbilisi:morpho:verb:imperf}

 \begin{tabular}{|l|ll|ll| }
 \hline & \multicolumn{2}{l|}{Tbilisi }& \multicolumn{2}{l|}{cf. SEA } \\
 `I was speaking' & χos-um e-i-$\emptyset$ & \armenian{խօսում էի} & χos-um ej-i-$\emptyset$ & \armenian{խոսում էի} \\
 `I was saying' & ɑs-um e-i-$\emptyset$ & \armenian{ասում էի} & ɑs-um ej-i-$\emptyset$ & \armenian{ասում էի} \\
 & \multicolumn{2}{l|}{$\sqrt{}$-{\impfcvb} {\aux}-{\pst}-1{\sg} }& \multicolumn{2}{l|}{$\sqrt{}$-{\impfcvb} {\aux}-{\pst}-1{\sg} } \\
 \hline 
 \end{tabular} 
\end{table}




\begin{adjarianpage}\label{page:56}\end{adjarianpage}% should be 56

\paragraph{Allomorphy of the future formative /ku/ <\armenian{կու}>}

The future formative is /ku/ <\armenian{կու}> instead of /kə/ <\armenian{կը}> (Table \ref{tab:Tbilisi:morpho:verb:futKU}).\footnote{\translatorHD{Some modern grammars of SEA treat the formative /k/ as as a conditional future marker \citep[253ff]{DumTragut-2009-ArmenianReferenceGrammar}. But to maintain consistency with Adjarian, I gloss it as a future marker. } }



\begin{table}[H]
 \centering
 \caption{Future formative as /ku/ <\armenian{կու}> in the Tbilisi dialect }
 \label{tab:Tbilisi:morpho:verb:futKU}

 \begin{tabular}{|l|ll|ll| }
 \hline & \multicolumn{2}{l|}{Tbilisi }& \multicolumn{2}{l|}{cf. SEA } \\
 `I will like' & ku siɾ-i-m & \armenian{կու սիրիմ} & kə-siɾ-e-m & \armenian{կսիրեմ} \\
 `I will bring' & ku beɾ-i-m & \armenian{կու բէրիմ} & kə-beɾ-e-m & \armenian{կբերեմ} \\
 & \multicolumn{2}{l|}{{\fut}-$\sqrt{}$-{\thgloss}-1{\sg} }& \multicolumn{2}{l|}{{\fut}-$\sqrt{}$-{\thgloss}-1{\sg} } \\
 \hline 
 \end{tabular} 
\end{table}

Before vowel-initial verbs, this participle is sometimes shortened to \armenian{կ} /k/, but a lot of times it stays constant (Table \ref{tab:Tbilisi:morpho:verb:imperfIrregIs}). 



\begin{table}[H]
 \centering
 \caption{Variable shortening of the future formative as /ku/ <\armenian{կու}> before vowel-initial verbs in the Tbilisi dialect }
 \label{tab:Tbilisi:morpho:verb:futKConst}

 \begin{tabular}{|l|ll|ll| }
 \hline & \multicolumn{2}{l|}{Tbilisi }& \multicolumn{2}{l|}{cf. SEA } \\
 ? & k-eh-ɑ-m & \armenian{կէհամ} & & \\
 `I will go' & k-eɾtʰ-ɑ-m & \armenian{կէրթամ} & k-eɾtʰ-ɑ-m & \armenian{կերթամ} \\
 `I will have' & k-unen-ɑ-m & \armenian{կունէնամ} & k-eɾtʰ-ɑ-m & \armenian{կունենամ} \\
 `I will free' & ku ɑzɑt-i-m & \armenian{կու ազատիմ} & k-ɑzɑt-e-m & \armenian{կազատեմ} \\
 `I will pray' & ku ɑʁotʰ-i-m & \armenian{կու աղօթիմ} & k-ɑʁotʰ-e-m & \armenian{կաղոթեմ} \\
 `I will burn' & ku eɾ-i-m & \armenian{կու էրիմ} & k-ɑjɾ-e-m & \armenian{կայրեմ} \\
 `I will know' & ku imɑn-ɑ-m & \armenian{կու իմանամ} & k-ɑjɾ-ɑ-m & \armenian{կիմանամ} \\
 & \multicolumn{2}{l|}{{\fut}-$\sqrt{}$-{\thgloss}-1{\sg} }& \multicolumn{2}{l|}{{\fut}-$\sqrt{}$-{\thgloss}-1{\sg} } \\
 `I will take' & ku ɑr-n-i-m & \armenian{կու առնիմ} & k-ɑr-n-e-m & \armenian{կառնեմ} \\
 & \multicolumn{2}{l|}{{\fut}-$\sqrt{}$-{\vx}-{\thgloss}-1{\sg} }& \multicolumn{2}{l|}{{\fut}-$\sqrt{}$-{\vx}-{\thgloss}-1{\sg} } \\

 \hline 
 \end{tabular} 
\end{table}

The particle becomes voiced /ɡ/ <\armenian{գ}> before the verb `to want' (Table \ref{tab:Tbilisi:morpho:verb:imperfIrregIs}). 



\begin{table}[H]
 \centering
 \caption{Voicing of the future formative as /ɡ/ <\armenian{գ}> for the verb `to want' in the Tbilisi dialect }
 \label{tab:Tbilisi:morpho:verb:futKG}

 \begin{tabular}{|l|ll|ll|l| }
 \hline & \multicolumn{2}{l|}{Tbilisi }& \multicolumn{2}{l|}{cf. SEA } & \\
 `I will want' & ɡ-uz-i-m & \armenian{գուզիմ} & k-uz-e-m & \armenian{կուզեմ} & {\fut}-$\sqrt{}$-{\thgloss}-1{\sg} \\
 `you will want' & ɡ-uz-i-s & \armenian{գուզիս} & k-uz-e-s & \armenian{կուզես} & {\fut}-$\sqrt{}$-{\thgloss}-2{\sg} \\
 `I was wanting' & ɡ-uz-e-i-$\emptyset$ & \armenian{գուզէի} & k-uz-e-s & \armenian{կուզես} & {\fut}-$\sqrt{}$-{\thgloss}-{\pst}-1{\sg} \\


 \hline 
 \end{tabular} 
\end{table}

It is also voiced as in Table\ref{tab:Tbilisi:morpho:verb:futKgu}. \translatorHD{Note that he doesn't state what is the root of these verbs. So I segment the prefix as /ɡu/ based on the contrast against SEA, but Adjarian might have meant that the prefix as /ɡ-/ while the verb was /ukʰɑm/.}



\begin{table}[H]
	\centering
	\caption{Voicing of the future formative as /ɡ(u?)/ <\armenian{գու}> for the verb `to come' in the Tbilisi dialect }
	\label{tab:Tbilisi:morpho:verb:futKgu}
	
	\begin{tabular}{|l|ll|ll|l| }
		\hline & \multicolumn{2}{l|}{Tbilisi }& \multicolumn{2}{l|}{cf. SEA } & \\
		`I will come' & ɡu-kʰ-ɑ-m & \armenian{գուքամ} & kə-ɡ-ɑ-m & \armenian{կգամ} & {\fut}-$\sqrt{}$-{\thgloss}-1{\sg} \\
		`you will come' & ɡu-kʰ-ɑ-s & \armenian{գուքաս} & kə-ɡ-ɑ-s & \armenian{կգաս} & {\fut}-$\sqrt{}$-{\thgloss}-2{\sg} \\
		\hline 
	\end{tabular} 
\end{table}


Before the verbs in Table\ref{tab:Tbilisi:morpho:verb:futKCoales}, the particle is fused with the /ɑ/ <\armenian{ա}>, and becomes /ko/ <\armenian{կօ}>. 



\begin{table}[H]
 \centering
 \caption{Coalesence of merger of the future formative /ku/ <\armenian{կու}> and verb-initial /ɑ/ <\armenian{ա}> as /ko/ <\armenian{կօ}> in the Tbilisi dialect }
 \label{tab:Tbilisi:morpho:verb:futKCoales}

 \begin{tabular}{|l|ll|ll| }
 \hline & \multicolumn{2}{l|}{Tbilisi }& \multicolumn{2}{l|}{cf. SEA } \\
 \hline Infinitive && &\multicolumn{2}{l|}{$\sqrt{}$-{\thgloss}-{\infgloss}} \\
 `to lay' & & &ɑt͡s-e-l & \armenian{ածել} \\
 `to do'& & &ɑn-e-l & \armenian{անել} \\
 `to say' & & &ɑs-e-l & \armenian{ասել} \\
 \hline Future 1{\sg} &\multicolumn{2}{l|}{{\fut}-$\sqrt{}$-{\thgloss}-1{\sg}} &\multicolumn{2}{l|}{{\fut}-$\sqrt{}$-{\thgloss}-1{\sg}} \\
 `I will lay &k-ot͡s-i-m &\armenian{կօծիմ} &k-ɑt͡s-e-m & \armenian{կածեմ} \\
 `I will do' &k-on-i-m &\armenian{կօնիմ} &k-ɑn-e-m & \armenian{կանեմ} \\
 `I will say' &k-os-i-m & \armenian{կօսիմ} & k-ɑs-e-m & \armenian{կասեմ} \\
 \hline Future perfect 1{\sg} &\multicolumn{2}{l|}{{\fut}-$\sqrt{}$-{\thgloss}-{\pst}-1{\sg}} &\multicolumn{2}{l|}{{\fut}-$\sqrt{}$-{\thgloss}-{\pst}-1{\sg}} \\
 `I was going to lay &k-ot͡s-e-i-$\emptyset$ &\armenian{կօծէի} &k-ɑt͡s-ej-i-$\emptyset$ & \armenian{կածեի} \\
 `I was going to do' &k-on-e-i-$\emptyset$ &\armenian{կօնէի} &k-ɑn-ej-i-$\emptyset$ & \armenian{կանեի} \\
 `I was going to say' &k-os-e-i-$\emptyset$ & \armenian{կօսէի} & k-ɑs-ej-i-$\emptyset$ & \armenian{կասեի} \\
 \hline 
 \end{tabular} 
\end{table}

\paragraph{Past participle or perfective converb with /-il,-i/ <\armenian{իլ, ի}> }

The past participle has the ending /-il/ <\armenian{իլ}>. This is used to form the present perfect (\armenian{յարակատար}), past perfect (\armenian{գերակատար}), and the negative (\armenian{բացասական}) forms (Table \ref{tab:Tbilisi:morpho:verb:pastPart}). 



\begin{table}[H]
 \centering
 \caption{Past participle or perfective converb with /-il/ <\armenian{իլ}> in the Tbilisi dialect }
 \label{tab:Tbilisi:morpho:verb:pastPart}

 \begin{tabular}{|l|ll|ll|l| }
 \hline & \multicolumn{2}{l|}{Tbilisi }& \multicolumn{2}{l|}{cf. SEA } & \\
 `I have liked' & siɾ-il i-m & \armenian{սիրիլ իմ} & siɾ-el e-m & \armenian{սիրել եմ} & $\sqrt{}$-{\perfcvb} {\aux}-1{\sg} \\
 `I had liked' & siɾ-il e-i-$\emptyset$ & \armenian{սիրիլ էի} & siɾ-el ej-i-$\emptyset$ & \armenian{սիրել էի} & $\sqrt{}$-{\perfcvb} {\aux}-{\pst}-1{\sg} \\
 \hline 
 \end{tabular} 
\end{table}

But when the auxiliary is before the participle, the last letter of the suffix is lost (Table \ref{sent:Tbilisi:morpho:verb:pastPartMove}). 


\begin{exe}
 \ex Tbilisi dialect \label{sent:Tbilisi:morpho:verb:pastPartMove} \begin{xlist}
 \ex \gll t͡ʃʰ-i-m siɾ-i \\
 {\neggloss}-{\aux}-1{\sg} like-{\perfcvb} \\
 \trans `I have not liked.' \\
 \armenian{չիմ սիրի}
\ex \gll jis i-m beɾ-i \\
 I {\aux}-1{\sg} bring-{\perfcvb} \\
 \trans `\textbf{I} have brought (as opposed to someone else).' \\
 \armenian{յիս իմ բէրի}
 \end{xlist}
\end{exe}
\subsubsection{General paradigm}

Here we show the most often used forms of the verb `to like', as a reflex from Classical /siɾ-e-l/ <\armenian{սիրել}>. 


{\paradigmExplanation}

\paragraph{Indicative present and past imperfective}\label{section:Tbilisi:morpho:verb:paradigm:indcPresPst}

\translatorHD{The indicative present in SEA is formed via periphrasis (Table \ref{tab:Tbilisi:morpho:verb:paradigm:presentIndc}). The verb is in a converb form called the imperfective converb with the suffix /-um/. Tense and agreement is on an inflected auxiliary. The Tbilisi dialect shows the same strategy with one major difference. The auxiliary /e/ is replaced by /i/ for all but the present 3SG. }


\begin{table}[H]
 \centering
 \caption{Indicative present <\armenian{ներկայ}> of the verb `to like' in the Tbilisi dialect}
 \label{tab:Tbilisi:morpho:verb:paradigm:presentIndc}
 \begin{tabular}{|l|ll|ll|}
 \hline & \multicolumn{2}{l|}{Tbilisi} & \multicolumn{2}{l|}{cf. SEA} \\
1SG & siɾ-um i-m & \armenian{սիրում իմ} & siɾ-um e-m & \armenian{սիրում եմ} \\
2SG & siɾ-um i-s & \armenian{սիրում իս} & siɾ-um e-s & \armenian{սիրում ես} \\
3SG & siɾ-um e & \armenian{սիրում է} & siɾ-um e & \armenian{սիրում է} \\
1PL & siɾ-um i-nkʰ & \armenian{սիրում ինք} & siɾ-um e-ŋkʰ & \armenian{սիրում ենք}\\
2PL & siɾ-um i-kʰ & \armenian{սիրում իք} & siɾ-um e-kʰ & \armenian{սիրում եք} \\
3PL & siɾ-um i-n & \armenian{սիրում ին} & siɾ-um e-n & \armenian{սիրում են} \\
& \multicolumn{2}{l|}{$\sqrt{}$-{\impfcvb} {\aux}-{\agr}}& \multicolumn{2}{l|}{$\sqrt{}$-{\impfcvb} {\aux}-{\agr}}
 \\ \hline 
 \end{tabular} \end{table}


 
\translatorHD{The indicative past imperfective uses the same imperfective converb as in the present (Table \ref{tab:Tbilisi:morpho:verb:paradigm:pastImpfIndc}). The difference is that auxiliary is now in the past tense. In both SEA and Tbilisi, the auxiliary has the shape /e/.}



\begin{table}[H]
 \centering
 \caption{Indicative past imperfective <\armenian{անկատար}> of the verb `to like' in the Tbilisi dialect}
 \label{tab:Tbilisi:morpho:verb:paradigm:pastImpfIndc}
 \begin{tabular}{|l|ll|ll|}
\hline & \multicolumn{2}{l|}{Tbilisi} & \multicolumn{2}{l|}{cf. SEA} \\
1SG & siɾ-um e-i-$\emptyset$ & \armenian{սիրում էի} & siɾ-um ej-i-$\emptyset$ & \armenian{սիրում էի} \\
2SG & siɾ-um e-i-ɾ & \armenian{սիրում էիր} & siɾ-um ej-i-ɾ & \armenian{սիրում էիր} \\
3SG & siɾ-um e-$\emptyset$-ɾ & \armenian{սիրում էր} & siɾ-um e-$\emptyset$-ɾ & \armenian{սիրում էր} \\
1PL & siɾ-um e-i-nkʰ & \armenian{սիրում էինք} & siɾ-um ej-i-ŋkʰ & \armenian{սիրում էինք} \\
2PL & siɾ-um e-i-kʰ & \armenian{սիրում էիք} & siɾ-um ej-i-kʰ & \armenian{սիրում էիք} \\
3PL & siɾ-um e-i-n & \armenian{սիրում էին} & siɾ-um ej-i-n & \armenian{սիրում էին} \\
& \multicolumn{2}{l|}{$\sqrt{}$-{\impfcvb} {\aux}-{\pst}-{\agr}}& \multicolumn{2}{l|}{$\sqrt{}$-{\impfcvb} {\aux}-{\pst}-{\agr}} \\
\hline 
\end{tabular}
\end{table}

\translatorHD{Thus in Tbilisi, the auxiliary shows variation in its morphs: /e/ for some inflection cells, while /i/ for other cells. See \S\ref{sec:Tbilisi:morpho:verb:various:eImpf}.}


\paragraph{Present perfect and past perfect}

\translatorHD{The present perfect (Table \ref{tab:Tbilisi:morpho:verb:paradigm:presentPerfect}) and past perfect (Table \ref{tab:Tbilisi:morpho:verb:paradigm:pastPerfect}) in SEA are formed with periphrasis. The verb is in the form of the perfective converb with the suffix /-el/. The present tense auxiliary is added for the present perfect, while the past auxiliary for the past perfect. The Tbilisi dialect essentially uses the same strategy but with two differences. First, the converb suffix is /-il/ not /-el/. Second, the auxiliary shows the same changes in its shape as for the indicative present and past. }

\begin{table}[H]
 \centering
 \caption{Present perfect <\armenian{յարակատար}> of the verb `to like' in the Tbilisi dialect}
 \label{tab:Tbilisi:morpho:verb:paradigm:presentPerfect}
 \begin{tabular}{|l|ll|ll|}
\hline & \multicolumn{2}{l|}{Tbilisi} & \multicolumn{2}{l|}{cf. SEA} \\
1SG & siɾ-il i-m & \armenian{սիրիլ իմ} & siɾ-el e-m & \armenian{սիրել եմ} \\
2SG & siɾ-il i-s & \armenian{սիրիլ իս} & siɾ-el e-s & \armenian{սիրել ես} \\
3SG & siɾ-il e & \armenian{սիրիլ է} & siɾ-el e & \armenian{սիրել է} \\
1PL & siɾ-il i-nkʰ & \armenian{սիրիլ ինք} & siɾ-el e-ŋkʰ & \armenian{սիրել ենք} \\
2PL & siɾ-il i-kʰ & \armenian{սիրիլ իք} & siɾ-el e-kʰ & \armenian{սիրել եք} \\
3PL & siɾ-il i-n & \armenian{սիրիլ ին} & siɾ-el e-n & \armenian{սիրել են} \\
& \multicolumn{2}{l|}{$\sqrt{}$-{\perfcvb} {\aux}-{\agr}}& \multicolumn{2}{l|}{$\sqrt{}$-{\perfcvb} {\aux}-{\agr}}\\ 

\hline 
\end{tabular}
\end{table}

\begin{table}[H]
 \centering
 \caption{Past perfect <\armenian{գերակատար}> of the verb `to like' in the Tbilisi dialect}
 \label{tab:Tbilisi:morpho:verb:paradigm:pastPerfect}
 \begin{tabular}{|l|ll|ll| }
\hline & \multicolumn{2}{l|}{Tbilisi} & \multicolumn{2}{l|}{cf. SEA} \\
1SG & siɾ-il e-i-$\emptyset$ & \armenian{սիրիլ էի} & siɾ-el ej-i-$\emptyset$ & \armenian{սիրել էի} \\
2SG & siɾ-il e-i-ɾ & \armenian{սիրիլ էիր} & siɾ-el ej-i-ɾ & \armenian{սիրել էիր} \\
3SG & siɾ-il e-$\emptyset$-ɾ & \armenian{սիրիլ էր} & siɾ-el e-$\emptyset$-ɾ & \armenian{սիրել էր} \\
1PL & siɾ-il e-i-nkʰ & \armenian{սիրիլ էինք} & siɾ-el ej-i-ŋkʰ & \armenian{սիրել էինք} \\
2PL & siɾ-il e-i-kʰ & \armenian{սիրիլ էիք} & siɾ-el ej-i-kʰ & \armenian{սիրել էիք} \\
3PL & siɾ-il e-i-n & \armenian{սիրիլ էին} & siɾ-el ej-i-n & \armenian{սիրել էին} \\
& \multicolumn{2}{l|}{$\sqrt{}$-{\perfcvb} {\aux}-{\pst}-{\agr}}& \multicolumn{2}{l|}{$\sqrt{}$-{\perfcvb} {\aux}-{\pst}-{\agr}}\\ 

\hline 
\end{tabular}
\end{table}

\paragraph{Past perfective or aorist}

\translatorHD{The past perfective (Table \ref{tab:Tbilisi:morpho:verb:paradigm:pastperfectiveAorist}) is also called the aorist. In SEA for /siɾ-e-l/ `to like', the past perfective is formed by taking the root and theme vowel, adding the aorist or perfective suffix /-t͡sʰ-/, and then adding the past suffix /-i/ and the appropriate agreement suffixes. The 3SG uses covert tense and agreement suffixes. The Tbilisi dialect behaves the same. }


\begin{table}[H]
 \centering
 \caption{Past perfective or aorist <\armenian{կատարեալ}> of the verb `to like' in the Tbilisi dialect}
 \label{tab:Tbilisi:morpho:verb:paradigm:pastperfectiveAorist}
 \begin{tabular}{|l|ll|ll|}
\hline & \multicolumn{2}{l|}{Tbilisi} & \multicolumn{2}{l|}{cf. SEA} \\
1SG & siɾ-e-t͡sʰ-i-$\emptyset$ & \armenian{սիրէցի} & siɾ-e-t͡sʰ-i-$\emptyset$ & \armenian{սիրեցի} \\
2SG & siɾ-e-t͡sʰ-i-ɾ & \armenian{սիրէցիր} & siɾ-e-t͡sʰ-i-ɾ & \armenian{սիրեցիր} \\
3SG & siɾ-i-t͡sʰ-$\emptyset$-$\emptyset$ & \armenian{սիրից} & siɾ-e-t͡sʰ-$\emptyset$-$\emptyset$ & \armenian{սիրեց} \\
1PL & siɾ-e-t͡sʰ-i-nkʰ & \armenian{սիրէցինք} & siɾ-e-t͡sʰ-i-ŋkʰ & \armenian{սիրեցինք} \\
2PL & siɾ-e-t͡sʰ-i-kʰ & \armenian{սիրէցիք} & siɾ-e-t͡sʰ-i-kʰ & \armenian{սիրեցիք} \\
3PL & siɾ-e-t͡sʰ-i-n & \armenian{սիրէցին} & siɾ-e-t͡sʰ-i-n & \armenian{սիրեցին} \\
& \multicolumn{2}{l|}{$\sqrt{}$-{\thgloss}-{\aor}-{\pst}-{\agr}}& \multicolumn{2}{l|}{$\sqrt{}$-{\thgloss}-{\aor}-{\pst}-{\agr}}\\ 

\hline 
\end{tabular}
\end{table}

\translatorHD{Note though that theme vowel is /e/ for all but the 3SG. The past perfective 3SG instead uses the theme vowel /i/. }


\paragraph{Subjunctive present and past imperfective } 

\translatorHD{In SEA, the subjunctive present (Table \ref{tab:Tbilisi:morpho:verb:paradigm:subjPresent}) is formed by adding agreement suffixes after the theme vowel. These are the same agreement suffixes that are added onto the present auxiliary in the indicative present. For a verb like `to like', the 3SG involves changing the theme vowel /e/ to /i/ in the 3SG. The Tbilisi dialect follows the same system but with the opposite choice of vowels. The theme vowel is /e/ for the present 3SG, and /i/ elsewhere. } 


\begin{table}[H]
 \centering
 \caption{Subjunctive present <\armenian{ստորադասական ներկայ}> of the verb `to like' in the Tbilisi dialect}
 \label{tab:Tbilisi:morpho:verb:paradigm:subjPresent}
 \begin{tabular}{|l|ll|ll|}
\hline & \multicolumn{2}{l|}{Tbilisi} & \multicolumn{2}{l|}{cf. SEA} \\
1SG & siɾ-i-m & \armenian{սիրիմ} & siɾ-e-m & \armenian{սիրեմ} \\
2SG & siɾ-i-s & \armenian{սիրիս} & siɾ-e-s & \armenian{սիրես} \\
3SG & siɾ-e-$\emptyset$ & \armenian{սիրէ} & siɾ-i-$\emptyset$ & \armenian{սիրի} \\
1PL & siɾ-i-nkʰ & \armenian{սիրինք} & siɾ-e-ŋkʰ & \armenian{սիրենք} \\
2PL & siɾ-i-k & \armenian{սիրիք} & siɾ-e-k & \armenian{սիրեք} \\
3PL & siɾ-i-n & \armenian{սիրին} & siɾ-e-n & \armenian{սիրեն} \\
& \multicolumn{2}{l|}{$\sqrt{}$-{\thgloss}-{\agr}}& \multicolumn{2}{l|}{$\sqrt{}$-{\thgloss}-{\agr}}\\ 

\hline 
\end{tabular}
\end{table}

\translatorHD{In SEA, the subjunctive past imperfective (Table \ref{tab:Tbilisi:morpho:verb:paradigm:subjPast}) is formed by adding the past suffix /i/ and agreement suffixes after the theme vowel. In Tbilisi, the same is used. Note how the theme vowel stays a constant /e/ in the past, unlike the variation in the present. }



\begin{table}[H]
 \centering
 \caption{Subjunctive past <\armenian{ստորադասական անցեալ}> of the verb `to like' in the Tbilisi dialect}
 \label{tab:Tbilisi:morpho:verb:paradigm:subjPast}
 \begin{tabular}{|l|ll|ll|}
\hline & \multicolumn{2}{l|}{Tbilisi} & \multicolumn{2}{l|}{cf. SEA} \\
1SG & siɾ-e-i-$\emptyset$ & \armenian{սիրէի} & siɾ-ej-i-$\emptyset$ & \armenian{սիրեի} \\
2SG & siɾ-e-i-ɾ & \armenian{սիրէիր} & siɾ-ej-i-ɾ & \armenian{սիրեիր} \\
3SG & siɾ-e-$\emptyset$-ɾ & \armenian{սիրէր} & siɾ-e-$\emptyset$-ɾ & \armenian{սիրեր} \\
1PL & siɾ-e-i-nkʰ & \armenian{սիրէինք} & siɾ-ej-i-nkʰ & \armenian{սիրեինք} \\
2PL & siɾ-e-i-kʰ & \armenian{սիրէիք} & siɾ-ej-i-kʰ & \armenian{սիրեիք} \\ 
3PL & siɾ-e-i-n & \armenian{սիրէին} & siɾ-ej-i-n & \armenian{սիրեին} \\ 
& \multicolumn{2}{l|}{$\sqrt{}$-{\thgloss}-{\pst}-{\agr}}& \multicolumn{2}{l|}{$\sqrt{}$-{\thgloss}-{\pst}-{\agr}}\\ 

\hline 
\end{tabular}
\end{table}

\paragraph{Tenses built from the subjunctive: Future and debitive}
 
 
 \translatorHD{In Tbilisi, many other tenses seem to be built off of the subjunctive (Table \ref{tab:Tbilisi:morpho:verb:paradigm:complexSubjunctive}). The future and future perfect are built by adding the prefix /ku/ before the subjunctive present and subjunctive past. The debitive and debitive perfect are formed also by adding the proclitic /piti/ before the appropriate subjunctive form. I don't provide morpheme glosses for these forms for space. SEA behaves essentially the same (with the expected difference in theme vowels) and I don't provide its paradigm. }
 

\begin{table}[H]
 \centering
 \caption{Forms that are built from the subjunctive forms for the verb `to like' in the Tbilisi dialect}
 \label{tab:Tbilisi:morpho:verb:paradigm:complexSubjunctive}
 \begin{tabular}{|l|ll|ll|}
\hline & 
\multicolumn{2}{l|}{Future <\armenian{ապառնի}>} & \multicolumn{2}{l|}{Future perfect <\armenian{անցեալ ապառնի}> } \\
1SG & ku siɾ-i-m & \armenian{կու սիրիմ} & ku siɾ-e-i-$\emptyset$ & \armenian{կու սիրէի} \\
2SG & ku siɾ-i-s & \armenian{կու սիրիս} & ku siɾ-e-i-ɾ & \armenian{կու սիրէիր} \\
3SG & ku siɾ-e-$\emptyset$ & \armenian{կու սիրէ} & ku siɾ-e-$\emptyset$-ɾ & \armenian{կու սիրէր} \\
1PL & ku siɾ-i-nkʰ & \armenian{կու սիրինք} & ku siɾ-e-i-nkʰ & \armenian{կու սիրէինք} \\
2PL & ku siɾ-i-k & \armenian{կու սիրիք} & ku siɾ-e-i-kʰ & \armenian{կու սիրէիք} \\
3PL & ku siɾ-i-n & \armenian{կու սիրին} & ku siɾ-e-i-n & \armenian{կու սիրէին} \\
& \multicolumn{2}{l|}{{\fut} $\sqrt{}$-{\thgloss}-{\agr}}& \multicolumn{2}{l|}{{\fut} $\sqrt{}$-{\thgloss}-{\pst}-{\agr}}
\\ \hline 
& \multicolumn{2}{l|}{Debitive \armenian{պարտաւորական ներկայ} } & \multicolumn{2}{l|}{Debitive perfect \armenian{պարտաւորական անցեալ} } \\
1SG & piti siɾ-i-m & \armenian{պիտի սիրիմ} & piti siɾ-e-i-$\emptyset$ & \armenian{պիտի սիրէի} \\
2SG & piti siɾ-i-s & \armenian{պիտի սիրիս} & piti siɾ-e-i-ɾ & \armenian{պիտի սիրէիր} \\
3SG & piti siɾ-e-$\emptyset$ & \armenian{պիտի սիրէ} & piti siɾ-e-$\emptyset$-ɾ & \armenian{պիտի սիրէր} \\
1PL & piti siɾ-i-nkʰ & \armenian{պիտի սիրինք} & piti siɾ-e-i-nkʰ & \armenian{պիտի սիրէինք} \\
2PL & piti siɾ-i-k & \armenian{պիտի սիրիք} & piti siɾ-e-i-kʰ & \armenian{պիտի սիրէիք} \\
3PL & piti siɾ-i-n & \armenian{պիտի սիրին} & piti siɾ-e-i-n & \armenian{պիտի սիրէին} \\ 
& \multicolumn{2}{l|}{{\deb} $\sqrt{}$-{\thgloss}-{\agr}}& \multicolumn{2}{l|}{{\deb} $\sqrt{}$-{\thgloss}-{\pst}-{\agr}}
\\\hline \end{tabular}
\end{table}

\translatorHD{The debitive forms show an alternative strategy. The previously discussed strategy in Table \ref{tab:Tbilisi:morpho:verb:paradigm:complexSubjunctive} was to place the particle /piti/ before the inflected verb. The verb carries tense and agreement inflection. In contrast, an alternative strategy (Table \ref{tab:Tbilisi:morpho:verb:paradigm:debitiveOther}) is to place the tense and agreement morphology onto the particle /piti/. The verb is then in a constant shape: /siɾ-i/ for `to like'. }

\begin{table}[H]
 \centering
 \caption{Alternative forms for the debitive for the verb `to like' in the Tbilisi dialect with mobile agreement}
 \label{tab:Tbilisi:morpho:verb:paradigm:debitiveOther}
 \begin{tabular}{|l|ll|ll|}
\hline 
& \multicolumn{2}{l|}{Debitive \armenian{պարտաւորական} } & \multicolumn{2}{l|}{Debitive perfect \armenian{անցեալ պարտաւորական} } \\
1SG & pit-i-m siɾ-i & \armenian{պիտիմ սիրի} & pit-e-i-$\emptyset$ siɾ-i & \armenian{պիտէի սիրի} \\
2SG & pit-i-s siɾ-i & \armenian{պիտիս սիրի} & pit-e-i-ɾ siɾ-i & \armenian{պիտէիր սիրի} \\
3SG & pit-i-$\emptyset$ siɾ-i & \armenian{պիտի սիրի} & pit-e-$\emptyset$-ɾ siɾ-i & \armenian{պիտէր սիրի} \\
1PL & pit-i-nk siɾ-i & \armenian{պիտինք սիրի} & pit-e-i-nk siɾ-i & \armenian{պիտէինք սիրի} \\
2PL & pit-i-kʰ siɾ-i & \armenian{պիտիք սիրի} & pit-e-i-kʰ siɾ-i & \armenian{պիտէիք սիրի} \\
3PL & pit-i-n siɾ-i & \armenian{պիտին սիրի} & pit-e-i-n siɾ-i & \armenian{պիտէին սիրի} \\
& \multicolumn{2}{l|}{{\deb}-?-{\agr} $\sqrt{}$-? }& \multicolumn{2}{l|}{{\deb}-?-{\pst}-{\agr} $\sqrt{}$-? }
\\\hline \end{tabular}
\end{table}


\translatorHD{Adjarian doesn't ambiguously state if the verb in this alternative strategy is a specific participle, or if all verbs show the same type of final vowel. It's also unclear to me what is the proper glossing for the inflected forms of the particle /piti/. The second vowel alternates between /i,e/ in exactly the same way as the theme vowel of the verb to like /siɾ-e,i-/ in the above paradigms. It's unclear to me if the second vowel in this particle is thus still the same debitive morpheme, vs. a theme vowel, vs. a some other morpheme. }


\paragraph{Other tenses built from participles}

\translatorHD{Adjarian provides a paradigm for something he calls the `debitive past perfect'. It consists of the debitive particle /piti/, plus what appears to be the perfective converb with /-il/, and then the past auxiliary. }


\begin{table}[H]
 \centering
 \caption{Debitive past perfective <\armenian{պարտաւորական գերակատար}> for the verb `to like' in the Tbilisi dialect}
 \label{tab:Tbilisi:morpho:verb:paradigm:debitivePerf}
 \begin{tabular}{|l|ll|}
\hline 
1SG & piti siɾ-il e-i-$\emptyset$ & \armenian{պիտի սիրիլ էի} \\
2SG & piti siɾ-il e-i-ɾ & \armenian{պիտի սիրիլ էիր} \\
3SG & piti siɾ-il e-$\emptyset$-ɾ & \armenian{պիտի սիրիլ էր} \\
1PL & piti siɾ-il e-i-nkʰ & \armenian{պիտի սիրիլ էինք} \\
2PL & piti siɾ-il e-i-kʰ & \armenian{պիտի սիրիլ էիք} \\
3PL & piti siɾ-il e-i-n & \armenian{պիտի սիրիլ էին} \\ 
& \multicolumn{2}{l|}{{\deb} $\sqrt{}$-{\perfcvb} {\aux}-{\pst}-{\agr} }
\\\hline \end{tabular}
\end{table}

\paragraph{Imperative and prohibitive}

\translatorHD{For the imperative 2SG, SEA adds the morph /-iɾ/ after the root for a verb like `to like' (Table \ref{tab:Tbilisi:morpho:verb:paradigm:Imp}). For the 2PL, archaic SEA adds the sequence /-e-t͡sʰ-ekʰ/ after the root such that /-e-t͡sʰ/ forms the aorist stem, while /-ekʰ/ is the agreement marker. More modern registers use just the suffix /-ekʰ/ without the aorist stem. The prohibitive is marked by just adding the proclitic /mi/ before the verb}

\translatorHD{Tbilisi uses the same strategy for the imperative 2PL. FOr the 2SG, the post-root vowel is /e/. It's unclear if this /e/ is a special agremenet morpheme or if it's the theme vowel. For the prohibitive though, Tbilisi ends up using /i/ for the 2SG, and /ekʰ/ for the 2PL. Thus the imperative and negative imperative (prohibitive) are not obviously connected. }


\begin{table}[H]
 \centering
 \caption{Imperative and negative imperative forms <\armenian{հրամայական}> for the verb `to like' in the Tbilisi dialect}
 \label{tab:Tbilisi:morpho:verb:paradigm:Imp}
 \begin{tabular}{|l|ll|lll|l|}
\hline & \multicolumn{2}{l|}{Tbilisi} & \multicolumn{3}{l|}{cf. SEA} & \\\hline
{\imp} 2SG & siɾ-\'e & \armenian{սիրէ՛} & siɾ-iɾ & \armenian{սիրիր} & & $\sqrt{}$-{\imp}.2{\sg}\\
{\imp} 2PL & siɾ-e-t͡sʰ-ekʰ & \armenian{սիրէցէք} & siɾ-e-t͡sʰ-ekʰ & \armenian{սիրեցեք} & dated & $\sqrt{}$-{\thgloss}-{\aor}-{\imp}.2{\pl} \\
 & & & siɾ-ekʰ & \armenian{սիրեք} & modern& $\sqrt{}$-{\imp}.2{\pl} \\ \hline 
{\proh} 2SG & m\i siɾ-i & \armenian{մի՛ սիրի} & mi siɾ-iɾ & \armenian{մի սիրիր} & & {\proh} $\sqrt{}$-{\imp}.2{\sg} \\
{\proh} 2PL & & & mi siɾ-e-t͡sʰ-ekʰ & \armenian{մի սիրեցեք} & dated & {\proh} $\sqrt{}$-{\thgloss}-{\aor}-{\imp}.2{\pl} \\
 & m\i siɾ-ekʰ & \armenian{մի սիրէք} & mi siɾ-ekʰ & \armenian{մի սիրեք} & modern
 & {\proh} $\sqrt{}$-{\imp}.2{\pl} 
\\\hline \end{tabular}
\end{table}
 

\paragraph{Non-finite forms}

\translatorHD{Finally, Adjarian lists the following non-finite forms of this verb (participles or converbs) in Table \ref{tab:Tbilisi:morpho:verb:paradigm:participle}. Note that present participle is also called the subject participle. What Adjarian calls the past participle is differentiated in SEA as a resultative participle with /-ɑt͡s/ and a perfective converb with /-el/.} 

\begin{table}[H]
 \centering
 \caption{Participles or converbs <\armenian{դերբայներ}> for the verb `to like' in the Tbilisi dialect}
 \label{tab:Tbilisi:morpho:verb:paradigm:participle}
 \begin{tabular}{|ll|ll|ll|l|}
\hline & & \multicolumn{2}{l|}{Tbilisi} & \multicolumn{2}{l|}{cf. SEA} & \\
 Infinitive& \armenian{անորոշ} & siɾ-i-l & \armenian{սիրիլ} & siɾ-e-l & \armenian{սիրել} & $\sqrt{}$-{\thgloss}-{\infgloss} \\
 Present & \armenian{ներկայ} & siɾ-oʁ & \armenian{սիրօղ} & siɾ-oʁ &\armenian{սիրող} & $\sqrt{}$-{\sptcp} \\
 Past & \armenian{անցեալ} & siɾ-il & \armenian{սիրիլ} & siɾ-el & \armenian{սիրել} & $\sqrt{}$-{\perfcvb} \\
& & siɾ-i & \armenian{սիրի}& && $\sqrt{}$-{\perfcvb} 
\\
& & siɾ-ɑt͡s & \armenian{սիրած} & siɾ-ɑt͡s & \armenian{սիրած} & $\sqrt{}$-{\rptcp} \\
 Future & \armenian{ապառնի} & siɾ-e-l-u & \armenian{սիրէլու} & siɾ-e-l-u & \armenian{սիրելու} & $\sqrt{}$-{\thgloss}-{\infgloss}-{\futcvb} \\
 & & siɾ-e-l-ɑt͡sʰu & \armenian{սիրէլացու} & & & $\sqrt{}$-{\thgloss}-{\infgloss}-? 
\\\hline \end{tabular}
\end{table}

\begin{adjarianpage}\label{page:57}\end{adjarianpage}% should be 57

\subsubsection{Other conjugation classes}

The other conjugations follow this pattern for the most part. The present, imperfective, and the future use the same strategy. It is only the past perfective and the imperative which have their own construction, in accordance with Classical Armenian.

\translatorHD{He means that the past perfective and imperative have class-specific construction rules, similar to CA and to SEA. Table {tab:Tbilisi:morpho:verb:otherClass:aorist:ICLass} shows the paradigm for the I-Class. The I-Class with theme /-il/ does not exist in SEA, but it does in SWA. For easier contrast, we contrast Tbilisi with SWA. }


\begin{table}[H]
 \centering
 \caption{Past perfectives (aorists) and imperatives for the I-Class verb /-il/ <\armenian{իլ}> `to live' in the Tbilisi dialect}
 \label{tab:Tbilisi:morpho:verb:otherClass:aorist:ICLass}
 \begin{tabular}{|l|ll|ll|}
\hline & \multicolumn{2}{l|}{Tbilisi} & \multicolumn{2}{l|}{cf. SWA} \\ \hline 
 Infinitive & & &\multicolumn{2}{l|}{$\sqrt{}$-{\thgloss}-{\infgloss}} \\
 & & & ɑbɾ-i-l & \armenian{ապրիլ} \\ \hline 
 Past perfective & \multicolumn{2}{l|}{$\sqrt{}$-{\thgloss}-{\aor}-{\pst}-{\agr}} & \multicolumn{2}{l|}{$\sqrt{}$-{\thgloss}-{\aor}-{\pst}-{\agr}}\\
 1SG & ɑpɾ-e-t͡sʰ-ɑ-$\emptyset$ &\armenian{ապրէցա} & ɑbɾ-e-t͡sʰ-ɑ-$\emptyset$ & \armenian{ապրեցայ} \\
2SG & ɑpɾ-e-t͡sʰ-ɑ-ɾ &\armenian{ապրէցար} & ɑbɾ-e-t͡sʰ-ɑ-ɾ & \armenian{ապրեցար} \\
3SG & ɑpɾ-e-t͡sʰ-ɑ-v &\armenian{ապրէցավ}& ɑbɾ-e-t͡sʰ-ɑ-v & \armenian{ապրեցաւ} \\
1PL & ɑpɾ-e-t͡sʰ-ɑ-nkʰ &\armenian{ապրէցանք}& ɑbɾ-e-t͡sʰ-ɑ-ŋkʰ & \armenian{ապրեցանք} \\
2PL & ɑpɾ-e-t͡sʰ-ɑ-kʰ &\armenian{ապրէցաք}& ɑbɾ-e-t͡sʰ-ɑ-kʰ & \armenian{ապրեցաք} \\
3PL & ɑpɾ-e-t͡sʰ-ɑ-n &\armenian{ապրէցան}& ɑbɾ-e-t͡sʰ-ɑ-nʰ & \armenian{ապրեցան} \\
\hline 
Imperative & \multicolumn{2}{l|}{$\sqrt{}$-{\thgloss}-({\aor})-{\agr}} & \multicolumn{2}{l|}{$\sqrt{}$-{\thgloss}-({\aor})-{\agr}}\\
2SG & ɑpɾ-\'i-$\emptyset$ &\armenian{ապրի՛} & ɑbɾ-i-ɾ & \armenian{ապրիր} \\
2PL & ɑpɾ-e-t͡sʰ-ekʰ &\armenian{ապրէցէք}& ɑbɾ-e-t͡sʰ-ɑ-kʰ & \armenian{ապրեցէք} \\\hline 
Prohibitive & \multicolumn{2}{l|}{{\proh} $\sqrt{}$-{\thgloss}?-{\agr}} & \multicolumn{2}{l|}{{\proh} $\sqrt{}$-{\thgloss}-{\agr}}\\
2SG & m\'i ɑpɾ-\'i-$\emptyset$ & \armenian{մի՛ ապրի} &mi ɑbɾ-i-ɾ &\armenian{մի ապրիր} \\
2PL & m\'i ɑpɾ-e-kʰ &\armenian{մի՛ ապրէք}& mi ɑbɾ-i-kʰ & \armenian{մի ապրիք} \\
\hline \end{tabular}
\end{table}

\translatorHD{Another class is the irregular infixed verbs that end in the morph sequence /-n-i-l/. The /n/ is a meaningless stem-extender that's deleted in the past perfective. In SEA, the theme vowel /i/ is replaced by /e/. We show just the Tbilisi and SWA paradigms for illustration (Table \ref{tab:Tbilisi:morpho:verb:otherClass:aorist:infix}). }



\begin{table}[H]
 \centering
 \caption{Past perfectives (aorists) and imperatives for the infixed verb /hɑs-/ `to reach' and /hɑkʰ-/ `to wear' in the Tbilisi dialect}
 \label{tab:Tbilisi:morpho:verb:otherClass:aorist:infix}
 \begin{tabular}{|l|ll|ll|}
\hline & \multicolumn{2}{l|}{Tbilisi} & \multicolumn{2}{l|}{cf. SWA} \\ \hline \hline 
 Infinitive & & & \multicolumn{2}{l|}{$\sqrt{}$-{\vx}-{\thgloss}-{\infgloss}} \\
 & & & hɑs-n-i-l & \armenian{հասնիլ} \\ \hline 
 Past perfective & \multicolumn{2}{l|}{$\sqrt{}$-{\pst}-{\agr}} & \multicolumn{2}{l|}{$\sqrt{}$-{\pst}-{\agr}}\\
 1SG & hɑs-ɑ-$\emptyset$ &\armenian{հասա} & hɑs-ɑ-$\emptyset$ & \armenian{հասայ} \\
2SG & hɑs-ɑ-ɾ &\armenian{հասար}& hɑs-ɑ-ɾ & \armenian{հասար} \\
3SG & hɑs-ɑ-v &\armenian{հասավ}& hɑs-ɑ-v & \armenian{հասաւ} \\
\hline \hline 
 Infinitive & & & \multicolumn{2}{l|}{$\sqrt{}$-{\vx}-{\thgloss}-{\infgloss}} \\
 & & & hɑkʰ-n-i-l & \armenian{հագնիլ} \\ \hline 
Imperative & \multicolumn{2}{l|}{$\sqrt{}$-{\thgloss}-({\aor})-{\agr}} & \multicolumn{2}{l|}{$\sqrt{}$-{\thgloss}-({\aor})-{\agr}}\\
2SG & hɑkʰ-\'i-$\emptyset$ &\armenian{հաքի՛} & hɑkʰ-i-ɾ & \armenian{հագիր} \\
2PL & hɑkʰ-\'ekʰ &\armenian{հաքէ՛ք}& hɑkʰ-\'ekʰ & \armenian{հագէք} \\ \hline
Prohibitive & \multicolumn{2}{l|}{{\proh} $\sqrt{}$-{\vx}-{\thgloss}?-{\agr}} & \multicolumn{2}{l|}{{\proh} $\sqrt{}$-{\vx}-{\thgloss}-{\agr}}\\
2SG & m\'i hɑkʰ-n-\'i-$\emptyset$ & \armenian{մի՛ հաքնի} &mi hɑkʰ-n-i-ɾ &\armenian{մի հագնիր} \\
2PL & m\'i hɑkʰ-n-e-kʰ &\armenian{մի՛ հաքնէք}& mi hɑkʰ-n-i-kʰ & \armenian{մի հագնիք} \\
\hline \end{tabular}
\end{table}

\translatorHD{The A-Class uses the theme /-ɑ/ and it's found in both SEA and Tbilisi. The two dialects utilize the same strategies for the perfective and imperative. Though in the prohibitive, SEA just uses the particle /mi/ plus the imperative, while Tbilisi uses the particle and a different sequence of verbal suffixes (Table \ref{tab:Tbilisi:morpho:verb:otherClass:aorist:aclass}). }



\begin{table}[H]
 \centering
 \caption{Past perfectives (aorists) and imperatives for the A-Class `to stay' in the Tbilisi dialect}
 \label{tab:Tbilisi:morpho:verb:otherClass:aorist:aclass}
 \begin{tabular}{|l|ll|ll|}
\hline & \multicolumn{2}{l|}{Tbilisi} & \multicolumn{2}{l|}{cf. SEA} \\ \hline 
 Infinitive & & & \multicolumn{2}{l|}{$\sqrt{}$-{\thgloss}-{\infgloss}} \\
 & & & mən-ɑ-l & \armenian{մնալ} \\ \hline 
 Past perfective & \multicolumn{2}{l|}{$\sqrt{}$-{\aor}-{\pst}-{\agr}} & \multicolumn{2}{l|}{$\sqrt{}$-{\aor}-{\pst}-{\agr}}\\
 1SG & mən-ɑ-t͡sʰ-i-$\emptyset$ & \armenian{մնացի} & mən-ɑ-t͡sʰ-i-$\emptyset$ & \armenian{մնացի} \\
2SG & mən-ɑ-t͡sʰ-i-ɾ & \armenian{մնացիր} & mən-ɑ-t͡sʰ-i-ɾ & \armenian{մնացիր} \\
3SG & mən-ɑ-t͡sʰ-$\emptyset$-$\emptyset$ & \armenian{մնաց} & mən-ɑ-t͡sʰ-$\emptyset$-$\emptyset$ & \armenian{մնաց} \\
1PL & mən-ɑ-t͡sʰ-i-nkʰ & \armenian{մնացինք} & mən-ɑ-t͡sʰ-i-ŋkʰ & \armenian{մնացինք} \\
2PL & mən-ɑ-t͡sʰ-i-kʰ & \armenian{մնացիք} & mən-ɑ-t͡sʰ-i-kʰ & \armenian{մնացիք} \\
3PL & mən-ɑ-t͡sʰ-i-nʰ & \armenian{մնացին} & mən-ɑ-t͡sʰ-i-nʰ & \armenian{մնացին} \\
\hline 
Imperative & \multicolumn{2}{l|}{$\sqrt{}$-{\thgloss}-({\aor})-{\agr}} & \multicolumn{2}{l|}{$\sqrt{}$-{\thgloss}-({\aor})-{\agr}}\\
2SG & mən-\'ɑ-$\emptyset$ &\armenian{մնա՛} & mən-ɑ-$\emptyset$ &\armenian{մնա}\\
2PL & mən-ɑ-t͡sʰ-\'ekʰ &\armenian{մնացէ՛ք}& mən-ɑ-t͡sʰ-ekʰ &\armenian{մնացեք}\\
 \hline 
Prohibitive & \multicolumn{2}{l|}{{\proh} $\sqrt{}$-{\thgloss}?-{\agr}} & \multicolumn{2}{l|}{{\proh} $\sqrt{}$-{\thgloss}-({\aor})-{\agr}}\\
2SG & m\'i mən-ɑ-$\emptyset$ & \armenian{մի՛ մնա} &mi mən-ɑ-$\emptyset$ &\armenian{մի մնա} \\
2PL & m\'i mən-ɑ-kʰ &\armenian{մի՛ մնաք}& mi mən-ɑ-t͡sʰ-ekʰ & \armenian{մի մնաք} \\
\hline \end{tabular}
\end{table}

\translatorHD{Inchoative verbs end in /-nɑl/ (Table \ref{tab:Tbilisi:morpho:verb:otherClass:aorist:inch}). The aorist patterns the same across Tbilisi and SEA. But the prohibitive again uses different suffixes for Tbilisi. Note that I suspect the Tbilisi verb is a reflex of Classical /herɑnɑl/ <\armenian{հեռանալ}> but I'm not sure. }


\begin{table}[H]
 \centering
 \caption{Past perfectives (aorists) and imperatives for the inchoative `to go away' in the Tbilisi dialect}
 \label{tab:Tbilisi:morpho:verb:otherClass:aorist:inch}
 \begin{tabular}{|l|ll|ll|}
\hline & \multicolumn{2}{l|}{Tbilisi} & \multicolumn{2}{l|}{cf. SEA} \\ \hline 
 Infinitive & & & \multicolumn{2}{l|}{$\sqrt{}$-{\lvgloss}-{\inch}-{\thgloss}-{\infgloss}} \\
 & & & her-ɑ-n-ɑ-l & \armenian{հեռանալ} \\ \hline 
 Past perfective & \multicolumn{2}{l|}{$\sqrt{}$-{\aor}-{\pst}-{\agr}} & \multicolumn{2}{l|}{$\sqrt{}$-{\aor}-{\pst}-{\agr}}\\
 1SG & hir-ɑ-t͡sʰ-ɑ-$\emptyset$ & \armenian{հիռացա} & her-ɑ-t͡sʰ-ɑ-$\emptyset$ & \armenian{հեռացա} \\
2SG & hir-ɑ-t͡sʰ-ɑ-ɾ & \armenian{հիռացար} & her-ɑ-t͡sʰ-ɑ-ɾ & \armenian{հեռացար} \\
3SG & hir-ɑ-t͡sʰ-ɑ-v & \armenian{հիռացավ} & her-ɑ-t͡sʰ-ɑ-v & \armenian{հեռացավ} \\
\hline 
Imperative & \multicolumn{2}{l|}{$\sqrt{}$-{\lvgloss}-{\aor}-{\agr}} & \multicolumn{2}{l|}{$\sqrt{}$-{\lvgloss}-{\aor}-{\agr}}\\
2SG & hir-ɑ-t͡sʰ-\'i &\armenian{հիռացի՛} & her-ɑ-t͡sʰ-iɾ &\armenian{հեռացիր}\\
2PL & hir-ɑ-t͡sʰ-\'ekʰ &\armenian{հիռացէ՛ք} & her-ɑ-t͡sʰ-ekʰ &\armenian{հեռացեք}\\
 \hline 
Prohibitive & \multicolumn{2}{l|}{$\sqrt{}$-{\lvgloss}-{\inch}-{\thgloss}-{\agr}} & \multicolumn{2}{l|}{$\sqrt{}$-{\lvgloss}-{\aor}-{\agr}}\\
2SG & m\'i hir-ɑ-n-ɑ-kʰ & \armenian{մի՛ հիռանա} &m\'i her-ɑ-t͡sʰ-iɾ &\armenian{մի հեռացիր} \\
2PL & m\'i hir-ɑ-n-ɑ-kʰ &\armenian{մի՛ հիռանաք}& m\'i her-ɑ-t͡sʰ-ekʰ & \armenian{մի հեռացեք} \\
\hline \end{tabular}
\end{table}

\section{Literature}

 
As of now, there have been three studies on the Tbilisi dialect. The first was by Gevorg Akhverdian (\armenian{Գէորգ Ախվէրդեան}), in the beginning of his published work on Sayat Nova (\armenian{Սայեաթ-Նօվայ}) \citep[1-41]{SayatNovaAdjarian}, and almost everywhere after that in a note. The second is by the Armenologist \citet{Petermann-1867-Tiflis}. The third is the work by Armenologist Thomson (Томсонъ; \translatorHD{modern Russian: Александр Томсон}) in Russian \citep{Thomson-1890-Tiflis}.\footnote{\translatorHD{Adjarian's original citation is an abbreviation: Грам. современ. Армянскаго языка гор. Тифлисъ \armenian{Բեդրսպուրկ} 1890.}} This work was summarized in German by L. Patrubȧni in the periodical \textit{Sprachwissenschaftliche Abhandlungen}, volume 1, page 289-302.

Besides these, there are many works that are written in the Tbilisi dialect, mostly in comedies. From these, we mention the main ones.


\begin{adjarianpage}\label{page:58}\end{adjarianpage}% should be 58

{\litoverview}

\begin{itemize}
 \item Literature with the Tbilisi dialect
 \begin{itemize}
 \item \armenian{Գէորգ Տէր-Աղէքսանդրեան}
\begin{itemize}
\item \armenian{Թիֆլիսեցոց մտաւոր կեանքը (հաւաքածու Բանաւոր գրականութեան}). \armenian{Թիֆլիս}, 1885
\item \armenian{Ուխտագնացութիւն ի Թէլէթ. Կռունկ} 1860, page 808-922
\end{itemize}
\item \armenian{Գէորգ Ախվերդեան} - \armenian{Սայեաթ-Նօվա. Մոսկվա}, 1852
\item \armenian{Գաբրիէլ Սունդուկեանց} - \armenian{Պէպօ. Թիֆլիս}, 1876
\begin{itemize}
\item \armenian{Խաթաբալա. Թիֆլիս}, 1881
\item \armenian{Քանդած օջավ. Թիֆլիս}, 1882
\item \armenian{Էլի մէկ զոհ. Թիֆլիս}, 1884
\item \armenian{Գիշերվա սարբը խէր է. Թիֆլիս}, 1881
\item \armenian{Օսկան Պետրովիչը դժուխկումը}

 \end{itemize}
 \item \armenian{Երէցփոխեան Գ}. - \armenian{Ա՛յ քեզ օյին. Թիֆլիս}, 1886
\item \armenian{Եսայեան Յարութիւն} - \armenian{Սօնայի նշանդրէքը. Թիֆլիս}, 1904
\item \armenian{Պատկանեան Միքայէլ Նինօյի նշնիլը}
\begin{itemize}
\item \armenian{Վույ քի իմ վէչէր}
\item \armenian{Պէպօյի ակճուր}
\item \armenian{Պառաւներուն խրատ}
\item \armenian{Էս էլ քի մօցիքլութին}


 \end{itemize}
\item \armenian{Փուզինեան Նիկոդայոս} - \armenian{Դալալ Ղՙազօ}
\item \armenian{Փառնակէս} - \armenian{Գրականական երեկոյ. Թիֆլիս}, 1886
\item \armenian{Սարգիս} - \armenian{Ռուստավելի. Ընձու մորթի հագաղ մարդ. Կռունկ}, 1860
\item \armenian{Քախկըցի Դաբաղ Ղՙազօի մասլտաթը. Կռունկ}, 1862, page 454-498
\item \armenian{Գէօ Աւետիսով} - \armenian{Քախկըցի Շաքար Մանուշակեանցի բարովազրի ջուղաբը. Կռունկ}, 1862, page 135-152

 \end{itemize}
 
\end{itemize}

Besides these, there are many small funny articles that have been published in Tbilisi periodicals, especially in \citeauthor{khatabala} and \textit{Hayeli} (\citeauthor{hayeliTiflis}), which we thought would be superfluous to discuss in detail.

\section{Text samples}

{\sampleoverview}
 
\subsection{Sample 1}

Adjarian's source: \armenian{Սայեաթ-Նօվա} (page 139)

\armenian{Պատկիրքըդ ղՙալամօվ քաշած, Թահրըդ օանգէ ռանգ իս անում}.

\armenian{Էրէսիդ խալըն ծածկում է մազիրըդ, խափանզ իս անում}.

\armenian{Բացվիլ իս կարմիր վարթի պէս, բըլբուլի հիգ հանգ իս անում}.

\armenian{Ակռէքըդ օսկումըն շարած, պըռօշըդ մահանգ իս անում։}

~\\

\armenian{Էրէսըդ նուր լուսնի նման՝ քանի կէհա՝ կու բօլըրվի}.

\armenian{Դաստա մաղըդ նամ չի ուզի, առանց հուսիլ կու օլըրվի}.

\armenian{Էնդու համա քու տէնօղըն իր ճամփէմէն կու մօլըրվի}

\armenian{Յիփ մըտնում իս մէջլիսումըն, շանգ շուխի շաբանց իս անում։}

\begin{adjarianpage}\label{page:59}\end{adjarianpage}% should be 59

\armenian{Էրէսըդ տէսնէլու գուքան քաղաք քախկօվ, գիղ գիղի պէս}.

\armenian{Մէռնողըն քիզմէն կու առնէ անմահական դիգ, դիգի պէս}.

\armenian{Յիփ տիզէմէդ ժաժ իս գալի, շըխշըխկում իս ջիղջիղի պէս}.

\armenian{Ինչ կ՚օնիս սանթուր, քամանչէն. զուքսըդ չօնգուր, չանգ իս անում։}

~ \\

\armenian{Ծուցիդ մէչեն վարթ, մանիշակ. սընբուլ ու սուսան իս շինի}.

\armenian{Քու տէրըն բազըն ի՛նչ կ՚օնէ, քու հուտըն ռէհան իս շինի}.

\armenian{Քամին մէչըն անց է կէնում՝ մասիրըդ յէլքան իս շինի}.

\armenian{Աշխարքըն ծօվ, դուն մէչըն նավ ման իս գալի, լանգ իս անում։}


\subsection{Sample 2}

Adjarian's source: \armenian{Էլի մէկ զոհ}, page 1-4

– \armenian{Էսէնգ էլ իր ասածի՜. «հա ու չէ» բէ իմանում է՛լի։}

– \armenian{Բթխիխտ խօ չի՞ս կանա անի խէխճին. տէսնում իս չէ ուզում, զօռօվ բան կուլի՜։}

– \armenian{Զէ՛նդ, զէ՛նդ, դիփ քու միզն է, Բարբարէ, վուր էնէնց ղՙայիմ է կանգնած։}

– \armenian{Վունց չէ, մէ իմ խիլքօվ է ապրում, մէկ էլ քու խիլքօվ։}

– \armenian{Յիս էլ էտ իմ ասում է՜, վուր ինչ ուզից՝ հիդը գրանցիր. ի՛նչ ասավ հիդը բանի տվիր ու վիրչը բէրիր էն տիղը, վուր վունց հօրն է լսում, վունց մօրը։}

– \armenian{Թէ կի նա իր հօր վրա էլ ու մօր վրա էլ խէլօք է, յիս ի՞նչ անիմ էտումը, քա՜։}

– \armenian{Ա՜յ, ա՜յ, էտէնց իս խօսում դիփ վուր իրան էլ իս գժվէցնում է՜։}

– \armenian{Դուն թէ գժվէցնում իս, թէ չէ յիս իսկի էլ չիմ գժվէցնում։}

– \armenian{Ի՜նչ, ի՜նչ}… \armenian{յի՞ս իմ գժվէցնո՜ւմ. արի ու հիդը խօսի։}

– \armenian{Բաս ի՞նչ իս անում. ամալ աշքարա ասում է վուր չէ ուզում, դուն կի ուզոմւ իս զօռով ուզիլ տա. ավար վո՞ւր խէլօքը կու լսէ. հա գժվէցնիլ է ու գժվէցնիլ։}

– \armenian{Տօ, Ստէփան Դանէլիչը, Ստէփան Դանէլիչը, էն միլյօններու տէրը, ախչիկ ըլի տալի ագանչաքօվ պաղանտաքօվ, էնղՙադա փուղ ու բաժնքօվ ու խէլօքը չուզէ՜։ Տօ՛, հազիր ասիս թէ՝ վո՞ւր խէլօքը չի ուղի՜։}

– \armenian{Ի՜նչ անիմ. խօ տէ՞սնում իս վուր նրա ուշկ ու միտքը Անաին է։}

– \armenian{Յիս նրան Անանի կու շանց տամ. հալա մէ մուլափ տա։} 

\begin{adjarianpage}\label{page:60}\end{adjarianpage}% should be 60


\armenian{Է՛սէնց էլ օ՜յին. մարթ ձիյէմէն վէր գա իշին նտսի՜. մարթ խալ ու խալիչէն թօղնէ գէդնի վրա գլօրվի՜}… \armenian{Տէր օղօրմած Աստուձ}… \armenian{Ը՛մ, Յա՛գօր Սիմօնիչ, է՞ս էր քու մտկումն է՛լի}… \armenian{Էնքան էկավ ու գնաց, էնքան տարավ ու էրի (եբեր}), \armenian{ինչրու աղունակի պէս, էրէխիս խիլքէմէն արավ իստակ։ Բաս թէ յիս քու տակը մնացի, Յագօր Սիմօնիչ, էլ յիս մարթ չիմ ըլի, էլ էս գդակը գլխիս գդակ չի ըլի}… \armenian{Էս ի՛նչ ընտիր մօղա էկավ, ա՛խպէր, թէ յաղի (օտար) տանը յադի տղէն տուն ու գուս անէ, ճաշ գնա, իրիգուն գնա, ախչկա հիգ սազ ու բազ (խօսակցիլ) անէ, կ՚օսիս նրա բիձու (հօրեղբայր) տղէն ըլի, ի՞նչ է հարէֆնիր ինք, կ՚օսէ. հարէվնիր չդառան՝ ցավ դառան. յիրգնուց պատիժ էկան գլխիս է՛լի։ Դուն էլ ամէն սահաթի էս ճաշ սարքէ նրանց համա, էս մուրաբէք (քաղցրաւենիք) մօդ տար, էս միրք առնուլ տու}… \armenian{Է՞ս էիր ուզում էլի։ Աստուձ քիզ կու հարցնէ, քի՛զ, Բարբարէ, Միխէիլի գժվէցնօղը բաշտան ջէր (առաջին անգամ) դուն իս։}

– \armenian{Ի՜նչ հանգն իս խօսում, ա՛ մարթ. դուն վուր հէր իս, իս մէր չի՞մ. դուն վուր ուզում իս Միխէիլի լավութինը, յիս չիմ յուզո՜ւմ։ Տէսնում իս իր ասածն է։ Ախար վրէն չարանում իս, էն խիլքի տէրն է վուր վախէնա՜։}

– \armenian{Բարէմց ասա ձէռնէրուն էլ պաչ անիմ է՛լի։}

– \armenian{Օ՜վ է ասում վուր ձէռին պաչ անիս, ամա ամազ իս արի, աշխատանք իս քաշի, ուսում իս տվի, բէրիլ իս մարթ իս շինի. քա, թօղ ի՛նչ քէփը տա էն անէ է՜, քի՞զ ինչ։}





\chapter{Karabakh}
\begin{adjarianpage}\label{page:61}\end{adjarianpage}% should be 61

\section{Background}

Among the 31 New Armenian dialects, the largest and the most widespread is the Karabakh dialect. Its borders go from the north until the final edges of the Caucasus, from the south until Tabriz, from the east until the shores of the Caspian Sea, from the west until Lake Sevan and the borders of the dialects of Yerevan and Karin. 

The Karabakh dialect has likewise gone further beyond these borders to far away places. In Asia minor, next to Smyrna and Aydın there is an old Armenian settlement, which one or two centuries ago came from Karabakh and established those lands. Although this community has for the most part become Turkish-speaking, but there are two places (Burdur and Ödemiş) which still haven't lost their native dialect. 

Because the Karabakhians are a very tall, strict mercantile, clever, capable, and entrepreneurial people, they have recently crossed to the other side of the Caspian sea and came to the various cities of Turkistan, Tatarstan and Manchuria, such as Krasnovodsk, Samarkand, Tashkent, and so on. But because these are not established migrant communities yet, we have not included them into our borders. 

In this way, the main places where the Karabakh dialect is spoken are the following: Shushi, Gandzak, Nukha, Baku, Darband, Shamakhi villages, Agstafa, Dilijan, Gharakilisa, Gazakh province, the northern part of Tabriz in the Armenian-populated village of Mujumbar, the Lilava district of Tabriz which is a settlement of Mujumbar and Karadagh, also in Ottoman Turkey in Ödemiş and Burdur. 

For a dialect that is so widespread, it would not be possible for the dialect to maintain its unity, and it would naturally develop many subidlacts. But the Karabakh dialect is not like this. Baku, Shamakhi villages, ... 

\begin{adjarianpage}\label{page:62}\end{adjarianpage}% should be 62

... Darband, Nukha and its villages, and Bolnis-Khachini are largely the same as the Shushi dialect. Only Gandzak differs from the main dialect, which has more purer forms and according to this is found in the middle between the Karabakh and Yerevan dialects. Pure subdialects are Karadagh and Gazakh, which we later talk about individually.

\section{Phonology}
\subsection{Segment inventory}

The phonetic system of Karabakh is rich both in its vowels and consonants. It has in total 46 sounds. 
\subsubsection{Monophthong vowel inventory}
\translatorHD{It has the monophthong vowels in Table \ref{tab:Karabakh:vowels}.}

\begin{table}[H]
 \centering
 \caption{Monophthongal vowels of the Karabakh dialect}
 \label{tab:Karabakh:vowels}
 \begin{tabular}{|llll|}
 \hline 
 /i/ <\armenian{ի}>& /ʏ/ <\armenian{իւ}> &/ə̟/ <\armenian{ըէ}>& /u/ <\armenian{ու}> \\
 /e/ <\armenian{է}> & /œ/ <\armenian{էօ}>& /ə/ <\armenian{ը}> & /o/ <\armenian{օ}>\\
 /æ/ <\armenian{ա}̈>& && /ɑ/ <\armenian{ա}>
 \\ \hline 
 \end{tabular}
\end{table}

\translatorHD{Note the vowel <\armenian{ըէ}> which is quite difficult to understand. He and other Armenian sources describe this vowel as a midpoint between /ə/ and /e/, and that it forms minimal pairs with /ə/ \citep[6-7]{Sargsyan-2013-KarabakhDictionary}. Auditorily however, it's not clear what acoustic cues are used by dialectologists to create this description. We have found the following contradictory information: 
	\begin{itemize}
\item Adjarian implies this sound is a monophthong. But others have said this is a diphthong \citep[16]{Davtyan-1966-KarabakhDialectMap}. 
\item 
In his earlier work, Adjarian says this sound is the same as the French letter <\'e>, suggesting that the <\armenian{ը, ըէ}> contrast is between /ɛ, e/ \citep[7]{Adjarian-1901-Kharabagh}
 \item Bert Vaux reports that such a vowel sounds like a backed schwa and easily confusable as a /we, u̯e/ sequence.
 \item Victoria Khurshudyan reports that this vowel is backer than the schwa, and that it's close to Russian <ы> /ɨ/. 
 \item A speaker from Karabakh told me that, for her, the sounds /ə/ and <\armenian{ըէ}> are interchangeable, and that <\armenian{ըէ}> feels like a diphthong but with the schwa part shorter.
 \item In some of the few acoustic samples that I could find, I sense that this vowel had a wide variation of pronunciations -- sometimes it sounds lower, higher, or with an offglide /j/ -- it has been hard for me to pinpoint it down to a single type of central vowel.
\end{itemize}
 For this translation, I treat the closest IPA approximation as /ə̟/ -- an /ə/ with some fronting. Previous transcriptions that I've come across include <\mydoT{ə}> \citep[25]{Adjarian-1909-ClassificationArmenianDialect}. The Karabakh dialect is not moribund, so future work could look into the exact yer variable acoustic values of this vowel. }

\subsubsection{Diphthong vowel inventory}\label{sec:Karabakh:phono:segments:diph}

\translatorHD{Karabakh has the diphthong vowels in Table \ref{tab:Karabakh:diphthong}. } 

\begin{table}[H]
 \centering
 \caption{Diphthongal vowels of the Karabakh dialect}
 \label{tab:Karabakh:diphthong}
 \begin{tabular}{|ll|}
 \hline 
 /ei̯/ <\armenian{էյ}> & /oi̯/ <\armenian{օյ}> \\
 & /u̯ɑ/ <\armenian{ուա}> 
 \\ \hline
 \end{tabular}
\end{table}


\subsubsection{Consonant inventory}


\translatorHD{It has the consonants in Table \ref{tab:Karabakh:Consonant}. } 

\begin{table}[H]
 \centering
 \caption{Consonants of the Karabakh dialect}
 \label{tab:Karabakh:Consonant}
 \begin{tabular}{|l|lll|llll|lll|}
 \hline 
 & \multicolumn{3}{l|}{Labial}& \multicolumn{4}{l|}{Coronal}& \multicolumn{3}{l|}{Dorsal/Back}\\
 Stops& /b/ & /p/ & /pʰ/ & /d/ & /t/ & /tʰ/& & /ɡ/ & /k/ & /kʰ/ 
 \\
 & <\armenian{բ}> &<\armenian{պ}>& <\armenian{փ}> &<\armenian{դ}>& <\armenian{տ}> &<\armenian{թ}>&& <\armenian{գ}>& <\armenian{կ}>& <\armenian{ք}>\\
 & & & & & & && /ɡʲ/ & /kʲ/ & /kʰʲ/ \\
 & & & && & && <\armenian{գյ}>& <\armenian{կյ}>& <\armenian{քյ}>\\
 \hline 
 Affricates & && & /d͡z/ & /t͡s/ & /t͡sʰ/ & && & \\
 & && &<\armenian{ձ}>& <\armenian{ծ}>& <\armenian{ց}> & & & & \\
 & && & /d͡ʒ/ & /t͡ʃ/ & / t͡ʃʰ/ && & & \\
 & & & &<\armenian{ջ}>& <\armenian{ճ}>& <\armenian{չ}> & & && \\
 \hline 
 Fricatives& /v/ && &/s/& /z/& /ʃ/& /ʒ/& /χ/ & /ʁ/ & /h/ \\
 & <\armenian{վ}>&& & <\armenian{ս}>& <\armenian{զ}>& <\armenian{շ}>& <\armenian{ժ}>& <\armenian{խ}> & <\armenian{ղ}> & <\armenian{հ}> \\
 && & & & & & & & & /hʲ/ \\
 && & & & & & & & & <\armenian{հյ}>
\\ \hline 
 Sonorants & /m/ & /n/& & /ɾ/ & /r/& /l/ & /j/ && & \\
& <\armenian{մ}> & <\armenian{ն}> && <\armenian{ր}>& <\armenian{ռ}>& <\armenian{լ}>& <\armenian{յ}> && & 
\\ \hline 
 \end{tabular}
\end{table}

\subsection{Stress and vowel deletion}

As in the Yerevan and Tbilisi dialect, the Karabakh dialect places stress on on the penultimate syllable. In these two other dialects, the change in stress did not cause other changes. But in the Karabakh dialect, this change has caused the lost of vowels. Every vowel that is found before stress is either turned into a schwa /ə/ <\armenian{ը}> or completely lost (Table \ref{tab:Karabakh:stress:deletion}).\footnote{\translatorHD{The word `swallow' in Classical was <\armenian{ծիծեռն}> /t͡sit͡serən/. The form I provide is hypothetical, but Adjarian treats it as non-hypothetical. }}

\begin{table}[H]
 \centering
 \caption{Penultimate stress and vowel deletion in the Karabakh dialect}
 \label{tab:Karabakh:stress:deletion}
 \begin{tabular}{|l|ll|ll|ll|}
 \hline & \multicolumn{2}{l|}{Classical Armenian}& \multicolumn{2}{l|}{> Karabakh }& \multicolumn{2}{l|}{cf. SEA }
 \\
 `gospel'& ɑu̯etɑɾ\'ɑn & \armenian{աւետարան} & əvət\'ɑɾɑn & \armenian{ըվըտա՛րան} & ɑvetɑɾ\'ɑn & \armenian{ավետարան} \\
 `request'& ɑɬɑt͡ʃ\'ɑnkʰ & \armenian{աղաչանք} & əʁ\'ɑt͡ʃʰɑnkʰ & \armenian{ըղա՛չանք} & ɑʁɑt͡ʃ\'ɑŋkʰ & \armenian{աղաչանք} \\
 & & & ʁ\'ɑt͡ʃʰɑnkʰ & \armenian{ղա՛չանք} & & \\
 `request'& nɑ{wɑ}kɑt\'ikʰ & \armenian{նաւակատիք} & nəvək\'ɑtei̯ɡʲ& \armenian{նըվըկա՛տէյգյ} & nɑvɑkɑt\'ikʰ & \armenian{նավակատիք} \\
 `fawning'& eɾespɑʃtutʰ\'iu̯n & \armenian{երեսպաշտութիւն} & əɾəspəʃt\'otʰun& \armenian{ըրըսպըշտօ՛թուն} & jeɾespɑʃtutʰj\'un & \armenian{երեսպաշտություն} \\
 `today'& ɑi̯s\'ɑu̯ɾ & \armenian{այսաւր} & soɾ & \armenian{սօր} & ɑjs\'oɾ & \armenian{այսօր} \\
 `swallow'& *t͡sit͡sern\'ɑk & \armenian{ծիծեռնակ} & t͡sʰət͡sʰ\'ernɑk & \armenian{ցըցէ՛ռնակ} & t͡sit͡sern\'ɑk & \armenian{ծիծեռնակ} \\
 `razor'& ɑt͡sel\'i & \armenian{ածելի} & t͡s\'ili & \armenian{ծի՛լի} & ɑt͡sel\'i & \armenian{ածելի} \\
 `pigeon'& ɑɬɑu̯n\'i & \armenian{աղաւնի} & jəʁ\'ojnei̯ɡʲ& \armenian{յըղօ՛յնէյգյ} & ɑt͡sel\'i & \armenian{աղավնի} \\
 & & & ʁ\'oi̯nei̯ɡʲ & \armenian{ղօ՛յնէյգյ} & & \\
 `evening'& eɾek\'oi̯ & \armenian{երեկոյ} & ɾ\'ʏɡʏ & \armenian{րի՛ւգիւ} & jeɾek\'o & \armenian{երեկո} \\
 \hline
 \end{tabular}
 
\end{table} 

\subsection{Sound changes}

Of the splendid phonetic changes in the dialect, we mention the following important ones.

\subsubsection{Monophthong vowel changes}

\paragraph{Classical Armenian /ɑ/ <\armenian{ա}>} 

Classical Armenian /ɑ/ <\armenian{ա}> became /ɑ/ <\armenian{ա}> for the words in Table \ref{tab:Karabakh:phonology:soundChange:monoph:a:a}. 


\begin{table}[H]
 \centering
 \caption{Change from Classical Armenian /ɑ/ <\armenian{ա}> to /ɑ/ <\armenian{ա}> in the Karabakh dialect}
 \label{tab:Karabakh:phonology:soundChange:monoph:a:a}
 \begin{tabular}{|l| ll|ll| ll|}
 \hline & \multicolumn{2}{l|}{Classical Armenian} &\multicolumn{2}{l|}{> Karabakh} & \multicolumn{2}{l|}{cf. SEA} \\ 
`thick' &tʰ\'ɑnd͡zəɾ& \armenian{թանձր} & tʰ\'ɑnd͡zəɾ & \armenian{թա՛նձըր} &tʰ\'ɑnd͡zəɾ& \armenian{թանձր} \\
`to rise' &bɑɾd͡zɾɑn\'ɑl& \armenian{բարձրանալ} & bət͡sʰəɾ\'ɑnɑl & \armenian{պըցըրա՛նալ} &bɑɾt͡sʰɾɑn\'ɑl& \armenian{բարձրանալ} \\
`account' &hɑm\'ɑɾ& \armenian{համար} & məhɑɾ & \armenian{մըհար} &hɑm\'ɑɾ& \armenian{համար} \\
 \hline 
 \end{tabular}
\end{table}

Classical Armenian /ɑ/ <\armenian{ա}> became /æ/ <\armenian{ա}̈> for the words in Table \ref{tab:Karabakh:phonology:soundChange:monoph:a:ae}. 


\begin{table}[H]
 \centering
 \caption{Change from Classical Armenian /ɑ/ <\armenian{ա}> to /æ/ <\armenian{ա}̈> in the Karabakh dialect}
 \label{tab:Karabakh:phonology:soundChange:monoph:a:ae}
 \begin{tabular}{|l| ll|ll| ll|}
 \hline & \multicolumn{2}{l|}{Classical Armenian} &\multicolumn{2}{l|}{> Karabakh} & \multicolumn{2}{l|}{cf. SEA} \\ 
`tail' &ɑɡ\'i& \armenian{ագի} & h\'ækʰʏ & \armenian{հա̈՛քիւ} &ɑɡ\'i& \armenian{ագի} \\
`field' &\'ɑnd& \armenian{անդ} & hænd & \armenian{հա̈նդ} & \'ɑnd& \armenian{անդ} \\
`good' &l\'ɑu̯& \armenian{լաւ} & læv & \armenian{լա̈վ} & l\'ɑv& \armenian{լաւ} \\
`lightning' &kɑi̯t͡s\'ɑkən & \armenian{կայծակն} & k\'æt͡sɑk & \armenian{կա̈՛ծակ} & kɑjt͡s\'ɑk & \armenian{կայծակ} \\
`spring' &ɡɑɾ\'un & \armenian{գարուն} & kʲ\'æɾunkʰ & \armenian{կյա̈՛րունք} & ɡɑɾ\'un & \armenian{գարուն} \\
 \hline 
 \end{tabular}
\end{table}


Classical Armenian /ɑ/ <\armenian{ա}> became /e/ <\armenian{է}> for the words in Table \ref{tab:Karabakh:phonology:soundChange:monoph:a:e}. 


\begin{table}[H]
 \centering
 \caption{Change from Classical Armenian /ɑ/ <\armenian{ա}> to /e/ <\armenian{է}> in the Karabakh dialect}
 \label{tab:Karabakh:phonology:soundChange:monoph:a:e}
 \begin{tabular}{|l| ll|ll| ll|}
 \hline & \multicolumn{2}{l|}{Classical Armenian} &\multicolumn{2}{l|}{> Karabakh} & \multicolumn{2}{l|}{cf. SEA} \\ 
`thin' &bɑɾ\'ɑk & \armenian{բարակ} & b\'eɾɑk & \armenian{պէ՛րակ} & bɑɾ\'ɑk & \armenian{բարակ} \\
`thing' &bɑn & \armenian{բան} & ben & \armenian{պէն} & bɑn & \armenian{բան} \\
`cotton' &bɑmb\'ɑk & \armenian{բամբակ} & b\'embɑk & \armenian{պէ՛մբակ} & bɑmb\'ɑk & \armenian{բամբակ} \\
`turtle dove' &tɑtɾ\'ɑk& \armenian{տատրակ} & t\'etɾɑk & \armenian{տէ՛տրակ} & tɑtɾ\'ɑk & \armenian{տատրակ} \\
`water-mill' &d͡ʒəɾɑɬ\'ɑt͡sʰ& \armenian{ջրաղաց} & t͡ʃ\'eʁɑt͡sʰ &\armenian{ճէ՛ղաց} & d͡ʒəɾɑʁ\'ɑt͡sʰ& \armenian{ջրաղաց} \\
`empty' &dɑt\'ɑɾk& \armenian{դատարկ} & t\'eɾtɑk &\armenian{տէ՛րտակ} & dɑt\'ɑɾk & \armenian{դատարկ} \\
`to conquer' &jɑɬtʰ\'el & \armenian{յաղթել} & h\'eχnel &\armenian{յէ՛խնէլ} & hɑχtʰ\'el & \armenian{հաղթել} \\
 \hline 
 \end{tabular}
\end{table}

\paragraph{Classical Armenian /e/ <\armenian{ե}>} 

Classical Armenian /e/ <\armenian{ե}> became /e/ <\armenian{է}> for the words in Table \ref{tab:Karabakh:phonology:soundChange:monoph:e:e}. 


\begin{table}[H]
 \centering
 \caption{Change from Classical Armenian /e/ <\armenian{ե}> to /e/ <\armenian{է}> in the Karabakh dialect}
 \label{tab:Karabakh:phonology:soundChange:monoph:e:e}
 \begin{tabular}{|l| ll|ll| ll|}
 \hline & \multicolumn{2}{l|}{Classical Armenian} &\multicolumn{2}{l|}{> Karabakh} & \multicolumn{2}{l|}{cf. SEA} \\ 
`wife's father' &ɑn\'eɾ& \armenian{աներ} & h\'ɑneɾ & \armenian{հա՛նէր} &ɑn\'eɾ& \armenian{աներ} \\
`grave' &ɡeɾezm\'ɑn& \armenian{գերեզման} & kʲəɾ\'ezmɑn & \armenian{կյըրէ՝զման} &ɡeɾezm\'ɑn & \armenian{գերեզման} \\
`thirty' &eɾes\'un& \armenian{երեսուն} & əɾ\'esun & \armenian{ըրէ՛սուն} &jeɾes\'un & \armenian{երեսուն} \\
`hand' &d͡zer-kʰ (plural) & \armenian{ձեռք} & t͡serkʰ & \armenian{ծէռք} &d͡zerkʰ & \armenian{ձեռք} \\
`mouth' &beɾ\'ɑn & \armenian{բերան} & p\'eɾɑn & \armenian{պէ՛րան} &beɾ\'ɑn & \armenian{բերան} \\
 \hline 
 \end{tabular}
\end{table}

\begin{adjarianpage}\label{page:63}\end{adjarianpage}% should be 63


Classical Armenian /e/ <\armenian{ե}> became /ə̟/ <\armenian{ըէ}> for the words in Table \ref{tab:Karabakh:phonology:soundChange:monoph:e:əFront}, though some Karabakh villages use /ə/ <\armenian{ը}>. 


\begin{table}[H]
 \centering
 \caption{Change from Classical Armenian /e/ <\armenian{ե}> to /ə̟/ <\armenian{ըէ}> or /ə/ <\armenian{ը}> or in the Karabakh dialect}
 \label{tab:Karabakh:phonology:soundChange:monoph:e:əFront}
 \begin{tabular}{|l| ll|ll| ll|}
 \hline & \multicolumn{2}{l|}{Classical Armenian} &\multicolumn{2}{l|}{> Karabakh} & \multicolumn{2}{l|}{cf. SEA} \\ 
`you.{\pl}.{\dat}' &d͡zez & \armenian{ձեզ} & t͡sə̟z& \armenian{ծըէզ} &d͡zez & \armenian{ձեզ} \\
 & & & t͡səz& \armenian{ծըզ} & & \\
`our' &meɾ & \armenian{մեր} & mə̟ɾ& \armenian{մըէր} &meɾ & \armenian{մեր} \\
 & & & məɾ& \armenian{մըր} & & \\
`big' &met͡s & \armenian{մեծ} & mə̟t͡st͡s& \armenian{մըէծծ} &met͡s & \armenian{մեծ} \\
 & & & mət͡s& \armenian{մըծ} & & \\
`bridegroom' &pes\'ɑi̯ & \armenian{փեսայ} & pʰ\'ə̟sɑ& \armenian{փըէ՛սա} &pes\'ɑ & \armenian{փեսա} \\
 & & & pʰəsɑ& \armenian{փըսա} & & \\
`to die' &merɑn\'el & \armenian{մեռանել} & m\'ə̟rnel& \armenian{մըէ՛ռնէլ} &mern\'el & \armenian{մեռնել} \\
 & & & mərnel& \armenian{մըռնէլ} & & \\
 \hline 
 \end{tabular}
\end{table}


Classical Armenian /e/ <\armenian{ե}> became /je/ <\armenian{յէ}> for the words in Table \ref{tab:Karabakh:phonology:soundChange:monoph:e:je}. This happens at the beginning of both monosyllabic and polysyllabic words.



\begin{table}[H]
 \centering
 \caption{Change from Classical Armenian /e/ <\armenian{ե}> to /je/ <\armenian{յէ}> or in the Karabakh dialect}
 \label{tab:Karabakh:phonology:soundChange:monoph:e:je}
 \begin{tabular}{|l| ll|ll| ll|}
 \hline & \multicolumn{2}{l|}{Classical Armenian} &\multicolumn{2}{l|}{> Karabakh} & \multicolumn{2}{l|}{cf. SEA} \\ 
`church' &ekeɬet͡sʰ\'i & \armenian{եկեղեցի} & j\'eχt͡se & \armenian{յէ՛խծէ} &jekeʁet͡sʰ\'i & \armenian{եկեղեցի} \\
`sky' &eɾk\'in-kʰ (plural) & \armenian{երկինք} & j\'eɾɡinkʰʲ & \armenian{յէ՛րգինքյ} &jeɾk\'iŋkʰ & \armenian{երկինք} \\
`ox' &\'ezən & \armenian{եզն} & j\'eznə & \armenian{յէ՛զնը} &jez & \armenian{եզ} \\
`I' & es & \armenian{ես} & jes & \armenian{յէս} &jes & \armenian{ես} \\
 \hline 
 \end{tabular}
\end{table}


Classical Armenian /e/ <\armenian{ե}> became /i/ <\armenian{ի}> for some words (Table \ref{tab:Karabakh:phonology:soundChange:monoph:e:i}a). This changes happens especially in those words where the Classical form had two two subsequent /e/ <e> sounds (Table \ref{tab:Karabakh:phonology:soundChange:monoph:e:i}b). 



\begin{table}[H]
 \centering
 \caption{Change from Classical Armenian /e/ <\armenian{ե}> to /i/ <\armenian{ի}> or in the Karabakh dialect}
 \label{tab:Karabakh:phonology:soundChange:monoph:e:i}
 \begin{tabular}{|ll| ll|ll| ll|}
 \hline & & \multicolumn{2}{l|}{Classical Armenian} &\multicolumn{2}{l|}{> Karabakh} & \multicolumn{2}{l|}{cf. SEA} \\ 
a. & `thread' &tʰel & \armenian{թել} &tʰil & \armenian{թիլ} &tʰel & \armenian{թել} \\
& `sun' &ɑɾeɡ\'ɑkən & \armenian{արեգակն} &əɾ\'ikʰʲnɑk & \armenian{ըրի՛քյնակ} &ɑɾeɡ\'ɑk & \armenian{արեգակ} \\
& `more' & ɑɾɑ{w\'e}l & \armenian{առաւել} & \'ivil & \armenian{ի՛վիլ} &ɑɾɑv\'el & \armenian{առավել} \\
b. & `ladle' &ʃeɾ\'epʰ & \armenian{շերեփ} &ʃ\'iɾepʰ & \armenian{շի՛րէփ} &ʃeɾ\'epʰ & \armenian{շերեփ} \\
 & `daytime' &t͡sʰeɾ\'ek & \armenian{ցերեկ} &t͡sʰ\'iɾek & \armenian{ցի՛րէկ} &t͡sʰeɾ\'ek & \armenian{ցերեկ} \\
 & `face' &eɾ\'es & \armenian{երես} &\'iɾes & \armenian{ի՛րէս} &jeɾ\'es & \armenian{երես} \\
 & `leaf' &teɾ\'eu̯ & \armenian{տերեւ} &t\'iɾev & \armenian{տի՛րէվ} &teɾ\'ev & \armenian{տերև} \\
 & `three' &eɾ\'ekʰ & \armenian{երեք} & \'iɾekʰ & \armenian{ի՛րէք} &jeɾ\'ekʰ & \armenian{երեք} \\
 & `light (weight)' &tʰetʰ\'eu̯ & \armenian{թեթեւ} & tʰ\'itʰev & \armenian{թի՛թէվ} &tʰetʰ\'ev & \armenian{թեթև} \\
 \hline 
 \end{tabular}
\end{table}

\paragraph{Classical Armenian /ē/ <\armenian{է}>} 

Classical Armenian /ē/ <\armenian{է}> became /e/ <\armenian{է}> for the words in Table \ref{tab:Karabakh:phonology:soundChange:monoph:ee:e}. 


\begin{table}[H]
 \centering
 \caption{Change from Classical Armenian /ē/ <\armenian{է}> to /e/ <\armenian{է}> in the Karabakh dialect}
 \label{tab:Karabakh:phonology:soundChange:monoph:ee:e}
 \begin{tabular}{|l| ll|ll| ll|}
 \hline & \multicolumn{2}{l|}{Classical Armenian} &\multicolumn{2}{l|}{> Karabakh} & \multicolumn{2}{l|}{cf. SEA} \\ 
`gum' &χēʒ & \armenian{խէժ} & χ\'eʒnə & \armenian{խէ՛ժնը} &χeʒ & \armenian{խեժ} \\
`female' & ēɡ & \armenian{էգ} & ekʰʲ & \armenian{էքյ} & eɡ & \armenian{էգ} \\
`fox' & ɑɬu.\'ēs & \armenian{աղուէս} & \'ɑʁves & \armenian{ա՛ղվէս} & ɑʁv\'es & \armenian{աղվես} \\
`donkey' & ēʃ & \armenian{էշ} & eʃ & \armenian{էշ} & eʃ & \armenian{էշ} \\
 \hline 
 \end{tabular}
\end{table}


Classical Armenian /ē/ <\armenian{է}> became /ə̟/ <\armenian{ըէ}> for the words in Table \ref{tab:Karabakh:phonology:soundChange:monoph:ee:əFront}. 


\begin{table}[H]
 \centering
 \caption{Change from Classical Armenian /ē/ <\armenian{է}> to /ə̟/ <\armenian{ըէ}> in the Karabakh dialect}
 \label{tab:Karabakh:phonology:soundChange:monoph:ee:əFront}
 \begin{tabular}{|l| ll|ll| ll|}
 \hline & \multicolumn{2}{l|}{Classical Armenian} &\multicolumn{2}{l|}{> Karabakh} & \multicolumn{2}{l|}{cf. SEA} \\ 
`half' &kēs & \armenian{կէս} & kə̟s & \armenian{կըէս} &kes & \armenian{կես} \\
`point' &kēt & \armenian{կէտ} & kə̟t & \armenian{կըէտ} &ket & \armenian{կետ} \\
 \hline 
 \end{tabular}
\end{table}


Classical Armenian /ē/ <\armenian{է}> became /i/ <\armenian{ի}> for the words in Table \ref{tab:Karabakh:phonology:soundChange:monoph:ee:i}. 


\begin{table}[H]
 \centering
 \caption{Change from Classical Armenian /ē/ <\armenian{է}> to /i/ <\armenian{ի}> in the Karabakh dialect}
 \label{tab:Karabakh:phonology:soundChange:monoph:ee:i}
 \begin{tabular}{|l| ll|ll| ll|}
 \hline & \multicolumn{2}{l|}{Classical Armenian} &\multicolumn{2}{l|}{> Karabakh} & \multicolumn{2}{l|}{cf. SEA} \\ 
`heap' &dēz & \armenian{դէզ} & tə̟z & \armenian{տիզ} &dez & \armenian{դեզ} \\
`heap' &ʃəɾēʃ & \armenian{շրէշ} & ʃɾiʃ & \armenian{շրիշ} &ʃəɾeʃ & \armenian{շրեշ} \\
 \hline 
 \end{tabular}
\end{table}


\paragraph{Classical Armenian /i/ <\armenian{ի}>} 

Classical Armenian /i/ <\armenian{ի}> became /i/ <\armenian{ի}> for the words in Table \ref{tab:Karabakh:phonology:soundChange:monoph:i:i}. 


\begin{table}[H]
 \centering
 \caption{Change from Classical Armenian /i/ <\armenian{ի}> to /i/ <\armenian{ի}> in the Karabakh dialect}
 \label{tab:Karabakh:phonology:soundChange:monoph:i:i}
 \begin{tabular}{|l| ll|ll| ll|}
 \hline & \multicolumn{2}{l|}{Classical Armenian} &\multicolumn{2}{l|}{> Karabakh} & \multicolumn{2}{l|}{cf. SEA} \\ 
`nine' &\'inən & \armenian{ինն} & \'innə & \armenian{ի՛ննը} &\'inən & \armenian{ինն} \\
`full' &l\'i & \armenian{լի} & l\'ijnə & \armenian{լի՛յնը} &l\'i & \armenian{լի} \\
`louse' &od͡ʒ\'il & \armenian{ոջիլ} & v\'it͡ʃʰil & \armenian{վի՛չիլ} &vot͡ʃʰ\'il& \armenian{ոջիլ} \\
`wine' &ɡin\'i & \armenian{գինի} & k\'ini &\armenian{կի՛նի} &ɡin\'i & \armenian{գինի} \\
`what' &int͡ʃʰ & \armenian{ինչ} & hint͡ʃʰ &\armenian{հինչ}&int͡ʃʰ & \armenian{ինչ} \\
`chickpea' &sis\'erən & \armenian{սիսեռն} & s\'isernə &\armenian{սի՛սէռնը}& sis\'er & \armenian{սիսեռ} \\
 \hline 
 \end{tabular}
\end{table}


Classical Armenian /i/ <\armenian{ի}> became /e/ <\armenian{է}> for the words in Table \ref{tab:Karabakh:phonology:soundChange:monoph:i:e}.\footnote{\translatorHD{For the word `barley', Adjarian gives a Karabakh form <\armenian{կյէ՛օրէ}>. The most straightforward transcription would be /kʲ\'œɾe/, but then this contradicts his generalization. An alternative less obvious transcription would be /kʲ\'e.oɾe/ with vowel hiatus. }}


\begin{table}[H]
 \centering
 \caption{Change from Classical Armenian /i/ <\armenian{ի}> to /e/ <\armenian{է}> in the Karabakh dialect}
 \label{tab:Karabakh:phonology:soundChange:monoph:i:e}
 \begin{tabular}{|l| ll|ll| ll|}
 \hline & \multicolumn{2}{l|}{Classical Armenian} &\multicolumn{2}{l|}{> Karabakh} & \multicolumn{2}{l|}{cf. SEA} \\ 
`nose' &kʰitʰ & \armenian{քիթ} &kʰetʰ & \armenian{քէթ} &kʰitʰ & \armenian{քիթ} \\
`year' &tɑɾ\'i & \armenian{տարի} &t\'ɑɾe & \armenian{տա՛րէ} &tɑɾ\'i & \armenian{տարի} \\
`church' &ekeɬet͡sʰ\'i & \armenian{եկեղեցի} &j\'eχt͡se & \armenian{յէ՛խծէ} &ekeɬet͡sʰ\'i & \armenian{եկեղեցի} \\
`yellow' &deɬ\'in & \armenian{դեղին} &t\'eʁen & \armenian{տէ՛ղէն} &deʁ\'in & \armenian{դեղին} \\
`barley' &ɡɑɾ\'i & \armenian{գարի} &kʲ\'œɾe & \armenian{կյէ՛օրէ} &ɡɑɾ\'i & \armenian{գարի} \\
`bitter' &leɬ\'i & \armenian{լեղի} &lʁe & \armenian{լղէ} &leʁ\'i & \armenian{լեղի} \\
`five' &hinɡ & \armenian{հինգ} &henɡʲ & \armenian{հէնգյ} &hiŋɡ & \armenian{հինգ} \\
 \hline 
 \end{tabular}
\end{table}


Classical Armenian /i/ <\armenian{ի}> became /ə̟/ <\armenian{ըէ}> for the words in Table \ref{tab:Karabakh:phonology:soundChange:monoph:i:əFront}.


\begin{table}[H]
 \centering
 \caption{Change from Classical Armenian /i/ <\armenian{ի}> to /ə̟/ <\armenian{ըէ}> in the Karabakh dialect}
 \label{tab:Karabakh:phonology:soundChange:monoph:i:əFront}
 \begin{tabular}{|l| ll|ll| ll|}
 \hline & \multicolumn{2}{l|}{Classical Armenian} &\multicolumn{2}{l|}{> Karabakh} & \multicolumn{2}{l|}{cf. SEA} \\ 
`one' &mi & \armenian{մի} &mə̟ɾ & \armenian{մըէր} &mi & \armenian{մի} \\ 
`oak' & kɑɬn\'i & \armenian{կաղնի} &k\'ɑʁnə̟ & \armenian{կա՛ղնըէ} &kɑʁn\'i & \armenian{կաղնի} \\ 
`month' & ɑm\'is & \armenian{ամիս} & \'ɑmə̟s & \armenian{ա՛մըէս} &ɑm\'is & \armenian{ամիս} \\ 
`meat' & mis & \armenian{միս} & mə̟s & \armenian{մըէս} &mis & \armenian{միս} \\ 
`apricot' & t͡siɾ\'ɑn & \armenian{ծիրան} & t͡s\'ə̟ɾɑn & \armenian{ծըէ՛րան} &t͡siɾɑn & \armenian{ծիրան} \\ 
`heart' & siɾt & \armenian{սիրտ} & sə̟ɾt & \armenian{սըէրտ} &siɾt & \armenian{սիրտ} \\ 
 \hline 
 \end{tabular}
\end{table}


\paragraph{Classical Armenian /o/ <\armenian{ո}>} 

Classical Armenian /o/ <\armenian{ո}> became /o/ <\armenian{օ}> for the words in Table \ref{tab:Karabakh:phonology:soundChange:monoph:o:o}. 


\begin{table}[H]
 \centering
 \caption{Change from Classical Armenian /o/ <\armenian{ո}> to /o/ <\armenian{օ}> in the Karabakh dialect}
 \label{tab:Karabakh:phonology:soundChange:monoph:o:o}
 \begin{tabular}{|l| ll|ll| ll|}
 \hline & \multicolumn{2}{l|}{Classical Armenian} &\multicolumn{2}{l|}{> Karabakh} & \multicolumn{2}{l|}{cf. SEA} \\ 
`ash' &moχ\'iɾ & \armenian{մոխիր} & m\'oχeɾ & \armenian{մօ՛խէր} & moχ\'iɾ & \armenian{մոխիր} \\
`kernel' &koɾ\'iz & \armenian{կորիզ} & k\'oɾə̟z & \armenian{կօ՛րըէզ} & koɾ\'iz& \armenian{կորիզ} \\
`wheat' &t͡sʰoɾe̯\'ɑn & \armenian{ցորեան} & t͡sʰ\'oɾen & \armenian{ցօ՛րէն} & t͡sʰoɾ\'en& \armenian{ցորեն} \\
 \hline 
 \end{tabular}
\end{table}


Classical Armenian /o/ <\armenian{ո}> became /œ/ <\armenian{էօ}> for the words in Table \ref{tab:Karabakh:phonology:soundChange:monoph:o:eo}, but only next to the sounds /ɾ, r, ʁ, χ/ <\armenian{ր, ռ, ղ, խ}>. \footnote{\translatorHD{It's unclear though if the vowel has to be next to those sounds in the Classical form vs. the modern form. } }


\begin{table}[H]
 \centering
 \caption{Change from Classical Armenian /o/ <\armenian{ո}> to /œ/ <\armenian{էօ}> in the Karabakh dialect}
 \label{tab:Karabakh:phonology:soundChange:monoph:o:eo}
 \begin{tabular}{|l| ll|ll| ll|}
 \hline & \multicolumn{2}{l|}{Classical Armenian} &\multicolumn{2}{l|}{> Karabakh} & \multicolumn{2}{l|}{cf. SEA} \\ 
`to twist' &oloɾ\'el & \armenian{ոլորել} & həll\'œɾel & \armenian{հըլլէօ՛րէլ} & voloɾ\'el& \armenian{ոլորել} \\
`valley' &d͡zoɾ & \armenian{ձոր} & t͡sœɾ & \armenian{ծէօր} & d͡zoɾ & \armenian{ձոր} \\
`four' &t͡ʃʰoɾs & \armenian{չորս} & & & t͡ʃʰoɾs & \armenian{չորս} \\
 &t͡ʃʰoɾkʰ & \armenian{չորք} & t͡sœɾkʰ & \armenian{չէօրք} & & \\
`plum' &sɑloɾ & \armenian{սալոր} & ʃəllœɾ & \armenian{շըլլէօր} & sɑloɾ & \armenian{սալոր} \\
`thief' &ɡoɬ & \armenian{գող} & kʲœʁ & \armenian{կյէօղ} & ɡoʁ & \armenian{գող} \\
`work' &ɡoɾt͡s & \armenian{գործ} & kʲœɾt͡s & \armenian{կյէօրծ} & ɡoɾt͡s & \armenian{գործ} \\
 \hline 
 \end{tabular}
\end{table}


Classical Armenian /o/ <\armenian{ո}> became /u/ <\armenian{ու}> for the words in Table \ref{tab:Karabakh:phonology:soundChange:monoph:o:u}.\footnote{\translatorHD{For the word `cress', Adjarian provides an ancestor form <\armenian{կոտեմն}>, but I've had difficulty verifying if this word existed in Classical Armenian. Instead the form I found in dictionaries like Calfa was <\armenian{կոտիմն}>. For the word `dirty', Adjarian provides the word <\armenian{աղտոտ}>. I couldn't determine if this word existed in Classical Armenian; but this word is a compound of Classical roots, so it's possible. }}


\begin{table}[H]
 \centering
 \caption{Change from Classical Armenian /o/ <\armenian{ո}> to /u/ <\armenian{ու}> in the Karabakh dialect}
 \label{tab:Karabakh:phonology:soundChange:monoph:o:u}
 \begin{tabular}{|l| ll|ll| ll|}
 \hline & \multicolumn{2}{l|}{Classical Armenian} &\multicolumn{2}{l|}{> Karabakh} & \multicolumn{2}{l|}{cf. SEA} \\ 
`madder' &toɾ\'on & \armenian{տորոն} & t\'uɾun & \armenian{տո՛ւրուն} & toɾ\'on & \armenian{տորոն} \\
`cress' & kot\'imən & \armenian{կոտիմն} & k\'utemnə & \armenian{կո՛ւտէմնը} & kot\'em & \armenian{կոտեմ} \\
`to steal' & ɡoɬɑn\'ɑl & \armenian{գողանալ} & kʲuʁ\'ɑnɑl & \armenian{կյուղա՛նալ} & ɡoʁɑn\'ɑl & \armenian{գողանալ} \\
`dirty' & ɑɬt\'ot & \armenian{աղտոտ} & j\'eχtut & \armenian{յէ՛խտուտ} & ɑχt\'ot & \armenian{աղտոտ} \\
`grape' & χɑɬ\'oɬ & \armenian{խաղող} & h\'ɑʁuʁ & \armenian{հա՛ղուղ} & χɑʁ\'oʁ & \armenian{խաղող} \\
 \hline 
 \end{tabular}
\end{table}

Classical Armenian /o/ <\armenian{ո}> became /ə̟/ <\armenian{ըէ}> for the words in Table \ref{tab:Karabakh:phonology:soundChange:monoph:o:əFront}, always after the sound /v/ <\armenian{վ}>. 


\begin{table}[H]
 \centering
 \caption{Change from Classical Armenian /o/ <\armenian{ո}> to /ə̟/ <\armenian{ըէ}> in the Karabakh dialect}
 \label{tab:Karabakh:phonology:soundChange:monoph:o:əFront}
 \begin{tabular}{|l| ll|ll| ll|}
 \hline & \multicolumn{2}{l|}{Classical Armenian} &\multicolumn{2}{l|}{> Karabakh} & \multicolumn{2}{l|}{cf. SEA} \\ 
`king' &tʰɑɡɑ{w\'o}ɾ & \armenian{թագաւոր} & tʰkʰ\'ɑvə̟ɾ & \armenian{թքա՛վըէր} & tʰɑkʰɑv\'oɾ & \armenian{թագավոր} \\
`graceful' &ʃənoɾhɑ{w\'o}ɾ & \armenian{շնորհաւոր} & ʃənəh\'ɑvə̟ɾ & \armenian{շընըհա՛վըէր} & ʃənoɾɑv\'oɾ & \armenian{շնորհավոր} \\
`to get accustomed' &sovoɾ\'il & \armenian{սովորիլ} & səv\'ə̟ɾil & \armenian{սըվըէ՛րիլ} & sovoɾ\'el & \armenian{սովորել} \\
`smell' &hot & \armenian{հոտ} & və̟t & \armenian{վըէտ} & hot & \armenian{հոտ} \\
`hole (CA); pit (SEA)' &hoɾ & \armenian{հոր} & və̟ɾ & \armenian{վըէր} & hoɾ & \armenian{հոր} \\ 
`earth' &hoɬ & \armenian{հող} & və̟ʁ & \armenian{վըէղ} & hoʁ & \armenian{հող} \\
 \hline 
 \end{tabular}
\end{table}

Classical Armenian /o/ <\armenian{ո}> became /ə̟/ <\armenian{վըէ}> for the words in Table \ref{tab:Karabakh:phonology:soundChange:monoph:o:vəFront}, at the beginning of monosyllabic and polysyllabic words. 


\begin{table}[H]
 \centering
 \caption{Change from Classical Armenian /o/ <\armenian{ո}> to /və̟/ <\armenian{վըէ}> in the Karabakh dialect}
 \label{tab:Karabakh:phonology:soundChange:monoph:o:vəFront}
 \begin{tabular}{|l| ll|ll| ll|}
 \hline & \multicolumn{2}{l|}{Classical Armenian} &\multicolumn{2}{l|}{> Karabakh} & \multicolumn{2}{l|}{cf. SEA} \\ 
`prey' &oɾs & \armenian{որս} & və̟ɾs & \armenian{վըէրս} & voɾs & \armenian{որս} \\
`foot' &\'otən & \armenian{ոտն} & v\'ə̟nnə & \armenian{վըէ՛ննը} & v\'ot & \armenian{ոտ} \\
`sheep' &ot͡ʃʰχ\'ɑɾ & \armenian{ոչխար} & v\'ə̟χt͡ʃɑɾ & \armenian{վըէ՛խճար} & vot͡ʃʰχ\'ɑɾ & \armenian{ոչխար} \\
`bone' &\'oskəɾ & \armenian{ոսկր} & v\'ə̟skə̟r & \armenian{վըէ՛սկըէռ} & vosk\'oɾ & \armenian{ոսկոր} \\
`buttocks' & or & \armenian{ոռ} & və̟r & \armenian{վըէռ} & vor & \armenian{ոռ} \\
 \hline 
 \end{tabular}
\end{table}

\paragraph{Classical Armenian /u/ <\armenian{ու}>} 

Classical Armenian /u/ <\armenian{ու}> became /v/ <\armenian{վ}> for the words in Table \ref{tab:Karabakh:phonology:soundChange:monoph:u:v}, when next to a vowel. 


\begin{table}[H]
 \centering
 \caption{Change from Classical Armenian /u/ <\armenian{ու}> to /v/ <\armenian{վ}> in the Karabakh dialect}
 \label{tab:Karabakh:phonology:soundChange:monoph:u:v}
 \begin{tabular}{|l| ll|ll| ll|}
 \hline & \multicolumn{2}{l|}{Classical Armenian} &\multicolumn{2}{l|}{> Karabakh} & \multicolumn{2}{l|}{cf. SEA} \\ 
`fox' &ɑɬu.\'es & \armenian{աղուէս} & \'ɑʁves & \armenian{ա՛ղվէս} & ɑʁv\'es & \armenian{աղվես} \\ 
`to appear' &tʰu.\'il& \armenian{թուիլ} & tʰvɑl & \armenian{թվալ} & tʰəv\'el & \armenian{թվել} \\ 
 \hline 
 \end{tabular}
\end{table}


With the subsequent Classical vowel /ɑ/ <\armenian{ա}>, it forms the diphthong /u̯ɑ/ <\armenian{ուա}>... 

\begin{adjarianpage}\label{page:64}\end{adjarianpage}% should be 64

... in the following three words (Table \ref{tab:Karabakh:phonology:soundChange:monoph:u:ua}). In Shushi however, these words follow the general rule and are pronounced, \translatorHD{meaning they're pronounced as in SEA with a /əv/ sequence instead of /u̯ɑ/.\footnote{I couldn't unambiguously track down what the word <\armenian{թթուեալ}> meant, so I couldn't determine its SEA reflex. } }

 
\begin{table}[H]
 \centering
 \caption{Change from Classical Armenian /u/ <\armenian{ու}> to /v/ <\armenian{վ}> in the Karabakh dialect}
 \label{tab:Karabakh:phonology:soundChange:monoph:u:ua}
 \begin{tabular}{|l| ll|ll| ll|}
 \hline & \multicolumn{2}{l|}{Classical Armenian} &\multicolumn{2}{l|}{> Karabakh} & \multicolumn{2}{l|}{cf. SEA} \\ 
 `rope' &t͡ʃʰu̯ɑn & \armenian{չուան} & t͡ʃʰu̯ɑn & \armenian{չուան} & t͡ʃʰəvɑn & \armenian{չվան} \\ 
 & & & t͡ʃʰəvɑn (Shushi) & \armenian{չըվան} && \\ 
 `sourish' &tʰətʰu̯ɑʃ & \armenian{թթուաշ} & tʰtʰu̯ɑʃ & \armenian{թթուաշ} & tʰətʰvɑʃ & \armenian{թթվաշ} \\ 
 & & & tʰtʰvɑʃ (Shushi) & \armenian{թթվաշ} && \\ 
 maybe `to get sour' &tʰətʰue̯ɑl & \armenian{թթուեալ} (?) & tʰətʰu̯ɑl & \armenian{թթուալ} & & \\ 
 & & & tʰtʰvɑl (Shushi) & \armenian{թթվալ} && \\ 
 \hline 
 \end{tabular}
\end{table}


Classical Armenian /u/ <\armenian{ու}> became /o/ <\armenian{օ}> for the words in Table \ref{tab:Karabakh:phonology:soundChange:monoph:u:o}, when next to a vowel. 


\begin{table}[H]
 \centering
 \caption{Change from Classical Armenian /u/ <\armenian{ու}> to /o/ <\armenian{օ}> in the Karabakh dialect}
 \label{tab:Karabakh:phonology:soundChange:monoph:u:o}
 \begin{tabular}{|l| ll|ll| ll|}
 \hline & \multicolumn{2}{l|}{Classical Armenian} &\multicolumn{2}{l|}{> Karabakh} & \multicolumn{2}{l|}{cf. SEA} \\ 
`dog' &ʃun & \armenian{շուն} & ʃon & \armenian{շօն} & ʃun & \armenian{շուն} \\ 
`mulberry' &tʰutʰ & \armenian{թութ} & tʰotʰ & \armenian{թօթ} & tʰutʰ & \armenian{թութ} \\ 
`smoke' &t͡suχ & \armenian{ծուխ} & t͡soχ & \armenian{ծօխ} & t͡suχ & \armenian{ծուխ} \\ 
`sour' &tʰətʰu & \armenian{թթու} & tʰtʰo & \armenian{թթօ} & tʰətʰu & \armenian{թթու} \\ 
`pomegranate' &n\'urən & \armenian{նուռն} & n\'ornə & \armenian{նօ՛ռնը} & nur & \armenian{նուռ} \\ 
`I have' &un\'im & \armenian{ունիմ} & \'onim & \armenian{օ՛նիմ} & un\'em & \armenian{ունեմ} \\ 
`colt' &kʰur\'ɑk & \armenian{քուռակ} & kʰ\'orɑk & \armenian{քօ՛ռակ} & kʰur\'ɑk & \armenian{քուռակ} \\ 
 \hline 
 \end{tabular}
\end{table}

 

Classical Armenian /u/ <\armenian{ու}> became /u/ <\armenian{ու}> for the words in Table \ref{tab:Karabakh:phonology:soundChange:monoph:u:u}. 


\begin{table}[H]
 \centering
 \caption{Change from Classical Armenian /u/ <\armenian{ու}> to /u/ <\armenian{ու}> in the Karabakh dialect}
 \label{tab:Karabakh:phonology:soundChange:monoph:u:u}
 \begin{tabular}{|l| ll|ll| ll|}
 \hline & \multicolumn{2}{l|}{Classical Armenian} &\multicolumn{2}{l|}{> Karabakh} & \multicolumn{2}{l|}{cf. SEA} \\ 
`cat' &kɑt\'u & \armenian{կատու} & k\'ɑtu & \armenian{կա՛տու} & kɑt\'u & \armenian{կատու} \\ 
`hail' &kɑɾk\'ut & \armenian{կարկուտ} & k\'ɑɾkut & \armenian{կա՛րկուտ} & kɑɾk\'ut & \armenian{կարկուտ} \\ 
`two' & eɾk\'u & \armenian{երկու} & \'eɾku & \armenian{է՛րկու} & jeɾk\'u & \armenian{երկու} \\ 
`tear' & ɑɾtɑs\'ukʰ & \armenian{արտասուք} & əst\'ɑsunkʰ & \armenian{ըստա՛սունք} & ɑɾtɑs\'ukʰ & \armenian{արտասուք} \\ 
`name' & ɑn\'un & \armenian{անուն} & \'ɑnum & \armenian{ա՛նում} &ɑn\'un & \armenian{անուն} \\ 
`coal' & ɑt͡s\'uχ & \armenian{ածուխ} & and͡zuʁ & \armenian{անձուղ} & ɑt͡s\'uχ & \armenian{ածուխ} \\ 
 \hline 
 \end{tabular}
\end{table}

Classical Armenian /u/ <\armenian{ու}> became /ʏ/ <\armenian{իւ}> for the words in Table \ref{tab:Karabakh:phonology:soundChange:monoph:u:ʏ}. 


\begin{table}[H]
 \centering
 \caption{Change from Classical Armenian /u/ <\armenian{ու}> to /ʏ/ <\armenian{իւ}> in the Karabakh dialect}
 \label{tab:Karabakh:phonology:soundChange:monoph:u:ʏ}
 \begin{tabular}{|l| ll|ll| ll|}
 \hline & \multicolumn{2}{l|}{Classical Armenian} &\multicolumn{2}{l|}{> Karabakh} & \multicolumn{2}{l|}{cf. SEA} \\ 
`fish' &d͡z\'ukən & \armenian{ձուկն} & t͡s\'ʏknə & \armenian{ծի՛ւկնը} & d͡z\'uk & \armenian{ձուկ} \\ 
`egg' &d͡zu & \armenian{ձու} & d͡zʏ & \armenian{ձիւ} & d͡zu & \armenian{ձու} \\ 
`water' &d͡ʒuɾ & \armenian{ջուր} & t͡ʃʏɾ & \armenian{ճիւր} & d͡ʒuɾ & \armenian{ջուր} \\ 
`flea' &lu & \armenian{լու} & lʏ & \armenian{լիւ} & lu & \armenian{լու} \\ 
`oath' &eɾd\'umən & \armenian{երդումն} & \'ʏɾtʰʏmnə & \armenian{ի՛ւրթիւմնը} & jeɾtʰ\'um & \armenian{երդում} \\ 
 \hline 
 \end{tabular}
\end{table}


Classical Armenian /u/ <\armenian{ու}> became /œ/ <\armenian{էօ}> for the words in Table \ref{tab:Karabakh:phonology:soundChange:monoph:u:œ}. 


\begin{table}[H]
 \centering
 \caption{Change from Classical Armenian /u/ <\armenian{ու}> to /œ/ <\armenian{էօ}> in the Karabakh dialect}
 \label{tab:Karabakh:phonology:soundChange:monoph:u:œ}
 \begin{tabular}{|l| ll|ll| ll|}
 \hline & \multicolumn{2}{l|}{Classical Armenian} &\multicolumn{2}{l|}{> Karabakh} & \multicolumn{2}{l|}{cf. SEA} \\ 
`fawn' & ul & \armenian{ուլ} & hœl & \armenian{հէօլ} & ul & \armenian{ուլ} \\ 
`Friday' & uɾb\'ɑtʰ & \armenian{ուրբաթ} & \'œɾpʰætʰ & \armenian{է՛օրփա̈թ} & uɾpʰ\'ɑtʰ & \armenian{ուրբաթ} \\ 
`head' & ɡəl\'uχ& \armenian{գլուխ} & kʲəlœχ & \armenian{կյըլէօխ} & ɡəl\'uχ & \armenian{գլուխ}\\ 
\hline 
 \end{tabular}
\end{table}


\subsubsection{Diphthong changes}


\paragraph{Classical Armenian /ɑi̯/ <\armenian{այ}>} 

Classical Armenian /ɑi̯/ <\armenian{այ}> became /e/ <\armenian{է}> for the words in Table \ref{tab:Karabakh:phonology:soundChange:diphthong:ɑi:e}. 


\begin{table}[H]
 \centering
 \caption{Change from Classical Armenian /ɑi̯/ <\armenian{այ}> to /e/ <\armenian{է}> in the Karabakh dialect}
 \label{tab:Karabakh:phonology:soundChange:diphthong:ɑi:e}
 \begin{tabular}{|l| ll|ll| ll|}
 \hline & \multicolumn{2}{l|}{Classical Armenian} &\multicolumn{2}{l|}{> Karabakh} & \multicolumn{2}{l|}{cf. SEA} \\ 
`wide' & lɑi̯n & \armenian{լայն} & len & \armenian{լէն} & lɑjn & \armenian{լայն} \\ 
`goat' & ɑi̯t͡s & \armenian{այծ} & et͡s & \armenian{էծ} & ɑjt͡s & \armenian{այծ} \\ 
`father' & hɑi̯ɾ & \armenian{հայր} & heɾ & \armenian{հէր} & hɑjɾ & \armenian{հայր} \\ 
`brother' & eɬb\'ɑi̯ɾ & \armenian{եղբայր} & \'ɑχpeɾ & \armenian{ա՛խպէր} & jeχp\'ɑjɾ & \armenian{եղբայր} \\ 
 \hline 
 \end{tabular}
\end{table}

Classical Armenian /ɑi̯/ <\armenian{այ}> became /ɑ/ <\armenian{ա}> for the words in Table \ref{tab:Karabakh:phonology:soundChange:diphthong:ɑi:ɑ}, at the end of the word. 


\begin{table}[H]
 \centering
 \caption{Change from Classical Armenian /ɑi̯/ <\armenian{այ}> to /ɑ/ <\armenian{ա}> in the Karabakh dialect}
 \label{tab:Karabakh:phonology:soundChange:diphthong:ɑi:ɑ}
 \begin{tabular}{|l| ll|ll| ll|}
 \hline & \multicolumn{2}{l|}{Classical Armenian} &\multicolumn{2}{l|}{> Karabakh} & \multicolumn{2}{l|}{cf. SEA} \\ 
`broad bean' & bɑkl\'ɑi̯ & \armenian{բակլայ} & p\'eklɑ & \armenian{պէ՛կլա} & bɑkl\'ɑ & \armenian{բակլա} \\ 
`on' & i veɾ\'ɑi̯ & \armenian{ի վերայ} & jəɾ\'ɑ & \armenian{յըրա՛} & vəɾ\'ɑ & \armenian{վրա} \\ 
`(male?) child' & təɬ\'ɑi̯ & \armenian{տղայ} & təʁɑ & \armenian{տղա} & təʁ\'ɑ & \armenian{տղա} \\ 
`Satan' & sɑtɑn\'ɑi̯ & \armenian{սատանայ} & sut\'ɑnɑ & \armenian{սուտա՛նա} & sɑtɑn\'ɑ & \armenian{սատանա} \\ 
 \hline 
 \end{tabular}
\end{table}


\paragraph{Classical Armenian /ɑu̯/ <\armenian{աւ}>} 

Classical Armenian /ɑu̯/ <\armenian{աւ}> became /ɑv/ <\armenian{ավ}> when next to a vowel and word-final, as in Table \ref{tab:Karabakh:phonology:soundChange:diphthong:ɑu:ɑv}.



\begin{table}[H]
 \centering
 \caption{Change from Classical Armenian /ɑu̯/ <\armenian{աւ}> to /ɑv/ <\armenian{ավ}> in the Karabakh dialect}
 \label{tab:Karabakh:phonology:soundChange:diphthong:ɑu:ɑv}
 \begin{tabular}{|l| ll|ll| ll|}
 \hline & \multicolumn{2}{l|}{Classical Armenian} &\multicolumn{2}{l|}{> Karabakh} & \multicolumn{2}{l|}{cf. SEA} \\ 
`bird (CA); chicken (SEA)' & hɑu̯ & \armenian{հաւ} & hɑv & \armenian{հավ} & hɑv & \armenian{հավ} \\ 
`to like' & hɑ{wɑ}n\'il & \armenian{հաւանիլ} & h\'ɑvɑn kenɑl & \armenian{հա՛վան կէնալ} & hɑvɑn\'el & \armenian{հավանել} \\ 
 \hline 
 \end{tabular}
\end{table}


Classical Armenian /ɑu̯/ <\armenian{աւ}> became /o/ <\armenian{օ}> when next to a consonant as in Table \ref{tab:Karabakh:phonology:soundChange:diphthong:ɑu:o}. 



\begin{table}[H]
 \centering
 \caption{Change from Classical Armenian /ɑu̯/ <\armenian{աւ}> to /o/ <\armenian{օ}> in the Karabakh dialect}
 \label{tab:Karabakh:phonology:soundChange:diphthong:ɑu:o}
 \begin{tabular}{|l| ll|ll| ll|}
 \hline & \multicolumn{2}{l|}{Classical Armenian} &\multicolumn{2}{l|}{> Karabakh} & \multicolumn{2}{l|}{cf. SEA} \\ 
`pigeon' & ɑɬɑu̯n\'i & \armenian{աղաւնի} & jəʁ\'onei̯ɡʲ & \armenian{յըղօ՛նէյգյ} & ɑʁɑvn\'i & \armenian{աղավնի} \\ 
`naphtha' & nɑu̯tʰ & \armenian{նաւթ} & notʰ & \armenian{նօթ} & nɑftʰ & \armenian{նավթ} \\ 
 \hline 
 \end{tabular}
\end{table}


\paragraph{Classical Armenian /e̯ɑ, e̯ɑi̯/ <\armenian{եա, եայ}>} 

Classical Armenian /e̯ɑ, e̯ɑi̯/ <\armenian{եա, եայ}> became /e/ <\armenian{է}> (Table \ref{tab:Karabakh:phonology:soundChange:diphthong:eɑi:e}). 


\begin{table}[H]
 \centering
 \caption{Change from Classical Armenian /e̯ɑ, e̯ɑi̯/ <\armenian{եա, եայ}> to /e/ <\armenian{է}> in the Karabakh dialect}
 \label{tab:Karabakh:phonology:soundChange:diphthong:eɑi:e}
 \begin{tabular}{|l| ll|ll| ll|}
 \hline & \multicolumn{2}{l|}{Classical Armenian} &\multicolumn{2}{l|}{> Karabakh} & \multicolumn{2}{l|}{cf. SEA} \\ 
`wheat' & t͡sʰoɾe̯\'ɑn& \armenian{ցորեան} & t͡sʰ\'oɾen &\armenian{ցօ՛րէն} & t͡sʰoɾ\'en & \armenian{ցորեն} \\ 
`threshold' & se̯\'ɑmkʰ& \armenian{սեամք} & ʃemkʰ & \armenian{շէմք} & ʃemkʰ & \armenian{շեմք} \\ 
`tortoise' & kəɾe̯\'ɑi̯& \armenian{կրեայ} & k\'oɾe, k\'oɾɑ & \armenian{կօ՛րէ, կօ՛րա}) & kəɾj\'ɑ & \armenian{կրիա} \\ 
 \hline 
 \end{tabular}
\end{table}

\paragraph{Classical Armenian /eu̯/ <\armenian{եւ}>} 

Classical Armenian /eu̯/ <\armenian{եւ}> became /ev/ <\armenian{էվ}> (Table \ref{tab:Karabakh:phonology:soundChange:diphthong:eu:ev}). 


\begin{table}[H]
 \centering
 \caption{Change from Classical Armenian /eu̯/ <\armenian{եւ}> to /ev/ <\armenian{էվ}> in the Karabakh dialect}
 \label{tab:Karabakh:phonology:soundChange:diphthong:eu:ev}
 \begin{tabular}{|l| ll|ll| ll|}
 \hline & \multicolumn{2}{l|}{Classical Armenian} &\multicolumn{2}{l|}{> Karabakh} & \multicolumn{2}{l|}{cf. SEA} \\ 
`light (weight)' & tʰetʰ\'eu̯& \armenian{թեթեւ} & tʰ\'itʰev &\armenian{թի՛թէվ} & tʰetʰ\'ev & \armenian{թեթև} \\ 
`sun' & ɑɾ\'eu̯& \armenian{արեւ} & \'ɑɾev & \armenian{ա՛րէվ} & ɑɾ\'ev & \armenian{արև} \\ 
`gray-haired' & ɑle{w\'o}ɾ & \armenian{ալեւոր} & hl\'evuɾ &\armenian{հլէ՛վուր} & ɑlev\'oɾ & \armenian{ալևոր} \\ 
 \hline 

 \end{tabular}
\end{table}

\paragraph{Classical Armenian /iu̯̯/ <\armenian{իւ}>} 

Classical Armenian /iu̯/ <\armenian{իւ}> became /ʏ/ <\armenian{իւ}> for the words in Table \ref{tab:Karabakh:phonology:soundChange:diphthong:iu:y}. 


\begin{table}[H]
 \centering
 \caption{Change from Classical Armenian /iu̯/ <\armenian{իւ}> to /ʏ/ <\armenian{իւ}> in the Karabakh dialect}
 \label{tab:Karabakh:phonology:soundChange:diphthong:iu:y}
 \begin{tabular}{|l| ll|ll| ll|}
 \hline & \multicolumn{2}{l|}{Classical Armenian} &\multicolumn{2}{l|}{> Karabakh} & \multicolumn{2}{l|}{cf. SEA} \\ 
`snow' & d͡ziu̯n& \armenian{ձիւն} & t͡sʏn & \armenian{ծիւն} & d͡zjun & \armenian{ձյուն} \\ 
`column' & siu̯n& \armenian{սիւն} & sʏn & \armenian{սիւն} & d͡zjun & \armenian{սյուն} \\ 
`hundred' & hɑɾiu̯ɾ & \armenian{հարիւր} & hɑɾʏɾ & \armenian{հա̈րիւր} & hɑɾjuɾ & \armenian{հարյուր} \\ 
 \hline 
 \end{tabular}
\end{table}


Classical Armenian /iu̯/ <\armenian{իւ}> became /iv/ <\armenian{իվ}> for the words in Table \ref{tab:Karabakh:phonology:soundChange:diphthong:iu:iv}, when word-final and next to a vowel. 


\begin{table}[H]
 \centering
 \caption{Change from Classical Armenian /iu̯/ <\armenian{իւ}> to /iv/ <\armenian{իվ}> in the Karabakh dialect}
 \label{tab:Karabakh:phonology:soundChange:diphthong:iu:iv}
 \begin{tabular}{|l| ll|ll| ll|}
 \hline & \multicolumn{2}{l|}{Classical Armenian} &\multicolumn{2}{l|}{> Karabakh} & \multicolumn{2}{l|}{cf. SEA} \\ 
`honor' & pɑt\'iu̯& \armenian{պատիւ} & pɑt\'iv & \armenian{պա՛տիվ} & pɑt\'iv & \armenian{պատիվ} \\ 
`honor' & ɑɾt͡s\'iu̯ & \armenian{արծիւ} & \'ɑrt͡siv & \armenian{ա՛ռծիվ} & ɑɾt͡s\'iv & \armenian{արծիվ} \\ 
`sick' & hi{w\'ɑ}nd & \armenian{հիւանդ} & h\'ivɑnd & \armenian{հի՛վանդ} & hiv\'ɑnd & \armenian{հիվանդ} \\ 
 \hline 
 \end{tabular}
\end{table}


Classical Armenian /iu̯/ <\armenian{իւ}> became /ev/ <\armenian{էվ}> for the words in Table \ref{tab:Karabakh:phonology:soundChange:diphthong:iu:iv}, when word-final. 


\begin{table}[H]
 \centering
 \caption{Change from Classical Armenian /iu̯/ <\armenian{իւ}> to /ev/ <\armenian{էվ}> in the Karabakh dialect}
 \label{tab:Karabakh:phonology:soundChange:diphthong:iu:ev}
 \begin{tabular}{|l| ll|ll| ll|}
 \hline & \multicolumn{2}{l|}{Classical Armenian} &\multicolumn{2}{l|}{> Karabakh} & \multicolumn{2}{l|}{cf. SEA} \\ 
`fight' & kəriu̯& \armenian{կռիւ} & krev & \armenian{կռէվ} & kəriv & \armenian{կռիվ} \\ 
`scattered' & t͡sʰəɾiu̯& \armenian{ցրիւ} & t͡sʰɾev & \armenian{ցրէվ} & t͡sʰəɾiv & \armenian{ցրիվ} \\ 
 \hline 
 \end{tabular}
\end{table}


\paragraph{Classical Armenian /oi̯/ <\armenian{ոյ}>} 

Classical Armenian /oi̯/ <\armenian{ոյ}> became /ʏ/ <\armenian{իւ}> (Table \ref{tab:Karabakh:phonology:soundChange:diphthong:oi:y}). 


\begin{table}[H]
 \centering
 \caption{Change from Classical Armenian /oi̯̯/ <\armenian{ոյ}> to /ʏ/ <\armenian{իւ}> in the Karabakh dialect}
 \label{tab:Karabakh:phonology:soundChange:diphthong:oi:y}
 \begin{tabular}{|l| ll|ll| ll|}
 \hline & \multicolumn{2}{l|}{Classical Armenian} &\multicolumn{2}{l|}{> Karabakh} & \multicolumn{2}{l|}{cf. SEA} \\ 
`nest' & boi̯n & \armenian{բոյն} &pʏn & \armenian{պիւն} & bujn & \armenian{բույն} \\ 
`evening' & eɾek\'oi̯ & \armenian{երեկոյ} &ɾ\'ʏɡʏ & \armenian{րի՛ւգիւ} & jeɾek\'o & \armenian{երեկո} \\ 
`blue' & kɑp\'oi̯t & \armenian{կապոյտ} &kʲ\'æpʏt & \armenian{կյա̈՛պիւտ} & kɑp\'ujt & \armenian{կապույտ} \\ 
 \hline 
 \end{tabular}
\end{table}

\paragraph{Classical Armenian /ov/ <\armenian{ով}>} 

Classical Armenian /ov/ <\armenian{ով}> became /ɑv/ <\armenian{ավ}> (Table \ref{tab:Karabakh:phonology:soundChange:diphthong:ov:ɑv}).\footnote{\translatorHD{I find it odd that Adjarian calls this sequence a diphthong because <\armenian{վ}> most likely was a /v/ sound. This suggests that Adjarian may have actually thought that <\armenian{ով}> was pronounced as /ou̯/ instead of /ov/. }}


\begin{table}[H]
 \centering
 \caption{Change from Classical Armenian /ov/ <\armenian{ով}> to /ɑv/ <\armenian{ավ}> in the Karabakh dialect}
 \label{tab:Karabakh:phonology:soundChange:diphthong:ov:ɑv}
 \begin{tabular}{|l| ll|ll| ll|}
 \hline & \multicolumn{2}{l|}{Classical Armenian} &\multicolumn{2}{l|}{> Karabakh} & \multicolumn{2}{l|}{cf. SEA} \\ 
`to roast' & χoɾov\'el & \armenian{խորովել} & χɾ\'ɑvel & \armenian{խրա՛վէլ} & χoɾov\'el & \armenian{խորովել} \\ 
`cow' & kov & \armenian{կով} & kɑv &\armenian{կավ} & kov & \armenian{կով} \\ 
`salt-{\ins}' & & & \'ɑʁɑv & \armenian{ա՛ղավ} & ɑʁ-\'ov & \armenian{աղով} \\ 
`with-{\ins}' & kov & \armenian{կով} & pʰ\'ə̟dɑv &\armenian{փըէ՛դավ} & pʰɑjt-\'ov & \armenian{փայտով} \\ 
 \hline 
 \end{tabular}
\end{table}

\subsubsection{Consonant changes}

\paragraph{Voicing changes}

The consonants in the Karabakh dialect have undergone general circle-like sound changes (\armenian{ձայնաշրջութիւն}). 

The voiced consonants of Old Armenian become voiceless. They are unchanged only when next to the nasals /m,n/ <\armenian{մ, ն}>... 

\begin{adjarianpage}\label{page:65}\end{adjarianpage}% should be 65


... The voiceless consonants of Old Armenian stay unchanged, but they are voiced after the nasals. The Classical voiced sounds\footnote{\translatorHD{On page 306, Adjarian provides an erratum about a missing word; I fixed it.}} are voiceless aspirated after the sound /ɾ/ <\armenian{ր}>. Examples are in Table \ref{tab:Karabakh:phonology:soundChange:cons:voicing}. 


\begin{table}[H]
 \centering
 \caption{Changes in laryngeal voicing from Classical Armenian to the Karabakh dialect}
 \label{tab:Karabakh:phonology:soundChange:cons:voicing}
 \begin{tabular}{|l| ll|ll| ll|}
 \hline & \multicolumn{2}{l|}{Classical Armenian} &\multicolumn{2}{l|}{> Karabakh} & \multicolumn{2}{l|}{cf. SEA} \\ 
`mouth' & beɾ\'ɑn & \armenian{բերան} & p\'eɾɑn & \armenian{պէ՛րան} & beɾ\'ɑn & \armenian{բերան} \\ 
`thing' & bɑn & \armenian{բան} & pen & \armenian{պէն} & bɑn & \armenian{բան} \\ 
`door' & d\'urən & \armenian{դուռն} & t\'œrnə & \armenian{տէօ՛ռնը} & dur & \armenian{դուռ} \\ 
`sound' & d͡zɑi̯n & \armenian{ձայն} & t͡sen & \armenian{ծէն} & d͡zɑjn & \armenian{ձայն} \\ 
`water-mill' & d͡ʒəɾɑɬ\'ɑt͡sʰ & \armenian{ջրաղաց} & t͡ʃ\'eʁɑt͡sʰ & \armenian{ճէ՛ղաց} & d͡ʒəɾɑʁ\'ɑt͡sʰ & \armenian{ջրաղաց} \\ 
`cotton' & bɑmb\'ɑk & \armenian{բամբակ} & p\'embɑk & \armenian{պէ՛մբակ} & bɑmb\'ɑk & \armenian{բամբակ} \\ 
`orphan' & oɾb & \armenian{որբ} & v\'ə̟ɾpʰ & \armenian{վըէրփ} & voɾpʰ & \armenian{որբ} \\ 
`cloud' & ɑmp & \armenian{ամպ} & ɑmb & \armenian{ամբ} & ɑmp & \armenian{ամպ} \\ 
`wool' & buɾd & \armenian{բուրդ} & pʏɾtʰ & \armenian{պիւրթ} & buɾtʰ & \armenian{բուրդ} \\ 
`fever' & tend & \armenian{տենդ} & tə̟nd & \armenian{տըէնդ} & tend & \armenian{տենդ} \\ 
`to slander' & bɑmbɑs\'el & \armenian{բամբասել} & pəmb\'ɑsel & \armenian{պըմբա՛սէլ} & bɑmbɑs\'el & \armenian{բամբասել} \\ 
\armenian{՝}free, ownerless' & ɑnt\'ēɾ & \armenian{անտէր} & \'ɑndɑɾ &\armenian{ա՛նդար} & ɑnt\'eɾ & \armenian{անտեր} \\ 
\armenian{՝}lord' & tēɾ & \armenian{տէր} &tɑɾ & \armenian{տար} & teɾ & \armenian{տեր} \\ 
 \hline 
 \end{tabular}
\end{table}


\paragraph{Palatalization} 

The dorsal sounds from Classical /ɡ k kʰ/ <\armenian{գ կ ք}> preserve their simple pronunciation in various places, but they also soften in some places, \translatorHD{meaning they palatalize}. In accordance with the above rules, they turn into /ɡʲ kʲ kʰʲ/ <\armenian{գյ կյ քյ}>. 

It is notable that while the Classical sound /ɡ/ <\armenian{գ}> sound becomes /kʲ/ <\armenian{կյ}> word-initially, the Classical sounds /k, kʰ/ <\armenian{կ,ք}> do not palatalize in this context. The Classical sound /k/ <\armenian{կ}> becomes /ɡʲ/ <\armenian{գյ}> word-finally after /i/ <\armenian{ի}>, while Classical /kʰ/ <\armenian{ք}> becomes /kʰʲ/ <\armenian{քյ}> word-finally after /i, in, en/ <\armenian{ի, ին, էն}>. Similarly, the Classical sequence /nkn/ <\armenian{նկն}> becomes /nɡnə, nɡʲnə, jnə, ɡʲnə/ <\armenian{նգնը, նգյնը, յնը, գյնը}>. 

 

Examples are in Table \ref{tab:Karabakh:phonology:soundChange:cons:palatal}. 


\begin{table}[H]
 \centering
 \caption{Palatalization from Classical Armenian to the Karabakh dialect}
 \label{tab:Karabakh:phonology:soundChange:cons:palatal}
 \begin{tabular}{|l| ll|ll| ll|}
 \hline & \multicolumn{2}{l|}{Classical Armenian} &\multicolumn{2}{l|}{> Karabakh} & \multicolumn{2}{l|}{cf. SEA} \\ 
`lamb' & ɡ\'ɑrən & \armenian{գառն} & kʲ\'ɑrnə & \armenian{կյա՛ռնը} & ɡ\'ɑr & \armenian{գառ} \\ 
`wolf' & ɡɑi̯l & \armenian{գայլ} & kʲʏl & \armenian{կյիւլ} & ɡɑjl & \armenian{գայլ} \\ 
`wine' & ɡin\'i & \armenian{գինի} & kʲ\'ini & \armenian{կյի՛նի} & ɡin\'i & \armenian{գինի} \\ 
`cane' & ɡɑ{wɑ}z\'ɑn & \armenian{գաւազան} & kʲəv\'ɑzɑn & \armenian{կյըվա՛զան} & ɡɑvɑz\'ɑn & \armenian{գավազան} \\ 
`five' & hinɡ & \armenian{հինգ} & henɡʲ& \armenian{հէնգյ} & hinɡ & \armenian{հինգ} \\ 
`jug' & kuʒ & \armenian{կուժ} & koʒ& \armenian{կօժ} & kuʒ & \armenian{կուժ} \\ 
`kernel' & koɾ\'iz & \armenian{կորիզ} & k\'oɾə̟z & \armenian{կօ՛րըէզ} & koɾ\'iz & \armenian{կորիզ} \\ 
`flower' & t͡sɑɬ\'ik & \armenian{ծաղիկ} & t͡s\'ɑʁeɡʲ & \armenian{ծա՛ղէգյ} & t͡sɑʁ\'ik & \armenian{ծաղիկ} \\ 
`woman' & & & knə̟ɡʲ & \armenian{կնըէգյ} & kənik & \armenian{կնիկ} \\ 
`goatskin' & tik & \armenian{տիկ} & tei̯ɡʲ & \armenian{տէյգյ} & tik & \armenian{տիկ} \\ 
`how many' & kʰɑn\'i & \armenian{քանի} & kʰ\'ɑnə̟ & \armenian{քա՛նըէ} & kʰɑn\'i & \armenian{քանի} \\ 
`partridge' & kɑkʰ\'ɑu̯ & \armenian{կաքաւ} & k\'ɑkʰɑv &\armenian{կա՛քավ} & kɑkʰ\'ɑv& \armenian{կաքավ} \\ 
`manure' & tʰəɾ\'ikʰ & \armenian{թրիք} & tʰɾekʰʲ & \armenian{թրէքյ} & tʰəɾ\'ikʰ& \armenian{թրիք} \\ 
`wedding' & hɑɾsɑn\'ikʰ & \armenian{հարսանիք} & hɾs\'ɑnei̯nkʰʲ & \armenian{հրսա՛նէյնքյ} & hɑɾsɑn\'ikʰ& \armenian{հարսանիք} \\ 
`he' & \'inkʰən & \armenian{ինքն} & \'inkʰʲə & \armenian{ի՛նքյը} & \'iŋkʰən& \armenian{ինքն} \\ 
`mushroom' & s\'unkʰən & \armenian{սունկն} & sojnə & \armenian{սօյնը} & suŋk & \armenian{սունկ} \\ 
 & & & sonɡʲnə & \armenian{սօնգյնը} & & \\ 
 & & & sonɡnə & \armenian{սօնգնը} & & \\ 
`ear' & \'unkʰən & \armenian{ունկն} & \'ojnə & \armenian{օ՛յնը} & uŋk & \armenian{ունկ} \\ 
 & & & \'onɡnə & \armenian{օ՛նգնը} & & \\ 
`knee' & t͡s\'unɡəkʰ & \armenian{ծունգք} & t͡s\'ojnə & \armenian{ծօ՛յնը} & t͡suŋk & \armenian{ծունկ} \\ 
 & & & t͡s\'onɡʲnə & \armenian{ծօ՛նգյնը} & & \\ 
 & & & t͡s\'onɡnə & \armenian{ծօ՛նգնը} & & \\ 
\hline 
 \end{tabular}
\end{table}


\paragraph{Change of word-initial /h/ <\armenian{հ}> to /v/ <\armenian{վ}> } 


Classical Armenian /h/ <\armenian{հ}> becomes /v/ <\armenian{վ}> when word-initial before Classical /o/ <\armenian{ո}> and only in closed syllables (Table \ref{tab:Karabakh:phonology:soundChange:cons:hv}. 


\begin{table}[H]
 \centering
 \caption{Change from Classical Armenian /h/ <\armenian{հ}> to /v/ <\armenian{վ}> in the Karabakh dialect}
 \label{tab:Karabakh:phonology:soundChange:cons:hv}
 \begin{tabular}{|l| ll|ll| ll|}
 \hline & \multicolumn{2}{l|}{Classical Armenian} &\multicolumn{2}{l|}{> Karabakh} & \multicolumn{2}{l|}{cf. SEA} \\ 
`earth' & hoɬ & \armenian{հող} & və̟ʁ & \armenian{վըէղ} & hoʁ & \armenian{հող} \\ 
`smell' & hot & \armenian{հոտ} & və̟t & \armenian{վըէտ} & hot & \armenian{հոտ} \\ 
`hole (CA); pit (SEA)' &hoɾ & \armenian{հոր} & və̟ɾ & \armenian{վըէր} & hoɾ & \armenian{հոր} \\ 
 `soul' & hoɡ\'i & \armenian{հոգի} & h\'ʏkʰi & \armenian{հի՛ւքի} & hokʰ\'i & \armenian{հոգի} \\ 
\hline 
 \end{tabular}
\end{table}

\paragraph{Word-initial insertion of /h/ <\armenian{հ}> } 

At the beginning of many words, the sound /h/ <\armenian{հ}> is added in Karabakh, whereas it is absent in Old Armenian (Table \ref{tab:Karabakh:phonology:soundChange:cons:hinsert}.\footnote{\translatorHD{The original page had the reflex of `shame' <\armenian{ամաւթ, ամօթ}> be <\armenian{հա՛մօթ}>. But on page 306, Adjarian provides an erratum that this reflex is mistranscribed; I fixed it.}} 


\begin{table}[H]
 \centering
 \caption{Insertion of word-initial /h/ <\armenian{հ}> in the Karabakh dialect}
 \label{tab:Karabakh:phonology:soundChange:cons:hinsert}
 \begin{tabular}{|ll| ll|ll| ll|}
 \hline & & \multicolumn{2}{l|}{Classical Armenian} &\multicolumn{2}{l|}{> Karabakh} & \multicolumn{2}{l|}{cf. SEA} \\ 
a. `who' & ov & \armenian{ով} & huv & \armenian{հուվ} & ov & \armenian{ով} \\ 
&`who.{\gen}.{\sg}' & oi̯ɾ, oɾoi̯ & \armenian{ոյր, որոյ} & hʏɾ & \armenian{հիւր} & voɾi &\armenian{որի} \\ 
&`where' & uɾ & \armenian{ուր} & hoɾ & \armenian{հօր} & uɾ & \armenian{ուր} \\ 
&`how' & & & hunt͡sʰ & \armenian{հունց} & vont͡sʰ & \armenian{ոնց} \\ 
&`what' & int͡ʃʰ & \armenian{ինչ} & hint͡ʃʰ & \armenian{հինչ} & int͡ʃʰ & \armenian{ինչ} \\ 
b. & `friend' & ənkeɾ & \armenian{ընկեր} & hənɡeɾ & \armenian{հընգէր} & əŋkeɾ & \armenian{ընկեր} \\ 
&`shame' & ɑm\'ɑu̯tʰ & \armenian{ամաւթ} & h\'ɑmutʰ & \armenian{հա՛մութ} & ɑm\'otʰ & \armenian{ամոթ} \\ 
&`tail' & ɑɡ\'i & \armenian{ագի} & h\'ækʰʏ & \armenian{հա̈՛քիւ} &ɑɡ\'i& \armenian{ագի} \\ 
&`gray-haired' & ɑle{w\'o}ɾ & \armenian{ալեւոր} & hl\'evuɾ &\armenian{հլէ՛վուր} & ɑlev\'oɾ & \armenian{ալևոր} \\ 
&`idle' & pɑɾ\'ɑp & \armenian{պարապ} & həp\'ɑɾɑp & \armenian{հըպա՛րապ} & pɑɾ\'ɑp & \armenian{պարապ} \\ 
\hline 
 \end{tabular}
\end{table}

These are especially interesting because they show the oldest form of Armenian. In these examples, the words in Table \ref{tab:Karabakh:phonology:soundChange:cons:hinsert}a\footnote{\translatorHD{On page 306, Adjarian provides states the CA word <\armenian{ուր}> `where' should have been <\armenian{ու}> `who', but I think his correction is incorrect.}} previously had an initial /kʷ/ sound, which was later lost.\footnote{\translatorHD{I think he means that Proto-Armenian or Proto-Indo-European had this reconstructed sound */kʷ/. The Wiktionary page for a Classical word <\armenian{հի}> /hi/ `something' likewise provides an etymology from PIE */kʷ/, based on \citet[92]{Adjarian-1979-Etymology}. \url{https://en.wiktionary.org/wiki/\armenian{հի}-} } }The Karabakh sound /h/ <\armenian{հ}> is a continuation of this. 


\paragraph{Voicing assimilation between dorsal fricatives and stops} 

The Classical sounds /χ, ɬ/ <\armenian{խ, ղ}> merge with the following plosive to form a /χ/ <\armenian{խ}> + voiceless sequence (Table \ref{tab:Karabakh:phonology:soundChange:cons:velarassimilation}. 


\begin{table}[H]
 \centering
 \caption{Voicing assimilation between dorsal fricatives and stops in in the Karabakh dialect}
 \label{tab:Karabakh:phonology:soundChange:cons:velarassimilation}
 \begin{tabular}{|l| ll|ll| ll|}
 \hline & \multicolumn{2}{l|}{Classical Armenian} &\multicolumn{2}{l|}{> Karabakh} & \multicolumn{2}{l|}{cf. SEA} \\ 
 `fountain' & ɑɬb\'iu̯ɾ & \armenian{աղբիւր} & \'ɑχpʏɾ & \armenian{ա՛խպիւր} & ɑχpj\'uɾ & \armenian{աղբյուր} \\ 
 `horse-radish' & boɬk & \armenian{բողկ} & peχk & \armenian{պէխկ} & boχk & \armenian{բողկ} \\ 
 `sin' & meɬkʰ & \armenian{մեղք} & meχk & \armenian{մէխկ} & meχkʰ & \armenian{մեղք} \\ 
 `to strangle' & χeɬd\'el & \armenian{խեղդել} & χ\'eχtel & \armenian{խէ՛խտէլ} & χeχt\'el & \armenian{խեղդել} \\ 
 `filth' & ɑɬt & \armenian{աղտ} & jeχt &\armenian{յէխտ} & ɑχt & \armenian{աղտ} \\ 
 `paper' & tʰuɬtʰ & \armenian{թուղթ} & tʰoχt &\armenian{թօխտ} & tʰuχtʰ & \armenian{թուղթ} \\ 
`church' &ekeɬet͡sʰ\'i & \armenian{եկեղեցի} & j\'eχt͡se & \armenian{յէ՛խծէ} &jekeʁet͡sʰ\'i & \armenian{եկեղեցի} \\
`girl' &ɑɬd͡ʒ\'ik & \armenian{աղջիկ} & \'ɑχt͡ʃiɡʲ & \armenian{ա՛խճիգյ} &ɑχt͡ʃʰ\'ik & \armenian{աղջիկ} \\
`to flee' &pʰɑχt͡ʃʰ\'il & \armenian{փախչիլ} & pʰ\'ɑχt͡ʃil & \armenian{փա՛խճիլ} & pʰɑχt͡ʃʰ\'el & \armenian{փախչել} \\
`sheep' &ot͡ʃʰχ\'ɑɾ & \armenian{ոչխար} & v\'ə̟χt͡ʃɑɾ & \armenian{վըէ՛խճար} & vot͡ʃʰχ\'ɑɾ & \armenian{ոչխար} \\
\hline 
 \end{tabular}
\end{table}

\begin{adjarianpage}\label{page:66}\end{adjarianpage}% should be 66


\paragraph{Change from word-final Classical /ən/ to /nə/} 

The ending /n/ <\armenian{ն}> in Old Armenian was found in many words in Old Armenian (Table \ref{tab:Karabakh:phonology:soundChange:cons:nschwa}). \translatorHD{Note that orthographically, this was written as final <\armenian{ն}> /n/, but a schwa is epenthesized after consonants to create /ən/.} This ending is lost in almost all our dialects (\translatorHD{such as in SEA}). This form become /nə/ <\armenian{նը}> in Karabakh, creating a unique characteristic for this dialect.
 


\begin{table}[H]
 \centering
 \caption{Change from word-final Classical Armenian /(ə)n/ <\armenian{ն}> to /nə/ <\armenian{նը}> in the Karabakh dialect}
 \label{tab:Karabakh:phonology:soundChange:cons:nschwa}
 \begin{tabular}{|l| ll|ll| ll|}
 \hline & \multicolumn{2}{l|}{Classical Armenian} &\multicolumn{2}{l|}{> Karabakh} & \multicolumn{2}{l|}{cf. SEA} \\ 
`door' & d\'urən & \armenian{դուռն} & t\'œrnə & \armenian{տէ՛օռնը} & dur & \armenian{դուռ} \\ 
`fish' &d͡z\'ukən & \armenian{ձուկն} & t͡s\'ʏknə & \armenian{ծի՛ւկնը} & d͡z\'uk & \armenian{ձուկ} \\ 
`mouse' &m\'ukən & \armenian{մուկն} & m\'oknə & \armenian{մօ՛կնը}& m\'uk & \armenian{մուկ} \\ 
`pomegranate' &n\'urən & \armenian{նուռն} & n\'ornə & \armenian{նօ՛ռնը} & nur & \armenian{նուռ} \\ 
`milk' &k\'ɑtʰən & \armenian{կաթն} & k\'ɑtʰnə & \armenian{կա՛թնը} & kɑtʰ & \armenian{կաթ} \\ 
`finger' &m\'ɑtʰən & \armenian{մատն} & m\'ɑnnə & \armenian{մա՛ննը} & mɑt & \armenian{մատ} \\ 
`foot' &\'otən & \armenian{ոտն} & v\'ə̟nnə & \armenian{վըէ՛ննը} & v\'ot & \armenian{ոտ} \\
`cold' &s\'ɑrən & \armenian{սառն} & s\'ɑrnə & \armenian{սա՛ռնը} & s\'ɑrən & \armenian{սառն} \\
 \hline 
 \end{tabular}
\end{table}

\paragraph{Assimilation of /tn/ to /nn/} 

It is also typical that the sound /t/ <\armenian{տ}> becomes /n/ <\armenian{ն}> when before /n/ <\armenian{ն}>, as an assimilation (Table \ref{tab:Karabakh:phonology:soundChange:cons:nn}). 


\begin{table}[H]
 \centering
 \caption{Assimilation from Classical Armenian /tn/ <\armenian{տն}> to /nn/ <\armenian{նն}> in the Karabakh dialect}
 \label{tab:Karabakh:phonology:soundChange:cons:nn}
 \begin{tabular}{|l| ll|ll| ll|}
 \hline & \multicolumn{2}{l|}{Classical Armenian} &\multicolumn{2}{l|}{> Karabakh} & \multicolumn{2}{l|}{cf. SEA} \\ 
`foot' &\'otən & \armenian{ոտն} & v\'ə̟nnə & \armenian{վըէ՛ննը} & v\'ot & \armenian{ոտ} \\
 `finger' &m\'ɑtʰən & \armenian{մատն} & m\'ɑnnə & \armenian{մա՛ննը} & mɑt & \armenian{մատ} \\ 
 `thimble' &mɑtn\'ot͡sʰ & \armenian{մատնոց} & mnn\'ɑnut͡sʰ & \armenian{մննա՛նուց} & mɑtn\'ot͡sʰ & \armenian{մատնոց} \\ 
 `to enter' &mətɑn\'el & \armenian{մտանել} & mnn\'el & \armenian{մննէլ} & mətn\'el& \armenian{մտնել} \\ 
\hline 
 \end{tabular}
\end{table}


\paragraph{Absence of /f/ and adaptation of loan /f/ to /pʰ/} 

The sound /f/ <\armenian{ֆ}> has entered almost all the dialects of New Armenian. But it is absent in Karabakh, just as in Tbilisi. Here too, like in Old Armenian, the sound /f/ <\armenian{ֆ}> of foreign words has changed to /pʰ/ <\armenian{փ}>
(Table \ref{tab:Karabakh:phonology:soundChange:cons:f}). 


\begin{table}[H]
 \centering
 \caption{Change of borrowed /f/ to /pʰ/ <\armenian{փ}> in the Karabakh dialect}
 \label{tab:Karabakh:phonology:soundChange:cons:f}
 \begin{tabular}{|l|ll|ll| }
 \hline & \multicolumn{2}{l|}{Source language } &\multicolumn{2}{l|}{> Karabakh } \\
 
 `factory' &French & <fabrique> & pʰ\'ɑbɾik & \armenian{փա՛բրիկ}\\
 `surname' &French & <famille> & pʰ\'ɑmil & \armenian{փալ} \\
 `Fez' &Turkish & <fes> & pʰæs & \armenian{փա̈ս} \\
 `carriage' &Turkish & <fayton> & pʰ\'ɑjton & \armenian{փա՛յտօն} \\
 `lamp' &Turkish & <fener> & pʰ\'ænæɾ &\armenian{փա̈՛նա̈ր} \\
 \hline 
 \end{tabular}
\end{table}

\section{Morphology}

\subsection{Noun inflection or declension}

\subsubsection{General paradigm}
The declensions in Karabakh are the same as in the previous two dialects. Here we see the following differences:
\begin{itemize}
 \item The genitive is formed with the formative /-ə̟/ <\armenian{ըէ}> formative (or /-e,-i/ <\armenian{է, ի}>). 
 \item The ablative with the formatives /-ɑ, -ɑn/ <\armenian{ա, ան}>. 
 \item The instrumental with the formative /-ɑv/ <\armenian{ավ}>. 
 \item The plural with the formatives /-ə̟ɾ, -nə̟ɾ, -ne/ <\armenian{ըէր, նըէր, նէ}>.
\end{itemize}

See Table \ref{tab:Karabakh:morpho:noun:paradigm}. 

\begin{table}[H]
 \centering \caption{Paradigm of plural and case suffixes for nominal declension in the Karabakh dialect}
 \label{tab:Karabakh:morpho:noun:paradigm}
 \begin{tabular}{|l| ll| lll| }
 \hline & \multicolumn{2}{l|}{Singular} & \multicolumn{3}{l|}{Plural ({\pl}-{\kase})} \\
{\nom} & & & -ə̟ɾ,& -nə̟ɾ, -ne & -\armenian{ըէր}, -\armenian{նըէր}, -\armenian{նէ} \\
{\gen}, {\dat} & -ə̟, -e, -i & -\armenian{ըէ}, -\armenian{է}, -\armenian{ի}& -\'eɾ-i, &-n\'eɾ-i & -\armenian{է՛րի}, -\armenian{նէ՛րի} \\
{\abl} & -ɑ, -ɑn & -\armenian{ա}, -\armenian{ան} & -\'eɾ-ɑn, &-n\'eɾ-ɑn& -\armenian{է՛րան}, -\armenian{նէ՛րան} \\
{\ins} & -ɑv & -\armenian{ավ} & -\'eɾ-ɑv, &-n\'eɾ-ɑv & -\armenian{է՛րավ}, -\armenian{նէ՛րավ} \\
{\locgloss} & -um & -\armenian{ում} & -\'eɾ-um, &-n\'eɾ-um & -\armenian{է՛րում}, -\armenian{նէրում} \\ \hline 
\end{tabular}

\end{table}

\subsubsection{Genitive formation}
Unlike the Yerevan and Tbilisi dialects, the genitive here can take the definite article /-n/ <\armenian{ն}> when needed (\ref{sent:karabakh:morpho:noun:gen}). Thus the genitive is not differentiated from the dative, just like in the dialects of the /kə/ <\armenian{կը}> branch. 

\begin{exe}
 \ex \label{sent:karabakh:morpho:noun:gen}
 \begin{xlist}
 \ex Karabakh 
 \begin{xlist}
 \ex \gll tʰ\'ʏn-ʏ-n b\'ə̟li-n \\
 Harutyun-{\gen}-{\defgloss} godfather-{\defgloss}\\
 \trans `Harutyun's godfather' \\
 \armenian{Թի՛ւնիւն պըէ՛լին}
 \ex \gll kɾikʰ\'oɾ-e-n hæɾ-ə\\
 Krikor-{\gen}-{\defgloss} father-{\defgloss} \\
 \trans `Krikor's father' \\
 \armenian{Կրիքօ՛րէն հա̈րը} 
 \end{xlist}
 \ex cf. SEA 
 \begin{xlist}
 \ex \gll hɑɾutʰjun-\'i kəŋkʰɑhɑ\'jɾ-ə \\
 Harutyun-{\gen} godfather \\
 \trans `Harutyun's godfather' \\
 \armenian{Հարությունի կնքահայրը} 
 \ex \gll ɡəɾikʰoɾ-\'i hɑjɾ-ə\\
 Grikor-{\gen} father-{\defgloss} \\
 \trans `Grikor's father' \\
 \armenian{Գրիգորի հայրը} 
 \end{xlist}
\end{xlist}
\end{exe}

The infinitive participles.. 


\begin{adjarianpage}\label{page:67}\end{adjarianpage}% should be 67

... take the genitive /-i/ <\armenian{ի}> instead of /-u/ <\armenian{ու}> (\ref{sent:karabakh:morpho:noun:geninf}), in accordance with the general rule

 

\begin{exe}
 \ex \label{sent:karabakh:morpho:noun:geninf}
 \begin{xlist}
 \ex Karabakh 
 \begin{xlist}
 \ex \gll hint͡ʃʰ \'on-i-s əs\'e-l-i \\
 what have-{\thgloss}-2{\sg} say-{\thgloss}-{\infgloss}-{\dat}\\
 \trans `What do you have to say?' \\
 \armenian{հի՞նչ օ՛նիս ըսէ՛լի} 
 \ex \gll χos-\'e-l-i məhɑɾ\\
 speak-{\thgloss}-{\infgloss}-{\gen} for \\
 \trans `for speaking' \\
 \armenian{խօսէ՛լի մըհար} 
 \end{xlist}
 \ex cf. SEA 
 \begin{xlist}
 \ex \gll int͡ʃʰ un-e-s ɑs\'e-l-u \\
 what have-{\thgloss}-2{\sg} say-{\thgloss}-{\infgloss}-{\dat}\\
 \trans `What do you have to say?' \\
 \armenian{ի՞նչ ունես ասելու} 
 \ex \gll χos-\'e-l-u hɑmɑɾ\\
 speak-{\thgloss}-{\infgloss}-{\gen} for \\
 \trans `for speaking' \\
 \armenian{խոսելու համար} 
 \end{xlist}
\end{xlist}
\end{exe}

\subsubsection{Additional units}

Almost all the nominal case markers can be preceded by the additional units /-ɑn, -nɑn/ <\armenian{ան, նան}> (Table \ref{tab:Karabakh:morpho:noun:extender}). \translatorHD{It seems these morphs act as stem-extenders between stems and case suffixes. It's unclear if they're optional. }


\begin{table}[H]
 \centering \caption{Additional suffixes before case suffixes in the Karabakh dialect}
 \label{tab:Karabakh:morpho:noun:extender}
 \begin{tabular}{|l| lll| }
 \hline & `paternal aunt' & & \\
{\abl} & hɑkʰve-n\'ɑn-ɑ & \armenian{հաքվէնա՛նա} & $\sqrt{}$-?-{\abl} \\
{\ins} & hɑkʰve-n\'ɑn-ɑv & \armenian{հաքվէնա՛նավ} & $\sqrt{}$-?-{\ins} \\
{\locgloss} & hɑkʰve-n\'ɑn-um & \armenian{հաքվէնա՛նում} & $\sqrt{}$-?-{\locgloss} \\
 \hline 
\end{tabular}

\end{table}

\subsection{Pronoun inflection or declension}

The following are the pronoun declensions. 

\subsubsection{Personal pronouns}
\translatorHD{Table \ref{tab:Karabakh:morpho:pronoun:personal} lists the personal pronouns. }

\begin{table}[H]
\caption{Inflection paradigm for personal pronouns in the Karabakh dialect }\label{tab:Karabakh:morpho:pronoun:personal}
\centering 
\begin{tabular}{| l| llll| }
 \hline & 1SG & 2SG & 1PL & 2PL \\
 & `I' & `you' & `we'& `you' \\\hline 
{\nom} & jes & tʏ & munkʰ & tukʰ \\
 & \armenian{յէս} & \armenian{տիւ} & \armenian{մունք} & \armenian{տուք} \\ 
\hline {\gen} & im & kʰu & mə̟ɾ & t͡sə̟ɾ \\
 & \armenian{իմ} & \armenian{քու} & \armenian{մըէր} & \armenian{ծըէր} \\
\hline {\dat}, {\acc} & ind͡z & kʰez & mə̟z & t͡sə̟z \\
 & \armenian{ինձ} & \armenian{քէզ} & \armenian{մըէզ} & \armenian{ծըէզ} \\
\hline {\abl} & ənd͡z-\'ɑn-ɑ & kʰʲəz-\'ɑn-ɑ & məz-\'ɑn-ɑ & t͡səz-\'ɑn-ɑ \\
 & \armenian{ընձա՛նա} & \armenian{քյըզա՛նա} & \armenian{մըզա՛նա} & \armenian{ծըզա՛նա} \\
\hline {\ins} & ənd͡z-\'ɑn-ɑv & kʰʲəz-\'ɑn-ɑv & məz-\'ɑn-ɑv & t͡səz-\'ɑn-ɑv \\
 & \armenian{ընձա՛նավ} & \armenian{քյըզա՛նավ} & \armenian{մըզա՛նավ} & \armenian{ծըզա՛նավ} \\
\hline {\locgloss} & ənd͡z-\'ɑn-um & kʰʲəz-\'ɑn-um &məz-\'ɑn-um & t͡səz-\'ɑn-um \\
 & \armenian{ընձա՛նում} & \armenian{քյըզա՛նում} &\armenian{մըզա՛նում} & \armenian{ծըզա՛նում} 
 \\ \hline 
\end{tabular}
\end{table}

\subsubsection{Logophoric pronouns}
\translatorHD{For the third person personal pronouns, SEA uses two sets of pronouns: logophoric pronouns like /iŋkʰə/, and a non-logophoric pronoun that's connected to the distal demonstrative /nɑ/. The logophoric pronouns are in Table \ref{tab:Karabakh:morpho:pronoun:log}. }


\begin{table}[H]
\caption{Inflection paradigm for third person logophoric pronouns in the Karabakh dialect }\label{tab:Karabakh:morpho:pronoun:log}
\centering 
\begin{tabular}{| l| llll|}
 \hline & 3SG & & 3PL & \\
 & `he/she' & & `they'& \\
 \hline 
{\nom} & \'inkʰʲən & \armenian{ի՛նքյըն} & \'ʏɾɑnkʰ & \armenian{ի՛ւրանք} \\
{\gen} {\dat} {\acc} & \'ʏɾɑn & \armenian{ի՛ւրան} & \'ʏɾɑnt͡sʰ & \armenian{ի՛ւրանց} \\
{\abl} & ʏɾ\'ɑn-ɑn & \armenian{իւրա՛նան} & ʏɾ\'ɑnt͡sʰ-ɑn & \armenian{իւրա՛նցան}\\
{\ins} & ʏɾ\'ɑn-ɑv & \armenian{իւրա՛նավ} & ʏɾ\'ɑnt͡sʰ-ɑv & \armenian{իւրա՛նցավ}\\
{\locgloss} & ʏɾ\'ɑn-um & \armenian{իւրա՛նում} & ʏɾ\'ɑnt͡sʰ-um & \armenian{իւրա՛նցում}
\\ \hline 
\end{tabular}\end{table}

\subsubsection{Demonstrative pronouns}
\translatorHD{Demonstrative pronouns come in three sets: proximal, medial, and distal. Within each set, Karabakh seems to have use four separate lexemes or patterns with unclear semantic differences: singular in Pattern A (Table \ref{tab:Karabakh:morpho:pronoun:dem:A}), plural in Patterns B (Table \ref{tab:Karabakh:morpho:pronoun:dem:B}), C (Table \ref{tab:Karabakh:morpho:pronoun:dem:C}), and D (Table \ref{tab:Karabakh:morpho:pronoun:dem:D}). }


\begin{table}[H]
\caption{Inflection paradigm (Pattern A) for 3SG demonstrative pronouns in the Karabakh dialect }\label{tab:Karabakh:morpho:pronoun:dem:A}
\centering 
\begin{tabular}{| l| lll| }
 \hline & proximal & medial & distal \\
 & `this' & `that' & `yonder' \\
 \hline {\nom} & es & et & en \\
 & \armenian{էս} & \armenian{էտ} & \armenian{էն} \\
\hline {\gen} & estəɾɑ & ətɾɑ & əndəɾɑ \\
 & \armenian{ըստըրա} & \armenian{ըտրա} & \armenian{ընդըրա} \\
\hline {\dat} {\acc} & estəɾɑn & ətɾɑn & əndəɾɑn \\
 & \armenian{ըստըրան} & \armenian{ըտրան} & \armenian{ընդըրան} \\
\hline {\abl} & estəɾ\'ɑn-ɑ & ətɾ\'ɑn-ɑ & əndəɾ\'ɑn-ɑ \\
 & \armenian{ըստըրա՛նա} & \armenian{ըտրա՛նա} & \armenian{ընդըրա՛նա} \\
\hline {\ins} & estəɾ\'ɑn-ɑv & ətɾ\'ɑn-ɑv & əndəɾ\'ɑn-ɑv \\
 & \armenian{ըստըրա՛նավ} & \armenian{ըտրա՛նավ} & \armenian{ընդըրա՛նավ} \\
\hline {\locgloss} & estəɾ\'ɑn-um & ətɾ\'ɑn-um & əndəɾ\'ɑn-um \\
 & \armenian{ըստըրա՛նում} & \armenian{ըտրա՛նում} & \armenian{ընդըրա՛նում} \\ \hline
\end{tabular}
\end{table}



\begin{table}[H]
\caption{Inflection paradigm (Pattern B) for 3PL demonstrative pronouns in the Karabakh dialect }\label{tab:Karabakh:morpho:pronoun:dem:B}
\centering 
\begin{tabular}{| l| lll| }
 \hline & proximal & medial & distal \\
 & `these' & `those' & `those yonder' \\
\hline {\nom} & əstəhɑnkʰ & ətəhɑnkʰ & əndəhɑnkʰ \\
 & \armenian{ըստըհանք} & \armenian{ըտըհանք} & \armenian{ընդըհանք} \\\hline 
{\gen} & əstəhɑnt͡sʰ & ətəhɑnt͡sʰ & əndəhɑnt͡sʰ \\
 & \armenian{ըստըհանց} & \armenian{ըտըհանց} & \armenian{ընդըհանց} \\\hline 
{\dat} {\acc} & əstəhɑnt͡sʰ & ətəhɑnt͡sʰ & əndəhɑnt͡sʰ \\
 & \armenian{ըստըհանց} & \armenian{ըտըհանց} & \armenian{ընդըհանց} \\\hline 
{\abl} & əstəh\'ɑnt͡sʰ-ɑn & ətəh\'ɑnt͡sʰ-ɑn & əndəh\'ɑnt͡sʰ-ɑn \\
 & \armenian{ըստըհա՛նցան} & \armenian{ըտըհա՛նցան} & \armenian{ընդըհա՛նցան} \\\hline 
{\ins} & əstəh\'ɑnt͡sʰ-ɑv & ətəh\'ɑnt͡sʰ-ɑv & əndəh\'ɑnt͡sʰ-ɑv \\
 & \armenian{ըստըհա՛նցավ} & \armenian{ըտըհա՛նցավ} & \armenian{ընդըհա՛նցավ} \\\hline 
{\locgloss} & əstəh\'ɑnt͡sʰ-um & ətəh\'ɑnt͡sʰ-um & əndəh\'ɑnt͡sʰ-um \\
 & \armenian{ըստըհա՛նցում} & \armenian{ըտըհա՛նցում} & \armenian{ընդըհա՛նցում} \\ \hline 
\end{tabular}
\end{table}


\begin{table}[H]
\caption{Inflection paradigm (Pattern C) for 3PL demonstrative pronouns in the Karabakh dialect }\label{tab:Karabakh:morpho:pronoun:dem:C}
\centering 
\begin{tabular}{|l|lll|}
\hline & proximal& medial & distal \\
& `these' & `those' & `those yonder'\\\hline 
{\nom} & əstəɾɑnkʰ & ətəɾɑnkʰ & əndəɾɑnkʰ \\
& \armenian{ըստըրանք}& \armenian{ըտըրանք}& \armenian{ընդըրանք}\\\hline 
{\gen} & əstəɾɑnt͡sʰ& ətəɾɑnt͡sʰ& əndəɾɑnt͡sʰ\\
& \armenian{ըստըրանց}& \armenian{ըտըրանց}& \armenian{ընդըրանց}\\\hline 
{\dat} {\acc} & əstəɾɑnt͡sʰ& ətəɾɑnt͡sʰ& əndəɾɑnt͡sʰ\\
& \armenian{ըստըրանց}& \armenian{ըտըրանց}& \armenian{ընդըրանց}\\\hline 
{\abl} & əstəɾ\'ɑnt͡sʰ-ɑn & ətəɾ\'ɑnt͡sʰ-ɑn & əndəɾ\'ɑnt͡sʰ-ɑn \\
& \armenian{ըստըրա՛նցան}& \armenian{ըտըրա՛նցան}& \armenian{ընդըրա՛նցան}\\\hline 
{\ins} & əstəɾ\'ɑnt͡sʰ-ɑv & ətəɾ\'ɑnt͡sʰ-ɑv & əndəɾ\'ɑnt͡sʰ-ɑv \\
& \armenian{ըստըրա՛նցավ}& \armenian{ըտըրա՛նցավ}& \armenian{ընդըրա՛նցավ}\\\hline 
{\locgloss} & əstəɾ\'ɑnt͡sʰ-um & ətəɾ\'ɑnt͡sʰ-um & əndəɾ\'ɑnt͡sʰ-um \\
& \armenian{ըստըրա՛նցում} & \armenian{ըտըրա՛նցում} & \armenian{ընդըրա՛նցում} \\ \hline 
\end{tabular}
\end{table}

\begin{table}[H]
\caption{Inflection paradigm (Pattern D) for 3PL demonstrative pronouns in the Karabakh dialect }\label{tab:Karabakh:morpho:pronoun:dem:D}
\centering 
\begin{tabular}{|l|lll|}
 \hline & proximal & medial & distal \\
 & `these' & `those' & `those yonder' \\\hline 
{\nom} & səhɑnkʰ & təhɑnkʰ & nəhɑnkʰ \\
 & \armenian{սըհանք} & \armenian{տըհանք} & \armenian{նըհանք} \\\hline 
{\gen} & səhɑnt͡sʰ & təhɑnt͡sʰ & nəhɑnt͡sʰ \\
 & \armenian{սըհանց} & \armenian{տըհանց} & \armenian{նըհանց} \\\hline 
{\dat} {\acc} & səhɑnt͡sʰ & təhɑnt͡sʰ & nəhɑnt͡sʰ \\
 & \armenian{սըհանց} & \armenian{տըհանց} & \armenian{նըհանց} \\\hline 
{\abl} & səh \'ɑnt͡sʰ-ɑn & təh \'ɑnt͡sʰ-ɑn & nəh \'ɑnt͡sʰ-ɑn \\
 & \armenian{սըհա՛նցան} & \armenian{տըհա՛նցան} & \armenian{նըհա՛նցան} \\\hline 
{\ins} & səh \'ɑnt͡sʰ-ɑv & təh \'ɑnt͡sʰ-ɑv & nəhɑnt͡sʰ- \'ɑn-ɑv \\
 & \armenian{սըհա՛նցավ} & \armenian{տըհա՛նցավ} & \armenian{նըհանցա՛նավ} \\\hline 
{\locgloss} & səh \'ɑnt͡sʰ-um & təh \'ɑnt͡sʰ-um & nəhɑnt͡sʰ- \'ɑn-um \\
 & \armenian{սըհա՛նցում} & \armenian{տըհա՛նցում} & \armenian{նըհանցա՛նում} \\ \hline 
\end{tabular}
\end{table}

\subsubsection{Interrogative pronouns}
\translatorHD{Adjarian provides the set of interrogative pronouns in Table \ref{tab:Karabakh:morpho:pronoun:who} for `who'. Note that plural has two sets of declensions. }

\begin{table}[H]
\caption{Inflection paradigm for the interrogative pronoun `who' in the Karabakh dialect }\label{tab:Karabakh:morpho:pronoun:who}
\centering 
\begin{tabular}{|l|lll|}
\hline & Singular & Plural & \\\hline 
{\nom} & hu, huv & h\'uv-eɾkʰ & \\
 & \armenian{հու, հուվ} & \armenian{հո՛ւվէրք} & \\\hline 
{\gen} {\dat} {\acc} & hʏɾ & h\'uv-eɾt͡sʰ & h\'ʏɾ-ɑnt͡sʰ \\
 & \armenian{հիւր} & \armenian{հո՛ւվէրց} & \armenian{հի՛ւրանց} \\\hline 
{\abl} & hʏɾ-\'ɑn-ɑ & huv-\'eɾt͡sʰ-ɑn & hʏɾ-\'ɑnt͡sʰ-ɑn \\
 & \armenian{հիւրա՛նա} & \armenian{հուվէ՛րցան} & \armenian{հիւրա՛նցան} \\\hline 
{\ins} & hʏɾ-\'ɑn-ɑv & huv-\'eɾt͡sʰ-ɑv & hʏɾ-\'ɑnt͡sʰ-ɑv \\
 & \armenian{հիւրա՛նավ} & \armenian{հուվէ՛րցավ} & \armenian{հիւրա՛նցավ} \\\hline 
{\locgloss} & hʏɾ-\'ɑn-um & huv-\'eɾt͡sʰ-um & hʏɾ-\'ɑnt͡sʰ-um \\
 & \armenian{հիւրա՛նում} & \armenian{հուվէ՛րցում} & \armenian{հիւրա՛նցում} \\ \hline 
\end{tabular}
\end{table}

\begin{adjarianpage}\label{page:68}\end{adjarianpage}% should be 68

\subsection{Verb inflection or conjugation}
\subsubsection{Overview}\label{sec:Karabakh:morpho:verb:overview}

Verbal conjugations show some innovations. The stem of the present or imperfective is formed with the formatives /-um, -əm, -ɑm, -is, -es, ɑs/ <\armenian{ում, ըմ, ամ, իս, էս, աս}>. The first three belong to the Khachen province, while the last three belong to the Varanda and Dizak provinces. For example, all of the forms in Table \ref{tab:Karabakh:morpho:verb:um} mean the same thing.

\begin{table}[H]
 \centering
 \caption{Forms of the imperfective converb suffix in the Karabakh dialect with the verb `I like' in in the indicative present } 
 \label{tab:Karabakh:morpho:verb:um}\begin{tabular}{| l| ll| }
 \hline First group & siɾ-um ə-m& \armenian{սիրում ըմ} \\
 & siɾ-əm ə-m & \armenian{սիրըմ ըմ} \\
 & siɾ-ɑm ə-m & \armenian{սիրամ ըմ} \\
 \hline 
 Second group & siɾ-is ə-m & \armenian{սիրիս ըմ} \\
 & siɾ-es ə-m& \armenian{սիրէս ըմ} \\
 & siɾ-ɑs ə-m & \armenian{սիրաս ըմ} 
 \\\hline 
 &\multicolumn{2}{l|}{$\sqrt{}$-{\impfcvb} {\aux}-1{\sg}} \\ \hline
 \end{tabular}

\end{table}

 The imperfective is similarly formed (Table \ref{tab:Karabakh:morpho:verb:umPast} and so on). 

\begin{table}[H]
 \centering
 \caption{Forms of the imperfective converb suffix in the Karabakh dialect with the verb `I was liking' in in the indicative past imperfective } 
 \label{tab:Karabakh:morpho:verb:umPast}
 \begin{tabular}{| ll| } \hline 
 siɾ-um i-$\emptyset$-$\emptyset$ & \armenian{սիրում ի} \\
siɾ-əm i-$\emptyset$-$\emptyset$ & \armenian{սիրըմ ի} \\
 siɾ-es i-$\emptyset$-$\emptyset$ & \armenian{սիրէս ի} 
 \\\hline 
 \multicolumn{2}{l|}{ $\sqrt{}$-{\impfcvb} {\aux}-{\pst}-1{\sg}}\\ \hline
 \end{tabular}

\end{table}

The future is formed with the formative /kə/ <\armenian{կը}>, which becomes /kʰə/ <\armenian{քը}> next to voiceless sounds. 

The definite future (\armenian{որոշեալ ապառնի}) is formed with the formatives /ɑkɑn, ɑt͡sʰukʰ/ <\armenian{ական, ացուք}>. 

The forms /piti, pitim, petum ɑ/ <\armenian{պիտի, պիտիմ, պէտում ա}> are used to form the various tenses of the debitive mood (\armenian{պարտաւորական եղանակը}). 

Besides there, there are many so-called immediate (\armenian{անմիջական}) and narrative (\armenian{պատմական}) forms forms, which we show below along with the previously mentioned form. 


\subsubsection{General paradigm}

\translatorHD{We show the paradigm for the verb `to like', as a reflex from Classical /siɾ-e-l/ <\armenian{սիրել}>.} 


{\paradigmExplanation}

\paragraph{Indicative present and past imperfective}

\translatorHD{The indicative present in SEA is formed via periphrasis (Table \ref{tab:Karabakh:morpho:verb:paradigm:presentIndc}). The verb is in a converb form called the imperfective converb with the suffix /-um/. Tense and agreement is on an inflected auxiliary. The Tbilisi dialect shows the same strategy but with two major differences. First, the converb suffix is quite in general /-əm/. Second, the auxiliary has different morphs. The auxiliary is /e/ in SEA; but in Karabakh, the auxiliary is /ɑ/ in 3SG present, and /ə/ for the present forms. }


\begin{table}[H]
 \centering
 \caption{Indicative present <\armenian{ներկայ}> of the verb `to like' in the Karabakh dialect}
 \label{tab:Karabakh:morpho:verb:paradigm:presentIndc}
 \begin{tabular}{|l|ll|ll|}
 \hline & \multicolumn{2}{l|}{Karabakh} & \multicolumn{2}{l|}{cf. SEA} \\
1SG & s\'iɾ-əm ə-m & \armenian{սի՛րըմ ըմ} & siɾ-\'um e-m & \armenian{սիրում եմ} \\
2SG & s\'iɾ-əm ə-s & \armenian{սի՛րըմ ըս} & siɾ-\'um e-s & \armenian{սիրում ես} \\
3SG & s\'iɾ-əm ɑ & \armenian{սի՛րըմ ա} & siɾ-\'um e & \armenian{սիրում է} \\
1PL & s\'iɾ-əm ə-nkʰ & \armenian{սի՛րըմ ընք} & siɾ-\'um e-ŋkʰ & \armenian{սիրում ենք}\\
2PL & s\'iɾ-əm ə-kʰ & \armenian{սի՛րըմ ըք} & siɾ-\'um e-kʰ & \armenian{սիրում եք} \\
3PL & s\'iɾ-əm ə-n & \armenian{սի՛րըմ ըն} & siɾ-\'um e-n & \armenian{սիրում են} \\
& \multicolumn{2}{l|}{$\sqrt{}$-{\impfcvb} {\aux}-{\agr}}& \multicolumn{2}{l|}{$\sqrt{}$-{\impfcvb} {\aux}-{\agr}}
 \\ \hline 
 \end{tabular} \end{table}


 
\translatorHD{The indicative past imperfective uses the same imperfective converb as in the present (Table \ref{tab:Karabakh:morpho:verb:paradigm:pastImpfIndc}). The difference is that auxiliary is now in the past tense. In SEA, the auxiliary has the constant shape /e/ in the past. But in Karabakh, the auxiliary is /ɑ/ in the 3SG. For the other persons, SEA has an underlying sequence /e-i/ that surfaces with glide epenthesis [ej-i], glossed as {\aux}-{\pst}. But in Karabakh, this sequence is replaced by just [i]. Hypothetically, this Karabakh [i] can be derived from either the auxiliary or the past suffix. Data from the past perfective (\ref{sec:Karabakh:morphology:verb:paradigm:aor}) suggests that the past suffix is /-e/ in this dialect, and that this /e/ is deleted after theme vowels and auxiliaries like /i/. Thus, Karabakh and SEA switch glosses for the surface /i/ morph. I admit though that this analysis is tentative and based only on Adjarian's sample paradigms for only the reflex of the Classical E-Class. }



\begin{table}[H]
 \centering
 \caption{Indicative past imperfective <\armenian{անկատար}> of the verb `to like' in the Karabakh dialect}
 \label{tab:Karabakh:morpho:verb:paradigm:pastImpfIndc}
 \begin{tabular}{|l|ll|ll|}
\hline & \multicolumn{2}{l|}{Karabakh} & \multicolumn{2}{l|}{cf. SEA} \\
1SG & s\'iɾ-əm i-$\emptyset$-$\emptyset$ & \armenian{սի՛րըմ ի} & siɾ-\'um ej-i-$\emptyset$ & \armenian{սիրում էի} \\
2SG & s\'iɾ-əm i-$\emptyset$-ɾ & \armenian{սի՛րըմ իր} & siɾ-\'um ej-i-ɾ & \armenian{սիրում էիր} \\
3SG & s\'iɾ-əm ɑ-$\emptyset$-ɾ & \armenian{սի՛րըմ ար} & siɾ-\'um e-$\emptyset$-ɾ & \armenian{սիրում էր} \\
1PL & s\'iɾ-əm i-$\emptyset$-nkʰ & \armenian{սի՛րըմ ինք} & siɾ-\'um ej-i-ŋkʰ & \armenian{սիրում էինք} \\
2PL & s\'iɾ-əm i-$\emptyset$-kʰʲ &\armenian{սի՛րըմ իքյ} & siɾ-\'um ej-i-kʰ & \armenian{սիրում էիք} \\
3PL & s\'iɾ-əm i-$\emptyset$-n & \armenian{սի՛րըմ ին} & siɾ-\'um ej-i-n & \armenian{սիրում էին} \\
& \multicolumn{2}{l|}{$\sqrt{}$-{\impfcvb} {\aux}-{\pst}-{\agr}}& \multicolumn{2}{l|}{$\sqrt{}$-{\impfcvb} {\aux}-{\pst}-{\agr}} \\
\hline 
\end{tabular}
\end{table}

\translatorHD{Note that Adjarian transcribed the present 2PL of the auxiliary as /-kʰ/ in the present but /-kʰʲ/ in the past. I'm not sure if this is an error by Adjarian. The past and subjunctive forms from the following sections likewise use /-kʰʲ/. }

\paragraph{Past perfective or aorist}\label{sec:Karabakh:morphology:verb:paradigm:aor}

\translatorHD{The past perfective (Table \ref{tab:Karabakh:morpho:verb:paradigm:pastperfectiveAorist}) is also called the aorist. In SEA for /siɾ-e-l/ `to like', the past perfective is formed by taking the root and theme vowel, adding the aorist or perfective suffix /-t͡sʰ-/, and then adding the past suffix /-i/ and the appropriate agreement suffixes. The 3SG uses covert tense and agreement suffixes. The Karabakh dialect behaves quite differently: the past suffix is /-e/ instead of /-i/, the theme vowel is /i/ in the 3SG but /e/ elsewhere. }


\begin{table}[H]
 \centering
 \caption{Past perfective or aorist <\armenian{կատարեալ}> of the verb `to like' in the Karabakh dialect}
 \label{tab:Karabakh:morpho:verb:paradigm:pastperfectiveAorist}
 \begin{tabular}{|l|ll|ll|}
\hline & \multicolumn{2}{l|}{Karabakh} & \multicolumn{2}{l|}{cf. SEA} \\
1SG & siɾ-\'e-t͡sʰ-e-$\emptyset$ & \armenian{սիրէ՛ցէ} & siɾ-e-t͡sʰ-\'i-$\emptyset$ & \armenian{սիրեցի} \\
2SG & siɾ-\'e-t͡sʰ-e-ɾ & \armenian{սիրէ՛ցէր} & siɾ-e-t͡sʰ-\'i-ɾ & \armenian{սիրեցիր} \\
3SG & s\'iɾ-i-t͡sʰ-$\emptyset$-$\emptyset$ & \armenian{սի՛րից} & siɾ-\'e-t͡sʰ-$\emptyset$-$\emptyset$ & \armenian{սիրեց} \\
1PL & siɾ-\'e-t͡sʰ-e-nkʰʲ & \armenian{սիրէ՛ցէնքյ} & siɾ-e-t͡sʰ-\'i-ŋkʰ & \armenian{սիրեցինք} \\
2PL & siɾ-\'e-t͡sʰ-e-kʰʲ & \armenian{սիրէ՛ցէքյ} & siɾ-e-t͡sʰ-\'i-kʰ & \armenian{սիրեցիք} \\
3PL & siɾ-\'e-t͡sʰ-e-n & \armenian{սիրէ՛ցէն} & siɾ-e-t͡sʰ-\'i-n & \armenian{սիրեցին} \\
& \multicolumn{2}{l|}{$\sqrt{}$-{\thgloss}-{\aor}-{\pst}-{\agr}}& \multicolumn{2}{l|}{$\sqrt{}$-{\thgloss}-{\aor}-{\pst}-{\agr}}\\ 
\hline 
\end{tabular}
\end{table}

\translatorHD{Note though that theme vowel is /e/ for all but the 3SG. The past perfective 3SG instead uses the theme vowel /i/. }

\translatorHD{Note that Adjarian transcribed the present 1PL of the perfective as /-nkʰʲ/ while auxiliaries in the indicative present/past used /-nkʰ/. I'm not sure if this is an error by Adjarian. The subjunctive forms from the following section likewise use /-nkʰʲ/. }

\paragraph{Subjunctive present and past imperfective } 

\translatorHD{In SEA, the subjunctive present (Table \ref{tab:Karabakh:morpho:verb:paradigm:subjPresent}) is formed by adding agreement suffixes after the theme vowel /e/. These are the same agreement suffixes that are added onto the present auxiliary in the indicative present. For a verb like `to like', the 3SG involves changing the theme vowel /e/ to /i/ in the 3SG. The Karabakh dialect is similar with one main difference: the theme vowel is /i/ instead of /e/, much like how the auxiliary is /i/ instead of /e/. } 


\begin{table}[H]
 \centering
 \caption{Subjunctive present <\armenian{ստորադասական ներկայ}> of the verb `to like' in the Karabakh dialect}
 \label{tab:Karabakh:morpho:verb:paradigm:subjPresent}
 \begin{tabular}{|l|ll|ll|}
\hline & \multicolumn{2}{l|}{Karabakh} & \multicolumn{2}{l|}{cf. SEA} \\
1SG & s\'iɾ-i-m & \armenian{սի՛րիմ} & siɾ-\'e-m & \armenian{սիրեմ} \\
2SG & s\'iɾ-i-s & \armenian{սի՛րիս} & siɾ-\'e-s & \armenian{սիրես} \\
3SG & s\'iɾ-i-$\emptyset$ & \armenian{սի՛րի} & siɾ-\'i-$\emptyset$ & \armenian{սիրի} \\
1PL & s\'iɾ-i-nkʰʲ & \armenian{սի՛րինքյ} & siɾ-\'e-ŋkʰ & \armenian{սիրենք} \\
2PL & s\'iɾ-i-kʰʲ & \armenian{սի՛րիքյ} & siɾ-\'e-kʰ & \armenian{սիրեք} \\
3PL & s\'iɾ-i-n & \armenian{սի՛րին} & siɾ-\'e-n & \armenian{սիրեն} \\
& \multicolumn{2}{l|}{$\sqrt{}$-{\thgloss}-{\agr}}& \multicolumn{2}{l|}{$\sqrt{}$-{\thgloss}-{\agr}}\\ 
\hline 
\end{tabular}
\end{table}

\translatorHD{In SEA, the subjunctive past imperfective (Table \ref{tab:Karabakh:morpho:verb:paradigm:subjPast}) is formed by adding the past suffix /i/ and agreement suffixes after the theme vowel /e/. The underlying sequence /-e-i/ surfaces as [-ej-i]. In contrast in Karabakh, the sequence /e-i/ is replaced by [i]. Based on comparisons with the indicative past imperfective and the past perfective, it seems that the past suffix is /e/, and that this suffix is deleted after the theme vowel /i/. Thus the transformation is from underlying /-i-e/ to [-i]. Note that in both SEA and Karabakh, the past suffix is zero in the 3SG, while the theme and auxiliary is /ɑ/. }



\begin{table}[H]
 \centering
 \caption{Subjunctive past <\armenian{ստորադասական անցեալ}> of the verb `to like' in the Karabakh dialect}
 \label{tab:Karabakh:morpho:verb:paradigm:subjPast}
 \begin{tabular}{|l|ll|ll|}
\hline & \multicolumn{2}{l|}{Karabakh} & \multicolumn{2}{l|}{cf. SEA} \\
1SG & si\'ɾ-i-$\emptyset$-$\emptyset$ & \armenian{սի՛րի} & siɾ-ej-\'i-$\emptyset$ & \armenian{սիրեի} \\
2SG & s\'iɾ-i-$\emptyset$-ɾ & \armenian{սի՛րիր} & siɾ-ej-\'i-ɾ & \armenian{սիրեիր} \\
3SG & s\'iɾ-ɑ-$\emptyset$-ɾ & \armenian{սի՛րար} & siɾ-\'e-$\emptyset$-ɾ & \armenian{սիրեր} \\
1PL & s\'iɾ-i-$\emptyset$-nkʰʲ & \armenian{սի՛րինքյ} & siɾ-ej-\'i-ŋkʰ & \armenian{սիրեինք} \\
2PL & s\'iɾ-i-$\emptyset$-kʰʲ & \armenian{սի՛րիքյ} & siɾ-ej-\'i-kʰ & \armenian{սիրեիք} \\
3PL & s\'iɾ-i-$\emptyset$-n & \armenian{սի՛րին} & siɾ-ej-\'i-n & \armenian{սիրեին} \\
& \multicolumn{2}{l|}{$\sqrt{}$-{\thgloss}-{\pst}-{\agr}}& \multicolumn{2}{l|}{$\sqrt{}$-{\thgloss}-{\pst}-{\agr}}\\ 
\hline 
\end{tabular}
\end{table}

 
\paragraph{Tenses built from the subjunctive: Future }
 
 
 \translatorHD{In Karabakh, many other tenses seem to be built off of the subjunctive (Table \ref{tab:Karabakh:morpho:verb:paradigm:complexSubjunctive}). The future and future perfect are built by adding the prefix /kə/ before the subjunctive present and subjunctive past. (Note that this prefix is /kʰə/ before voiceless sounds, as stated by Adjarian (\S\ref{sec:Karabakh:morpho:verb:overview}). SEA behaves essentially the same and I don't provide its paradigm. }
 

\begin{table}[H]
 \centering
 \caption{Forms that are built from the subjunctive forms for the verb `to like' in the Karabakh dialect}
 \label{tab:Karabakh:morpho:verb:paradigm:complexSubjunctive}
 \begin{tabular}{|l|ll|ll|}
\hline & 
\multicolumn{2}{l|}{Future <\armenian{ապառնի}>} & \multicolumn{2}{l|}{Future perfect <\armenian{անցեալ ապառնի}> } \\
1SG & kʰə s\'iɾ-i-m & \armenian{քը սի՛րիմ} & kʰə s\'iɾ-i-$\emptyset$-$\emptyset$ & \armenian{քը սի՛րի} \\
2SG & kʰə s\'iɾ-i-s & \armenian{քը սի՛րիս}& kʰə s\'iɾ-i-$\emptyset$-ɾ & \armenian{քը սի՛րիր} \\
3SG & kʰə s\'iɾ-i-$\emptyset$ & \armenian{քը սի՛րի} & kʰə s\'iɾ-ɑ-$\emptyset$-ɾ & \armenian{քը սի՛րար} \\
1PL & kʰə s\'iɾ-i-nkʰʲ & \armenian{կի սի՛րինքյ}& kʰə s\'iɾ-i-$\emptyset$-nkʰʲ & \armenian{քը սի՛րինքյ} \\
2PL & kʰə s\'iɾ-i-kʰʲ & \armenian{քը սի՛րիքյ} & kʰə s\'iɾ-i-$\emptyset$-kʰʲ & \armenian{քը սի՛րիքյ} \\
3PL & kʰə s\'iɾ-i-n & \armenian{քը սի՛րին} & kʰə s\'iɾ-i-$\emptyset$-n & \armenian{քը սի՛րին} 
\\
& \multicolumn{2}{l|}{{\fut} $\sqrt{}$-{\thgloss}-{\agr}}& \multicolumn{2}{l|}{{\fut} $\sqrt{}$-{\thgloss}-{\pst}-{\agr}}
\\ \hline 
\end{tabular}
\end{table}


\paragraph{Imperative and prohibitive}

\translatorHD{For the imperative 2SG, SEA adds the morph /-iɾ/ after the root for a verb like `to like' (Table \ref{tab:Karabakh:morpho:verb:paradigm:Imp}). For the 2PL, archaic SEA adds the sequence /-e-t͡sʰ-ekʰ/ after the root such that /-e-t͡sʰ/ forms the aorist stem, while /-ekʰ/ is the agreement marker. More modern registers of SEA instead just add the sequence /-ekʰ/ directly after the root. Karabakh uses similar strategies: the 2SG marker is either /-i/ or /-e/. The 2PL system seems to match SEA. }


\begin{table}[H]
 \centering
 \caption{Imperative forms <\armenian{հրամայական}> for the verb `to like' in the Karabakh dialect}
 \label{tab:Karabakh:morpho:verb:paradigm:Imp}
 \begin{tabular}{|l|ll|ll|l|}
\hline & \multicolumn{2}{l|}{Karabakh} & \multicolumn{2}{l|}{cf. SEA} & \\
2SG & s\'iɾ-i & \armenian{սի՛րի} & siɾ-\'iɾ & \armenian{սիրի՛ր} & $\sqrt{}$-{\imp}.2{\sg}
\\
& s\'iɾ-e & \armenian{սի՛րէ} & & & $\sqrt{}$-{\imp}.2{\sg}
\\
2PL& siɾ-\'e-t͡sʰ-ekʰʲ& \armenian{սիրէ՛ցէքյ} & siɾ-e-t͡sʰ-\'ekʰ& \armenian{սիրեցեք} & $\sqrt{}$-{\thgloss}-{\aor}-{\imp}.2{\pl}
\\
& s\'iɾ-ekʰʲ &\armenian{սի՛րէքյ} & siɾ-\'ekʰ&\armenian{սիրեք}& $\sqrt{}$-{\imp}.2{\pl}
\\\hline \end{tabular}
\end{table}

\translatorHD{For the prohibitive or negative imperative (Table \ref{tab:Karabakh:morpho:verb:paradigm:Proh}), SEA simply adds the prohibitive formative /mi/ before the imperative form. Karabakh however uses a more complex system. One option is to give the verb a suffix /-il/ (possibly a non-finite form like the infinitive), and then add the prohibitive marker /mə̟ɾ/ for the 2SG or /mə̟kʰʲ/ for the 2PL. Another option is to inflect the verb with /-s/ for 2SG or /-kʰʲ/ for the 2PL, and then add the negation word /və̟t͡ʃʰ/ (likely a cognate of SEA `no' /vot͡ʃʰ/. The two strategies does differ in the placement of inflection: on either the verb or the post-verbal marker. Thus, agreement morphology is mobile } 


\begin{table}[H]
 \centering
 \caption{Negative imperative or prohibitive forms for the verb `to like' in the Karabakh dialect}
 \label{tab:Karabakh:morpho:verb:paradigm:Proh}
 \begin{tabular}{|l|ll|ll|ll|}
\hline & \multicolumn{4}{l|}{Karabakh} & \multicolumn{2}{l|}{cf. SEA} \\
2SG & s\'iɾ-i-l mə̟-ɾ & \armenian{սի՛րիլ մըէր} &s\'iɾ-i-s və̟t͡ʃʰ & \armenian{սի՛րիս վըէչ} & m\'i siɾ-iɾ & \armenian{մի՛ սիրիր} \\ 
2PL & s\'iɾ-i-l mə̟-kʰʲ & \armenian{սի՛րիլ մըէքյ} &s\'iɾ-i-kʲʰ və̟t͡ʃʰ & \armenian{սի՛րիքյ վըէչ} & m\'i siɾ-ekʰ& \armenian{մի՛ սիրեք} \\
& \multicolumn{2}{l|}{$\sqrt{}$-{\thgloss}-{\infgloss}? {\proh}-{\agr}}& \multicolumn{2}{l|}{$\sqrt{}$-{\thgloss}-{\agr}? {\neggloss}}& \multicolumn{2}{l|}{{\proh} $\sqrt{}$-{\agr}} \\
\hline \end{tabular}
\end{table}


\begin{adjarianpage}\label{page:69}\end{adjarianpage}% should be 69



\paragraph{Non-finite forms}

\translatorHD{Finally, Adjarian lists the following non-finite forms of this verb (participles or converbs) in Table \ref{tab:Karabakh:morpho:verb:paradigm:participle}. Note that present participle is also called the subject participle. What Adjarian calls the past participle is differentiated in SEA as a resultative participle with /-ɑt͡s/ and a perfective converb with /-el/. } 

\begin{table}[H]
 \centering
 \caption{Participles or converbs <\armenian{դերբայներ}> for the verb `to like' in the Karabakh dialect}
 \label{tab:Karabakh:morpho:verb:paradigm:participle}
 \begin{tabular}{|ll|ll|ll|l|}
\hline & & \multicolumn{2}{l|}{Karabakh} & \multicolumn{2}{l|}{cf. SEA} & \\
 Infinitive& \armenian{անորոշ} & s\'iɾ-i-l & \armenian{սի՛րիլ} & siɾ-\'e-l & \armenian{սիրել} & $\sqrt{}$-{\thgloss}-{\infgloss} \\
 Present & \armenian{ներկայ} & s\'iɾ-oʁ & \armenian{սի՛րօղ} & siɾ-\'oʁ &\armenian{սիրող} & $\sqrt{}$-{\sptcp} \\
 Past & \armenian{անցեալ} & s\'iɾ-ɑl & \armenian{սի՛րալ} & siɾ-\'el & \armenian{սիրել} & $\sqrt{}$-{\perfcvb} \\
& & s\'iɾ-ɑt͡s & \armenian{սի՛րած} & siɾ-\'ɑt͡s & \armenian{սիրած} & $\sqrt{}$-{\rptcp} \\
 Future & \armenian{ապառնի} && & siɾ-e-l-u & \armenian{սիրելու} & $\sqrt{}$-{\thgloss}-{\infgloss}-{\futcvb} \\
 & & siɾ-ə-l-\'ɑkɑn & \armenian{սիրըլա՛կան} & & & $\sqrt{}$-{\thgloss}-{\infgloss}-? \\
 & & siɾ-ə-l-\'ɑt͡sʰukʰ & \armenian{սիրըլա՛ցուք} & & & $\sqrt{}$-{\thgloss}-{\infgloss}-? 
\\\hline \end{tabular}
\end{table}

\paragraph{Other Complex or periphrastic forms}

Besides these, there are many composite \armenian{բաղադրեալ} forms, which are formed with the participles and with auxiliaries. The following is a list. 

\subparagraph{Indicative mood}
\translatorHD{In the indicative mood (\armenian{սահմանական եղանակ}), Adjarian lists the following other periphrastic tenses: the present perfect, the past perfect, the definite future, and the definite future perfect. }

\translatorHD{In SEA, the present perfect and past perfect are formed by taking the perfective converb of a verb (suffixed with /-el/: Table \ref{tab:Karabakh:morpho:verb:paradigm:presentpastPerfect}), and then adding present or past auxiliaries. Karabakh shows essentially the same strategy. The verb uses a non-finite form with either the perfective suffix /-ɑl/ or the resultative suffix /-ɑt͡s/. }


\begin{table}[H]
 \centering
 \caption{1SG present perfect <\armenian{յարակատար}> and past perfect <\armenian{գերակատար}> of the verb `to like' in the Karabakh dialect}
 \label{tab:Karabakh:morpho:verb:paradigm:presentpastPerfect}
 \begin{tabular}{|l|ll|ll|l| }
\hline & \multicolumn{2}{l|}{Karabakh} & \multicolumn{2}{l|}{cf. SEA} & \\
Pres. & s\'iɾ-ɑl ə-m & \armenian{սի՛րալ ըմ} & siɾ-\'el e-m & \armenian{սիրել եմ} & {$\sqrt{}$-{\perfcvb} {\aux}-{\agr}}\\ 
 & s\'iɾ-ɑt͡s ə-m & \armenian{սի՛րած ըմ} & siɾ-\'el e-m & \armenian{սիրել եմ} & {$\sqrt{}$-{\rptcp} {\aux}-{\agr}}\\ 
Past. & s\'iɾ-ɑl i-$\emptyset$-$\emptyset$ & \armenian{սի՛րալ ի} & siɾ-\'el ej-i-$\emptyset$ & \armenian{սիրել էի} & {$\sqrt{}$-{\perfcvb} {\aux}-{\pst}-{\agr}}\\
 & s\'iɾ-ɑt͡s i-$\emptyset$-$\emptyset$ & \armenian{սի՛րած ի} & siɾ-\'el ej-i-$\emptyset$ & \armenian{սիրել էի} & {$\sqrt{}$-{\rptcp} {\aux}-{\pst}-{\agr}}\\

\hline 
\end{tabular}
\end{table}

\translatorHD{Adjarian likewise mentions the definite future and the definite future perfect. They're formed by taking the future participle and adding the present or past auxiliaries. }
 
 
\begin{table}[H]
 \centering
 \caption{1SG definite future <\armenian{որոշեալ ապառնի}> and definite future perfect <\armenian{որոշեալ ապառնի անցեալ}> of the verb `to like' in the Karabakh dialect}
 \label{tab:Karabakh:morpho:verb:paradigm:defFuture}
 \begin{tabular}{|l|ll|l| }
\hline Fut. & siɾ-ə-l-\'ɑkɑn ə-m & \armenian{սիրըլա՛կան ըմ} & {$\sqrt{}$-{\thgloss}-{\infgloss}-{\futcvb} {\aux}-1{\sg}}\\
& siɾ-ə-l-\'ɑt͡sʰukʰ ə-m & \armenian{սիրըլա՛ցուք ըմ}& {$\sqrt{}$-{\thgloss}-{\infgloss}-{\futcvb} {\aux}-1{\sg}}\\
Fut Perf. & siɾ-ə-l-\'ɑkɑn i-$\emptyset$-$\emptyset$ & \armenian{սիրըլա՛կան ի} & {$\sqrt{}$-{\perfcvb} {\aux}-{\pst}-1{\sg}} \\
& siɾ-ə-l-\'ɑt͡sʰukʰ i-$\emptyset$-$\emptyset$ & \armenian{սիրըլա՛ցուք ի} & {$\sqrt{}$-{\perfcvb} {\aux}-{\pst}-1{\sg}}\\
\hline 
\end{tabular}
\end{table} 

\subparagraph{Narrative mood}

\translatorHD{In the narrative mood (\armenian{պատմական եղանակ}), Adjarian briefly illustrates 6 possible systems. These systems are formed by taking a pre-existing periphrastic tense, and then adding a formative /əlæl/ <\armenian{ըլա̈լ}>, which is likely a cognate to the SWA verb /əllɑl/ `to be'.\footnote{\translatorHD{A possible segmentation for this formative is /əl-æ-l/ `$\sqrt{}$-{\thgloss}-{\infgloss}'. Unfortunately, Adjarian doesn't provide enough data. For safety, I give a simple segmentation and gloss as {\narr}. }} the paradigm that Adjarian provides The 6 new periphrastic systems are morphologically built from the 6 that we previously described: the indicative present, the indicative past imperfective, the present perfect, the past perfect, the definite future, the definite future perfect. }


 
\begin{table}[H]
 \centering
 \caption{1SG narrative mood <\armenian{պատմական եղանակ}> of the verb `to like' in the Karabakh dialect}
 \label{tab:Karabakh:morpho:verb:paradigm:narative}
 \begin{tabular}{|l|ll|l| }
\hline Pres. & s\'iɾ-əm ə-m əlæl & \armenian{սի՛րըմ ըմ ըլա̈լ} & {$\sqrt{}$-{\impfcvb} {\aux}-1{\sg} {\narr}}\\
 Past Impf. & s\'iɾ-əm i-$\emptyset$-$\emptyset$ əlæl & \armenian{սի՛րըմ ի ըլա̈լ} & {$\sqrt{}$-{\impfcvb} {\aux}-{\pst}-1{\sg} {\narr}}\\
 Def. Fut. & siɾ-ə-l-\'ɑkɑn ə-m əlæl & \armenian{սիրըլա՛կան ըմ ըլա̈լ} & {$\sqrt{}$-{\thgloss}-{\infgloss}-{\futcvb}? {\aux}-1{\sg} {\narr}} \\ 
 & siɾ-ə-l-\'ɑt͡sʰukʰ ə-m əlæl & \armenian{սիրըլա՛ցուք ըմ ըլա̈լ} & {$\sqrt{}$-{\thgloss}-{\infgloss}-{\futcvb}? {\aux}-1{\sg} {\narr}} \\ 
 Def. Fut. Perf. & siɾ-ə-l-\'ɑkɑn i-$\emptyset$-$\emptyset$ əlæl & \armenian{սիրըլա՛կան ի ըլա̈լ} & {$\sqrt{}$-{\thgloss}-{\infgloss}-{\futcvb}? {\aux}-{\pst}-1{\sg} {\narr}} \\ 
 & siɾ-ə-l-\'ɑt͡sʰukʰ i-$\emptyset$-$\emptyset$ əlæl & \armenian{սիրըլա՛ցուք ի ըլա̈լ} & {$\sqrt{}$-{\thgloss}-{\infgloss}-{\futcvb}? {\aux}-{\pst}-1{\sg} {\narr}} \\ 
 Pres. Perf. & s\'iɾ-ɑt͡s ə-m əlæl & \armenian{սի՛րած ըմ ըլա̈լ} & {$\sqrt{}$-{\rptcp} {\aux}-1{\sg} {\narr}}\\
 & s\'iɾ-ɑl ə-m əlæl & \armenian{սի՛րալ ըմ ըլա̈լ} & {$\sqrt{}$-{\perfcvb} {\aux}-1{\sg} {\narr}}\\
 Past Perf. & s\'iɾ-ɑt͡s i-$\emptyset$-$\emptyset$ əlæl & \armenian{սի՛րած ի ըլա̈լ} & {$\sqrt{}$-{\rptcp} {\aux}-{\pst}-1{\sg} {\narr}}\\
 & s\'iɾ-ɑl i-$\emptyset$-$\emptyset$ əlæl & \armenian{սի՛րալ ի ըլա̈լ} & {$\sqrt{}$-{\perfcvb} {\aux}-{\pst}-1{\sg} {\narr}}\\
\hline 
\end{tabular}
\end{table} 


\subparagraph{Debitive mood}

\translatorHD{In the debitive mood (\armenian{պարտաւորական եղանակ}), Adjarian briefly illustrates 8 possible systems. These systems are formed by taking a finite or non-finite form of the verb, and then adding a version of the debitive formative /piti/. In some cases, this formative takes agreement, thus suggesting that agreement morphology is mobile. These 8 systems are not straightforwardly built from pre-existing tenses. Instead, each systems to use its own set of rules (\ref{sent:Karabakh:morpho:verb:Debitive}). The rules below are my own conjectures based on Adjarian's list of forms. Adjarian doesn't state at all what the rules should be. }

\begin{exe}
 \ex Debitive forms in Karabakh \label{sent:Karabakh:morpho:verb:Debitive}
\begin{xlist}
 \ex Present \gll 
 p\'et-ə-m ɑ s\'iɾ-i-m \\ 
 {\deb}-?-1{\sg} {\aux}.{\prs}.3{\sg} like-{\thgloss}-1{\sg} \\
 \trans \armenian{պէ՛տըմ ա սի՛րիմ} \\Rule: The debitive present is formed by inflecting the debitive, adding the present 3SG auxiliary, and adding the inflected present subjunctive. 
 \ex Past imperfective \gll 
 p\'et-ə-m ɑ s\'iɾ-i-$\emptyset$-$\emptyset$, p\'et-ə-m i-$\emptyset$-$\emptyset$ s\'iɾ-i-$\emptyset$-$\emptyset$ \\ 
 {\deb}-?-1{\sg} {\aux}.{\prs}.3{\sg} like-{\thgloss}-{\pst}-1{\sg}, {\deb}-?-1{\sg} {\aux}-{\pst}-1{\sg} like-{\thgloss}-{\pst}-1{\sg} \\
 \trans \armenian{պէ՛տըմ ա սի՛րի, պէ՛տըմ ի սի՛րի} \\
 Rule: The debitive past imperfective is formed by inflecting the debitive, adding either the present 3SG auxiliary /ɑ/ or the past inflected auxiliary, and then adding the inflected past subjunctive. 
 \ex Future \gll 
 s\'iɾ-ɑt͡s pit-i-m \\ 
 like-{\rptcp} {\deb}-?-1{\sg}\\ 
 \trans \armenian{սի՛րած պիտիմ} \\
 Rule: The debitive future is formed by taking the resultative participle (past participle with /-ɑt͡s/), and then adding a present-inflected debitive marker /piti/. 
 \ex Future perfect \gll 
 s\'iɾ-ɑt͡s pit-i-$\emptyset$-$\emptyset$ \\ 
 like-{\rptcp} {\deb}-?-{\pst}-1{\sg} \\ 
 \trans \armenian{սի՛րած պիտի}\\
 Rule: The debitive future perfect is formed by taking the resultative participle (past participle with /-ɑt͡s/), and then adding a past-inflected debitive marker /piti/.
 \ex Narrative present \gll 
 s\'iɾ-ɑt͡s pit-i-m əlæl\\ 
 like-{\rptcp} {\deb}-?-1{\sg} {\narr}\\ 
 \trans \armenian{սի՛րած պիտիմ ըլա̈լ} \\
 Rule: The debitive narrative present is formed by taking the resultative participle (past participle with /-ɑt͡s/), adding a present-inflected debitive marker /piti/, and then ading the narrative marker /əlæl/. 
 \ex Narrative past imperfective \gll 
 s\'iɾ-ɑt͡s pit-i-$\emptyset$-$\emptyset$ əlæl \\ 
 like-{\rptcp} {\deb}-?-{\pst}-1{\sg} {\narr} \\ 
 \trans \armenian{սի՛րած պիտի ըլա̈լ}\\
 Rule: The debitive narative past imperfective is formed by taking the resultative participle (past participle with /-ɑt͡s/), adding a past-inflected debitive marker /piti/, and then ading the narrative marker /əlæl/.
 \ex Definite future \gll 
 siɾ-ə-l-\'ɑkɑn pit-i-m \\ 
 like-{\thgloss}-{\infgloss}-{\futcvb}? {\deb}-?-1{\sg} \\ 
 \trans \armenian{սիրըլա՛կան պի՛տիմ}\\
 Rule: The debitive definite future is formed by taking the future participle, and then adding a present-inflected debitive marker /piti/.
 \ex Definite future perfect \gll 
 siɾ-ə-l-\'ɑkɑn pit-i-$\emptyset$-$\emptyset$ \\ 
 like-{\thgloss}-{\infgloss}-{\futcvb}? {\deb}-?-{\pst}-1{\sg} \\ 
 \trans \armenian{սիրըլա՛կան պիտի}\\
 Rule: The debitive definite future perfect is formed by taking the future participle, and then adding a past-inflected debitive marker /piti/. 
 
\end{xlist}
 

\end{exe}

 

\subparagraph{Intensive mood}

\translatorHD{In the intensive mood (\armenian{սաստկական եղանակ}), Adjarian briefly illustrates 4 possible systems. These systems are formed by taking a finite or non-finite form. Adjarian doesn't explain the structure of such systems. I conjecture the following rules in (\ref{sent:Karabakh:morpho:verb:intensive}). }


\begin{exe}
 \ex Intensive forms in Karabakh \label{sent:Karabakh:morpho:verb:intensive}
\begin{xlist}
 \ex Present (version 1) \gll 
 s\'iɾ-ɑt͡s piti p\'it-i-m \\ 
 like-{\rptcp} {\deb} {\deb}-?-1{\sg} \\ 
\trans \armenian{սի՛րած պիտի պի՛տիմ}\\
 Rule: The intensive present is formed by taking the stressed resultative participle (past participle with /-ɑt͡s/), adding an uninflected debitive marker /piti/, and then adding a stressed present-inflected debitive marker /piti/. 
\ex Present (version 2) \gll 
 siɾ-ɑt͡s piti piti \'ini-m \\ 
 like-{\rptcp} {\deb} {\deb} {?}-1{\sg} \\ 
\trans \armenian{սիրած պիտի պիտի ի՛նիմ}\\
 Rule: The intensive present is formed by taking the unstressed resultative participle (past participle with /-ɑt͡s/), adding two instances of an uninflected debitive marker /piti/, and then a stressed present-inflected formative /ini-/.
\ex Past (version 1) \gll 
 s\'iɾ-ɑt͡s piti p\'it-i-$\emptyset$-$\emptyset$ \\ 
 like-{\rptcp} {\deb} {\deb}-?-{\pst}-1{\sg} \\ 
 \trans \armenian{սի՛րած պիտի պի՛տի}\\
 Rule: The intensive past is formed by taking the stressed resultative participle (past participle with /-ɑt͡s/), adding an uninflected debitive marker /piti/, and then adding a stressed past-inflected debitive marker /piti/. 
 \ex Past (version 2) \gll 
 siɾ-ɑt͡s piti piti \'ini-$\emptyset$-$\emptyset$ \\ 
 like-{\rptcp} {\deb} {\deb} {?}-{\pst}-1{\sg} \\ 
\trans \armenian{սիրած պիտի պիտի ի՛նի} \\
 Rule: The intensive past is formed by taking the unstressed resultative participle (past participle with /-ɑt͡s/), adding two instances of an uninflected debitive marker /piti/, and then a stressed past-inflected formative /ini-/.

 \ex Future (version 1) \gll 
 siɾ-ə-l-\'ɑkɑn piti p\'it-i-m \\ 
 like-{\thgloss}-{\infgloss}-{\futcvb}? {\deb} {\deb}-?-1{\sg}\\ 
 \trans \armenian{սիրըլա՛կան պիտի պի՛տիմ} \\
 Rule: The intensive future is formed by taking the stressed future participle, adding an uninflected debitive marker /piti/, and then adding a stressed present-inflected debitive marker /piti/. 
 \ex Future (version 2) \gll 
 siɾ-ə-l-\'ɑkɑn piti piti \'ini-$\emptyset$ \\ 
 like-{\thgloss}-{\infgloss}-{\futcvb}? {\deb} {\deb} {?}-1{\sg}\\ 
\trans \armenian{սիրըլա՛կան պիտի պիտի ի՛նիմ} \\
 Rule: The intensive future is formed by taking the stressed future participle, adding two instances of an uninflected debitive marker /piti/, and then a stressed present-inflected formative /ini-/. 
\ex Future perfect (version 1) \gll 
 siɾ-ə-l-\'ɑkɑn piti pit-i-$\emptyset$-$\emptyset$ \\ 
 like-{\thgloss}-{\infgloss}-{\futcvb}? {\deb} {\deb}-?-{\pst}-1{\sg} \\ 
 \trans \armenian{սիրըլա՛կան պիտի պի՛տի} \\
 Rule: The intensive future perfect is formed by taking the stressed future participle, adding an uninflected debitive marker /piti/, and then adding a stressed past-inflected debitive marker /piti/. 
 \ex Future perfect (version 2) \gll 
 siɾ-ə-l-\'ɑkɑn piti piti \'ini-$\emptyset$-$\emptyset$ \\ 
 like-{\thgloss}-{\infgloss}-{\futcvb}? {\deb} {\deb} {?}-{\pst}-1{\sg} \\ 
\trans \armenian{սիրըլա՛կան պիտի պիտի ի՛նի} \\
 Rule: The intensive future perfect is formed by taking the stressed future participle adding two instances of an uninflected debitive marker /piti/, and then a stressed past-inflected formative /ini-/. 

 
\end{xlist}
 

\end{exe}


\subparagraph{Immediate mood}

\translatorHD{In the immediate mood (\armenian{անմիջական եղանակ}), Adjarian briefly illustrates 4 possible systems. Adjarian doesn't explain the structure of such systems. I conjecture that the morphological strategy is to take the instrumental form of the verb (suffixed with /-ɑv/), and then use combinations of auxiliaries and narrative formatives. (\ref{sent:Karabakh:morpho:verb:Immediate}). }


\begin{exe}
 \ex Immediate forms in Karabakh \label{sent:Karabakh:morpho:verb:Immediate}
\begin{xlist}
 \ex Present \gll 
 siɾ-\'e-l-ɑv ə-m \\ 
 like-{\thgloss}-{\infgloss}-{\ins} {\aux}-1{\sg} \\ 
\trans \armenian{սիրէ՛լավ ըմ}\\
 Rule: The immediate present is formed by taking the instrumental form of the verb, and then adding the present auxiliary. 
 \ex Past imperfective \gll 
 siɾ-\'e-l-ɑv i-$\emptyset$-$\emptyset$ \\ 
 like-{\thgloss}-{\infgloss}-{\ins} {\aux}-{\pst}-1{\sg} \\ 
\trans \armenian{սիրէ՛լավ ի}\\
 Rule: The immediate past imperfective is formed by taking the instrumental form of the verb, and then adding the past auxiliary. 
 \ex Narrative present \gll 
 siɾ-\'e-l-ɑv ə-m əlæl \\ 
 like-{\thgloss}-{\infgloss}-{\ins} {\aux}-1{\sg} {\narr} \\ 
\trans \armenian{սիրէ՛լավ ըմ ըլա̈լ}\\
 Rule: The immediate narrative present is formed by taking the instrumental form of the verb, adding the present auxiliary, and then adding the narrative formative /əlæl/. 
 \ex Past imperfective \gll 
 siɾ-\'e-l-ɑv i-$\emptyset$-$\emptyset$ əlæl \\ 
 like-{\thgloss}-{\infgloss}-{\ins} {\aux}-{\pst}-1{\sg} {\narr} \\ 
\trans \armenian{սիրէ՛լավ ի ըլա̈լ}\\
 Rule: The immediate narrative past imperfective is formed by taking the instrumental form of the verb, adding the past auxiliary, and then adding the narrative formative /əlæl/. 
\end{xlist}

 
\end{exe}

\begin{adjarianpage}\label{page:70}\end{adjarianpage}% should be 70

\section{Subdialects}




The description that we give is for the main dialect of Karabakh. Its subdialects (Gandzak, Gazakh, and Karadagh) show some or many differences. Because they have not been researched or scientifically verified, it is impossible for me to determine in detail the limits or borders of these differences. I am sufficed with using only my passing familiarity.

\subsection{Gandzak}

The Gandzak subdialect is extremely close to the main Karabakh dialect, only that it has more purer forms, meaning forms that are closer to the old language. For example in verbal conjugation, the copular verb does not have forms with schwa /ə/ (like 2PL /ə-kʰ/ <\armenian{ըք}>, 3PL /ə-n/ <\armenian{ըն}>) but instead forms with /e/ (Table \ref{tab:Karabakh:subdialect:gandzak:e}). 

\begin{table}[H]
 \centering
 \caption{Use of /e/ instead of /ə/ in the copula for the Gandzak subdialect of the Karabakh dialect}
 \label{tab:Karabakh:subdialect:gandzak:e}
 \begin{tabular}{|l| ll|ll| ll|}
 \hline & \multicolumn{2}{l|}{General Karabakh} & \multicolumn{2}{l|}{cf. Gandzak subdialect} & \multicolumn{2}{l|}{cf. SEA} \\ 
1SG `I am' &ə-m& \armenian{ըմ} & e-m & \armenian{էմ} &e-m& \armenian{եմ} \\
2SG `you are' &ə-s& \armenian{ըս} & e-s & \armenian{էս} &e-s& \armenian{ես} \\
1PL `I am' &ə-nkʰ& \armenian{ընք} & e-nkʰ & \armenian{էննք} &e-ŋkʰ& \armenian{ենք} \\
 \hline 
 \end{tabular}
\end{table}

The Classical sound /i/ <\armenian{ի}> does not become /ə̟/ <\armenian{ըէ}> and it stays constant (Table \ref{tab:Karabakh:subdialect:gandzak:i}). 

\begin{table}[H]
 \centering
 \caption{Lack of the sound change from /i/ <\armenian{ի}> in the Gandzak subdialect of the Karabakh dialect}
 \label{tab:Karabakh:subdialect:gandzak:i}
 \begin{tabular}{|l| ll | ll|ll| ll|}
 \hline & \multicolumn{2}{l|}{Classical Armenian}& \multicolumn{2}{l|}{> General Karabakh} & \multicolumn{2}{l|}{cf. Gandzak subdialect} & \multicolumn{2}{l|}{cf. SEA} \\ 
`neck' &viz& \armenian{վիզ} &və̟z & \armenian{վըէզ} & viz & \armenian{վիզ} &viz& \armenian{վիզ} \\
`year' &tɑɾ\'i& \armenian{տարի} & t\'ɑɾe & \armenian{տա՛րէ} & t\'ɑɾi & \armenian{տա՛րի} &tɑɾ\'i& \armenian{տարի} \\
`nose' &kʰitʰ& \armenian{քիթ} & kʰetʰ & \armenian{քէթ} &kʰitʰ& \armenian{քիթ} &kʰitʰ& \armenian{քիթ} \\
 \hline 
 \end{tabular}
\end{table}


 
The ending /n/ <\armenian{ն}> of Old Armenian became /nə/ <\armenian{նը}> in Karabakh, but it became /ə/ <\armenian{ը}> in Gandzak (Table \ref{tab:Karabakh:subdialect:gandzak:n}). 

\begin{table}[H]
 \centering
 \caption{Change from final /(ə)n/ <\armenian{ն}> in Classical to /ə/ <\armenian{ը}> in the Gandzak subdialect of the Karabakh dialect}
 \label{tab:Karabakh:subdialect:gandzak:n}
 \begin{tabular}{|l| ll | ll|ll| ll|}
 \hline & \multicolumn{2}{l|}{Classical Armenian}& \multicolumn{2}{l|}{> General Karabakh} & \multicolumn{2}{l|}{cf. Gandzak subdialect} & \multicolumn{2}{l|}{cf. SEA} \\ 
`fish' &d͡z\'ukən & \armenian{ձուկն} & t͡s\'ʏknə & \armenian{ծի՛ւկնը} & t͡s\'ukə & u\armenian{ծո՛ւկը} & d͡z\'uk & \armenian{ձուկ} \\ 
 \hline 
 \end{tabular}
\end{table}
 


\subsection{Gazakh}

The Gazakh subdialect, as can be seen in my published writings, is much more closer to the Yerevan dialect. The ablative formative is /-it͡sʰ/ <\armenian{ից}>, instead of the Karabakh form /-ɑn/ <\armenian{ան}>. 

The past participle ends in the formative /-el/ <\armenian{էլ}> and not with /ɑl/ <\armenian{ալ}> (\ref{sent:Karabakh:subdialct:Gazakh:participle}). 

\begin{exe}
 \ex \label{sent:Karabakh:subdialct:Gazakh:participle}
 \begin{xlist}
 \ex \begin{xlist}
 \ex Karabakh \gll 
 jes ə-m əl-æl \\
 I {\aux}-1{\sg} be-{\perfcvb} \\
 \trans \armenian{յէս ըմ ըլա̈լ}
 \ex Gazakh \gll 
 jes e-m l-el \\
 I {\aux}-1{\sg} be-{\perfcvb} \\
 \trans \armenian{յէս էմ լէլ}
 \ex SEA \gll 
 jes e-m eʁ-el \\
 I {\aux}-1{\sg} be-{\perfcvb} \\
 \trans \armenian{՝}I have been.' \\
 \armenian{ես եմ եղել}
 \end{xlist}
 \ex \begin{xlist}
 \ex Karabakh \gll 
 ənɡ-ɑl ə-m \\ 
 fall-{\perfcvb} {\aux}-1{\sg}\\
 \trans \armenian{ընգալ ըմ}
 \ex Gazakh \gll 
 ənk-el e-m \\ 
 fall-{\perfcvb} {\aux}-1{\sg}\\
 \trans \armenian{ընկէլ էմ}
 \ex SEA \gll 
 ənk-el e-m \\ 
 fall-{\perfcvb} {\aux}-1{\sg}\\
 \trans `I have fallen.' \\
 \armenian{ընկել եմ}
 
 \end{xlist}
 \end{xlist}
\end{exe}

However, before stress, the basic rule of losing vowels continue to apply (Table \ref{tab:Karabakh:subdialect:gazakh:vowel} and \ref{sent:Karabakh:subdialct:Gazakh:vowelDel}). 

\begin{table}[H]
 \centering
 \caption{Pre-tonic vowel deletion in the Gazakh subdialect of the Karabakh dialect}
 \label{tab:Karabakh:subdialect:gazakh:vowel} 
 \begin{tabular}{|l| ll | ll| ll|}
 \hline & \multicolumn{2}{l|}{Classical Armenian}& \multicolumn{2}{l|}{> Gandzak subdialect} & \multicolumn{2}{l|}{cf. SEA} \\ 
\armenian{՝}barking at a beast' & & &kzn\'ɑhɑt͡ʃʰ & \armenian{կզնա՛հաչ}&ɡɑzɑnɑh\'ɑt͡ʃʰ & \armenian{գազանահաչ}\\ 
`children' & eɾeχ\'ɑi̯kʰ & \armenian{երեխայք}& ɾ\'eχekʰ & \armenian{րէ՛խէք}& jeɾeχ\'ekʰ & \armenian{երեխեք}\\ 
`to get accustomed' &sovoɾ\'il & \armenian{սովորիլ} & səv\'oɾil & \armenian{սըվօրիլ}& sovoɾ\'el & \armenian{սովորել} \\
 \hline 
 \end{tabular}
\end{table} 

(\ref{sent:Karabakh:subdialct:Gazakh:participle}). 

\begin{exe}
 \ex \label{sent:Karabakh:subdialct:Gazakh:vowelDel}
 \begin{xlist}
 \ex Karabakh \gll 
 əʃχ\'ɑɾkʰ-e-s əɾ\'es-e-n \\
 world-{\gen}-{\possFsg} face-{\dat}-{\defgloss} \\
 \trans \armenian{ըշխա՛րքէս ըրէ՛սէն}
 \ex Gazakh \gll 
 əʃχ\'ɑɾ-i-s əɾ\'is-i-n \\
 world-{\gen}-{\possFsg} face-{\dat}-{\defgloss} \\
 \trans \armenian{ըշխա՛րիս ըրի՛սին}
 \ex SEA \gll 
 ɑʃχɑɾ-\'i-s jeɾes-\'i-n \\
 world-{\gen}-{\possFsg} face-{\dat}-{\defgloss} \\
 \trans `one the face of my world.' (likely idiomatic for `in my life')
 \armenian{աշխարհիս երեսին}
 \end{xlist}
 
\end{exe}



The debitive form (1SG /piti-m/ <\armenian{պիտիմ}>, 2SG /piti-s/ <\armenian{պիտիս}>) is shortened to /dem, des, den/ <\armenian{դէմ, դէս, դէն}> (\ref{sent:Karabakh:subdialct:Gazakh:deb}). For example, 


\begin{exe}
 \ex Gazakh \label{sent:Karabakh:subdialct:Gazakh:deb}
 \begin{xlist}
 \ex \gll \'int͡ʃʰ dem səvoɾ-il \\
 what {\deb}.1{\sg} study-? \\
 \trans The meaning might be `What do I have to study!' \\
 \armenian{ի՛նչ դէմ սըվօրիլ}
 \ex \gll meɾ ɾeχəkʰ-n \'int͡ʃʰ \'œɾnɑk dem ve kɑlnil \\
 our? children-{\defgloss} what ? {\deb}.3{\pl} ? ? \\
 \trans I don't know what this sentence means, nor do I know what most of the words means. \\
 \armenian{մէր րէխէքն ի՛նչ է՛օրնակ դէմ վէ կալնիլ}.
 \end{xlist}
 
\end{exe}

\subsection{Karadagh}

The Karadagh subdialect has a wide distribution. At the north side of Azerbaijan, there is the large and heavily Armenian-populated province of Karadagh, which was was previously Paytakaran. Besides that, the subdialect is also spoken in the Armenian-populated village of Mujumbar (close to Tabriz), and in the Armenian populace of the Lilava district of Tabriz, from which... 


\begin{adjarianpage}\label{page:71}\end{adjarianpage}% should be 71

... the Armenian settlements of Karadagh and Mujumbar were formed. This subdialect is very close to the Karabakh dialect. In this subdialect, we find the following:\begin{itemize}
 \item The stress change.
 \item The loss of pre-stress vowels (\translatorHD{= pre-tonic vowels}).
 \item The change from Classical /o/ <\armenian{ո}> to /o,œ/ <\armenian{օ,էօ}>.
 \item The change from Classical and /u/ <\armenian{ու}> to /o,ʏ/ <\armenian{օ,իւ}>. 
 \item The change from Classical /n/ <\armenian{ն}> to /nə/ <\armenian{նը}>. 
 \item The use of the past participle suffix \armenian{ալ} /ɑl/. 
\end{itemize}

The Karadagh dialect however did not change the Old Armenian voiced consonants to voiceless ones; they stayed voiced. 

\section{Literature}

The Karabakh dialect was studied first by Patkanian \citep[55-73]{Patkanian-1869-RussianDialects}, then some small pieces of information in Makar Barkhudariants's ``Pele Pughi'' (\armenian{Մ. Վ. Բարխուդարեանցի Պըլը-Պուղի}) and in Karapet Shahnazariants's work (\armenian{Կ. Մ. Շահնազարեանցի Ղըլըցէ կնանոց պընը փէշակը}; \translatorHD{SEA: /kɑɾɑpet ʃɑhnɑzɑɾjɑnt͡sʰ/}). The last time there was a detailed study was my own work was published \citep{Adjarian-1901-Kharabagh}. Of my work, the Armenologist Meillet wrote a review (Journal Asiatique, 1902, page 561-571),\footnote{\translatorHD{It wasn't clear to me how to cite this review, but I tracked down a URL: \url{https://babel.hathitrust.org/cgi/pt?id=mdp.39015024511522&view=1up&seq=623}. }} where he discusses all the interesting and phonologically-interesting points of this Karabakh dialect. 

The following works are written in the Karabakh dialect:

{\litoverview}

\begin{itemize}
 \item Literature with the Karabakh dialect
 \begin{itemize}
 \item In the main Karabakh dialect 
 \begin{itemize}
 \item \armenian{Մակար Վրդ. Բարխուդարեանց – Պըլը-Պուղի, Թիֆլիս}, 1883.
\item \armenian{Ոստա}X \armenian{Գէորգ Բարխուդարեանց} 
\begin{itemize}
 \item – \armenian{Արաղը տարին կտարի. Շուշի}, 1883.
 \item – \armenian{Չոնբանն ու նշանածը. Թիֆլիս}, 1896.
 \item – \armenian{Բաբոյական առածներ. Թիֆլիս}, 1898.
\end{itemize}
\item \armenian{Շիրմազանեան Գ}. – \armenian{Ասրի-րէգ եւ Գիւքի. Կռունկ}, 1862, page 896-930, 1863, page 113-137
\item \armenian{Կ. Մէլիք-Շահնազարեան – Ջուռնա-ամբլա}, 2 \armenian{հատոր. Վազարշապատ}, 1907-08
\item \armenian{Տիգօ – Ղալի աղաթներան պատկերներ. Ճպատըդ քօլադ կարի. Թիֆլիս}, 1889
\item \armenian{Ե. Լալայեան} 
\begin{itemize}
 \item – \armenian{Ժողովրդական երգեր (Գորիսի}). \armenian{Ազգ. Հանդ. Գ}. page 261-270
\item – \armenian{Ժողովրդական երգեր (Ջանդեղուրի}). \armenian{Ազգ. Հանդ. Դ}. page 113-116
 \end{itemize}
 \end{itemize}
\item Gazakh subdialect

\begin{itemize}
 \item \armenian{Տէր-Դաւթեան Դ. Փաստաբանի մօտ (վօդըվիլ}). \armenian{Թիֆլիս}, 1901
\item \armenian{Ճուզուրեան Յ}. 

\begin{itemize}
 \item – \armenian{Մորացուած աշխարհ}. 3 \armenian{հտ. Թիֆլիս}, 1895-6

\begin{adjarianpage}\label{page:72}\end{adjarianpage}% should be 72
\item – \armenian{Աղքատի հալը}
 \item – \armenian{Գիւղի այրին}
\end{itemize}

\end{itemize}

\item Gandzak subdialect
\begin{itemize}
 \item \armenian{Ե. Լալայեան – Բանաւոր գրականութիւն. Ազգ. Հանգ}., \armenian{Ջ}, page 372-382
\item \armenian{Ս. Աւետիքեան} 
\begin{itemize}
 \item – \armenian{Սամիտարնի. Թիֆլիս}, 1897
\item – \armenian{Նահատակ. Թիֆլիս}, 1898
\end{itemize}
\end{itemize}

\item Karadagh subdialect
\begin{itemize}
 \item \armenian{Ղազարեան Յ}. – \armenian{Մանկական բեմ. Թիֆլիս}, 1900
\armenian{Ս. Անգրէասեան – Առածներ. Բիւրակն}, 1898, page 460-461

\end{itemize}

\end{itemize}
\end{itemize}

\section{Text samples}

{\sampleoverview}
 
\subsection{Karabakh dialect} 


– \armenian{Պարի աճօ՛զում, ա՛պրէս, հըշ տըղա՞ն ըս կյամ։}

– \armenian{Ըստուծէ՛ն պա՛րին. Նէ՛րքէ շէ՛նան ըմ կյամ։}

– \armenian{Հինչո՞ւ մըհար իր քէ՛ցալ ընդէղ։}

– \armenian{Պէն օ՛նի. քէ՛ցալ ի ըխճըկա՛նըս ա՛կը տամ։}

– \armenian{Խէ՞, ա՛խճիկըտ ընդէղ ըս հըղէ ըրա՛լ։}

– \armenian{Բա հա̈՛լա̈ նօր ըս գի՛ւդում, է՛րկու տըռնան (տարիէն) ի՛վիլ ա վրէր Նըրքը-Շընա՛ցա մին մա՛րթու-յ-ըմ տըվալ։}

– \armenian{Է՜, փըսէտ հա՞վան ը՜ս, լա՞վ տըղա-յ-ա՜. ըխճըկանըտ լա՞վ ա մըղա՛ձիտ կէնը՜մ։}

– \armenian{Խէ՞ չի. Ըստուծա՛նա շընուրհա՛կալ ըմ. տօնը շէն, ա՛մբարը ցօ՛րնավ լի՛գյը, կյիւմը տըվա՛րավ լի՛գյը, վըէ՛խճարը սիւրիւ-սի՛ւրիւ կա՛գնած, ճօխտ ճօխտ ճըղըցնէն (ջրաղացներ) պէ՛նէ (կը բանի}). \armenian{ի՛նքյն էլ լա̈ վ բօ՛յավ բուսա՛թավ, վըէր յէ՛շըմ ըս քէ՛փըտ կյա̈ մ ա։}

– \armenian{Դէ վըէր ըտի՛-յ-տ (այդպէս է}), \armenian{լա̈ վ ա. Ա՛ստուծ է՛լ ի՛վիլ ա՛նէ։}

\subsection{Karadagh subdialect} 

Adjarian's source: communicated by the Karadaghian, Ms. S. Ter-Martirosian (\armenian{օր. Ս. Տէր-Մարտիրոսեան}; \translatorHD{SEA: /teɾ mɑɾtiɾosjɑn/}). 

\armenian{Վա՛նէսը ո՛ւրան կյա՛նքըմը ժամ չը՛լալ գէ՛ցած. գի՛դալ չըլալ խօստօ՛վանքը հա՛ղօրթը հի՞նչ ա։ Գէնցալ ա ո՛ւրան մին ծանօ՛թից հըցրալ ա թա «իս մըտքը՛մըս դրամ ըմ, վօր խօստօվա՛նվըմ ու հաղօ՛րթվըմ հի՛նչուր կա՛րըն»։ Էն ալ ասալ ա թա «հինչ ուր կա՛րըս, կի գինըս ժամ, քահա՛նան քի հինչ վօր կա՛սի ՝ դու ալ էն կա՛սըս»։ Վա՛նէսը գինէ՛ցալ ա ժամ, ա՛սալ ա. «Ա՜ դէր, ինձ} ... 

\begin{adjarianpage}\label{page:73}\end{adjarianpage}% should be 73

... \armenian{խօստօվանցրու վօր պտք ա հաղորթվում»։ Քահա՛նան ա՛սալ ա. «վօ՛րթի չօ՛քի». Վա՛նէսն ալ ա՛սալ ա «վօ՛րթի չօ՛քի»։ Քահա՛նան ա՛սալ ա «իրէ՛սէդ խաչակընքի. ա՛սի մէ՛ղա Աստծու». էն ալ ա՛սալ ա «իրէ՛սէդ խաչակընքի, ա՛սի մէ՛ղա Աստծու»։ Քահա՛նան ա՛սալ ա «հի՛նչ գուզէ՛թուն վօր ա՛րած ըս՝ ա՛սի»։ Վա՛նէսն ալ ա՛սալ ա «հի՛նչ գուզէ՛թուն վօր ա՛րած ըս ՝ ա՛սի»։ Էն սհա՛թէն Վա՛նէսը ձէ՛ռքը թա՛քուն տա՛րալ ա դէ՛րէն ջուրը, սհա՛թը հա՛նալ ա։ Քահա՛նան տէ՛սալ ա վօր հինչ վօր ըսի՛լիս ա՝ էն ալ ըսի՛լիս, սկսալ ա Վա՛նէսէ թա՛կիլ։ Վա՛նէսն ալ դէ՛րէն ա թա՛կալ։ Վա՛նէսը փա՛խալ ա քուչան, մա՛րթըրը տէ՛սալ ըն, հրցրալ. «Ա՜ մարթ, խէ՞-յ-ըս փխչի՛լիս»}. – \armenian{«Ախր խօստովա՛նված ըմ». ի՛նդի ի՛լալ գիդա՛լիս վօր խօստօվա՛նվողը կը փա՛խչի։ Յէ՛տէն քահա՛նան ձէ՛ռքը տա՛րալ ա ջո՛ւբը՝ սհա՛թէն յէշի՛լու, տէ՛սալ ա վօր ջուբո՛ւմը չի։ «Է՜յ անիծած, խօստօվա՛նքէն թա՛հրը (Թրք. կերպը) գիդա՛լիս չի, սհաթս ալ գուղա՛ցալ ա՝ տա՛րայ»։}

\subsection{Gazakh subdialect} 

Adjarian's source: Taken from \armenian{Ճուղուրեանի Մոռացուած աշխարհէն. հտ. Ա, էջ} 103-4. Unfortunately, it does not have scientific accuracy. 

\armenian{Էրկու ախպէր էն լըմ, մնի անըմը Կայան, մէկէլինը Աբէլ։ Կայանը շա՛տ օցի կծածն ա ըլըմ, հօրը, մօրն ու ախպօրն ըսկի՛ սիրէլիս չի ըլըմ։ Մը հէտ էրկու ախպէրն էլ ուզըմ էն Ասծուն մատաղ անէն։ Օց Կայանը ըռանչպար ա ըլըմ, Աբէլը չօբան։ Նրանք էլ մէզ պէս՝ չօռի ու ցավի ժամանակ Ասծուն միտնէրն էն քըցըմ՝ ուզըմ էն մտղանա տալ}…\armenian{։ Ասծու օխնած Աբէլը վօչխարը բէրըմ ա դիւզըմը կըղնըցընըմ, երիսին խաչ քաշըմ ու վօչխարի միջից՝ սրտալի մի թուխը-սախար վօչխար ա ջօկըմ, ձէռաց վէ քըցըմ մօրթէն, փէտ ու կրակ անըմ սկսըմ խրօվէլ։ Օց Կայանը, Ա՛ստօծ անըծի նրան, կալը նօր քամած՝ ցօրէնը կիտած ա ըլըմ. սա վէր ա կալնըմ՝ անսիրտ-անսիրտ ցօրէնը խախալըմ, տակ ու գլուխը բէրըմ մատաղանա տալի ու ինքն էլ փէտ ու կրակ անըմ իր առածը խրօվըմ։ Էրկուսի կրակն էլ մի դիւզ տէղ ա ըլըմ՝ բացր յէրկնքի տակ։}

\armenian{Աստօծ մտիկ ա տալի տէնըմ, վօր Աբէլը սրտով տըվէց՝ նրա մատաղի կրակի ծուխը ծլլի (ուղիղ) ա բացրանըմ, նրա մատաղն ընթունէլի ա անըմ։}


\begin{adjarianpage}\label{page:74}\end{adjarianpage}% should be 74

\armenian{Պռօթօղ-մռօթօղ, տակ ու գլուխ տվօղ Կայանի վրա Աստվածաբար չրանըմ ա ու նրա ցօրէնի հասկէրն անըծըմ, թէ «Թօղ քու ցօրէնը մի հասկանի մնա»։}

\armenian{Կայանը տէնալով վօր Աստօծ Աբէլի ղօլը պահէց, մի օր՝ ԱԲէլոին՝ վօչխարի մէչ շվաքըմը խօրը քնած վախտը՝ վէր ա կալնըմ կօվըզիզիկը (սեւ սուր քար}), \armenian{բուգը դուս ա կարըմ ու իրան չօմբախօվը կրկաժի կիսին տալի՝ ղուդը շաղ ա տալի հա՛ սպանը՜մ։ Կայանն անպատիժ չի մնըմ։ – Աստօծ նրան շաշվացնըմ ա սարէրն ու հանդէրը քցըմ}…\armenian{։}

\subsection{Gandzak subdialect} 

Adjarian's source: See \armenian{Ազգ. Հանդ. Զ, էջ} 372 \armenian{եւն}. I have kept the orthography unchanged, even if very inaccurate. 




\armenian{Մածուն եմ}{\footnote{\translatorHD{The original page had <\armenian{են}>. But on page 306, Adjarian provides an erratum that this should be <\armenian{եմ}>.}} \armenian{մերել սրալի},

\armenian{Եար ըմ փռնել սրտալի},

\armenian{Հով որ եարը եարէն հանի},

\armenian{Հոքին սընտանէն տանի։}

~\\

\armenian{Մատումս կայ մըտանի},

\armenian{Համ թառ ա համ ծիրանի},

\armenian{Արի քինանք դիւանը}

\armenian{Հով սիրել ա, նա՛ տանի։}

~\\

\armenian{Մեր վճխարին եաթաղը},

\armenian{Գլխիս դնեմ փափախը}.

\armenian{Զանդուրուկդ ետ քցես}

\armenian{Ըրեսիս տայ շատաղը։}

~\\

\armenian{Մտիկ ըրէք էն ղուշին},

\armenian{Վէտը տրէլ ա փուշին},

\armenian{Եըրանի ընդիւր կի՛լի}

\armenian{Թուշի տրել ա սրածին։}


~\\

\armenian{Ջուրս կէծ ա, ջուր պիրէք},

\armenian{Սպըհանայ հիւն պիրէք},

\armenian{Տուն հաւասար թամամ են},

\armenian{Ղարիբ խուշը տուն պիրէք։}



\begin{adjarianpage}\label{page:75}\end{adjarianpage}% should be 75

\armenian{Սազին կոթին կիր կանեմ},

\armenian{Վենը կաց գինի բերեմ},

\armenian{Ես էս պիծի տեզաւս},

\armenian{Ցրուած խելքդ տոն պիրեմ։}

~\\

\armenian{Սրերի խինձը ի՛նչ անի},

\armenian{Կաթնով բրինձը ի՛նչ անի},

\armenian{Իւրիւր սիրել ենք կառնենք},

\armenian{Ուրըշի խօսք ու զրիցն ինչ անի։}

~\\

\armenian{Տիւ կաղնել ես կիւթանիա},

\armenian{Շիրմըշըղի պերանիա},

\armenian{Տիւ պարակ, մէջքդ պարակ},

\armenian{Ղօլ քընամ ջէյրան ջանիդ։}

\subsection{Zangezur subdialect} 

Adjarian's source: See \armenian{Ազգ. Հանդ}., \armenian{Դ. էջ} 115. 

\armenian{Դարուազին տակը սառալ ա}

\armenian{Մատներս ղալամ դարալ ա}

\armenian{Թոխ եախաւոր, բոզ չուխաւոր}

\armenian{Օշս խելքս տարալ ա։}

~\\ 

\armenian{Եկէք գնանք ծեր անենք},

\armenian{Խնձոր կծենք թոլ անենք},

\armenian{Հուր որ մին եար չունի՝}

\armenian{Գլխին թխենք դէն անենք։}

\armenian{Գեանջու քամին կալիս է},

\armenian{Դռները թըրըխկալիս է},

\armenian{Հարիւր թիւման մըշտըլըղ},

\armenian{Սիմօն եարը կալիս է։}

~\\ 


\armenian{Վըրթիվերիս վադան ա},

\armenian{Միջի մարդին կադան ա},

\armenian{Միջի մարդը հո՛ւնց անի},

\armenian{Խնձորկեցի կադան ա։}



\chapter{Shamakhi}


\begin{adjarianpage}\label{page:76}\end{adjarianpage}% should be 76

\section{Background}
This dialect is spoken primarily in the city of Shamakhi and its surrounding villages, until Quba. The remaining villages are mostly from Karabakh, and some are settlements from Khoy and Salmast; they thus speak their native dialect and they aren't included in this region. In Baku also, there is a migrant community from Shamakhi, and this community uses its dialect, but this community is dissolving into the larger Karabakh community. Near Baku there is Ermenikend (Armenian village), which is mostly made up of Shamakhi people and they speak this dialect. 

The Shamakhi dialect forms a middle zone between the Karabakh and Julfa dialects. Its phonetic system and phonological changes are largely the same as in the Karabakh dialect, as well as many of the grammatical forms. Because of this, it would have been possible to not consider Shamakhi as its own dialect, but to have treated it as a subdialect of Karabakh. However we are forced to treat it as its own independent dialect because of the diverse forms for pronouns and because of the formation of the present, both of which are entirely different from Karabakh. 

\section{Phonology}

\subsection{Segment inventory}
The phonetic system of the Shamakhi dialect is the same as for the Karabakh dialect. The dialect is missing only the diphthongs and the sound /ə̟/ <\armenian{ըէ}>. The sound /f/ <\armenian{ֆ}> is widely used here in borrowed words. The sound /hʲ/ <\armenian{հյ}> is missing. 

\subsection{Stress}
Unlike the Karabakh dialect, stress is on the last syllable. 

\translatorHD{This is an incorrect overgeneralization. As we see in \S\ref{sec:Shamakhi:Phonology:change:vowel:harmonyschwa}, when a final unstressed schwa undergoes vowel harmony, its harmonized vowel like /i/ stays unstressed. The correct generalization is that stress is final in the word before the definite/possessive suffixes. In this way, Shamakhi has morphologized the otherwise phonologically predictable rule of final stress assignment of SEA. }
\subsection{Sound changes}

\subsubsection{Vowel changes}
The following vowel changes and diphthong changes are notable. 


\paragraph{Classical Armenian /u/ <\armenian{ու}>}

Classical Armenian /u/ <\armenian{ու}> became Shamakhi /ʏ/ <\armenian{իւ}> as in Table \ref{tab:Shamakhi:phonology:soundChange:vowel:u:ʏ}. 

\begin{table}[H]
 \centering
 \caption{Change from Classical Armenian /u/ <\armenian{ու}> to /ʏ/ <\armenian{իւ}> in the Shamakhi dialect}
 \label{tab:Shamakhi:phonology:soundChange:vowel:u:ʏ}
 \begin{tabular}{|l| ll|ll| ll|}
 \hline & \multicolumn{2}{l|}{Classical Armenian} &\multicolumn{2}{l|}{> Shamakhi} & \multicolumn{2}{l|}{cf. SEA} \\ 
`you ({\nom})' & du & \armenian{դու} & tʏ & \armenian{տիւ} & du & \armenian{դու} \\ 
`outside' & duɾs & \armenian{դուրս} & tʏs & \armenian{տիւս} & duɾs & \armenian{դուրս} \\ 
`tongue' & lezu & \armenian{լեզու} & lʏzʏ & \armenian{լիւզիւ} & lezu & \armenian{լեզու} \\ 
 \hline 
 \end{tabular}
\end{table}



\paragraph{Classical Armenian /oi̯/ <\armenian{ոյ}>}

Classical Armenian /oi̯/ <\armenian{ոյ}> became Shamakhi /ʏ/ <\armenian{իւ}> as in Table \ref{tab:Shamakhi:phonology:soundChange:vowel:oi̯:ʏ}. 

\begin{table}[H]
 \centering
 \caption{Change from Classical Armenian /u/ <\armenian{ոյ}> to /ʏ/ <\armenian{իւ}> in the Shamakhi dialect}
 \label{tab:Shamakhi:phonology:soundChange:vowel:oi̯:ʏ}
 \begin{tabular}{|l| ll|ll| ll|}
 \hline & \multicolumn{2}{l|}{Classical Armenian} &\multicolumn{2}{l|}{> Shamakhi} & \multicolumn{2}{l|}{cf. SEA} \\ 
`light' & loi̯s & \armenian{լոյս} & lʏs & \armenian{լիւս} & lujs & \armenian{լույս} \\ 
`sister' & kʰoi̯ɾ & \armenian{քոյր} & kʰʏɾ & \armenian{քիւր} & kʰujɾ & \armenian{քույր} \\ 
`blue' & kɑpoi̯t & \armenian{կապոյտ} & kʲæpʏt & \armenian{կյա̈պիւտ} & kɑpujt & \armenian{կապույտ} \\ 
 \hline 
 \end{tabular}
\end{table}


\begin{adjarianpage}\label{page:77}\end{adjarianpage}% should be 77



\paragraph{Classical Armenian /ɑi̯/ <\armenian{այ}>}

Classical Armenian /ɑi̯/ <\armenian{այ}> became Shamakhi /ɑ, æ, e/ <\armenian{ա, ա}̈, \armenian{է}> as in Table \ref{tab:Shamakhi:phonology:soundChange:vowel:ɑi̯:stuff}. 

\begin{table}[H]
 \centering
 \caption{Change from Classical Armenian /ɑi̯/ <\armenian{այ}> to /ɑ, æ, e/ <\armenian{ա, ա}̈, \armenian{է}> in the Shamakhi dialect}
 \label{tab:Shamakhi:phonology:soundChange:vowel:ɑi̯:stuff}
 \begin{tabular}{|l| ll|ll| ll|}
 \hline & \multicolumn{2}{l|}{Classical Armenian} &\multicolumn{2}{l|}{> Shamakhi} & \multicolumn{2}{l|}{cf. SEA} \\ 
`this' & ɑi̯s & \armenian{այս} & æs & \armenian{ա̈ս} & ɑjs & \armenian{այս} \\ 
`long' & eɾkɑi̯n & \armenian{երկայն} & eɾɡɑn & \armenian{էրգան} & jeɾkɑjn & \armenian{երկայն} \\ 
`father' & hɑi̯ɾ & \armenian{հայր} & heɾ & \armenian{հէր} & hɑjɾ & \armenian{հայր} \\ 
`mother' & mɑi̯ɾ & \armenian{մայր} & meɾ & \armenian{մէր} & mɑjɾ & \armenian{մայր} \\ 
 \hline 
 \end{tabular}
\end{table}



\paragraph{Classical Armenian /i/ <\armenian{այ}>}

Classical Armenian /i/ <\armenian{ի}> became Shamakhi /ə/ <\armenian{ը}> as in Table \ref{tab:Shamakhi:phonology:soundChange:vowel:i:ə}. 

\begin{table}[H]
 \centering
 \caption{Change from Classical Armenian /i/ <\armenian{ի}> to /ə/ <\armenian{ը}> in the Shamakhi dialect}
 \label{tab:Shamakhi:phonology:soundChange:vowel:i:ə}
 \begin{tabular}{|l| ll|ll| ll|}
 \hline & \multicolumn{2}{l|}{Classical Armenian} &\multicolumn{2}{l|}{> Shamakhi} & \multicolumn{2}{l|}{cf. SEA} \\ 
`heart' & siɾt & \armenian{սիրտ} & səɾt & \armenian{սըրտ} & siɾt & \armenian{սիրտ} \\ 
`mind' & mit-(ə?)kʰ (-{\pl}) & \armenian{միտք} & mətk & \armenian{մըտկ} & mitkʰ & \armenian{միտք} \\ 
 \hline 
 \end{tabular}
\end{table}

\paragraph{Vowel harmony of the schwa /ə/}\label{sec:Shamakhi:Phonology:change:vowel:harmonyschwa}
The sound /ə/ <\armenian{ը}> usually keeps its presence next to heavy vowel. But next to soft vowels, it turns to /i/ <\armenian{ի}> (Table \ref{tab:Shamakhi:phonology:soundChange:vowel:harmony}).\footnote{\translatorHD{It's not obvious to me what is the soft vs. hard distinction in vowels. I suspect he means that soft vowels are front vowels. }}

\translatorHD{Note that the wording implies that Adjarian is treating the original schwa as present in Classical Armenian. However, many of his examples involve the definite article which was /-n, ən/ in Classical Armenian but /-n, -ə/ in SEA. This article marks definiteness in SEA, but it marked distal deixis in CA. I suspect that he's actually reconstructing the harmonized schwa from a shared ancestor between Shamakhi and SEA instead of CA. }
 

\begin{table}[H]
 \centering
 \caption{Vowel harmony in the change from /ə/ <\armenian{ը}> to /i/ <\armenian{ի}> in the Shamakhi dialect}
 \label{tab:Shamakhi:phonology:soundChange:vowel:harmony}
 \begin{tabular}{|l|ll|ll| ll|}
 \hline &\multicolumn{2}{l|}{Classical Armenian}&\multicolumn{2}{l|}{> Shamakhi} & \multicolumn{2}{l|}{cf. SEA} \\ 
`girl-{\defgloss}' & ɑɬd͡ʒ\'ik-ən & \armenian{աղջիկն} & ɑχt͡ʃʰ\'iɡ-i & \armenian{ախչի՛գի} & ɑχt͡ʃʰ\'ik-ə & \armenian{աղջիկը} \\ 
`to listen' & & & mitiɡ ɑnil & \armenian{միտիգ անիլ}& mətik ɑnel & \armenian{մտիկ անել} \\ 
`neck-{\dimgloss}-{\possSsg}' & & & viz-\'ik-is & \armenian{վիզի՛գիս}& vəz-\'ik-əs & \armenian{վզիկս} \\ 
`he-{\defgloss}' & \'inkʰ-ən& \armenian{ինքն}& \'inkʰ-i & \armenian{ի՛նքի}&\'iŋkʰ-ə & \armenian{ինքը} \\ 
 \hline 
 \end{tabular}
\end{table}


\paragraph{Vowel lengthening of /o/}

The Shamakhi dialect has a special sound change where Classical /ɑu̯/ <\armenian{աւ}> becomes /o/ <\armenian{օ}> before the Classical vowels /o, ɑu̯, ō/ <\armenian{ո, աւ, օ}>. \translatorHD{To clarify, he means that if the Classical form had a substring <\armenian{աւո}> /ɑ{wo}/ (which he treats as after a diphthong /ɑu̯o/), this string became /oo/ in Shamakhi. In contrast in SEA, such a string became /ɑvo/. I suspect his transcription <\armenian{օօ}> /oo/ signifies a long vowel /oː/ but I'm not sure.}


It becomes /œo/ <\armenian{էօօ}> when next to soft vowels. \translatorHD{I assume his transcription <\armenian{էօօ}> /eoo/ is actually /œo/. I suspect this is a diphthong but I'm unsure. }



\begin{table}[H]
 \centering
 \caption{Vowel lengthening and fronting in the change from Classical Armenian /ɑ{wX}/ <\armenian{աւ}V> to /oo, œo/ <\armenian{օօ, էօօ}> in the Shamakhi dialect}
 \label{tab:Shamakhi:phonology:soundChange:vowel:olength}
 \begin{tabular}{|l|ll|ll| ll|}
 \hline &\multicolumn{2}{l|}{Classical Armenian}&\multicolumn{2}{l|}{> Shamakhi} & \multicolumn{2}{l|}{cf. SEA} \\ 
`necessary' & hɑɾkɑ{wo}ɾ & \armenian{հարկաւոր} & hɑɾkooɾ & \armenian{հարկօօր} & hɑɾkɑvoɾ & \armenian{հարկավոր} \\ 
`baptized' & kənkʰɑ{wo}ɾ & \armenian{կնքաւոր} & kənkʰooɾ & \armenian{կնքօօր} & kəŋkʰɑvoɾ & \armenian{կնքավոր} \\ 
`graceful (CA), & ʃənoɾhɑ{wo}ɾ & \armenian{շնորհաւոր} & ʃənooɾ & \armenian{շընօօր} & ʃənoɾɑvoɾ & \armenian{շնորհավոր} \\ 
 congratulations (SEA)'& & & &\\
`morning' & ɑrɑ{wɑ}u̯t & \armenian{առաւաւտ} & ɑroot & \armenian{առօօտ} & ɑrɑvot & \armenian{առավոտ} \\ 
`to be housed' & & & tnoorvil & \armenian{տնօօրվիլ} & tənɑvoɾvel & \armenian{տնավորվել} \\ 
`knife-bearing' &dɑnɑkɑ{wo}ɾ & \armenian{դանակաւոր} & tænæɡʲœoɾ & \armenian{տա̈նա̈գյէօօր} & dɑnɑkɑvoɾ & \armenian{դանակավոր} \\ 
 \hline 
 \end{tabular}
\end{table}

\paragraph{Pre-tonic vowel deletion}

The deletion of vowels before stress is not a general rule. But there are a few cases (Table \ref{tab:Shamakhi:phonology:soundChange:vowel:del}). 



\begin{table}[H]
 \centering
 \caption{Pre-tonic vowel deletion in the Shamakhi dialect}
 \label{tab:Shamakhi:phonology:soundChange:vowel:del}
 \begin{tabular}{|l|ll|ll| ll|}
 \hline &\multicolumn{2}{l|}{Classical Armenian}&\multicolumn{2}{l|}{> Shamakhi} & \multicolumn{2}{l|}{cf. SEA} \\ 
`to crush' & d͡ʒɑχd͡ʒɑχel & \armenian{ջախջախել} & t͡ʃət͡ʃeχel & \armenian{ճըճէխէլ} & d͡ʒɑχd͡ʒɑχel & \armenian{ջախջախել} \\ 
`Satan' & sɑtɑnɑi̯ & \armenian{սատանայ} & stɑnɑ & \armenian{ստանա} & sɑtɑnɑi & \armenian{սատանա} \\ 
`to wrap up' & pʰɑtʰɑtʰel & \armenian{փաթաթել} & pʰtʰɑtʰel & \armenian{փթաթէլ} & pʰɑtʰɑtʰel & \armenian{փաթաթել} \\ 
 \hline 
 \end{tabular}
\end{table}


In this case, the loss of a vowel has caused the rise of a schwa /ə/ <\armenian{ը}>, and sometimes it assimilates to the form of the stressed vowel (Table \ref{tab:Shamakhi:phonology:soundChange:vowel:delepenth}). \translatorHD{I don't see how Adjarian's examples relate to his proposed process of schwa epenthesis and vowel harmony. }



\begin{table}[H]
 \centering
 \caption{Pre-tonic vowel deletion feeds vowel epenthesis and harmony in the Shamakhi dialect}
 \label{tab:Shamakhi:phonology:soundChange:vowel:delepenth}
 \begin{tabular}{|l|ll|ll| ll|}
 \hline &\multicolumn{2}{l|}{Classical Armenian}&\multicolumn{2}{l|}{> Shamakhi} & \multicolumn{2}{l|}{cf. SEA} \\ 
 `child' & eɾɑχɑi̯ & \armenian{երախայ} & ɑɾɑχɑ & \armenian{արախա} & jeɾeχɑ & \armenian{երեխա} \\
 `childhood, catechumenism' & eɾɑχɑ{ju}tʰiu̯n & \armenian{երախայութիւն} & ɑɾɑχɑtʰun & \armenian{արախաթուն} & jeɾeχɑjutʰjun & \armenian{երեխայություն} \\
 \hline 
 \end{tabular}
\end{table}. 

\subsubsection{Consonant changes}

\paragraph{Voicing changes}
The voiced stops became voiceless, and they kept their voicing only after nasals. In this situation, the voiceless stops became voiced (Table \ref{tab:Shamakhi:phonology:soundChange:cons:voice}).



\begin{table}[H]
 \centering
 \caption{Voicing changes in stops and affricates in the Shamakhi dialect}
 \label{tab:Shamakhi:phonology:soundChange:cons:voice}
 \begin{tabular}{|l|ll|ll| ll|}
 \hline &\multicolumn{2}{l|}{Classical Armenian}&\multicolumn{2}{l|}{> Shamakhi} & \multicolumn{2}{l|}{cf. SEA} \\ 
 `head' & ɡəluχ & \armenian{գլուխ} & kloχ & \armenian{կլօխ} & ɡəluχ & \armenian{գլուխ} \\
 `to bring' & beɾel & \armenian{բերել} & peɾil & \armenian{պէրիլ} & beɾel & \armenian{բերել} \\
 `snake' & od͡z & \armenian{օձ} & ot͡s & \armenian{օծ} & ot͡sʰ & \armenian{օձ} \\
`water-mill' &d͡ʒəɾɑɬɑt͡sʰ& \armenian{ջրաղաց} & t͡ʃɑʁɑt͡sʰ & \armenian{ճաղաց} & d͡ʒəɾɑʁɑt͡sʰ& \armenian{ջրաղաց} \\
 \armenian{՝}ear' & ɑkɑnd͡ʒ & \armenian{ականջ}& ɑnɡod͡ʒ & \armenian{անգօջ} & ɑkɑnd͡ʒ & \armenian{ականջ} \\
`wine' &ɡini & \armenian{գինի} & kini &\armenian{կինի} &ɡini & \armenian{գինի} \\
`thing' & bɑn & \armenian{բան} & pæn & \armenian{պա̈ն} & bɑn & \armenian{բան} \\ 
\hline 
 \end{tabular}
\end{table}


The word-final Classical sound /k/ <\armenian{կ}> became /ɡ/ <\armenian{գ}> in many cases (Table \ref{tab:Shamakhi:phonology:soundChange:cons:voicek}).



\begin{table}[H]
 \centering
 \caption{Voicing changes for word-final /k/ <\armenian{կ}> in the Shamakhi dialect}
 \label{tab:Shamakhi:phonology:soundChange:cons:voicek}
 \begin{tabular}{|l|ll|ll| ll|}
 \hline &\multicolumn{2}{l|}{Classical Armenian}&\multicolumn{2}{l|}{> Shamakhi} & \multicolumn{2}{l|}{cf. SEA} \\ 
 `board' & tɑχtɑk & \armenian{տախտակ} & tɑχtɑɡ & \armenian{տախտագ} & tɑχtɑk & \armenian{տախտակ} \\
`girl' & ɑɬd͡ʒik & \armenian{աղջիկ} & ɑχt͡ʃʰiɡ & \armenian{ախչիգ} & ɑχt͡ʃʰik & \armenian{աղջիկ} \\ 
`woman' & & & kniɡ & \armenian{կնիգ} & kənik & \armenian{կնիկ} \\ 
\hline 
 \end{tabular}
\end{table}

\paragraph{Loss of the rhotic /ɾ/ <\armenian{ր}>}

The sound /ɾ/ <\armenian{ր}> is lost in the words in Table \ref{tab:Shamakhi:phonology:soundChange:cons:r}a. In contrast, the \armenian{ր} */ɾ/ is stronger in the words in Table \ref{tab:Shamakhi:phonology:soundChange:cons:r}b. \translatorHD{Note that these multi-word phrases likely didn't exist in Classical Armenian, so we can assume that Adjarian was treating the shared closest ancestor of Shamakhi and SEA as being an SEA-like variety.}



\begin{table}[H]
 \centering
 \caption{Loss of rhotic /ɾ/ <\armenian{ր}> in the Shamakhi dialect}
 \label{tab:Shamakhi:phonology:soundChange:cons:r}
 \begin{tabular}{|ll|ll| ll| ll|}
 \hline & & \multicolumn{2}{l|}{Classical Armenian}& \multicolumn{2}{l|}{> Shamakhi} & \multicolumn{2}{l|}{cf. SEA} \\ 
a. & `this night' & ɑi̯s, ɡiʃeɾ & \armenian{այս, գիշեր} & æs kʰiʃe & \armenian{ա̈ս քիշէ} & ɑjs ɡiʃeɾ & \armenian{այս գիշեր} \\ 
 & `to go up' &veɾ, elɑn\'el &\armenian{վեր, ելանել} & v\'ellil & \armenian{վէ՛լլիլ} & veɾ jeln\'el & \armenian{վեր ելնել} \\ 
 & `hand' &d͡zer-kʰ (-{\pl}) & \armenian{ձեռք} & t͡sʰekʰ & \armenian{ցէք} & d͡zerkʰ & \armenian{ձեռք} \\ 
b. & `hundred' & hɑɾiu̯ɾ & \armenian{հարիւր} & hɑrur & \armenian{հառուռ} & hɑɾjuɾ & \armenian{հարյուր} \\ 
\hline 
 \end{tabular}
\end{table}

\paragraph{Insertion of word-initial /h/ <\armenian{հ}>}

Before word-initial vowels, the sound /h/ <\armenian{հ}> is sometimes added, just as in the Karabakh dialect (Table \ref{tab:Shamakhi:phonology:soundChange:cons:h}).



\begin{table}[H]
 \centering
 \caption{Insertion of word-initial /h/ <\armenian{հ}> in the Shamakhi dialect}
 \label{tab:Shamakhi:phonology:soundChange:cons:h}
 \begin{tabular}{|l|ll|ll| ll|}
 \hline & \multicolumn{2}{l|}{Classical Armenian}& \multicolumn{2}{l|}{> Shamakhi} & \multicolumn{2}{l|}{cf. SEA} \\ 
 `what' & int͡ʃʰ & \armenian{ինչ} & hint͡ʃʰ & \armenian{հինչ} & int͡ʃʰ & \armenian{ինչ} \\ 
 `who' & ov & \armenian{ով} & hov & \armenian{հօվ} & ov & \armenian{ով} \\ 
 \armenian{՝}when' & eɾb & \armenian{երբ} & hepʰ & \armenian{հէփ} & jeɾpʰ & \armenian{երբ} \\
`tail' &ɑɡ\'i& \armenian{ագի} & hækʰi & \armenian{հա̈քի} &ɑɡ\'i& \armenian{ագի} \\
\hline 
 \end{tabular}
\end{table}

\paragraph{Retention of word-final /n/ <\armenian{ն}>}

The Classical rime /n/ <\armenian{ն}> is kept here too (Table \ref{tab:Shamakhi:phonology:soundChange:cons:n}a) but not in Table \ref{tab:Shamakhi:phonology:soundChange:cons:n}b.



\begin{table}[H]
 \centering
 \caption{Retention or loss of final Classical /n/ <\armenian{ն}> in the Shamakhi dialect}
 \label{tab:Shamakhi:phonology:soundChange:cons:n}
 \begin{tabular}{|ll|ll|ll| ll|}
 \hline & & \multicolumn{2}{l|}{Classical Armenian}& \multicolumn{2}{l|}{> Shamakhi} & \multicolumn{2}{l|}{cf. SEA} \\ 
a. &`bitter' & d\'ɑrən & \armenian{դառն} & t\'ærnə & \armenian{տա̈՛ռնը} & d\'ɑrən & \armenian{դառն} \\ 
 & `fish' &d͡z\'ukən & \armenian{ձուկն} & t͡s\'ʏɡni & \armenian{ծի՛ւգնի} & d͡z\'uk & \armenian{ձուկ} \\ 
 & `mouse' &m\'ukən & \armenian{մուկն} & m\'uknə & \armenian{մո՛ւկնը} & m\'uk & \armenian{մուկ} \\ 
&`milk' &k\'ɑtʰən & \armenian{կաթն} & k\'ɑtʰnə & \armenian{կա՛թնը} & kɑtʰ & \armenian{կաթ} \\ 
& `pomegranate' &n\'urən & \armenian{նուռն} & n\'ornə & \armenian{նօ՛ռնը} & nur & \armenian{նուռ} \\ 
b. & `finger' &mɑtʰən & \armenian{մատն} & mɑt & \armenian{մատ} & mɑtʰ & \armenian{մատ} \\ 
 & \armenian{՝}foot' & otən & \armenian{ոտն} & vot & \armenian{վօտ} & votkʰ & \armenian{ոտք} \\
 & \armenian{՝}bride' & hɑɾsən & \armenian{հարսն} & hɑɾs & \armenian{հարս} & hɑɾs & \armenian{հարս} \\

\hline 
 \end{tabular}
\end{table}

\section{Morphology}
\subsection{Noun inflection or declension}

\subsubsection{Ablative marking with /-ɑn/ <\armenian{ան}>}
In declension, it is notable that the ablative formative is /-ɑn/ <\armenian{ան}> (Table \ref{tab:Shamakhi:morpho:noun:abl}), same as in Karabakh. 


\begin{table}[H]
 \centering
 \caption{Ablatives with /-ɑn/ <\armenian{ան}> in the Shamakhi dialect}
 \label{tab:Shamakhi:morpho:noun:abl}
 \begin{tabular}{|l|ll| ll|}
 \hline & \multicolumn{2}{l|}{Shamakhi} & \multicolumn{2}{l|}{cf. SEA} \\ 
 `under-{\abl}' & tɑk-ɑn & \armenian{տական} & tɑk-it͡sʰ & \armenian{տակից} \\ 
 `place-{\abl}' & teʁ-ɑn & \armenian{տէղան} & teʁ-it͡sʰ & \armenian{տեղից} \\ 
 `childhood-{\abl}' & ɑɾɑχɑtʰun-ɑn & \armenian{արախաթունան} & jeɾeχɑjutʰjun-it͡sʰ & \armenian{երեխայությունից} \\
\hline 
 \end{tabular}
\end{table}




\begin{adjarianpage}\label{page:78}\end{adjarianpage}% should be 78


Similar to the Karabakh dialect, we can also add here the formative /-ɑnɑ/ <\armenian{անա}>. For example /knɡɑns-ɑnɑ/ `from my wife'.\footnote{\translatorHD{Adjarian doesn't explain this form, but it seems it's morphologically decomposable to /knɡ-ɑn-s-ɑnɑ/ with the gloss `wife-{\obl}-{\possSsg}-{\abl}'. Note the unexpected presence of a possessive marker before the ablative marker. }}

\subsubsection{Instrumental marking with /-ov/ <\armenian{օվ}>}
The instrumental uses the formative /-ov/ <\armenian{օվ}>, while the Karabakh dialect has /-ɑv/ <\armenian{ավ}> (Table \ref{tab:Shamakhi:morpho:noun:abl}). 


\begin{table}[H]
 \centering
 \caption{Instrumentals with /-ov/ <\armenian{օվ}> in the Shamakhi dialect}
 \label{tab:Shamakhi:morpho:noun:ins}
 \begin{tabular}{|l|ll| ll| ll|}
 \hline & \multicolumn{2}{l|}{Shamakhi} & \multicolumn{2}{l|}{cf. Karabakh} & \multicolumn{2}{l|}{cf. SEA} \\ 
 `mouth-{\ins}' & peɾɑn-\'ov & \armenian{պէրանօվ} & piɾ\'æn-ɑv & \armenian{պիրա̈՛նավ} & beɾɑn-\'ov & \armenian{բերանով}\\ 
 `hand-{\ins}' & t͡sʰekʰ-\'ov & \armenian{ցէքօվ} & t͡s\'erkʰ-ɑv & \armenian{ծէ՛ռքավ} & d͡zerkʰ-\'ov & \armenian{ձեռքով}\\ 
\hline 
 \end{tabular}
\end{table} 


\subsubsection{Locative marking with /-əm, -im/ <\armenian{ըմ, իմ}>}
The locative formative /-um/ <\armenian{ում}> is shortened to /əm/ <\armenian{ըմ}> and then became \armenian{իմ} because of the rule of harmony (\armenian{ներդաշնակութեան օրէնքով}) (Table \ref{tab:Shamakhi:morpho:noun:loc}). 


\begin{table}[H]
 \centering
 \caption{Locatives with /-əm, -im/ <\armenian{ըմ, իմ}> in the Shamakhi dialect}
 \label{tab:Shamakhi:morpho:noun:loc}
 \begin{tabular}{|l|ll| ll| }
 \hline & \multicolumn{2}{l|}{Shamakhi} & \multicolumn{2}{l|}{cf. SEA} \\ 
 `thing-{\locgloss}-{\defgloss}' & pæn-\'im-i & \armenian{պա̈նի՛մի} & bɑn-\'um-ə & \armenian{բանումը}\\ 
\hline 
 \end{tabular}
\end{table} 

\subsubsection{Vowel harmony of the definite article /ə/ <\armenian{ը}> }

Based on the rule of vowel harmony, the article is /-ə/ <\armenian{ը}> (or /-i/ <\armenian{ի}>), and /-n/ <\armenian{ն}> (Table \ref{tab:Shamakhi:morpho:noun:article:vowel}). 


\begin{table}[H]
 \centering
 \caption{Vowel harmony of the definite article /ə/ <\armenian{ը}> to /-i/ <\armenian{ի}> in the Shamakhi dialect}
 \label{tab:Shamakhi:morpho:noun:article:vowel}
 \begin{tabular}{|l|ll| ll| }
 \hline & \multicolumn{2}{l|}{Shamakhi} & \multicolumn{2}{l|}{cf. SEA} \\ 
 `heart-{\defgloss}' & s\'əɾt-ə & \armenian{սը՛րտը} & s\'iɾt-ə & \armenian{սիրտը}\\ 
 `mind-{\defgloss}' & m\'ətk-ə & \armenian{մը՛տկը} & m\'itkʰ-ə & \armenian{միտքը}\\ 
 `hand-{\defgloss}' & t͡sʰ\'ekʰ-ə & \armenian{ցէ՛քը} & d͡z\'erkʰ-ə & \armenian{ձեռքը}\\ 
`girl-{\defgloss}' & ɑχt͡ʃʰ\'iɡ-i & \armenian{ախչի՛գի} & ɑχt͡ʃʰ\'ik-ə & \armenian{աղջիկը} \\ 
 `voice-{\defgloss}' & t͡s\'æn-i & \armenian{ծա̈՛նի} & d͡z\'ɑjn-ə & \armenian{ձայնը}\\ 
\hline 
 \end{tabular}
\end{table}

The same process occurs next to the possessive suffixes (\armenian{դիմորոշներուն քով}) from CA /-s, -d, -n/ <\armenian{ս, դ, ն}> (Table \ref{tab:Shamakhi:morpho:noun:article:poss}). \translatorHD{He means that we also find harmony for the schwa that is epenthesized between a stem-final consonant and a possessive suffixes. }




\begin{table}[H]
 \centering
 \caption{Vowel harmony of the possessive schwa in the Shamakhi dialect}
 \label{tab:Shamakhi:morpho:noun:article:poss}
 \begin{tabular}{|l|ll| ll| }
 \hline & \multicolumn{2}{l|}{Shamakhi} & \multicolumn{2}{l|}{cf. SEA} \\ 
 `mind-{\possSsg}' & t͡s\'æn-itʰ & \armenian{ծա̈՛նիթ} & d͡z\'ɑjn-ət & \armenian{ձայնդ}\\ 
 `heart-{\possFsg}' & s\'əɾt-əs & \armenian{սը՛րտըս} & s\'iɾt-əs & \armenian{սիրտս}\\ 
 `mind-{\possFsg}' & m\'ətk-əs & \armenian{մը՛տկըս} & m\'itkʰ-əs & \armenian{միտքս}\\ 
 `thing-{\possFsg}' & p\'æn-itʰ & \armenian{պա̈՛նիթ} & b\'ɑn-ət & \armenian{բանդ}\\ 
\hline 
 \end{tabular}
\end{table}


In these words, stress is on the penultimate syllable. The genitive, which has the same form, is distinguished from these words only by stress (Table \ref{tab:Shamakhi:morpho:noun:article:geb}). 

\begin{table}[H]
 \centering
 \caption{Stress distinctions between the stressed genitive and the unstressed harmonized schwa in the Shamakhi dialect}
 \label{tab:Shamakhi:morpho:noun:article:gen}
 \begin{tabular}{|l|ll| ll| }
 \hline & \multicolumn{2}{l|}{Shamakhi} & \multicolumn{2}{l|}{cf. SEA} \\ 
 `thing-{\possFsg}' & p\'æn-itʰ & \armenian{պա̈՛նիթ} & bɑn-ət & \armenian{բանդ}\\ 
 `thing-{\gen}-{\possFsg}' & pæn-\'i-tʰ &\armenian{պա̈նի՛թ} & bɑn-\'i-t & \armenian{բանիդ}\\ 
 `heart-{\dimgloss}-{\possFsg}' & sɾt-\'iɡ-is & \armenian{սրտի՛գիս} & səɾt-\'ik-əs & \armenian{սրտիկս}\\ 
 `heart-{\dimgloss}-{\gen}-{\possFsg}' & sɾt-iɡ-\'i-s &\armenian{սրտիգի՛ս} & səɾt-ik-\'is & \armenian{սրտիկիս}\\ 
\hline 
 \end{tabular}
\end{table} 

\subsection{Pronoun inflection or declension}

\subsubsection{Personal pronouns}
Pronouns are declined in the following way (Table \ref{tab:Shamakhi:morpho:pronoun:personal}). 

\begin{table}[H]
\caption{Inflection paradigm for personal pronouns in the Shamakhi dialect }\label{tab:Shamakhi:morpho:pronoun:personal}
\centering
\begin{tabular}{| l| llllll| }
\hline
 & 1SG & 2SG & 3SG & 1PL & 2SG & 3PL \\
 & `I' & `you' & `he/she' & `we' & `you' & `they' \\ \hline
{\nom} & jes & tʏ & nɑ & mekʰ & tʏkʰ & nɾɑnkʰ \\
 & \armenian{յէս} & \armenian{տիւ} & \armenian{նա} & \armenian{մէք} & \armenian{տիւք} & \armenian{նրանք} \\\hline
{\gen} & im & kʰu & nɾɑ & meɾ & t͡seɾ & nɾɑnt͡sʰ \\
 & \armenian{իմ} & \armenian{քու} & \armenian{նրա} & \armenian{մէր} & \armenian{ծէր} & \armenian{նրանց} \\\hline
{\dat} & ind͡z, ind͡z-i & kʰez, kʰez-i & nɾɑn & mez, mez-i & t͡sez, t͡sez-i & nɾɑnt͡sʰ \\
 & \armenian{ինձ, ինձի} & \armenian{քէզ, քէզի} & \armenian{նրան} & \armenian{մէզ, մէզի} & \armenian{ծէզ, ծէզի} & \armenian{նրանց} \\\hline
{\abl} & ind͡z-ɑn-ɑ & kʰez-ɑn-ɑ & nɾɑn-ɑ & mez-ɑn-ɑ & t͡sez-ɑn-ɑ & nɾɑnt͡sʰ-ɑn-ɑ \\
 & \armenian{ինձանա} & \armenian{քէզանա} & \armenian{նրանա} & \armenian{մէզանա} & \armenian{ծէզանա} & \armenian{նրանցանա} \\\hline
{\ins} & ind͡z-ɑn-ov & kʰez-ɑn-ov & nɾɑn-ov & mez-ɑn-ov & t͡sez-ɑn-ov & nɾɑnt͡sʰ-ɑn-ov \\
 & \armenian{ինձանով} & \armenian{քէզանօվ} & \armenian{նրանօվ} & \armenian{մէզանօվ} & \armenian{ծէզանօվ} & \armenian{նրանցանօվ} \\ \hline
\end{tabular}
\end{table}

\subsubsection{Repeated addition of the formative /-ik/ <\armenian{իկ}> }

The pronouns also have another interesting form which is unique to this dialect. Here, this is the addition of formative /-ik/ <\armenian{իկ}> . Although this formative appears in other dialects and its Classical Armenian, but there it is only added for the demonstratives (Table \ref{tab:Shamakhi:morpho:pronoun:ClasicalDem}a) and it is not declined, such as Table \ref{tab:Shamakhi:morpho:pronoun:ClasicalDem}b). \translatorHD{Note that Adjarian isn't clear on which Armenian varieties have the demonstratives in Table \ref{tab:Shamakhi:morpho:pronoun:ClasicalDem}. I had to personally catalog them and track down their attestations. }


\begin{table}[H]
\caption{Sample of demonstratives in Classical Armenian and other dialects }\label{tab:Shamakhi:morpho:pronoun:ClasicalDem}
\centering
\begin{tabular}{| l| llll l| }
\hline & CA & SEA & SWA & Istanbul& \\
\hline
a. Simple demonstratives & & & & &\\
 proximal `this' & ɑi̯s & ɑjs & ɑjs &&\armenian{այս} \\
 medial `that' & ɑi̯d &ɑjd & ɑjtʰ & & \armenian{այդ} \\
distal `that yonder' & ɑi̯n &ɑjn& ɑjn& & \armenian{այն} \\
\hline
b. Complex demonstratives & & & && \\
 proximal `this' & & & is-ik& ɑjs&\armenian{իսիկ} \\
 & & & ɑsiɡɑ &&\armenian{ասիկա} \\
 & ɑi̯soɾ-ik& & &&\armenian{այսորիկ} \\
 & ɑi̯sm-ik& & &&\armenian{այսմիկ} \\
 medial `that' & & & & it-ik & \armenian{իտիկ} \\
 & & & ɑd-iɡɑ &&\armenian{ատիկա} \\
 \hline
\end{tabular}
\end{table}


But here in Shamakhi, the reflex of the formative /-ik/ <\armenian{իկ}> (\translatorHD{as /-iɡ/ <\armenian{իգ}>}) is added on the pronouns `I', `you.{\sg} ', `he' and other pronouns, in all their case declensions. And it can be repeated up to three times (Table \ref{tab:Shamakhi:morpho:pronoun:repeated}). 

\translatorHD{Adjarian doesn't segment or explain the meaning of these complex pronouns with /-ik/. Based on sentence (\ref{sent:Shamakhi:morpho:pronoun:ikNoun:viz}), I suspect this /-iɡ/ may act as a diminutive suffix; the cognate is a diminutive suffix in SEA. But then it's unclear to me how why these pronouns would have these hypothetical diminutive. } 

\begin{table}[H]
\caption{Sample A of pronouns in Shamakhi with the repeated /-iɡ/ <\armenian{իկ}> formative }\label{tab:Shamakhi:morpho:pronoun:repeated}
\centering
\begin{tabular}{| l lll | }
\hline 1SG `I'&jes & \armenian{յէս} & {\pro} \\
&jes-iɡ&\armenian{յէսիգ}& {\pro}-? \\
&jes-\'iɡ-is& \armenian{յէսի՛գիս} & {\pro}-?-{\possFsg}\\
&jes-iɡ-\'iɡ-is &\armenian{յէսիգի՛գիս} & {\pro}-?-?-{\possFsg} \\
&jes-iɡ-iɡ-\'iɡ-is&\armenian{յէսիգիգի՛գիս} & {\pro}-?-?-?-{\possFsg} \\
\hline 
2SG `you' &tʏ & \armenian{տիւ} & {\pro} \\
&tʏ-iɡ& \armenian{տիւիգ} & {\pro}-? \\
& tʏ-\'iɡ-itʰ & \armenian{տիւի՛գիթ}& {\pro}-?-{\possSsg} \\ 
&tʏ-iɡ-\'iɡ-itʰ& \armenian{տիւիգի՛գիթ} & {\pro}-?-?-{\possSsg} \\
&tʏ-iɡ-iɡ-\'iɡ-itʰ& \armenian{տիւիգիգի՛գիթ} & {\pro}-?-?-{\possSsg} \\ 
\hline 
 3SG `he' & nɑ & \armenian{նա} & {\pro} \\
 & nɑ-\'iɡ-i & \armenian{նաի՛գի} & {\pro}-?--{\defgloss} \\
 & nɑ-iɡ-\'iɡ-i& \armenian{նաիգի՛գի} & {\pro}-?-?-{\defgloss} \\ \hline
3PL `they' & nɾɑnkʰ-\'iɡ-i& \armenian{նրանքի՛գի} & {\pro}-?-{\defgloss} \\
 & nɾɑnkʰ-iɡ-iɡ-\'iɡ-i & \armenian{նրանքիգիգի՛գի} & {\pro}-?-?-?-{\defgloss}\\
 \hline
\end{tabular}
\end{table}


\begin{adjarianpage}\label{page:79}\end{adjarianpage}% should be 79


In the other case declensions, the formative is added between the word and the case ending. As for in the plural, the formative /neɾ/ <\armenian{նէր}> is also required (Table \ref{tab:Shamakhi:morpho:pronoun:repeatedb}).



\begin{table}[H]
\caption{Sample B of pronouns in Shamakhi with the repeated /-iɡ/ <\armenian{իկ}> formative }\label{tab:Shamakhi:morpho:pronoun:repeatedb}
\centering
\begin{tabular}{| ll ll | }
\hline 2SG `you' ({\acc}, {\dat}) & kʰez-iɡ-i-n & \armenian{քէզիգին} & {\pro}-?-{\dat}-{\defgloss} \\
 & kʰez-iɡ-i-tʰ & \armenian{քէզիգիթ}& {\pro}-?-{\dat}-{\possSsg} \\
2SG `you' ({\abl}) & kʰez-ɑn-iɡᶨ-æn & \armenian{քէզանիգյա̈ն}& {\pro}-{\nx}-?-{\abl} \\
2SG `you' ({\ins}) & kʰez-ɑn-iɡᶨ-œv & \armenian{քէզանիգյէօվ}& {\pro}-{\nx}-?-{\ins} \\
\hline
1SG `mine' & im-\'iɡ-is & \armenian{իմի՛գիս} & {\pro}-?-{\possFsg} \\
1SG `I' ({\acc}, {\dat}) & ind͡z-iɡ-i-s & \armenian{ինձիգիս} & {\pro}-?-{\dat}-{\possFsg} \\
 & ind͡z-iɡ-i-n & \armenian{ինձիգին} & {\pro}-?-{\dat}-{\defgloss} \\
1SG `I' ({\abl}) & ind͡z-ɑn-iɡʲ-æ-s & \armenian{ինձանիգյա̈ս} & {\pro}-{\nx}-?-{\abl}-{\possFsg} \\
 & ind͡z-ɑn-iɡʲ-æn & \armenian{ինձանիգյա̈ն} & {\pro}-{\nx}-?-{\abl} \\
1SG `I' ({\ins}) & ind͡z-ɑn-iɡʲ-œov & \armenian{ինձանիգյէօվ} & {\pro}-{\nx}-?-{\ins} \\
 & ind͡z-ɑn-iɡʲ-\'œov-is & \armenian{ինձանիգյէ՛օվիս} & {\pro}-{\nx}-?-{\ins}-{\possFsg} \\
 3SG `he' & nɾɑn-\'iɡ-i & \armenian{նրանի՛գի} & {\pro}-?-{\defgloss} \\
 3SG `he' ({\abl}) & nɾɑn-iɡ-æn & \armenian{նրանիգա̈ն} & {\pro}-?-{\abl} \\
 3PL `they' ({\abl}) & nɾɑnt͡sʰ-ɑn-iɡʲ-æn & \armenian{նրանցանիգյա̈ն} & {\pro}-{\nx}-?-{\abl} \\
 1PL `we' & mekʰ-neɾ-\'iɡʲ-is & \armenian{մէքնէրի՛գիս} & {\pro}-{\pl}-?-{\possFsg} \\
 & mekʰ-neɾ-iɡ-iɡ-\'iɡʲ-is & \armenian{մէքնէրիգիգի՛գիս} & {\pro}-{\pl}-?-?-?-{\possFsg} \\
 2PL `you' & tʏkʰ-neɾ-iɡ-iɡ-\'iɡʲ-itʰ & \armenian{տիւքնէրիգիգի՛գիթ} & {\pro}-{\pl}-?-?-?-{\possSsg} \\
 \hline
\end{tabular}
\end{table}


The \armenian{իկ} /ik/ formative is so common in the Shamakhi dialect that it can added on almost any word (\ref{sent:Shamakhi:morpho:pronoun:ikNoun}). 

\begin{exe}
 \ex Shamakhi \label{sent:Shamakhi:morpho:pronoun:ikNoun}
 \begin{xlist}
 \ex \gll kʰu hoɾ tun-\'iɡ-i \\
 you.{\pl}.{\gen} father.{\gen} house-?-{\defgloss} \\
 \trans `your father's house' \\
 \armenian{քու հօր տունի՛գի} 
 \ex \gll viz-\'iɡ-i \\
 neck-?-{\defgloss} \\
 \trans `my little neck' \label{sent:Shamakhi:morpho:pronoun:ikNoun:viz} \\
 \armenian{վիզի՛գիս} 
 \end{xlist}
\end{exe}

\subsubsection{Other innovations}
Some pronoun innovations in Shamakhi are in Table \ref{tab:Shamakhi:morpho:pronoun:innovation}a. Some of their more widespread forms are in Table \ref{tab:Shamakhi:morpho:pronoun:innovation}b. They originate from Table \ref{tab:Shamakhi:morpho:pronoun:innovation}c, cf. Karabakh forms in Table \ref{tab:Shamakhi:morpho:pronoun:innovation}d.\footnote{\translatorHD{Adjarian doesn't state this, but it's possible that this construction is grammaticalized from a construction made up of a genitive pronoun + the postposition /het/ `with'. For example, Adjarian provides phrases like /kʰez heti/ which resemble an SEA phrase like /kʰez het/ <\armenian{քեզ հետ}> `with you'. }}


\begin{table}[H]
\caption{Pronoun innovations in the Shamakhi dialect }\label{tab:Shamakhi:morpho:pronoun:innovation}
\centering
\begin{tabular}{| ll ll |l| }
\hline 
& `for you.{\sg}' & `for us' & `for you.{\pl}'& Dialect \\
a.& kʰez-ti & mez-ti & t͡sez-ti & Shamakhi\\
&\armenian{քէզտի} & \armenian{մէզտի}& \armenian{ծէզտի} & \\
b. &kʰez-eti & mez-eti & t͡sez-eti & Shamakhi\\
& \armenian{քէզէտի} & \armenian{մէզէտի} & \armenian{ծէզէտի} & \\
c. &kʰez heti & mez heti & t͡sez heti & Shamakhi?\\
 & \armenian{քէզ հէտի} & \armenian{մէզ հէտի} & \armenian{ծէզ հէտի} & \\
d. &kʰʲəz h\'ete & məz h\'ete & t͡səz h\'ete & Karabakh \\
 &\armenian{քյըզ հէ՛տէ}& \armenian{մըզ հէ՛տէ}& \armenian{ծըզ հէ՛տէ} & \\
\hline
\end{tabular}
\end{table}

\subsection{Verb inflection or conjugation}

\subsubsection{Overview and general properties}
Verbal conjugation in the Shamakhi dialect is sometimes the same as in the Karabakh dialect, and sometimes it distances itself from Karabakh and approaches the Julfa dialect.


\paragraph{Copula with /ɑ/ <\armenian{ա}> }
For the copular verb in the present tense, the Classical sound /e/ <\armenian{ե}> sound becomes /ɑ/ <\armenian{ա}> next to nasals (Table \ref{tab:Shamakhi:morpho:verb:copula}). 


\begin{table}[H]
 \centering
 \caption{Copula with /ɑ/ <\armenian{ա}> instead of /e/ <\armenian{է}> in the Shamakhi dialect}
 \label{tab:Shamakhi:morpho:verb:copula}
 \begin{tabular}{|l|ll| ll| }
 \hline & \multicolumn{2}{l|}{Shamakhi} & \multicolumn{2}{l|}{cf. SEA} \\ 
1SG `I am' &ɑ-m & \armenian{ամ} &e-m & \armenian{եմ}\\ 
2SG `you are' &e-s & \armenian{էս} &e-s & \armenian{ես}\\ 
3SG `he is' &ɑ & \armenian{ա} &e & \armenian{է}\\ 
1PL `we are' &ɑ-nkʰ & \armenian{անք} &e-ŋkʰ & \armenian{ենք}\\ 
2PL `you are' &e-kʰ & \armenian{էք} &e-kʰ & \armenian{եք}\\ 
3PL `they are' &ɑ-n & \armenian{ան} &e-n & \armenian{են}\\ 
& \multicolumn{2}{l|}{{\aux}-{\agr}}& \multicolumn{2}{l|}{{\aux}-{\agr}} \\
\hline 
 \end{tabular}
\end{table}

\paragraph{Present and past imperfective}\label{sec:Shamakhi:morpholgoy:verb:overview:prespastimperfectivevowels}

In this way (\translatorHD{meaning with the above copulas}), the present form of other verbs is formed. The imperfective is similar to Karabakh. 

\paragraph{Past participle or perfective converb} 

The past participle ends in /-ɑl/ <\armenian{ալ}> (Table \ref{tab:Shamakhi:morpho:verb:pastpart}). 


\begin{table}[H]
 \centering
 \caption{Past participle or perfective converb with /-ɑl/ <\armenian{ալ}> in the Shamakhi dialect}
 \label{tab:Shamakhi:morpho:verb:pastpart}
 \begin{tabular}{|l|ll| ll| }
 \hline & \multicolumn{2}{l|}{Shamakhi} & \multicolumn{2}{l|}{cf. SEA} \\ 
`to tie' &kɑp-ɑl & \armenian{կապալ} &kɑp-el & \armenian{կապել}\\ 
`to fall' &ənɡ-ɑl & \armenian{ընգալ} &ənk-ɑl & \armenian{ընկել}\\ 
& \multicolumn{2}{l|}{$\sqrt{}$-{\perfcvb}}& \multicolumn{2}{l|}{$\sqrt{}$-{\perfcvb}} \\
\hline 
 \end{tabular}
\end{table}

\paragraph{Infinitives take dative /-i/ <\armenian{ի}> } 

The infinitive is case marked with the formative /-i/ <\armenian{ի}> (Table \ref{tab:Shamakhi:morpho:verb:dative}). 


\begin{table}[H]
 \centering
 \caption{Infinitives with dative /-i/ <\armenian{ի}> instead of /-u/ <\armenian{ու}> in the Shamakhi dialect}
 \label{tab:Shamakhi:morpho:verb:dative}
 \begin{tabular}{|l|ll| ll|l| }
 \hline & \multicolumn{2}{l|}{Shamakhi} & \multicolumn{2}{l|}{cf. SEA} & \\ 
`to say' &ɑs-i-l-i & \armenian{ասիլի} & ɑs-e-l-i & \armenian{ասելու} & $\sqrt{}$-{\thgloss}-{\infgloss}-{\dat}\\ 
`to throw' &kʰit͡sʰ-i-l-i & \armenian{քիցիլի} &ɡət͡sʰ-e-l-u & \armenian{գցելու} & $\sqrt{}$-{\thgloss}-{\infgloss}-{\dat}\\ 
`to die' &mer-n-e-l-i & \armenian{մէռնէլի} &mer-n-e-l-u & \armenian{մեռնելու} & $\sqrt{}$-{\vx}-{\thgloss}-{\infgloss}-{\dat}\\ 
\hline 
 \end{tabular}
\end{table}
 
\paragraph{Causative suffix /-t͡sʰu/ <\armenian{ցու}>} 


The causative (\armenian{անցողական}) formative is /-t͡sʰu/ <\armenian{ցու}> (Table \ref{tab:Shamakhi:morpho:verb:causative}). \translatorHD{Unfortunately, Adjarian doesn't translate his examples to SEA or SWA, so I can only guess what they're supposed to mean or how they should be segmented based on the SEA forms that resemble them the most.} 


\begin{table}[H]
 \centering
 \caption{Causatives with dative /-i/ <\armenian{ի}> instead of /-u/ <\armenian{ու}> in the Shamakhi dialect}
 \label{tab:Shamakhi:morpho:verb:causative}
 \begin{tabular}{|l|lll | }
 \hline & \multicolumn{3}{l|}{Shamakhi and cf. SEA }\\ \hline 
`I approach' (?) & mot-ɑ-t͡sʰun-ɑ-m & \armenian{մօտացունամ} &Sh. \\
&mot-e-t͡sʰn-e-m & \armenian{մոտեցնեմ} &SEA\\ 
& \multicolumn{3}{l|}{$\sqrt{}$-{\lvgloss}-{\caus}-{\thgloss}-1{\sg}} \\ \hline
`to lose' (?) &kʰoɾ-ɑ-t͡sʰun-i-l &\armenian{քօռացունիլ}&Sh. \\
&koɾ-t͡sn-e-l & \armenian{կորցնել} &SEA\\
& \multicolumn{3}{l|}{$\sqrt{}$-({\lvgloss})-{\caus}-{\thgloss}-{\infgloss}} \\ \hline
`he has fed' (?) &ut-ɑ-t͡sʰuɾ-ɑl ɑ &\armenian{ուտացուրալ ա}&Sh. \\
& ut-e-t͡sʰɾ-el e& \armenian{ուտեցրել է} &SEA \\
& \multicolumn{3}{l|}{$\sqrt{}$-{\lvgloss}-{\caus}-{\perfcvb} {\aux}} \\ \hline
`I have delivered' (?) &hɑs-ɑ-t͡sʰuɾ-ɑl ɑ-m &\armenian{հասացուրալ ամ} &Sh. \\
& hɑs-t͡sʰɾ-el e-m& \armenian{հասցրել եմ} &SEA\\
& \multicolumn{3}{l|}{$\sqrt{}$-{\lvgloss}-{\caus}-{\perfcvb} {\aux}-1{\sg}} \\ \hline
? &hɑnɡ-ɑ-t͡sʰuɾ-i-$\emptyset$ &\armenian{հանգացուրի} &Sh. \\
& \multicolumn{3}{l|}{$\sqrt{}$-{\lvgloss}-{\caus}-{\pst}-1{\sg}} \\ \hline
\armenian{՝}we have raised' (?) &pɑɾt͡sʰɾ-ɑ-t͡sʰuɾ-ɑl ɑ-nkʰ &\armenian{պարցրացուրալ անք} &Sh. \\
& bɑɾt͡sʰɾ-ɑ-t͡sʰɾ-el e-ŋkʰ &\armenian{բարձրացրել ենք} &SEA \\
& \multicolumn{3}{l|}{$\sqrt{}$-{\lvgloss}-{\caus}-{\perfcvb} {\aux}-1{\pl}} \\
\hline 
 \end{tabular}
\end{table}

After soft vowels (\translatorHD{I think he means front vowels}), the sound /ɑ/ <\armenian{ա}> becomes /æ/ <\armenian{ա}̈ > (Table \ref{tab:Shamakhi:morpho:verb:soft}).


\begin{table}[H]
 \centering
 \caption{Vowel fronting for /ɑ/ <\armenian{ա}> in the Shamakhi dialect}
 \label{tab:Shamakhi:morpho:verb:soft}
 \begin{tabular}{|l|ll| ll|l| }
 \hline & \multicolumn{2}{l|}{Shamakhi} & \multicolumn{2}{l|}{cf. SEA} & \\ 
`I do' &ɑn-əm æ-m & \armenian{անըմ ա̈մ} & ɑn-um e-m & \armenian{անում եմ} & $\sqrt{}$-{\impfcvb} {\aux}-1{\sg}\\ 
`done' &il-æl & \armenian{իլա̈լ} & el-el & \armenian{եղել} & $\sqrt{}$-{\perfcvb} \\ 
`he has thrown' &kʰit͡sʰ-æl æ & \armenian{քիցա̈լ ա}̈ & ɡət͡sʰ-el e& \armenian{գցել է} & $\sqrt{}$-{\perfcvb} {\aux} \\ 

\hline 
 \end{tabular}
\end{table}

\subsubsection{Verb paradigms}
 \translatorHD{Adjarian does not give any complete verb paradigms for this dialect. He only provides some datasets. }


The following are the important tenses for the verb `to like' from Classical /siɾ-e-l/ <\armenian{սիրել}>. 

\paragraph{Indicative present and past imperfective}

\translatorHD{In SEA, the indicative present is formed by combining the imperfective converb with the present auxiliary (Table \ref{tab:Shamakhi:morpho:verb:paradigm:presentIndc}). This converb uses the suffix /-um/. For Shamakhi, Adjarian provides a complete paradigm for the indicative present. Shamakhi uses the same system as SEA. The only difference is that the converb suffix is /-əm/ and the auxiliary is /ɑ/ for the 3SG and before nasals. }


\begin{table}[H]
 \centering
 \caption{Indicative present <\armenian{ներկայ}> of the verb `to like' in the Shamakhi dialect}
 \label{tab:Shamakhi:morpho:verb:paradigm:presentIndc}
 \begin{tabular}{|l|ll|ll|}
\hline & \multicolumn{2}{l|}{Shamakhi} & \multicolumn{2}{l|}{cf. SEA} \\
1SG & siɾ-əm ɑ-m & \armenian{սիրում էմ} & siɾ-um e-m &\armenian{սիրում եմ} \\
2SG & siɾ-əm e-s & \armenian{սիրում էս} & siɾ-um e-s &\armenian{սիրում ես} \\
3SG & siɾ-əm ɑ & \armenian{սիրում ա} & siɾ-um e &\armenian{սիրում է} \\
1PL & siɾ-əm ɑ-nkʰ & \armenian{սիրում էնք} & siɾ-um e-ŋk &\armenian{սիրում ենք} \\
2PL & siɾ-əm e-kʰ & \armenian{սիրում էք} & siɾ-um e-kʰ &\armenian{սիրում եք} \\
3PL& siɾ-əm ɑ-n & \armenian{սիրում էն} & siɾ-um e-n &\armenian{սիրում են} \\
& \multicolumn{2}{l|}{$\sqrt{}$-{\impfcvb} {\aux}-{\agr}}& \multicolumn{2}{l|}{$\sqrt{}$-{\impfcvb} {\aux}-{\agr}}\\
\hline 
\end{tabular}
\end{table}


\translatorHD{In SEA, the indicative past imperfective is formed by combining the imperfective converb with the past auxiliary (Table \ref{tab:Shamakhi:morpho:verb:paradigm:pastImpfIndc}). Adjarian does not provide a complete paradigm for Shamakhi. He provides only the 1SG and 2SG. He suggested earlier in \S\ref{sec:Shamakhi:morpholgoy:verb:overview:prespastimperfectivevowels} that Shamakhi uses the same set of past auxiliary morphs as the Karabakh dialect. }

 




\begin{table}[H]
 \centering
 \caption{Indicative past imperfective <\armenian{անկատար}> of the verb `to like' in the Shamakhi dialect}
 \label{tab:Shamakhi:morpho:verb:paradigm:pastImpfIndc}
 \begin{tabular}{|l|ll|ll|ll|l|}
\hline & \multicolumn{2}{l|}{Shamakhi} & \multicolumn{2}{l|}{cf. SEA} \\
1SG & siɾ-əm $\emptyset$-i-$\emptyset$ & \armenian{սիրըմ ի} & siɾ-um ej-i-$\emptyset$ &\armenian{սիրում էի} \\
2SG& siɾəm $\emptyset$-i-ɾ & \armenian{սիրըմ իր} & siɾ-um ej-i-ɾ &\armenian{սիրում էիր} \\
& \multicolumn{2}{l|}{$\sqrt{}$-{\impfcvb} {\aux}-{\pst}-{\agr}} & \multicolumn{2}{l|}{$\sqrt{}$-{\impfcvb} {\aux}-{\pst}-{\agr}}\\
\hline 
\end{tabular}
\end{table}

\translatorHD{Note that unlike the Karabakh dialect, data from the past perfective suggests that the past suffix is /-i/ in this dialect, like SEA. }

\paragraph{Present perfect and past perfect}

\translatorHD{In SEA, the present perfect (Table \ref{tab:Shamakhi:morpho:verb:paradigm:presentPerfect}) and past perfect (Table \ref{tab:Shamakhi:morpho:verb:paradigm:pastPerfect}) are formed by combining the perfective converb with the present/past auxiliary. For SEA, this converb uses the suffix /-el/. Shamakhi uses the same system, but with converb suffix as /-ɑl/. Adjarian provides a complete paradigm for the present perfect, but an incomplete one for the past perfect. See \S\ref{sec:Shamakhi:morpholgoy:verb:overview:prespastimperfectivevowels} for brief discussion on what the past auxiliaries could be. }

\begin{table}[H]
 \centering
 \caption{Present perfect <\armenian{յարակատար}> of the verb `to like' in the Shamakhi dialect}
 \label{tab:Shamakhi:morpho:verb:paradigm:presentPerfect}
 \begin{tabular}{|l|ll|ll|}
\hline & \multicolumn{2}{l|}{Shamakhi} & \multicolumn{2}{l|}{cf. SEA} \\
1SG &siɾ-ɑl ɑ-m & \armenian{սիրալ ամ} & siɾ-el e-m &\armenian{սիրել եմ} \\
2SG &siɾ-ɑl e-s & \armenian{սիրալ էս} & siɾ-el e-s &\armenian{սիրել ես} \\
3SG &siɾ-ɑl ɑ & \armenian{սիրալ ա} & siɾ-el e &\armenian{սիրել է} \\
1PL&siɾ-ɑl ɑ-nkʰ & \armenian{սիրալ անք} & siɾ-el e-ŋk &\armenian{սիրել ենք} \\
2PL&siɾ-ɑl e-kʰ & \armenian{սիրալ էք}& siɾ-el e-kʰ &\armenian{սիրել եք} \\
3PL &siɾ-ɑl ɑ-n & \armenian{սիրալ ան} & siɾ-el e-n &\armenian{սիրել են} \\
& \multicolumn{2}{l|}{$\sqrt{}$-{\perfcvb} {\aux}-{\agr}}& \multicolumn{2}{l|}{$\sqrt{}$-{\perfcvb} {\aux}-{\agr}}\\ 

\hline 
\end{tabular}
\end{table}


\begin{table}[H]
 \centering
 \caption{Past perfect <\armenian{գերակատար}> of the verb `to like' in the Shamakhi dialect}
 \label{tab:Shamakhi:morpho:verb:paradigm:pastPerfect}
 \begin{tabular}{|l|ll|ll| }
\hline & \multicolumn{2}{l|}{Shamakhi} & \multicolumn{2}{l|}{cf. SEA} \\
1SG &siɾ-ɑl $\emptyset$-i-$\emptyset$ & \armenian{սիրալ ի} & siɾ-el ej-i-$\emptyset$ &\armenian{սիրել էի} \\
2SG &siɾ-ɑl $\emptyset$-i-ɾ & \armenian{սիրալ իր} & siɾ-el ej-i-ɾ &\armenian{սիրել էիր} \\
& \multicolumn{2}{l|}{$\sqrt{}$-{\perfcvb} {\aux}-{\pst}-{\agr}}& \multicolumn{2}{l|}{$\sqrt{}$-{\perfcvb} {\aux}-{\pst}-{\agr}}\\ 

\hline 
\end{tabular}
\end{table}

\paragraph{Future and future perfect}

\translatorHD{In SEA, one strategy to form the future (Table \ref{tab:Shamakhi:morpho:verb:paradigm:future}) and future perfect (Table \ref{tab:Shamakhi:morpho:verb:paradigm:futurePerfect}) is to use periphrasis. The future converb is combined with the present/past auxiliary. For SEA, this converb is formed by adding the suffix /-u/ onto the infinitive. Shamakhi uses the same system, but with converb suffix as /-ʏ/. Adjarian provides incomplet paradigms for these two tenses. They would use the same present and past auxiliaries as the previous periphrastic tenses (indicative present/past, and present/past perfect). } 



\begin{table}[H]
 \centering
 \caption{Future <\armenian{ապառնի}> of the verb `to like' in the Shamakhi dialect}
 \label{tab:Shamakhi:morpho:verb:paradigm:future}
 \begin{tabular}{|l|ll|ll|}
\hline & \multicolumn{2}{l|}{Shamakhi} & \multicolumn{2}{l|}{cf. SEA} \\
1SG &siɾ-e-l-ʏ ɑ-m & \armenian{սիրէլիւ ամ} & siɾ-e-l-u e-m &\armenian{սիրելու եմ} \\
2SG &siɾ-e-l-ʏ e-s & \armenian{սիրէլիւ էս} & siɾ-e-l-u e-s &\armenian{սիրելու ես} \\
3SG &siɾ-e-l-ʏ ɑ & \armenian{սիրէլիւ ա} & siɾ-e-l-u e &\armenian{սիրելու է} \\
& \multicolumn{2}{l|}{$\sqrt{}$-{\thgloss}-{\infgloss}-{\futcvb} {\aux}-{\agr}}& \multicolumn{2}{l|}{$\sqrt{}$-{\thgloss}-{\infgloss}-{\futcvb} {\aux}-{\agr}}\\ 

\hline 
\end{tabular}
\end{table}


\begin{table}[H]
 \centering
 \caption{Future perfect <\armenian{անցեալ ապառնի}> of the verb `to like' in the Shamakhi dialect}
 \label{tab:Shamakhi:morpho:verb:paradigm:futurePerfect}
 \begin{tabular}{|l|ll|ll| }
\hline & \multicolumn{2}{l|}{Shamakhi} & \multicolumn{2}{l|}{cf. SEA} \\
1SG &siɾ-e-l-ʏ $\emptyset$-i-$\emptyset$ & \armenian{սիրէլիւ ի} & siɾ-el ej-i-$\emptyset$ &\armenian{սիրելու էի} \\
2SG &siɾ-e-l-ʏ $\emptyset$-i-ɾ & \armenian{սիրէլիւ իր} & siɾ-el ej-i-ɾ &\armenian{սիրելու էիր} \\
& \multicolumn{2}{l|}{$\sqrt{}$-{\thgloss}-{\infgloss}-{\futcvb} {\aux}-{\pst}-{\agr}}& \multicolumn{2}{l|}{$\sqrt{}$-{\thgloss}-{\infgloss}-{\futcvb} {\aux}-{\pst}-{\agr}}\\ 

\hline 
\end{tabular}
\end{table}

\paragraph{Past perfective or aorist}

\translatorHD{In SEA, the past perfective or aorist (Table \ref{tab:Shamakhi:morpho:verb:paradigm:pastperfectiveAorist}) is formed in the following way for /siɾ-e-l/ `to like'. After the root and theme vowel, we add the the aorist or perfective suffix /-t͡sʰ-/, and then adding the past suffix /-i/ and the appropriate agreement suffixes. The 3SG uses covert tense and agreement suffixes. Adjarian unfortunately only provides the 1SG form for Shamakhi. I suspect that his omission implies that Shamakhi followed the same past perfective system as SEA> } 

\begin{table}[H]
 \centering
 \caption{Past perfective or aorist <\armenian{կատարեալ}> of the verb `to like' in the Shamakhi dialect}
 \label{tab:Shamakhi:morpho:verb:paradigm:pastperfectiveAorist}
 \begin{tabular}{|l|ll|ll|}
\hline & \multicolumn{2}{l|}{Shamakhi} & \multicolumn{2}{l|}{cf. SEA} \\
1SG & siɾ-e-t͡sʰ-i-$\emptyset$ & \armenian{սիրէցի} & siɾ-e-t͡sʰ-i-$\emptyset$ & \armenian{սիրեցի} \\
& \multicolumn{2}{l|}{$\sqrt{}$-{\thgloss}-{\aor}-{\pst}-{\agr}}& \multicolumn{2}{l|}{$\sqrt{}$-{\thgloss}-{\aor}-{\pst}-{\agr}}\\ 

\hline 
\end{tabular}
\end{table}

\paragraph{Subjunctive present and past imperfective } 

\translatorHD{In SEA, the subjunctive present (Table \ref{tab:Shamakhi:morpho:verb:paradigm:subjPresent}) is formed by adding agreement suffixes after the theme vowel. These are the same agreement suffixes that are added onto the present auxiliary in the indicative present. For a verb like `to like', the 3SG involves changing the theme vowel /e/ to /i/ in the 3SG. The Shamakhi dialect follows the same system but with the following changes: the theme vowel is /ɑ/ in the 3SG or before nasals. } 


\begin{table}[H]
 \centering
 \caption{Subjunctive present <\armenian{ստորադասական ներկայ}> of the verb `to like' in the Shamakhi dialect}
 \label{tab:Shamakhi:morpho:verb:paradigm:subjPresent}
 \begin{tabular}{|l|ll|ll|}
\hline & \multicolumn{2}{l|}{Shamakhi} & \multicolumn{2}{l|}{cf. SEA} \\
1SG & siɾ-ɑ-m & \armenian{սիրամ} & siɾ-e-m & \armenian{սիրեմ} \\
2SG & siɾ-e-s & \armenian{սիրէս} & siɾ-e-s & \armenian{սիրես} \\
3SG & siɾ-ɑ-$\emptyset$ & \armenian{սիրա} & siɾ-i-$\emptyset$ & \armenian{սիրի} \\
1PL & siɾ-ɑ-nkʰ & \armenian{սիրանք} & siɾ-e-ŋkʰ & \armenian{սիրենք} \\
2PL & siɾ-e-k & \armenian{սիրէք} & siɾ-e-k & \armenian{սիրեք} \\
3PL & siɾ-ɑ-n & \armenian{սիրան} & siɾ-e-n & \armenian{սիրեն} \\
& \multicolumn{2}{l|}{$\sqrt{}$-{\thgloss}-{\agr}}& \multicolumn{2}{l|}{$\sqrt{}$-{\thgloss}-{\agr}}\\ 

\hline 
\end{tabular}
\end{table}

\translatorHD{In SEA, the subjunctive past imperfective (Table \ref{tab:Shamakhi:morpho:verb:paradigm:subjPast}) is formed by adding the past suffix /i/ and agreement suffixes after the theme vowel. For Shamakhi, Adjarian does not provide a complete paradigm. But it seems that the past suffix is added and it deletes the preceding /e/ theme vowel. It's unclear how the 3SG would be; I suspect it would resemble the Karabakh system in the choice of surface morphs. } 




\begin{table}[H]
 \centering
 \caption{Subjunctive past <\armenian{ստորադասական անցեալ}> of the verb `to like' in the Shamakhi dialect}
 \label{tab:Shamakhi:morpho:verb:paradigm:subjPast}
 \begin{tabular}{|l|ll|ll|}
\hline & \multicolumn{2}{l|}{Shamakhi} & \multicolumn{2}{l|}{cf. SEA} \\
1SG & siɾ-$\emptyset$-i-$\emptyset$ & \armenian{սիրի} & siɾ-ej-i-$\emptyset$ & \armenian{սիրեի} \\
2SG & siɾ-$\emptyset$-i-ɾ & \armenian{սիրիր} & siɾ-ej-i-ɾ & \armenian{սիրեիր} \\
& \multicolumn{2}{l|}{$\sqrt{}$-{\thgloss}-{\pst}-{\agr}}& \multicolumn{2}{l|}{$\sqrt{}$-{\thgloss}-{\pst}-{\agr}}\\ 

\hline 
\end{tabular}
\end{table}





\paragraph{Imperative and prohibitive}

\translatorHD{For the imperative, SEA distinguishes the 2SG from the 2PL. Unfortunately, Adjarian only provides 2SG forms for Shamakhi so we only discuss those. In SEA, the imperative 2SG is formed by adding the morph /-iɾ/ after the root for a verb like `to like' (Table \ref{tab:Shamakhi:morpho:verb:paradigm:Imp}). Shamakhi seems to use the morph /-i/ instead. }


\begin{table}[H]
 \centering
 \caption{Imperative forms <\armenian{հրամայական}> for the verb `to like' in the Shamakhi dialect}
 \label{tab:Shamakhi:morpho:verb:paradigm:Imp}
 \begin{tabular}{|l|ll|ll|l|}
\hline & \multicolumn{2}{l|}{Shamakhi} & \multicolumn{2}{l|}{cf. SEA} & \\
2SG & siɾ-i & \armenian{սիրի} & siɾ-iɾ & \armenian{սիրիր} & $\sqrt{}$-{\imp}.2{\sg}
\\\hline \end{tabular}
\end{table}

\translatorHD{For the prohibitive or negative imperative (Table \ref{tab:Shamakhi:morpho:verb:paradigm:Proh}), SEA simply adds the prohibitive formative /mi/ before the imperative form. Shamakhi seems to add this marker, and then change the verb into a non-finite form with /-ɑl/. } 


\begin{table}[H]
 \centering
 \caption{Negative imperative or prohibitive forms for the verb `to like' in the Shamakhi dialect}
 \label{tab:Shamakhi:morpho:verb:paradigm:Proh}
 \begin{tabular}{|l|ll|ll|}
\hline & \multicolumn{2}{l|}{Shamakhi} & \multicolumn{2}{l|}{cf. SEA} \\
2SG & m\'i siɾ-ɑl & \armenian{մի՛ սիրալ} & m\'i siɾ-iɾ & \armenian{մի՛ սիրիր} \\ 
& \multicolumn{2}{l|}{{\proh} $\sqrt{}$-?} & \multicolumn{2}{l|}{{\proh} $\sqrt{}$-{\agr}} \\
\hline \end{tabular}
\end{table}


\paragraph{Non-finite forms}

\translatorHD{Finally, Adjarian lists the following non-finite forms of this verb (participles or converbs) in Table \ref{tab:Shamakhi:morpho:verb:paradigm:participle}. Unfortunately, he doesn't give names to these forms. So I have to guess what they are based on phonological similarities to SEA. } 

\begin{table}[H]
 \centering
 \caption{Participles or converbs <\armenian{դերբայներ}> for the verb `to like' in the Shamakhi dialect}
 \label{tab:Shamakhi:morpho:verb:paradigm:participle}
 \begin{tabular}{|ll|ll|ll|l|}
\hline & & \multicolumn{2}{l|}{Shamakhi} & \multicolumn{2}{l|}{cf. SEA} & \\
 Infinitive& & siɾ-i-l & \armenian{սիրիլ} & siɾ-\'e-l & \armenian{սիրել} & $\sqrt{}$-{\thgloss}-{\infgloss} \\
 Present? & (subject participle) & siɾ-i-l-ɑn & \armenian{սիրիլան} & & & $\sqrt{}$-{\thgloss}-{\infgloss}-? \\
 Past & (perfective) & siɾ-ɑl & \armenian{սիրալ} & siɾ-el & \armenian{սիրել} & $\sqrt{}$-{\perfcvb} \\
 Future & &siɾ-e-l-ʏ & \armenian{սիրէլիւ} & siɾ-e-l-u & \armenian{սիրէլու} & $\sqrt{}$-{\thgloss}-{\infgloss}-{\futcvb} \\
\hline \end{tabular}
\end{table}

\begin{adjarianpage}\label{page:80}\end{adjarianpage}% should be 80

\section{Literature}

As of now, there is no study on the Shamakhi dialect. I am preparing a study of the Shamakhi dialect with the dramatist Mr. A. Abelian (\armenian{Թատերագիր պր. Ա. Աբէլեան}; \translatorHD{ɑbeljɑn}). I have collected the previous information from this work. There are few published works that use this dialect. The following are the primary ones:


{\litoverview}

\begin{itemize}
 \item Literature with the Shamakhi dialect
 \begin{itemize}
 \item \armenian{Ալ. Աբէլեանց} 
 \begin{itemize}
 \item \armenian{Մկիճի ապահարզանը (Ֆարս-վօդըվիլ}). \armenian{Բագու}, 1899
\item – \armenian{Մկիճի հարսանիքը (պիէս} 1 \armenian{գործ}). \armenian{Բագու}, 1903
 \end{itemize}
\item \armenian{Ս. Գարագաշ – Քաղցած փեսանըրը եւ Գէօգարջինի բալան. Բագու}, 1898
\item \armenian{Շիրվանզադէ – Նամուս. Թիֆլիս}, 1883. Besides this novel takes place in Shamakhi, oftentimes the author will make the characters use this dialect. At the end of the book, there is also list of words and forms from the Shamakhi dialect. 
 \item \armenian{Մ. արքեպս. Սմբատեան – Նկարագիր Ս. Ստեփաննոսի Վանաց Սաղիանի. Թիֆլիս}, 1896. In this, pages 283-286 have a text sample of the Shamakhi dialect. 

 \end{itemize}
 \end{itemize}

\section{Text samples}

{\sampleoverview}
 
Adjarian's source: See \armenian{Ա. Աբէլեան, Մկիճի հարսանիքը}, pages 5-12 with scientific accuracy. 

\armenian{Խէխճ ապէր։ Ղուշըտ ղա̈ֆա̈սա̈ն փախալ ա, մնացալ էս յէրվիլօվ։ Ինձ հարցունէս, մէղայ Ասսու, ուզում ամ ասամ լափ տէղն ա։ Մարթս պէտկանամ պա̈նի լա̈վ ֆիքիր անի։ Ալչան լա̈վ պտուղ ա, համա դէ նրա չհասածը այնա մին զա̈հրմար։ Իւրիւշ պա̈ն ա̈ նրա հասածը. վօր տինիմ էս էրկու ազուիտ արանքըմը հուպ տամ, շրախկօցը գյէօգն ա̈ պա̈ցրանըմ։ Տիւ ապէրս, ասա, ի՛նքիտ մէծ հօրս յաշը՛մը, քինա̈ցա̈լ էս մին կնիգյ էս տռալ քի վօտը հա̈լա̈ հէրու չէ մէկալ տարի փայիզին տասնութումը տիրա̈վ։ Իլիր Մկիճի պէս։ Ի՛նքիս քսանօխտ տարական}, ... 


\begin{adjarianpage}\label{page:81}\end{adjarianpage}% should be 81

... \armenian{ախչիգ ա̈մ սիրա̈լ յէրէսնօխտ տարական։ Դէ ուղօրթ ա, յէքուցվան օրը ա̈նա̈նց կնի՛գի ա̈թա̈նց տղին կը թօղնի, քու Շիւշտնիտ պէս հէրանց տուն կը փախչի։ Հա̈ր պա՛նիս ա̈լ ա̈թա̈նց ա̈։ Մտկօվ, խէլքօվ, խիտրութունօվ (ռուս. խօրամանկութեամբ}), \armenian{ջուվալլաղութունօվ (թրք. չարամտութեամբ)։ Ու՜ֆ, պէրանս հախ ա ուրախ ա խօսըմ, համա սը՛րտըս մըկըտամ ա։ Քու հօր տունի՛գի բարբադ իլի. սէ՜ր։ Տիւի՛գիթ մին ա̈լա̈մա̈թ ցավ էս։ Քէզի տէսնամ քի դա̈րդամահ, իլէս, սատկէս, տմկէս, չօրանաց, տիզ իլէս կպչէս դուվարան, ա̈՛լ պուք չի կյա̈ս։ Ա՛չի (թրք. ա՛յ մարդ) ա̈ջա̈բ խաթի չընգա՞նք։ Տրանա առաչ յէս ի սիրաահարվօղնէրի վէրտ ծըծաղըմ, հիմիգ քի ինձ ա̈մ միտիգ անըմ, լափ ծէր արէվ, ծըծաղ քի չէ, խէխչս ա կյա̈մ ինձ, խէխչս։ Ախր Մկիճը հօ՜վ, սէ՜րը հօվ։ Մկիճը հորդէ՜, Անթառանը հօրդէ՜։ Տէս՛ պա̈՛նի հօրդէ ա հասալ, քի Անթաօանիս վէրտ շարաթրանք ալ ամ կիրա̈լ։} … \armenian{Անգօջ տիր, տէս հի՛նչ սըրտի կանձ խօսկէր ա։}

 \armenian{Մազէրթ սէվ հիւլէօր հիւլէօր},

 \armenian{Պռօշնէրըտ կըլօր կըլօր։}

 \armenian{Ժամի տուռնան լէն ուսէրիտ},

 \armenian{Մատաղ արա, ա՛ս Մկիճիտ։}

~\\

 \armenian{Անթառանս հավ կը խաշի},

 \armenian{Հօրի մէչան ճիւր կը քաշի։}

 \armenian{Հաստ կռնէրը սըրտ կը մաշի},

 \armenian{Մատաղ կանի աս Մկիճին։}

~\\

 \armenian{Տափը սիպտա̈գ նախշ ունքէրիտ},

 \armenian{Շէկ մազէրօվ խէլունք կլխիտ},

 \armenian{Տէղին խունգի յէրդան վիզիտ},

 \armenian{Թօղ փըթաթվի աս Մկիճիտ։}

~\\

 \armenian{Ճակատըտ ա վօսկի հէյլի},

 \armenian{Յէս Մէժլում ամ, տիւ մին Լէյլի},

 \armenian{Վօր քէզ հազար սիրօղ իլի},

 \armenian{Ցէքըտ մէկնիր աս Մկիճիտ։}


~\\

 \armenian{Յէս ծէր բախչին բաղ ամ ասըմ}.

 \armenian{Յէկ քաշամք դամաղ ամ ասըմ}.

 \armenian{Զարգարի պէս հաղ ամ ասըմ},

 \armenian{Ցէք մի՛ քիցիլ ա̈ս Մկիճիտ։}






\chapter{Astrakhan}

\section{Background}

\begin{adjarianpage}\label{page:82}\end{adjarianpage}% should be 82

This dialect is spoken primarily in the city of Astrakhan. This dialect also includes the various corners of the North Caucasus. When the first pioneers of Eastern literature wants to establish a literary language, they first chose the Astrakhan dialect as the basis for the literary language. But they quickly left it and chose the Yerevan dialect, and it eventually took its present form. 

There is no study on the Astrakhan dialect. The only thing we have are short pieces of information on this dialect in the work of Patkanian \citep[24]{Patkanian-1869-RussianDialects}. Patkanian considers this dialect to be extremely close to the literary language, and thus thinks it is excessive to talk more about it. 

As for published samples of the Astrakhan dialect, the first are excerpts in the novels of Raphael Patkanian (\armenian{Ռափայէլ Պատկանեան}; \translatorHD{SEA: /rɑpʰɑjel pɑtkɑnjɑn/}); see his \armenian{Երկասիրութիւնները} (1893, \armenian{հատ. Բ. էջ} 18-19, 23-24, 75, 76, 178-179, 183-186, 192-193, 210, 218-222, and 231-232). There is abundant material in the periodicals of the Astrakhan: Lraber (\citeauthor{LraberAstrakhan}; \translatorHD{SEA: /ləɾɑbeɾ/}) and Gorc (\armenian{Գործ}; \translatorHD{SEA: /ɡoɾt͡s/}).\footnote{\translatorHD{Unfortunately, I couldn't find an online record of a journal called from Astrakhan. There are many such periodicals however with the same name from elsewhere in modern Azerbaijan. }} But these unfortunately don't have perfect scientific accuracy. 

\textit{Footnote from Adjarian}: My deep gratitude to the most honorable Father G. Mkrtumian from Astrakhan (\armenian{Աստրախանցի Արժանապատիւ Տէր Գ. քհ. Մկրտումեանին}; \translatorHD{SEA: /teɾ məkəɾtumjɑn/}), who was kind enough to offer me issues from \textit{Lraber} (\citeauthor{LraberAstrakhan}) that have the best samples of the Astrakhan dialect.

Judging by the language of these publications, we shall see that Patkanian's ideas are not correct. The Astrakhan dialect occupies a middle ground between the Shamakhi and Yerevan dialects, but it is different from both.

\section{Phonology}

The consonants follow the phonetic rules of Shamakhi or Karabakh, but its vowels... 



\begin{adjarianpage}\label{page:83}\end{adjarianpage}% should be 83

... generally follow the Yerevan system. 

\subsection{Consonant voicing}
In this way, the voiced sounds of Old Armenian have become voiceless, and they are unchanged only after nasals (Table \ref{tab:Astrakhan:phono:change:voice} and sentence ).

\begin{table}[H]
 \centering
 \caption{Consonant voicing changes from Classical Armenian to the Astrakhan dialect}
 \label{tab:Astrakhan:phono:change:voice}
 \begin{tabular}{|l|ll|ll|ll|}
 \hline & \multicolumn{2}{l|}{Classical Armenian}& \multicolumn{2}{l|}{> Astrakhan }& \multicolumn{2}{l|}{cf. SEA }
 \\
 \armenian{՝}thin' & bɑɾɑk & \armenian{բարակ} & pɑɾɑk & \armenian{պարակ} & bɑɾɑk & \armenian{բարակ} \\
 \armenian{՝}head' & ɡəluχ & \armenian{գլուխ} & kluχ & \armenian{կլուխ} & ɡəluχ & \armenian{գլուխ} \\
 \armenian{՝}water' & d͡ʒuɾ & \armenian{ջուր} & t͡ʃuɾ & \armenian{ճուր} & d͡ʒuɾ & \armenian{ջուր} \\
 \armenian{՝}to put' & dənel & \armenian{դնել} & tinel & \armenian{տինէլ} & dənel & \armenian{դնել} \\
`wool' & buɾd & \armenian{բուրդ} & puɾtʰ & \armenian{պուրթ} & buɾtʰ & \armenian{բուրդ} \\ 
 `sound' & d͡zɑi̯n & \armenian{ձայն} & t͡sen & \armenian{ծէն} & d͡zɑjn & \armenian{ձայն} \\ 
 `egg' & d͡zu& \armenian{ձու} &t͡su & \armenian{ծու}& d͡zu& \armenian{ձու} \\
\hline
 \end{tabular}
 
\end{table}

\begin{exe}
 \ex \begin{xlist}
 \ex Astrakhan \gll 
 inɡ-n-əm e-m \\
 fall-{\vx}-{\impfcvb} {\aux}-1{\sg} \\
 \trans `I fall.' \\
 \armenian{ինգնըմ էմ}
 \ex cf. SEA \gll 
 ənk-n-um e-m \\
 fall-{\vx}-{\impfcvb} {\aux}-1{\sg} \\
 \trans `I fall.' \\
 \armenian{ընկնում եմ}
 \end{xlist}
\end{exe}

\subsection{Vowel inventory} 
Among vowels, the sounds /æ, ʏ, œ/ <\armenian{ա}̈, \armenian{իւ, էօ}> are missing. 


\subsection{Vowel changes}
There is no rule of deleting vowels before the stressed syllable.


There are some notable vowel changes and diphthong changes.

\subsubsection{Classical Armenian /ɑi̯/ <\armenian{այ}> }

Classical Armenian /ɑi̯/ <\armenian{այ}> became /e/ <\armenian{է}> (Table \ref{tab:Astrakhan:phonology:soundChange:diphthong:ɑi:e}). 


\begin{table}[H]
 \centering
 \caption{Change from Classical Armenian /ɑi̯/ <\armenian{այ}> to /e/ <\armenian{է}> in the Astrakhan dialect}
 \label{tab:Astrakhan:phonology:soundChange:diphthong:ɑi:e}
 \begin{tabular}{|l| ll|ll| ll|}
 \hline & \multicolumn{2}{l|}{Classical Armenian} &\multicolumn{2}{l|}{> Astrakhan} & \multicolumn{2}{l|}{cf. SEA} \\ 
 `sound' & d͡zɑi̯n & \armenian{ձայն} & t͡sen & \armenian{ծէն} & d͡zɑjn & \armenian{ձայն} \\ 
`that' & ɑi̯n & \armenian{այն} & en & \armenian{էն} & ɑjn & \armenian{այն} \\ 
`wide' & lɑi̯n & \armenian{լայն} & len & \armenian{լէն} & lɑjn & \armenian{լայն} \\ 
`on' & i veɾ\'ɑi̯ & \armenian{ի վերայ} & veɾ\'e & \armenian{վէրէ} & vəɾ\'ɑ & \armenian{վրա} \\ 
 \hline 
 \end{tabular}
\end{table}


\subsubsection{Classical Armenian /iu̯, oi̯, u/ <\armenian{իւ, ոյ, ու}> }

Classical Armenian /iu̯, oi̯, u/ <\armenian{իւ, ոյ, ու}> became /e/ <\armenian{ու}> (Table \ref{tab:Astrakhan:phonology:soundChange:diphthong:ɑi:u}). 


\begin{table}[H]
 \centering
 \caption{Change from Classical Armenian /iu̯, oi̯, u/ <\armenian{իւ, ոյ, ու}> to /e/ <\armenian{ու}> in the Astrakhan dialect}
 \label{tab:Astrakhan:phonology:soundChange:diphthong:iu:u}
 \begin{tabular}{|l| ll|ll| ll|}
 \hline & \multicolumn{2}{l|}{Classical Armenian} &\multicolumn{2}{l|}{> Astrakhan} & \multicolumn{2}{l|}{cf. SEA} \\ 
`hundred' & hɑɾiu̯ɾ & \armenian{հարիւր} & hɑɾuɾ & \armenian{հարուր} & hɑɾjuɾ & \armenian{հարյուր} \\ 
 \armenian{՝}light' & loi̯s & \armenian{լոյս}& lus & \armenian{լուս} & lujs & \armenian{լույս} \\
`fish' &d͡zukən & \armenian{ձուկն} & t͡suknə & \armenian{ծուգնը} & d͡zuk & \armenian{ձուկ} \\ 
 \hline 
 \end{tabular}
\end{table}

\section{Morphology}
\subsection{Noun inflection or declension}

Case declensions are the same as in Yerevan: genitive /-i/ <\armenian{ի}>, ablative /-it͡sʰ/ <\armenian{ից}>, instrumental /-ov/ <\armenian{օվ}>, locative /-əm/ <\armenian{ըմ}> (Table \ref{tab:Astrakhan:morphology:noun:decl}).\footnote{\translatorHD{For Astrakhan /met͡ʃʰ-əm-ə/, I suspect the final schwa is a typo because in SEA the definite suffix is banned after the locative suffix. } }


\begin{table}[H]
 \centering
 \caption{Sample of noun declension in the Astrakhan dialect}
 \label{tab:Astrakhan:morphology:noun:decl}
 \begin{tabular}{|l| ll| ll|}
 \hline & \multicolumn{2}{l|}{Astrakhan} & \multicolumn{2}{l|}{cf. SEA} \\ 
`close-{\abl}' & mot-it͡sʰ & \armenian{մօտից} &mot-it͡sʰ & \armenian{մոտից}\\ 
`head-{\abl}' & klχ-it͡sʰ & \armenian{կլխից} &ɡəlχ-it͡sʰ & \armenian{գլխից}\\ 
`outside-{\abl}' & tus-it͡sʰ & \armenian{տուսից} &dəɾs-it͡sʰ & \armenian{դրսից}\\
`speech-{\ins}' & χoskʰ-ov & \armenian{խօսքօվ} &χoskʰ-ov & \armenian{խոսքով}\\ 
`eye-{\ins}' & ɑt͡ʃʰkʰ-ov & \armenian{աչքօվ} &ɑt͡ʃʰkʰ-ov & \armenian{աչքով}\\ 
`inside-{\locgloss}-({\defgloss})' & met͡ʃʰ-əm-ə & \armenian{մէչըմը} &met͡ʃʰ-um & \armenian{մեջում}\\ 
`place-{\locgloss}' & teʁ-əm & \armenian{տէղըմ} &teʁ-um & \armenian{տեղում}\\ 
\hline 
 \end{tabular}
\end{table}


In the excerpts from Patkanian, the ablative uses the Karabakh system with the forms /-ɑ, -ɑn/ <\armenian{ա, ան}>. But this is not found in the others. Sometimes, instead of ablative /-it͡sʰ/ <\armenian{ից}>, I've seen the formative /-it͡s/ <\armenian{իծ}> (Table \ref{tab:Astrakhan:morphology:noun:decl}). 


\begin{table}[H]
 \centering
 \caption{Ablative /-it͡s/ in the Astrakhan dialect}
 \label{tab:Astrakhan:morphology:noun:its}
 \begin{tabular}{|l| ll| ll|}
 \hline & \multicolumn{2}{l|}{Astrakhan} & \multicolumn{2}{l|}{cf. SEA} \\ 
`mouth-{\abl}' & peɾɑn-it͡s & \armenian{պէրանիծ} &beɾɑn-it͡sʰ & \armenian{բերանից}\\ 
`writing-{\thgloss}-{\infgloss}-{\abl}' & kɾ-e-l-it͡s & \armenian{կրէլիծ} &ɡəɾ-e-l-ut͡sʰ & \armenian{գրելուց} \\
`outside-{\abl}' & tus-it͡s & \armenian{տուսիծ} &dəɾs-it͡sʰ & \armenian{դրսից}\\ 
`speech-{\ins}' & χoskʰ-ov & \armenian{խօսքօվ} &χoskʰ-ov & \armenian{խոսքով}\\ 
`eye-{\ins}' & ɑt͡ʃʰkʰ-ov & \armenian{աչքօվ} &ɑt͡ʃʰkʰ-ov & \armenian{աչքով}\\ 
`inside-{\locgloss}-({\defgloss})' & met͡ʃʰ-əm-ə & \armenian{մէչըմը} &met͡ʃʰ-um & \armenian{մեջում}\\ 
`place-{\locgloss}' & teʁ-əm & \armenian{տէղըմ} &teʁ-um & \armenian{տեղում}\\ 
\hline 
 \end{tabular}
\end{table}

This change from the Classical sound /t͡sʰ/ <\armenian{ց}> is also found in the word /kʰɑsibɑnot͡s/ <\armenian{քասիբանօծ}> `\armenian{աղքատանոց}'. \translatorHD{This word is made up of a borrowed root /kʰɑsib/ plus an Armenian derivational suffix: SEA /-ɑnot͡sʰ/ <-\armenian{անոց}>.}\footnote{\translatorHD{Adjarian says that this root /kʰɑsib/ is from Turkish, but Wiktionary treats the Azerbaijani form <kasıb> as the source for Armenian. Ottoman Turkish also had a cognate \textarab{كاسب} <kasib>. }}

\subsection{Pronoun inflection or declension}

The pronouns almost all follow the Yerevan system (Table \ref{tab:Astrakhan:morphology:pronoun:sample}). \translatorHD{Adjarian doesn't translate these pronouns; so I'm guessing what they are.}


\begin{table}[H]
 \centering
 \caption{Sample of pronouns in the Astrakhan dialect}
 \label{tab:Astrakhan:morphology:pronoun:sample}
 \begin{tabular}{|l ll|}
\hline 
personal 1SG {\nom} `I' &jes & \armenian{յէս} \\
personal 2SG {\nom} `you' &tu & \armenian{տու} \\
personal 3SG {\nom} `he' &en & \armenian{էն} \\
personal 1SG {\abl} `from me' &ind͡z-ɑn-it͡sʰ & \armenian{ինձանից} \\
personal 1PL {\dat} `to us' &mez-i & \armenian{մէզի} \\
personal 1PL {\locgloss} `in us' &mez-ɑn-əm & \armenian{մէզանըմ} \\
demonstrative medial {\sg} {\gen} `that' &tɾɑ & \armenian{տրա} \\
demonstrative medial {\sg} {\dat} `that' &tɾɑn & \armenian{տրան} \\
demonstrative proximal {\pl} {\nom} `these' &sɾɑnkʰ & \armenian{սրանք} \\
demonstrative medial {\pl} {\nom} `those' &tɾɑnkʰ & \armenian{տրանք} \\
demonstrative distal {\pl} {\nom} `those yonder' &nɾɑnkʰ & \armenian{նրանք} \\
demonstrative proximal {\pl} {\acc} `these' &sɾɑnt͡sʰ & \armenian{սրանց} \\
demonstrative medial {\pl} {\acc} `those' &tɾɑnt͡sʰ & \armenian{տրանց} \\
demonstrative medial {\pl} {\abl} `those' &tɾɑnt͡sʰ-it͡sʰ & \armenian{տրանցից} \\
demonstrative proximal {\sg} {\gen} `this' &estuɾ & \armenian{էստուր} \\
demonstrative proximal {\pl} {\nom} `these' &estunkʰ & \armenian{էստունք} \\
\hline 
 \end{tabular}
\end{table}


What's interesting is the form /estuɾ-neɾ-i/ `this-{\pl}-{\dat}' and the nominally declined forms of the word `who' (Table \ref{tab:Astrakhan:morphology:pronoun:who}). 



\begin{table}[H]
 \centering
 \caption{Paradigm of the interrogative pronoun `who' in the Astrakhan dialect}
 \label{tab:Astrakhan:morphology:pronoun:who}
 \begin{tabular}{|l| ll| }
 \hline 
 `who' ({\nom}) & hov & \armenian{հօ՞վ} \\
`to who' ({\dat}) & hov-i & \armenian{հօվի՞}\\
`from who' ({\abl}) & hov-it͡sʰ & \armenian{հօվի՞ց} \\
 `who' ({\pl}) & hov-eɾ & \armenian{հօվէ՞ր}\\
 `to who' ({\pl}-{\dat}) & hov-eɾ-i & \armenian{հօվէրի՞}\\
 `from who' ({\pl}-{\abl}) & hov-eɾ-it͡sʰ & \armenian{հօվէրի՞ց}\\
\hline 
 \end{tabular}
\end{table}

\subsection{Verb inflection or conjugation}

In verbal conjugation, the formation of the present tense is similar to the Yerevan dialect. 

\subsubsection{Copula with /e/ and 3SG /ɑ/}
The copula is in Table \ref{tab:Astrakhan:morpho:verb:copula}. 


\begin{table}[H]
 \centering
 \caption{Copula in the Astrakhan dialect}
 \label{tab:Astrakhan:morpho:verb:copula}
 \begin{tabular}{|l|ll| ll| }
 \hline & \multicolumn{2}{l|}{Astrakhan} & \multicolumn{2}{l|}{cf. SEA} \\ 
1SG `I am' &e-m & \armenian{էմ} &e-m & \armenian{եմ}\\ 
2SG `you are' &e-s & \armenian{էս} &e-s & \armenian{ես}\\ 
3SG `he is' &ɑ & \armenian{ա} &e & \armenian{է}\\ 
1PL `we are' &e-nkʰ & \armenian{էնք} &e-ŋkʰ & \armenian{ենք}\\ 
2PL `you are' &e-kʰ & \armenian{էք} &e-kʰ & \armenian{եք}\\ 
3PL `they are' &e-n & \armenian{էն} &e-n & \armenian{են}\\ 
& \multicolumn{2}{l|}{{\aux}-{\agr}}& \multicolumn{2}{l|}{{\aux}-{\agr}} \\
\hline 
 \end{tabular}
\end{table}
 
\subsubsection{Indicative present forms}

The formative (\translatorHD{of the imperfective converb}) is /-əm, -is/ <\armenian{ըմ, իս}> (\ref{sent:Astrakhan:morpho:verb:presIndc}).

\begin{exe}
 \ex Astrakhan \label{sent:Astrakhan:morpho:verb:presIndc}
 \begin{xlist}
 \ex \gll ɑs-əm e-m \\
 say-{\impfcvb} {\aux}-1{\sg} \\
 \trans `I say.'\\
 \armenian{ասըմ էմ} 
 \ex \gll ls-əm e-m \\
 hear-{\impfcvb} {\aux}-1{\sg} \\
 \trans `I hear.'\\
 \armenian{լսըմ էմ}
 \ex \gll inɡ-n-əm ɑ \\
 fall-{\vx}-{\impfcvb} {\aux} \\
 \trans `He falls.'\\
 \armenian{ինգնըմ ա} 
 \ex \gll k-ɑ-l-is e-nkʰ \\
 come-{\thgloss}-{\infgloss}-{\impfcvb} {\aux}-1{\pl} \\
 \trans `We are coming.'\\
 \armenian{կալիս էնք}
 \ex \gll lɑt͡sʰ e-kʰ il-əm \\
 cry {\aux}-2{\pl} be-{\impfcvb} \\
 \trans `You.{\pl} are crying.'\\
 \armenian{լաց էք իլըմ}
 \ex \gll t͡ʃʰ-e-n ls-əm \\
 {\neggloss}-{\aux}-2{\pl} hear-{\impfcvb} \\
 \trans `They don't hear.'\\
 \armenian{չէն լսըմ}.
 \end{xlist}
\end{exe}

The Karabakh-style forms are used (\ref{sent:Astrakhan:morpho:verb:presIndcKara}). \translatorHD{It's not clear to me how this is like Karabakh. }

\begin{exe}
 \ex Astrakhan \label{sent:Astrakhan:morpho:verb:presIndcKara}
 \begin{xlist}
 \ex \gll k-ɑ-m ɑ, \textit{or} k-ɑ-l-is ɑ \\
 come-{\thgloss}-? {\aux}, ~ come-{\thgloss}-{\infgloss}-{\impfcvb} {\aux} \\
 \trans `He comes.'\\
 \armenian{կալիս էնք}, or \armenian{կալիս ա}
 \ex \gll t-ɑ-m ɑ, \textit{or} t-ɑ-l-is ɑ \\
 give-{\thgloss}-? {\aux}, ~ give-{\thgloss}-{\infgloss}-{\impfcvb} {\aux} \\
 \trans `He gives.'\\
 \armenian{տամ ա}, or \armenian{տալիս ա} 
 \end{xlist}
\end{exe}

The verbs with the vowel /u/ <\armenian{ու}> get the formative /-um/ <\armenian{ում}> (\ref{sent:Astrakhan:morpho:verb:presU}). 
\begin{exe}
 \ex Astrakhan \label{sent:Astrakhan:morpho:verb:presU}
 \begin{xlist}
 \ex \gll uz-um e-m \\
 want-{\impfcvb} {\aux}-1{\sg} \\
 \trans `I want.'\\
\armenian{ուզում էմ} 
\ex \gll uz-um ɑ \\
 want-{\impfcvb} {\aux} \\
 \trans `He wants.'\\
\armenian{ուզում ա} 
\end{xlist}
\end{exe}

\subsubsection{Vowel assimilation (/e/ <\armenian{է}> to /i/ <\armenian{ի}>) and past /-m/ <\armenian{մ}> }

In the Astrakhan dialect, the sound /e/ <\armenian{է}> becomes /i/ <\armenian{ի}> when before an /i/ <\armenian{ի}>. 

Besides this, it receives the first person formative /-m/ <\armenian{մ}> formative in the imperfective and perfective, based off of the present tense. For example, the copula can or not (Table \ref{tab:Astrakhan:morpho:verb:copulaPast}) 

\begin{table}[H]
 \centering
 \caption{Past copula or past auxiliary in the Astrakhan dialect}
 \label{tab:Astrakhan:morpho:verb:copulaPast}
 \begin{tabular}{|l|ll| ll| ll| }
 \hline & \multicolumn{4}{l|}{Astrakhan (with and without assimilation)} & \multicolumn{2}{l|}{cf. SEA} \\ 
1SG `I was' &i-i-m & \armenian{իիմ} &e-i-$\emptyset$ & \armenian{էի} & ej-i-$\emptyset$ & \armenian{էի}\\ 
2SG `you were' &i-i-ɾ & \armenian{իիր} &e-i-ɾ & \armenian{էիր}&ej-i-ɾ & \armenian{էիր}\\ 
3SG `he was' &e-$\emptyset$-ɾ & \armenian{էր} &e-$\emptyset$-ɾ & \armenian{էր}&e-$\emptyset$-ɾ & \armenian{էր}\\ 
1PL `we were' &i-i-nkʰ & \armenian{իինք} &e-i-ŋkʰ & \armenian{էինք}&ej-i-ŋkʰ & \armenian{էինք}\\ 
2PL `you were' &i-i-kʰ & \armenian{իիք} &e-i-kʰ & \armenian{էիք} &ej-i-kʰ & \armenian{էիք}\\ 
3PL `they were' &i-i-n & \armenian{իին} &e-i-n & \armenian{էին}&ej-i-n & \armenian{էին}\\ 
& \multicolumn{2}{l|}{{\aux}-{\pst}-{\agr}}& \multicolumn{2}{l|}{{\aux}-{\pst}-{\agr}}& \multicolumn{2}{l|}{{\aux}-{\pst}-{\agr}} \\
\hline 
 \end{tabular}
\end{table}


\begin{adjarianpage}\label{page:84}\end{adjarianpage}% should be 84

\translatorHD{The negative also shows assimilation (Table \ref{tab:Astrakhan:morpho:verb:copulaPastNeg}) .} 

 

\begin{table}[H]
 \centering
 \caption{Negative past copula or past auxiliary in the Astrakhan dialect}
 \label{tab:Astrakhan:morpho:verb:copulaPastNeg}
 \begin{tabular}{|l|ll | ll| }
 \hline & \multicolumn{2}{l|}{Astrakhan (with assimilation)} & \multicolumn{2}{l|}{cf. SEA} \\ 
1SG `I was not' &t͡ʃʰ-i-i-m & \armenian{չիիմ} & t͡ʃʰ-ej-i-$\emptyset$ & \armenian{չէի}\\ 
2SG `you were not' &t͡ʃʰ-i-i-ɾ &\armenian{չ իիր} &t͡ʃʰ-ej-i-ɾ & \armenian{չէիր}\\ 
3SG `he was not' &t͡ʃʰ-e-$\emptyset$-ɾ & \armenian{չէր}&t͡ʃʰ-e-$\emptyset$-ɾ & \armenian{չէր}\\ 
1PL `we were not' &t͡ʃʰ-i-i-nkʰ & \armenian{չիինք} & t͡ʃʰ-ej-i-ŋkʰ & \armenian{չէինք}\\ 
2PL `you were not' &t͡ʃʰ-i-i-kʰ & \armenian{չիիք} & t͡ʃʰ-ej-i-kʰ & \armenian{չէիք}\\ 
3PL `they were not' &t͡ʃʰ-i-i-n & \armenian{չիին} &t͡ʃʰ-ej-i-n & \armenian{չէին}\\ 
 & \multicolumn{2}{l|}{{\neggloss}-{\aux}-{\pst}-{\agr}}& \multicolumn{2}{l|}{{\neggloss}-{\aux}-{\pst}-{\agr}} \\
\hline 
 \end{tabular}
\end{table}

 
 \translatorHD{See (\ref{sent:Astrakhan:morpho:verb:conjEx}) for examples of assimilation and 1SG /-m/ in conjugation}
 
 \begin{exe}
 \ex \label{sent:Astrakhan:morpho:verb:conjEx}
 \begin{xlist}
 \ex Astrakhan
 \gll ɑs-əm e-i-n, \textit{or} ɑs-əm i-i-n \\
 say-{\impfcvb} {\aux}-{\pst}-3{\pl}, ~ say-{\impfcvb} {\aux}-{\pst}-3{\pl} \\
 \trans `They were saying.'\\
 \armenian{ասըմ էին} or \armenian{ասըմ իին}
\ex \begin{xlist}
\ex Astrakhan \gll inɡ-n-i-i-m, inɡ-n-i-i-nkʰ\\
 fall-{\vx}-{\thgloss}-{\pst}-1{\sg}, fall-{\vx}-{\thgloss}-{\pst}-1{\pl}\\
 \trans \armenian{ինգնիիմ, ինգնիինք}
 \ex cf. SEA \gll əŋk-n-ej-i-$\emptyset$, əŋk-n-ej-i-ŋkʰ\\
 fall-{\vx}-{\thgloss}-{\pst}-1{\sg}, fall-{\vx}-{\thgloss}-{\pst}-1{\pl}\\
 \trans `(If) I were to fall; (if) we were to fall'\\
 \armenian{ընկնեի, ընկնեինք}
 \end{xlist}
 \ex \begin{xlist}
\ex Astrakhan \gll kə-χɑʁ-ɑj-i-m, k-eɾtʰ-ɑj-i-m, kə-peɾ-e-i-m\\
 {\fut}-play-{\thgloss}-{\pst}-1{\sg}, {\fut}-go-{\thgloss}-{\pst}-1{\sg}, {\fut}-bring-{\thgloss}-{\pst}-1{\sg}\\
 \trans \armenian{կը խաղայիմ, կէրթայիմ, կը պէրէիմ}
 \ex cf. SEA \gll kə-χɑʁ-ɑj-i-$\emptyset$, k-eɾtʰ-ɑj-i-$\emptyset$, kə-beɾ-ej-i-$\emptyset$\\
 {\fut}-play-{\thgloss}-{\pst}-1{\sg}, {\fut}-go-{\thgloss}-{\pst}-1{\sg}, {\fut}-bring-{\thgloss}-{\pst}-1{\sg}\\
 \trans `I would have played, I would have gone, I would have brought.'\\
 \armenian{կխաղայի, կերթայի, կբերեի}
 \end{xlist} 

 \end{xlist}
 \end{exe}
 
 \translatorHD{Similarly the aorist or past perfective uses the 1SG marker /-m/ in Astrakhan but not SEA. Note that Adjarian doesn't give translations or reflexes for most of these perfectives; I had to guess what they meant and then guess their SEA cognate. Note that some of these perfectives have a larger structure in their more conservative SEA cognates. }



\begin{table}[H]
 \centering
 \caption{Use of 1SG marker /-m/ <\armenian{մ}> in the aorist or past perfective in the Astrakhan dialect}
 \label{tab:Astrakhan:morpho:verb:pastPerfM}
 \begin{tabular}{|l|ll | ll| }
\hline & \multicolumn{2}{l|}{Astrakhan} & \multicolumn{2}{l|}{cf. SEA} \\ 
 \hline 
`I said' &ɑs-ɑ-t͡sʰ-i-m & \armenian{ասացիմ} &ɑs-ɑ-t͡sʰ-i-$\emptyset$ &\armenian{ասացի} \\ 
`I went' &kn-ɑ-t͡sʰ-i-m & \armenian{կնացիմ} &ɡən-ɑ-t͡sʰ-i-$\emptyset$ &\armenian{գնացի} \\ 
`I liked' &siɾ-e-t͡sʰ-i-m & \armenian{սիրէցիմ} &siɾ-e-t͡sʰ-i-$\emptyset$ &\armenian{սիրեցի} \\ 
&\multicolumn{2}{l|}{$\sqrt{}$-{\thgloss}-{\aor}-{\pst}-1{\sg}}&\multicolumn{2}{l|}{$\sqrt{}$-{\thgloss}-{\aor}-{\pst}-1{\sg}}\\
\hline 
 `I said' &ɑs-ɑ-m & \armenian{ասամ} &ɑs-ɑ-t͡sʰ-i-$\emptyset$ &\armenian{ասացի} \\ 
`I gave' &tv-ɑ-m & \armenian{տվամ} &təv-e-t͡sʰ-i-$\emptyset$ &\armenian{տվեցի} \\ 
`I brought' &peɾ-ɑ-m &\armenian{պէրամ} &beɾ-e-t͡sʰ-i-$\emptyset$ &\armenian{բերեցի} \\ 
`I called' &kɑnt͡ʃʰ-ɑ-m &\armenian{կանչամ} &kɑnt͡ʃʰ-e-t͡sʰ-i-$\emptyset$ &\armenian{կանչեցի} \\ 
`I put ({\pst})' &tiɾ-ɑ-m &\armenian{տիրամ} &dəɾ-e-t͡sʰ-i-$\emptyset$ &\armenian{դրեցի} \\ 
`I removed' (?) &hɑn-ɑ-m &\armenian{հանամ} &hɑn-e-t͡sʰ-i-$\emptyset$ &\armenian{հանեցի} \\ 
`I allowed' &tʰoʁ-ɑ-m &\armenian{թօղամ} & tʰoʁ-e-t͡sʰ-i-$\emptyset$ &\armenian{թողեցի} \\ 
&\multicolumn{2}{l|}{$\sqrt{}$-{\pst}-1{\sg}}&\multicolumn{2}{l|}{$\sqrt{}$-{\thgloss}-{\aor}-{\pst}-1{\sg}}\\
\hline 
`I came' &ek-ɑ-m & \armenian{էկամ} & jek-ɑ-$\emptyset$& \armenian{եկա} \\ 
 `I found' &kʰtʰ-ɑ-m &\armenian{քթամ} & ɡət-ɑ-$\emptyset$& \armenian{գտա} \\ 
`I took' &ɑɾ-ɑ-m &\armenian{առամ} &ɑr-ɑ-$\emptyset$ &\armenian{առա} \\ 
& \multicolumn{2}{l|}{$\sqrt{}$-{\pst}-1{\sg}}&\multicolumn{2}{l|}{$\sqrt{}$-{\pst}-1{\sg}}\\
 \hline 
 \end{tabular}
\end{table}

\subsubsection{Past participle or perfective converb with /-el/ <\armenian{էլ}>}

The past participle ends in /-el/ <\armenian{էլ}>, and in this way it forms the present perfect and past perfect (\armenian{յարակատարն ու գերակատարը}) in (\ref{sent:Astrakhan:morpho:verb:perfectivePart}). \translatorHD{Adjarian did not provide translations or reflexes, so I had to guess their meanings. }

\begin{exe}
 \ex Astrakhan \label{sent:Astrakhan:morpho:verb:perfectivePart}
 \begin{xlist}
 \ex \gll siɾ-el ɑ \\ 
 like-{\perfcvb} {\aux} \\
 \trans `I have liked.' \\ 
 \armenian{սիրէլ ա}
 \ex \gll ek-el e-i-m \\ 
 come-{\perfcvb} {\aux}-{\pst}-1{\sg} \\
 \trans `I had come.' \\ 
 \armenian{էկէլ էիմ}
 \ex \gll kʰtʰ-el e-s\\ 
 find-{\perfcvb} {\aux}-2{\sg} \\
 \trans `You have found.' \\ 
 \armenian{քթէլ էս}
 \ex \gll inɡ-el e-n\\ 
 fall-{\perfcvb} {\aux}-3{\pl} \\
 \trans `They have fallen.' \\ 
 \armenian{ինգէլ էն}
 \ex \gll t͡ʃʰ-i-i-m ls-el \\ 
 {\neggloss}-{\aux}-{\pst}-1{\sg} hear-{\perfcvb}\\
 \trans `I had not listened.' \\
 \armenian{չիիմ լսէլ}
 \ex \gll ɑmɑnt͡ʃʰ-el i-i-m \\ 
 shy?-{\perfcvb} {\aux}-{\pst}-1{\sg} \\
 \trans I'm not but I think this means `I had felt shy.' \\
 \armenian{ամանչէլ իիմ} 
 \end{xlist}
\end{exe}

The reflex of verb /linel/ <\armenian{լինել}> `to be' uses the formative \armenian{էր} /eɾ/ in order to distinguish the consonants (\ref{sent:Astrakhan:morpho:verb:perfectivePartLinel}).\footnote{\translatorHD{Adjarian treats the verb in (\ref{sent:Astrakhan:morpho:verb:perfectivePartLinel:Fut}) as made up of a past participle or perfective converb. However, his SEA translation /linelu e/ <\armenian{լինելու}> uses a future converb. I think he gave the incorrect description for this sentence's morphology; it does not use a perfective converb. }}


\begin{exe}
 \ex Astrakhan \label{sent:Astrakhan:morpho:verb:perfectivePartLinel}
 \begin{xlist}
 \ex \gll il-eɾ e-n \\ 
 be-{\perfcvb} {\aux}-3{\pl} \\
 \trans `They have been.' \\ 
 \armenian{իլէր էն} 
 \ex \gll il-eɾ ɑ \\ 
 be-{\perfcvb} {\aux} \\
 \trans `He has been.' \\ 
 \armenian{իլէր ա}
 \ex \gll il-e-ɾ-u ɑ \\ 
 be-{\thgloss}-{\infgloss}-{\futcvb} {\aux} \\
 \trans `He will be.' \label{sent:Astrakhan:morpho:verb:perfectivePartLinel:Fut}\\ 
 \armenian{իլէրու ա} 
 \end{xlist}
\end{exe}

This sound change is also found in the reflex of the Classical conjunction /ɑjl/ <\armenian{այլ}> which uses the form /el/ <\armenian{էլ}> or /eɾ/ <\armenian{էր}> in the Astrakhan dialect. 

\subsubsection{Infinitival genitive with /-i/ <\armenian{ի}>}

The genitive of the infinitive is formed with /-i/ <\armenian{ի}>, similarly to the the Karabakh dialect (Table \ref{tab:Astrakhan:morpho:verb:genI}).


\begin{table}[H]
 \centering
 \caption{Infinitives take genitive /-i/ <\armenian{ի}> instead of /-u/ <\armenian{ու}> in the Astrakhan dialect}
 \label{tab:Astrakhan:morpho:verb:genI}
 \begin{tabular}{|l|ll | ll| }
\hline & \multicolumn{2}{l|}{Astrakhan} & \multicolumn{2}{l|}{cf. SEA} \\ 
`to put' ({\gen}) & tn-e-l-i & \armenian{տնէլի} & dən-e-l-u & \armenian{դնելու}\\
`to speak' ({\gen}) & χos-e-l-i & \armenian{խօսէլի} & χos-e-l-u & \armenian{խոսելու}\\
`to sew' ({\gen}) & kɑɾ-e-l-i & \armenian{կարէլի} & kɑɾ-e-l-u & \armenian{կարելու}\\
& \multicolumn{2}{l|}{$\sqrt{}$-{\thgloss}-{\infgloss}-{\gen}} & \multicolumn{2}{l|}{$\sqrt{}$-{\thgloss}-{\infgloss}-{\gen}}\\
 \hline 
 \end{tabular}
\end{table}

\section{Text samples}

{\sampleoverview}

Adjarian's source: Taken from \textit{Lraber} (\citeauthor{LraberAstrakhan}, year 1909, number 19)

– \armenian{Ադա Մոսկովից Լէքսէյ Իվանիճի տղան էր էկէլ, Միշան. կընըմ էր Պետրովըծկա։ Քշէրը էկաւ. դէ մեր տուն ցած էկավ. ա (Ռուս. իսկ) առաւօտը իննը սահաթին պրօխօդը կընըմ ա։ Դէ սաղ քշէր խօսանք, հօրը հարցրամ. հէ՜յ գիդի տարիներ. ի՜նչ քէֆեր էին արել}…\armenian{՝ Դէ, տա քի, Արտեմ Վանիճ, առաւօտը մին ծի վեր առնենք ստեղի հայի պաները շանց տուր։}

\armenian{Լա՛ւ, ասամ։}

\armenian{Ո՛չ իիմ ասել։}

\armenian{Առաւօտը բագաժը ըդըրգանք, ութին կէս կար, ծի վեր առանք, տուս էկանք։ Սամի առաչինը պէրամ սրան Պետրոս-Պօղոսի ժամի խաչելութիւնի պատկերքը շանց տալի։ Շատ հաւանաւ. ասըմ ա հովի՞ ծեռքաճուրն ա։ Չեմ մանըմ, ասըմ էմ, Րաֆայէլինն ա, թէ նրա աշկերտինը։ Տեսամ, որ շատ խորը խորը մտիկ ա անըմ՝ ուզամ փոքր պարծանալի}. – \armenian{ասըմ էմ, տա խօ}... 

\begin{adjarianpage}\label{page:85}\end{adjarianpage}% should be 85


... \armenian{էսպէս չի՛ իլէր առաչ. դուզ կտաւի վերէն ա քաշած իլէր. էտով մեր էրէսփոխներից մինը պռնել ա՝ էրէսին լակ ա քսել տվել։ Ադա էս խօսքս ասելը իմացամ, տա վերէս պաց չի ինգնի՞լ. քի «այս ի՞նչ վանդալութի՛ւն ա»։ Տեսամ, որ շատ ա տաքանըմ ՝ հանդարտ փեշից քաշամ, ասըմ էմ խըբար էս ինչ խաբար ա՞, փոքր հանդարտիր, ժամըմն էնք, համ էր խառը վախտեր ա. բիրդան պերանիծդ մի խօսկ պաց կթողնես՝ ստակ կկորչենք։} 

\armenian{Ադա տուռնից տուս կալի վախտին՝ սրա աչքէն էլի մին զադ չի ինգնի՞լ. կայնաւ։}

– \armenian{Տա ի՞նչ ա, ասըմ ա, էս տուռնի գըմանի պատը քերվե՞լ ա ի՛նչ ա։}

– \armenian{Չէ՛, ասըմ էմ, ստեղ պան ա կրած. հոր թիվին քըցաղ ա ժամը, եփ օծած ա. հով ա օծել, հովերի հետ։ Կրելիծ էտով էր, բուկվաների վերէն զարվարաղ էն քսած իլէր։ Դէ տարիներ էն անցկացել, ադա մարթ ա վեր ինգնըմ մեռնըմ, ի՞նչ պան ա որ սա էր փչացած իլի, էնա ինգել ա փչանալի վերչն էլ մին քանի հետ ժամը տուսիծ նորոքելի վախտին տըրան էլ բելիտ էն արել. ա՜յ, ասըմ էմ, մին հետ էր բելիտ անեն ՝ ստակ կբարաբարվի։}

\armenian{Ադա սա թազադանիծ չի թընդըւի՞լ։}

– \armenian{Սա հնոութիւն ա, ասըմ ա, ի՞նչպէս կարելի-յ-ա ոնչըչացնել. սա ի՞նչ խելք ա, ի՞նչ հասկացողուլիւն ա։ Լաւ, ասըմ ա, էս ծեր քաղաքըմ մին դանա հասկացող մարթ չի կա՞յ, որ էս պատկերքի համար էր խօսէր. էս}\footnote{\translatorHD{The original page had <\armenian{կս}>. But on page 306, Adjarian provides an erratum that this should be <\armenian{էս}>; I fixed it.}} \armenian{կրածի համար էր։}

– \armenian{Ստեղ ուժ ես ինքս տաքացում. ներողութիւն, ասըմ էմ, շատ իզուր էս մեղի էսպէս անպատիւ անըմ. հասկացող մարթ մեզանըմ ինչքան ուզես. ա՜յ համեցէք մեր ժողովքները, թամաշա արա։ Հէնց մեր տունը ասըմ էմ, էնդուր համար ա քանդըվել, որ շատ հասկացողներ ունենք. դաժը էթէ կուզես՝ անհասկացող մարթ չի կայ, ոխչով հասկացող էն։ Էն պատկերքի պանը լսելը բաշտ մունչերի սպըխվատիտծը են իլէր՝ նա կէսը ուժ լակ ա քսած իլէր, դէ թողել էն որ պըրծացնի. իսկ էս կրածը, այ մին էրկու տարի կիլի, որ խօսկ իլէր ա, շուտով կվճռվի, թազադանից կրել կտանք։ Կնանք, ասըմ էմ, ծին սպասըմ ա։ Մին կերպ սրան դրոգ քցամ, տարամ ախչիգերքի շկօլայի մօտ։}

– \armenian{Այ, ասըմ էմ, սա մեր շկօլան ա։}

\begin{adjarianpage}\label{page:86}\end{adjarianpage}% should be 86

– \armenian{Բէս, ասըմ ա, Ռուստի վըվիսկա ինչի՞ ա։}

– \armenian{Քրէյով էնք տվել. ասըմ էմ, ա՜յ հոքաւոր տէրը կկայ՝ հուսումարանի պանը կպրծացնի՝ մենք էլի ետ ստեղ կքաշվենք։ լավ րեմօնտ կանենք, տուսից էլ բելիտ կանենք. մին փոքր կէսատ մնացած պան կա։}

– \armenian{Բէս սա՞ ինչ ա, հարցնըմ ա ինձանից, հովի՞ տունն ա։}

– \armenian{Մերն ա, ասըմ էմ. քասիբանոծինը. քանի տարի եա ասել ա, պէտք ա քանդի, թազանը քցի, կործը կէսատ ու մնացել հելէ։}
– \armenian{Բէս սա ի՞նչ պան ա, ասըմ ա, կէսը ճուր, կէսը հող։}

– \armenian{Սա, ասըմ էմ, մեր կանավն ա, փորըմ էն, կէսատ ա։}

\armenian{Ա տղայ, տա բիրդան ինձի չի ասի՞լ – քի, ասա պաժալըստա տուք ինքներդ էր կիսա՞տ էք, թէ թամամ խալխ էք։ Մատաղ, ասըմ ա, ծեզի սկի թամամացրած, պրծացրած պան չունէ՞ք, որ տեսնենք։}

– \armenian{Ինչի՞ չէ, ասըմ էմ, ա՛յ Կըտերինայի մատուռը, ստակ թամամացրած պրծացրած ա։}

– \armenian{Ինձադը՞, ասըմ ա. նա ի՞նչ պան ա։}

– \armenian{Մատուռ ա, ասըմ էմ, աղօթք անելի տեղ, չասօվնա՛. ուզում էս կնանք։}
\armenian{Ադա տա բիրդան ծիտն շուր չի տալ, հարայ չի տալ. «պրեամը նա պարախօդ»։}
\armenian{Պրօխօդ էկանք թէ չէ, սվիստոհը տըվան։ Սա ինգաւ ոխչին թարիֆ անելի, քի Հաշտարխանի հայերի ոխչը պանը կէսատ ա. ծըծաղ պաց քըցան որ։}
\armenian{Ասըմ էմ ինչի՞ էս հոքիիդ մեղք անըմ, քեզի խօ ասա՞մ, որ պրծացրած պան էր ունենք։}

– \armenian{Հա, ասըմ ա, մին դանա չասօվնա ունեն պրծացրած. ինչպէս ա անըմը՞։}

– \armenian{Կըտերնայի մատուռ։}

\armenian{Ինքըս էլ փոշմանամ։ Ադա սաղ կայնած խալխը էսթայից էր էնթայից էր, եքքէ ծէնով ծըծախ պաց չեն քըցի՞լ. կասես մին վեդրէ հէրման ճուր ածան վերէս։ Ինքըս էր չի հասկացամ, որ ի՞նչպէս ընդեղիծ տուս էկամ, տուն էկամ։ Մունչուրի էս սահաթը չէմ կարանըմ մոռանալի. որ միտքըս չի ընգնը՞մ, սաղ վերէս ալավ ա տամ։}



\chapter{Julfa} 

\begin{adjarianpage}\label{page:87}\end{adjarianpage}% should be 87

\section{Overview}

The homeland of this dialect is the village called Julfa (now called Old Julfa), found near the shores of Araks, at the Persia-Russia borders. In the old times, Julfa played a large role in national commerce. Julfa Armenians had spread out until Italy and Holland, and accumulated great wealth in these fields. During the time of Shah Abbas the Great (Armenian: \armenian{Շահաբաս}), large numbers of the Armenian population of the Araratian plains (including Julfa Armenians) were forced to leave their homeland, and were taken to Isfahan. Here, in the southern part of the city, they established the New Julfa suburb, which over a small period of time became a lot bigger and richer. It had up to 25,000 Armenian residents. The majority of its residents became involved in commerce and established a few settlements in India, Birmania (Burma), Java and Sumatra. These latter settlements are now almost all gone, and the few remaining Armenians have become English-speakers. 

The Julfa dialect is still alive in Old Julfa, New Julfa, and in a few cities in Persia, such as Shiraz, Hamadan, Bushehr, Tehran, Anzali, Qazvin, Rasht, and so on, where New Julfa migrants have settled. 

There are is extensive manuscript that is written in the Julfa dialect; this is the chronology of Petros di Sarkis Gilanentz (Armenian: \armenian{Պետրոս Դի Սարգիս Գիլանենց}).\footnote{\translatorHD{His name is also romanized with \textit{Sargis} instead of \textit{Sarkis}, and \textit{Gilanents} instead of \textit{Gilanentz}. An English translation can be found online \citep{gilanentz-1959-chronicle}.}}. This was published first in the periodical \textit{Krunk Handes} (Armenian: \armenian{Կռունկ Հանդէս}, 1863, February, March) and then published on its own.\footnote{\translatorHD{I think he means the periodical \citeauthor{KrunkHayotsAshkharin}. } } One can also find articles written in the more recent dialect in the local New Julfa periodical of \textit{Lraber} (\citeauthor{LraberNewJulfa}), which is still published to this day. Because I don't have this newspaper, I could not use it. 

The Julfa dialect was studied by Patkanian (\armenian{Պատկանեան}) in his... 

\begin{adjarianpage}\label{page:88}\end{adjarianpage}% should be 88

... work \citep[76-103]{Patkanian-1869-RussianDialects}, page 76-103. Thus by benefiting from this work, we can compose our description of the Julfa dialect. 

\section{Phonology}
The phonetic system of this dialect is like the Yerevan system, or more exactly like the Tabriz subdialect. 

\subsection{Change from Classical /h/ <\armenian{հ}> to /χ/ <\armenian{խ}>}

The primary borderline of its sound changes is how the Classical Armenian /h/ <\armenian{հ}> became /χ/ <\armenian{խ}> (Table \ref{tab:Julfa:phono:hkh}). 


\begin{table}[H]
 \centering
 \caption{Change from Classical Armenian /h/ <\armenian{հ}> to /χ/ <\armenian{խ}> in the Julfa dialect}
 \label{tab:Julfa:phono:hkh}
 \begin{tabular}{|l|ll|ll|ll| }
 \hline & \multicolumn{2}{l|}{Classical Armenian } & \multicolumn{2}{l|}{> Julfa }& \multicolumn{2}{l|}{cf. SEA } \\
 `Armenian' & hɑi̯ & \armenian{հայ} & χɑj & \armenian{խայ} & hɑj & \armenian{հայ} \\
 `bread' & hɑt͡sʰ & \armenian{հաց} & χɑt͡sʰ & \armenian{խաց} & hɑt͡sʰ & \armenian{հաց} \\
 `father' & hɑi̯ɾ & \armenian{հայր} & χeɾ & \armenian{խէր} & hɑjɾ & \armenian{հայր} \\
`graceful' &ʃənoɾhɑ{wo}ɾ & \armenian{շնորհաւոր} & ʃənɑχɑvoɾ & \armenian{շընախավօր} & ʃənoɾɑvoɾ & \armenian{շնորհավոր} \\
 `fear' & ɑh& \armenian{ահ} & ɑχ & \armenian{ախ} & ɑh & \armenian{ահ} \\
 
 \hline 
 \end{tabular}


\end{table}


\subsection{Word-initial insertion of /h/ <\armenian{հ}> } 



Many vowel-initial words received an initial /h/ <\armenian{հ}> (Table \ref{tab:Julfa:phonology:hinsert}. 

\begin{table}[H]
 \centering
 \caption{Insertion of word-initial /h/ <\armenian{հ}> in the Julfa dialect}
 \label{tab:Julfa:phonology:hinsert}
 \begin{tabular}{| l| ll|ll| ll|}
 \hline & \multicolumn{2}{l|}{Classical Armenian} &\multicolumn{2}{l|}{> Julfa} & \multicolumn{2}{l|}{cf. SEA} \\ 
 \armenian{՝}when' & eɾb & \armenian{երբ} & hipʰ & \armenian{հիփ} & jeɾpʰ & \armenian{երբ} \\
`cheap' & ɑɾʒɑn & \armenian{արժան}& heʒɑn & \armenian{հէժան}& ɑɾʒɑn & \armenian{արժան} \\
`front' & ɑrɑd͡ʒ & \armenian{առաջ}& hɑred͡ʒ & \armenian{հառէջ}& ɑrɑt͡ʃʰ & \armenian{առաջ} \\
`long' & eɾkɑɾ & \armenian{երկար}& heɾkɑɾ & \armenian{հէրկար}& jeɾkɑɾ & \armenian{երկար} \\
`evening & eɾekun & \armenian{երեկուն}& & & iɾikun & \armenian{իրիկուն} \\
`close of the day' & & & hɑɾɑknɑdem & \armenian{հարակնադէմ}& iɾiknɑdem & \armenian{իրիկնադէմ} \\
\hline 

 \end{tabular}
\end{table}


\subsection{Change from Classical /iu̯/ <\armenian{իւ}> } 



The word-initial /iu̯/ <\armenian{իւ}> sound became /u/ <\armenian{ու}> (Table \ref{tab:Julfa:phonology:iu}. 

\begin{table}[H]
 \centering
 \caption{Change from Classical Armenian /iu̯/ <\armenian{իւ}> to /u/ <\armenian{ու}> in the Julfa dialect}
 \label{tab:Julfa:phonology:iu}
 \begin{tabular}{| l| ll|ll| ll|}
 \hline & \multicolumn{2}{l|}{Classical Armenian} &\multicolumn{2}{l|}{> Julfa} & \multicolumn{2}{l|}{cf. SEA} \\ 
 \armenian{՝}oil' & iu̯ɬ & \armenian{իւղ}& uʁ & \armenian{ուղ} & juʁ & \armenian{յուղ} \\
 \armenian{՝}self' & iu̯ɾ & \armenian{իւր}& uɾ & \armenian{ուր} & juɾ & \armenian{յուր} \\
\hline 
 \end{tabular}
\end{table}

\subsection{Nasal insertion for the word `no' /ot͡ʃʰ/}
The negative formative from Classical /ot͡ʃʰ/ <\armenian{ոչ}> became /mot͡ʃʰ/ <\armenian{մօչ}>. This interesting word is formed in the following way. First, the Classical word /ot͡ʃʰ/ <\armenian{ոչ}> became /vot͡ʃʰ/ <\armenian{վօչ}>, as in many dialects of New Armenian. The sound /n/ <\armenian{ն}> was then added to this word to get /vont͡ʃ/ <\armenian{վօնչ}>, and this form is used in the Yerevan dialect. The labial (\armenian{շրթնական}) sound /v/ in /vont͡ʃ/ was affected by the nasality of this /n/ <\armenian{ն}>, and became a labial nasal /m/ <\armenian{մ}>. From this form, we got the lenited form (\armenian{սղմամբ}) /mot͡ʃʰ/ <\armenian{մօչ}>. In the Julfa dialect, the two forms are also used (Table \ref{tab:Julfa:phonology:vots}). 


\begin{table}[H]
 \centering
 \caption{Insertion of nasal /m/ <\armenian{մ}> in negative words in the Julfa dialect}
 \label{tab:Julfa:phonology:vots}
 \begin{tabular}{| l| ll|ll| ll|}
 \hline & \multicolumn{2}{l|}{Classical Armenian} &\multicolumn{2}{l|}{> Julfa} & \multicolumn{2}{l|}{cf. SEA} \\ 
 \armenian{՝}nothing' & ot͡ʃʰint͡ʃʰ & \armenian{ոչինչ}& mot͡ʃʰint͡ʃʰ, & \armenian{մօչինչ} & vot͡ʃʰint͡ʃʰ & \armenian{ոչինչ} \\
 & & & mont͡ʃʰint͡ʃʰ & \armenian{մօնչինչ} & & \\
 \armenian{՝}no one' & ot͡ʃʰ okʰ & \armenian{ոչ ոք}& mot͡ʃʰov & \armenian{մօչօվ} & vot͡ʃʰ vokʰ & \armenian{ոչ ոք} \\
 \armenian{՝}no one's' & & & mot͡ʃʰum & \armenian{մօչում} & vot͡ʃʰ meki & \armenian{ոչ մեկի} \\
\hline 
 \end{tabular}
\end{table}
\section{Morphology}
\subsection{Noun inflection or declension}

\subsubsection{Case marking for singular nouns}
In noun declension, the genitive-dative case is formed with the formative /-e/ <\armenian{է}> as in Karabakh, or with the formative /-i/ <\armenian{ի}> as in Yerevan. 

The ablative is formed with the formative /-e/ <\armenian{է}>. But for with words with the rhyme /-u/ <\armenian{ու}>, the ablative uses the formatives /-it͡sʰ, ut͡sʰ/, <\armenian{ից, ուց}> (Table \ref{tab:Julfa:morphology:noun:abl}). 


\begin{table}[H]
 \centering
 \caption{Ablative in the Julfa dialect}
 \label{tab:Julfa:morphology:noun:abl}
 \begin{tabular}{| l|ll| ll|}
 \hline &\multicolumn{2}{l|}{Julfa} & \multicolumn{2}{l|}{cf. SEA} \\ 
 \armenian{՝}house' & & & tun & \armenian{տուն} \\
 \armenian{՝}house-{\abl}' & tən-e & \armenian{տընէ} & tən-it͡sʰ & \armenian{տնից} \\
 ?-{\abl} & ɑrv-it͡sʰ & \armenian{առվից} & & \\
 \armenian{՝}soul' & & & hokʰi & \armenian{հոգի} \\
 \armenian{՝}soul-{\abl}' & χokʰ-ut͡sʰ & \armenian{խօքուց} & hokʰ-ut͡sʰ & \armenian{հոգուց} \\
\hline 
 \end{tabular}
\end{table}

The instrumental formative is the usual formative /-ov/ <\armenian{օվ}>, and the locative is /-um/ <\armenian{ում}>.

\subsubsection{Case marking for plural nouns}

For the plural, its declension endings are in Table \ref{tab:Julfa:morphology:noun:plural}. 


\begin{table}[H]
 \centering
 \caption{Declensions for plurals in the Julfa dialect}
 \label{tab:Julfa:morphology:noun:plural}
 \begin{tabular}{| l|ll| ll|}
 \hline 
 {\nom} & /-eɾ/ & <\armenian{էր}> & /-neɾ/ & < \armenian{նէր}> \\
 {\gen},{\dat} & /-eɾ-i, -eɾ-ot͡sʰ/& <\armenian{էրի, էրօց}> & /-neɾ-i/& <\armenian{նէրի}>\\
 {\abl} & /-eɾ-e, -eɾ-ot͡sʰ-e/ & <\armenian{էրէ, էրօցէ}> & & \\
 {\ins} & /-eɾ-ov/ &<\armenian{էրօվ}> & /-neɾ-ov/ & <\armenian{նէրօվ}> \\
 {\locgloss} & /-eɾ-um/ & <\armenian{էրում}> & /-neɾ-um/ & <\armenian{նէրում}>
\\
\hline 
 \end{tabular}
\end{table}


To form the plural in some situations, the dialect uses the formatives /-ekʰ, -ɑni, -eɾ-ɑni, -neɾ-ɑni, -ɑrenkʰ/ <\armenian{էք, անի, էրանի, նէրանի, արէնք}>. 

\begin{table}[H]
 \centering
 \caption{Example of plurals in the Julfa dialect}
 \label{tab:Julfa:morphology:noun:plexample}
 \begin{tabular}{| l|ll| ll|}
 \hline &\multicolumn{2}{l|}{Julfa} & \multicolumn{2}{l|}{cf. SEA} \\ 
 \hline \armenian{՝}Russian' & & & rus & \armenian{ռուս} \\
 \armenian{՝}Russian-{\pl}' & əɾust-ɑni & \armenian{ըռուստանի} & rus-neɾ & \armenian{ռուսներ} \\
 \armenian{՝}hunter' & & & voɾsoʁ & \armenian{որսող} \\
 \armenian{՝}hunter-{\pl}' & voɾsoʁ-ɑni & \armenian{վօրսօղանի} & voɾsoʁ-neɾ & \armenian{որսողներ} \\
 \armenian{՝}other' & & & uɾiʃ & \armenian{ուրիշ} \\
 \armenian{՝}other-{\pl}' & uɾiʃ-ɑni & \armenian{ուրիշանի} & uɾiʃ-neɾ & \armenian{ուրիշներ} \\
 \hline \armenian{՝}head' & & & ɡəluχ & \armenian{գլուխ} \\
 \armenian{՝}head-{\pl}-({\pl})' & ɡluχ-neɾ-ɑni & \armenian{գլուխնէրանի} & ɡəluχ-neɾ & \armenian{գլուխներ} \\
 \armenian{՝}voice' & & & d͡zɑjn & \armenian{ձայն} \\
 \armenian{՝}voice-{\pl}-({\pl})' & d͡zen-eɾ-ɑni & \armenian{ձէնէրանի} & d͡zɑjn-e & \armenian{ձայներ} \\
 \hline \armenian{՝}place' & & & teʁ & \armenian{տեղ} \\
 \armenian{՝}place-{\pl}' & teʁ-ɑɾenkʰ & \armenian{տէղարէնք} & teʁ-eɾ & \armenian{տեղեր} \\
 \armenian{՝}village' & & & ɡʏuʁ & \armenian{գիւղ} \\
 \armenian{՝}village-{\pl}' & ɡeʁ-ɑrenkʰ & \armenian{գէղարէնք} & ɡʏuʁ-eɾ & \armenian{գյուղեր} \\
\hline 
 \end{tabular}
\end{table}


\subsection{Pronoun inflection or declension}
The declension of pronouns is the same as in the Yerevan dialect, ... 

\begin{adjarianpage}\label{page:89}\end{adjarianpage}% should be 89

... ... so we think it's superfluous to show them. The pronouns only in the ablative, where they take the formative /-e/ <\armenian{Է}> (Table \ref{tab:Julfa:morphology:pronoun:ablSample}). 

\begin{table}[H]
 \centering
 \caption{Sample of ablative pronouns in the Julfa dialect}
 \label{tab:Julfa:morphology:pronoun:ablSample}
 \begin{tabular}{|l ll|}
\hline 
personal 1SG `from me' &jinen & \armenian{յինէն} \\
personal 2SG `from you' &kʰezne & \armenian{քէզնէ} \\
personal 1PL `from us' &mezne & \armenian{մէզնէ} \\
personal 1PL `from us' &mezɑne & \armenian{մէզանէ} \\
personal 2PL `from you' &d͡zezne & \armenian{ձէզնէ} \\
personal 2PL `from you' &d͡zezɑne & \armenian{ձէզանէ} \\
demonstrative proximal {\sg} `from this' & esti & \armenian{էստի} \\
demonstrative proximal {\sg} `from this' & soɾɑne & \armenian{սօրանէ} \\
demonstrative proximal {\pl} `from these' & estont͡sʰme & \armenian{էստօնցմէ} \\
demonstrative proximal {\pl} `from these' & sot͡sʰɑne & \armenian{սօցանէ} \\
\hline 
 \end{tabular}
\end{table}



Some interesting forms are in Table \ref{tab:Julfa:morphology:pronoun:other}). 

\begin{table}[H]
 \centering
 \caption{Sample of other pronouns in the Julfa dialect}
 \label{tab:Julfa:morphology:pronoun:other}
 \begin{tabular}{|l ll|}
\hline 
reflexive {\pl} `selves' &uɾenkʰ & \armenian{ուրէնք} \\
reflexive {\sg} {\abl} `from self' &uɾnen & \armenian{ուրնէն} \\
reflexive {\pl} {\gen} `of selves' &uɾent͡sʰ & \armenian{ուրէնց} \\
reflexive {\pl} {\gen} `of selves' &uɾt͡sʰent͡sʰ & \armenian{ուրցէնց} \\
demonstrative proximal {\pl} {\gen} `of these' &sɑnt͡sʰɑn & \armenian{սանցան} \\
demonstrative medial {\pl} {\gen} `of those' &dɑnt͡sʰɑn & \armenian{դանցան} \\
\hline 
 \end{tabular}
\end{table}

\subsection{Verb inflection or conjugation}

\subsubsection{Overview and morphological changes}

For verb conjugation, the most characteristic forms are the following. 


\paragraph{Present copula with /ɑ/ <\armenian{ա}> }
For present tense form of the copular verb, the Classical sound /e/ <\armenian{ե}> sound of all the persons has changed to /ɑ/ <\armenian{ա}> 
(Table \ref{tab:Julfa:morpho:verb:copula}). 


\begin{table}[H]
 \centering
 \caption{Copula with /ɑ/ <\armenian{ա}> instead of /e/ <\armenian{է}> in the Julfa dialect}
 \label{tab:Julfa:morpho:verb:copula}
 \begin{tabular}{|l|ll| ll| }
 \hline & \multicolumn{2}{l|}{Julfa} & \multicolumn{2}{l|}{cf. SEA} \\ 
1SG `I am' &ɑ-m & \armenian{ամ} &e-m & \armenian{եմ}\\ 
2SG `you are' &ɑ-s & \armenian{աս} &e-s & \armenian{ես}\\ 
3SG `he is' &ɑ & \armenian{ա} &e & \armenian{է}\\ 
1PL `we are' &ɑ-nkʰ & \armenian{անք} &e-ŋkʰ & \armenian{ենք}\\ 
2PL `you are' &ɑ-kʰ & \armenian{աք} &e-kʰ & \armenian{եք}\\ 
3PL `they are' &ɑ-n & \armenian{ան} &e-n & \armenian{են}\\ 
& \multicolumn{2}{l|}{{\aux}-{\agr}}& \multicolumn{2}{l|}{{\aux}-{\agr}} \\
\hline 
 \end{tabular}
\end{table}



This of course originated via analogy to the present 3SG, which as we know is /ɑ/ <\armenian{ա}> in the dialects of Yerevan, Karabakh, Shamakhi, Astrakhan, and Agulis.


\paragraph{Past copula with /i/ <\armenian{ի}> }
Its imperfective (\translatorHD{= the past form of the copula}) is formed like in Yerevan 
(Table \ref{tab:Julfa:morpho:verb:copulaPast}). \translatorHD{This means that in Julfa, the /e-i/ {\aux}-{\pst} sequence surfaces as a single [i] via deletion of the /e/. }



\begin{table}[H]
 \centering
 \caption{Past copula with /\armenian{ի}/ <\armenian{ի}> instead of /ei/ <\armenian{էի}> in the Julfa dialect}
 \label{tab:Julfa:morpho:verb:copulaPast}
 \begin{tabular}{|l|ll| ll| }
 \hline & \multicolumn{2}{l|}{Julfa} & \multicolumn{2}{l|}{cf. SEA} \\ 

1SG `I was' & $\emptyset$-i-$\emptyset$ & \armenian{ի} & ej-i-$\emptyset$ & \armenian{էի} \\
2SG `you was' & $\emptyset$-i-ɾ & \armenian{իր} & ej-i-ɾ & \armenian{էիր} \\
3SG `he was'& e-$\emptyset$-ɾ & \armenian{էր} & e-$\emptyset$-ɾ & \armenian{էր} \\
1PL `we were'& $\emptyset$-i-nkʰ & \armenian{ինք} & ej-i-ŋkʰ & \armenian{էինք} \\
2PL `you were'& $\emptyset$-i-kʰ & \armenian{իք}& ej-i-kʰ & \armenian{էիք} \\
3PL `they were' & $\emptyset$-i-n & \armenian{ին} & ej-i-n & \armenian{էին} \\ 
& \multicolumn{2}{l|}{{\aux}-{\pst}-{\agr}}& \multicolumn{2}{l|}{{\aux}-{\pst}-{\agr}}\\ 

\hline 
\end{tabular}
\end{table}



\paragraph{Imperfective converb with /-mɑn/ <\armenian{ման}> } \label{sec:Julfa:morphology:verb:overview:impfcvb}
 The indicative present and imperfective of every verb is made with the formative /-mɑn/ <\armenian{ման}> (/-ɑmɑn/ <\armenian{աման}>) (Table \ref{tab:Julfa:morpho:verb:converb}).\footnote{\translatorHD{I suspect that the formative /-ɑmɑn/ is actually segmentable as /-ɑ-mɑn/: the theme vowel /ɑ/ plus the converb suffix /-mɑn/. }}



\begin{table}[H]
 \centering
 \caption{Imperfective converb with /-mɑn/ <\armenian{ման}> (/-ɑmɑn/ <\armenian{աման}>) in the Julfa dialect}
 \label{tab:Julfa:morpho:verb:converb}
 \begin{tabular}{|l|ll|ll | }
 \hline & \multicolumn{2}{l|}{Shamakhi}& \multicolumn{2}{l|}{cf. SEA }\\ \hline 
`I go' & ɡn-ɑ-mɑn ɑ-m & \armenian{գնաման ամ}&ɡn-um e-m & \armenian{գնում եմ} \\ 
& \multicolumn{2}{l|}{$\sqrt{}$-{\thgloss}-{\impfcvb} {\aux}-1{\sg}} & \multicolumn{2}{l|}{$\sqrt{}$-{\impfcvb} {\aux}-1{\sg}} \\ 
\hline
 `I see' & tes-mɑn ɑ-m & \armenian{տէսման ամ}&tes-n-um e-m & \armenian{տեսնում եմ} \\ 
`I flee' & pʰɑχ-mɑn ɑ-m & \armenian{փախման ամ}&pʰɑχ-t͡ʃʰ-um e-m & \armenian{փախչում եմ} \\ 
& \multicolumn{2}{l|}{$\sqrt{}$-{\impfcvb} {\aux}-1{\sg}} & \multicolumn{2}{l|}{$\sqrt{}$-{\vx}-{\impfcvb} {\aux}-1{\sg}} \\ 
\hline
`I was going' & ɡn-ɑ-mɑn $\emptyset$-i-$\emptyset$ & \armenian{գնաման ի}&ɡn-um ej-i-$\emptyset$ & \armenian{գնում էի} \\ 
& \multicolumn{2}{l|}{$\sqrt{}$-{\thgloss}-{\impfcvb} {\aux}-1{\sg}} & \multicolumn{2}{l|}{$\sqrt{}$-{\impfcvb} {\aux}-1{\sg}} \\ 
\hline
 `I was seeing' & tes-mɑn $\emptyset$-i-$\emptyset$ & \armenian{տէսման ի}&tes-n-um ej-i-$\emptyset$ & \armenian{տեսնում էի} \\ 
`I was fleeing' & pʰɑχ-mɑn $\emptyset$-i-$\emptyset$ & \armenian{փախման ի}&pʰɑχ-t͡ʃʰ-um ej-i-$\emptyset$ & \armenian{փախչում էի} \\ 
& \multicolumn{2}{l|}{$\sqrt{}$-{\impfcvb} {\aux}-{\pst}-1{\sg}} & \multicolumn{2}{l|}{$\sqrt{}$-{\vx}-{\impfcvb} {\aux}-{\pst}-1{\sg}} \\ 
\hline
 \end{tabular}\end{table}

\subsubsection{General paradigms for the reflex of the A-Class}
The following are the primary tenses of the reflex of the Classical verb /ɡən-ɑ-l/ <\armenian{գնալ}> `to go'. \translatorHD{This is the A-Class with theme vowel /-ɑ-/. }





{\paradigmExplanation}
\paragraph{Indicative present and past imperfective}

\translatorHD{In SEA, the indicative present and past imperfective are formed by combining the imperfective converb (a verb with suffix /-um/) with an inflected auxiliary (Table \ref{tab:Julfa:morpho:verb:paradigm:presentIndc}, \ref{tab:Julfa:morpho:verb:paradigm:pastImpfIndc}). In Julfa, essentially the same strategy is used with the following differences: the converb suffix is /-mɑn/, and the auxiliary has different forms in Julfa. These forms were discussed in \S\ref{sec:Julfa:morphology:verb:overview:impfcvb}. }


\begin{table}[H]
 \centering
 \caption{Indicative present <\armenian{ներկայ}> of the verb `to go' in the Julfa dialect}
 \label{tab:Julfa:morpho:verb:paradigm:presentIndc}
 \begin{tabular}{|l|ll|ll|}
\hline & \multicolumn{2}{l|}{Julfa} & \multicolumn{2}{l|}{cf. SEA} \\
1SG & ɡn-ɑ-mɑn ɑ-m & \armenian{գնաման ամ} & ɡən-um e-m &\armenian{գնում եմ} \\
2SG & ɡn-ɑ-mɑn ɑ-s & \armenian{գնաման աս} & ɡən-um e-s &\armenian{գնում ես} \\
3SG & ɡn-ɑ-mɑn ɑ & \armenian{գնաման ա} & ɡən-um e &\armenian{գնում է} \\
1PL & ɡn-ɑ-mɑn ɑ-nkʰ & \armenian{գնաման անք} & ɡən-um e-ŋk &\armenian{գնում ենք} \\
2PL & ɡn-ɑ-mɑn ɑ-kʰ & \armenian{գնաման աք} & ɡən-um e-kʰ &\armenian{գնում եք} \\
3PL& ɡn-ɑ-mɑn ɑ-n & \armenian{գնաման ան} & ɡən-um e-n &\armenian{գնում են} \\
& \multicolumn{2}{l|}{$\sqrt{}$-{\impfcvb} {\aux}-{\agr}}& \multicolumn{2}{l|}{$\sqrt{}$-{\impfcvb} {\aux}-{\agr}}\\
\hline 
\end{tabular}
\end{table}


\begin{table}[H]
 \centering
 \caption{Indicative past imperfective <\armenian{անկատար}> of the verb `to go' in the Julfa dialect}
 \label{tab:Julfa:morpho:verb:paradigm:pastImpfIndc}
 \begin{tabular}{|l|ll|ll|l|}
\hline & \multicolumn{2}{l|}{Julfa} & \multicolumn{2}{l|}{cf. SEA} \\
1SG & ɡn-ɑ-mɑn $\emptyset$-i-$\emptyset$ $\emptyset$-i-$\emptyset$ & \armenian{գնաման ի} & ɡən-um ej-i-$\emptyset$ &\armenian{գնում էի} \\
2SG& ɡn-ɑ-mɑn $\emptyset$-i-ɾ $\emptyset$-i-ɾ & \armenian{գնաման իր} & ɡən-um ej-i-ɾ &\armenian{գնում էիր} \\
3SG& ɡn-ɑ-mɑn e-$\emptyset$-ɾ & \armenian{գնաման էր} & ɡən-um e-$\emptyset$-ɾ &\armenian{գնում էր} \\
1PL& ɡn-ɑ-mɑn $\emptyset$-i-nkʰ & \armenian{գնաման ինք} & ɡən-um ej-i-ŋkʰ &\armenian{գնում էինք} \\
2PL& ɡn-ɑ-mɑn $\emptyset$-i-kʰ &\armenian{գնաման իք} & ɡən-um ej-i-kʰ &\armenian{գնում էիք} \\
3PL& ɡn-ɑ-mɑn $\emptyset$-i-n & \armenian{գնաման ին} & ɡən-um ej-i-n &\armenian{գնում էին} \\
& \multicolumn{2}{l|}{$\sqrt{}$-{\thgloss}-{\impfcvb} {\aux}-{\pst}-{\agr}} & \multicolumn{2}{l|}{$\sqrt{}$-{\impfcvb} {\aux}-{\pst}-{\agr}}\\
\hline 
\end{tabular}
\end{table}
 




\paragraph{Past perfective or aorist}

\translatorHD{In SEA, the past perfective or aorist (Table \ref{tab:Julfa:morpho:verb:paradigm:pastperfectiveAorist}) is for /ɡən-ɑ-l/ `to go' is formed by taking taking the root and theme vowel, adding the aorist or perfective suffix /-t͡sʰ-/, and then adding the past suffix /-i/ and the appropriate agreement suffixes. The 3SG uses covert tense and agreement suffixes. Julfa uses the same strategy. }


\begin{table}[H]
 \centering
 \caption{Past perfective or aorist <\armenian{կատարեալ}> of the verb `to go' in the Julfa dialect}
 \label{tab:Julfa:morpho:verb:paradigm:pastperfectiveAorist}
 \begin{tabular}{|l|ll|ll|}
\hline & \multicolumn{2}{l|}{Julfa} & \multicolumn{2}{l|}{cf. SEA} \\
1SG & ɡn-ɑ-t͡sʰ-i-$\emptyset$ & \armenian{գնացի} & ɡən-ɑ-t͡sʰ-i-$\emptyset$ & \armenian{գնացի} \\
2SG& ɡn-ɑ-t͡sʰ-i-ɾ & \armenian{գնացիր} & ɡən-ɑ-t͡sʰ-i-ɾ &\armenian{գնացիր} \\
3SG & ɡn-ɑ-t͡sʰ-$\emptyset$-$\emptyset$ & \armenian{գնաց} &ɡən-ɑ-t͡sʰ-$\emptyset$-$\emptyset$ &\armenian{գնաց} \\
1PL & ɡn-ɑ-t͡sʰ-i-nkʰ & \armenian{գնացինք} & ɡən-ɑ-t͡sʰ-i-ŋkʰ &\armenian{գնացինք} \\
2PL & ɡn-ɑ-t͡sʰ-i-kʰ & \armenian{գնացիք} &ɡən-ɑ-t͡sʰ-i-kʰ &\armenian{գնացիք}\\
3PL & ɡn-ɑ-t͡sʰ-i-n & \armenian{գնացին} & ɡən-ɑ-t͡sʰ-i-n &\armenian{գնացին} \\
& \multicolumn{2}{l|}{$\sqrt{}$-{\thgloss}-{\aor}-{\pst}-{\agr}}& \multicolumn{2}{l|}{$\sqrt{}$-{\thgloss}-{\aor}-{\pst}-{\agr}}\\ 

\hline 
\end{tabular}
\end{table}
\paragraph{Subjunctive present and past imperfective } 

\translatorHD{In SEA, the subjunctive present (Table \ref{tab:Julfa:morpho:verb:paradigm:subjPresent}) is formed by adding agreement suffixes after the theme vowel /ɑ/. These are the same agreement suffixes that are added onto the present auxiliary in the indicative present. The Julfa dialect behaves the same but with one difference: the theme vowel of the 2PL changes from /ɑ/ to /e/. } 


\begin{table}[H]
 \centering
 \caption{Subjunctive present <\armenian{ստորադասական ներկայ}> of the verb `to go' in the Julfa dialect}
 \label{tab:Julfa:morpho:verb:paradigm:subjPresent}
 \begin{tabular}{|l|ll|ll|}
\hline & \multicolumn{2}{l|}{Julfa} & \multicolumn{2}{l|}{cf. SEA} \\
1SG & ɡn-ɑ-m & \armenian{գնամ} & ɡən-ɑ-m & \armenian{գնամ} \\
2SG & ɡn-ɑ-s & \armenian{գնաս} & ɡən-ɑ-s & \armenian{գնաս} \\
3SG & ɡn-ɑ-$\emptyset$ & \armenian{գնա} & ɡən-ɑ-$\emptyset$ & \armenian{գնա} \\
1PL & ɡn-ɑ-nkʰ & \armenian{գնանք} & ɡən-ɑ-ŋkʰ & \armenian{գնանք} \\
2PL & ɡn-e-kʰ & \armenian{գնէք} & ɡən-ɑ-kʰ & \armenian{գնաք} \\
3PL & ɡn-ɑ-n & \armenian{գնան} & ɡən-ɑ-n & \armenian{գնան} \\
& \multicolumn{2}{l|}{$\sqrt{}$-{\thgloss}-{\agr}}& \multicolumn{2}{l|}{$\sqrt{}$-{\thgloss}-{\agr}}\\ 

\hline 
\end{tabular}
\end{table}









\translatorHD{In SEA, the subjunctive past imperfective (Table \ref{tab:Julfa:morpho:verb:paradigm:subjPast}) is formed by adding the past suffix /i/ and agreement suffixes after the theme vowel. In Julfa, the theme vowel /ɑ/ is deleted before the past suffix /i/. In the 3SG, the theme vowel is changed to /e/. }



\begin{table}[H]
 \centering
 \caption{Subjunctive past <\armenian{ստորադասական անցեալ}> of the verb `to go' in the Julfa dialect}
 \label{tab:Julfa:morpho:verb:paradigm:subjPast}
 \begin{tabular}{|l|ll|ll|}
\hline & \multicolumn{2}{l|}{Julfa} & \multicolumn{2}{l|}{cf. SEA} \\
1SG & ɡn-$\emptyset$-i-$\emptyset$ & \armenian{գնի} & ɡən-ɑj-i-$\emptyset$ & \armenian{գնայի} \\
2SG & ɡn-$\emptyset$-i-ɾ & \armenian{գնիր} & ɡən-ɑj-i-ɾ & \armenian{գնայիր} \\
3SG & ɡn-e-$\emptyset$-ɾ & \armenian{գնէր} & ɡən-ɑ-$\emptyset$-ɾ & \armenian{գնար} \\
1PL & ɡn-$\emptyset$-i-nkʰ & \armenian{գնինք} & ɡən-ɑj-i-ŋkʰ & \armenian{գնայինք} \\
2PL & ɡn-$\emptyset$-i-kʰ & \armenian{գնիք} & ɡən-ɑj-i-kʰ & \armenian{գնայիք} \\
3PL & ɡn-$\emptyset$-i-n & \armenian{գնին} & ɡən-ɑj-i-n & \armenian{գնային} \\
& \multicolumn{2}{l|}{$\sqrt{}$-{\thgloss}-{\pst}-{\agr}}& \multicolumn{2}{l|}{$\sqrt{}$-{\thgloss}-{\pst}-{\agr}}\\ 

\hline 
\end{tabular}
\end{table}









 
\paragraph{Tenses built from the subjunctive: Future }
 
 
 \translatorHD{In Julfa, the future and future perfect are built off of the subjunctive by adding the prefix /kə/ <\armenian{կը}> (Table \ref{tab:Julfa:morpho:verb:paradigm:complexSubjunctive}). SEA behaves essentially the same and I don't provide its paradigm. }
 

\begin{table}[H]
 \centering
 \caption{Future <\armenian{ապառնի}> and future perfect <\armenian{անցեալ ապառնի}> for the verb `to go' in the Julfa dialect}
 \label{tab:Julfa:morpho:verb:paradigm:complexSubjunctive}
 \begin{tabular}{|l|ll|ll|}
\hline & 
\multicolumn{2}{l|}{Future <\armenian{ապառնի}>} & \multicolumn{2}{l|}{Future perfect <\armenian{անցեալ ապառնի}> } \\
1SG & kə ɡn-ɑ-m & \armenian{կը գնամ} & kə ɡən-$\emptyset$-i-$\emptyset$ & \armenian{կը գնի} \\
2SG & kə ɡn-ɑ-s & \armenian{կը գնաս}& kə ɡən-$\emptyset$-i-ɾ & \armenian{կը գնիր} \\
3SG & kə ɡn-ɑ-$\emptyset$ & \armenian{կը գնա} & kə ɡən-e-$\emptyset$-ɾ & \armenian{կը գնէր} \\
1PL & kə ɡn-ɑ-nkʰ & \armenian{կի գնանք}& kə ɡən-$\emptyset$-i-nkʰ & \armenian{կը գնինք} \\
2PL & kə ɡn-e-kʰ & \armenian{կը գնէք} & kə ɡən-$\emptyset$-i-kʰ & \armenian{կը գնիք} \\
3PL & kə ɡn-ɑ-n & \armenian{կը գնան} & kə ɡən-$\emptyset$-i-n & \armenian{կը գնին} 
\\
& \multicolumn{2}{l|}{{\fut} $\sqrt{}$-{\thgloss}-{\agr}}& \multicolumn{2}{l|}{{\fut} $\sqrt{}$-{\thgloss}-{\pst}-{\agr}} 
\\\hline \end{tabular}
\end{table} 

 

\paragraph{Imperative and prohibitive}

\translatorHD{For the imperative 2SG, SEA adds a zero morph after the root for an A-Class verb like `to go' (Table \ref{tab:Julfa:morpho:verb:paradigm:Imp}). For the 2PL, SEA adds the sequence /-ɑ-t͡s-ekʰ/ after the root such that /-ɑ-t͡sʰ/ for the aorist stem, while /-ekʰ/ is the agreement marker. Julfa uses similar strategies with one difference: the 2PL can omit the /-ɑ-t͡sʰ/ sequence. } 


\begin{table}[H]
 \centering
 \caption{Imperative forms <\armenian{հրամայական}> for the verb `to go' in the Julfa dialect}
 \label{tab:Julfa:morpho:verb:paradigm:Imp}
 \begin{tabular}{|l|ll|ll|l|}
\hline & \multicolumn{2}{l|}{Julfa} & \multicolumn{2}{l|}{cf. SEA} & \\
2SG & ɡn-ɑ-$\emptyset$ & \armenian{գնա} & ɡn-ɑ-$\emptyset$ & \armenian{գնա} & $\sqrt{}$-{\thgloss}-{\imp}.2{\sg}
\\
2PL& ɡn-ɑ-t͡sʰ-ekʰ& \armenian{գնացէք} & ɡn-ɑ-t͡sʰ-ekʰ& \armenian{գնացեք} & $\sqrt{}$-{\thgloss}-{\aor}-{\imp}.2{\pl}
\\
& ɡn-ekʰ&\armenian{գնէք} & & & $\sqrt{}$-{\imp}.2{\pl}
 
\\\hline \end{tabular}
\end{table}
 


\translatorHD{For the prohibitive or negative imperative (Table \ref{tab:Julfa:morpho:verb:paradigm:Proh}), SEA simply adds the prohibitive formative /mi/ before the imperative form. For Julfa, the prohibitive is formed by placing the /mi/ after the verb. The verb is a non-finite form with /-ɑl/ (possibly the infinitive). In the 2PL, the prohibitive marker carries plural agreement, thus agreement morphology is mobile. } 


\begin{table}[H]
 \centering
 \caption{Negative imperative or prohibitive forms for the verb `to go' in the Yerevan dialect}
 \label{tab:Julfa:morpho:verb:paradigm:Proh}
 \begin{tabular}{|l|lll| l|}
\hline & \multicolumn{3}{l|}{Julfa and cf. SEA} & \\\hline 
2SG & ɡn-ɑ-l m\'i &\armenian{գնալ մի՛} &$\sqrt{}$-{\thgloss}-{\infgloss}? {\proh} & Julfa \\
& m\'i ɡən-ɑ-$\emptyset$ & \armenian{մի՛ գնա} & {\proh} $\sqrt{}$-{\thgloss}-{\imp}.2{\sg} & SEA\\ \hline 
2PL & ɡn-ɑ-l m-ekʰ & \armenian{գնալ մէք} & $\sqrt{}$-{\thgloss}-{\infgloss}? {\proh}-{\imp}.2{\pl} & Julfa \\
 & mi ɡən-ɑ-t͡sʰ-ekʰ& \armenian{մի գնացեք} & {\proh} $\sqrt{}$-{\thgloss}-{\aor}-{\imp}.2{\pl} & SEA \\
 \hline \end{tabular}
\end{table}

\paragraph{Non-finite forms}

\translatorHD{Finally, Adjarian lists the following non-finite forms of this verb (participles or converbs) in Table \ref{tab:Julfa:morpho:verb:paradigm:participle}. Note that the present participle and past participle are also called the imperfective converb and the perfective converb. } 

\begin{table}[H]
 \centering
 \caption{Participles or converbs <\armenian{դերբայներ}> for the verb `to go' in the Julfa dialect}
 \label{tab:Julfa:morpho:verb:paradigm:participle}
 \begin{tabular}{|ll|lll|l|}
\hline & & \multicolumn{3}{l|}{Julfa and cf. SEA} & \\
 Infinitive & \armenian{անորոշ} & ɡn-ɑ-l & \armenian{գնալ} & $\sqrt{}$-{\thgloss}-{\infgloss} & Julfa\\ 
 & & ɡn-ɑ-l & \armenian{գնալ} & $\sqrt{}$-{\thgloss}-{\infgloss} & SEA \\
 Present & \armenian{ներկայ} & ɡn-ɑ-mɑn & \armenian{գնաման} & $\sqrt{}$-{\thgloss}-{\impfcvb} & Julfa \\
& & ɡən-um & \armenian{գնում} & $\sqrt{}$-{\impfcvb} & SEA \\
 Past & \armenian{անցեալ} & ɡn-ɑ-t͡sʰ-el & \armenian{գնացէլ} & $\sqrt{}$-{\thgloss}-{\aor}-{\perfcvb} & Julfa\\
& & ɡən-ɑ-t͡sʰ-el & \armenian{գնացել} & $\sqrt{}$-{\thgloss}-{\aor}-{\perfcvb} & SEA \\
\hline \end{tabular}
\end{table}


\begin{adjarianpage}\label{page:90}\end{adjarianpage}% should be 90


\section{Text samples}

Adjarian's sample: Taken from \armenian{Կռունկ}, 1863, \armenian{էջ} 92-94

{\sampleoverview}

1. \armenian{Համադանցի Սահիջանի վօրթի Հովսէփն արէկ քաղաքս, ասաց թէ Համադանա մին ամիս ա վօր դուս ամ. յէս Համադան իքան՝ Բաղդադա մին խայ յէկավ Համադան, ասաց թէ Ըստամբօլու շատ ջաբախանա յէկավ Բաղդադ, ամա ասկար չէկավ, եւ ասկար կալէ ձէն էլ չկէր։ Բաղդադա փաշէն վօր Ըռուստի ասկարին Գիլան գօլն հիմացավ՝ Բաղդադա բէրթըն ինչ քանդած տէղ վօր կէր՝ Թամամին շինէլ արէտ, վօր Ըռուստիցն շատ ախ էր քաշում։}

2. \armenian{Վէրօ Հօվսէփն ասաց թէ յէս վօր յէկի Ղազվին, Ղազվինցիք ասում ին թէ մէր սարդարն տէղէս փախավ՝ մէնք մնացինք անտէր. մէր ճարն ի՞նչպէս գընի. մէր ճարն էս ա վօր հէփ մէր ախն շատանա վօր Աղվանն մէզ մօտկանա, պիտի վօր գրէնք Ըռէշտ՝ Ըռուստի սարդարին վէրա, վօր մէզ տիրութին առի եւ մէնք Ըռուստի ղուլուղ առէնք։}

3. \armenian{Օգօստօսի} 2\armenian{քումն} 2 \armenian{շամախցի թուրք Թարվիզու շահիցն չափարարէկ Ըռէշտ՝ վօր գնա Թիմիջանա վէզրին կուշտն։ Էս} 2 \armenian{չափարն ասէլ ան թէ՝ Վախտանկ խանըն եւ Կախէթու վալի Մամատ Ղուլի խանըն խաշտէլ ան (հաշտուեր են) եւ միատէղ ուրէնց ասկարօվն գնացէլ ան Հարէվան (Երեւան)։ Հարէվանա կշտին} 4 \armenian{օսմանցու փաշա դընի ուրէնց ասկարօվն}. 4 \armenian{փաշին խէտ կռիվ կը տան, վօր Օսմանցու ասկարէն շատ ջառթէլ ան. մնացյալն փախէլ ա եւ մին փաշէն ուր ասկարօվն մին ղայիմ տէղ ա՝ վօր կարէլ չէն գրիշմիշ լինէլն։}

4. \armenian{Ջուղայէցի Վօհանէսի վօրթի Թօրօսն Օգօստօսի} 3\armenian{ումն յէկավ Ըռէշտ. ասաց թէ յէս Արզրում էի՝ վօր խաբար յէկավ թէ} 4 \armenian{փաշա գնացին Թիֆլիզ առին եւ ընկէլ ան Վախտանկ խանին հէտնէն վօր բռնէն. Արզրում} 3 \armenian{օր դօնամա արարին եւ} ... 

\begin{adjarianpage}\label{page:91}\end{adjarianpage}% should be 91

... \armenian{թօփվէր գցէցին. հէտօ քանի օրէն հէտ խաբար արէկ՝ թէ Վախտանկ խանն շատ ասկարօվ հէտ ա դառցել օսմանցուն վէրա. շատ ուժօվ շքաստա տվէլ. եւ օսմանլվին ասկարն վօր փախէլ էր՝} 100-\armenian{օվ} 200-\armenian{օվ հէտ ան գօլման Արզրում. վօր յէկինք Բայազիդն էլ տէսման ինք, վօր հէտ ին փախման։ Թավրիզու վօր դախիլ յէլանք՝ էլ էսպէս լսէցինք. եւ Վրաստանա չափար արէկ Թարվէզ՝ էլ էսպէս ասաց՝ վօր վէրէվումն գրած ա։}

5. \armenian{Հուլիսի} 28. \armenian{Թիֆլիզէցի} X \armenian{վօր Հայօց ղավակ ա, վօր կաթօլիկ ա դառցէլ, վօր ֆռանկսըզի կումալանուն դիլմաճ յէլէլ, վօր էս Հօսէփն Ըսպահանա փախէլ էր, Համադանա վէրա արէկ Ըռէշտ, վօր Աղվանին Ըսպահան առուլն խաքիյաթ արար՝ թէ ի՛նչպէս ան առէլ. թիվն} 1722 \armenian{փէտրվարի} 18. \armenian{Աղվան Միրվէսի վօրթի Մամուդ խանըն} 12,000 \armenian{ասկարօվ Քրմանա վրա յէկավ. Ըսպահանա վէրա, վօր էլ էս օր Աղվանն յէկէլ ա Վարզան դախիլ յէլէլ, վօր մինչի Ըսպահան} 16 \armenian{աղաջ ա, վօր է} 80 \armenian{վէրստ։}

6. \armenian{Արապի սարդարն էլ էն օրն չափար ա ղարկէլ Ըսպահան շահին՝ թէ Թախիխ Մամուդն ուր ասկարօվն Ըսպահանա վէրա կալման, շուտօվ էստուր ֆիքրն արա։}

7. \armenian{Շահ Մամուդին գալն վօր կը լսի՝ թէ Թախիխ կալիս ա, հուքմ կառի ուր բէկլարին՝ թէ վօ՛րչանք օմարա, խան, բէկ, բէկզարա, ղուլ, ղօոչի կա՝ հազըվէն եւ թօփ եւ ջաբախանա հազրէցէք՝ որ բիտի գնէք Ազվանին ղարշուն, վօր թօշէք վօչ Աղվանն Ըսպահան դօ. սօքա} 18,000 \armenian{մարթ եւ} 24 \armenian{թօփ կը հազրէն։}

8. \armenian{Էլ էն օրն մուասիլ կաղարկէն գեղարէնքն՝} 12,000 \armenian{մարթ թվանկչի կը բօլօրէն կը բէրէն Ըսպահան։}


\chapter{Agulis}\label{chapter:Agulis}

\section{Background}
\begin{adjarianpage}\label{page:92}\end{adjarianpage}% should be 92


The Agulis dialect is spread in a small border near Nakhichevan, whose center is the village-city of Agulis. The surrounding villages are Çənnəb, Əndəmic, Danaqırt, Ramis, Dasht, Kələki, and so on. All of these constitute the branches of this dialect. 

The Agulis dialect is so far away from the common Armenian language, that its surrounding populations have thought that this dialect was a foreign language and called it \textit{Zok} (Armenian: \armenian{Զոկերէն}, SEA /zokeɾen/), just as the people are called \textit{Zok} (Armenian: \armenian{Զոկ}, SEA /zok/).

\section{Phonology}
\subsection{Segment inventory}\label{sec:Agulis:phono:segment}

The phonetic system of Agulis is similar to the Yerevan system. It has added only the vowels /æ, ʏ, œ/ <\armenian{ա}̈, \armenian{իւ, էօ}>, and the consonants /ɡʲ, kʲ, kʰʲ/ <\armenian{գյ, կյ, քյ}>. 
\subsection{Sound changes}
Its sound changes have rendered this language unrecognizable, and they are the following. 

\subsubsection{Monophthong vowel changes}

\paragraph{Classical Armenian /ɑ/ <\armenian{ա}>}

Classical Armenian /ɑ/ <\armenian{ա}> became /ɑ/ <\armenian{ա}> for the words in Table \ref{tab:Agulis:phonology:soundChange:monoph:a:a}. 

\begin{table}[H]
 \centering
 \caption{Change from Classical Armenian /ɑ/ <\armenian{ա}> to /ɑ/ <\armenian{ա}> in the Agulis dialect}
 \label{tab:Agulis:phonology:soundChange:monoph:a:a}
 \begin{tabular}{|l| ll|ll| ll|}
 \hline & \multicolumn{2}{l|}{Classical Armenian} &\multicolumn{2}{l|}{> Agulis} & \multicolumn{2}{l|}{cf. SEA} \\ 
`happy' & uɾ\'ɑχ & \armenian{ուրախ} & \'oɾɑχ & \armenian{օ՛րախ} & uɾ\'ɑχ & \armenian{ուրախ} \\
`game' & χɑɬ & \armenian{խաղ} & hɑʁ & \armenian{հաղ} & χɑʁ & \armenian{խաղ} \\
`tail' &ɑɡ\'i& \armenian{ագի} & \'ɑɡi & \armenian{ա՛գի} &ɑɡ\'i& \armenian{ագի} \\
`crow' &ɑɡr\'ɑu̯ & \armenian{ագռաւ} & ɑkr\'ɑv &\armenian{ա՛կռավ} & ɑɡr\'ɑv & \armenian{ագռավ} \\
`salt' & ɑɬ & \armenian{աղ} & ɑʁ & \armenian{աղ} & ɑʁ & \armenian{աղ} \\
`vessel' & ɑm\'ɑn & \armenian{աման} & \'ɑmɑn & \armenian{աղ} & ɑm\'ɑn & \armenian{աման} \\
`summer' & ɑm\'ɑrən & \armenian{ամառն} & \'ɑmɑr & \armenian{ա՛մառ} & ɑm\'ɑr& \armenian{ամառ} \\
`peak' & ɡɑɡ\'ɑtʰən & \armenian{գագաթն} & ɡʲ\'eɡʲɑtʰ & \armenian{գյէ՛գյաթ} & ɡɑɡ\'ɑtʰ& \armenian{գագաթ} \\
 \hline 
 \end{tabular}
\end{table}


Classical Armenian /ɑ/ <\armenian{ա}> became /æ/ <\armenian{ա}̈> for the words in Table \ref{tab:Agulis:phonology:soundChange:monoph:a:ae}. 



\begin{table}[H]
 \centering
 \caption{Change from Classical Armenian /ɑ/ <\armenian{ա}> to /æ/ <\armenian{ա}̈> in the Agulis dialect}
 \label{tab:Agulis:phonology:soundChange:monoph:a:ae}
 \begin{tabular}{|l| ll|ll| ll|}
 \hline & \multicolumn{2}{l|}{Classical Armenian} &\multicolumn{2}{l|}{> Agulis} & \multicolumn{2}{l|}{cf. SEA} \\ 
`mouth' &beɾ\'ɑn & \armenian{բերան} & b\'æɾæn & \armenian{բա̈՛րա̈ն} &beɾ\'ɑn & \armenian{բերան} \\
`sheep' & ot͡ʃʰχ\'ɑɾ & \armenian{ոչխար} & \'eχt͡ʃʰæɾ & \armenian{է՛խչա̈ր} & vot͡ʃʰχ\'ɑɾ & \armenian{ոչխար} \\
`flour' & ɑl\'iu̯ɾ & \armenian{ալիւր} & \'ælʏɾ & \armenian{ա̈՛լիւր} & ɑlj\'uɾ & \armenian{ալյուր} \\
 \armenian{՝}blood' & ɑɾ\'iu̯n & \armenian{արիւն}& \'æɾʏn & \armenian{ա̈՛րիւն} & ɑɾj\'un & \armenian{արյուն} \\
`thin' &bɑɾ\'ɑk & \armenian{բարակ} & b\'æɾæk & \armenian{բա̈՛րա̈կ} & bɑɾ\'ɑk & \armenian{բարակ} \\
`spring' &ɡɑɾ\'un & \armenian{գարուն} & ɡʲ\'æɾunkʰ & \armenian{գյա̈՛րունք} & ɡɑɾ\'un & \armenian{գարուն} \\
\hline 
 \end{tabular}
\end{table}

Classical Armenian /ɑ/ <\armenian{ա}> became /o/ <\armenian{օ}> for the words in Table \ref{tab:Agulis:phonology:soundChange:monoph:a:o}, only in the last syllable. \translatorHD{But contrast their genitive forms which show an /ɑ/. }



\begin{table}[H]
 \centering
 \caption{Change from Classical Armenian /ɑ/ <\armenian{ա}> to /o/ <\armenian{օ}> in the Agulis dialect}
 \label{tab:Agulis:phonology:soundChange:monoph:a:o}
 \begin{tabular}{|l| ll|ll| ll|}
 \hline & \multicolumn{2}{l|}{Classical Armenian} &\multicolumn{2}{l|}{> Agulis} & \multicolumn{2}{l|}{cf. SEA} \\ 
`man' &mɑɾd & \armenian{մարդ} & moɾd & \armenian{մօրդ} &mɑɾtʰ & \armenian{մարդ} \\
`man-{\gen}' & & & m\'ɑɾd-i & \armenian{մա՛րդի} &mɑɾtʰ-\'u & \armenian{մարդու} \\
 \armenian{՝}bride' & hɑɾsən & \armenian{հարսն} & hoɾs & \armenian{հօրս} & hɑɾs & \armenian{հարս} \\
 \armenian{՝}bride-{\gen}' & & & h\'ɑɾs-i & \armenian{հա՛րսի} & hɑɾs-\'i & \armenian{հարսի} \\
 \armenian{՝}death' & mɑh & \armenian{մահ} & moh & \armenian{մօհ} & mɑh & \armenian{մահ} \\
 \armenian{՝}death-{\gen}' & & & m\'ɑh-i & \armenian{մա՛հի} & mɑh-\'i & \armenian{մահի} \\
 \armenian{՝}lamp' & t͡ʃəɾɑɡ & \armenian{ճրագ} & t͡ʃɾoɡʲ & \armenian{ճրօգյ} & t͡ʃəɾɑkʰ & \armenian{ճրագ} \\
 \armenian{՝}lamp-{\gen}' & & & t͡ʃɾ\'ɑɡʲ-i & \armenian{ճրա՛գյի} & t͡ʃəɾɑkʰ-\'i & \armenian{ճրագի} \\
\hline 
 \end{tabular}
\end{table}


Classical Armenian /ɑ/ <\armenian{ա}> became /e/ <\armenian{է}> for very few words (Table \ref{tab:Agulis:phonology:soundChange:monoph:a:e}).\footnote{\translatorHD{For the word `dirty', Adjarian provides the word <\armenian{աղտոտ}>. I couldn't determine if this word existed in Classical Armenian; but this word is a compound of Classical roots, so it's possible. }} 



\begin{table}[H]
 \centering
 \caption{Change from Classical Armenian /ɑ/ <\armenian{ա}> to /e/ <\armenian{է}> in the Agulis dialect}
 \label{tab:Agulis:phonology:soundChange:monoph:a:e}
 \begin{tabular}{|l| ll|ll| ll|}
 \hline & \multicolumn{2}{l|}{Classical Armenian} &\multicolumn{2}{l|}{> Agulis} & \multicolumn{2}{l|}{cf. SEA} \\ 
`man' &mɑɾd & \armenian{մարդ} & moɾd & \armenian{մօրդ} &mɑɾtʰ & \armenian{մարդ} \\
`man-{\gen}' & & & m\'ɑɾd-i & \armenian{մա՛րդի} &mɑɾtʰ-\'u & \armenian{մարդու} \\
`dirt' & ɑɬt & \armenian{աղտ} & eχt & \armenian{էխտ} & ɑχt & \armenian{աղտ} \\
`dirty' & ɑɬt\'ot & \armenian{աղտոտ} & j\'eχtut & \armenian{յէ՛խտուտ} & ɑχt\'ot & \armenian{աղտոտ} \\
`peak' & ɡɑɡ\'ɑtʰən & \armenian{գագաթն} & ɡʲ\'eɡʲɑtʰ & \armenian{գյէ՛գյաթ} & ɡɑɡ\'ɑtʰ& \armenian{գագաթ} \\
`barley' &ɡɑɾ\'i & \armenian{գարի} &ɡʲ\'eɾi & \armenian{գյէ՛րի} &ɡɑɾ\'i & \armenian{գարի} \\
\hline 
 \end{tabular}
\end{table}
 

Classical Armenian /ɑ/ <\armenian{ա}> became /œ/ <\armenian{էօ}> for the words in Table \ref{tab:Agulis:phonology:soundChange:monoph:a:œ}



\begin{table}[H]
 \centering
 \caption{Change from Classical Armenian /ɑ/ <\armenian{ա}> to /œ/ <\armenian{էօ}> in the Agulis dialect}
 \label{tab:Agulis:phonology:soundChange:monoph:a:œ}
 \begin{tabular}{|l| ll|ll| ll|}
 \hline & \multicolumn{2}{l|}{Classical Armenian} &\multicolumn{2}{l|}{> Agulis} & \multicolumn{2}{l|}{cf. SEA} \\ 
`pillow' &bɑɾd͡z & \armenian{բարձ} & bœɾd͡z & \armenian{բէօրձ} &bɑɾt͡sʰ & \armenian{բարձ} \\
`church' &ʒɑm & \armenian{ժամ} & ʒœm & \armenian{ժէօմ} &ʒɑm & \armenian{ժամ} \\
`spade' &bɑh & \armenian{բահ} & bœh & \armenian{բէօհ} &bɑh & \armenian{բահ} \\
`high' &b\'ɑɾd͡zəɾ & \armenian{բարձր} & bœd͡zəɾ & \armenian{բէօձըր} & b\'ɑɾt͡sʰəɾ & \armenian{բարձր} \\
`open' &bɑt͡sʰ & \armenian{բաց} & bœt͡sʰ & \armenian{բէօց} & bɑt͡sʰ & \armenian{բաց} \\
`to go' &ɡənɑl & \armenian{գնալ} & nœl & \armenian{նէօլ} & ɡənɑl & \armenian{գնալ} \\
`bitter' & d\'ɑrən & \armenian{դառն} & d\'œrnə & \armenian{դէ՛օռնը} & d\'ɑrən & \armenian{դառն} \\ 
\hline 
 \end{tabular}
\end{table}


Classical Armenian /ɑ/ <\armenian{ա}> became /jœ/ <\armenian{յէօ}> for the words in Table \ref{tab:Agulis:phonology:soundChange:monoph:a:jœ}

 

\begin{table}[H]
 \centering
 \caption{Change from Classical Armenian /ɑ/ <\armenian{ա}> to /jœ/ <\armenian{յէօ}> in the Agulis dialect}
 \label{tab:Agulis:phonology:soundChange:monoph:a:jœ}
 \begin{tabular}{|l| ll|ll| ll|}
 \hline & \multicolumn{2}{l|}{Classical Armenian} &\multicolumn{2}{l|}{> Agulis} & \multicolumn{2}{l|}{cf. SEA} \\
`right (side)' &ɑd͡ʒ & \armenian{աջ} & jœd͡ʒ & \armenian{յէօջ} &ɑt͡ʃʰ & \armenian{աջ} \\
\hline 
 \end{tabular}
\end{table}

\begin{adjarianpage}\label{page:93}\end{adjarianpage}% should be 93


Classical Armenian /ɑ/ <\armenian{ա}> became /ɑj/ <\armenian{այ}> for the words in Table \ref{tab:Agulis:phonology:soundChange:monoph:a:ɑj}



\begin{table}[H]
 \centering
 \caption{Change from Classical Armenian /ɑ/ <\armenian{ա}> to /ɑj/ <\armenian{այ}> in the Agulis dialect}
 \label{tab:Agulis:phonology:soundChange:monoph:a:ɑj}
 \begin{tabular}{|l| ll|ll| ll|}
 \hline & \multicolumn{2}{l|}{Classical Armenian} &\multicolumn{2}{l|}{> Agulis} & \multicolumn{2}{l|}{cf. SEA} \\
`needle' & ɑs\'eɬən & \armenian{ասեղն} & \'ɑjsæʁ & \armenian{ա՛յսա̈ղ} &ɑs\'eʁ & \armenian{ասեղ} \\
`happy! (interjection)' & eɾɑn\'i & \armenian{երանի} & həɾ\'ɑjnɑk & \armenian{հըրա՛յնակ} &jeɾɑn\'i & \armenian{երանի} \\
`to fold' & t͡sɑl\'el & \armenian{ծալել} & t͡s\'ɑjlil & \armenian{ծա՛յլիլ} &t͡sɑl\'el & \armenian{ծալել} \\
`to melt' & hɑl\'el & \armenian{հալել} & h\'ɑjlil & \armenian{հա՛յլիլ} & hɑl\'el & \armenian{հալել} \\
\hline 
 \end{tabular}
\end{table}


Classical Armenian /ɑ/ <\armenian{ա}> became /u/ <\armenian{ու}> for the words in Table \ref{tab:Agulis:phonology:soundChange:monoph:a:u}, only before nasal consonants. 



\begin{table}[H]
 \centering
 \caption{Change from Classical Armenian /ɑ/ <\armenian{ա}> to /u/ <\armenian{ու}> in the Agulis dialect}
 \label{tab:Agulis:phonology:soundChange:monoph:a:u}
 \begin{tabular}{|l| ll|ll| ll|}
 \hline & \multicolumn{2}{l|}{Classical Armenian} &\multicolumn{2}{l|}{> Agulis} & \multicolumn{2}{l|}{cf. SEA} \\
`similar' & nəmɑn & \armenian{նման} & nmun & \armenian{նմուն} &nəmɑn & \armenian{նման} \\
`sign' & nəʃɑn & \armenian{նշան} & nʃun & \armenian{նշուն} &nəʃɑn & \armenian{նշան} \\
`monastery' & vɑnkʰ & \armenian{վանք} & vunkʰ & \armenian{վունք} &vɑŋkʰ & \armenian{վանք} \\
`thick' & tʰ\'ɑnd͡zəɾ & \armenian{թանձր} & tʰ\'und͡zɾ & \armenian{թո՛ւնձր} & tʰ\'ɑnd͡zəɾ & \armenian{թանձր} \\
`heavy' & t͡s\'ɑnəɾ & \armenian{ծանր} & t͡s\'undəɾ & \armenian{ծո՛ւնդըր} & t͡s\'ɑnəɾ & \armenian{ծանր} \\
\hline 
 \end{tabular}
\end{table}



Classical Armenian /ɑ/ <\armenian{ա}> became /ʏ/ <\armenian{իւ}> for the words in Table \ref{tab:Agulis:phonology:soundChange:monoph:a:ʏ}. 



\begin{table}[H]
 \centering
 \caption{Change from Classical Armenian /ɑ/ <\armenian{ա}> to /ʏ/ <\armenian{իւ}> in the Agulis dialect}
 \label{tab:Agulis:phonology:soundChange:monoph:a:ʏ}
 \begin{tabular}{|l| ll|ll| ll|}
 \hline & \multicolumn{2}{l|}{Classical Armenian} &\multicolumn{2}{l|}{> Agulis} & \multicolumn{2}{l|}{cf. SEA} \\
`to come' & ɡɑl & \armenian{գալ} & ɡʲʏl & \armenian{գյիւլ} &ɡɑl & \armenian{գալ} \\
`thing' &bɑn & \armenian{բան} & bʏn & \armenian{բիւն} & bɑn & \armenian{բան} \\
`spoon' &tɑɾɡɑl & \armenian{տարգալ} & dəɡʏl & \armenian{դըգիւլ} & ɡətʰɑl & \armenian{գդալ} \\
 `swallow'& t͡sit͡sern\'ɑk & \armenian{ծիծեռնակ} & t͡sʰət͡sʰ\'ærnʏk & \armenian{ցըցա̈՛ռնիւկ} & t͡sit͡sern\'ɑk & \armenian{ծիծեռնակ} \\
`apricot' & t͡siɾ\'ɑn & \armenian{ծիրան} & t͡s\'æɾʏn & \armenian{ծա̈՛րիւն} &t͡siɾ\'ɑn & \armenian{ծիրան} \\ 
\hline 
 \end{tabular}
\end{table}

\paragraph{Classical Armenian /e/ <\armenian{ե}>}

Classical Armenian /e/ <\armenian{ե}> became /ʏ/ <\armenian{իւ}> for the words in Table \ref{tab:Agulis:phonology:soundChange:monoph:e:ʏ}. 



\begin{table}[H]
 \centering
 \caption{Change from Classical Armenian /e/ <\armenian{ե}> to /ʏ/ <\armenian{իւ}> in the Agulis dialect}
 \label{tab:Agulis:phonology:soundChange:monoph:e:ʏ}
 \begin{tabular}{|l| ll|ll| ll|}
 \hline & \multicolumn{2}{l|}{Classical Armenian} &\multicolumn{2}{l|}{> Agulis} & \multicolumn{2}{l|}{cf. SEA} \\
`two' & eɾk\'u & \armenian{երկու} & \'æɾkʏ & \armenian{ա̈՛րկիւ} & jeɾk\'u & \armenian{երկու} \\ 
`on' & i veɾ\'ɑi̯ & \armenian{ի վերայ} & v\'æɾæ & \armenian{վա̈՛րա}̈ & vəɾ\'ɑ & \armenian{վրա} \\ 
 `bridegroom' & pʰes\'ɑi̯ & \armenian{փեսայ} & pʰ\'æsæ & \armenian{փա̈՛սա}̈ & pʰes\'ɑ & \armenian{փեսա} \\
 `corpse' & mere̯\'ɑl & \armenian{մեռեալ} & m\'æɾæl & \armenian{մա̈՛ռա̈լ} & merj\'ɑl & \armenian{մեռյալ} \\
 `ground' & ɡet\'in & \armenian{գետին}& ɡʲ\'ætin & \armenian{գյա̈՛տին} & ɡet\'in& \armenian{գետին} \\
`mouth' &beɾ\'ɑn & \armenian{բերան} & b\'æɾæn & \armenian{բա̈՛րա̈ն} &beɾ\'ɑn & \armenian{բերան} \\
`maternal uncle' &kʰer\'i & \armenian{քեռի} & kʰ\'æɾi & \armenian{քա̈՛ռի} & kʰer\'i & \armenian{քեռի} \\
`needle' & ɑs\'eɬən & \armenian{ասեղն} & \'ɑjsæʁ & \armenian{ա՛յսա̈ղ} &ɑs\'eʁ & \armenian{ասեղ} \\
`beam' & ɡeɾ\'ɑn & \armenian{գերան} & ɡʲ\'æɾæn & \armenian{գյա̈՛րա̈ն} &ɡeɾ\'ɑn& \armenian{գերան} \\

\hline 
 \end{tabular}
\end{table}


Classical Armenian /e/ <\armenian{ե}> became /ɑ/ <\armenian{ա}> for the words in Table \ref{tab:Agulis:phonology:soundChange:monoph:e:ɑ}. 

 
\begin{table}[H]
 \centering
 \caption{Change from Classical Armenian /e/ <\armenian{ե}> to /ɑ/ <\armenian{ա}> in the Agulis dialect}
 \label{tab:Agulis:phonology:soundChange:monoph:e:ɑ}
 \begin{tabular}{|l| ll|ll| ll|}
 \hline & \multicolumn{2}{l|}{Classical Armenian} &\multicolumn{2}{l|}{> Agulis} & \multicolumn{2}{l|}{cf. SEA} \\
`dream' & eɾ\'ɑz & \armenian{երազ} & \'ɑɾɑz & \armenian{ա՛րազ} & jeɾ\'ɑz & \armenian{երազ} \\
`come (participle)' & eke̯\'ɑl & \armenian{եկեալ} & \'ɑkɑl & \armenian{ա՛կալ} & jek\'e̯l & \armenian{եկել} \\
 \armenian{՝}iron' & eɾk\'ɑtʰ & \armenian{երկաթ} & \'ɑɾkɑtʰ & \armenian{ա՛րկաթ} & jeɾk\'ɑtʰ & \armenian{երկաթ} \\
 \armenian{՝}come! ({\imp}.2{\pl})' & ek\'ɑi̯kʰ & \armenian{եկա՛յք} & \'ɑkikʰʲ & \armenian{ա՛կիքյ} & jek\'ekʰ & \armenian{եկե՛ք} \\
`brother' & eɬb\'ɑi̯ɾ & \armenian{եղբայր} & \'ɑχpɑɾ & \armenian{ա՛խպար}& jeχp\'ɑjɾ & \armenian{եղբայր} \\ 
`student' & ɑʃɑk\'eɾt & \armenian{աշակերտ} & \'ɑʃkɑɾt & \armenian{ա՛շկարտ}& ɑʃɑk\'eɾt & \armenian{աշակերտ} \\ 
`thirty' &eɾes\'un& \armenian{երեսուն} & ɑɾ\'ɑssun & \armenian{արա՛սսուն} &jeɾes\'un & \armenian{երեսուն} \\
\hline 
 \end{tabular}
\end{table}


Classical Armenian /e/ <\armenian{ե}> became /i/ <\armenian{ի}> for the words in Table \ref{tab:Agulis:phonology:soundChange:monoph:e:i}. 

\begin{table}[H]
 \centering
 \caption{Change from Classical Armenian /e/ <\armenian{ե}> to /i/ <\armenian{ի}> in the Agulis dialect}
 \label{tab:Agulis:phonology:soundChange:monoph:e:i}
 \begin{tabular}{|l| ll|ll| ll|}
 \hline & \multicolumn{2}{l|}{Classical Armenian} &\multicolumn{2}{l|}{> Agulis} & \multicolumn{2}{l|}{cf. SEA} \\
`big' &met͡s & \armenian{մեծ} & mit͡s& \armenian{միծ} &met͡s & \armenian{մեծ} \\
`I ({\nom})' &es & \armenian{ես} & is & \armenian{իս} &jes & \armenian{ես} \\
`ox' &\'ezən & \armenian{եզն} & \'iznə & \armenian{ի՛զնը} &jez & \armenian{եզ} \\
`river' &ɡet & \armenian{գետ} & ɡit & \armenian{գիտ} &ɡet & \armenian{գետ} \\
`wife's father' &ɑn\'eɾ& \armenian{աներ} & \'ɑniɾ & \armenian{ա՛նիր} &ɑn\'eɾ& \armenian{աներ} \\
`broom' &ɑ{w\'e}l& \armenian{աւել} & \'ɑvil & \armenian{ա՛վիլ} &ɑv\'el& \armenian{ավել} \\
 \armenian{՝}to bring' & beɾ\'el & \armenian{բերել} & b\'iɾil & \armenian{բի՛րիլ} & beɾ\'el & \armenian{բերել} \\
 \armenian{՝}when' & eɾb & \armenian{երբ} & ib & \armenian{իբ} & jeɾpʰ & \armenian{երբ} \\
 \armenian{՝}face' & eɾ\'es & \armenian{երես} & \'iɾis & \armenian{ի՛րիս}& jeɾ\'es & \armenian{երես} \\
 `three' &eɾ\'ekʰ & \armenian{երեք} & \'iɾikʰ & \armenian{ի՛րիք} &jeɾ\'ekʰ & \armenian{երեք} \\
 `to cook' & epʰ\'el & \armenian{եփել} & \'ipʰil & \armenian{ի՛փիլ} & jepʰ\'el & \armenian{եփել} \\
`light (weight)' & tʰetʰ\'eu̯& \armenian{թեթեւ} & tʰ\'itʰiv &\armenian{թի՛թիվ} & tʰetʰ\'ev & \armenian{թեթև} \\ 
\hline 
 \end{tabular}
\end{table}



Classical Armenian /e/ <\armenian{ե}> became /e/ <\armenian{է}> for the words in Table \ref{tab:Agulis:phonology:soundChange:monoph:e:e}. 

 \begin{table}[H] \centering
 \caption{Change from Classical Armenian /e/ <\armenian{ե}> to /e/ <\armenian{է}> in the Agulis dialect}
 \label{tab:Agulis:phonology:soundChange:monoph:e:e}
 \begin{tabular}{|l| ll|ll| ll|}
 \hline & \multicolumn{2}{l|}{Classical Armenian} &\multicolumn{2}{l|}{> Agulis} & \multicolumn{2}{l|}{cf. SEA} \\
`burden' &b\'erən & \armenian{բեռն} & b\'ernə & \armenian{բէ՛ռնը} &ber & \armenian{բեռ} \\ 
`yellow' &deɬ\'in & \armenian{դեղին} &d\'eʁin & \armenian{դէ՛ղին} &deʁ\'in & \armenian{դեղին} \\
`nail (finger/toe)' &eɬ\'unɡ & \armenian{եղունգ} & \'eʁunkʰ & \armenian{է՛ղունք} &jeʁ\'uŋɡ& \armenian{եղունգ} \\
`winter' & d͡zə\'merən & \armenian{ձմեռն} & d͡zəm\'ernə & \armenian{ձըմէ՛ռնը} & d͡zə\'mer & \armenian{ձմեռ} \\
 `hand' &d͡zer-kʰ (-{\pl}) & \armenian{ձեռք} & d͡zerkʰ & \armenian{ձէռք} & d͡zerkʰ & \armenian{ձեռք} \\ 
\hline 
 \end{tabular}
\end{table}


Classical Armenian /e/ <\armenian{ե}> became /e/ <\armenian{էօ}> for the words in Table \ref{tab:Agulis:phonology:soundChange:monoph:e:œ}. 

 \begin{table}[H] \centering
 \caption{Change from Classical Armenian /e/ <\armenian{ե}> to /e/ <\armenian{էօ}> in the Agulis dialect}
 \label{tab:Agulis:phonology:soundChange:monoph:e:œ}
 \begin{tabular}{|l| ll|ll| ll|}
 \hline & \multicolumn{2}{l|}{Classical Armenian} &\multicolumn{2}{l|}{> Agulis} & \multicolumn{2}{l|}{cf. SEA} \\
`oath' &eɾd\'umən & \armenian{երդումն} & \'œɾdʏm & \armenian{էօ՛րդիւմ} & jeɾtʰ\'um & \armenian{երդում} \\ 
\hline 
 \end{tabular}
\end{table}

\paragraph{Classical Armenian /ē/ <\armenian{է}>}

Classical Armenian /ē/ <\armenian{է}> became /e/ <\armenian{է}> for the words in Table \ref{tab:Agulis:phonology:soundChange:monoph:ee:e}. 

\begin{table}[H]
 \centering
 \caption{Change from Classical Armenian /ē/ <\armenian{է}> to /e/ <\armenian{է}> in the Agulis dialect}
 \label{tab:Agulis:phonology:soundChange:monoph:ee:e}
 \begin{tabular}{|l| ll|ll| ll|}
 \hline & \multicolumn{2}{l|}{Classical Armenian} &\multicolumn{2}{l|}{> Agulis} & \multicolumn{2}{l|}{cf. SEA} \\ 
`donkey' & ēʃ & \armenian{էշ} & eʃ & \armenian{էշ} & eʃ & \armenian{էշ} \\
 `half' &kēs & \armenian{կէս} & kes & \armenian{կէս} &kes & \armenian{կես} \\
 `olive oil' &d͡zētʰ & \armenian{ձէթ} & d͡zetʰ & \armenian{ձէթ} &d͡zetʰ & \armenian{ձեթ} \\
\hline 
 \end{tabular}
\end{table}


Classical Armenian /ē/ <\armenian{է}> became /ej/ <\armenian{էյ}> for the words in Table \ref{tab:Agulis:phonology:soundChange:monoph:ee:ej}. 

\begin{table}[H]
 \centering
 \caption{Change from Classical Armenian /ē/ <\armenian{է}> to /ej/ <\armenian{էյ}> in the Agulis dialect}
 \label{tab:Agulis:phonology:soundChange:monoph:ee:ej}
 \begin{tabular}{|l| ll|ll| ll|}
 \hline & \multicolumn{2}{l|}{Classical Armenian} &\multicolumn{2}{l|}{> Agulis} & \multicolumn{2}{l|}{cf. SEA} \\ 
`heap' &dēz & \armenian{դէզ} & dejz & \armenian{դէյզ} &dez & \armenian{դեզ} \\
\hline 
 \end{tabular}
\end{table}

Classical Armenian /ē/ <\armenian{է}> became /i/ <\armenian{ի}> for the words in Table \ref{tab:Agulis:phonology:soundChange:monoph:ee:i}. 

\begin{table}[H]
 \centering
 \caption{Change from Classical Armenian /ē/ <\armenian{է}> to /i/ <\armenian{ի}> in the Agulis dialect}
\label{tab:Agulis:phonology:soundChange:monoph:ee:i}
 \begin{tabular}{|l| ll|ll| ll|}
 \hline & \multicolumn{2}{l|}{Classical Armenian} &\multicolumn{2}{l|}{> Agulis} & \multicolumn{2}{l|}{cf. SEA} \\ 
`heap' &dēz & \armenian{դէզ} & dejz & \armenian{դէյզ} &dez & \armenian{դեզ} \\
`fox' &ɑɬu.\'es & \armenian{աղուէս} & \'ɑʁvis &\armenian{ա՛ղվիս}& ɑʁv\'es & \armenian{աղվես} \\ 
`curse' &ɑn\'ēt͡skʰ & \armenian{անէծք} & \'ɑnit͡skʰ & \armenian{ա՛նիծք}& ɑn\'et͡skʰ & \armenian{անեծք} \\ 
\hline 
 \end{tabular}
\end{table}

Classical Armenian /ē/ <\armenian{է}> became /ɑj/ <\armenian{այ}> for the words in Table \ref{tab:Agulis:phonology:soundChange:monoph:ee:ɑj}. 

\begin{table}[H]
 \centering
 \caption{Change from Classical Armenian /ē/ <\armenian{է}> to /ɑj/ <\armenian{այ}> in the Agulis dialect}
\label{tab:Agulis:phonology:soundChange:monoph:ee:ɑj}
 \begin{tabular}{|l| ll|ll| ll|}
 \hline & \multicolumn{2}{l|}{Classical Armenian} &\multicolumn{2}{l|}{> Agulis} & \multicolumn{2}{l|}{cf. SEA} \\ 
\armenian{՝}lord' & tēɾ & \armenian{տէր} &tɑjɾ & \armenian{տայր} & teɾ & \armenian{տեր} \\ 
\armenian{՝}dormouse (CA); rat (SEA)' & ɑrnēt & \armenian{առնէտ} &ərnɑjt & \armenian{ըռնայտ} & ɑrnet & \armenian{առնետ} \\ 
\armenian{՝}need' & ɑrnēt & pētkʰ &pɑjtkʰ & \armenian{պայտք} & petkʰ & \armenian{պետք} \\ 
\hline 
 \end{tabular}
\end{table}


Classical Armenian /ē/ <\armenian{է}> became /ɑj/ <\armenian{ա}> for the words in Table \ref{tab:Agulis:phonology:soundChange:monoph:ee:ɑ}. 

\begin{table}[H]
 \centering
 \caption{Change from Classical Armenian /ē/ <\armenian{է}> to /ɑ/ <\armenian{ա}> in the Agulis dialect}
\label{tab:Agulis:phonology:soundChange:monoph:ee:ɑ}
 \begin{tabular}{|l| ll|ll| ll|}
 \hline & \multicolumn{2}{l|}{Classical Armenian} &\multicolumn{2}{l|}{> Agulis} & \multicolumn{2}{l|}{cf. SEA} \\ 
`female' & ēɡ & \armenian{էգ} & ɑɡʲ & \armenian{ագյ} & eɡ & \armenian{էգ} \\
`inside' & mēd͡ʒ & \armenian{մէջ} & & & met͡ʃʰ & \armenian{մեջ} \\
`inside-{\defgloss}' & & & m\'ɑd͡ʒ-ə & \armenian{մա՛ջը} & m\'et͡ʃʰ-ə & \armenian{մեջը} \\
\hline 
 \end{tabular}
\end{table}


\paragraph{Classical Armenian /i/ <\armenian{ի}>}

Classical Armenian /i/ <\armenian{ի}> became /i/ <\armenian{ի}> for the words in Table \ref{tab:Agulis:phonology:soundChange:monoph:i:i}. 
 
\begin{table}[H]
 \centering
 \caption{Change from Classical Armenian /i/ <\armenian{ի}> to /i/ <\armenian{ի}> in the Agulis dialect}
 \label{tab:Agulis:phonology:soundChange:monoph:i:i}
 \begin{tabular}{|l| ll|ll| ll|}
 \hline & \multicolumn{2}{l|}{Classical Armenian} &\multicolumn{2}{l|}{> Agulis} & \multicolumn{2}{l|}{cf. SEA} \\ 
`red' & kɑɾm\'iɾ & \armenian{կարմիր} & k\'ɑɾmiɾ & \armenian{կա՛րմիր} & kɑɾm\'iɾ& \armenian{կարմիր} \\
`month' & ɑm\'is & \armenian{ամիս} & \'ɑmis & \armenian{ա՛միս} &ɑm\'is & \armenian{ամիս} \\ 
`nit' & ɑn\'it͡s & \armenian{անիծ} & \'ɑnit͡s & \armenian{ա՛նիծ} &ɑn\'it͡s & \armenian{անիծ} \\ 
`nit' & bəɾind͡z& \armenian{բրինձ} & bɾind͡z & \armenian{բրինձ} & bəɾind͡z & \armenian{բրինձ} \\ 
 `barley' &ɡɑɾ\'i & \armenian{գարի} &ɡʲ\'eɾi & \armenian{գյէ՛րի} &ɡɑɾ\'i & \armenian{գարի} \\
`wine' &ɡin\'i & \armenian{գինի} & ɡʲ\'ini &\armenian{գյի՛նի} &ɡin\'i & \armenian{գինի} \\
`ninety' & innəs\'un & \armenian{իննսուն} & inn\'ɑsun & \armenian{իննա՛սուն} &innəs\'un & \armenian{իննսուն} \\
 `horse' & d͡zi& \armenian{ձի} & d͡zi& \armenian{ձի} & d͡zi& \armenian{ձի} \\
\hline 
 \end{tabular}
\end{table}


Classical Armenian /i/ <\armenian{ի}> became /ej/ <\armenian{էյ}> for the words in Table \ref{tab:Agulis:phonology:soundChange:monoph:i:ej}. 

\begin{table}[H]
 \centering
 \caption{Change from Classical Armenian /i/ <\armenian{ի}> to /ej/ <\armenian{էյ}> in the Agulis dialect}
 \label{tab:Agulis:phonology:soundChange:monoph:i:ej}
 \begin{tabular}{|l| ll|ll| ll|}
 \hline & \multicolumn{2}{l|}{Classical Armenian} &\multicolumn{2}{l|}{> Agulis} & \multicolumn{2}{l|}{cf. SEA} \\ 
`to lick' & liz\'el & \armenian{լիզել} & l\'ejzil & \armenian{լէ՛յզիլ} & liz\'el& \armenian{լիզել} \\ 
\hline 
 \end{tabular}
\end{table}


Classical Armenian /i/ <\armenian{ի}> became /uj/ <\armenian{ույ}> for the words in Table \ref{tab:Agulis:phonology:soundChange:monoph:i:uj}. 

\begin{table}[H]
 \centering
 \caption{Change from Classical Armenian /i/ <\armenian{ի}> to /uj/ <\armenian{ույ}> in the Agulis dialect}
\label{tab:Agulis:phonology:soundChange:monoph:i:uj}
 \begin{tabular}{|l| ll|ll| ll|}
 \hline & \multicolumn{2}{l|}{Classical Armenian} &\multicolumn{2}{l|}{> Agulis} & \multicolumn{2}{l|}{cf. SEA} \\ 
`one' &mi & \armenian{մի} &mujn & \armenian{մույն} &mi & \armenian{մի} \\ 
\hline 
 \end{tabular}
\end{table}

 
Classical Armenian /i/ <\armenian{ի}> became /ɑj/ <\armenian{այ}> for the words in Table \ref{tab:Agulis:phonology:soundChange:monoph:i:ɑj}. 

\begin{table}[H]
 \centering
 \caption{Change from Classical Armenian /i/ <\armenian{ի}> to /ɑj/ <\armenian{այ}> in the Agulis dialect}
\label{tab:Agulis:phonology:soundChange:monoph:i:ɑj}
 \begin{tabular}{|l| ll|ll| ll|}
 \hline & \multicolumn{2}{l|}{Classical Armenian} &\multicolumn{2}{l|}{> Agulis} & \multicolumn{2}{l|}{cf. SEA} \\ 
`meat' & mis & \armenian{միս} & mɑjs & \armenian{մայս} &mis & \armenian{միս} \\ 
`to like' & siɾ\'el & \armenian{սիրել} & s\'ɑjɾil & \armenian{սա՛յրիլ} &siɾ\'el & \armenian{սիրել} \\ 
`oar' & tʰi & \armenian{թի} & tʱ\'ɑjnə & \armenian{թա՛յնը} &tʰi & \armenian{թի} \\ 
`nine' & \'inən & \armenian{ինն} & \'ɑjnə & \armenian{ա՛յնը} &\'inə & \armenian{ինը} \\ 
`laughter' & t͡sit͡s\'ɑɬ & \armenian{ծիծաղ} & t͡s\'ɑjt͡sæʁ& \armenian{ծա՛յծա̈ղ} &t͡sit͡s\'ɑʁ & \armenian{ծիծաղ} \\ 
`lime' & liɾ & \armenian{կիր} & kɑjɾ & \armenian{կայր} liɾ & \armenian{կիր} \\ 
`old' & hin & \armenian{հին} & hɑjn & \armenian{հայն} hin & \armenian{հին} \\ 
`nose' &kʰitʰ& \armenian{քիթ} & kʰɑjntʰ & \armenian{քայնթ} &kʰitʰ& \armenian{քիթ} \\
`mind' & mit-(ə?)kʰ (-{\pl}) & \armenian{միտք} & mɑjtkʰ & \armenian{մայտք} & mitkʰ & \armenian{միտք} \\ 
\hline 
 \end{tabular}
\end{table}


Classical Armenian /i/ <\armenian{ի}> became /æ/ <\armenian{ա}̈> for the words in Table \ref{tab:Agulis:phonology:soundChange:monoph:i:æ}. 
 
\begin{table}[H]
 \centering
 \caption{Change from Classical Armenian /i/ <\armenian{ի}> to /æ/ <\armenian{ա}̈> in the Agulis dialect}
\label{tab:Agulis:phonology:soundChange:monoph:i:æ}
 \begin{tabular}{|l| ll|ll| ll|}
 \hline & \multicolumn{2}{l|}{Classical Armenian} &\multicolumn{2}{l|}{> Agulis} & \multicolumn{2}{l|}{cf. SEA} \\ 
`beautiful' & siɾ\'un & \armenian{սիրուն} & s\'æɾʏn & \armenian{սա̈՛րիւն} & siɾ\'un & \armenian{սիրուն} \\ 
`I ({\dat})' & ind͡z & \armenian{ինձ} & ænd͡z & \armenian{ա̈նձ} & ind͡z & \armenian{ինձ} \\ 
`I ({\gen})' & im & \armenian{իմ} & æm & \armenian{ա̈մ} & im & \armenian{իմ} \\ 
`apricot' & t͡siɾ\'ɑn & \armenian{ծիրան} & t͡s\'æɾʏn & \armenian{ծա̈՛րիւն} &t͡siɾ\'ɑn & \armenian{ծիրան} \\ 
`one' &mi & \armenian{մի} &mæn & \armenian{մա̈ն} &mi & \armenian{մի} \\ 
\hline 
 \end{tabular}
\end{table}

Classical Armenian /i/ <\armenian{ի}> became /ɑ/ <\armenian{ա}> for the words in Table \ref{tab:Agulis:phonology:soundChange:monoph:i:ɑ}.\footnote{\translatorHD{The original page had <\armenian{մայտք}> as part of the change from <\armenian{ի}> to <\armenian{ա}>. But on page 306, Adjarian provides an erratum that this should be part of the change from <\armenian{ի}> to \armenian{այ}>; I fixed it.}} 
 
\begin{table}[H]
 \centering
 \caption{Change from Classical Armenian /i/ <\armenian{ի}> to /ɑ/ <\armenian{ա}> in the Agulis dialect}
\label{tab:Agulis:phonology:soundChange:monoph:i:ɑ}
 \begin{tabular}{|l| ll|ll| ll|}
 \hline & \multicolumn{2}{l|}{Classical Armenian} &\multicolumn{2}{l|}{> Agulis} & \multicolumn{2}{l|}{cf. SEA} \\ 
`heart' & siɾt & \armenian{սիրտ} & sɑɾt & \armenian{սարտ} &siɾt & \armenian{սիրտ} \\ 
 `five' &hinɡ & \armenian{հինգ} &hɑnɡʲ & \armenian{հանգյ} &hiŋɡ & \armenian{հինգ} \\
`kernel' &koɾ\'iz & \armenian{կորիզ} & kʁ\'ɑz & \armenian{կղազ} & koɾ\'iz& \armenian{կորիզ} \\
\hline 
 \end{tabular}
\end{table}

 
\begin{adjarianpage}\label{page:94}\end{adjarianpage}% should be 94


\paragraph{Classical Armenian /o/ <\armenian{ո}>}

Classical Armenian /o/ <\armenian{ո}> became /o/ <\armenian{օ}> for the words in Table \ref{tab:Agulis:phonology:soundChange:monoph:o:o}. 

\begin{table}[H]
 \centering
 \caption{Change from Classical Armenian /o/ <\armenian{ո}> to /o/ <\armenian{օ}> in the Agulis dialect}
 \label{tab:Agulis:phonology:soundChange:monoph:o:o}
 \begin{tabular}{|l| ll|ll| ll|}
 \hline & \multicolumn{2}{l|}{Classical Armenian} &\multicolumn{2}{l|}{> Agulis} & \multicolumn{2}{l|}{cf. SEA} \\ 
`eyelid' &kop& \armenian{կոպ} & kop & \armenian{կօպ} & kop& \armenian{կոպ} \\ 
\hline 
 \end{tabular}
\end{table}


Classical Armenian /o/ <\armenian{ո}> became /u/ <\armenian{ու}> for the words in Table \ref{tab:Agulis:phonology:soundChange:monoph:o:u}. 

\begin{table}[H]
 \centering
 \caption{Change from Classical Armenian /o/ <\armenian{ո}> to /u/ <\armenian{ու}> in the Agulis dialect}
 \label{tab:Agulis:phonology:soundChange:monoph:o:u}
 \begin{tabular}{|l| ll|ll| ll|}
 \hline & \multicolumn{2}{l|}{Classical Armenian} &\multicolumn{2}{l|}{> Agulis} & \multicolumn{2}{l|}{cf. SEA} \\ 
`smell' &hot & \armenian{հոտ} & hut & \armenian{հուտ} & hot & \armenian{հոտ} \\
`loan' &pʰoχ & \armenian{փոխ} & pʰuh & \armenian{փուհ} & pʰoχ & \armenian{փոխ} \\
`foot' &\'otən & \armenian{ոտն} & \'utnə & \armenian{ո՛ւտնը}& v\'ot & \armenian{ոտ} \\
 \armenian{՝}belly' & pʰoɾ & \armenian{փոր}& pʰuɾ & \armenian{փուր} & pʰoɾ & \armenian{փոր} \\
`who' & ov & \armenian{ով} & uv & \armenian{ուվ} & ov & \armenian{ով} \\ 
`no' & ot͡ʃʰ & \armenian{ոչ} & ut͡ʃʰ & \armenian{ուչ} & vot͡ʃʰ & \armenian{ոչ} \\ 
`chair' & ɑtʰ\'or & \armenian{աթոռ} & \'ɑtʰurkʰ & \armenian{ա՛թուռք} & ɑtʰ\'or & \armenian{աթոռ} \\ 
`trembling' & doɬ & \armenian{դող} & duʁ & \armenian{դուղ} & doʁ & \armenian{դող}\\ 
`bone' &\'oskəɾ & \armenian{ոսկր} & \'uskər & \armenian{ո՛ւսկըռ} & vosk\'oɾ & \armenian{ոսկոր} \\
`lentil' & ospən & \armenian{ոսպն} & usp & \armenian{ուսպ} & vosp & \armenian{ոսպ} \\
`wall' & \'oɾmən & \armenian{որմն} & \'uɾmɑn & \armenian{ո՛ւրման} & voɾm & \armenian{որմ} \\
\hline 
 \end{tabular}
\end{table}


Classical Armenian /o/ <\armenian{ո}> became /ʏ/ <\armenian{իւ}> for the words in Table \ref{tab:Agulis:phonology:soundChange:monoph:o:ʏ}. 
 
\begin{table}[H]
 \centering
 \caption{Change from Classical Armenian /o/ <\armenian{ո}> to /ʏ/ <\armenian{իւ}> in the Agulis dialect}
 \label{tab:Agulis:phonology:soundChange:monoph:o:ʏ}
 \begin{tabular}{|l| ll|ll| ll|}
 \hline & \multicolumn{2}{l|}{Classical Armenian} &\multicolumn{2}{l|}{> Agulis} & \multicolumn{2}{l|}{cf. SEA} \\ 
`thief' &ɡoɬ & \armenian{գող} & ɡʲʏʁ & \armenian{գյիւղ} & ɡoʁ & \armenian{գող} \\
`essential' &ɡəlχɑ{w\'o}ɾ & \armenian{գլխաւոր} & ɡʲəlh\'ævʏɾ & \armenian{գյըլհա̈՛վիւր} & ɡəlχɑv\'oɾ & \armenian{գլխավոր} \\
`gray-haired' & ɑle{w\'o}ɾ & \armenian{ալեւոր} & həl\'ævʏɾ & \armenian{հըլա̈՛վիւր}& ɑlev\'oɾ & \armenian{ալևոր} \\ 
`flame' &bot͡sʰ & \armenian{բոց} & bʏt͡sʰ & \armenian{բիւց} & bot͡sʰ & \armenian{բոց} \\
`lap' &ɡoɡ & \armenian{գոգ} & ɡʲʏɡʲ & \armenian{գյիւգյ} & ɡokʰ & \armenian{գոգ} \\
`apple' & χənd͡z\'oɾ & \armenian{խնձոր} & χənd͡z\'ʏɾ & \armenian{խընձիւր} & χənd͡z\'oɾ & \armenian{խնձոր} \\ 
`worm' & \'oɾdən & \armenian{որդն} & \'ʏrnə & \armenian{իւռնը} & v\'oɾtʰ & \armenian{որդ} \\ 
\hline 
 \end{tabular}
\end{table}

Classical Armenian /o/ <\armenian{ո}> became /e/ <\armenian{է}> for the words in Table \ref{tab:Agulis:phonology:soundChange:monoph:o:e}. 
 
\begin{table}[H]
 \centering
 \caption{Change from Classical Armenian /o/ <\armenian{ո}> to /e/ <\armenian{է}> in the Agulis dialect}
 \label{tab:Agulis:phonology:soundChange:monoph:o:e}
 \begin{tabular}{|l| ll|ll| ll|}
 \hline & \multicolumn{2}{l|}{Classical Armenian} &\multicolumn{2}{l|}{> Agulis} & \multicolumn{2}{l|}{cf. SEA} \\ 
`to try' &pʰoɾd͡z\'el & \armenian{փորձել} & pʰ\'eɾd͡zil & \armenian{փէ՛րձիլ} & pʰoɾt͡sʰ\'el & \armenian{փորձել} \\
`mule' & d͡ʒoɾ\'i & \armenian{ջորի} & d͡ʒ\'eɾi & \armenian{ջէ՛րի} & d͡ʒoɾ\'i & \armenian{ջորի} \\
 `soul' & hoɡ\'i & \armenian{հոգի} & h\'eɡi & \armenian{հէ՛գի} & hokʰ\'i & \armenian{հոգի} \\
`sheep' & ot͡ʃʰχ\'ɑɾ & \armenian{ոչխար} & \'eχt͡ʃʰæɾ & \armenian{է՛խչա̈ր} & vot͡ʃʰχ\'ɑɾ & \armenian{ոչխար} \\
`louse' &od͡ʒ\'il & \armenian{ոջիլ} & \'ed͡ʒil & \armenian{է՛ջիլ} &vot͡ʃʰ\'il& \armenian{ոջիլ} \\
\hline 
 \end{tabular}
\end{table}

Classical Armenian /o/ <\armenian{ո}> became /æ/ <\armenian{ա}̈> for the words in Table \ref{tab:Agulis:phonology:soundChange:monoph:o:æ}.% \footnote{\translatorHD{For the word `cress', Adjarian provides an ancestor form <\armenian{կոտեմն}>, but I've had difficulty verifying if this word existed in Classical Armenian. Instead the form I found in dictionaries like Calfa was <\armenian{կոտիմն}>. }} 
 
\begin{table}[H]
 \centering
 \caption{Change from Classical Armenian /o/ <\armenian{ո}> to /æ/ <\armenian{ա}̈> in the Agulis dialect}
 \label{tab:Agulis:phonology:soundChange:monoph:o:æ}
 \begin{tabular}{|l| ll|ll| ll|}
 \hline & \multicolumn{2}{l|}{Classical Armenian} &\multicolumn{2}{l|}{> Agulis} & \multicolumn{2}{l|}{cf. SEA} \\ 
`to try' & ɡoɾt͡s\'el & \armenian{գործել} & ɡʲ\'æɾd͡zil & \armenian{գյա̈՛րձիլ} & ɡoɾt͡s\'el & \armenian{գործել} \\ 
 \armenian{՝}to praise' & ɡov\'el & \armenian{գովել}& ɡʲ\'ɑvil & \armenian{գյա̈՛վիլ} & ɡov\'el & \armenian{գովել} \\
`cress' & kot\'imən & \armenian{կոտիմն} & kʲ\'ætim & \armenian{կյա̈՛տիմ} & kot\'em & \armenian{կոտեմ} \\
\hline 
 \end{tabular}
\end{table}


Classical Armenian /o/ <\armenian{ո}> became /ɑ/ <\armenian{ա}> for the words in Table \ref{tab:Agulis:phonology:soundChange:monoph:o:ɑ}.%\footnote{\translatorHD{For the word `calf', Adjarian provides an Classical ancestor /hoɾtʰ/ <\armenian{հորթ}>. But the most prescriptive Classical form is /oɾtʰ/ <\armenian{որթ}>. I changed his example for accuracy. }}
 
\begin{table}[H]
 \centering
 \caption{Change from Classical Armenian /o/ <\armenian{ո}> to /ɑ/ <\armenian{ա}> in the Agulis dialect}
 \label{tab:Agulis:phonology:soundChange:monoph:o:ɑ}
 \begin{tabular}{|l| ll|ll| ll|}
 \hline & \multicolumn{2}{l|}{Classical Armenian} &\multicolumn{2}{l|}{> Agulis} & \multicolumn{2}{l|}{cf. SEA} \\ 
`smell' &hot & \armenian{հոտ} & & & hot & \armenian{հոտ} \\
`smelled (participle)' & & & h\'ætɑt͡s & \armenian{հա՛տած} & hot\'ɑt͡s & \armenian{հոտած} \\
`to change' & pʰoχ\'el &\armenian{փոխել} & pʰ\'ɑhil & \armenian{փա՛հիլ}& pʰoχ\'el & \armenian{փոխել} \\
`grandchild' & tʰ\'orən &\armenian{թոռն} & tʰ\'ɑrnə & \armenian{թա՛ռնը} & tʰor & \armenian{թոռ} \\
`ash' &moχ\'iɾ & \armenian{մոխիր} & m\'ɑχiɾ & \armenian{մա՛խիր} & moχ\'iɾ & \armenian{մոխիր} \\
`gold' & osk\'i & \armenian{ոսկի} & \'ɑski & \armenian{ա՛սկի} & vosk\'i& \armenian{ոսկի} \\
`smell' & oɾtʰ & \armenian{որթ} & \'ɑɾtʰuk & \armenian{ա՛րթուկ} & hoɾt & \armenian{հորթ} \\
\hline 
 \end{tabular}
\end{table} 

\paragraph{Classical Armenian /u/ <\armenian{ու}>}


Classical Armenian /u/ <\armenian{ու}> became /u/ <\armenian{ու}> for the words in Table \ref{tab:Agulis:phonology:soundChange:monoph:u:u}. 

\begin{table}[H]
 \centering
 \caption{Change from Classical Armenian /u/ <\armenian{ու}> to /u/ <\armenian{ու}> in the Agulis dialect}
 \label{tab:Agulis:phonology:soundChange:monoph:u:u}
 \begin{tabular}{|l| ll|ll| ll|}
 \hline & \multicolumn{2}{l|}{Classical Armenian} &\multicolumn{2}{l|}{> Agulis} & \multicolumn{2}{l|}{cf. SEA} \\ 
`other' & uɾ\'iʃ & \armenian{ուրիշ} & \'uɾiʃ & \armenian{ո՛ւրիշ} & uɾ\'iʃ & \armenian{ուրիշ} \\ 
 `name' & ɑn\'un & \armenian{անուն} & \'ɑnun & \armenian{ա՛նուն} &ɑn\'un & \armenian{անուն} \\ 
 `autumn' & ɑʃ\'un & \armenian{աշուն} & & & ɑʃ\'un& \armenian{աշուն} \\ 
 & ɑʃ\'un-kʰ (-{\pl})& \armenian{աշունք} & \'ɑʃunkʰ &\armenian{ա՛շունք} & \\ 
 `elbow' & ɑɾm\'ukən & \armenian{արմուկն} & \'ɑɾmunɡ & \armenian{ա՛րմունգ} &ɑɾm\'uŋk & \armenian{արմունկ} \\ 
`fish' &d͡z\'ukən & \armenian{ձուկն} & d͡z\'uknə & \armenian{ձո՛ւկնը} & d͡z\'uk & \armenian{ձուկ} \\ 
\hline 
 \end{tabular}
\end{table}

Classical Armenian /u/ <\armenian{ու}> became /ʏ/ <\armenian{իւ}> for the words in Table \ref{tab:Agulis:phonology:soundChange:monoph:u:ʏ}. 

\begin{table}[H]
 \centering
 \caption{Change from Classical Armenian /u/ <\armenian{ու}> to /ʏ/ <\armenian{իւ}> in the Agulis dialect}
 \label{tab:Agulis:phonology:soundChange:monoph:u:ʏ}
 \begin{tabular}{|l| ll|ll| ll|}
 \hline & \multicolumn{2}{l|}{Classical Armenian} &\multicolumn{2}{l|}{> Agulis} & \multicolumn{2}{l|}{cf. SEA} \\ 
`you ({\nom})' & du & \armenian{դու} & dʏ & \armenian{դիւ} & du & \armenian{դու} \\ 
`wool' & buɾd & \armenian{բուրդ} & bʏɾtʰ & \armenian{բիւրդ} & buɾtʰ & \armenian{բուրդ} \\ 
`abyss' & ɑnd\'und & \armenian{անդունդ} & & & ɑnd\'und & \armenian{անդունդ} \\ 
 & ɑnd\'und-(ə?)kʰ (-{\pl}) & \armenian{անդունդք} & \'ændʏndkʰ &\armenian{ա̈՛նդիւնդք} & & \\ 
 `head' & ɡəluχ & \armenian{գլուխ} & ɡʲəlʏh & \armenian{գյըլիւհ} & ɡəluχ & \armenian{գլուխ} \\
`oath' &eɾd\'umən & \armenian{երդումն} & \'œɾdʏm & \armenian{էօ՛րդիւմ} & jeɾtʰ\'um & \armenian{երդում} \\ 
\hline 
 \end{tabular}
\end{table}


Classical Armenian /u/ <\armenian{ու}> became /o/ <\armenian{օ}> for the words in Table \ref{tab:Agulis:phonology:soundChange:monoph:u:o}. 

\begin{table}[H]
 \centering
 \caption{Change from Classical Armenian /u/ <\armenian{ու}> to /o/ <\armenian{օ}> in the Agulis dialect}
 \label{tab:Agulis:phonology:soundChange:monoph:u:o}
 \begin{tabular}{|l| ll|ll| ll|}
 \hline & \multicolumn{2}{l|}{Classical Armenian} &\multicolumn{2}{l|}{> Agulis} & \multicolumn{2}{l|}{cf. SEA} \\ 
`false' & sut & \armenian{սուտ} & sot & \armenian{սօտ} & sut & \armenian{սուտ} \\ 
`cold' & t͡sʰuɾt & \armenian{ցուրտ} & t͡sʰoɾt & \armenian{ցօրտ} & t͡sʰuɾt & \armenian{ցուրտ} \\ 
`shoulder' & us & \armenian{ուս} & jons & \armenian{յօնս} & us & \armenian{ուս} \\ 
`camel' & uɬt & \armenian{ուղտ} & oʁt & \armenian{օղտ} & uχt & \armenian{ուղտ} \\ 
`happy' & uɾ\'ɑχ & \armenian{ուրախ} & \'oɾɑχ & \armenian{օ՛րախ} & uɾ\'ɑχ & \armenian{ուրախ} \\
 `to eat' & ut\'el & \armenian{ուտել} & \'otil & \armenian{օ՛տիլ} & ut\'el & \armenian{ուտել} \\
`to have' &uni\'l & \armenian{ունիմ} & \'onel & \armenian{օ՛նիլ} & un\'el & \armenian{ունել} \\ 
`sour' &tʰətʰu & \armenian{թթու} & tʰtʰo & \armenian{թթօ} & tʰətʰu & \armenian{թթու} \\ 
`fig' &tʰuz & \armenian{թուզ} & tʰ\'oznə & \armenian{թօ՛զնը} & tʰuz & \armenian{թուզ} \\ 
\hline 
 \end{tabular}
\end{table}

Classical Armenian /u/ <\armenian{ու}> became /ej/ <\armenian{էյ}> for the words in Table \ref{tab:Agulis:phonology:soundChange:monoph:u:ej}. 

\begin{table}[H]
 \centering
 \caption{Change from Classical Armenian /u/ <\armenian{ու}> to /ej/ <\armenian{էյ}> in the Agulis dialect}
 \label{tab:Agulis:phonology:soundChange:monoph:u:ej}
 \begin{tabular}{|l| ll|ll| ll|}
 \hline & \multicolumn{2}{l|}{Classical Armenian} &\multicolumn{2}{l|}{> Agulis} & \multicolumn{2}{l|}{cf. SEA} \\ 
`knee' & t͡s\'unɡəkʰ & \armenian{ծունգք} & t͡s\'ejnə & \armenian{ծէ՛յնը} & t͡suŋk & \armenian{ծունկ} \\ 
\hline 
 \end{tabular}
\end{table}
 
Classical Armenian /u/ <\armenian{ու}> became /e/ <\armenian{է}> for the words in Table \ref{tab:Agulis:phonology:soundChange:monoph:u:e}. 


\begin{table}[H]
 \centering
 \caption{Change from Classical Armenian /u/ <\armenian{ու}> to /e/ <\armenian{է}> in the Agulis dialect}
 \label{tab:Agulis:phonology:soundChange:monoph:u:e}
 \begin{tabular}{|l| ll|ll| ll|}
 \hline & \multicolumn{2}{l|}{Classical Armenian} &\multicolumn{2}{l|}{> Agulis} & \multicolumn{2}{l|}{cf. SEA} \\ 
`belly (CA); satiated (SEA)' & kuʃt & \armenian{կուշտ} & keʃt & \armenian{կէշտ} & kuʃt & \armenian{կուշտ} \\ 
`Friday' & uɾb\'ɑtʰ & \armenian{ուրբաթ} & \'eɾbætʰ & \armenian{է՛րբա̈թ} & uɾpʰ\'ɑtʰ & \armenian{ուրբաթ} \\ 
`to swallow' & kul tɑl & \armenian{կուլ տալ} & kel tol & \armenian{կէլ տօլ} & kul tɑl & \armenian{կուլ տալ} \\ 
`jug' & kuʒ & \armenian{կուժ} & keʒ& \armenian{կէժ} & kuʒ & \armenian{կուժ} \\ 
`thorn' & pʰuʃ & \armenian{փուշ} & pʰeʃ& \armenian{փէշ} & pʰuʃ & \armenian{փուշ} \\ 
\hline 
 \end{tabular}
\end{table}
 
\subsubsection{Diphthong changes}

\paragraph{Classical Armenian /ɑi̯/ <\armenian{այ}>}
 
Classical Armenian /ɑi̯/ <\armenian{այ}> became /ɑj/ <\armenian{այ}> for the words in Table \ref{tab:Agulis:phonology:soundChange:diphth:ɑi:ɑj}. 

\begin{table}[H]
 \centering
 \caption{Change from Classical Armenian /ɑi̯/ <\armenian{այ}> to /ɑj/ <\armenian{այ}> in the Agulis dialect}
 \label{tab:Agulis:phonology:soundChange:diphth:ɑi:ɑj}
 \begin{tabular}{|l| ll|ll| ll|}
 \hline & \multicolumn{2}{l|}{Classical Armenian} &\multicolumn{2}{l|}{> Agulis} & \multicolumn{2}{l|}{cf. SEA} \\ 
`father' & hɑi̯ɾ & \armenian{հայր} & hɑjɾ & \armenian{հայր} & hɑjɾ & \armenian{հայր} \\ 
`mother' & mɑi̯ɾ & \armenian{մայր} & mɑjɾ & \armenian{մայր} & mɑjɾ & \armenian{մայր} \\ 
`wood' & pʰɑi̯t & \armenian{փայտ} & pʰɑjt & \armenian{փայտ} &pʰɑjt & \armenian{փայտ} \\
`goat' & ɑi̯t͡s & \armenian{այծ} & ɑjt͡s & \armenian{այծ} & ɑjt͡s & \armenian{այծ} \\ 
`sound' & d͡zɑi̯n & \armenian{ձայն} & d͡zɑjn & \armenian{ձայն} & d͡zɑjn & \armenian{ձայն} \\ 
`lightning' &kɑi̯t͡s\'ɑkən & \armenian{կայծակն} & k\'ɑjt͡sɑk & \armenian{կա՛յծա̈կ} & kɑjt͡s\'ɑk & \armenian{կայծակ} \\
 \hline 
 \end{tabular}
\end{table}


Classical Armenian /ɑi̯/ <\armenian{այ}> became /ɑ/ <\armenian{ա}> for the words in Table \ref{tab:Agulis:phonology:soundChange:diphth:ɑi:ɑ}. 

\begin{table}[H]
 \centering
 \caption{Change from Classical Armenian /ɑi̯/ <\armenian{այ}> to /ɑ/ <\armenian{ա}> in the Agulis dialect}
 \label{tab:Agulis:phonology:soundChange:diphth:ɑi:ɑ}
 \begin{tabular}{|l| ll|ll| ll|}
 \hline & \multicolumn{2}{l|}{Classical Armenian} &\multicolumn{2}{l|}{> Agulis} & \multicolumn{2}{l|}{cf. SEA} \\ 
 medial `that' &ɑi̯d& \armenian{այդ} & ɑt & \armenian{ատ} &ɑjd& \armenian{այդ} \\
 distal `that yonder' &ɑi̯n& \armenian{այն} & ɑn & \armenian{ան} &ɑjn& \armenian{այն} \\
 `other' &ɑi̯l& \armenian{այլ} & ɑl & \armenian{ալ} &ɑjl& \armenian{այլ} \\
 \hline 
 \end{tabular}
\end{table}

Classical Armenian /ɑi̯/ <\armenian{այ}> became /e/ <\armenian{է}> for only the word in Table \ref{tab:Agulis:phonology:soundChange:diphth:ɑi:e}. 

\begin{table}[H]
 \centering
 \caption{Change from Classical Armenian /ɑi̯/ <\armenian{այ}> to /e/ <\armenian{է}> in the Agulis dialect}
 \label{tab:Agulis:phonology:soundChange:diphth:ɑi:e}
 \begin{tabular}{|l| ll|ll| ll|}
 \hline & \multicolumn{2}{l|}{Classical Armenian} &\multicolumn{2}{l|}{> Agulis} & \multicolumn{2}{l|}{cf. SEA} \\ 
 `vineyard' &ɑi̯ɡ\'i& \armenian{այգի} & \'eɡʲi & \armenian{է՛գյի} &ɑjɡ\'i& \armenian{այգի} \\
 \hline 
 \end{tabular}
\end{table}

Classical Armenian /ɑi̯/ <\armenian{այ}> became /ʏ/ <\armenian{իւ}> for only the word in Table \ref{tab:Agulis:phonology:soundChange:diphth:ɑi:ʏ}. 

\begin{table}[H]
 \centering
 \caption{Change from Classical Armenian /ɑi̯/ <\armenian{այ}> to /ʏ/ <\armenian{իւ}> in the Agulis dialect}
 \label{tab:Agulis:phonology:soundChange:diphth:ɑi:ʏ}
 \begin{tabular}{|l| ll|ll| ll|}
 \hline & \multicolumn{2}{l|}{Classical Armenian} &\multicolumn{2}{l|}{> Agulis} & \multicolumn{2}{l|}{cf. SEA} \\ 
`wolf' & ɡɑi̯l & \armenian{գայլ} & ɡʲʏl & \armenian{գյիւլ} & ɡɑjl & \armenian{գայլ} \\ 
 \hline 
 \end{tabular}
\end{table}


Classical Armenian /ɑi̯/ <\armenian{այ}> became /i/ <\armenian{ի}> for only the word in Table \ref{tab:Agulis:phonology:soundChange:diphth:ɑi:i}. 

\begin{table}[H]
 \centering
 \caption{Change from Classical Armenian /ɑi̯/ <\armenian{այ}> to /i/ <\armenian{ի}> in the Agulis dialect}
 \label{tab:Agulis:phonology:soundChange:diphth:ɑi:i}
 \begin{tabular}{|l| ll|ll| ll|}
 \hline & \multicolumn{2}{l|}{Classical Armenian} &\multicolumn{2}{l|}{> Agulis} & \multicolumn{2}{l|}{cf. SEA} \\ 
`wide' & lɑi̯n & \armenian{լայն} & lin & \armenian{լին} & lɑjn & \armenian{լայն} \\ 
\hline 
 \end{tabular}
\end{table}


\paragraph{Classical Armenian /ɑu̯/ <\armenian{աւ}>}
 
Classical Armenian /ɑu̯/ <\armenian{աւ}> became /ɑv/ <\armenian{ավ}> for the words in Table \ref{tab:Agulis:phonology:soundChange:diphth:ɑu:ɑv}. 

\begin{table}[H]
 \centering
 \caption{Change from Classical Armenian /ɑu̯/ <\armenian{աւ}> to /ɑv/ <\armenian{ավ}> in the Agulis dialect}
 \label{tab:Agulis:phonology:soundChange:diphth:ɑu:ɑv}
 \begin{tabular}{|l| ll|ll| ll|}
 \hline & \multicolumn{2}{l|}{Classical Armenian} &\multicolumn{2}{l|}{> Agulis} & \multicolumn{2}{l|}{cf. SEA} \\ 
`crow' &ɑɡr\'ɑu̯ & \armenian{ագռաւ} & ɑkr\'ɑv &\armenian{ա՛կռավ} & ɑɡr\'ɑv & \armenian{ագռավ} \\
`sand' & ɑ{w\'ɑ}z & \armenian{աւազ} & \'ɑvɑz &\armenian{ա՛վազ} & ɑv\'ɑz & \armenian{ավազ} \\ 
`thirsty' & t͡sɑɾ\'ɑu̯ & \armenian{ծարաւ} & t͡s\'ɑɾɑv & \armenian{ծա՛րավ} & t͡sɑɾ\'ɑv & \armenian{ծարավ} \\ 
`partridge' & kɑkʰ\'ɑu̯ & \armenian{կաքաւ} & k\'ɑkʰɑv &\armenian{կա՛քավ} & kɑkʰ\'ɑv& \armenian{կաքավ} \\ 
 \hline 
 \end{tabular}
\end{table}

Classical Armenian /ɑu̯/ <\armenian{աւ}> became /ov/ <\armenian{օվ}> for the words in Table \ref{tab:Agulis:phonology:soundChange:diphth:ɑu:ov}. 
 
\begin{table}[H]
 \centering
 \caption{Change from Classical Armenian /ɑu̯/ <\armenian{աւ}> to /ov/ <\armenian{օվ}> in the Agulis dialect}
 \label{tab:Agulis:phonology:soundChange:diphth:ɑu:ov}
 \begin{tabular}{|l| ll|ll| ll|}
 \hline & \multicolumn{2}{l|}{Classical Armenian} &\multicolumn{2}{l|}{> Agulis} & \multicolumn{2}{l|}{cf. SEA} \\ 
`cot' &χɑu̯ & \armenian{խաւ} & χov &\armenian{խօվ} & χɑv & \armenian{խավ} \\
 `bird (CA); chicken (SEA)' & hɑu̯ & \armenian{հաւ} & hov & \armenian{հօվ} & hɑv & \armenian{հավ} \\ 
 `agreeable' & hɑ{w\'ɑ}n & \armenian{հաւան} & h\'ovɑn & \armenian{հօ՛վան} & hɑv\'ɑn & \armenian{հավան} \\ 
\hline 
 \end{tabular}
\end{table}

Classical Armenian /ɑu̯/ <\armenian{աւ}> became /o/ <\armenian{օ}> for the words in Table \ref{tab:Agulis:phonology:soundChange:diphth:ɑu:o}. 
 
\begin{table}[H]
 \centering
 \caption{Change from Classical Armenian /ɑu̯/ <\armenian{աւ}> to /o/ <\armenian{օ}> in the Agulis dialect}
 \label{tab:Agulis:phonology:soundChange:diphth:ɑu:o}
 \begin{tabular}{|l| ll|ll| ll|}
 \hline & \multicolumn{2}{l|}{Classical Armenian} &\multicolumn{2}{l|}{> Agulis} & \multicolumn{2}{l|}{cf. SEA} \\ 
`speech' &χ\'ɑu̯skʰ & \armenian{խաւսք} & χoskʰ &\armenian{խօ՛սք} & χ\'oskʰ & \armenian{խոսք} \\
`mother ({\gen})' &mɑu̯ɾ & \armenian{մաւր} & moɾ &\armenian{մօր} &moɾ & \armenian{մոր} \\
`father ({\gen})' &hɑu̯ɾ & \armenian{հաւր} & hoɾ &\armenian{հօր} & hoɾ & \armenian{հոր} \\
 `pigeon'& ɑɬɑu̯n\'i & \armenian{աղաւնի} & jəʁ\'oni & \armenian{ըղօ՛նի} & ɑt͡sel\'i & \armenian{աղավնի} \\
\hline 
 \end{tabular}
\end{table}

Classical Armenian /ɑu̯/ <\armenian{աւ}> became /u/ <\armenian{ու}> for the words in Table \ref{tab:Agulis:phonology:soundChange:diphth:ɑu:u}. 
 
\begin{table}[H]
 \centering
 \caption{Change from Classical Armenian /ɑu̯/ <\armenian{աւ}> to /u/ <\armenian{ու}> in the Agulis dialect}
 \label{tab:Agulis:phonology:soundChange:diphth:ɑu:u}
 \begin{tabular}{|l| ll|ll| ll|}
 \hline & \multicolumn{2}{l|}{Classical Armenian} &\multicolumn{2}{l|}{> Agulis} & \multicolumn{2}{l|}{cf. SEA} \\ 
`prayer' &ɑɬ\'ɑu̯tʰəkʰ & \armenian{աղաւթք} & \'ɑʁutʰkʰ & \armenian{ա՛ղութք} & ɑʁ\'otʰkʰ & \armenian{աղոթք} \\
`shame' & ɑm\'ɑu̯tʰ & \armenian{ամաւթ} & \'ɑmutʰ & \armenian{ա՛մութ} & ɑm\'otʰ & \armenian{ամոթ} \\ 
`to speak' & χɑu̯s\'il & \armenian{խաւսիլ} & χ\'usil & \armenian{խո՛ւսիլ} & χos\'el & \armenian{խոսել} \\ 
`eyebrow' & jɑu̯n-kʰ (-{\pl}) & \armenian{յաւնք} & hunkʰ & \armenian{յունք} & hoŋkʰ & \armenian{հոնք} \\ 
\hline 
 \end{tabular}
\end{table}

\begin{adjarianpage}\label{page:95}\end{adjarianpage}% should be 95


Classical Armenian /ɑu̯/ <\armenian{աւ}> became /ɑ/ <\armenian{ա}> for the words in Table \ref{tab:Agulis:phonology:soundChange:diphth:ɑu:ɑ}. This change happens in the declined forms of some of the previous words.

\begin{table}[H]
 \centering
 \caption{Change from Classical Armenian /ɑu̯/ <\armenian{աւ}> to /ɑ/ <\armenian{ա}> in the Agulis dialect}
 \label{tab:Agulis:phonology:soundChange:diphth:ɑu:ɑ}
 \begin{tabular}{|l| ll|ll| ll|}
 \hline & \multicolumn{2}{l|}{Classical Armenian} &\multicolumn{2}{l|}{> Agulis} & \multicolumn{2}{l|}{cf. SEA} \\ 
`speech' &χɑu̯skʰ & \armenian{խաւսք} & χoskʰ &\armenian{խօսք} & χoskʰ & \armenian{խոսք} \\
`speech-{\gen}' & & & χɑskʰ-i & \armenian{խասքի} & χoskʰ-i & \armenian{խոսքի} \\
`eyebrow' & jɑu̯n-kʰ (-{\pl}) & \armenian{յաւնք} & hunkʰ & \armenian{յունք} & hoŋkʰ & \armenian{հոնք} \\ 
`eyebrow-{\gen}' & & & hɑnkʰ-i & \armenian{յանքի} & hoŋkʰ-i & \armenian{հոնքի} \\ 
\hline 
 \end{tabular}
\end{table}


\paragraph{Classical Armenian /eu̯/ <\armenian{եւ}>}
 
Classical Armenian /eu̯/ <\armenian{եւ}> became /æv/ <\armenian{ա̈վ}> for the words in Table \ref{tab:Agulis:phonology:soundChange:diphth:eu̯:æv}. 
 

\begin{table}[H]
 \centering
 \caption{Change from Classical Armenian /eu̯/ <\armenian{եւ}> to /æv/ <\armenian{ա̈վ}> in the Agulis dialect}
\label{tab:Agulis:phonology:soundChange:diphth:eu̯:æv}
 \begin{tabular}{|l| ll|ll| ll|}
 \hline & \multicolumn{2}{l|}{Classical Armenian} &\multicolumn{2}{l|}{> Agulis} & \multicolumn{2}{l|}{cf. SEA} \\ 
`gray-haired' & ɑle{w\'o}ɾ & \armenian{ալեւոր} & həl\'ævʏɾ & \armenian{հըլա̈՛վիւր}& ɑlev\'oɾ & \armenian{ալևոր} \\ 
\hline 
 \end{tabular}
\end{table}

Classical Armenian /eu̯/ <\armenian{եւ}> became /iv/ <\armenian{իվ}> for the words in Table \ref{tab:Agulis:phonology:soundChange:diphth:eu̯:iv}. 

\begin{table}[H]
 \centering
 \caption{Change from Classical Armenian /eu̯/ <\armenian{եւ}> to /iv/ <\armenian{իվ}> in the Agulis dialect}
\label{tab:Agulis:phonology:soundChange:diphth:eu̯:iv}
 \begin{tabular}{|l| ll|ll| ll|}
 \hline & \multicolumn{2}{l|}{Classical Armenian} &\multicolumn{2}{l|}{> Agulis} & \multicolumn{2}{l|}{cf. SEA} \\ 
`rain' & ɑnd͡zɾ\'eu̯ & \armenian{անձրեւ} & \'ɑnd͡zɾiv & \armenian{ա՛նձրիվ} & ɑnd͡zɾ\'ev & \armenian{անձրեւ} \\ 
 `sun' & ɑɾ\'eu̯& \armenian{արեւ} & \'ɑɾiv &\armenian{ա՛րիվ} & ɑɾ\'ev & \armenian{արև} \\ 
 `light (weight)' &tʰetʰ\'eu̯ & \armenian{թեթեւ} & tʰ\'itʰiv & \armenian{թի՛թիվ} &tʰetʰ\'ev & \armenian{թեթև} \\
 `form' &d͡z'eu̯ & \armenian{ձեւ} & d͡ziv & \armenian{ձիվ} &d͡z'ev & \armenian{ձեւ} \\
\hline 
 \end{tabular}
\end{table}

\paragraph{Classical Armenian /iu̯/ <\armenian{իւ}>}
 
Classical Armenian /iu̯/ <\armenian{իւ}> became /iv/ <\armenian{իվ}> for the words in Table \ref{tab:Agulis:phonology:soundChange:diphth:iu̯:iv}. 

\begin{table}[H]
 \centering
 \caption{Change from Classical Armenian /iu̯/ <\armenian{իւ}> to /iv/ <\armenian{իվ}> in the Agulis dialect}
 \label{tab:Agulis:phonology:soundChange:diphth:iu̯:iv}
 \begin{tabular}{|l| ll|ll| ll|}
 \hline & \multicolumn{2}{l|}{Classical Armenian} &\multicolumn{2}{l|}{> Agulis} & \multicolumn{2}{l|}{cf. SEA} \\ 
`honor' & ɑɾt͡s\'iu̯ & \armenian{արծիւ} & \'ɑɾt͡siv & \armenian{ա՛րծիվ} & ɑɾt͡s\'iv & \armenian{արծիվ} \\ 
`account' & hɑʃ\'iu̯ & \armenian{հաշիւ} & h\'ɑʃiv & \armenian{հա՛շիվ} & hɑʃ\'iv & \armenian{հաշիվ} \\ 
 \hline 
 \end{tabular}
\end{table}


Classical Armenian /iu̯/ <\armenian{իւ}> became /ʏ/ <\armenian{իւ}> for the words in Table \ref{tab:Agulis:phonology:soundChange:diphth:iu̯:ʏ}.\footnote{\translatorHD{The original page had CA <\armenian{ամուր}> as undergoing this sound change to <\armenian{ա̈՛միւր}>. But on page 306, Adjarian provides an erratum that the change is from <\armenian{ալիւր}> to <\armenian{ա̈՛լիւր}>; I fixed it.}} 

\begin{table}[H]
 \centering
 \caption{Change from Classical Armenian /iu̯/ <\armenian{իւ}> to /ʏ/ <\armenian{իւ}> in the Agulis dialect}
 \label{tab:Agulis:phonology:soundChange:diphth:iu̯:ʏ}
 \begin{tabular}{|l| ll|ll| ll|}
 \hline & \multicolumn{2}{l|}{Classical Armenian} &\multicolumn{2}{l|}{> Agulis} & \multicolumn{2}{l|}{cf. SEA} \\ 
 `durable' & ɑl\'iu̯ɾ & \armenian{ալիւր} & \'ælʏɾ & \armenian{ա̈՛լիւր} & ɑlj\'uɾ & \armenian{ալյուր} \\ 
`blood' & ɑɾ\'iu̯n & \armenian{արիւն} & \'æɾʏn & \armenian{ա̈՛րիւն} & ɑɾj\'un & \armenian{արյուն} \\
 \armenian{՝}hundred' & hɑɾ\'iu̯ɾ & \armenian{հարիւր}& h\'æɾʏɾ & \armenian{հա̈՛րիւր} & hɑɾj\'uɾ & \armenian{հարյուր} \\
 \armenian{՝}snow' & d͡ziu̯n & \armenian{ձիւն}& d͡zʏn & \armenian{ձիւն} & d͡zjun & \armenian{ձյուն} \\
 \hline 
\end{tabular}
\end{table}

Classical Armenian /iu̯/ <\armenian{իւ}> became /i/ <\armenian{ի}> for the words in Table \ref{tab:Agulis:phonology:soundChange:diphth:iu̯:i}. This happens for the Classical ending /-tʰiu̯n/ <\armenian{թիւն}>.

\begin{table}[H]
 \centering
 \caption{Change from Classical Armenian /iu̯/ <\armenian{իւ}> to /i/ <\armenian{ի}> in the Agulis dialect}
 \label{tab:Agulis:phonology:soundChange:diphth:iu̯:i}
 \begin{tabular}{|l| ll|ll| ll|}
 \hline & \multicolumn{2}{l|}{Classical Armenian} &\multicolumn{2}{l|}{> Agulis} & \multicolumn{2}{l|}{cf. SEA} \\ 
 `theft' & ɡoɬutʰ\'iu̯n & \armenian{գողութիւն} & kʲəʁ\'otʰin & \armenian{կյըղօ՛թին} & ɡoʁutʰj\'un & \armenian{գողություն} \\ 
 `remission' & tʰoɬutʰ\'iu̯n & \armenian{թողութիւն} & tʰʁ\'otʰin & \armenian{թղօ՛թին} & ʰoʁutʰj\'un & \armenian{թողություն} \\ 
 \hline \end{tabular}
\end{table}


Classical Armenian /iu̯/ <\armenian{իւ}> became /ɑjv/ <\armenian{այվ}> for the words in Table \ref{tab:Agulis:phonology:soundChange:diphth:iu̯:ɑjv}. 

\begin{table}[H]
 \centering
 \caption{Change from Classical Armenian /iu̯/ <\armenian{իւ}> to /ɑjv/ <\armenian{այվ}> in the Agulis dialect}
 \label{tab:Agulis:phonology:soundChange:diphth:iu̯:ɑjv}
 \begin{tabular}{|l| ll|ll| ll|}
 \hline & \multicolumn{2}{l|}{Classical Armenian} &\multicolumn{2}{l|}{> Agulis} & \multicolumn{2}{l|}{cf. SEA} \\ 
`fight' & kəriu̯& \armenian{կռիւ} & krɑjv & \armenian{կռայվ} & kəriv & \armenian{կռիվ} \\ 
 \hline \end{tabular}
\end{table}


\paragraph{Classical Armenian /oi̯/ <\armenian{ոյ}>}

Classical Armenian /oi̯/ <\armenian{ոյ}> became /ʏ/ <\armenian{իւ}> for the words in Table \ref{tab:Agulis:phonology:soundChange:diphth:oi̯:ʏ}. 

\begin{table}[H]
 \centering
 \caption{Change from Classical Armenian /oi̯/ <\armenian{ոյ}> to /ʏ/ <\armenian{իւ}> in the Agulis dialect}
 \label{tab:Agulis:phonology:soundChange:diphth:oi̯:ʏ}
 \begin{tabular}{|l| ll|ll| ll|}
 \hline & \multicolumn{2}{l|}{Classical Armenian} &\multicolumn{2}{l|}{> Agulis} & \multicolumn{2}{l|}{cf. SEA} \\ 
`thumb (CA, SEA); finger (Agulis)' & boi̯tʰ & \armenian{բոյթ} & bʏtʰ & \armenian{բիւթ} & butʰ & \armenian{բութ} \\ 
`nest' & boi̯n & \armenian{բոյն} &bʏn & \armenian{բիւն} & bujn & \armenian{բույն} \\ 
`walnut' & ənk\'oi̯z & \armenian{ընկոյզ} &\'ænɡʲʏz & \armenian{ա̈՛նգյիւզ} & əŋk\'ujz & \armenian{ընկույզ} \\
`light' & loi̯s & \armenian{լոյս} & lʏs & \armenian{լիւս} & lujs & \armenian{լույս} \\ 
`lazy' & t͡soi̯l & \armenian{ծոյլ} & t͡sʏl & \armenian{ծիւլ} & t͡sujl & \armenian{ծույլ} \\ 
 `blue' & kɑp\'oi̯t & \armenian{կապոյտ} & kʲ\'æpʏt & \armenian{կյա̈՛պիւտ} & kɑp\'ujt & \armenian{կապույտ} \\ 
\hline 
 \end{tabular}
\end{table}


\paragraph{Classical Armenian /ɑi̯/ <\armenian{ով}>}

 \translatorHD{I find it odd that Adjarian calls this sequence a diphthong because <\armenian{վ}> most likely was a /v/ sound. This suggests that Adjarian may have actually thought that <\armenian{ով}> was pronounced as /ou̯/ instead of /ov/. }
 
Classical Armenian /ov/ <\armenian{ով}> became /ov/ <\armenian{օվ}> for the words in Table \ref{tab:Agulis:phonology:soundChange:diphth:ov:ov}.
 

\begin{table}[H]
 \centering
 \caption{Change from Classical Armenian /ov/ <\armenian{ով}> to /ov/ <\armenian{օվ}> in the Agulis dialect}
 \label{tab:Agulis:phonology:soundChange:diphth:ov:ov}
 \begin{tabular}{|l| ll|ll| ll|}
 \hline & \multicolumn{2}{l|}{Classical Armenian} &\multicolumn{2}{l|}{> Agulis} & \multicolumn{2}{l|}{cf. SEA} \\ 
`sea' & t͡sov & \armenian{ծով} & t͡sov & \armenian{ծօվ} & t͡sov & \armenian{ծով} \\
 \hline 
 \end{tabular}
\end{table}

Classical Armenian /ov/ <\armenian{ով}> became /uv/ <\armenian{ուվ}> for the words in Table \ref{tab:Agulis:phonology:soundChange:diphth:ov:uv}.
 

\begin{table}[H]
 \centering
 \caption{Change from Classical Armenian /ov/ <\armenian{ով}> to /uv/ <\armenian{ուվ}> in the Agulis dialect}
 \label{tab:Agulis:phonology:soundChange:diphth:ov:uv}
 \begin{tabular}{|l| ll|ll| ll|}
 \hline & \multicolumn{2}{l|}{Classical Armenian} &\multicolumn{2}{l|}{> Agulis} & \multicolumn{2}{l|}{cf. SEA} \\ 
`cow' & kov & \armenian{կով} & kuv &\armenian{կուվ} & kov & \armenian{կով} \\ 
 \hline 
 \end{tabular}
\end{table}

Classical Armenian /ov/ <\armenian{ով}> became /ɑv/ <\armenian{ավ}> for the words in Table \ref{tab:Agulis:phonology:soundChange:diphth:ov:ɑv}.
 

\begin{table}[H]
 \centering
 \caption{Change from Classical Armenian /ov/ <\armenian{ով}> to /ɑv/ <\armenian{ավ}> in the Agulis dialect}
 \label{tab:Agulis:phonology:soundChange:diphth:ov:ɑv}
 \begin{tabular}{|l| ll|ll| ll|}
 \hline & \multicolumn{2}{l|}{Classical Armenian} &\multicolumn{2}{l|}{> Agulis} & \multicolumn{2}{l|}{cf. SEA} \\ 
`sea' & t͡sov & \armenian{ծով} & t͡sov & \armenian{ծօվ} & t͡sov & \armenian{ծով} \\
`sea-{\gen}' & & & t͡s\'ɑv-i & \armenian{ծա՛վի} & t͡sov-\'i & \armenian{ծովի} \\
`to be disturbed' & χərov\'il & \armenian{խռովիլ} & hr\'ɑvil & \armenian{հռա՛վիլ} & χərov\'el & \armenian{խռովել} \\
`accustomed' &sov\'oɾ& \armenian{սովոր} & s\'ɑvuɾ & \armenian{սա՛վուր} & sov\'oɾ & \armenian{սովոր} \\
 \hline 
 \end{tabular}
\end{table}

\subsubsection{Stress and pre-tonic vowel deletion}

Like the Karabakh dialect, stress has moved to the penultimate dialect. Because of this, all pre-tonic vowels have been lost, as the above examples show.
\subsubsection{Consonant changes}


\paragraph{Voicing changes}

The consonants in the Agulis dialect have preserved their native and original pronunciation, more than any dialect. As is clear, the New dialects, including the Tbilisi and Yerevan dialects, have changed voiced sounds to voiceless aspirates after the sound /ɾ/ <\armenian{ր}>. But the Agulis dialect is an exception to this rule, and these same consonants preserve their original pronunciation (Table \ref{tab:Agulis:phonology:soundChange:cons:voice}). 
 

\begin{table}[H]
 \centering
 \caption{Change from Classical Armenian /ov/ <\armenian{ով}> to /ɑv/ <\armenian{ավ}> in the Agulis dialect}
 \label{tab:Agulis:phonology:soundChange:cons:voice} 
 \begin{tabular}{|l| ll|ll| ll|}
 \hline & \multicolumn{2}{l|}{Classical Armenian} &\multicolumn{2}{l|}{> Agulis} & \multicolumn{2}{l|}{cf. SEA} \\ 
`man' &mɑɾd & \armenian{մարդ} & moɾd & \armenian{մօրդ} &mɑɾtʰ & \armenian{մարդ} \\
`rose' &vɑɾd & \armenian{վարդ} & vœɾd & \armenian{վէօրդ} &vɑɾtʰ & \armenian{վարդ} \\
`Transfiguration' &vɑɾdɑvɑr & \armenian{վարդավառ} & vərdɑvʏɾ & \armenian{վըռդավիւր} &vɑɾtʰɑvɑr & \armenian{վարդավառ} \\
`male' &oɾd͡z & \armenian{որձ} & ʏrd͡z & \armenian{իւռձ} &voɾt͡sʰ & \armenian{որձ} \\
`bear' &ɑɾd͡ʒ & \armenian{արջ} & oɾd͡ʒ & \armenian{օրջ} &ɑɾt͡ʃʰ & \armenian{արջ} \\
 \hline 
 \end{tabular}
\end{table}


However among the consonants, some of them have undergone various changes. 

\paragraph{Palatalization of velar stops}

Classical Armenian /ɡ/ <\armenian{գ}> has changed to /ɡʲ/ <\armenian{գյ}> everywhere. But only after the vowel /o/ <\armenian{օ}>, it is pronounced as /ɡ/ <\armenian{գ}>. Similar to this, the sounds /k,kʰ/ <\armenian{կ, ք}> have changed everywhere to /kj, kʰʲ/ <\armenian{կյ, քյ}>. 

\paragraph{Change from Classical /χ/ <\armenian{խ}> to /h/ <\armenian{հ}> }

Classical Armenian /χ/ <\armenian{խ}> has changed everywhere to /h/ <\armenian{հ}> (Table \ref{tab:Agulis:phonology:soundChange:cons:χ}).\footnote{\translatorHD{For the word `locust', Adjarian provides an ancestor <\armenian{մարեխ}> /mɑɾeχ/. Other attested Classical forms include <\armenian{մարախ}> /mɑɾɑχ/, which I suspect is a closer ancestor for Agulis based on the difference in vowels. }} 


\begin{table}[H]
 \centering
 \caption{Change from Classical Armenian /χ/ <\armenian{խ}> to /h/ <\armenian{հ}> in the Agulis dialect}
 \label{tab:Agulis:phonology:soundChange:cons:χ} 
 \begin{tabular}{|l| ll|ll| ll|}
 \hline & \multicolumn{2}{l|}{Classical Armenian} &\multicolumn{2}{l|}{> Agulis} & \multicolumn{2}{l|}{cf. SEA} \\ 
`stall' &ɑχ\'or & \armenian{ախոռ} & \'ɑhur & \armenian{ա՛հուռ} &ɑχ\'or & \armenian{ախոռ} \\ 
 `head' & ɡəluχ & \armenian{գլուխ} & ɡʲəlʏh & \armenian{գյըլիւհ} & ɡəluχ & \armenian{գլուխ} \\
`game' & χɑɬ & \armenian{խաղ} & hɑʁ & \armenian{հաղ} & χɑʁ & \armenian{խաղ} \\
`grape' & χɑɬ\'oɬ & \armenian{խաղող} & h\'ɑʁuʁ & \armenian{հա՛ղուղ} & χɑʁ\'oʁ & \armenian{խաղող} \\
 `to strangle' & χeɬd\'el & \armenian{խեղդել} & h\'eχχil & \armenian{հէ՛ղղիլ} & χeχt\'el & \armenian{խեղդել} \\ 
`deep' & χoɾ & \armenian{խոր} & huɾ & \armenian{հուր} & χoɾ & \armenian{խոր} \\
`bundle' & χuɾd͡z & \armenian{խուրձ} & hœɾznə & \armenian{հէօրզնը} & χuɾt͡sʰ & \armenian{խուրձ} \\
`confused' & χərov & \armenian{խռով} & hrov & \armenian{հռով} & χərov & \armenian{խռով} \\
`to be disturbed' & χərov\'il & \armenian{խռովիլ} & hr\'ɑvil & \armenian{հռա՛վիլ} & χərov\'el & \armenian{խռովել} \\
`advice' & χəɾɑt & \armenian{խրատ} & hɾɾot & \armenian{հրրօտ} & χəɾɑt & \armenian{խրատ} \\
`to spend (CA); to sell (SEA)' & t͡sɑχ\'el & \armenian{ծախել} & t͡s\'ɑhil & \armenian{ծա՛հիլ} & t͡sɑχ\'el & \armenian{ծախել} \\
`smoke' &t͡suχ & \armenian{ծուխ} & t͡soh & \armenian{ծոհ} & t͡suχ & \armenian{ծուխ} \\ 
`to hang' & kɑχ\'el & \armenian{կախել} & k\'ɑhil & \armenian{կա՛հիլ} & kɑχ\'el & \armenian{կախել} \\
`to trample' & koχ\'el & \armenian{կոխել} & k\'ɑhil & \armenian{կա՛հիլ} & koχ\'el & \armenian{կոխել} \\
`pulley' & t͡ʃɑχɑɾ\'ɑk & \armenian{ճախարակ} & t͡ʃh\'ɑɾɑk & \armenian{ճհա՛րակ} & t͡ʃɑχɑɾ\'ɑk & \armenian{ճախարակ} \\
`locust' & mɑɾ\'ɑχ & \armenian{մարախ} & m\'ɑɾɑh & \armenian{մա՛րահ} & mɑɾ\'ɑχ & \armenian{մարախ} \\
`herd of cattle' & nɑχ\'iɾ & \armenian{նախիր} & n\'ɑhiɾ & \armenian{նա՛հիր} & nɑχ\'iɾ& \armenian{նախիր} \\
`onion' & soχ & \armenian{սոխ} & suh & \armenian{սուհ} & soχ & \armenian{սոխ} \\
`vinegar' & kʰɑt͡sʰ\'ɑχ & \armenian{քացախ} & kʰ\'ɑt͡sʰɑh & \armenian{քա՛ցահ} & kʰɑt͡sʰ\'ɑχ & \armenian{քացախ} \\
`sheepfold' & pʰɑɾ\'ɑχ & \armenian{փարախ} & pʰ\'ɑɾɑh & \armenian{փա՛րահ} & pʰɑɾ\'ɑχ & \armenian{փարախ} \\
\hline 
 \end{tabular}
\end{table}

 \begin{adjarianpage}\label{page:96}\end{adjarianpage}% should be 96

\paragraph{Change from Classical /s/ <\armenian{ս}> to /h/ <\armenian{հ}> }

Classical Armenian /s/ <\armenian{ս}> has changed to /h/ <\armenian{հ}> in one word (Table \ref{tab:Agulis:phonology:soundChange:cons:sh}). 


\begin{table}[H]
 \centering
 \caption{Change from Classical Armenian /s/ <\armenian{ս}> to /h/ <\armenian{հ}> in the Agulis dialect}
 \label{tab:Agulis:phonology:soundChange:cons:sh} 
 \begin{tabular}{|l| ll|ll| ll|}
 \hline & \multicolumn{2}{l|}{Classical Armenian} &\multicolumn{2}{l|}{> Agulis} & \multicolumn{2}{l|}{cf. SEA} \\ 
`to say' &ɑs\'el & \armenian{ասել} & \'ɑhil & \armenian{ա՛հիլ} & ɑs\'el & \armenian{ասել} \\ 
\hline 
 \end{tabular}
\end{table}



\paragraph{Change from Classical /t͡sʰ/ <\armenian{ց}> to /h/ <\armenian{հ}> }

In imperatives and in the past participles, the Classical sound /t͡sʰ/ <\armenian{ց}> has changed to /h/ <\armenian{հ}> (Table \ref{tab:Agulis:phonology:soundChange:cons:tsh}). 


\begin{table}[H]
 \centering
 \caption{Change from Classical Armenian /\armenian{Ս}/ <\armenian{ս}> to /h/ <\armenian{հ}> in the Agulis dialect}
 \label{tab:Agulis:phonology:soundChange:cons:tsh} 
 \begin{tabular}{|l| l l l |}
 \hline & \multicolumn{1}{l}{Classical Armenian} &\multicolumn{1}{l}{> Agulis} & \multicolumn{1}{l|}{cf. SEA} \\ 
`I have forgotten' & mor-ɑ-t͡sʰ-e̯\'ɑl e-m & mr-æ-h-\'æl ə-m & mor-ɑ-t͡sʰ-\'el e-m \\ 
 & \armenian{մոռացեալ եմ} & \armenian{մռա̈՛հա̈լ ըմ} & \armenian{մոռացել եմ} \\ 
 & \multicolumn{3}{l|}{$\sqrt{}$-{\lvgloss}-{\aor}-{\perfcvb} {\aux}-1{\sg}} \\ 
 \hline & \multicolumn{1}{l}{Classical Armenian} &\multicolumn{1}{l}{> Agulis} & \multicolumn{1}{l|}{cf. SEA} \\ 
`forget!' & mor-ɑ-t͡sʰ-\'iɾ & mr-æ-h-\'i & mor-ɑ-t͡sʰ-\'iɾ \\ 
 & \armenian{մոռացիր} & \armenian{մռա̈՛հի} & \armenian{մոռացիր} \\ 
 & \multicolumn{3}{l|}{$\sqrt{}$-{\lvgloss}-{\aor}-{\imp}.2{\sg}}
\\ \hline 
 \end{tabular}
\end{table}


\paragraph{Change from Classical ending /n/ <\armenian{ն}>}
 

The ending /n/ <\armenian{ն}> from Old Armenian has changed to /nə/ <\armenian{նը}>, keeping its native form, similar to the Karabakh dialect (Table \ref{tab:Agulis:phonology:soundChange:cons:n}). 


\begin{table}[H]
 \centering
 \caption{Change from Classical ending /n/ <\armenian{ն}> to /nə/ <\armenian{նը}> in the Agulis dialect}
 \label{tab:Agulis:phonology:soundChange:cons:n} 
 \begin{tabular}{|l| ll|ll| ll|}
 \hline & \multicolumn{2}{l|}{Classical Armenian} &\multicolumn{2}{l|}{> Agulis} & \multicolumn{2}{l|}{cf. SEA} \\ 
`mouse' &m\'ukən & \armenian{մուկն} & m\'oknə & \armenian{մօ՛կնը}& muk & \armenian{մուկ} \\ 
`pomegranate' &n\'urən & \armenian{նուռն} & n\'ornə & \armenian{նօ՛ռնը} & nur & \armenian{նուռ} \\ 
`wall' & \'oɾmən & \armenian{որմն} & \'uɾmɑn & \armenian{ո՛ւրման} & voɾm & \armenian{որմ} \\
`worm' & \'oɾdən & \armenian{որդն} & \'ʏrnə & \armenian{իւռնը} & voɾtʰ & \armenian{որդ} \\ 
\hline 
 \end{tabular}
\end{table}.

\section{Morphology}

\subsection{Noun inflection or declension}

\subsubsection{Case suffixes}
In case declension, the following formatives are used. 
\begin{itemize}
 \item {\gen}, {\dat}: The genitive-dative is formed generally with the formative /-i/ <\armenian{ի}>. Proper nouns or names take the formative /-ɑ/ <\armenian{ա}>; but when stress is on the final syllable this /-ɑ/ <\armenian{ա}> turns to /œ/ <\armenian{էօ}>. 
\item {\acc}: The accusative is the same as either the nominative or dative, just as in the Yerevan dialect, based on whether the object is inanimate or animate.
\item {\abl}: The ablative uses the formative /-it͡sʰ/ <\armenian{ից}>, similar to the Yerevan dialect, whereas the Karabakh dialect has /-ɑ, -ɑn/, <\armenian{ա, ան}>. 
\item {\ins}: The instrumental formative is /-ɑv/ <\armenian{ավ}> after stress, but /-œv/ <\armenian{էօվ}> when stressed. 
\item {\locgloss}: The locative takes the unstressed formative \armenian{ամ} /ɑm/, and stressed formative \armenian{ում} /um/
\end{itemize}

\subsubsection{Case infix}
Before these formatives, we can sometimes place the infix (\armenian{միջամասնիկ}) /-hæn/ <\armenian{հա̈ն}>, which corresponds to the Karabakh formatives /-ɑn, -ɑnɑ/ <\armenian{ան, անա}>. In this way, we have the secondary formatives:\begin{itemize}
 \item Ablative: /-h\'æn-it͡sʰ/ <\armenian{հա̈՛նից}>
 \item Instrumental: /-h\'æn-æv/ <\armenian{հա̈՛նա̈վ}>
 \item Locative: /-h\'æn-æm/ <\armenian{հա̈՛նա̈մ}>
\end{itemize} 

\translatorHD{It's not clear to me what this formative is supposed to be. It could be a meaningless stem-extender, as found in SEA pronouns: `from me' /ind͡z-ɑn-it͡sʰ/ glossed as 1{\sg}.{\dat}-{\nx}-{\abl}. Or it could be an oblique marker like the one's used in irregular SEA kinship words: `from a friend' /əŋkeɾ-ot͡ʃʰ-it͡sʰ/ glossed as `friend-{\obl}-{\abl}. }

\subsubsection{Plural declension}

The nominative plural uses the following formatives:\begin{itemize}
 \item /-ɑɾ/ <\armenian{ար}> for monosyllabic words
 \item /-kʰ/ <\armenian{ք}> for vowel-final words
 \item /-neɾ/ <\armenian{նէր}> for the remaining words
\end{itemize} 

The other cases are formed in this way.\begin{itemize}
 \item Genitive-Dative: /-(n)eɾ-i/ <(\armenian{ն)էրի}> 
 \item Ablative: /-(n)eɾ-it͡s/ <(\armenian{ն)էրից}> 
 \item Instrumental: /-(n)eɾ-æv/ <(\armenian{ն)էրա̈վ}> 
 \item Locative: /-(n)eɾ-æm/ <(\armenian{ն)էրա̈մ}> 
\end{itemize} 

\subsection{Pronoun inflection or declension}

\subsubsection{Personal pronouns}
\translatorHD{Table \ref{tab:Agulis:morpho:pronoun:personal} lists the personal pronouns. }

\begin{table}[H]
\caption{Inflection paradigm for personal pronouns in the Agulis dialect }\label{tab:Agulis:morpho:pronoun:personal}
\centering 
\begin{tabular}{| l| llll| }
 \hline & 1SG & 2SG & 1PL & 2PL \\
 & `I' & `you' & `we'& `you' \\\hline 
{\nom} & is& dʏ &mikʰʲ & dʏkʰ \\
 & \armenian{իս} & \armenian{դիւ} & \armenian{միքյ} & \armenian{դիւք} \\
{\gen} & æm&kʰu &miɾ &d͡ziɾ \\
 & \armenian{ա̈մ} & \armenian{քու} & \armenian{միր} & \armenian{ձիր} \\
{\dat},{\acc} &ænd͡z & kʰiz & miz&d͡ziz \\
 & \armenian{ա̈նձ} & \armenian{քիզ} & \armenian{միզ} & \armenian{ձիզ} \\
{\abl} & ænd͡z-\'æn-it͡sʰ & kʰiz-\'æn-it͡sʰ & miz-\'æn-it͡sʰ &d͡ziz-\'æn-it͡sʰ \\
 & \armenian{ա̈նձա̈՛նից} & \armenian{քիզա̈՛նից} & \armenian{միզա̈՛նից} & \armenian{ձիզա̈՛նից} \\
{\ins} & ænd͡z-\'æn-æv& kʰiz-\'æn-æv&miz-\'æn-æv &d͡ziz-\'æn-æv \\
 & \armenian{ա̈նձա̈՛նա̈վ} & \armenian{քիզա̈՛նա̈վ} & \armenian{միզա̈՛նա̈վ} & \armenian{ձիզա̈՛նա̈վ} \\
{\locgloss} & ænd͡z-\'æn-æm & kʰiz-\'æn-æm & miz-\'æn-æm & d͡ziz-\'æn-æm \\
 & \armenian{ա̈նձա̈՛նա̈մ} & \armenian{քիզա̈՛նա̈մ} & \armenian{միզա̈՛նա̈մ} & \armenian{ձիզա̈՛նա̈մ}
 \\ \hline
\end{tabular}
\end{table}

\begin{adjarianpage}\label{page:97}\end{adjarianpage}% should be 97

\subsubsection{Demonstrative pronouns}


In Agulis, the Armenian forms have changed (Table \ref{tab:Agulis:morphology:pronoun:dem:change}). 





\begin{table}[H]
 \centering
 \caption{Changes from Classical demonstratives in the Agulis dialect}
 \label{tab:Agulis:morphology:pronoun:dem:change}
 \begin{tabular}{|l| ll|ll| ll|}
 \hline & \multicolumn{2}{l|}{Classical Armenian} &\multicolumn{2}{l|}{> Agulis} & \multicolumn{2}{l|}{cf. SEA} \\ 
proximal `this' &ɑi̯s, sɑ & \armenian{այս, սա} & hok, so, ɑs & \armenian{հոկ, սօ, աս} & ɑjs, sɑ & \armenian{այս, սա}\\ 
medial `that' &ɑi̯d, dɑ & \armenian{այդ, դա} & dok, do, ɑt & \armenian{դոկ, դօ, ատ} & ɑjd, dɑ & \armenian{այդ, դա}\\ 
distal `that yonder' &ɑi̯n, nɑ & \armenian{այն, նա} & nok, no, ɑn & \armenian{նոկ, նօ, ան} & ɑjn, nɑ & \armenian{այն, նա}\\ 
\hline 
 \end{tabular}
\end{table}


Their declined forms are in Table \ref{tab:Agulis:morpho:pronoun:dem:decl}. \translatorHD{Adjarian provides only a partial paradigm.}

\begin{table}[H]
 \centering
 \caption{Declension of proximinal demonstrative `this' in the Agulis dialect} \label{tab:Agulis:morpho:pronoun:dem:decl}

 \begin{tabular}{|l|ll|ll|}
\hline & \multicolumn{2}{l|}{Singular} & \multicolumn{2}{l|}{Plural} \\
\hline 
{\nom} & ɑs & \armenian{աս} & \'æstʏkʰ& \armenian{ա̈՛ստիւք} \\
{\gen} & ɑs\'uɾ & \armenian{ատո՛ւր} &ɑstut͡sʰ & \armenian{աստուց}\\
{\dat}& ɑst\'uɾ & \armenian{աստո՛ւր} &&\\
{\abl} & ɑstuɾ-\'ɑn-it͡sʰ & \armenian{աստուրա՛նից} & & \\
{\ins} & ɑstuɾ-\'ɑn-ɑv & \armenian{աստուրա՛նավ} & &\\
{\locgloss} & ɑstuɾ-\'ɑn-ɑm & \armenian{աստուրա՛նամ} & & \\
\hline 
\end{tabular}
\end{table}

The form /so/ <\armenian{սօ}> `this' also has the for /zo/ <\armenian{զօ}>. This latter form is used only for animates, while /so/ <\armenian{սօ}> is for both animates and inanimates. 

\subsection{Verb inflection or conjugation}

\subsubsection{Conjugation classes}
In the Agulis dialect, verbal conjugation is very interesting, because many form changes have occurred. Of the four conjugation classes from Old Armenian, only two are kept; these are the /il/ <\armenian{իլ}> and /ol/ <\armenian{օլ}>. 




\begin{table}[H]
 \centering
 \caption{Change in conjugation classes from Classical Armenian to the Agulis dialect}
 \label{tab:Agulis:morphology:verb:class}
 \begin{tabular}{|l| ll|ll| ll|l| }
 \hline & \multicolumn{2}{l|}{Classical Armenian} &\multicolumn{2}{l|}{> Agulis} & \multicolumn{2}{l|}{cf. SEA} & \\ 
`to say' &ɑs-e-l & \armenian{ասել} & ɑh-i-l & \armenian{ահիլ} & ɑs-e-l & \armenian{ասել} & $\sqrt{}$-{\thgloss}-{\infgloss}\\ 
`to go away' &her-ɑ-n-\'ɑ-l & \armenian{հեռանալ} & hr-\'æ-n-i-l & \armenian{հռա̈՛նիլ} &her-ɑ-n-\'ɑ-l & \armenian{հեռանալ} & $\sqrt{}$-{\lvgloss}-{\inch}-{\thgloss}-{\infgloss}\\ 
`to cough' &hɑz-ɑ-l & \armenian{հազալ} & hɑzz-o-l & \armenian{հազզօլ}& hɑz-ɑ-l & \armenian{հազալ} & $\sqrt{}$-{\thgloss}-{\infgloss}\\ 
\hline 
 \end{tabular}
\end{table}

\subsubsection{Copular verb or auxiliary}
 \paragraph{Present copula with /ə-/}


The copular verb has kept only its present tense (Table \ref{tab:Agulis:morpho:verb:prescopula}). The 1PL and 2PL are homophonous. 

\begin{table}[H]
 \centering
 \caption{Present copula with the vowel /ə-/ in the Agulis dialect }
 \label{tab:Agulis:morpho:verb:prescopula}
 \begin{tabular}{|l|ll|ll| }
 \hline & \multicolumn{2}{l|}{Agulis }& \multicolumn{2}{l|}{cf. SEA } \\
 1{\sg} `I am' & ə-m & \armenian{ըմ} & e-m & \armenian{եմ} \\
 2{\sg} `you are' & ə-s & \armenian{ըս} & e-s & \armenian{ես} \\
 3{\sg} `he is' & ɑ & \armenian{ա} & e & \armenian{է}\\
 1{\pl} `we are' & ə-kʰ & \armenian{ըք} & e-ŋkʰ & \armenian{ենք} \\
 2{\pl} `you are' & ə-kʰ & \armenian{ըք} & e-kʰ & \armenian{եք} \\
 3{\pl} `they are' & ə-n & \armenian{ըն} & e-n & \armenian{են} \\
 & \multicolumn{2}{l|}{{\aux}-{\agr}}& \multicolumn{2}{l|}{{\aux}-{\agr}}
\\ \hline 
 \end{tabular}


\end{table}

 \paragraph{Past copula with /ə-/}

The past imperfective has been lost; in its place, the dialect has invented a new strategy (Table \ref{tab:Agulis:morpho:verb:copulaPast}). \translatorHD{Whereas CA and SEA have a suffix /-i/ to encode past tense, Agulis instead uses /nel/ as a particle that marks the past.}


\begin{table}[H]
 \centering
 \caption{Past copula or past auxiliary in the Agulis dialect}
 \label{tab:Agulis:morpho:verb:copulaPast}
 \begin{tabular}{|l|ll| ll| }
 \hline & \multicolumn{2}{l|}{Agulis} & \multicolumn{2}{l|}{cf. SEA} \\ 
1SG `I was' & nel ə-m & \armenian{նէլ ըմ} & ej-i-$\emptyset$ & \armenian{էի}\\ 
2SG `you were' &nel ə-s & \armenian{նէլ ըս} &ej-i-ɾ & \armenian{էիր}\\ 
3SG `he was' & nel ɑ & \armenian{նէլ ա} &e-$\emptyset$-ɾ & \armenian{էր}\\ 
1PL `we were' &nel ə-kʰ & \armenian{նէլ ըք} &ej-i-ŋkʰ & \armenian{էինք}\\ 
2PL `you were' &nel ə-kʰ & \armenian{նէլ ըք} &ej-i-kʰ & \armenian{էիք}\\ 
3PL `they were' & nel ə-n & \armenian{նէլ ըն}&ej-i-n & \armenian{էին}\\ 
& \multicolumn{2}{l|}{{\pst} {\aux}-{\agr}}& \multicolumn{2}{l|}{{\aux}-{\pst}-{\agr}} \\
\hline 
 \end{tabular}
\end{table}


As can be seen, this new strategy for the past auxiliary is a reduced and shortened form of the Classical phrase /eɬe̯ɑl em/ <\armenian{եղեալ եմ}> or /le̯ɑl em/ <\armenian{լեալ եմ}> `I have been'. \translatorHD{Note that this Classical phrase consists of the participle of the verb `to be' /linel/ <\armenian{լինել}>, plus the copula as an auxiliary. }


We shall find a similar usage in the Suceava dialect below. According to this, the imperfective of the Agulis dialect is originally the present perfect (\armenian{յարակատար}). 

\subsubsection{Inflectional paradigm}

\translatorHD{In contrast to the rest of /um/ <\armenian{ում}> branch dialects, Adjarian discusses the Agulis paradigms in depth. His original descriptions and my explanations are interspersed. }

\paragraph{Indicative present and past imperfective}

\translatorHD{The indicative present and past imperfective in SEA are formed via periphrasis (Table \ref{tab:Agulis:morpho:verb:paradigm:presentIndc:cut}). The verb is in a converb form called the imperfective converb. For most regular verbs, this converb is made up of the stem plus the suffix /-um/, without the theme vowel. Irregular monosyllabic verbs instead form the converb by adding the suffix /-is/ after the infinitive (Table \ref{tab:Agulis:morpho:verb:paradigm:presentIndc:give}). Tense and agreement is on the inflected auxiliary, whether present or past. What follows is how Adjarian describes Agulis. Note the difference in the use of converb suffix and the use of a prefix. }

To form present and imperfective indicative stem of verbs, we place the formatives /-um/ <\armenian{ում}> (Table \ref{tab:Agulis:morpho:verb:paradigm:presentIndc:cut}), or /-ɑm, -æm/ <\armenian{ամ, ա̈մ}> (Table \ref{tab:Agulis:morpho:verb:paradigm:presentIndc:goaway}). The first is used when stressed (Table \ref{tab:Agulis:morpho:verb:paradigm:presentIndc:cut}), while the second for unstressed (Table \ref{tab:Agulis:morpho:verb:paradigm:presentIndc:goaway}). \translatorHD{It seems that /-ɑn, -æm/ are allomorphs based on vowel harmony. }

\begin{table}[H]
 \centering
 \caption{Indicative present <\armenian{ներկայ}> of the verb `to cut' in the Agulis dialect, using the converb suffix /-um/ <\armenian{ում}>}
 \label{tab:Agulis:morpho:verb:paradigm:presentIndc:cut}
 \begin{tabular}{|l|ll|ll|}
 \hline & \multicolumn{2}{l|}{Agulis} & \multicolumn{2}{l|}{cf. SEA} \\
1SG & ktɾ-\'um ə-m & \armenian{կտրո՛ւմ ըմ} & kətɾ-\'um e-m & \armenian{կտրում եմ} \\
2SG & ktɾ-\'m ə-s & \armenian{կտրո՛ւմ ըս} & kətɾ-\'um e-s & \armenian{կտրում ես} \\
3SG & ktɾ-\'m ɑ & \armenian{կտրո՛ւմ ա} & kətɾ-\'um e & \armenian{կտրում է} \\
1PL & ktɾ-\'m ə-kʰ & \armenian{կտրո՛ւմ ըք} & kətɾ-\'um e-ŋkʰ & \armenian{կտրում ենք}\\
2PL & ktɾ-\'m ə-kʰ & \armenian{կտրո՛ւմ ըք} & kətɾ-\'um e-kʰ & \armenian{կտրում եք} \\
3PL & ktɾ-\'m ə-n & \armenian{կտրո՛ւմ ըն} & kətɾ-\'um e-n & \armenian{կտրում են} \\
& \multicolumn{2}{l|}{$\sqrt{}$-{\impfcvb} {\aux}-{\agr}}& \multicolumn{2}{l|}{$\sqrt{}$-{\impfcvb} {\aux}-{\agr}}
 \\ \hline 
 \end{tabular} \end{table}
 

\begin{table}[H]
 \centering
 \caption{Indicative present <\armenian{ներկայ}> of the verb `to go away' in the Agulis dialect, using the converb suffix /-æm/ <\armenian{ա̈մ}>}
 \label{tab:Agulis:morpho:verb:paradigm:presentIndc:goaway}
 \begin{tabular}{|l|ll|ll|}
 \hline & \multicolumn{2}{l|}{Agulis} & \multicolumn{2}{l|}{cf. SEA} \\
1SG & hɾ-\'æ-n-æm ə-m & \armenian{հռա̈՛նա̈մ ըմ} & her-ɑ-n-\'um e-m & \armenian{հեռանում եմ} \\
2SG & hɾ-\'æ-n-æm ə-s & \armenian{հռա̈՛նա̈մ ըս} & her-ɑ-n-\'um e-s & \armenian{հեռանում ես} \\
3SG &hɾ-\'æ-n-æm ɑ & \armenian{հռա̈՛նա̈մ ա} & her-ɑ-n-\'um e & \armenian{հեռանում է} \\
1PL & hɾ-\'æ-n-æm ə-kʰ & \armenian{հռա̈՛նա̈մ ըք} & her-ɑ-n-\'um e-ŋkʰ & \armenian{հեռանում ենք}\\
2PL & hɾ-\'æ-n-æm ə-kʰ & \armenian{հռա̈՛նա̈մ ըք} & her-ɑ-n-\'um e-kʰ & \armenian{հեռանում եք} \\
3PL & hɾ-\'æ-n-æm ə-n & \armenian{հռա̈՛նա̈մ ըն} & her-ɑ-n-\'um e-n & \armenian{հեռանում են} \\
& \multicolumn{2}{l|}{$\sqrt{}$-{\lvgloss}-{\inch}-{\impfcvb} {\aux}-{\agr}}& \multicolumn{2}{l|}{$\sqrt{}$-{\lvgloss}-{\inch}-{\impfcvb} {\aux}-{\agr}}
 \\ \hline 
 \end{tabular} \end{table}
 




Monosyllabic verbs take /-is/ <\armenian{իս}> (Table \ref{tab:Agulis:morpho:verb:paradigm:presentIndc:give}). 


\begin{table}[H]
 \centering
 \caption{Indicative present <\armenian{ներկայ}> of the verb `to give' in the Agulis dialect, using the converb suffix /-is/ <\armenian{իս}>}
 \label{tab:Agulis:morpho:verb:paradigm:presentIndc:give}
 \begin{tabular}{|l|ll|ll|}
 \hline & \multicolumn{2}{l|}{Agulis} & \multicolumn{2}{l|}{cf. SEA} \\
1SG & t-\'ɑ-l-is ə-m & \armenian{տա՛լիսըմ} & t-ɑ-l-\'is e-m & \armenian{տալիս եմ} \\
2SG & t-\'ɑ-l-is ə-s & \armenian{տա՛լիս ըս} & t-ɑ-l-\'is e-s & \armenian{տալիս ես} \\
3SG & t-\'ɑ-l-is ɑ & \armenian{տա՛լիս ա} & t-ɑ-l-\'is e & \armenian{տալիս է} \\
1PL & t-\'ɑ-l-is ə-kʰ & \armenian{տա՛լիս ըք} & t-ɑ-l-\'is e-ŋkʰ & \armenian{տալիս ենք}\\
2PL & t-\'ɑ-l-is ə-kʰ & \armenian{տա՛լիս ըք} & t-ɑ-l-\'is e-kʰ & \armenian{տալիս եք} \\
3PL & t-\'ɑ-l-is ə-n & \armenian{տա՛լիս ըն} & t-ɑ-l-\'is e-n & \armenian{տալիս են} \\
& \multicolumn{2}{l|}{$\sqrt{}$-{\thgloss}?-{\infgloss}-{\impfcvb} {\aux}-{\agr}}& \multicolumn{2}{l|}{$\sqrt{}$-{\thgloss}-{\infgloss}-{\impfcvb} {\aux}-{\agr}}
 \\ \hline 
 \end{tabular} \end{table}

 
Besides these, vowel-initial verbs take the prefix /n-/ <\armenian{ն}> (Table \ref{tab:Agulis:morpho:verb:paradigm:presentIndc:say}).



\begin{table}[H]
 \centering
 \caption{Indicative present <\armenian{ներկայ}> of the verb `to cut' in the Agulis dialect, using the converb suffix /-ɑm/ <\armenian{ամ}>, and prefix /n-/ <\armenian{ն}>}
 \label{tab:Agulis:morpho:verb:paradigm:presentIndc:say}
 \begin{tabular}{|l|ll|ll|}
 \hline & \multicolumn{2}{l|}{Agulis} & \multicolumn{2}{l|}{cf. SEA} \\
1SG & n-\'ɑh-ɑm ə-m & \armenian{նա՛համ ըմ} & ɑs-\'um e-m & \armenian{ասում եմ} \\
2SG & n-\'ɑh-ɑm ə-s & \armenian{նա՛համ ըս} & ɑs-\'um e-s & \armenian{ասում ես} \\
3SG & n-\'ɑh-ɑm ɑ & \armenian{նա՛համ ա} & ɑs-\'um e & \armenian{ասում է} \\
1PL & n-\'ɑh-ɑm ə-kʰ & \armenian{նա՛համ ըք} & ɑs-\'um e-ŋkʰ & \armenian{ասում ենք}\\
2PL & n-\'ɑh-ɑm ə-kʰ & \armenian{նա՛համ ըք} & ɑs-\'um e-kʰ & \armenian{ասում եք} \\
3PL & n-\'ɑh-ɑm ə-n & \armenian{նա՛համ ըն} & ɑs-\'um e-n & \armenian{ասում են} \\
& \multicolumn{2}{l|}{?-$\sqrt{}$-{\impfcvb} {\aux}-{\agr}}& \multicolumn{2}{l|}{$\sqrt{}$-{\impfcvb} {\aux}-{\agr}}
 \\ \hline 
 \end{tabular} \end{table}
 
 The past imperfective is formed by adding the form /nel/ <\armenian{նէլ}> to the present. \translatorHD{That is, whereas SEA uses a dedicated past auxiliary, Agulis combines the present auxiliary with a past particle to create the past tense. This past particle /nel/ is then added onto the indicative present to create the indicative past imperfective, regardless if the converb uses /-um/ (Table \ref{tab:Agulis:morpho:verb:paradigm:pastImpfIndc:cut}), /-æm/ (Table \ref{tab:Agulis:morpho:verb:paradigm:pastImpfIndc:goaway}), /-is/ (Table \ref{tab:Agulis:morpho:verb:paradigm:pastImpfIndc:give}), or a prefix /n-/ (Table \ref{tab:Agulis:morpho:verb:paradigm:pastImpfIndc:say}). }

\begin{table}[H]
 \centering
 \caption{Indicative past imperfective <\armenian{անկատար}> of the verb `to cut' in the Agulis dialect}
 \label{tab:Agulis:morpho:verb:paradigm:pastImpfIndc:cut}
 \begin{tabular}{|l|ll|ll|}
\hline & \multicolumn{2}{l|}{Agulis} & \multicolumn{2}{l|}{cf. SEA} \\
1SG & ktɾ-\'um ə-m nel & \armenian{կտրո՛ւմ ըմ նէլ} & kətɾ-\'um ej-i-$\emptyset$ & \armenian{կտրում էի} \\
2SG & ktɾ-\'um ə-s nel & \armenian{կտրո՛ւմ ըս նէլ} & kətɾ-\'um ej-i-ɾ & \armenian{կտրում էիր} \\
3SG & ktɾ-\'um ɑ nel & \armenian{կտրո՛ւմ ա նէլ}& kətɾ-\'um e-$\emptyset$-ɾ & \armenian{կտրում էր} \\
1PL & ktɾ-\'um ə-kʰ nel & \armenian{կտրո՛ւմ ըք նէլ} & kətɾ-\'um ej-i-ŋkʰ & \armenian{կտրում էինք} \\
2PL & ktɾ-\'um ə-kʰ nel &\armenian{կտրո՛ւմ ըք նէլ} & kətɾ-\'um ej-i-kʰ & \armenian{կտրում էիք} \\
3PL & ktɾ-\'um ə-n nel & \armenian{կտրո՛ւմ ըն նէլ} & kətɾ-\'um ej-i-n & \armenian{կտրում էին} \\
& \multicolumn{2}{l|}{$\sqrt{}$-{\impfcvb} {\aux}-{\agr} {\pst}}& \multicolumn{2}{l|}{$\sqrt{}$-{\impfcvb} {\aux}-{\pst}-{\agr}} \\
\hline 
\end{tabular}
\end{table}

\begin{table}[H]
 \centering
 \caption{Indicative past imperfective <\armenian{անկատար}> of the verb `to go away' in the Agulis dialect}
 \label{tab:Agulis:morpho:verb:paradigm:pastImpfIndc:goaway}
 \begin{tabular}{|l|ll|ll|}
\hline & \multicolumn{2}{l|}{Agulis} & \multicolumn{2}{l|}{cf. SEA} \\
1SG & hr-\'æ-n-æm ə-m nel & \armenian{հռա̈՛նա̈մ ըմ նէլ} & her-ɑ-n-\'um ej-i-$\emptyset$ & \armenian{հեռանոում էի} \\
2SG & hr-\'æ-n-æm ə-s nel &\armenian{հռա̈՛նա̈մ ըս նէլ} & her-ɑ-n-\'um ej-i-ɾ & \armenian{հեռանոում էիր} \\
3SG &hr-\'æ-n-æm ɑ nel & \armenian{հռա̈՛նա̈մ ա նէլ}& her-ɑ-n-\'um e-$\emptyset$-ɾ & \armenian{հեռանոում էր} \\
1PL & hr-\'æ-n-æm ə-kʰ nel &\armenian{հռա̈՛նա̈մ ըք նէլ} & her-ɑ-n-\'um ej-i-ŋkʰ & \armenian{հեռանոում էինք} \\
2PL & hr-\'æ-n-æm ə-kʰ nel & \armenian{հռա̈՛նա̈մ ըք նէլ} & her-ɑ-n-\'um ej-i-kʰ & \armenian{հեռանոում էիք} \\
3PL & hr-\'æ-n-æm ə-n nel & \armenian{հռա̈՛նա̈մ ըն նէլ} & her-ɑ-n-\'um ej-i-n & \armenian{հեռանոում էին} \\
& \multicolumn{2}{l|}{$\sqrt{}$-{\lvgloss}-{\inch}-{\impfcvb} {\aux}-{\agr} {\pst}}& \multicolumn{2}{l|}{$\sqrt{}$-{\lvgloss}-{\inch}-{\impfcvb} {\aux}-{\pst}-{\agr}} \\
\hline 
\end{tabular}
\end{table}


\begin{table}[H]
 \centering
 \caption{Indicative past imperfective <\armenian{անկատար}> of the verb `to give' in the Agulis dialect}
 \label{tab:Agulis:morpho:verb:paradigm:pastImpfIndc:give}
 \begin{tabular}{|l|ll|ll|}
\hline & \multicolumn{2}{l|}{Agulis} & \multicolumn{2}{l|}{cf. SEA} \\
1SG & t-\'ɑ-l-is ə-m nel & \armenian{տա՛լիս ըմ նէլ} &t-ɑ-l-\'is ej-i-$\emptyset$ & \armenian{տալիս էի} \\
2SG & t-\'ɑ-l-is ə-s nel &\armenian{տա՛լիս ըս նէլ} & t-ɑ-l-\'is ej-i-ɾ & \armenian{տալիս էիր} \\
3SG &t-\'ɑ-l-is ɑ nel &\armenian{տա՛լիս ա նէլ}& t-ɑ-l-\'is e-$\emptyset$-ɾ & \armenian{տալիս էր} \\
1PL & t-\'ɑ-l-is ə-kʰ nel & \armenian{տա՛լիս ըք նէլ} & t-ɑ-l-\'is ej-i-ŋkʰ & \armenian{տալիս էինք} \\
2PL & t-\'ɑ-l-is ə-kʰ nel & \armenian{տա՛լիս ըք նէլ} & t-ɑ-l-\'is ej-i-kʰ & \armenian{տալիս էիք} \\
3PL & t-\'ɑ-l-is ə-n nel & \armenian{տա՛լիս ըն նէլ} &t-ɑ-l-\'is ej-i-n & \armenian{տալիս էին} \\
& \multicolumn{2}{l|}{$\sqrt{}$-{\thgloss}?-{\infgloss}-{\impfcvb} {\aux}-{\agr} {\pst}}& \multicolumn{2}{l|}{$\sqrt{}$-{\thgloss}?-{\infgloss}-{\impfcvb} {\aux}-{\pst}-{\agr}} \\
\hline 
\end{tabular}
\end{table}

\begin{table}[H]
 \centering
 \caption{Indicative past imperfective <\armenian{անկատար}> of the verb `to say' in the Agulis dialect}
 \label{tab:Agulis:morpho:verb:paradigm:pastImpfIndc:say}
 \begin{tabular}{|l|ll|ll|}
\hline & \multicolumn{2}{l|}{Agulis} & \multicolumn{2}{l|}{cf. SEA} \\
1SG & n-\'ɑh-ɑm ə-m nel &\armenian{նա՛համ ըմ նէլ} & ɑs-\'um ej-i-$\emptyset$ & \armenian{ասում էի} \\
2SG & n-\'ɑh-ɑm ə-s nel & \armenian{նա՛համ ըս նէլ} & ɑs-\'um ej-i-ɾ & \armenian{ասում էիր} \\
3SG & n-\'ɑh-ɑm ɑ nel &\armenian{նա՛համ ա նէլ}& ɑs-\'um e-$\emptyset$-ɾ & \armenian{ասում էր} \\
1PL & n-\'ɑh-ɑm ə-kʰ nel & \armenian{նա՛համ ըք նէլ} & ɑs-\'um ej-i-ŋkʰ & \armenian{ասում էինք} \\
2PL & n-\'ɑh-ɑm ə-kʰ nel &\armenian{նա՛համ ըք նէլ} & ɑs-\'um ej-i-kʰ & \armenian{ասում էիք} \\
3PL & n-\'ɑh-ɑm ə-n nel & \armenian{նա՛համ ըն նէլ} & ɑs-\'um ej-i-n & \armenian{ասում էին} \\
& \multicolumn{2}{l|}{?-$\sqrt{}$-{\impfcvb} {\aux}-{\agr} {\pst}}& \multicolumn{2}{l|}{$\sqrt{}$-{\impfcvb} {\aux}-{\pst}-{\agr}} \\
\hline 
\end{tabular}
\end{table}

\begin{adjarianpage}\label{page:98}\end{adjarianpage}% should be 98

\paragraph{Past perfective or aorist}

The past perfective is lost. Agulis has replaced it with either the present perfect (\armenian{յարակատար}) or with a new strategy, which is similar to the Old Armenian present.

\translatorHD{What he means is the following. In SEA, the aorist or past perfective is marked synthetically by using the aorist stem (Table \ref{tab:Agulis:morpho:verb:paradigm:pastperfectiveAorist:cut}). For a verb like `to cut' /kətɾ-e-l/, the past perfective is marked by adding the aorist suffix /-t͡sʰ/ after the theme vowel, and then adding the past suffix /-i/ and the appropriate agreement suffixes. The 3SG uses a covert tense and agreement suffix. In contrast in Agulis, the past perfective is marked periphrastically. The first strategy that Adjarian describes is by combining the perfective converb (with suffix /-el/, also called the past participle) with an inflected auxiliary. In Agulis, there is evidence that the vowel /e/ in /-el/ is actually a separate theme vowel. }

\begin{table}[H]
 \centering
 \caption{Past perfective or aorist <\armenian{կատարեալ}> of the verb `to cut' in the Agulis dialect}
 \label{tab:Agulis:morpho:verb:paradigm:pastperfectiveAorist:cut}
 \begin{tabular}{|l|ll|ll|}
\hline & \multicolumn{2}{l|}{Agulis} & \multicolumn{2}{l|}{cf. SEA} \\
1SG & kətɾ-e-l ə-m & \armenian{կտրէլ ըմ} & kətɾ-e-t͡sʰ-i-$\emptyset$ & \armenian{կտրեցի} \\
2SG & kətɾ-e-l ə-s & \armenian{կտրէլ ըս} & kətɾ-e-t͡sʰ-i-ɾ & \armenian{կտրեցիր} \\
3SG & kətɾ-e-l ɑ & \armenian{կտրէլ ա} & kətɾ-e-t͡sʰ-$\emptyset$-$\emptyset$ & \armenian{կտրեց} \\
1PL & kətɾ-e-l ə-kʰ &\armenian{կտրէլ ըք} & kətɾ-e-t͡sʰ-i-ŋkʰ & \armenian{կտրեցինք} \\
2PL & kətɾ-e-l ə-kʰ &\armenian{կտրէլ ըք} & kətɾ-e-t͡sʰ-i-kʰ & \armenian{կտրեցիք} \\
3PL & kətɾ-e-l ə-n & \armenian{կտրէլ ըն} & kətɾ-e-t͡sʰ-i-n & \armenian{կտրեցին} \\
& \multicolumn{2}{l|}{$\sqrt{}$-{\thgloss}-{\perfcvb} {\aux}-{\agr}}& \multicolumn{2}{l|}{$\sqrt{}$-{\thgloss}-{\aor}-{\pst}-{\agr}}\\ 
\hline 
\end{tabular}
\end{table}

\translatorHD{Such a morphological structure (perfective converb + auxiliary) exists in SEA too, but as a marker of the present perfect, not the past perfective. Note the following contrasts in (\ref{sent:agulis:morpho:verb:seapast}) for better illustration. }

\begin{exe}
\ex \label{sent:agulis:morpho:verb:seapast}\begin{xlist}
 
 \ex Agulis \gll ktɾ-e-l ə-n \\
 cut-{\thgloss}-{\perfcvb} {\aux}-3{\pl} \\
 \trans `They cut (in the past).' \\
 \armenian{կտրէլ ըն}
\ex SEA \gll kətɾ-el e-n \\
 cut-{\perfcvb} {\aux}-3{\pl} \\
 \trans `They have cut' \\
 \armenian{կտրել են}
\end{xlist}
\end{exe}

\translatorHD{In Agulis, the use of the perfective converb (past participle) to mark the past perfective is robust. Adjarian provides paradigms for two other verbs that use the past perfective in this way (Table \ref{tab:Agulis:morpho:verb:paradigm:pastperfectiveAorist:other}). Note how some verbs like `to cut' use a suffix /-el/, while `to say' uses /-ɑl/ and `to go for away' uses /-æl/. Inter-verb variation suggests that morphemes like /-el, -ɑl, -æl/ are actually bimorphemic with a theme vowel: /-e-l, -ɑ-l, -æ-l/.}


\begin{table}[H]
 \centering
 \caption{Past perfective or aorist <\armenian{կատարեալ}> of the verbs `to say' and `to go away' in the Agulis dialect}
 \label{tab:Agulis:morpho:verb:paradigm:pastperfectiveAorist:other}
 \begin{tabular}{|l|ll|ll|}
\hline & \multicolumn{2}{l|}{`to say'} & \multicolumn{2}{l|}{`to go away'} \\
1SG &\'ɑh-ɑ-l ə-m & \armenian{ա՛հալ ըմ} & hɾ-\'æ-h-æ-l ə-m & \armenian{հռա̈՛հա̈լ ըմ} \\
2SG &\'ɑh-ɑ-l ə-s & \armenian{ա՛հալ ըս} & hɾ-\'æ-h-æ-l ə-s & \armenian{հռա̈՛հա̈լ ըս} \\
3SG &\'ɑh-ɑ-l ɑ &\armenian{ա՛հալ ա} & hɾ-\'æ-h-æ-l ɑ &\armenian{հռա̈՛հա̈լ ա} \\
1PL & \'ɑh-ɑ-l ə-kʰ &\armenian{կտրէլ ըք} & hɾ-\'æ-h-æ-l ə-kʰ &\armenian{հռա̈՛հա̈լ ըք} \\
2PL & \'ɑh-ɑ-l ə-kʰ &\armenian{ա՛հալ ըք}& hɾ-\'æ-h-æ-l ə-kʰ &\armenian{հռա̈՛հա̈լ ըք} \\
3PL & \'ɑh-ɑ-l ə-n & \armenian{ա՛հալ ըն} & hɾ-\'æ-h-æ-l ə-n & \armenian{հռա̈՛հա̈լ ըն} \\
& \multicolumn{2}{l|}{$\sqrt{}$-{\thgloss}-{\perfcvb} {\aux}-{\agr}}& \multicolumn{2}{l|}{$\sqrt{}$-{\lvgloss}-{\aor}-{\thgloss}-{\perfcvb} {\aux}-{\agr}}\\ 
\hline 
\end{tabular}
\end{table}


\translatorHD{To help understand the above paradigms, consider the perfective converbs of these two verbs across the two dialects (\ref{sent:agulis:morpho:verb:seapastMore}). The converb uses a non-alternative suffix /-el/ in SEA, while this converb's theme changes in Agulis based on the verb. For inchoative verbs like `to go away', this converb uses the aorist suffix /-t͡sʰ-/ in SEA and /-h-/ in Agulis.}


\begin{exe}
\ex \label{sent:agulis:morpho:verb:seapastMore}\begin{xlist}
 
 \ex Agulis \gll \'ɑh-ɑ-l ə-m, hr-\'æ-h-æ-l ə-m \\
 say-{\thgloss}-{\perfcvb} {\aux}-3{\pl}, go.away-{\lvgloss}-{\aor}-{\thgloss}-{\perfcvb} {\aux}-3{\pl} \\
 \trans `They said; they went away.' \\
 \armenian{ա՛հալ ըն, հռա̈՛հա̈լ ըն}
\ex SEA \gll ɑs-\'el e-n her-ɑ-t͡sʰ-\'el e-n\\
 say-{\perfcvb} {\aux}-3{\pl}, go.away-{\lvgloss}-{\aor}-{\perfcvb} {\aux}-3{\pl} \\
 \trans `They have said; they have gone away.' \\
 \armenian{ասել են, հեռացել են}
\end{xlist}
\end{exe}



\translatorHD{The second strategy that Adjarian describes is the following. Classical Armenian had a synthetic construction for the indicative present in which the present agreement suffixes are added after the theme vowel. SEA inherited this construction and uses it to mark the subjunctive present. In contrast, Agulis uses it to mark the past perfective. I illustrate by contrasting the SEA subjunctive present against the Agulis past perfective (Table \ref{tab:Agulis:morpho:verb:paradigm:PastPerfsubjPresent}). Note that the 3SG is ineffable in Agulis with this strategy. The 1PL and 2PL are homophonous. For `to go away', this verb uses its inchoative form /-n-/ in SEA, but Agulis uses the cognate form /-h-/ of the aorist suffix /-t͡sʰ-/. }


\begin{table}[H]
 \centering
 \caption{Past perfective <\armenian{կատարեալ}> of the verb `to cut', `to say', and `to go away' in the Agulis dialect, contrasting against the subjunctive present of SEA}
 \label{tab:Agulis:morpho:verb:paradigm:PastPerfsubjPresent}
 \begin{tabular}{|l|ll|ll|}
\hline & \multicolumn{2}{l|}{Agulis past perfective} & \multicolumn{2}{l|}{cf. SEA subjunctive present} \\ \hline
& `to cut' & & & \\
1SG & ktɾ-e-m &\armenian{կտրէմ} & kətɾ-e-m & \armenian{կտրեմ} \\
2SG & ktɾ-e-s & \armenian{կտրէս} & kətɾ-e-s & \armenian{կտրես} \\
3SG & & & kətɾ-i-$\emptyset$ & \armenian{կտրի} \\
1PL & ktɾ-e-kʰ &\armenian{կտրէք} & kətɾ-e-ŋkʰ & \armenian{կտրենք} \\
2PL & ktɾ-e-kʰ & \armenian{կտրէք} & kətɾ-e-kʰ & \armenian{կտրեք} \\
3PL & ktɾ-e-n &\armenian{կտրէն} & kətɾ-e-n & \armenian{կտրեն} \\
& \multicolumn{2}{l|}{$\sqrt{}$-{\thgloss}-{\agr}}& \multicolumn{2}{l|}{$\sqrt{}$-{\thgloss}-{\agr}}\\ 
\hline 
& `to say' & & & \\
1SG & \'ɑh-ɑ-m &\armenian{կտրէմ} & ɑs-\'e-m & \armenian{ասեմ} \\
2SG & \'ɑh-ɑ-s & \armenian{կտրէս} & ɑs-\'e-s & \armenian{ասես} \\
3SG & & & ɑs-\'i-$\emptyset$ & \armenian{ասի} \\
1PL & \'ɑh-ɑ-kʰ &\armenian{կտրէք} & ɑs-\'e-ŋkʰ & \armenian{ասենք} \\
2PL & \'ɑh-ɑ-kʰ & \armenian{կտրէք} & ɑs-\'e-kʰ & \armenian{ասեք} \\
3PL & \'ɑh-ɑ-n &\armenian{կտրէն} & ɑs-\'e-n & \armenian{ասեն} \\
& \multicolumn{2}{l|}{$\sqrt{}$-{\thgloss}-{\agr}}& \multicolumn{2}{l|}{$\sqrt{}$-{\thgloss}-{\agr}}\\ 
\hline 
& `to go away' & & & \\
1SG & hr-\'æ-h-æ-m &\armenian{կտրէմ} & her-ɑ-n-\'ɑ-m & \armenian{հեռանամ} \\
2SG & hr-\'æ-h-æ-s & \armenian{կտրէս} &her-ɑ-n-\'ɑ-s & \armenian{հեռանաս} \\
3SG & & & her-ɑ-n-\'ɑ-$\emptyset$ & \armenian{հեռանա} \\
1PL & hr-\'æ-h-æ-kʰ &\armenian{կտրէք} & her-ɑ-n-\'ɑ-ŋkʰ & \armenian{հեռանանք} \\
2PL & hr-\'æ-h-æ-kʰ & \armenian{կտրէք} & her-ɑ-n-\'ɑ-kʰ & \armenian{հեռանաք} \\
3PL & hr-\'æ-h-æ-n &\armenian{կտրէն} & her-ɑ-n-\'ɑ-n & \armenian{հեռանան} \\
& \multicolumn{2}{l|}{$\sqrt{}$-{\thgloss}-{\agr}}& \multicolumn{2}{l|}{$\sqrt{}$-{\lvgloss}-{\inch}-{\thgloss}-{\agr}}\\ 
\hline 
\end{tabular}
\end{table}



\paragraph{Present perfect and past perfect}

The present perfect (\armenian{յարակատար}) is replaced by the second form of the past participle (with the formative /-ɑt͡s/ <\armenian{ած}>) (Table \ref{tab:Agulis:morpho:verb:paradigm:presentPerfect}). On this form, the familiar formative /nel/ <\armenian{նէլ}> is added to create the past perfect (\armenian{գերակատար}).

\translatorHD{Adjarian prose is quite succinct for a complicated topic. Essentially, for the present perfect, Agulis developed a periphrastic strategy that is more like SWA than SEA. In SEA, there are two past-oriented participles. One is the perfective converb with suffix /-el/. This converb is used for the present perfect. There is another non-finite form called the resultative participle with suffix /-ɑt͡s/. This form is not used in any periphrastic tenses in SEA. In contrast in SWA, there is no perfective converb. The cognate of the resultative suffix /-ɑd͡z/ is used to mark the present perfect. Agulis behaves like SWA because it uses the cognate of the resultative suffix /-e-t͡s/ to mark the present perfect. }


\begin{adjarianpage}\label{page:99}\end{adjarianpage}% should be 99

\begin{table}[H]
 \centering
 \caption{Present perfect <\armenian{յարակատար}> of the verb `to cut' in the Agulis dialect}
 \label{tab:Agulis:morpho:verb:paradigm:presentPerfect}
 \begin{tabular}{|l|ll|ll|ll|}
\hline & \multicolumn{2}{l|}{Agulis} & \multicolumn{2}{l|}{cf. SWA} & \multicolumn{2}{l|}{cf. SEA} \\
1SG & ktɾ-e-t͡s ə-m & \armenian{կտրէծ ըմ} & ɡədɾ-ɑd͡z e-m & \armenian{կտրած եմ} & kətɾ-el e-m & \armenian{կտրել եմ} \\
2SG & ktɾ-e-t͡s ə-s & \armenian{կտրէծ ըս} & ɡədɾ-ɑd͡z e-s & \armenian{կտրած ես} & kətɾ-el e-s & \armenian{կտրել ես} \\
3SG & ktɾ-e-t͡s ɑ & \armenian{կտրէծ ա} & ɡədɾ-ɑd͡z e & \armenian{կտրած է} & kətɾ-el e & \armenian{կտրել է} \\
1PL &ktɾ-e-t͡s ə-kʰ & \armenian{կտրէծ ըք} & ɡədɾ-ɑd͡z e-ŋkʰ & \armenian{կտրած ենք} & kətɾ-el e-ŋkʰ & \armenian{կտրել ենք} \\
2PL & ktɾ-e-t͡s ə-kʰ & \armenian{կտրէծ ըք} & ɡədɾ-ɑd͡z e-kʰ & \armenian{կտրած էք} & kətɾ-el e-kʰ & \armenian{կտրել եք} \\
3PL & ktɾ-e-t͡s ə-n & \armenian{կտրէծ ըն} & ɡədɾ-ɑd͡z e-n & \armenian{կտրած են} & kətɾ-el e-n & \armenian{կտրել են} \\
& \multicolumn{2}{l|}{$\sqrt{}$-{\thgloss}-{\rptcp} {\aux}-{\agr}}& \multicolumn{2}{l|}{$\sqrt{}$-{\rptcp} {\aux}-{\agr}}& \multicolumn{2}{l|}{$\sqrt{}$-{\perfcvb} {\aux}-{\agr}}\\ 

\hline 
\end{tabular}
\end{table}

\translatorHD{In SEA and SEA, the resultative suffix is a non-alternative suffix: SEA /-ɑt͡s/ and SWA /-ɑd͡z/. But in Agulis, there is evidence that the vowel is a separate theme vowel because it alternates across verbs: /-e-t͡s/ for `to cut', /-ɑ-t͡s/ for `to say', and /-æ-t͡s/ for `to go away' (Table \ref{tab:Agulis:morpho:verb:paradigm:presentPerfectMore}). }





\begin{table}[H]
 \centering
 \caption{Present perfect <\armenian{յարակատար}> of the verb `to say' and `to go away' in the Agulis dialect}
 \label{tab:Agulis:morpho:verb:paradigm:presentPerfectMore}
 \begin{tabular}{|l|ll|ll|ll|}
\hline & \multicolumn{2}{l|}{Agulis} & \multicolumn{2}{l|}{cf. SWA} & \multicolumn{2}{l|}{cf. SEA} \\ \hline 
& `to say' & & & & & \\
1SG & \'ɑh-ɑ-t͡s ə-m & \armenian{ա՛հած ըմ} & əs-\'ɑd͡z e-m & \armenian{ըսած եմ} & ɑs-\'el e-m & \armenian{ասել եմ} \\
2SG & \'ɑh-ɑ-t͡s ə-s & \armenian{ա՛հած ըս} & əs-\'ɑd͡z e-s & \armenian{ըսած ես} & ɑs-\'el e-s & \armenian{ասել ես} \\
3SG & \'ɑh-ɑ-t͡s ɑ & \armenian{ա՛հած ա} & əs-\'ɑd͡z e & \armenian{ըսած է} & ɑs-\'el e & \armenian{ասել է} \\
1PL &\'ɑh-ɑ-t͡s ə-kʰ & \armenian{ա՛հած ըք} & əs-\'ɑd͡z e-ŋkʰ & \armenian{ըսած ենք} & ɑs-\'el e-ŋkʰ & \armenian{ասել ենք} \\
2PL & \'ɑh-ɑ-t͡s ə-kʰ & \armenian{ա՛հած ըք} & əs-\'ɑd͡z e-kʰ & \armenian{ըսած էք} & ɑs-\'el e-kʰ & \armenian{ասել եք} \\
3PL & \'ɑh-ɑ-t͡s ə-n & \armenian{ա՛հած ըն} & əs-\'ɑd͡z e-n & \armenian{ըսած են} & ɑs-\'el e-n & \armenian{ասել են} \\
& \multicolumn{2}{l|}{$\sqrt{}$-{\thgloss}-{\rptcp} {\aux}-{\agr}}& \multicolumn{2}{l|}{$\sqrt{}$-{\rptcp} {\aux}-{\agr}}& \multicolumn{2}{l|}{$\sqrt{}$-{\perfcvb} {\aux}-{\agr}}\\ 
\hline 
& `to go away' & & & & & \\
1SG &hr-\'æ-h-æ-t͡s ə-m & \armenian{հռա̈՛հա̈ծ ըմ} & heɾ-ɑ-t͡sʰ-\'ɑd͡z e-m & \armenian{հեռացած եմ} & her-ɑ-t͡sʰ-\'el e-m & \armenian{հեռացել եմ} \\
2SG & hr-\'æ-h-æ-t͡s ə-s & \armenian{հռա̈՛հա̈ծ ըս} & heɾ-ɑ-t͡sʰ-\'ɑd͡z e-s & \armenian{հեռացած ես} & her-ɑ-t͡sʰ-\'el e-s & \armenian{հեռացել ես} \\
3SG & hr-\'æ-h-æ-t͡s ɑ & \armenian{հռա̈՛հա̈ծ ա} & heɾ-ɑ-t͡sʰ-\'ɑd͡z e & \armenian{հեռացած է} & her-ɑ-t͡sʰ-\'el e & \armenian{հեռացել է} \\
1PL &hr-\'æ-h-æ-t͡s ə-kʰ &\armenian{հռա̈՛հա̈ծ ըք} & heɾ-ɑ-t͡sʰ-\'ɑd͡z e-ŋkʰ & \armenian{հեռացած ենք} & her-ɑ-t͡sʰ-\'el e-ŋkʰ & \armenian{հեռացել ենք} \\
2PL & hr-\'æ-h-æ-t͡s ə-kʰ &\armenian{հռա̈՛հա̈ծ ըք} & heɾ-ɑ-t͡sʰ-\'ɑd͡z e-kʰ & \armenian{հեռացած էք} & her-ɑ-t͡sʰ-\'el e-kʰ & \armenian{հեռացել եք} \\
3PL & hr-\'æ-h-æ-t͡s ə-n & \armenian{հռա̈՛հա̈ծ ըն} &heɾ-ɑ-t͡sʰ-\'ɑd͡z e-n & \armenian{հեռացած են} & her-ɑ-t͡sʰ-\'el e-n & \armenian{հեռացել են} \\
& \multicolumn{2}{l|}{$\sqrt{}$-{\lvgloss}-{\aor}-{\thgloss}-{\rptcp} {\aux}-{\agr}}& \multicolumn{2}{l|}{$\sqrt{}$-{\lvgloss}-{\aor}-{\rptcp} {\aux}-{\agr}}& \multicolumn{2}{l|}{$\sqrt{}$-{\lvgloss}-{\aor}-{\perfcvb} {\aux}-{\agr}}\\ 
\hline 
\end{tabular}
\end{table}
 
\translatorHD{To form the past perfect, SEA and SWA replace the present auxiliary with the past auxiliary: 3PL present /e-n/ `they are' <\armenian{են}> vs. past /ej-i-n/ `they were' <\armenian{էին}>. In Agulis, there is no dedicated morph for the past auxiliary; instead the `past auxiliary' is made up of the present auxiliary plus the past particle /nel/: 3PL present /ə-n/ `they are' <\armenian{ըն}> vs. past /ə-n nel/ <\armenian{ըն նէլ}> (Table \ref{tab:Agulis:morpho:verb:paradigm:pastPerfect}).}


\begin{table}[H]
 \centering
 \caption{Past perfect <\armenian{գերակատար}> of the verb `to cut', `to say', and `to go away' in the Agulis dialect}
 \label{tab:Agulis:morpho:verb:paradigm:pastPerfect}
 \begin{tabular}{|l|ll|ll|ll| }
\hline & \multicolumn{2}{l|}{Agulis} & \multicolumn{2}{l|}{cf. SWA}& \multicolumn{2}{l|}{cf. SEA} \\\hline 
& `to cut' & & & & & \\
1SG & ktɾ-e-t͡s ə-m nel & \armenian{կտրէծ ըմ նէլ} & ɡədɾ-ɑd͡z ej-i-$\emptyset$ & \armenian{կտրած էի} & kətɾ-el ej-i-$\emptyset$ & \armenian{կտրել էի} \\
2SG & ktɾ-e-t͡s ə-s nel & \armenian{կտրէծ ըս նէլ} & ɡədɾ-ɑd͡z ej-i-ɾ & \armenian{կտրած էիր} & kətɾ-el ej-i-ɾ & \armenian{կտրել էիր} \\
3SG &ktɾ-e-t͡s ɑ nel & \armenian{կտրէծ ա նէլ} & ɡədɾ-ɑd͡z e-$\emptyset$-ɾ & \armenian{կտրած էր} & kətɾ-el e-$\emptyset$-ɾ & \armenian{կտրել էր} \\
1PL &ktɾ-e-t͡s ə-kʰ nel & \armenian{կտրէծ ըք նէլ} & ɡədɾ-ɑd͡z ej-i-ŋkʰ & \armenian{կտրած էինք}& kətɾ-el ej-i-ŋkʰ & \armenian{կտրել էինք} \\
2PL & ktɾ-e-t͡s ə-kʰ nel & \armenian{կտրէծ ըք նէլ} & ɡədɾ-ɑd͡z ej-i-kʰ & \armenian{կտրած էիք} & kətɾ-el ej-i-kʰ & \armenian{կտրել էիք} \\
3PL &ktɾ-e-t͡s ə-n nel & \armenian{կտրէծ ըն նէլ} & ɡədɾ-ɑd͡z ej-i-n & \armenian{կտրած էին} & kətɾ-el ej-i-n & \armenian{կտրել էին} \\
& \multicolumn{2}{l|}{$\sqrt{}$-{\thgloss}-{\rptcp} {\aux}-{\agr} {\pst}}& \multicolumn{2}{l|}{$\sqrt{}$-{\rptcp} {\aux}-{\pst}-{\agr}}& \multicolumn{2}{l|}{$\sqrt{}$-{\perfcvb} {\aux}-{\pst}-{\agr}}\\ 
\hline 
& `to say' & & & & & \\
1SG & \'ɑh-ɑ-t͡s ə-m nel & \armenian{ա՛հած ըմ նէլ} & əs-\'ɑd͡z ej-i-$\emptyset$ & \armenian{ըսած էի} & ɑs-\'el ej-i-$\emptyset$ & \armenian{ասել էի} \\
2SG & \'ɑh-ɑ-t͡s ə-s nel & \armenian{ա՛հած ըս նէլ} & əs-\'ɑd͡z ej-i-ɾ & \armenian{ըսած էիր} & ɑs-\'el ej-i-ɾ & \armenian{ասել էիր} \\
3SG &\'ɑh-ɑ-t͡s ɑ nel & \armenian{ա՛հած ա նէլ} & əs-\'ɑd͡z e-$\emptyset$-ɾ & \armenian{ըսած էր} & ɑs-\'el e-$\emptyset$-ɾ & \armenian{ասել էր} \\
1PL &\'ɑh-ɑ-t͡s ə-kʰ nel & \armenian{ա՛հած ըք նէլ} & əs-\'ɑd͡z ej-i-ŋkʰ & \armenian{ըսած էինք}& ɑs-\'el ej-i-ŋkʰ & \armenian{ասել էինք} \\
2PL & \'ɑh-ɑ-t͡s ə-kʰ nel & \armenian{ա՛հած ըք նէլ} & əs-\'ɑd͡z ej-i-kʰ & \armenian{ըսած էիք} & ɑs-\'el ej-i-kʰ & \armenian{ասել էիք} \\
3PL &\'ɑh-ɑ-t͡s ə-n nel & \armenian{ա՛հած ըն նէլ} & əs-\'ɑd͡z ej-i-n & \armenian{ըսած էին} & ɑs-\'el ej-i-n & \armenian{ասել էին} \\
& \multicolumn{2}{l|}{$\sqrt{}$-{\thgloss}-{\rptcp} {\aux}-{\agr} {\pst}}& \multicolumn{2}{l|}{$\sqrt{}$-{\rptcp} {\aux}-{\pst}-{\agr}}& \multicolumn{2}{l|}{$\sqrt{}$-{\perfcvb} {\aux}-{\pst}-{\agr}}\\ 
\hline 
& `to go away' & & & & & \\
1SG & hr-\'æ-h-æ-t͡s ə-m nel & \armenian{հռա̈՛հա̈ծ ըմ նէլ} & heɾ-ɑ-t͡sʰ-\'ɑd͡z ej-i-$\emptyset$ & \armenian{հեռացած էի} & her-ɑ-t͡sʰ-e\'-el ej-i-$\emptyset$ & \armenian{հեռացել էի} \\
2SG & hr-\'æ-h-æ-t͡s ə-s nel & \armenian{հռա̈՛հա̈ծ ըս նէլ} & heɾ-ɑ-t͡sʰ-\'ɑd͡z ej-i-ɾ & \armenian{հեռացած էիր} & her-ɑ-t͡sʰ-e\'-el ej-i-ɾ & \armenian{հեռացել էիր} \\
3SG &hr-\'æ-h-æ-t͡s ɑ nel & \armenian{հռա̈՛հա̈ծ ա նէլ} & heɾ-ɑ-t͡sʰ-\'ɑd͡z e-$\emptyset$-ɾ & \armenian{հեռացած էր} & her-ɑ-t͡sʰ-e\'-el e-$\emptyset$-ɾ & \armenian{հեռացել էր} \\
1PL &hr-\'æ-h-æ-t͡s ə-kʰ nel & \armenian{հռա̈՛հա̈ծ ըք նէլ} & heɾ-ɑ-t͡sʰ-\'ɑd͡z ej-i-ŋkʰ & \armenian{հեռացած էինք}& her-ɑ-t͡sʰ-e\'-el ej-i-ŋkʰ & \armenian{հեռացել էինք} \\
2PL & hr-\'æ-h-æ-t͡s ə-kʰ nel & \armenian{հռա̈՛հա̈ծ ըք նէլ} & heɾ-ɑ-t͡sʰ-\'ɑd͡z ej-i-kʰ & \armenian{հեռացած էիք} & her-ɑ-t͡sʰ-e\'-el ej-i-kʰ & \armenian{հեռացել էիք} \\
3PL &hr-\'æ-h-æ-t͡s ə-n nel & \armenian{հռա̈՛հա̈ծ ըն նէլ} & heɾ-ɑ-t͡sʰ-\'ɑd͡z ej-i-n & \armenian{հեռացած էին} & her-ɑ-t͡sʰ-e\'-el ej-i-n & \armenian{հեռացել էին} \\
& \multicolumn{2}{l|}{$\sqrt{}$-{\lvgloss}-{\aor}-{\thgloss}-{\rptcp} {\aux}-{\agr} {\pst}}& \multicolumn{2}{l|}{$\sqrt{}$-{\lvgloss}-{\aor}-{\rptcp} {\aux}-{\pst}-{\agr}}& \multicolumn{2}{l|}{$\sqrt{}$-{\lvgloss}-{\aor}-{\perfcvb} {\aux}-{\pst}-{\agr}}\\ 
\hline 
\end{tabular}
\end{table}

\translatorHD{For further illustration, Table \ref{tab:Agulis:morpho:verb:paradigm:resultativeperfective} shows the perfective and resultative non-finite forms across Agulis, SWA, and SEA and Agulis. }


\begin{table}[H]
 \centering
 \caption{Resultative participles and perfective converbs across Agulis, SWA, and SEA }
 \label{tab:Agulis:morpho:verb:paradigm:resultativeperfective}
 \resizebox{\textwidth}{!}{%
\begin{tabular}{|l|ll |ll |l l| }
\hline & \multicolumn{2}{l|}{Agulis} & \multicolumn{2}{l|}{cf. SWA}& \multicolumn{2}{l|}{cf. SEA} \\
& Perfective & Resultative & Perfective & Resultative & Perfective & Resultative \\\hline 
`to cut' & kətɾ-e-l & kətɾ-e-t͡s & N/A & ɡədɾ-ɑd͡z & kətɾ-el & kətɾ-ɑt͡s 
\\
& $\sqrt{}$-{\thgloss}-{\perfcvb}& $\sqrt{}$-{\thgloss}-{\rptcp}& & $\sqrt{}$-{\rptcp} & $\sqrt{}$--{\perfcvb}& $\sqrt{}$-{\rptcp}\\
& \armenian{կտրէլ} & \armenian{կտրէծ} & & \armenian{կտրած} & \armenian{կտրել} & \armenian{կտրած} \\
\hline 
`to say' & \'ɑh-ɑ-l & \'ɑh-ɑ-t͡s & N/A & əs-ɑd͡z & ɑs-el & ɑs-ɑt͡s 
\\
& $\sqrt{}$-{\thgloss}-{\perfcvb}& $\sqrt{}$-{\thgloss}-{\rptcp}& & $\sqrt{}$-{\rptcp}& $\sqrt{}$--{\perfcvb}& $\sqrt{}$-{\rptcp}\\
& \armenian{ա՛հալ} & \armenian{ա՛հած} & & \armenian{ըսած} & \armenian{ասել} & \armenian{ասած} \\
\hline 
`to go away' & hr-\'æ-h-æ-l & hr-\'æ-h-æ-t͡s & N/A & heɾ-ɑ-t͡sʰ-ɑd͡z & her-ɑ-t͡sʰ-el & her-ɑ-t͡sʰ-ɑt͡s 
\\
& $\sqrt{}$-{\lvgloss}-{\aor}-{\thgloss}-{\perfcvb}& $\sqrt{}$-{\thgloss}-{\lvgloss}-{\aor}-{\rptcp}& & $\sqrt{}$-{\lvgloss}-{\aor}-{\rptcp}& $\sqrt{}$--{\lvgloss}-{\aor}-{\perfcvb}& $\sqrt{}$-{\lvgloss}-{\aor}-{\rptcp}\\
& \armenian{հռա̈՛հա̈լ} & \armenian{հռա̈՛հա̈ծ} & & \armenian{հեռացած} & \armenian{հեռացել} & \armenian{հեռացած} \\
\hline 
Usage: & Perfective & Present perfect& & Present perfect& Present perfect & Not used for inflection
\\
\hline 
\end{tabular}
}
\end{table}


\paragraph{Future and future perfect} 

The future does not use the formatives /kə/ <\armenian{կը}> or /piti/ <\armenian{պիտի}>. It is formed by combining the infinitive with the inflected copular verb. 


\translatorHD{We elaborate in Table \ref{tab:Agulis:morpho:verb:paradigm:Futu}. In SEA, the future is formed synthetically. The particle /kə/ is placed before the verb. Tense and agreement is placed on the verb, after the theme vowel. In contrast, Agulis uses a periphrastic construction: the infinitive plus the inflected auxiliary. Note how the theme vowel of a verb like `to cut' alternates between /-e-/ in the resultative participle and perfective converb (Table \ref{tab:Agulis:morpho:verb:paradigm:resultativeperfective}, but uses /-i-/ in the infinitive. The verb `to say' uses the theme /-ɑ-/ in the previous two non-finite forms, but uses /-i-/ in the infinitive as well. But oddly, the verb `to go away' uses /-æ-/ in all three.}


\begin{table}[H]
 \centering
 \caption{Future <\armenian{ապառնի}> of the verb `to cut', `to say', and `to go away' in the Agulis dialect}
 \label{tab:Agulis:morpho:verb:paradigm:Futu}
 \begin{tabular}{|l|ll|ll| }
\hline & \multicolumn{2}{l|}{Agulis} & \multicolumn{2}{l|}{cf. SEA} \\\hline 
& `to cut' & & & \\
1SG & ktɾ-\'i-l ə-m & \armenian{կտրի՛լ ըմ} & kə kətɾ-\'e-m & \armenian{կկտրեմ} \\
2SG & ktɾ-\'i-l ə-s & \armenian{կտրի՛լ ըս} & kəkətɾ-\'e-s & \armenian{կկտրես} \\
3SG &ktɾ-\'i-l ɑ & \armenian{կտրի՛լ ա} &kə kətɾ-\'i-$\emptyset$ & \armenian{կկտրի} \\
1PL &ktɾ-\'i-l ə-kʰ & \armenian{կտրի՛լ ըք} & kə kətɾ-\'e-ŋkʰ & \armenian{կկտրենք} \\
2PL & ktɾ-\'i-l ə-kʰ & \armenian{կտրի՛լ ըք} &kə kətɾ-\'e-kʰ & \armenian{կկտրեք} \\
3PL &ktɾ-\'i-l ə-n & \armenian{կտրի՛լ ըն} & kə kətɾ-\'e-n & \armenian{կկտրեն} \\
& \multicolumn{2}{l|}{$\sqrt{}$-{\thgloss}-{\infgloss} {\aux}-{\agr} }& \multicolumn{2}{l|}{{\fut}-$\sqrt{}$-{\thgloss}-{\agr}}\\ 
\hline 
& `to say' & & & \\
1SG & n-\'ɑh-i-l ə-m & \armenian{նա՛հիլ ըմ} & k-ɑs-\'e-m & \armenian{կասեմ} \\
2SG & n-\'ɑh-i-l ə-s & \armenian{նա՛հիլ ըս} & k-ɑs-\'e-s & \armenian{կասես} \\
3SG &n-\'ɑh-i-l ɑ & \armenian{նա՛հիլ ա} & k-ɑs-\'i-$\emptyset$ & \armenian{կասի} \\
1PL &n-\'ɑh-i-l ə-kʰ & \armenian{նա՛հիլ ըք} & k-ɑs-\'e-ŋkʰ & \armenian{կասենք} \\
2PL & n-\'ɑh-i-l ə-kʰ & \armenian{նա՛հիլ ըք} & k-ɑs-\'e-kʰ & \armenian{կասեք} \\
3PL &n-\'ɑh-i-l ə-n & \armenian{նա՛հիլ ըն} & k-ɑs-\'e-n & \armenian{կասեն} \\
& \multicolumn{2}{l|}{?-$\sqrt{}$-{\thgloss}-{\infgloss} {\aux}-{\agr} }& \multicolumn{2}{l|}{{\fut}-$\sqrt{}$-{\thgloss}-{\agr}}\\ 
\hline 
& `to go away' & & & \\
1SG & hr-\'æ-n-æ-l ə-m & \armenian{հռա̈՛նա̈լ ըմ} & kə her-ɑ-n-\'ɑ-m & \armenian{կհեռանամ} \\
2SG & hr-\'æ-n-æ-l ə-s & \armenian{հռա̈՛նա̈լ ըս} & kə her-ɑ-n-\'ɑ-s & \armenian{կհեռացնաս} \\
3SG &hr-\'æ-n-æ-l ɑ & \armenian{հռա̈՛նա̈լ ա} &kə her-ɑ-n-\'ɑ-$\emptyset$ & \armenian{կհեռանա} \\
1PL &hr-\'æ-n-æ-l ə-kʰ & \armenian{հռա̈՛նա̈լ ըք} & kə her-ɑ-n-\'ɑ-ŋkʰ & \armenian{կհեռանանք} \\
2PL & hr-\'æ-n-æ-l ə-kʰ & \armenian{հռա̈՛նա̈լ ըք} & kə her-ɑ-n-\'ɑ-kʰ & \armenian{կհեռանաք} \\
3PL &hr-\'æ-n-æ-l ə-n & \armenian{հռա̈՛նա̈լ ըն} & kə her-ɑ-n-\'ɑ-n & \armenian{կհեռանան} \\
& \multicolumn{2}{l|}{$\sqrt{}$-{\lvgloss}-{\inch}-{\thgloss}-{\infgloss} {\aux}-{\agr} }& \multicolumn{2}{l|}{{\fut}-$\sqrt{}$-{\lvgloss}-{\inch}-{\thgloss}-{\agr}}\\ 
\hline 
\end{tabular}
\end{table}

\translatorHD{For the future perfect, SEA adds the tense suffix after the theme vowel, alongside the proper agreement morphs. In contrast, Agulis simply adds the past particle /nel/ after the future construction. Adjarian states the following.}


To form the past future, we must add the formative /nel/ <\armenian{նէլ}> on the above forms (Table \ref{tab:Agulis:morpho:verb:paradigm:FuturePerfect}). 

\translatorHD{He gives only a partial paradigm. }


\begin{table}[H]
 \centering
 \caption{Future perfect <\armenian{անցեալ ապառնի}>of the verb `to cut', `to say', and `to go away' in the Agulis dialect}
 \label{tab:Agulis:morpho:verb:paradigm:FuturePerfect}
 \begin{tabular}{|l|ll|ll| }
\hline & \multicolumn{2}{l|}{Agulis} & \multicolumn{2}{l|}{cf. SEA} \\\hline 
1SG `to cut'& ktɾ-\'i-l ə-m nel & \armenian{կտրի՛լ ըմ նէլ} & kə kətɾ-ej-\'i-$\emptyset$ & \armenian{կկտրեի} \\
& \multicolumn{2}{l|}{$\sqrt{}$-{\thgloss}-{\infgloss} {\aux}-1{\sg} {\pst} }& \multicolumn{2}{l|}{{\fut}-$\sqrt{}$-{\thgloss}-{\pst}-1{\sg}}\\ 
\hline 
2SG `to say'& n-\'ɑh-i-l ə-s nel & \armenian{նա՛հիլ ըս նէլ} & k-ɑs-ej-\'i-ɾ & \armenian{կասեիր} \\
& \multicolumn{2}{l|}{?-$\sqrt{}$-{\thgloss}-{\infgloss} {\aux}-2{\sg} {\pst} }& \multicolumn{2}{l|}{{\fut}-$\sqrt{}$-{\thgloss}-{\pst}-2{\sg}}\\ 
\hline 
3SG `to go away' &hr-\'æ-n-æ-l ɑ nel & \armenian{հռա̈՛նա̈լ ա նէլ} &kə her-ɑ-n-\'ɑ-$\emptyset$-ɾ & \armenian{կհեռանար} \\
& \multicolumn{2}{l|}{$\sqrt{}$-{\lvgloss}-{\inch}-{\thgloss}-{\infgloss} {\aux} {\pst} }& \multicolumn{2}{l|}{{\fut}-$\sqrt{}$-{\lvgloss}-{\inch}-{\thgloss}-{\pst}-3{\sg}}\\ 
\hline 
\end{tabular}
\end{table}


\paragraph{Imperative and prohibitive}

\translatorHD{For the imperative, Adjarian gives a list of formatives, but he is vague on their distribution. He states the following.}

The imperative is formed with the formatives /e/ <\armenian{է}>, /hi/ <\armenian{հի}>, /ɑ/ <\armenian{ա}> (Table \ref{tab:Agulis:morpho:verb:paradigm:Imp}).

\translatorHD{His prose is vague but it implies the following: the /-e/ is used for reflexes of the E-Class (verbs with the /-e-/ theme vowel), the /-hi/ is actually the aorist /-h/ plus imperative marker /-i/ that's used for inchoatives (verbs with the ending /-ɑ-n-ɑ-l/ in SEA), and /-ɑ/ is used for the A-Class (verbs with the /-ɑ-/ theme vowel). }

\begin{table}[H]
 \centering
 \caption{Imperative forms <\armenian{հրամայական}> for verbs in the Agulis dialect}
 \label{tab:Agulis:morpho:verb:paradigm:Imp}
 \begin{tabular}{|l|lll|lll|}
\hline & \multicolumn{3}{l|}{Agulis} & \multicolumn{3}{l|}{cf. SEA} \\
2SG of `to cut' & ktɾ-\'e-$\emptyset$ & $\sqrt{}$-{\thgloss}-{\imp}.2{\sg} & \armenian{կտրէ՛} & kətɾ-iɾ & \armenian{կտրի՛ր} & $\sqrt{}$-{\imp}.2{\sg} \\
2SG of `to forget' & mr-\'æ-h-i & $\sqrt{}$-{\lvgloss}-{\aor}-{\imp}.2{\sg}& \armenian{մռա̈՛հի} & mor-ɑ-t͡sʰ-iɾ &\armenian{մոռացի՛ր} & $\sqrt{}$-{\lvgloss}-{\aor}-{\imp}.2{\sg}\\
2SG of unclear & tʰ\'ɑk-ɑ-$\emptyset$ & $\sqrt{}$-{\thgloss}-{\imp}.2{\sg} & \armenian{թա՛կա} & & & $\sqrt{}$-{\thgloss}-{\imp}.2{\sg}\\
\hline \end{tabular}
\end{table}

\translatorHD{Unfortunately, he does not state how Agulis forms the imperative 2PL.}

\translatorHD{For SEA, the prohibitive is formed by just adding the particle /mi/ before the imperative form. In contrast, for Agulis, Adjarian states the following.}

Their prohibitive (\armenian{արգելական}) is formed by taking the infinitive and then adding the prohibitive formative /mæ/ <\armenian{մա}̈>. The final /l/ <\armenian{լ}> of the infinitive can also be deleted (Table \ref{tab:Agulis:morpho:verb:paradigm:Proh}). 

\translatorHD{Note how the infinitives here all share the same theme vowel /-i-/. }


\begin{table}[H]
 \centering
 \caption{Negative imperative or prohibitive forms for verbs in the Shamakhi dialect}
 \label{tab:Agulis:morpho:verb:paradigm:Proh}
 \begin{tabular}{|l|ll|ll|}
\hline & \multicolumn{2}{l|}{Agulis} & \multicolumn{2}{l|}{cf. SEA} \\
\hline 2SG of `to tie?' & k\'ɑb-i-l m\'æ & \armenian{կա՛պիլ մա}̈ & m\'i kɑb-iɾ & \armenian{մի՛ կապիր}\\ 
& k\'ɑb-i m\'æ & \armenian{կա՛պի մա}̈ & & \\ 
& \multicolumn{2}{l|}{$\sqrt{}$-{\thgloss}-{\infgloss} {\proh}} & \multicolumn{2}{l|}{{\proh} $\sqrt{}$-{\imp}.2{\sg}}\\
\hline 2SG of `to say' & n-\'ɑh-i-l m\'æ & \armenian{նա՛հիլ մա}̈ & m\'i ɑs-ɑ-$\emptyset$ & \armenian{մի՛ ասա} \\
& n-\'ɑh-i m\'æ & \armenian{նա՛հի մա}̈ & & \\ 
& \multicolumn{2}{l|}{?-$\sqrt{}$-{\thgloss}-{\infgloss} {\proh} } & \multicolumn{2}{l|}{{\proh} $\sqrt{}$-{\thgloss}-{\imp}.2{\sg}}\\
\hline 2SG of `to go away' & hr-\'æ-n-i-l m\'æ & \armenian{հռա̈՛նիլ մա}̈ & m\'i her-ɑ-t͡sʰ-iɾ & \armenian{մի՛ հեռացիր} \\
& hr-\'æ-n-i m\'æ & \armenian{հռա̈՛նի մա}̈ && \\ 
& \multicolumn{2}{l|}{$\sqrt{}$-{\lvgloss}-{\thgloss}-{\infgloss} {\proh} } &\multicolumn{2}{l|}{{\proh} $\sqrt{}$-{\lvgloss}-{\aor}-{\imp}.2{\sg} } \\
\hline \end{tabular}
\end{table}

 \paragraph{Subjunctive present and past imperfective } 

The subjunctive (\armenian{ստորադասական}) is formed similarly to the other dialects, but its past form is made with the formative /nel/ <\armenian{նէլ}>.

\translatorHD{What he means is that in SEA, the subjunctive present is formed by adding tense-agreement after the theme vowel. In Agulis, the present uses essentially the same morphological strategy (Table \ref{tab:Agulis:morpho:verb:paradigm:subjPresent}). The theme vowel is however a constant vowel /-i-/ across the different classes. }


\begin{table}[H]
 \centering
 \caption{Subjunctive present <\armenian{ստորադասական ներկայ}> of verbs in the Agulis dialect}
 \label{tab:Agulis:morpho:verb:paradigm:subjPresent}
 \begin{tabular}{|l|ll|ll| }
\hline & \multicolumn{2}{l|}{Agulis} & \multicolumn{2}{l|}{cf. SEA} \\\hline 
& `to cut' & & & \\
1SG & ktɾ-i-m & \armenian{կտրիմ} & kətɾ-e-m & \armenian{կտրեմ} \\
2SG & ktɾ-i-s & \armenian{կտրիս} & kə kətɾ-e-s & \armenian{կտրես} \\
3SG &ktɾ-i-$\emptyset$ & \armenian{կտրի} &kətɾ-i-$\emptyset$ & \armenian{կտրի} \\
1PL &ktɾ-i-kʰ & \armenian{կտրիք} & kətɾ-e-ŋkʰ & \armenian{կտրենք} \\
2PL & ktɾ-i-kʰ & \armenian{կտրիք} &kətɾ-e-kʰ & \armenian{կտրեք} \\
3PL &ktɾ-i-n & \armenian{կտրին} & kətɾ-e-n & \armenian{կտրեն} \\
& \multicolumn{2}{l|}{$\sqrt{}$-{\thgloss}-{\agr} }& \multicolumn{2}{l|}{$\sqrt{}$-{\thgloss}-{\agr}}\\ 
\hline 
& `to say' & & & \\
1SG & \'ɑh-i-m & \armenian{ա՛հիմ} & ɑs-\'e-m & \armenian{ասեմ} \\
2SG & \'ɑh-i-s & \armenian{ա՛հիս} & ɑs-\'e-s & \armenian{ասես} \\
3SG &\'ɑh-i-$\emptyset$ & \armenian{ա՛հի} & ɑs-\'i-$\emptyset$ & \armenian{ասի} \\
1PL &\'ɑh-i-kʰ & \armenian{ա՛հիք} & ɑs-\'e-ŋkʰ & \armenian{ասենք} \\
2PL & \'ɑh-i-kʰ & \armenian{ա՛հիք} & ɑs-\'e-kʰ & \armenian{ասեք} \\
3PL & \'ɑh-i-n & \armenian{ա՛հին} & ɑs-\'e-n & \armenian{ասեն} \\
& \multicolumn{2}{l|}{$\sqrt{}$-{\thgloss}-{\agr} }& \multicolumn{2}{l|}{$\sqrt{}$-{\thgloss}-{\agr}}\\ 
\hline 
& `to go away' & & & \\
1SG & hr-\'æ-n-i-m & \armenian{հռա̈՛նիմ} & her-ɑ-n-\'ɑ-m & \armenian{հեռանամ} \\
2SG & hr-\'æ-n-i-s & \armenian{հռա̈՛նիս} & her-ɑ-n-\'ɑ-s & \armenian{հեռացնաս} \\
3SG &hr-\'æ-n-i-$\emptyset$ & \armenian{հռա̈՛նի} & her-ɑ-n-\'ɑ-$\emptyset$ & \armenian{հեռանա} \\
1PL &hr-\'æ-n-i-kʰ & \armenian{հռա̈՛նք} & her-ɑ-n-\'ɑ-ŋkʰ & \armenian{հեռանանք} \\
2PL & hr-\'æ-n-i-kʰ & \armenian{հռա̈՛նք} & her-ɑ-n-\'ɑ-kʰ & \armenian{հեռանաք} \\
3PL &hr-\'æ-n-i-n & \armenian{հռա̈՛նին} & her-ɑ-n-\'ɑ-n & \armenian{հեռանան} \\
& \multicolumn{2}{l|}{$\sqrt{}$-{\lvgloss}-{\inch}-{\thgloss}-{\agr} }& \multicolumn{2}{l|}{$\sqrt{}$-{\lvgloss}-{\inch}-{\thgloss}-{\agr}}\\ 
\hline 
\end{tabular}
\end{table}

\begin{adjarianpage}\label{page:100}\end{adjarianpage}% should be 100


\translatorHD{For the subjunctive past imperfective, SEA places a tense suffix onto the verb. In contrast, Agulis adds the past marker /nel/ after the present construction (Table \ref{tab:Agulis:morpho:verb:paradigm:subjPast}).}


\begin{table}[H]
 \centering
 \caption{Subjunctive past <\armenian{ստորադասական անցեալ}> of verbs in the Agulis dialect}
 \label{tab:Agulis:morpho:verb:paradigm:subjPast}
 \begin{tabular}{|l|ll|ll| }
\hline & \multicolumn{2}{l|}{Agulis} & \multicolumn{2}{l|}{cf. SEA} \\\hline 
& `to cut' & & & \\
1SG & ktɾ-i-m nel & \armenian{կտրիմ նէլ} & kətɾ-ej-i-$\emptyset$ & \armenian{կտրեի} \\
2SG & ktɾ-i-s nel & \armenian{կտրիս նէլ} & kətɾ-ej-i-ɾ & \armenian{կտրեիր} \\
3SG &ktɾ-i-$\emptyset$ nel & \armenian{կտրի նէլ} &kətɾ-e-$\emptyset$-ɾ & \armenian{կտրեր} \\
1PL &ktɾ-i-kʰ nel & \armenian{կտրիք նէլ}& kətɾ-ej-i-ŋkʰ & \armenian{կտրեինք} \\
2PL & ktɾ-i-kʰ nel & \armenian{կտրիք նէլ} &kətɾ-ej-i-kʰ & \armenian{կտրեիք} \\
3PL &ktɾ-i-n nel & \armenian{կտրին նէլ} & kətɾ-ej-i-n & \armenian{կտրեին} \\
& \multicolumn{2}{l|}{$\sqrt{}$-{\thgloss}-{\agr} {\pst}}& \multicolumn{2}{l|}{$\sqrt{}$-{\thgloss}-{\pst}-{\agr}}\\ 
\hline 
& `to say' & & & \\
1SG & \'ɑh-i-m nel & \armenian{ա՛հիմ նէլ} & ɑs-ej-\'i-$\emptyset$ & \armenian{ասեի} \\
2SG & \'ɑh-i-s nel & \armenian{ա՛հիս նէլ} & ɑs-ej-\'i-ɾ & \armenian{ասեիր} \\
3SG &\'ɑh-i-$\emptyset$ nel & \armenian{ա՛հի նէլ} & ɑs-\'e-$\emptyset$-ɾ & \armenian{ասեր} \\
1PL &\'ɑh-i-kʰ nel & \armenian{ա՛հիք նէլ}& ɑs-ej-\'i-ŋkʰ & \armenian{ասեինք} \\
2PL & \'ɑh-i-kʰ nel & \armenian{ա՛հիք նէլ} & ɑs-ej-\'i-kʰ & \armenian{ասեիք} \\
3PL & \'ɑh-i-n nel & \armenian{ա՛հին նէլ} & ɑs-ej-\'i-n & \armenian{ասեին} \\
& \multicolumn{2}{l|}{$\sqrt{}$-{\thgloss}-{\agr} {\pst}}& \multicolumn{2}{l|}{$\sqrt{}$-{\thgloss}-{\pst}-{\agr}}\\ 
\hline 
& `to go away' & & & \\
1SG & hr-\'æ-n-i-m nel & \armenian{հռա̈՛նիմ նէլ} & her-ɑ-n-ɑj-\'i-$\emptyset$ & \armenian{հեռանայի} \\
2SG & hr-\'æ-n-i-s nel & \armenian{հռա̈՛նիս նէլ} & her-ɑ-n-ɑj-\'i-ɾ & \armenian{հեռացնայիր} \\
3SG &hr-\'æ-n-i-$\emptyset$ nel & \armenian{հռա̈՛նի նէլ} & her-ɑ-n-\'ɑ-$\emptyset$-ɾ & \armenian{հեռանար} \\
1PL &hr-\'æ-n-i-kʰ nel & \armenian{հռա̈՛նք նէլ} & her-ɑ-n-ɑj-\'i-ŋkʰ & \armenian{հեռանայինք} \\
2PL & hr-\'æ-n-i-kʰ nel & \armenian{հռա̈՛նք նէլ} & her-ɑ-n-ɑj-\'i-kʰ & \armenian{հեռանայիք} \\
3PL &hr-\'æ-n-i-n nel & \armenian{հռա̈՛նին նէլ} & her-ɑ-n-ɑj-\'i-n & \armenian{հեռանային} \\
& \multicolumn{2}{l|}{$\sqrt{}$-{\lvgloss}-{\inch}-{\thgloss}-{\agr} {\pst}}& \multicolumn{2}{l|}{$\sqrt{}$-{\lvgloss}-{\inch}-{\thgloss}-{\pst}-{\agr}}\\ 
\hline 
\end{tabular}
\end{table}

\section{Subdialects}

As a subdialect, we can consider the village of Çənnəb. This holds a middle ground between the dialects of Agulis and Karabakh, and it leans towards both. Its consonant system is entirely the same as the Karabakh dialect. Here as well, the voiced sounds became voiceless unaspirated. 

\section{Literature}

For the Agulis dialect, there have been three studies. The first was the work of Kerovbe Patkanian (\armenian{Պատկանեան}) in German \citep{Patkanoff-1866-Agulis}. The second is by the same author \citep[27-55]{Patkanian-1869-RussianDialects}. The third is \citet{Sargiants-1883-Agulis}. This is the most extensive and unique work because the author is a native from Agulis. 

There are few pieces that are written in the Agulis dialect. I am familiar with only the following. 

{\litoverview}

\begin{itemize}
 \item Literature with the Agulis dialect
 \begin{itemize}
 \item \armenian{Զարգութեանց Աւետիս – Գիւլը նէան դուռնան. Արարատ}, 1877, \armenian{էջ} 461–462
\item \armenian{Շահնաբաթեան Մարտ – Երգ ագուլեցոց (կէս գրական լեզուով}). \armenian{Կռունկ}, 1862, \armenian{էջ} 163–166
\item \armenian{Ս. Սարգիսեանց – Ագուլեցոց բարբառը, Բ մաս. էջ} 5-72

 \end{itemize}
\end{itemize}

From the last extensive collection, we extract the following samples (page 39-42).



\begin{adjarianpage}\label{page:101}\end{adjarianpage}% should be 101
\section{Text samples}

{\sampleoverview}

\subsection{Sample 1}

\armenian{Մույն օր մույն Ըգյըլա̈՛ցի ձիը է՛լած՝ դյէ՛լիս ա Ցա՛ղնա. ձիը քշում ա նէ՛լիս դիւզ կըրմընջո՛ւմը կա՛յնամ. տի՛սնա̈մ ա ՝ աստէղ ըսկի միծ մարդիքյ չի կօն՝ մա̈՛նա̈կյ մույն քընա ըրխաք ըն հըվա̈՛քվա̈ծ հաղ նա՛րամ։ Ըգյըլա̈՛ցին ձիո՛ւն վա̈՛րա̈ն հպարտ հպարտ նըստէծ՝ ձա̈րքա̈րը դրէծ չընըքտա՛կին՝ հրցանամ ա թա՝}

– \armenian{Ա՜յ տղարք, բա̈ս ձիր ախսախկալնէրը ըշտէ՞ղ ըն։}

\armenian{Տղարքը ջղօբ ըն տա՛լիս թա̈՝}

– \armenian{Նա̈՛հա̈ծ ըն էքին հա̈՛չա̈ պէրին տօն չընըքտակիտ։}

\armenian{Ըգյըլա̈՛ցին մանդրվամ ա. համա հրդէօ ինքյն ա նի՛զամ ա̈նօղ տօ Ցղնա̈՛ցա̈ց, նա՛համ ա}.

– \armenian{Ա՛ղո՞ւրդ ա օր ա՛ստէղ մաշկ ըն մա՛նդրամ։}

– \armenian{Օշկըտ ալ չին հա՛նո՜ւմ։}

\armenian{Ըգյելա̈՛ցին մայտք ա նա՛րամ՝ թա̈ աստէղ օր ըրխաքը աստի բա̈՛զզա̈թ (թրք. խորամանկ) ըն, բա̈ս սրուց միծարը ո՛ւնչպէս ըն նիլ}. – \armenian{քշում ա ձիը, յունց կալիս նէլիս։}

\subsection{Sample 2}

\armenian{Ի՛րիքյ սա̈րիւն սկա̈՛հա̈ծ մարդիքյ՝ ձիո՛ւ վա՛րտ նստէծ՝ նէլիս ըն նէլ։ Ճընըփի ռաստ ըն գյէ՛լիս մույն գյէղըցո՛ւ. նի՛զամ ըն սրօ անօզ տօն. նրուց մույնը նա՛համ ա։}

– \armenian{Իս հաստադիլ ըմ օր դիւ ջա՛մուշ ըս։}

– \armenian{Աս զարմանա՛լի չի}, – \armenian{նա՛հա̈մ ա գյէղա̈՛ցին, իս հաստա՛դիլ ըմ օր ձիր թամքարը ջէ՛րիքյ ըն։}

– \armenian{Հի՞մա̈ր ըս ունչ ա. դէ հաստա՛դա տիսնիքյ։}

– \armenian{Իս շօտ անքամ լսէծ ըմ օր՝ ունչ օր կօ ձիո՛ւ ընա̈ն էշի մաջտէղը՝ նօ ջէ՛րի ա։}
\armenian{Ալ ան մարդիքյը վօ՛չինչ չին գըրա̈՛հա̈լ ա՛հին, հռա̈՛հա̈ն նա̈՛հա̈լ։}

\subsection{Sample 3}

\armenian{Մույն օր մույն Ռըմա̈՛ցի նէ՛լիս ա օ՛րտը, տի՛սնա̈մ ա մույն ձի մտէծ ա̈ մաջը ըրըծա̈՛հա̈ծ, մա մաջտէղա̈մն ալ վէր ընգէծ ստա՛կած։ Նէ՛լիս ա մույն քա՛նի մա՛րդիքյ հըվա̈՛քա̈մ բի՛րա̈մ օր արտին մաջիցը քաշին ձիո՛ւն ջա̈՛մդա̈քը հանին։ Մըտըկա՛նամ ըն}, ... 


\begin{adjarianpage}\label{page:102}\end{adjarianpage}% should be 102

\armenian{քա̈նդիրը կապամ ստակած ձիո՛ւ ատնէրիցը, մայտք նա՛րամ՝ դիբի ուր յան քաշին օր վէօրը ջարդվի ուչ խարաբ էլի։}

– \armenian{Ա՛կիքյ դի՛բի աս յան քա՛շիքյ, նա՛համ ա մույնը։ Քա՛շամ ըն, քա՛շամ, տի՛սնա̈մ ըն՝ չէ, վէօրը խիլի (թրք. շատ) խարաբ ա է՛լա̈լ։}

– \armenian{Չէ, ա՛սմաս է՛լա̈լ չի. ա՛կիքյ դի՛բի աս յան քա՛շիքյ}. – \armenian{նա՛համ ա մույն ո՛ւրիշը։ Սկսում ըն քաշին թօզա (թրք. թազա՝ նոր) ճընըփօվ. տի՛սնա̈մ ըն՝ չէ. ալ վէօրը խիլի տէղ ջարդան խարօբ արալ։}

– \armenian{Ա՛ստի ալ չի է՛լա̈լ – նա՛համ ա իրիքիմջի՛ն – ա՛կիքյ հրէս աս ղրա՛ղավ (թրք. եզերք) քա՛շիքյ։ Քա՛շամ ըն, յիտ մտակյ նա՛րամ՝ ալ վէօրը ջա՛րդած ըն խա՛րօբ ա՛րած։ Ալ սկսում ըն մույն ո՛ւրիշ տէ՛ղավ քա՛շամ։ Ա՛նքքամ դէս ու դին ըն քշպատամ ջա̈՛մդա̈քը արտին մաջին, մի՛նչէվ օր կուհ տա՛լիս, ջա՛րդամ, խա՛րօբ նա՛րամ դիփ օ՛րտը։}

\subsection{Sample 4}

\armenian{Ա̈րկու ճանապա՛րհօրթ մույն հօվ օ՛նին նէլ ըտէ՛լու։ Մույնը ա՛հալ ա մյո՛ւսին։}

– \armenian{Հօկ հօվը միզ հրաքյ չի. թուղ ուխ սա̈՛րիւն ա՛րազ ա տի՛սնիլ, նօ օ՛տի։ Աս ա՛հուղը քիւն ա̈ է՛լա̈լ. մյուսը կա̈՛րա̈լ ա̈ հօվը։ Առաչինը ըրթնա՛հալ ա, սկսէլ պա՛տմի}.


– \armenian{Զարմանա՛լի ա՛րազ ըմ տա̈՛հա̈լ. ա̈նձ հրէշտա̈կնէրը տա՛նամ ըն նէլ յէրգյինքը։}
\armenian{Հա՛վավ կշտա՛հածը ա՛հալ ա։}

– \armenian{Իս ալ տա̈՛հա̈մ օր դիւ բըձրա̈՛նա̈մ ըս, ինքյըս ա̈մ մա՛ջիս ա՛համ թա̈ զօ յիտ չի դա՛ռնիլ, կա̈՛րա̈մ հօվը։}




\part{The /kə/ <\armenian{կը}> branch}
\begin{adjarianpage}\label{page:103}\end{adjarianpage}% should be 103

The /kə/ <\armenian{կը}> branch has 21 dialects:

\begin{enumerate}
\item Dialect of Karin 
\item Dialect of Mush
\item Dialect of Van 
\item Dialect of Tigranakert 
\item Dialect of Kharberd and Yerznka 
\item Dialect of Şebinkarahisar 
\item Dialect of Trabzon 
\item Dialect of Hamshen
\item Dialect of Malatya 
\item Dialect of Cilicia 
\item Dialect of Syria
\item Dialect of Arapgir 
\item Dialect of Akn 
\item Dialect of Sebastia 
\item Dialect of Evdokia 
\item Dialect of Smyrna 
\item Dialect of Nicomedia 
\item Dialect of Istanbul 
\item Dialect of Rodosto 
\item Dialect of Crimea 
\item Dialect of Austro-Hungary

 
\end{enumerate}

\chapter{Karin}\label{chapter:Karin}
\section{Background}
\begin{adjarianpage}\label{page:104}\end{adjarianpage}% should be 104


The center of this widely-spread dialect is Karin (Turkish: Erzurum). From the south, it spreads until near Hınıs, but without entering this small town (\armenian{աւան}). From the west, it reaches until Yerznka and Gümüşhane. During the last two Russo-Turkish wars, large migrant communities spread from the eastern and northern borders of this dialect to very far places, until Yerevan and Tbilisi. Four cities of the Caucasus (Kars, Alexandropol, Akhalkalak, and Akhaltskha) were filled with these same migrants, and now the entire Armenian population of those cities speaks the same dialect as the Armenians of Karin. 

\section{Phonology}

\subsection{Segment inventory}

\subsubsection{Vowels}
When we compare the phonetic system of this dialect against Old Armenian, we see that the vowels have been preserved almost unchanged. This dialect knows how to distinguish the sounds /i̯e/ <\armenian{ե}> vs. /e/ <\armenian{է}>, and /u̯o/ <\armenian{ո}> vs. /o/ <\armenian{օ}>. The vowel /æ/ <\armenian{ա}̈ > is included. The vowels /œ, ʏ/ <\armenian{էօ, իւ}> are found in those words that are taken from Turkish; they do not exist at all in native Armenian words. Meanwhile in other dialects, such as Karabakh, Agulis, and even Istanbul, these vowels are found even in native words because of natural sound changes.

\paragraph{Vowel /æ/ <\armenian{ա}̈ >}
The sound /æ/ <\armenian{ա}̈ > in Karin is also foreign, and it is found primarily in loanwords from Turkish. But there are also some Armenian words where this sound has entered, whether because of Turkish influence or because of independent sound changes (Table \ref{tab:Karin:phono:segment:vowel:ae}).

\begin{table}[H]
 \centering
 \caption{Presence of the vowel /æ/ <\armenian{ա}̈ > in the Karin dialect}
 \label{tab:Karin:phono:segment:vowel:ae}
 \begin{tabular}{|l| ll|ll| ll|}
 \hline & \multicolumn{2}{l|}{Classical Armenian} &\multicolumn{2}{l|}{> Karin} & \multicolumn{2}{l|}{cf. SEA} \\ 
`sugar' & ʃɑkʰɑɾ & \armenian{շաքար}& ʃækʰæɾ & \armenian{շա̈քա̈ր} & ʃɑkʰɑɾ & \armenian{շաքար} \\ 
`beam' & mɑɾdɑk & \armenian{մարդակ}& mæɾtʰæk & \armenian{մա̈րթա̈կ} & mɑɾtʰɑkn & \armenian{մարդակ} \\ 
`marble' & mɑɾmɑɾ & \armenian{մարմար}& mæɾmæɾ & \armenian{մա̈րմա̈ր} & mɑɾmɑɾ & \armenian{մարմար} \\ 
`to bleat' & mɑjel & \armenian{մայել}& mæjel & \armenian{մա̈յէլ} & mɑjel & \armenian{մայել} \\ 
`Sunday' & kiɾɑkē & \armenian{կիրակէ}& kiɾæki & \armenian{կիրա̈կի} & kiɾɑki & \armenian{կիրակի} \\ 
\hline 
 \end{tabular}
\end{table}

 
The first three... 




\begin{adjarianpage}\label{page:105}\end{adjarianpage}% should be 105


... are also used in Turkish, and the influence of Turkish is probable. But the latter three words are native Armenian words.

\paragraph{Diphthongs /u̯o, i̯e/ <\armenian{ո, ե}>}

The sounds <\armenian{ո, ե}> have a diphthongal pronunciation /u̯o, i̯e/ <\armenian{ուօ, իէ}>, and they originate from the Classical Armenian mid vowels /o, e/ <\armenian{ո, ե}>; they are found only in the language of villages. Urban speakers do not have these sounds. As for migrants of the Caucasus, those people who have a rural origin likewise pronounce the reflexes of Classical /o, ō/ <\armenian{ո, օ}> with a certain pronunciation; while those who are urban folk do not use these sounds. 

\subsubsection{Consonants}

\paragraph{Origin of the fricative /f/ <\armenian{ֆ}>}
For the consonants, let us first talk about the sound /f/ <\armenian{ֆ}>. 

The sound /f/ <\armenian{ֆ}> has two origins. First, it is found in foreign words that are borrowed from Turkish. Second, it developed in Armenian via natural sound changes. This latter origin also has two routes. 

\subparagraph{First route for origin of /f/}

Word-initially, the Classical sound /h/ <\armenian{հ}> becomes /f/ <\armenian{ֆ}> before Classical /o/ <\armenian{ո}> (\translatorHD{which became /uo̯/}) (Table \ref{tab:Karin:phono:segment:cons:f:1}).\footnote{\translatorHD{For the word `here', Adjarian provides an ancestor form <\armenian{հոս}>, but this form is not clearly attested in Classical Armenian. I treat it as a reconstruction. For `article', Adjarian provides an ancestor <\armenian{հոդ}>, but such a form does not exist but is instead <\armenian{յոդ}>. }}
 

\begin{table}[H]
 \centering
 \caption{Origin of /f/ <\armenian{ֆ}> from word-initial /h/ <\armenian{հ}> in the Karin dialect}
 \label{tab:Karin:phono:segment:cons:f:1}
 \begin{tabular}{|l| ll|ll| ll|}
 \hline & \multicolumn{2}{l|}{Classical Armenian} &\multicolumn{2}{l|}{> Karin} & \multicolumn{2}{l|}{cf. SEA} \\ 
`earth' &hoɬ & \armenian{հող} & fu̯oʁ & \armenian{ֆող} & hoʁ & \armenian{հող} \\
`smell' & hot & \armenian{հոտ} & fu̯ot & \armenian{ֆոտ} & hot & \armenian{հոտ} \\ 
`hole (CA); pit (SEA)' &hu̯oɾ & \armenian{հոր} & foɾ & \armenian{ֆոր} & hoɾ & \armenian{հոր} \\ 
`here' &*hos & *\armenian{հոս} & fu̯os & \armenian{ֆոս} & hos & \armenian{հոս} \\
`article' & jɑu̯d & \armenian{յաւդ} & fu̯od & \armenian{ֆոդ} & hod & \armenian{հոդ} \\ 
`there' & hon & \armenian{հոն} & fu̯on & \armenian{ֆոն} & hon & \armenian{հոն} \\
\hline 
 \end{tabular}
\end{table}

However, next to Classical /ɑu̯/ <\armenian{աւ}> (\translatorHD{which became /ō/ <\armenian{օ}>}, this change does not happen (Table \ref{tab:Karin:phono:segment:cons:f:1h}). 


\begin{table}[H]
 \centering
 \caption{Words with word-initial /h/ <\armenian{հ}> in the Karin dialect}
 \label{tab:Karin:phono:segment:cons:f:1h}
 \begin{tabular}{|l| ll|ll| ll|}
 \hline & \multicolumn{2}{l|}{Classical Armenian} &\multicolumn{2}{l|}{> Karin} & \multicolumn{2}{l|}{cf. SEA} \\ 
`father.{\gen}' &hɑu̯ɾ & \armenian{հաւր} & hoɾ & \armenian{հօր} & hoɾ & \armenian{հոր} \\ \hline 
 \end{tabular}
\end{table}

 
It is notable that this sound change is specific to the rural language. The urban sound /h/ <\armenian{հ}> sound stays unchanged, and the reason for this is as follows. As can be seen above, the origin of the sound /f/ <\armenian{ֆ}> is the diphthongal pronunciation /u̯o/ <\armenian{ուօ}> of the reflex of Classical /o/ <\armenian{ո}>, because no such change occurs next to the reflex of Classical /ɑu̯/ <\armenian{աւ}> (\translatorHD{also written as /ō/ <\armenian{օ}>}). Now, because urban folk don't have the sound /u̯o/ <\armenian{ո}> and pronounce it as just /o/ <\armenian{օ}>, then naturally they do not have this type of /f/ <\armenian{ֆ}>.

\subparagraph{Second route for origin of /f/}

The second route for the origin of the sound /f/ <\armenian{ֆ}> is the sound /v/ <\armenian{վ}>, which gets devoiced to /f/ <\armenian{ֆ}> (Table \ref{tab:Karin:phono:segment:cons:f:2}).

 

\begin{table}[H]
 \centering
 \caption{Origin of /f/ <\armenian{ֆ}> from devoiced /v/ <\armenian{վ}> in the Karin dialect}
 \label{tab:Karin:phono:segment:cons:f:2}
 \begin{tabular}{|l| ll|ll| ll|}
 \hline & \multicolumn{2}{l|}{Classical Armenian} &\multicolumn{2}{l|}{> Karin} & \multicolumn{2}{l|}{cf. SEA} \\ 
`equal' &hɑ{wɑ}sɑɾ & \armenian{հաւասար} & hɑfsɑɾ & \armenian{հաֆսար} & hɑvɑsɑɾ & \armenian{հավասար} \\
`to be gathered' & *hɑ{wɑ}kʰil & *\armenian{հաւաքիլ} & hɑfkʰil & \armenian{հաֆքիլ} & hɑvɑkʰvel & \armenian{հավաքվել} \\
`to suffice (CA); to roast (SEA)' & bovel & \armenian{բովել} & bʰoɾfel & \armenian{բՙօրֆէլ} & bovel & \armenian{բովել} \\
`south' & hɑɾɑu̯ & \armenian{հարաւ} & hɑɾɑf & \armenian{հարաֆ} & hɑɾɑv & \armenian{հարավ} \\
`to mew' & nu̯ɑl & \armenian{նուալ} & nfɑl & \armenian{նֆալ} & nəvɑl & \armenian{նվալ} \\
\hline 
 \end{tabular}
\end{table}

\paragraph{Consonant voicing}

In the consonant series, the Karin dialect has undergo a huge innovation, just as the Mush dialect has. 

We know that Old Armenian distinguished three groups of consonants. The Karin dialect has added a fourth series, entirely different from the others, which we called the voiced aspirates (Armenian: \textit{\armenian{թրթռուն շնշաւոր}}, French: \textit{sonore aspiree}). We represent them as /bʰ, ɡʰ, dʰ, d͡zʰ, d͡ʒʰ/ < \armenian{բՙ, գՙ, դՙ, ձՙ, ջՙ}> . Among the European phoneticians, Sievers was the first to notice the existence of voiced experiences... 



\begin{adjarianpage}\label{page:106}\end{adjarianpage}% should be 106

... in the pronunciation of Ashtarak (a Yerevan dialect),\footnote{\translatorHD{The prose is unclear, but I suspect he means \citet{sievers-1901-grundzuge}[436, 442], based on a similar citation by \citet[438]{Schirru-2012-LaryngealFeatureArmenianDialect}. I couldn't verify this however. }} but no person focused in depth on these sounds. And the existence of four degrees of consonants in Armenia is a novelty. For the first time, I had the opportunity to use an experimental method to study these same sounds in Paris, using the phonetic machines (\armenian{ձայնագիտական մեքենաներ}) of Abbé Rousselot (Jean-Pierre Rousselot, Armenian: \armenian{Աբբա Ռուսլօ}), with young people from Mush, Sebastia, and others. The results of this study were published in a small work, where they present the four degrees of plosive letters in Armenian according to the pronunciation of six vernaculars (Istanbul, Aslanbeg, Nukha, Shushi, Sebastia, and Mush), summarized in four phototype images (\armenian{լուսատիպ պատկերներ}). See \citet{Adjarian-1899-ArmenianExplosives}. 

I ascertained the existence of four degrees of consonants a year later in my study of the Suceava dialect (see \citeauthor{Bazmaveb}, 1899, page 219-220). Because we consider it excessive to further talk about this matter, we refer readers to the study. In passing, we only state that the pronunciation of the voiced aspirate consonants is close to the sounds /bh, ɡh, dh, d͡zh, d͡ʒh/ <\armenian{բհ, գհ, դհ, ձհ, ջհ}>, and these sounds are similar in manner to the Sanskrit consonants <bh, gh, dh, jh>. 

And thus we see a general picture of the plosive consonants of the Karin dialect (Table \ref{tab:Karin:phono:segment:cons:voice:voiceaspriate}).

\begin{table}[H]
\caption{Voiced aspirates in the Karin dialect}\label{tab:Karin:phono:segment:cons:voice:voiceaspriate}\centering 
\begin{tabular}{|l|ll|ll|ll|ll|}
\hline & \multicolumn{2}{l|}{Voiced} & \multicolumn{2}{l|}{Voiced aspirate} & \multicolumn{2}{l|}{Voiceless} & \multicolumn{2}{l|}{Voiceless aspirate} \\
Armenian name & \multicolumn{2}{l|}{\armenian{թրթռուն}} & \multicolumn{2}{l|}{\armenian{թրթռուն շնչաւոր}} & \multicolumn{2}{l|}{\armenian{խուլ}} & \multicolumn{2}{l|}{\armenian{խուլ շնչաւոր}} \\
French name &\multicolumn{2}{l|}{sonore} & \multicolumn{2}{l|}{sonore asp.} & \multicolumn{2}{l|}{sourde} & \multicolumn{2}{l|}{sourde asp.} \\
\hline & b &\armenian{բ} & bʰ& \armenian{բՙ} & p & \armenian{պ} & pʰ& \armenian{փ} \\
& ɡ &\armenian{գ} & ɡʰ & \armenian{գՙ} & k & \armenian{կ} & kʰ & \armenian{ք} \\
&d& \armenian{դ} & dʰ & \armenian{դՙ} & t & \armenian{տ} & tʰ & \armenian{թ} \\
&d͡z& \armenian{ձ} & d͡zʰ& \armenian{ձՙ} & t͡s & \armenian{ծ} & t͡sʰ & \armenian{ց} \\
& d͡ʒ &\armenian{ջ} & d͡ʒʰ & \armenian{ջՙ} & t͡ʃ & \armenian{ճ} & t͡ʃʰ & \armenian{չ} 
\\ \hline 
\end{tabular}
\end{table}

\paragraph{Voiced glottal fricative /ɦ/ <\armenian{յ}̵>voicing}

In the Karin dialect, the reflex of the Classical sound /j/ <\armenian{յ}> has a pronunciation similar to the voiced aspirated; this sound is also found in the Mush dialect, and we represent it as /ɦ/ <\armenian{յ}̵>. This sound is found against the Old Armenian sound /j/ <\armenian{յ}> (Table \ref{tab:Karin:phono:segment:cons:hh}).




\begin{table}[H]
 \centering
 \caption{Voiced glottal fricative /ɦ/ <\armenian{յ}̵> in the Karin dialect}
 \label{tab:Karin:phono:segment:cons:hh}
 \begin{tabular}{|l| ll|ll| ll|}
 \hline & \multicolumn{2}{l|}{Classical Armenian} &\multicolumn{2}{l|}{> Karin} & \multicolumn{2}{l|}{cf. SEA} \\ 
given name &jɑɾutʰiu̯n & \armenian{Յարութիւն} & ɦɑɾutʰ & \armenian{յ}̵\armenian{արութ} & hɑɾutʰjun & \armenian{Հարություն} \\
given name &jɑkob & \armenian{Յակոբ} & ɦɑko & \armenian{յ}̵\armenian{ակօ} & hɑkopʰ & \armenian{Հակոբ} \\
\hline 
 \end{tabular}
\end{table}


With this, the dialect has two types of glottal (\armenian{հագագային}) sounds: /ɦ, h/ <\armenian{յ}̵, \armenian{հ}>. 


\begin{adjarianpage}\label{page:107}\end{adjarianpage}% should be 107

\subsection{Sound changes}

The Karin dialect is not very rich in sound changes; and after indicating the above cases, few things remain. 

\subsubsection{Monophthongal vowel changes}
\paragraph{Vowel syncope of Classical Armenian /ɑ/ <\armenian{ա}>}
As general rule for all dialects in the /kə/ <\armenian{կը}> branch, in polysyllabic words, the reflex of the Classical vowel /ɑ/ <\armenian{ա}> of a word-medial syllable is deleted or changes to /ə/ <\armenian{ը}> (Table \ref{tab:Karin:phono:segment:cons:f:2}). We don't return to this general rule elsewhere. 

 

\begin{table}[H]
 \centering
 \caption{Medial vowel syncope in various Western dialects (Karin, Istanbul)}
 \label{tab:Karin:phono:change:vowel:syncope}
 \resizebox{\textwidth}{!}{% 
 \begin{tabular}{|l| ll|ll|ll|ll|ll|}
 \hline & \multicolumn{2}{l|}{Classical Armenian} &\multicolumn{2}{l|}{> Karin} & \multicolumn{2}{l|}{cf. Istanbul} & \multicolumn{2}{l|}{cf. SWA} & \multicolumn{2}{l|}{cf. SEA} \\ 
 `to recognize' & t͡ʃɑnɑt͡ʃʰel & \armenian{ճանաչել} & t͡ʃɑnt͡ʃʰel & \armenian{ճանչել} &t͡ʃɑʃnɑl & \armenian{ճաշնալ} & d͡ʒɑnt͡ʃʰnɑl & \armenian{ճանչնալ} & t͡ʃɑnɑt͡ʃʰel & \armenian{ճանաչել} \\
`sickly' & hi{wɑ}ndot & \armenian{հիւանդոտ} & hivəndot& \armenian{հիվընդոտ} & hivɑndod, & \armenian{հիվանդօդ}, & hivɑntʰod & \armenian{հիւանդոտ}& hivɑndot & \armenian{հիվանդոտ} \\ 
& & & & & hivəndod & \armenian{հիվընդօդ} & & & & \\ 
`mouth-{\gen}' & beɾɑn-i & \armenian{բերանի} & bʰeɾn-i &\armenian{բՙէրնի} & beɾn-i& \armenian{բէրնի} &pʰeɾn-i & \armenian{բերնի} &beɾɑn-i & \armenian{բերանի} \\
\hline 
 \end{tabular}
 }
\end{table}

\paragraph{Classical Armenian /e/ <\armenian{ե}>}


At the beginning of monosyllabic words (Table \ref{tab:Karin:phono:change:vowel:e}a), the Classical sound /e/ <\armenian{ե}> has turned to /je/ <\armenian{յէ}> or /ji̯e/ <\armenian{յե}> (the latter for villagers). At the beginning of polysyllabic words, the sound is /e/ <\armenian{է}> (Table \ref{tab:Karin:phono:change:vowel:e}b). And word-medially, it is /e/ <\armenian{է}> or /i̯e/ <\armenian{ե}> (Table \ref{tab:Karin:phono:change:vowel:e}c). 




\begin{table}[H]
 \centering
 \caption{Change from Classical Armenian /e/ <\armenian{ե}> to /je, ji̯e, e, i̯e/ <\armenian{յէ, յե, է, ե}> in the Karin dialect}
 \label{tab:Karin:phono:change:vowel:e}
 \begin{tabular}{|ll| ll|ll| ll|}
 \hline & & \multicolumn{2}{l|}{Classical Armenian} &\multicolumn{2}{l|}{> Karin} & \multicolumn{2}{l|}{cf. SEA} \\ 
 a. & \armenian{՝}ox' & ezən & \armenian{եզն} & je & \armenian{յէզ} & jez & \armenian{եզ} \\
 & \armenian{՝}boiling (CA); tingling (SEA)' & er-kʰ (-{\pl}) & \armenian{եռք} & jerkʰ & \armenian{յէռք} & jer-kʰ & \armenian{եռք} \\
&`I' & es & \armenian{ես} & jes & \armenian{յէս} &jes & \armenian{ես} \\
 & \armenian{՝}when' & eɾb & \armenian{երբ} & jepʰ & \armenian{յէփ} & jeɾpʰ & \armenian{երբ} \\
b. & `to cook' & epʰel & \armenian{եփել} & epʰel & \armenian{էփէլ} & jepʰel & \armenian{եփել} \\
& `dream' & eɾɑz & \armenian{երազ} & eɾɑz & \armenian{էրազ} & jeɾɑz & \armenian{երազ} \\
c. & \armenian{՝}to bring' & beɾel & \armenian{բերել} & bʰeɾel & \armenian{բՙէրէլ} & beɾel & \armenian{բերել} \\
& `big' &met͡s & \armenian{մեծ} & met͡s (urban) & \armenian{մէծ} &met͡s & \armenian{մեծ} \\
& & & & mi̯et͡s (villager) & \armenian{մեծ} & & \\
\hline 
 \end{tabular}
\end{table}


\paragraph{Classical Armenian /o/ <\armenian{ո}>}


 At the beginning of monosyllabic words, the Classical sound /o/ <\armenian{ո}> (Table \ref{tab:Karin:phono:change:vowel:o}) has changed to /vo/ <\armenian{վօ}>, /o/ <\armenian{օ}>, or /vu̯o/ <\armenian{վո}>; at the beginning of polysyllabic words to /o/ <\armenian{օ}>; and word-medially to /o/ <\armenian{օ}> or /u̯o/ <\armenian{ո}> (the forms /vu̯o, u̯o/ and <\armenian{վո, ո}> are rural). The word for `who' has a typical form /vev/ <\armenian{վէվ}>. 




\begin{table}[H]
 \centering
 \caption{Change from Classical Armenian /o/ <\armenian{ո}> to /vo, o, vu̯o, u̯o, ve/ <\armenian{վօ, օ, վո, ո, վէ}> in the Karin dialect}
 \label{tab:Karin:phono:change:vowel:o}
 \begin{tabular}{| l| ll|ll| ll|}
 \hline & \multicolumn{2}{l|}{Classical Armenian} &\multicolumn{2}{l|}{> Karin} & \multicolumn{2}{l|}{cf. SEA} \\ 
 \armenian{՝}that' & oɾ & \armenian{որ} & voɾ & \armenian{վօր} & voɾtʰi & \armenian{որ} \\
 \armenian{՝}to take pity on' & oɬoɾmil & \armenian{ողորմիլ} & oʁoɾmil & \armenian{օղօրմիլ} & voʁoɾmel & \armenian{ողորմել} \\
 \armenian{՝}to ruminate' & oɾot͡ʃɑl & \armenian{որոճալ} & oɾot͡ʃɑl & \armenian{օրօճալ} & voɾot͡ʃɑl & \armenian{որոճալ} \\
`who' & ov & \armenian{ով} & vev & \armenian{վէվ} & ov & \armenian{ով} \\ 
\hline 
 \end{tabular}
\end{table}

\subsubsection{Diphthong changes}

\paragraph{Classical Armenian /ɑi̯/ <\armenian{այ}>}


 
The Classical Armenian diphthong /ɑi̯/ <\armenian{այ}> has changed to /ɑ/ <\armenian{ա}> or city people, and /e/ <\armenian{է}> for villagers. For settlements in the Caucasus, Akhaltskha has the form /ɑ/ <\armenian{ա}>, while Alexandropol has the form /e/ <\armenian{է}> (Table \ref{tab:Karin:phono:change:diphth:aj}). 



\begin{table}[H]
 \centering
 \caption{Change from Classical Armenian /ɑi̯/ <\armenian{այ}> to /ɑ, e/ <\armenian{ա, է}> in the Karin dialect}
 \label{tab:Karin:phono:change:diphth:aj}
 \begin{tabular}{| l| ll|ll| ll|}
 \hline & \multicolumn{2}{l|}{Classical Armenian} &\multicolumn{2}{l|}{> Karin} & \multicolumn{2}{l|}{cf. SEA} \\ 
`father' & hɑi̯ɾ & \armenian{հայր} & hɑɾ, heɾ & \armenian{հար, հէր} & hɑjɾ & \armenian{հայր} \\ 
`wood' & pʰɑi̯t & \armenian{փայտ} & pʰɑt, pʰet & \armenian{փատ, փէտ} &pʰɑjt & \armenian{փայտ} \\
 `mother' & mɑi̯ɾ & \armenian{մայր} & mɑɾ, meɾ & \armenian{մար, մէր} & mɑjɾ & \armenian{մայր} \\
`goat' & ɑi̯t͡s & \armenian{այծ} & ɑt͡s, et͡s & \armenian{ած, էծ} & ɑjt͡s & \armenian{այծ} \\ 
\hline 
 \end{tabular}
\end{table}


\paragraph{Classical Armenian /oi̯/ <\armenian{ոյ}>}


 The Classical Armenian diphthong /oi̯/ <\armenian{ոյ}> changed to /u/ <\armenian{ու}> (Table \ref{tab:Karin:phono:change:diphth:oj}). 




\begin{table}[H]
 \centering
 \caption{Change from Classical Armenian /oi̯/ <\armenian{ոյ}> to /u/ <\armenian{ու}> in the Karin dialect}
 \label{tab:Karin:phono:change:diphth:oj}
 \begin{tabular}{| l| ll|ll| ll|}
 \hline & \multicolumn{2}{l|}{Classical Armenian} &\multicolumn{2}{l|}{> Karin} & \multicolumn{2}{l|}{cf. SEA} \\ 
`weak' & tʰoi̯l & \armenian{թոյլ} & tʰul & \armenian{թուլ} & tʰujl & \armenian{թույլ} \\ 
 `blue' & kɑpoi̯t & \armenian{կապոյտ} & kɑput & \armenian{կապուտ} & kɑpujt & \armenian{կապույտ} \\ 
 \armenian{՝}light' & loi̯s & \armenian{լոյս}& lus & \armenian{լուս} & lujs & \armenian{լույս} \\
\hline 
 \end{tabular}
\end{table}


\paragraph{Classical Armenian /iu̯/ <\armenian{իւ}>}


 The Classical Armenian diphthong /iu̯/ <\armenian{իւ}> changed to /u/ <\armenian{ու}> (Table \ref{tab:Karin:phono:change:diphth:iu̯}). 




\begin{table}[H]
 \centering
 \caption{Change from Classical Armenian /iu̯/ <\armenian{իւ}> to /u/ <\armenian{ու}> in the Karin dialect}
 \label{tab:Karin:phono:change:diphth:iu̯}
 \begin{tabular}{| l| ll|ll| ll|}
 \hline & \multicolumn{2}{l|}{Classical Armenian} &\multicolumn{2}{l|}{> Karin} & \multicolumn{2}{l|}{cf. SEA} \\ 
`weak' & tʰoi̯l & \armenian{թոյլ} & tʰul & \armenian{թուլ} & tʰujl & \armenian{թույլ} \\ 
`flour' & ɑliu̯ɾ & \armenian{ալիւր} & ɑluɾ & \armenian{ալուր} & ɑljuɾ & \armenian{ալյուր} \\ 
 `fountain' & ɑɬbiu̯ɾ & \armenian{աղբիւր} & ɑχbʰuɾ & \armenian{ախբՙուր} & ɑχpjuɾ & \armenian{աղբյուր} \\ 
 \armenian{՝}snow' & d͡ziu̯n & \armenian{ձիւն}& d͡zʰun & \armenian{ձՙուն} & d͡zjun & \armenian{ձյուն} \\
\hline 
 \end{tabular}
\end{table}

\subsubsection{Consonant changes}

\paragraph{Voicing changes}
For the consonants, the Old Armenian voiceless unaspirates and voiceless aspirates remain unchanged. The voiced sound have become voiced aspirates in general; but after nasals, they remain voiced unaspirated (Table \ref{tab:Karin:phono:change:cons:voice}). 




\begin{table}[H]
 \centering
 \caption{Voicing changes for stops and affricates in the Karin dialect}
 \label{tab:Karin:phono:change:cons:voice}
 \begin{tabular}{| l| ll|ll| ll|}
 \hline & \multicolumn{2}{l|}{Classical Armenian} &\multicolumn{2}{l|}{> Karin} & \multicolumn{2}{l|}{cf. SEA} \\ 
`thing' &bɑn & \armenian{բան} & bʰɑn & \armenian{բՙան} & bɑn & \armenian{բան} \\
`mouth' & beɾɑn & \armenian{բերան} & bʰeɾɑn &\armenian{բՙէրան} & beɾɑn & \armenian{բերան} \\
`hand' &d͡zer-kʰ (plural) & \armenian{ձեռք} & d͡zʰerkʰ & \armenian{ձՙէռք} &d͡zerkʰ & \armenian{ձեռք} \\
`I.{\dat}' &ind͡z& \armenian{ինձ} & ind͡zi & \armenian{ինձի} &ind͡z & \armenian{ինձ} \\
`apple' & χənd͡zoɾ & \armenian{խնձոր} & χənd͡zoɾ & \armenian{խընձօր} & χənd͡zoɾ & \armenian{խնձոր} \\ 
`cat' &kɑtu & \armenian{կատու} & kɑtu & \armenian{կատու} & kɑtu & \armenian{կատու} \\ 
`wool' & buɾd & \armenian{բուրդ} & bʰuɾdʰ & \armenian{բՙուրդՙ} & buɾtʰ & \armenian{բուրդ} \\ 
`sour' &tʰətʰu & \armenian{թթու} & tʰətʰu & \armenian{թըթու} & tʰətʰu & \armenian{թթու} \\ 
\hline 
 \end{tabular}
\end{table}

\paragraph{Assimilation of /t/ <\armenian{տ}> to a /r, t͡ʃ/ <\armenian{ր, ճ}>}

When the Classical sound /t/ <\armenian{տ}> is before the sounds /ɾ, r, t͡ʃ, ʒ/ <\armenian{ր, ռ, ճ, ժ}>, it merges with those sounds. Only in this situation does the Classical sound /ɾ/ <\armenian{ր}> become /r/ <\armenian{ռ}>, and the sound /ʒ/ <\armenian{ժ}> becomes /t͡ʃ/ <\armenian{ճ}> (Table \ref{tab:Karin:phono:change:cons:d}). 




\begin{table}[H]
 \centering
 \caption{Change from Classical Armenian /t/ <\armenian{տ}> to a /r, t͡ʃ/ <\armenian{ր, ճ}> in the Karin dialect}
 \label{tab:Karin:phono:change:cons:d}
 \begin{tabular}{| l| ll|ll| ll|}
 \hline & \multicolumn{2}{l|}{Classical Armenian} &\multicolumn{2}{l|}{> Karin} & \multicolumn{2}{l|}{cf. SEA} \\ 
`to tear' &pɑtɑrel & \armenian{պատառել} & pɑrrel & \armenian{պառռէլ} & pɑtɑrel, pɑtrel & \armenian{պատառել, պատռել} \\
`to divide' &kətɾel & \armenian{կտրել} & krrel & \armenian{կռռէլ} & kətɾel & \armenian{կտրել} \\
`to break' &kotoɾel & \armenian{կոտորել} & korrel & \armenian{կօռռէլ} & kotoɾel, kotɾel & \armenian{կոտորել, կոտրել} \\
`ready' &pɑtɾɑst & \armenian{պատրաստ} & pɑrrɑst & \armenian{պառռաստ} & pɑtɾɑst & \armenian{պատրաստ} \\
name `Peter' &petɾos & \armenian{Պետրոս} & perros & \armenian{Պէռռօս} & petɾos & \armenian{Պետրոս} \\
 `to punish' &pɑtʒel & \armenian{պատժել} & pɑt͡ʃt͡ʃel & \armenian{պաճճէլ} & pɑtʒel & \armenian{պատժել} \\
 `reason' &pɑtʒɑr & \armenian{պատճառ} & pɑt͡ʃt͡ʃɑr & \armenian{պաճճառ} & pɑtʒɑr & \armenian{պատճառ} \\
\hline 
 \end{tabular}
\end{table}

\begin{adjarianpage}\label{page:108}\end{adjarianpage}% should be 108

\subsection{Lexical change}

The Classical verb /ɑrnel/ <\armenian{առնել}> \armenian{՝}to do' becomes /enel/ <\armenian{էնէլ}>, whereas that word is /ɑnel/ <\armenian{անել}> or /ənel/ <\armenian{ընել}> in other places. \translatorHD{To clarify, the reflex of this verb is /ɑnel/ in SEA, and /ənel/ in SWA. }

\subsection{Stress}

In the Karin dialect, as in all the other dialects of the \armenian{կը} /kə/ branch, stress is on the last syllable. However, stress in Karin is an especially peculiar accent (\armenian{առագոնութիւն}) that it leaves a very pleasant impression. It is difficult for me to give a scientific explanation for this, but the following things are apparent. Stress in Karin is higher than the stress in other dialects; thus the difference in degree between unstressed and stressed syllables is very big. At the same time, because the pronunciation is more relaxed and elongated, during the descent, the sound goes through many musical notes, and it almost forms a song. 

\section{Morphology}
\subsection{Noun inflection or declension}

\subsubsection{Inflection for singular nouns}

Like all the dialect of the /kə/ <\armenian{կը}> branch, the Karin dialect has 6 cases, which are normative, genitive-dative, accusative, ablative, and instrumental. The locative is missing. 

However, the Karin dialect differ from the other dialects of the /kə/ <\armenian{կը}> branch; in the accusative, it distinguishes between animate and inanimate objects, similar to the /um/ <\armenian{ում}> branch. The accusative of inanimates is the same as the nominative, while the animates use the dative (\ref{sent:Karin:morpho:noun:DOM}). 


\begin{exe}
 \ex Karin \label{sent:Karin:morpho:noun:DOM}
 \begin{xlist}
 \ex \gll kɑtu-i-n səpɑn-e-t͡sʰ-i-$\emptyset$ \\ 
 cat-{\dat}-{\defgloss} kill-{\thgloss}-{\aor}-{\pst}-1{\sg} \\
 \trans `I killed the cat.'\\
 \armenian{կատուին սըպանէցի}
 \ex \gll kov-i-n moɾtʰ-e-t͡sʰ-i-$\emptyset$ \\ 
 cow-{\dat}-{\defgloss} slaughter-{\thgloss}-{\aor}-{\pst}-1{\sg} \\
 \trans `I slaughtered the cow.'\\
 \armenian{կօվին մօրթէցի}. 
 \end{xlist}
\end{exe}

As is the norm, the ablative uses the formative /-en/ <\armenian{էն}>, while the instrumental uses the formative /ov/ <\armenian{օվ}>. 

\subsubsection{Inflection for plural nouns}

In accordance with the general rule, the plural is formed with the formatives /-eɾ/ <\armenian{էր}> or /-neɾ/ <\armenian{նէր}>. But in this dialect, we also find the formative /estɑn/ <\armenian{Էստան}>. This formative is a reflex of the Old Armenian formative /-stɑn/ <\armenian{ստան}>, which is a location formative; the formative forms collective nouns and it can also receive the formative /neɾ/ <\armenian{նէր}> (Table \ref{tab:Karin:morpho:noun:pl}). 




\begin{table}[H]
 \centering
 \caption{Plural suffixes in the Karin dialect}
 \label{tab:Karin:morpho:noun:pl}
 \begin{tabular}{| l|ll| ll|}
 \hline & \multicolumn{2}{l|}{Karin} & \multicolumn{2}{l|}{cf. SEA} \\ 
`key-{\pl}' &bʰɑnl-estɑn & \armenian{բՙանլէստան} & bɑnɑli-neɾ & \armenian{բանալիներ} \\
 ~ ~ \textit{or} &bʰɑnl-estən-neɾ & \armenian{բՙանլէստըննէր} & & \\
`bathroom-{\pl}' &bʰɑʁn-estɑn & \armenian{բՙաղնէստան} & bɑʁnikʰ-neɾ & \armenian{բաղնիքներ} \\
~ ~ \textit{or} &bʰɑʁn-estən-neɾ & \armenian{բՙաղնէստըննէր} & & \\
`ring-{\pl}' &mɑtn-estɑn & \armenian{մատնէստան} & mɑtɑni-neɾ & \armenian{մատանիներ} \\
`dormer.window-{\pl}' &eɾdʰ-estɑn & \armenian{էրդՙէստան} & jeɾtikʰ-neɾ & \armenian{երդիքներ} \\
`intestine-{\pl}' &ɑʁ-estɑn & \armenian{աղէստան} & ɑʁikʰ-neɾ & \armenian{աղիքներ} \\
`bride-{\pl}' &hɑɾsn-estɑn & \armenian{հարսնէստան} & hɑɾs-eɾ & \armenian{հարսեր} \\
`underpants-{\pl}' &vɑɾt-estɑn & \armenian{վարտէստան} & vɑɾtikʰ-eɾ & \armenian{վարտիքներ} \\
`year-{\pl}' &tɑɾ-estɑn & \armenian{տարէստան} & tɑɾi-eɾ & \armenian{տարիներ} \\
\hline 
 \end{tabular}
\end{table}



As we can see from the examples, this formative is placed only after words that end in /-ikʰ/ <\armenian{իք}>. \translatorHD{I don't understand this generalization because it seems falsified by Adjarian's data. }

The other case markers of the plural are like the those of the singular, except for the genitive-dative which, in all the /kə/ <\armenian{կը}> branch dialects, uses the form /-u/ <\armenian{ու}> (Table \ref{tab:Karin:morpho:noun:plGen}). 


\begin{table}[H]
 \centering
 \caption{Genitive-dative of the plural in the Karin dialect}
 \label{tab:Karin:morpho:noun:plGen}
 \begin{tabular}{| l|ll| ll|}
 \hline & \multicolumn{2}{l|}{Karin} & \multicolumn{2}{l|}{cf. SEA} \\ 
`city-{\pl}-{\gen}/{\dat}' &kʰɑʁɑkʰ-neɾ-u & \armenian{քաղաքներու} & kʰɑʁɑkʰ-neɾ-i & \armenian{քաղաքների} \\
\hline 
 \end{tabular}
\end{table}



\begin{adjarianpage}\label{page:109}\end{adjarianpage}% should be 109

\subsection{Pronoun inflection or declension}

For pronouns, we note the following (Table \ref{tab:Karin:morphology:pronoun:sample}). 

\begin{table}[H]
 \centering
 \caption{Sample of pronouns in the Karin dialect}
 \label{tab:Karin:morphology:pronoun:sample}
 \begin{tabular}{|l ll| }
\hline 
personal 1SG {\nom} `I' &jes & \armenian{յէս} \\
personal 2SG {\nom} `you' &dʰu & \armenian{դՙու} \\
personal 1PL {\nom} `we' &menkʰ & \armenian{մէնք} \\
personal 2PL {\nom} `you' &dʰukʰ & \armenian{դՙուք} \\
demonstrative proximal {\sg} `this' & ɑs & \armenian{աս} \\
demonstrative medial {\sg} `that' & ɑdʰ & \armenian{ադՙ} \\
demonstrative distal {\sg} `that yonder' & ɑn & \armenian{ան} \\
demonstrative proximal {\pl} `these' & ɑsonkʰ & \armenian{ասօնք} \\
demonstrative medial {\pl} `those' & ɑtonkʰ & \armenian{ատօնք} \\
demonstrative distal {\pl} `those yonder' & ɑnonkʰ & \armenian{անօնք} \\
demonstrative proximal {\sg} `that' & isi & \armenian{իսի} \\
demonstrative medial {\sg} `that' & iti & \armenian{իտի} \\
demonstrative distal {\sg} `that yonder' & ini & \armenian{ինի} \\
demonstrative proximal {\sg} `that' & isik & \armenian{իսիկ} \\
demonstrative medial {\sg} `that' & itik & \armenian{իտիկ} \\
demonstrative distal {\sg} `that yonder' & inik & \armenian{ինիկ} \\
demonstrative proximal {\pl} `these' & isonkʰ & \armenian{իսօնք} \\
demonstrative medial {\pl} `those' & itonkʰ & \armenian{իտօնք} \\
demonstrative distal {\pl} `those yonder' & inonkʰ & \armenian{ինօնք} \\

\hline 
 \end{tabular}
\end{table}

In accordance with the norm, the first ones are not a unique innovation. The latter words /isik, inik, inin/ <\armenian{իսիկ, իտիկ, ինիկ}> are un-declinable; the others are declined in the following way (Table \ref{tab:Karin:morpho:pronoun}). 


\begin{table}[H]
 \centering
 \caption{Demonstrative pronouns in the Karin dialect} \label{tab:Karin:morpho:pronoun}

 \begin{tabular}{|l|lll|lll|}
 \hline & \multicolumn{3}{c|}{Singular} & \multicolumn{3}{c|}{Plural} \\
 & proximal & medial & distal & proximal & medial & distal \\
 & `this' & `that' & `yonder' & `these' & `those' & `those yonder' \\
 \hline {\nom} & isi & iti & ini & isonkʰ & itonkʰ & inonkʰ \\
 & \armenian{իսի} & \armenian{իտի} & \armenian{ինի} & \armenian{իսօնք} & \armenian{իտօնք} & \armenian{ինօնք} \\
{\gen},{\dat} & isoɾ & itoɾ & inoɾ & isont͡sʰ & itont͡sʰ & inont͡sʰ \\
 & \armenian{իսօր} & \armenian{իտօր} & \armenian{ինօր} & \armenian{իսօնց} & \armenian{իտօնց} & \armenian{ինօնց} \\
\hline {\abl} & isoɾ-en & itoɾ-en & inoɾ-en & isont͡sʰ-en & itont͡sʰ-en & inont͡sʰ-en \\
 & isoɾ-m-en & itoɾ-m-en & inoɾ-m-en & isont͡sʰ-m-en & itont͡sʰ-m-en & inont͡sʰ-m-en \\
 & \armenian{իսօրէն} & \armenian{իտօրէն} & \armenian{ինօրէն} & \armenian{իսօնցէն} & \armenian{իտօնցէն} & \armenian{ինօնցէն} \\
 & \armenian{իսօրմէն} & \armenian{իտօրմէն} & \armenian{ինօրմէն} & \armenian{իսօնցմէն} & \armenian{իտօնցմէն} & \armenian{ինօնցմէն} \\
\hline {\ins} & isoɾ-ov & itoɾ-ov & inoɾ-ov & isont͡sʰ-ov & itont͡sʰ-ov & inont͡sʰ-ov \\
 & isoɾ-m-ov & itoɾ-m-ov & inoɾ-m-ov & isont͡sʰ-m-ov & itont͡sʰ-m-ov & inont͡sʰ-m-ov \\
 & \armenian{իսօրօվ} & \armenian{իտօրօվ} & \armenian{ինօրօվ} & \armenian{իսօնցօվ} & \armenian{իտօնցօվ} & \armenian{ինօնցօվ} \\
 & \armenian{իսօրմօվ} & \armenian{իտօրմօվ} & \armenian{ինօրմօվ} & \armenian{իսօնցմօվ} & \armenian{իտօնցմօվ} & \armenian{ինօնցմօվ} 
\\ \hline
 \end{tabular}
\end{table}

\subsection{Verb inflection or conjugation}

\subsubsection{Indicative present and past imperfective: allomorphy of the indicative morpheme}

The formation of verbs is very similar. There are no tenses that are constructed with /-um/ <\armenian{ում}>, as in all the /kə/ <\armenian{կը}> dialects. The indicative present and imperfective are formed similar to Old Armenian, but here we add the formative \armenian{կը} /kə/, which is placed at the end of the verb in the Karin dialect. 

Verbs that start with a vowel also receive the formative /k-/ <\armenian{կ}> at the beginning; the verbs /əllil/ <\armenian{ըլլիլ}> `to be', /əjnil/ \armenian{ըյնիլ} <to fall>, /uzel/ <\armenian{ուզէլ}> `to want', etc. take the form /ɡʰ-/ <\armenian{գՙ}>. Monosyllabic verbs take /ku-/ <\armenian{կու}>; only the verb /ɡʰɑm/ <\armenian{գՙամ}> `I come' takes /ɡʰu/ <\armenian{գՙու}> (in assimilation with the verb's initial sound /ɡʰ/ <\armenian{գՙ}>). Thus are all the forms of these verbs:

\paragraph{Suffix or enclitic /kə/ for consonant-initial verbs}

\translatorHD{I clarify what he means. In Eastern dialects like SEA, the indicative present and past imperfective are formed periphrastically with a non-finite converb plus a finite auxiliary. But in Western dialects like SWA and Karin, these forms are created synthetically. Tense and inflection is on the finite verb, while the indicative mood is marked by adding a morpheme that looks like /kə/. In SWA, this morpheme is /ɡu-/ before monosyllabic roots, /ɡ-/ before vowels, and /ɡə-/ elsewhere before consonants. Karin uses cognates of these affix shapes with essentially the same distribution.}

\translatorHD{First consider a typical consonant-initial verb like /siɾ-e-l/ `to like'. In SWA, the indicative present is formed by adding agreement markers after the theme vowel (Table \ref{tab:Karin:morpho:verb:paradigm:presentPastIndc}). The indicative past imperfective includes a past marker /-i-/ between the theme and agreement. For both the present and the past, the 3SG is missing either a past marker or an agreement marker. For both tenses, this verb takes the indicative prefix /ɡə-/. Karin uses essentially the same strategy, but the indicative marker is an enclitic or suffix /kə/. }


\begin{table}[H]
 \centering
 \caption{Indicative present <\armenian{ներկայ}> and indicative past imperfective <\armenian{անկատար}> of the verb `to like' in the Karin dialect}
\label{tab:Karin:morpho:verb:paradigm:presentPastIndc}
 \begin{tabular}{|l|ll|ll|}
 \hline & \multicolumn{4}{l|}{Indicative present <\armenian{ներկայ}>} \\
 & \multicolumn{2}{l|}{Karin} & \multicolumn{2}{l|}{cf. SWA} \\ \hline 
1SG & siɾ-e-m kə & \armenian{սիրէմ կը} & ɡə siɾ-e-m & \armenian{կը սիրեմ} \\
2SG & siɾ-e-s kə & \armenian{սիրէս կը} & ɡə siɾ-e-s & \armenian{կը սիրես} \\
3SG & siɾ-e-$\emptyset$ kə & \armenian{սիրէ կը} & ɡə siɾ-e-$\emptyset$ & \armenian{կը սիրէ} \\
1PL & siɾ-e-nkʰ kə & \armenian{սիրէնք կը} & ɡə siɾ-e-ŋkʰ & \armenian{կը սիրենք} \\
2PL & siɾ-e-kʰ kə & \armenian{սիրէք կը} & ɡə siɾ-e-kʰ & \armenian{կը սիրէք} \\
3PL & siɾ-e-n kə & \armenian{սիրէն կը} & ɡə siɾ-e-n & \armenian{կը սիրեն} \\
& \multicolumn{2}{l|}{$\sqrt{}$-{\thgloss}-{\agr} {\ind}} & \multicolumn{2}{l|}{{\ind} $\sqrt{}$-{\thgloss}-{\agr}}
 \\ \hline 
\hline & \multicolumn{4}{l|}{Indicative past imperfective <\armenian{անկատար}> }\\
& \multicolumn{2}{l|}{Karin} & \multicolumn{2}{l|}{cf. SWA} \\
1SG & siɾ-e-i-$\emptyset$ kə & \armenian{սիրէի կը} & ɡə siɾ-ej-i-$\emptyset$ & \armenian{կը սիրէի} \\
2SG & siɾ-e-i-ɾ kə & \armenian{սիրէիր կը} & ɡə siɾ-ej-i-ɾ & \armenian{կը սիրէիր} \\
3SG & siɾ-e-$\emptyset$-ɾ kə & \armenian{սիրէր կը} & ɡə siɾ-e-$\emptyset$-ɾ & \armenian{կը սիրէր} \\
1PL & siɾ-e-i-nkʰ kə & \armenian{սիրէինք կը} & ɡə siɾ-ej-i-ŋkʰ & \armenian{կը սիրէինք} \\
2PL & siɾ-e-i-kʰ kə & \armenian{սիրէիք կը} & ɡə siɾ-ej-i-kʰ & \armenian{կը սիրէիք} \\
3PL & siɾ-e-i-n kə & \armenian{սիրէին կը} & ɡə siɾ-ej-i-n & \armenian{կը սիրէին} \\
& \multicolumn{2}{l|}{$\sqrt{}$-{\thgloss}-{\pst}-{\agr} {\ind}}& \multicolumn{2}{l|}{{\ind} $\sqrt{}$-{\thgloss}-{\pst}-{\agr} } \\
\hline 
\end{tabular}
\end{table}


\paragraph{Prefix and suffix/enclitic /k-...-kə/ or /ɡʰ-...-kə/ for vowel-initial verbs}

\begin{adjarianpage}\label{page:110}\end{adjarianpage}% should be 110

\translatorHD{For a vowel-initial verb like /ən-e-l/ `to do' (Table \ref{tab:Karin:morpho:verb:paradigm:presentPastIndcDo}), SWA uses an indicative prefix /ɡ-/ instead of /ɡə-/. In Karin, this verb uses both an indicative prefix /k-/ and an indicative suffix/enclitic /kə/. }


\begin{table}[H]
 \centering
 \caption{Indicative present <\armenian{ներկայ}> and indicative past imperfective <\armenian{անկատար}> of the verb `to do' in the Karin dialect}
 \label{tab:Karin:morpho:verb:paradigm:presentPastIndcDo}
 \begin{tabular}{|l|ll|ll|}
 \hline & \multicolumn{4}{l|}{Indicative present <\armenian{ներկայ}>} \\
 & \multicolumn{2}{l|}{Karin} & \multicolumn{2}{l|}{cf. SWA} \\ \hline 
1SG & k-ən-e-m kə & \armenian{կընէմ կը} & ɡ-ən-e-m & \armenian{կ՚ընեմ} \\
2SG & k-ən-e-s kə & \armenian{կընէս կը} & ɡ-ən-e-s & \armenian{կ՚ընես} \\
3SG & k-ən-e-$\emptyset$ kə & \armenian{կընէ կը} & ɡ-ən-e-$\emptyset$ & \armenian{կ՚ընէ} \\
1PL & k-ən-e-nkʰ kə & \armenian{կընէնք կը} & ɡ-ən-e-ŋkʰ & \armenian{կ՚ընենք} \\
2PL & k-ən-e-kʰ kə & \armenian{կընէք կը} & ɡ-ən-e-kʰ & \armenian{կ՚ընէք} \\
3PL & k-ən-e-n kə & \armenian{կընէն կը} & ɡ-ən-e-n & \armenian{կ՚ընեն} \\
& \multicolumn{2}{l|}{{\ind}-$\sqrt{}$-{\thgloss}-{\agr} {\ind}} & \multicolumn{2}{l|}{{\ind}-$\sqrt{}$-{\thgloss}-{\agr}}
 \\ \hline 
\hline & \multicolumn{4}{l|}{Indicative past imperfective <\armenian{անկատար}> }\\
& \multicolumn{2}{l|}{Karin} & \multicolumn{2}{l|}{cf. SWA} \\
1SG & k-ən-e-i-$\emptyset$ kə & \armenian{կընէի կը} & ɡ-ən-ej-i-$\emptyset$ & \armenian{կ՚ընէի} \\
2SG & k-ən-e-i-ɾ kə & \armenian{կընէիր կը} & ɡ-ən-ej-i-ɾ & \armenian{կ՚ընէիր} \\
3SG & k-ən-e-$\emptyset$-ɾ kə & \armenian{կընէր կը} & ɡ-ən-e-$\emptyset$-ɾ & \armenian{կ՚ընէր} \\
1PL & k-ən-e-i-nkʰ kə & \armenian{կընէինք կը} & ɡ-ən-ej-i-ŋkʰ & \armenian{կ՚ընէինք} \\
2PL & k-ən-e-i-kʰ kə & \armenian{կընէիք կը} & ɡ-ən-ej-i-kʰ & \armenian{կ՚ընէիք} \\
3PL & k-ən-e-i-n kə & \armenian{կընէին կը} & ɡ-ən-ej-i-n & \armenian{կ՚ընէին} \\
& \multicolumn{2}{l|}{{\ind}-$\sqrt{}$-{\thgloss}-{\pst}-{\agr} {\ind}}& \multicolumn{2}{l|}{{\ind}-$\sqrt{}$-{\thgloss}-{\pst}-{\agr} } \\
\hline 
\end{tabular}
\end{table}


\translatorHD{In Karin, for some exceptional vowel-initial verbs like `to fall', the indicative prefix is a voiced aspirate /ɡʰ/ (Table \ref{tab:Karin:morpho:verb:paradigm:presentPastIndcFall}). The indicative suffix/enclitic is still just /kə/. No such exceptionality is found in SWA.} 


\begin{table}[H]
 \centering
 \caption{Indicative present <\armenian{ներկայ}> and indicative past imperfective <\armenian{անկատար}> of the verb `to fall' in the Karin dialect}
 \label{tab:Karin:morpho:verb:paradigm:presentPastIndcFall}
 \begin{tabular}{|l|ll|ll|}
 \hline & \multicolumn{4}{l|}{Indicative present <\armenian{ներկայ}>} \\
 & \multicolumn{2}{l|}{Karin} & \multicolumn{2}{l|}{cf. SWA} \\ \hline 
1SG & ɡʰ-əjn-i-m kə & \armenian{գՙըյնիմ կը} & ɡ-ijn-ɑ-m & \armenian{կ՚իյնամ} \\
2SG & ɡʰ-əjn-i-s kə & \armenian{գՙըյնիս կը} & ɡ-ijn-ɑ-s & \armenian{կ՚իյնաս} \\
3SG & ɡʰ-əjn-i-$\emptyset$ kə & \armenian{գՙըյնի կը} & ɡ-ijn-ɑ-$\emptyset$ & \armenian{կ՚իյնայ} \\
1PL & ɡʰ-əjn-i-nkʰ kə & \armenian{գՙըյնինք կը} & ɡ-ijn-ɑ-ŋkʰ & \armenian{կ՚իյնանք} \\
2PL & ɡʰ-əjn-i-kʰ kə & \armenian{գՙըյնիք կը} & ɡ-ijn-ɑ-kʰ & \armenian{կ՚իյնաք} \\
3PL & ɡʰ-əjn-i-n kə & \armenian{գՙըյնին կը} & ɡ-ijn-ɑ-n & \armenian{կ՚իյնան} \\
& \multicolumn{2}{l|}{{\ind}-$\sqrt{}$-{\thgloss}-{\agr} {\ind}} & \multicolumn{2}{l|}{{\ind}-$\sqrt{}$-{\thgloss}-{\agr}}
 \\ \hline 
\hline & \multicolumn{4}{l|}{Indicative past imperfective <\armenian{անկատար}> }\\
& \multicolumn{2}{l|}{Karin} & \multicolumn{2}{l|}{cf. SWA} \\
1SG & ɡʰ-əjn-e-i-$\emptyset$ kə & \armenian{գՙըյնէի կը} & ɡ-ijn-ɑj-i-$\emptyset$ & \armenian{կ՚իյնայի} \\
2SG & ɡʰ-əjn-e-i-ɾ kə & \armenian{գՙըյնէիր կը} & ɡ-ijn-ɑj-i-ɾ & \armenian{կ՚իյնայիր} \\
3SG & ɡʰ-əjn-e-$\emptyset$-ɾ kə & \armenian{գՙըյնէր կը} & ɡ-ijn-ɑ-$\emptyset$-ɾ & \armenian{կ՚իյնար} \\
1PL & ɡʰ-əjn-e-i-nkʰ kə & \armenian{գՙըյնէինք կը} & ɡ-ijn-ɑj-i-ŋkʰ & \armenian{կ՚իյնայինք} \\
2PL & ɡʰ-əjn-e-i-kʰ kə & \armenian{գՙըյնէիք կը} & ɡ-ijn-ɑj-i-kʰ & \armenian{կ՚իյնայիք} \\
3PL & ɡʰ-əjn-e-i-n kə & \armenian{գՙըյնէին կը} & ɡ-ijn-ɑj-i-n & \armenian{կ՚իյնային} \\
& \multicolumn{2}{l|}{{\ind}-$\sqrt{}$-{\thgloss}-{\pst}-{\agr} {\ind}}& \multicolumn{2}{l|}{{\ind}-$\sqrt{}$-{\thgloss}-{\pst}-{\agr} } \\
\hline 
\end{tabular}
\end{table}

\paragraph{Prefix and suffix/enclitic /ku-...-kə/ or /ɡʰu-...-kə/ for monosyllabic verbs}

\translatorHD{For monosyllabic verbs, SWA uses the indicative prefix /ɡu-/. In Karin, this prefix is /ku-/ or /ɡʰu-/. Note that in SWA and apparently in Karin, there are only three monosyllabic verbs that can take indicative morphology: `cry' (Table \ref{tab:Karin:morpho:verb:paradigm:presentPastIndcCry}), `to give' (Table \ref{tab:Karin:morpho:verb:paradigm:presentPastIndcGive}), and `to come' (Table \ref{tab:Karin:morpho:verb:paradigm:presentPastIndcCome}). The verb `to come' takes the voiced prefix /ɡʰu-/.}


\begin{table}[H]
 \centering
 \caption{Indicative present <\armenian{ներկայ}> and indicative past imperfective <\armenian{անկատար}> of the verb `to cry' in the Karin dialect}
 \label{tab:Karin:morpho:verb:paradigm:presentPastIndcCry}
 \begin{tabular}{|l|ll|ll|}
 \hline & \multicolumn{4}{l|}{Indicative present <\armenian{ներկայ}>} \\
& \multicolumn{2}{l|}{Karin} & \multicolumn{2}{l|}{cf. SWA} \\ \hline 
1SG & ku-l-ɑ-m kə & \armenian{կուլամ կը} & ɡu-l-ɑ-m & \armenian{կու լամ} \\
2SG & ku-l-ɑ-s kə & \armenian{կուլաս կը} & ɡu-l-ɑ-s & \armenian{կու լաս} \\
3SG & ku-l-ɑ-$\emptyset$ kə & \armenian{կուլա կը} & ɡu-l-ɑ-$\emptyset$ & \armenian{կու լայ} \\
1PL & ku-l-ɑ-nkʰ kə & \armenian{կուլանք կը} & ɡu-l-ɑ-ŋkʰ & \armenian{կու լանք} \\
2PL & ku-l-ɑ-kʰ kə & \armenian{կուլաք կը} & ɡu-l-ɑ-kʰ & \armenian{կու լաք} \\
3PL & ku-l-ɑ-n kə & \armenian{կուլան կը} & ɡu-l-ɑ-n & \armenian{կու լան} \\
& \multicolumn{2}{l|}{{\ind}-$\sqrt{}$-{\thgloss}-{\agr} {\ind}} & \multicolumn{2}{l|}{{\ind}-$\sqrt{}$-{\thgloss}-{\agr}}
 \\ \hline 
\hline & \multicolumn{4}{l|}{Indicative past imperfective <\armenian{անկատար}> }\\
& \multicolumn{2}{l|}{Karin} & \multicolumn{2}{l|}{cf. SWA} \\
1SG & ku-l-ɑj-i-$\emptyset$ kə & \armenian{կուլայի կը} & ɡu-l-ɑj-i-$\emptyset$ & \armenian{կու լայի} \\
2SG & ku-l-ɑj-i-ɾ kə & \armenian{կուլայիր կը} & ɡu-l-ɑj-i-ɾ & \armenian{կու լայիր} \\
3SG & ku-l-ɑ-$\emptyset$-ɾ kə & \armenian{կուլար կը} & ɡu-l-ɑ-$\emptyset$-ɾ & \armenian{կու լար} \\
1PL & ku-l-ɑj-i-nkʰ kə & \armenian{կուլայինք կը} & ɡu-l-ɑj-i-ŋkʰ & \armenian{կու լայինք} \\
2PL & ku-l-ɑj-i-kʰ kə & \armenian{կուլայիք կը} & ɡu-l-ɑj-i-kʰ & \armenian{կու լայիք} \\
3PL & ku-l-ɑj-i-n kə & \armenian{կուլային կը} & ɡu-l-ɑj-i-n & \armenian{կու լային} \\
& \multicolumn{2}{l|}{{\ind}-$\sqrt{}$-{\thgloss}-{\pst}-{\agr} {\ind}}& \multicolumn{2}{l|}{{\ind}-$\sqrt{}$-{\thgloss}-{\pst}-{\agr} } \\
\hline 
\end{tabular}
\end{table}


\begin{table}[H]
 \centering
 \caption{Indicative present <\armenian{ներկայ}> and indicative past imperfective <\armenian{անկատար}> of the verb `to give' in the Karin dialect}
 \label{tab:Karin:morpho:verb:paradigm:presentPastIndcGive}
 \begin{tabular}{|l|ll|ll|}
 \hline & \multicolumn{4}{l|}{Indicative present <\armenian{ներկայ}>} \\
 & \multicolumn{2}{l|}{Karin} & \multicolumn{2}{l|}{cf. SWA} \\ \hline 
1SG & ku-t-ɑ-m kə & \armenian{կուտամ կը} & ɡu-d-ɑ-m & \armenian{կու տամ} \\
2SG & ku-t-ɑ-s kə & \armenian{կուտաս կը} & ɡu-d-ɑ-s & \armenian{կու տաս} \\
3SG & ku-t-ɑ-$\emptyset$ kə & \armenian{կուտա կը} & ɡu-d-ɑ-$\emptyset$ & \armenian{կու տայ} \\
1PL & ku-t-ɑ-nkʰ kə & \armenian{կուտանք կը} & ɡu-d-ɑ-ŋkʰ & \armenian{կու տանք} \\
2PL & ku-t-ɑ-kʰ kə & \armenian{կուտաք կը} & ɡu-d-ɑ-kʰ & \armenian{կու տաք} \\
3PL & ku-t-ɑ-n kə & \armenian{կուտան կը} & ɡu-d-ɑ-n & \armenian{կու տան} \\
& \multicolumn{2}{l|}{{\ind}-$\sqrt{}$-{\thgloss}-{\agr} {\ind}} & \multicolumn{2}{l|}{{\ind}-$\sqrt{}$-{\thgloss}-{\agr}}
 \\ \hline 
\hline & \multicolumn{4}{l|}{Indicative past imperfective <\armenian{անկատար}> }\\
& \multicolumn{2}{l|}{Karin} & \multicolumn{2}{l|}{cf. SWA} \\
1SG & ku-t-ɑj-i-$\emptyset$ kə & \armenian{կուտայի կը} & ɡu-d-ɑj-i-$\emptyset$ & \armenian{կու տայի} \\
2SG & ku-t-ɑj-i-ɾ kə & \armenian{կուտայիր կը} & ɡu-d-ɑj-i-ɾ & \armenian{կու տայիր} \\
3SG & ku-t-ɑ-$\emptyset$-ɾ kə & \armenian{կուտար կը} & ɡu-d-ɑ-$\emptyset$-ɾ & \armenian{կու տար} \\
1PL & ku-t-ɑj-i-nkʰ kə & \armenian{կուտայինք կը} & ɡu-d-ɑj-i-ŋkʰ & \armenian{կու տայինք} \\
2PL & ku-t-ɑj-i-kʰ kə & \armenian{կուտայիք կը} & ɡu-d-ɑj-i-kʰ & \armenian{կու տայիք} \\
3PL & ku-t-ɑj-i-n kə & \armenian{կուտային կը} & ɡu-d-ɑj-i-n & \armenian{կու տային} \\
& \multicolumn{2}{l|}{{\ind}-$\sqrt{}$-{\thgloss}-{\pst}-{\agr} {\ind}}& \multicolumn{2}{l|}{{\ind}-$\sqrt{}$-{\thgloss}-{\pst}-{\agr} } \\
\hline 
\end{tabular}
\end{table}


\begin{table}[H]
 \centering
 \caption{Indicative present <\armenian{ներկայ}> and indicative past imperfective <\armenian{անկատար}> of the verb `to come' in the Karin dialect}
 \label{tab:Karin:morpho:verb:paradigm:presentPastIndcCome}
 \begin{tabular}{|l|ll|ll|}
 \hline & \multicolumn{4}{l|}{Indicative present <\armenian{ներկայ}>} \\
 & \multicolumn{2}{l|}{Karin} & \multicolumn{2}{l|}{cf. SWA} \\ \hline 
1SG & ɡʰu-ɡʰ-ɑ-m kə & \armenian{գՙուգՙամ կը} & ɡu-kʰ-ɑ-m & \armenian{կու գամ} \\
2SG & ɡʰu-ɡʰ-ɑ-s kə & \armenian{գՙուգՙաս կը} & ɡu-kʰ-ɑ-s & \armenian{կու գաս} \\
3SG & ɡʰu-ɡʰ-ɑ-$\emptyset$ kə & \armenian{գՙուգՙա կը} & ɡu-kʰ-ɑ-$\emptyset$ & \armenian{կու գայ} \\
1PL & ɡʰu-ɡʰ-ɑ-nkʰ kə & \armenian{գՙուգՙանք կը} & ɡu-kʰ-ɑ-ŋkʰ & \armenian{կու գանք} \\
2PL & ɡʰu-ɡʰ-ɑ-kʰ kə & \armenian{գՙուգՙաք կը} & ɡu-kʰ-ɑ-kʰ & \armenian{կու գաք} \\
3PL & ɡʰu-ɡʰ-ɑ-n kə & \armenian{գՙուգՙան կը} & ɡu-kʰ-ɑ-n & \armenian{կու գան} \\
& \multicolumn{2}{l|}{{\ind}-$\sqrt{}$-{\thgloss}-{\agr} {\ind}} & \multicolumn{2}{l|}{{\ind}-$\sqrt{}$-{\thgloss}-{\agr}}
 \\ \hline 
\hline & \multicolumn{4}{l|}{Indicative past imperfective <\armenian{անկատար}> }\\
& \multicolumn{2}{l|}{Karin} & \multicolumn{2}{l|}{cf. SWA} \\
1SG & ɡʰu-ɡʰ-ɑj-i-$\emptyset$ kə & \armenian{գՙուգՙայի կը} & ɡu-kʰ-ɑj-i-$\emptyset$ & \armenian{կու գայի} \\
2SG & ɡʰu-ɡʰ-ɑj-i-ɾ kə & \armenian{գՙուգՙայիր կը} & ɡu-kʰ-ɑj-i-ɾ & \armenian{կու գայիր} \\
3SG & ɡʰu-ɡʰ-ɑ-$\emptyset$-ɾ kə & \armenian{գՙուգՙար կը} & ɡu-kʰ-ɑ-$\emptyset$-ɾ & \armenian{կու գար} \\
1PL & ɡʰu-ɡʰ-ɑj-i-nkʰ kə & \armenian{գՙուգՙայինք կը} & ɡu-kʰ-ɑj-i-ŋkʰ & \armenian{կու գայինք} \\
2PL & ɡʰu-ɡʰ-ɑj-i-kʰ kə & \armenian{գՙուգՙայիք կը} & ɡu-kʰ-ɑj-i-kʰ & \armenian{կու գայիք} \\
3PL & ɡʰu-ɡʰ-ɑj-i-n kə & \armenian{գՙուգՙային կը} & ɡu-kʰ-ɑj-i-n & \armenian{կու գային} \\
& \multicolumn{2}{l|}{{\ind}-$\sqrt{}$-{\thgloss}-{\pst}-{\agr} {\ind}}& \multicolumn{2}{l|}{{\ind}-$\sqrt{}$-{\thgloss}-{\pst}-{\agr} } \\
\hline 
\end{tabular}
\end{table}

\paragraph{Omission or reduction of the indicative morpheme}

When a few present forms succeed each other, the formative /kə/ <\armenian{կը}> is placed only after the last one (\ref{sent:Karin:morpho:verb:indcOmissionKe}).

\begin{exe}
 \ex Karin \label{sent:Karin:morpho:verb:indcOmissionKe}
 \begin{xlist}
 \ex \gll t͡ʃɑmpʰɑ-n kʰun-ə tɑn-i-$\emptyset$ ɡ-əjn-i-$\emptyset$ kə\\
 road-{\defgloss} -{\defgloss} take-{\thgloss}-3{\sg} {\ind}-fall?-{\thgloss}-3{\sg} {\ind}
 \\
 \trans I'm not completely sure but I think this is loosely translated as `While on the road, he gets sleepy and falls.' Note that in modern Armenian, the idomatic phrase `to take sleep' means `to get sleepy'. I'm not sure if I correctly analyzed the second verb. \\
 \armenian{ճամփան քունը տանի գըյնի կը}
 \ex \gll zɑɾm-ɑ-n-ɑ-n mn-ɑ-n kə\\
 surprise-{\lvgloss}-{\inch}-{\defgloss}-3{\pl} stay-{\thgloss}-3{\pl} {\ind}
 \\
 \trans `They get surprised, they stay \\
 \armenian{զարմանան մնան կը} 
 \end{xlist}
\end{exe}


This is strengthened until the verb is separated from various other words (\ref{sent:Karin:morpho:verb:indcOmissionKeStrong}). \translatorHD{I don't understand what this example shows. }

\begin{exe}
 \ex Karin\label{sent:Karin:morpho:verb:indcOmissionKeStrong}
 \begin{xlist}
\ex \gll ɑɾun kʰɾtinkʰ mtn-i-n kə\\
blood sweat enter?-{\thgloss}-3{\pl} {\ind}\\
\trans I'm not sure but I think this means `they enter blood and sweat.' I'm not sure what the verb means. \\
\armenian{արուն քրտինք մտնին կը}.
\end{xlist}
\end{exe}


 When this formative is immediately before the forms /oɾ/ <\armenian{օր}> `that', ... 

\begin{adjarianpage}\label{page:111}\end{adjarianpage}% should be 111

... or /u/ <\armenian{ու}> `and', the formative /kə/ \armenian{կը} is reduced and merges with those words to form /koɾ, ku/ <\armenian{կօր, կու}>.\footnote{\translatorHD{Adjarian includes a parenthetical <(\armenian{իմա կ՚օր, կ՚ու})>. But I don't understand the parenthetical. I suspect he means that these fused forms are written as <\armenian{կ՚օր, կ՚ու}>. }} 

\begin{exe}
 \ex Karin\label{sent:Karin:morpho:verb:korku}
 \begin{xlist}
 \ex \gll kɑʃ-e-n k-oɾ\\ 
 see-{\thgloss}-3{\pl} {\ind}-that\\ 
 \trans `they see that...'\\
 \armenian{կաշէն կօր}
 \ex \gll bʰeɾ-e-$\emptyset$ k-u tɑn-i-$\emptyset$ kə\\ 
 bring-{\thgloss}-3{\sg} {\ind}-and take-{\thgloss}-3{\sg}\\ 
 \trans `He brings and he takes.'\\ 
 \armenian{բՙէրէ կու տանի կը}
 \end{xlist}
\end{exe}

\paragraph{Theme vowel change in the past imperfective 3SG}

Many times in the 3SG of the imperfective, the sound /e/ <\armenian{է}> becomes /i/ <\armenian{ի}>, such as (Table \ref{tab:Karin:morpho:verb:presentPastIndc:themeChange}). 


\begin{table}[H]
 \centering
 \caption{Theme vowel change in the 3SG past imperfective in the Karin dialect}
 \label{tab:Karin:morpho:verb:presentPastIndc:themeChange}
 \begin{tabular}{|l|ll|ll|}
 & \multicolumn{2}{l|}{Karin} & \multicolumn{2}{l|}{cf. SWA} \\ \hline 
3SG `to have' & un-i-$\emptyset$-ɾ & \armenian{ունիր} & un-e-$\emptyset$-ɾ & \armenian{ունէր} \\
3SG `to fall' & ɡ-əjn-i-$\emptyset$-ɾ & \armenian{գըյնիր} & un-e-$\emptyset$-ɾ & \armenian{կ՚իյնար} \\
& \multicolumn{2}{l|}{({\ind})?-$\sqrt{}$-{\thgloss}-{\pst}-3{\sg} {\ind}}& \multicolumn{2}{l|}{({\ind}?)-$\sqrt{}$-{\thgloss}-{\pst}-3{\sg} } \\
\hline 
\end{tabular}
\end{table}

\subsubsection{Future and future perfect}

The future and future perfect (\armenian{ապառնիին ներկան եւ անցեալը}) are formed with the formative /piti/ <\armenian{պիտի}>, which can be placed before or after the verb. 

\translatorHD{To clarify, in SWA, the future is formed by replacing the indicative morpheme of the indicative present with the future proclitic /bidi/ (\ref{sent:Karin:morpho:verb:fut}). The same tense and agreement markers are used. The future perfect is similarly formed in contrast to the indicative past imperfect. Karin seems to use the same strategies, though the future morpheme has variable positions. }
 

\begin{exe}
 \ex Karin \label{sent:Karin:morpho:verb:fut}
 \begin{xlist}
 \ex \gll piti siɾ-e-m, piti siɾ-e-i-$\emptyset$ \\
 {\fut} like-{\thgloss}-1{\sg}, {\fut} like-{\thgloss}-{\pst}-1{\sg} \\
\trans `I will like, I was going to like.' 
\armenian{պիտի սիրէմ, պիտի սիրէի} 
 \ex \gll siɾ-e-m piti, siɾ-e-i-$\emptyset$ piti\\
 like-{\thgloss}-1{\sg} {\fut}, like-{\thgloss}-{\pst}-1{\sg} {\fut} \\
\trans `I will like, I was going to like.' 
\armenian{սիրէմ պիտի, սիրէի պիտի}
 \end{xlist}
\end{exe}


\subsubsection{Perfective converb or past participle with /-eɾ, -e/ <\armenian{էր, է}>}

The past participle takes the formative /-eɾ/ <\armenian{էր}>. But when the verb is after the auxiliary, the final /ɾ/ <\armenian{ր}> is deleted (\ref{sent:Karin:morpho:verb:pastPart}). 

\begin{exe}
 \ex Karin \label{sent:Karin:morpho:verb:pastPart}
 \begin{xlist}
 \ex \gll siɾ-eɾ e \\
 like-{\perfcvb} {\aux} \\
 \trans `He has liked.'\\
 \armenian{սիրէր է}
 \ex \gll t͡ʃʰ-e-m siɾ-e \\
 {\neggloss}-{\aux}-1{\sg} like-{\perfcvb} \\
 \trans `I have not liked.'\\
 \armenian{չէմ սիրէ}
 \ex \gll dʰu e-s bʰeɾ-e \\
 you {\aux}-2{\sg} bring-{\perfcvb} \\
 \trans `YOU have brought.' (with focus on `you')\\
 \armenian{դՙու էս բՙէրէ}
 \ex \gll inik e-s bʰeɾ-e \\
 that {\aux}-2{\sg} bring-{\perfcvb} \\
 \trans `You have brought THAT.' (with focus on `that')\\
 \armenian{ինիկ էս բՙէրէ}
 \end{xlist}
\end{exe}

 \section{Subdialects}

 Despite its widespread distribution, the Karin dialect does not have many subdialects. The same dialect is spoken in Karin, Akhaltskha, Kars, Akhalkalak, Alexandropol, and in their villages. 
 
 Exceptions are made only for the sounds /i̯e, e, u̯o, o/ <\armenian{ե է ո օ}>, and the change from the Classical diphthong /ɑi̯/ <\armenian{այ}> to /ɑ/ <\armenian{ա}> or /e/ <\armenian{է}>. 
 
 The people of Akhaltskha and Karin use the form /ɡʰ-əll-i/ <\armenian{գՙըլլի}> `it is', while the people of Alexandropol use /k-eʁn-i/ <\armenian{կէղնի}> `it is'.\footnote{\translatorHD{These are glossed as `{\ind}-be-{\thgloss}' with a covert 3SG suffix.}} But these forms are even found in the villages that are next to Karin, and they don't form their own decisive gaps (\armenian{առանձին որոշողական անջրպետ}). 

\translatorHD{For some reason, Adjarian talks about subdialects here, but then switches topics to talk about pre-existing literatures. And then returns to talking about dialects. }

\section{Literature}

For the Karin dialect, there is only a small study in Russian \citep{Thomson-1887-Karin}.\footnote{\translatorHD{Adjarian's original citation uses the pre-revolution Russian spelling: Томсонъ, Лингвистическія изслҍдованія: Краткій очеркъ фонетики и морфологіи ахалцыхскаго говора. \armenian{Բեդրսպուրկ} 1887.}} 

The following are works that are written in this dialect. 


{\litoverview}
 
\begin{itemize}
 \item Literature with the Karin dialect
 \begin{itemize}
 \item \armenian{Ե. Լալայեանց – Ջաւախքի բուրմուն. Թիֆլիս} 1892
\item \armenian{Ջաւախեցի – Ջաւախքի աղէտը. Թիֆլիս} 1900
\item \armenian{Արամ Չարուգ – Բասենի ժողովրդ. երգերը. Ազգ. Հանդ. Զ. էջ} 383-390
\item \armenian{Ե. Լալայեան – Ջաւախք. նոյն Ա. էջ} 327, 364, and so on. 
\item \armenian{Դպիր – Նարմանցիին երգերը. Բիւրակն}, 1899, \armenian{էջ} 524-5
\item \armenian{Խօջայեանց Յովհ}. – \armenian{Ասածներ Ալէքսանդրապօլից. Արրտ}. 1870-1, \armenian{էջ} 249-250, 283-4, 309-312
\item \armenian{Եւ}. – \armenian{Վաչեան. Նոր-Դար} 1887, \armenian{էջ} 174-5
\item \armenian{Գեղամեանց Յ. Իմ մանկական յիշողբւթիւններից. Փորձ, Բ. թիւ} 2, \armenian{էջ} 269-296 (\armenian{Ախալքալաքի}). 

 \end{itemize}
\end{itemize}

\section{Subdialects (continued)}

The subdivisions of the Karin dialect are the subdialects of Baberd and Khodorchur. 


\begin{adjarianpage}\label{page:112}\end{adjarianpage}% should be 112

\subsection{Baberd}
There is no separate study on the Baberd dialect. There has likewise not been a simple manuscript published. In the periodical \citeauthor{Byurakn} (1899, page 567), there is a small collection of proverbs from Baberd; but because this bears the literary culture, then it unfortunately cannot fulfill our needs. H. Darbinian (\armenian{Յ. Դարբինեան}; \translatorHD{SEA: /dɑɾpʰinjɑn/, SWA: /tʰɑɾpʰinjɑn/}) in the Arevelk (\armenian{Արեւելք}, number 6693, 6695, 6697, and 6699) has an article about the Baberd with the title `Provincial dialect treasures, \armenian{Գաւառաբարբառին գանձերը}); but this an ordinary list of provincial words.\footnote{Unfortunately, because of limited resources, I couldn't track down the the publication venue because there were multiple journals with the name Arevelk (\armenian{Արեւելք}), and I couldn't track down this manuscript or author.} As a consequence, I am forced to be satisfied with my little familiarity with the dialect, which I gathered in 1984 by visiting Baberd for a day, and also with information that H. Darbinian (\armenian{Յ. Դարբինեան}) gave me in the summer of 1910 during my travel to Istanbul. 

\translatorHD{This subdialect has the following properties:}

\begin{itemize}
 \item The Baberd subdialect knows how to distinguish the three degrees of consonants: voiced aspirated, voiced, and voiceless aspirated. 
 \item To form the indicative present and imperfective, it uses the the postposed formative /kə/ <\armenian{կը}> formative. 
 \item The sounds /u̯o, i̯e/ <\armenian{ո, ե}> are confused with the sounds /o, e/ <\armenian{օ, է}>.
 \item The sound change of Classical /h/ <\armenian{հ}> to /f/ <\armenian{ֆ}> does not exist.
 \item The accusative is always the same as the nominative, and there is no dative case for animate objects. 
 \item A separate innovation in Baberd is the progressive (\armenian{շարունակական}) form of the present and imperfective, which is formed with the formatives /ɡe, eɾ, ənɡe/ <\armenian{գէ, էր, ընգէ}>. 
\end{itemize}

These latter characteristics, especially the use of the formative /eɾ/ <\armenian{էր}>, show to us that the Baberd subdialect forms a middle ring between the Karin and Trabzon dialects. The villages of Baberd are more faithful to the mother dialect, and they are almost the same as the city of Karin. 
\subsection{Khodorchur}
Khodorchur also forms its own separate subdiaelct. As for its position between the Hamshen and Karin dialects, this is still not sufficiently clear to me. Recently, two significantly great volumes were published with the Khodorchur dialect, under the editorship of H. M Hadjian (\armenian{Հ. Մ. Հաճեան}; \translatorHD{SEA: /hɑt͡ʃjɑn/, SWA: /hɑd͡ʒjɑn/}). These are \textit{\armenian{Երգեր, առակներ, հանելուկներ}… \armenian{Խոտրջուր, Տփխիս}, 1904}, and \textit{\armenian{Հին աւանդական հէքիաթներ Խոտրջրոյ, Վիեննա} 1907}.\footnote{\translatorHD{Unfortunately, I haven't been able to track down these manuscripts, so that I could verify their bibliographic metadata. Though I believe I found the author that Adjarian mentions: \url{https://hyw.wikipedia.org/wiki/\armenian{Մատթէոս}_\armenian{Վրդ}._\armenian{Հաճեան}}}} Because the first is written in essentially the literary language, ... 


\begin{adjarianpage}\label{page:113}\end{adjarianpage}% should be 113

... it cannot offer any benefits for studying the dialect; and I still don't have a copy of the second one. Only one well-known characteristic of Khodorchur is known, and that is how the sound /ɾ/ <\armenian{ր}> changes to /j/ <\armenian{յ}> (Table \ref{tab:Karin:subdialect:Khodorchur:r}). 

\begin{table}[H]
 \centering
 \caption{Change from Classical Armenian /ɾ/ <\armenian{ր}> to /j/ <\armenian{յ}> in the Khodorchur subdialect of the Karin dialect}
 \label{tab:Karin:subdialect:Khodorchur:r}
 \begin{tabular}{|l| ll|ll| ll|}
 \hline & \multicolumn{2}{l|}{Classical Armenian} &\multicolumn{2}{l|}{> Karin} & \multicolumn{2}{l|}{cf. SEA} \\ 
`this belly (CA); my belly (SEA)' & pʰoɾ-əs & \armenian{փորս}& pʰoj-əs & \armenian{փօյըս} & pʰoɾ-əs & \armenian{փորս} \\
`person from Khodorchur' & & & χotujd͡ʒʰujt͡sʰi & \armenian{խօտույջՙույցի} & χotəɾd͡ʒuɾt͡sʰi & \armenian{խոտրջուրցի} \\
\hline 
 \end{tabular}
\end{table}


\section{Text samples}

{\sampleoverview}

\subsection{From Akhalkalak}
Adjarian's sample: Taken from \armenian{Ե. Լալայեանի Ջաւախքի բուրմունքէն}, page 44-45



\armenian{Սօղօմօն իմաստունին կնիկը սուտ հիվանդ գՙըլլի հէքիմին սիրէ կը։ Իրէն մարթուն յախան ա կըպչի կը թը հաֆքէրու օսկըռնէրէն ինձի մէ ղօնախանա մ՚ պիտի շինէս, թէպուլնէրէն ա յօրղան դՙօշակ մ՚ սարքէս։}

\armenian{Սօղօմօն իմաստունը կանչէ կը հաֆքէրուն, մօրթէ կ՚ւո օսկըռտանքն ու թէպուլնէրը թօփ կէնէ, օր կնգանը ուզածը հազըրէ։ Աշխարք ինչքան հաֆք կա գՙուգՙա կը, սալթ քօռ բՙուֆը չի գՙա։ Սօղօմօնը անղայ զմրութին ղրկէ կ՚օր գՙըտնի բՙէրէ։ Անղայ զմրութը ա̈զ մ՚ ման գՙուգՙա, անջաղ անջաղ գՙըտնի կը բՙընի մ՚ մէչ, ինչքան կանչէ ճըվա կը՝ չի դՙուս գՙա. ահ կուտա՝ չի ըլլի, խօստում կ՚էնէ ՝ չի ըլլի։ Խիւլասա բՙընին առաջը կայնի սիրուն քարօղ մ՚ խօսի կը, ասիկ գՙէլլէ դՙուս։ Ախըր անդէր քարօզը քար կը ծակէ։ Բՙուֆն օր գՙուս կ՚էլլէ՝ ասիկ քարօղը ծալէ՝ գՙըրկէ կ՚ու տանի կը Սօղօմօն իմաստունին։}

\armenian{Սօղօմօն իմաստունը հէրսօտի կը թը կանչէի կը ինչի՞ չէիր գՙա։ Բՙուբՙբՙուն կ՚եսէ կը. «Զոթք ցքէի կը թը աշխըրքիս մէչ տղամարթն է շատ, թէ կնիկ մարթը։ անդի ուշացա»։}

– \armenian{Է՜, կըսէ Սօղօմօն իմաստունը, իմացա՞ր, վէ՞րն է շատ։ Թը «կնիկ մարթը շատ է»։}

– \armenian{Ի՜նչըղ, կըսէ Սօղօմօն իմաստունը, տղամարթը շատ պիտի ըլլի։}

– \armenian{Ղօրթ է, կըսէ բՙուբՙբՙուն, հըմը, յէս, կընգանը խօսքը անգաջ էնօդին ա կնիկմարթ ցքի. կնիկմաթ չէ՞ անիկ օր կնգանը ձՙէռքը խաղալիք է գՙառէ. կնիկը մինդրին տակը յուխա (չոր խմորեղէն մը) լզէր սուտ հիվանդ է ձՙէվացէ ու հէքիմին սիրէ} ... 

\begin{adjarianpage}\label{page:114}\end{adjarianpage}% should be 114

... \armenian{կը, մարթուն ա չարչըրէլու հըմար հաֆքէրուն օսկըռտանքէն ղօնախանա գՙուզէ։ Ի՜նչքան հաֆք պիտի գՙընտէս ջՙարդՙէս օր անօնց օսկըռտանքէն ղօնախանա կտյնէցընէս։}

– \armenian{Խէլացի ըսաց, կըսէ Սօղօմօն խմաստունը ինքնիրէն, յէս ա կնիկմարթ էմ օր կընգանս խօսքօվը աշխըրքի հաֆքէրուն արունքը մտա։ Արթղ բՙիտտուն հաֆքէրուն բՙաց թօղնէ կը, օրը իրէք ղուշ ա բՙուբՙբՙուին կապէ կը։ Տէյ մ՚օր հիմի ա օրը իրէք ղուշ իրէնք իրէնք գՙուգՙան բՙուբՙբՙուին առաչը կայնին կը։ Բՙուբՙբՙուն էրկուսը կուտէ, մէկը Աստըծու սիրուն ազատէ կը։}

\subsection{From Basean}

Adjarian's source: See \citeauthor{AzgagrakanHandes}, volume 6 (\armenian{Զ}.), page 383, and so on


\armenian{Կաղաչեմ ինձի լսէ},

\armenian{Արի յ}̵\armenian{արտըսունքըս սրբՙէ}.

\armenian{Դՙանակըմ դՙու ինձի տու},

\armenian{յ}̵\armenian{էտեվ ըսէ մի՛ մօրթէ։}

~ 

\armenian{Սարեր, ձՙօրեր ու ջՙըրեր},

\armenian{Մարմանդ վազօղ ախբՙուրներ},

\armenian{Մէկ վեր կէցէք ու յ}̵\armenian{իմացէք}

\armenian{Տէսէք թէ վէ՞վ է էկեր։}

~ 



\armenian{Գՙէլը օչխըրին էկավ},

\armenian{Զարկեց գՙեր դՙառին տարավ}.

\armenian{Հայի տղէն ինչղ չը լա։}

\armenian{Յարը դիւշմանը տարավ։}


\armenian{Կօկոմս թօռմած մնաց},

\armenian{Սիրտըս կրակած մնաց},

\armenian{Ի՜նչ էնիմ յես ապրելը՝}

\armenian{յ}̵\armenian{աչքերըս լուս չմնաց։}


~ 


\armenian{Սեվ է յ}̵\armenian{աչքերըդՙ, կռունգ},

\armenian{Ճէրմակ է սիրտըդՙ, կռունգ},


\armenian{Ջՙուխտ գՙացիր մէնակ գՙուգՙոաս}, 

\armenian{Ո՞ւր է յ}̵\armenian{ընգերըդՙ, կռունգ։}

\begin{adjarianpage}\label{page:115}\end{adjarianpage}% should be 115

\armenian{Բաղի մէչը վարթ գՙըլլի},

\armenian{Բաղի շունը սարթ գՙըլլի},

\armenian{Շան ախճիկ, ուսուլ խօսէ},

\armenian{Տալտա տեղ է, մարթ գՙըլլի։}


\armenian{Մէրըս ինձի բՙէրեր է},

\armenian{Նխշուն բՙալուլ էրէր է},

\armenian{Նխշուն բՙալուլ մէռնէի},

\armenian{Մօրըս մտքէն յ}̵\armenian{էլլէի։}

~ 


\armenian{Ախճի, դՙու յես մէղավոր},

\armenian{Քէզի գՙուգՙան ուզավոր},

\armenian{Չէղնի էրթաս հէռու տեղ},

\armenian{Պագՙվիս կէղնիս յ}̵\armenian{ըռազիլ։}





\chapter{Mush}
\section{Background}

\begin{adjarianpage}\label{page:116}\end{adjarianpage}% should be 116

The Mush dialect is spread over the west side of the Van sea. Its center is the city of Mush. From the north, it spreads until Hınıs and Alashkert, from the south to Paghesh, from the east it reaches Moks from one side and Diyadin from the other side, from the west are Lice, Chapaghjur. Thus the Mush dialect is spoken in Mush, Sason, Paghesh, Hizan, Khlat, Arjesh, Bulanık, Manazkert, Hınıs, and Alashkert. During the last two Russo-Turkish wars, two large migrations happened from Mush and Alashkert, establishing settlements in the Yerevan province, at Aparan (near Alexandropol (Gyumri)) and the south sides of New Bayazet, on the shores of Lake Sevan. In the latter region, there are today 21 Armenian villages which speak the Mush or Alashkert dialect. These villages in order are Yeranos, Adamxan, Dzoragegh, Tsakkar, Gölköy, Tazakend, Lower and Upper Adyaman, Upper and Lower Karanlug, Avdalaghalu, Alikrykh, Zolakhach, Upper and Lower Gyuzeldara, Upper and Lower Kyolaghran, Lower Aluchalu, Gedakbulag, Zaghalu and Tyuskyulyu. A group of migrants from Hınıs also went to Akhalkalak, and they established the villages of Toria, Ujmana and Eshtia around the area. These also speak the dialect to this day.

\translatorHD{Note that more data is provided in \S\ref{section:Mush:text:migrationNote}.}
\section{Phonology}
\subsection{Vowel inventory and sound changes}

The Mush dialect does not have a rich phonetic system with respect to vowels. The vowels /æ, œ, ʏ/ <\armenian{ա}̈, \armenian{էօ, իւ}> are absent; and in this way we can form a characteristic border to distinguish the Mush dialect from the Van dialect, which has these vowels. 

The sounds <\armenian{ե,ո}> in Mush have a certain rich diphthongal pronunciation,\footnote{\translatorHD{The prose is vague, but I think he means that this dialect has the diphthongs /i̯e, u̯o/ <\armenian{ե, ո}>. }} and they originate from the Classical Armenian stressed sounds /e, o/ <\armenian{ե,ո}>. Without stress, these sounds became /e,o/ <\armenian{է,օ}>. ... 


\begin{adjarianpage}\label{page:117}\end{adjarianpage}% should be 117

... Word-initially in monosyllabic words, they turn to /je, vo/ <\armenian{յէ, վօ}>. At the beginning of polysyllabic words, they turn to /e,o/ <\armenian{է,օ}>. The Classical Armenian sounds /u/ <\armenian{ու}> and diphthongs /ɑi̯, oi̯. iu̯/ <\armenian{այ, ոյ, իւ}> are rendered as simple vowels: /u/ > /u/ (\armenian{ու} > \armenian{ու}), /ɑi̯/ > /e/ (\armenian{այ} > \armenian{է}), /oi̯/ > /u/ (\armenian{ոյ} > \armenian{ու}), /iu̯/ > /u/ (\armenian{իւ} > \armenian{ու}). 

\subsection{Consonant inventory and sound changes}

\subsubsection{Voicing changes and voiced aspirates}

In contrast, the Mush dialect is rich in consonants. Like the Karin dialect, here we find a group of voiced aspirated consonants, with which the dialect has four series of plosive consonants (Table \ref{tab:Mush:phono:segment:cons:voice}). 


\begin{table}[H]
\caption{Voicing contrasts in the Mush dialect}\label{tab:Mush:phono:segment:cons:voice}\centering 
\begin{tabular}{|ll|ll|ll|ll|}
\hline \multicolumn{2}{|l|}{Voiced} & \multicolumn{2}{l|}{Voiced aspirated} & \multicolumn{2}{l|}{Voiceless unaspirated} & \multicolumn{2}{l|}{Voiceless aspirate} \\
\hline 
b &\armenian{բ} & bʰ& \armenian{բՙ} & p & \armenian{պ} & pʰ& \armenian{փ} \\
 ɡ &\armenian{գ} & ɡʰ & \armenian{գՙ} & k & \armenian{կ} & kʰ & \armenian{ք} \\
d& \armenian{դ} & dʰ & \armenian{դՙ} & t & \armenian{տ} & tʰ & \armenian{թ} \\
d͡z& \armenian{ձ} & d͡zʰ& \armenian{ձՙ} & t͡s & \armenian{ծ} & t͡sʰ & \armenian{ց} \\
 d͡ʒ &\armenian{ջ} & d͡ʒʰ & \armenian{ջՙ} & t͡ʃ & \armenian{ճ} & t͡ʃʰ & \armenian{չ} 
\\ \hline 
\end{tabular}
\end{table}

Word-initially, the voiced sounds of Old Armenian become voiced aspirates. Word-medially, they voiceless or stay voiced. After the nasal /n/ <\armenian{ն}>, it is always the voiced sounds that are found. The voiceless unaspirated and voiceless aspirated groups are generally unchanged. But there are exceptions where the voiceless aspirates became voiceless unaspirated (Table \ref{tab:Mush:phono:cons:voice:deasp}). Discussing such forms requires a detailed study. 

\begin{table}[H]
 \centering
 \caption{Deaspiration from Classical Armenian voiced aspirates in the Mush dialect}
 \label{tab:Mush:phono:cons:voice:deasp}
 \begin{tabular}{|l| ll|ll| ll|}
 \hline & \multicolumn{2}{l|}{Classical Armenian} &\multicolumn{2}{l|}{> Mush} & \multicolumn{2}{l|}{cf. SEA} \\ 
 \armenian{՝}eye' & ɑt͡ʃʰ-(ə?)kʰ (-{\pl}) & \armenian{աչք} & ɑt͡ʃk & \armenian{աճկ} & ɑt͡ʃʰkʰ & \armenian{աչք} \\
 \armenian{՝}Armenianness' & hɑ{ju}tʰiu̯n & \armenian{հայութիւն} & hɑjuten & \armenian{հայուտեն} & hɑjutʰjun & \armenian{հայություն} \\
\hline 
 \end{tabular}
\end{table}

\subsubsection{Glottal fricatives /h, ɦ/ <\armenian{հ, յ}̵>}


Like the Karin dialect, the Mush dialect also has two types of glottal sounds (\armenian{հագագ}), which are /ɦ/ <\armenian{յ}̵> and /h/ <\armenian{հ}>. The use of these sounds is the same as in Karin. But the Mush dialect has the habit of adding the sound /ɦ/ <\armenian{յ}̵> to many vowel-initial words (Table \ref{tab:Mush:phono:cons:voice:initialH}). 


\begin{table}[H]
 \centering
 \caption{Insertion of word-initial voiced glottal fricative /ɦ/ <\armenian{յ}̵> in the Mush dialect}
 \label{tab:Mush:phono:cons:voice:initialH}
 \begin{tabular}{|l| ll|ll| ll|}
 \hline & \multicolumn{2}{l|}{Classical Armenian} &\multicolumn{2}{l|}{> Mush} & \multicolumn{2}{l|}{cf. SEA} \\ 
 \armenian{՝}cheap' & ɑʒɑn & \armenian{աժան} & ɦeʒɑn & \armenian{յ}̵\armenian{էժան} & ɑʒɑn & \armenian{աժան} \\
`stall' &ɑχor & \armenian{ախոռ} & ɦɑχor & \armenian{յ}̵\armenian{ախոռ} &ɑχor & \armenian{ախոռ} \\ 
 `fountain' & ɑɬbiu̯ɾ & \armenian{աղբիւր} & ɦɑχbuɾ & \armenian{յ}̵\armenian{ախբուր} & ɑχpjuɾ & \armenian{աղբյուր} \\ 
 `all' & ɑmenɑi̯n & \armenian{ամենայն} & ɦəmen & \armenian{յ}̵\armenian{ըմէն} & ɑmenɑjn & \armenian{ամենայն} \\ 
 `durable' & ɑmuɾ & \armenian{ամուր} & ɦɑmbuɾ & \armenian{յ}̵\armenian{ամբուր} & ɑmuɾ & \armenian{ամուր} \\ 
 `late' & ɑnɑɡɑn & \armenian{անագան} & ɦɑnɡɑn & \armenian{յ}̵\armenian{անգան} & ɑnɑɡɑn & \armenian{անագան} \\ 
\hline 
 \end{tabular}
\end{table}

\subsubsection{Using back fricatives to distinguish Mush from Van}

Because the Van dialect does not have voiced aspirates or the glottal sound /ɦ/ <\armenian{յ}̵ >, we are provided with a second significant method to distinguish these two dialects. 

The Classical Armenian sound /h/ <\armenian{հ}>, has two types of forms in the Mush dialect. We find the sound /h/ <\armenian{հ}> in Mush, Sason, Bulanık, Alashkert, Aparan and six villages on the shores of Lake Sevan; while in the other areas (Paghesh, Khlat, Arjesh and Artske), the sound has changed to /χ/ <\armenian{խ}>, as in the Van dialect. The last group ... 

\begin{adjarianpage}\label{page:118}\end{adjarianpage}% should be 118

... is distinguished from the Mush dialect in a few points. For example, the copular form has the form /ɑ/ <\armenian{ա}> (\ref{sent:Mush:phono:cons:participle:a}, \ref{sent:Mush:phono:cons:participle:b}), and there exists the sound /ɡʲ/ <\armenian{գյ}> (\ref{sent:Mush:phono:cons:voicepal}), with which it gets closer to the Van dialect. 

\begin{exe}
 \ex Mush (implied to be the Paghesh subdialect) 
 \begin{xlist}
 \ex \gll t͡ʃʰ-e ɡɑt͡sʰ-eɾ ɑ \\
 {\neggloss}-{\aux} go-{\perfcvb} {\aux} \\
 \trans `He has not gone.'\label{sent:Mush:phono:cons:participle:a} \\
 \armenian{չէ գացեր ա}
 \ex \gll t͡ʃʰ-e beɾ-eɾ ɑ \\
 {\neggloss}-{\aux} bring-{\perfcvb} {\aux} \\
 \trans `He has not brought.' \label{sent:Mush:phono:cons:participle:b}\\
 \ex \gll ku-ɡʲ-ɑ-$\emptyset$ \\
 {\ind}-come-{\thgloss}-3{\sg}\\
 \trans `He comes.' \label{sent:Mush:phono:cons:voicepal}\\
 \armenian{կուգյա}
 \end{xlist}
\end{exe}

It follows from here that the Mush dialect contains the subdialect of Paghesh, which contains also Khlat, Arjesh, and Artske. Unfortunately, for the materials or excerpts that we have at our disposal, these materials don't have the required scientific accuracy that we need in order to establish the characteristics of this subdialect. From the neighboring villages of the New Bayazet region, the most that we got was only the difference in the sound /χ/ <\armenian{խ}>; while for all the remaining points, the two branches are in agreement with each other. In the region of New Bayazet, the villages that belong to the Mush branch and that have the /h/ <\armenian{հ}> sound are Yeranos, Adamxan, Dzoragegh, Gölköy, Lower and Upper Adyaman; while the villages that have /χ/ <\armenian{խ}> are Tsakkar, Upper Karanlug, Avdalaghalu, Alikrykh, Zolakhach, Lower Gyuzeldara, Upper Gyuzeldara, Upper and Lower Kyolaghran, Lower Aluchalu, Gedakbulag and Zaghalu. We talk about the others much later. 

\section{Morphology}
\subsection{Noun inflection or declension}
The grammar of the Mush dialect has some interesting archaisms.

\subsubsection{Classical accusative prefix /z/ <\armenian{զ}>}
The accusative is formed similarly to Old Armenian, by using the prefix /z/ <\armenian{զ}> (\ref{sent:Mush:morpho:noun:accz}). 

\begin{exe}
 \ex \label{sent:Mush:morpho:noun:accz} \begin{xlist}
 \ex Mush 
 \gll əz hɑt͡sʰ, əz məɾ tun \\ 
 {\acc} bread, {\acc} our house \\
 \trans `bread ({\acc}), our house ({\acc})'\\
 \armenian{ըզ հաց, ըզ մըր տուն} 
 \ex cf. Classical Armenian 
 \gll əz-hɑt͡sʰ, əz-meɾ tun \\ 
 {\acc} bread, {\acc}-our house \\
 \trans `bread ({\acc}), our house ({\acc})' \\
 \armenian{զհաց, զմեր տուն} 
 \end{xlist}
 
\end{exe}
\subsubsection{Classical accusative prefix /i,j/ <\armenian{ի,յ}>}
The Classical accusative prefixal formatives /i,j/ <\armenian{ի,յ}> are still in use (\ref{sent:Mush:morpho:noun:accij}). 

\begin{exe}
 \ex Mush \label{sent:Mush:morpho:noun:accij}
 \begin{xlist}
 \ex \gll t͡ʃʰuɾ oɾik-n i mɑh-u-n \\ 
 water? day?-{\defgloss} {\acc} death-{\dat}-{\defgloss} \\
 \trans I don't understand this sentence, and I had the guess most of the words without surety.\\
 \armenian{չուր օրիկն ի մահուն}
 \ex \gll k-eɾtʰ-ɑ-m ɦ-ɑɾt \\ 
 {\ind}-go-{\thgloss}-1{\sg} {\acc}-? \\
 \trans I don't understand this sentence, and I had the guess most of the words without surety.\\
 \armenian{կէրթամ յ}̵\armenian{արտ}
 \end{xlist}
\end{exe} 

\subsubsection{Genitive possession without definite marking} 

After possessive adjectives, the noun does not have an article. This is like in Classical Armenian and in all the European languages, except for Italian. Modern Civil Armenian uses the definite article (\ref{sent:Mush:morpho:noun:possDef:sea}),\footnote{\translatorHD{Adjarian originally used <\armenian{քու}> /kʰu/ instead of <\armenian{քո}> /kʰo/ for the SEA 2PL possessor, but the form /kʰu/ is for SWA, not SEA. }} cf. Italian (\ref{sent:Mush:morpho:noun:possDef:it}),\footnote{\translatorHD{Adjarian used the word <pano> for Italian `bread', but this seems to be a typo for <pane>. }} but French (\ref{sent:Mush:morpho:noun:possDef:fr}). The Mush dialect also says (\ref{sent:Mush:morpho:noun:possDef:mush}), like Classical (\ref{sent:Mush:morpho:noun:possDef:ca}). 

 \begin{exe}
 \ex \begin{xlist}
 \ex SEA \label{sent:Mush:morpho:noun:possDef:sea}\gll
 meɾ hɑt͡sʰ-ə, d͡zeɾ tun-ə, im bɑɾekɑm-ə, kʰo ʒoʁovuɾtʰ-ə \\
 our bread-{\defgloss}, your.{\pl} house-{\defgloss}, my friend-{\defgloss}, your.{\sg} people-{\defgloss} \\
 \trans `our bread, your house, my friend, your people' \\ 
 \armenian{մեր հացը, ձեր տունը, իմ բարեկամը, քու ժողովուրդը}
 \ex Italian\label{sent:Mush:morpho:noun:possDef:it} \gll
 il nostro pane, la vostra casa, il mio amico, il tuo popolo
 \\ the our bread, the your house, the my friend, the your people \\
 \trans `our bread, your house, my friend, your people'
 \ex French\label{sent:Mush:morpho:noun:possDef:fr} \gll 
 notre pain, votre maison, mon ami, ton peuple \\
 our bread, your house, my friend, your people\\
 \trans `our bread, our house' \\
 \armenian{հաց մեր, տուն մեր} 
 \ex Mush \label{sent:Mush:morpho:noun:possDef:mush} \gll
 məɾ hɑt͡sʰ, məɾ tun \\
 our bread, our house \\
 \trans `our bread, our house' \\
 \armenian{մըր հաց, մըր տուն} 
 \ex Classical Armenian \label{sent:Mush:morpho:noun:possDef:ca}\gll
 hɑt͡sʰ meɾ, tun meɾ \\
 bread our, house our\\
 \end{xlist}
 \end{exe}

 \translatorHD{To clarify, Classical Armenian and most Western European langauges do not use a definite article for possessed nouns. However, Italian and SEA do use a definite article for possessed nouns. Mush violates this Modern Armenian tendency, and it does not use an article for possessed nouns. }

 \subsubsection{Post-nominal possesors} 
 
Possessive adjectives can be placed after the noun. This is not found in any dialect. But like in Classical Armenian, the people of Mush say the sentences in (\ref{sent:Mush:morpho:noun:postnomPoss}).

\begin{exe}
 \ex Mush \label{sent:Mush:morpho:noun:postnomPoss}
 \begin{xlist}
 \ex \gll d͡zʰern-e mzi \\
 hand-{\abl} our \\
 \trans `from our hand' \\
\armenian{ձՙէռնէ մզի} 
 \ex \gll bʰern-e kʰzi \\
 mouth-{\abl} your.{\sg} \\
 \trans `from your.{\sg} mouth' \\
 \armenian{բՙէրնէ քզի}
 \ex \gll siɾt d͡zʰəzi uɾɑɾ t͡ʃʰ-uz-ɑ-$\emptyset$ \\
 heart your.{\pl} each.other {\neggloss}-want-{\thgloss}-3{\sg} \\
 \trans `your hearts don't love/want each other' \\
 \armenian{սիրտ ձՙըզի ուրար չուզա}
 \ex \gll d͡zʰern-e ənd͡zi \\
 hand-{\abl} my \\
 \trans `from my hand' \\
 \armenian{ձՙէռնէ ընձի} 
 \ex \gll d͡zʰern-e d͡zʰzi \\
 hand-{\abl} your.{\pl} \\
 \trans `from your.{\pl} hand' \\
\armenian{ձՙէրնէ ձՙզի} 
 \ex \gll lez uɾɑnt͡sʰ, buk uɾɑnt͡sʰ \\
 language their, throat their \\ 
 \trans `their language, their throat' \\
 \armenian{լէզ ուրանց, բուկ ուրանց} 

 \end{xlist}
\end{exe}

\begin{adjarianpage}\label{page:119}\end{adjarianpage}% should be 119


\subsubsection{Retention of prepositions} 

For many prepositions that have become postpositions in other dialects, here they have kept their original position, sometimes as a formative (\ref{sent:Mush:morpho:noun:prepositionsNotPost}). 

\begin{exe}
 \ex\label{sent:Mush:morpho:noun:prepositionsNotPost} \begin{xlist}
 \ex `on our house' 
 \begin{xlist}
 \ex Mush \gll 
 vəɾ məɾ tɑn \\
 on our house.{\gen} \\
 \armenian{վըր մըր տան}
 \ex Classical Armenian \gll 
 i veɾɑi̯ meɾoi̯ tɑn \\
 to on our house.{\gen} \\
 \armenian{ի վերայ մերոյ տան}
 \ex SEA \gll 
 meɾ tɑn vəɾɑ \\
 our house.{\gen} on \\
 \armenian{մեր տան վրա}
 \end{xlist}
 \ex `in our house' 
 \begin{xlist}
 \ex Mush \gll 
 mət͡ʃʰ məɾ tɑn \\
 in our house.{\gen} \\
 \armenian{մըչ մըր տան}
 \ex Classical Armenian \gll 
 i mid͡ʒi meɾ tɑn \\
 to in our house.{\gen} \\
 \armenian{ի միջի մեր տան}
 \ex SEA \gll 
 meɾ tɑn met͡ʃʰ \\
 our house.{\gen} on \\
 \armenian{մեր տան մեջ}
 \end{xlist}
 \ex `near me' \footnote{\translatorHD{In Adjarian's original prose, it's not clear if he also proposes that Classical Armenian had the Mush-like order <\armenian{մաւտ ինձ}> /mɑu̯t ind͡z/. }}
 \begin{xlist}
 \ex Mush \gll 
 mət ənd͡zi \\
 near me.{\dat} \\
 \armenian{մըտ ընձի}
 
 \armenian{ի միջի մեր տան}
 \ex SEA \gll 
 ind͡z mot \\
 me.{\dat} near\\ 
 \armenian{ինձ մոտ}
 \end{xlist}
 \ex `how many years before you?'
 \begin{xlist}
 \ex Mush \gll 
 kʰɑni tɑri ɑrɑt͡ʃʰ kʰzne \\
 how.many year before you.{\sg}.{\abl} \\
 \armenian{քանի՞ տարի առաչ քզնէ}
 \ex SWA \gll 
 kʰezme kʰɑni dɑɾi ɑɾɑt͡ʃʰ \\
 you.{\sg}.{\abl} how.many year before \\ 
 \armenian{քեզմէ քանի՞ տարի առաջ}
 \end{xlist}
 \end{xlist}
\end{exe}

\subsubsection{Vocative case}

What is especially interesting is the vocative case; the vocative case is the ending the ending /o/ <\armenian{օ}>, and the people of Mush use it especially for proper names.


\begin{table}[H]
 \centering
 \caption{Vocative forms in the Mush dialect}
 \label{tab:Mush:morpho:noun:vocative}
 \begin{tabular}{|l|ll|ll|}
 \hline & \multicolumn{2}{l|}{Mush }& \multicolumn{2}{l|}{cf. SEA }
 \\
 \armenian{՝}Oh Harutyun!' & ɦɑɾo & \armenian{յ}̵\armenian{արօ} & ov hɑɾutʰjun &\armenian{ո՜վ Հարություն} \\
 \armenian{՝}Oh Hakop!' & ɦɑko & \armenian{յ}̵\armenian{ակօ} & ov hɑkopʰ &\armenian{ո՜վ Հակոբ} \\ 
 \armenian{՝}Oh Mariam!' & mɑɾo & \armenian{Մարօ} & ov mɑɾjɑm &\armenian{ո՜վ Մարիամ} \\ 
 \armenian{՝}Oh boy!' & l\'ɑo & \armenian{լա՛օ} & ov lɑt͡ʃ &\armenian{ո՜վ լաճ} \\ 
 \armenian{՝}Oh mom!' & m\'ɑmo & \armenian{մա՛մօ} & ov mɑm &\armenian{ո՜վ մամ} \\ 
 \armenian{՝}Oh sister!' & kʰuɾo & \armenian{քուրօ} & ov mɑm &\armenian{ո՜վ քույր} \\ 
 \hline
 \end{tabular}
 
\end{table}

\subsection{Pronoun inflection or declension}

Among pronouns, the following forms are worth noting (Table \ref{tab:Mush:morphology:pronoun:sample}).

\begin{table}[H]
 \centering
 \caption{Sample of pronouns in the Mush dialect}
 \label{tab:Mush:morphology:pronoun:sample}
 \begin{tabular}{|l ll| }
\hline 
personal 2SG {\dat} `to you' &kʰəzi & \armenian{քըզի} \\
personal 1PL {\acc} `us' & əzməzi, əzmi & \armenian{ըզմըզի, ըզմի} \\
personal 1PL {\gen} `our' &məɾ & \armenian{մըր} \\
personal 1PL {\dat} `to us' &məzi & \armenian{մըզի} \\
personal 1PL {\abl} `from us' & məzne & \armenian{մըզնէ} \\
personal 2PL {\acc} `you' & əzkʰəzi, əzkʰi & \armenian{ըզքըզի, ըզքի} \\
personal 2PL {\gen} `your' &d͡zʰəɾ & \armenian{ձՙըր} \\
personal 2PL {\dat} `to you' &d͡zʰəzi & \armenian{ձՙըզի} \\
personal 2PL {\abl} `from you' & d͡zʰəzne & \armenian{ձՙըզնէ} \\
demonstrative proximal {\sg} {\nom} `this' & isɑ & \armenian{իսա} \\
demonstrative proximal {\sg} {\ins} `with this' & estov & \armenian{էստով} \\
demonstrative medial {\sg} {\nom} `that' & idɑ & \armenian{իդա} \\
demonstrative distal {\sg} {\nom} `that yonder' & inɑ & \armenian{ինա} \\
demonstrative distal {\sg} {\ins} `with that yonder' & endov & \armenian{էնդով} \\
\hline 
 \end{tabular}
\end{table}


Finally, there are some very common forms (Table \ref{tab:Mush:morphology:pronoun:other}).

\begin{table}[H]
 \centering
 \caption{Sample of other pronouns in the Mush dialect}
 \label{tab:Mush:morphology:pronoun:other}
 \begin{tabular}{|l ll| }
\hline 
`why' &ɦoɾi & \armenian{յ}̵\armenian{օրի՞} \\
`other' &le & \armenian{լէ} \\
`now' &mkɑ & \armenian{մկա} \\
\hline 
 \end{tabular}
\end{table}



After the possessive suffixes, the formative /i/ <\armenian{ի}> is added very often (\ref{sent:Mush:morpho:pron:repI}).

\begin{exe}
 \ex Mush \label{sent:Mush:morpho:pron:repI} \gll 
 d͡zʰi-\'eɾ-d-i lɑv i-n, vɾ\'e-s-i k-iɡʰ-ɑ-s
 \\
 horse-{\pl}-{\possSsg}-? good {\aux}-3{\pl}, on?-{\possFsg}-? {\ind}-come-{\thgloss}-2{\sg}\\
 \trans Adjarian did not provide a SEA translation, so the segmentation and glosses are entirely my (unsure) guesses: `Your horses are good, you come onto me.'\\
 \armenian{ձՙիէ՛րդի լավ ին, վրէ՛սի կիգՙաս}
\end{exe}

\subsection{Verb inflection or conjugation}

\subsubsection{Changes in theme vowels and tense-agreement marking}
In the conjugation of verbs, the Classical Armenian sounds /e, ē/ <\armenian{ե, է}> have changed to /i/ <\armenian{ի}>, in the present, subjunctive present, future, and present perfect (Table \ref{tab:Mush:morpho:verb:themeI}). 


\begin{table}[H]
 \centering
 \caption{Theme vowel change from Classical /e/ <\armenian{ե}> to /i/ <\armenian{ի}> in the Mush dialect}
 \label{tab:Mush:morpho:verb:themeI}
 \begin{tabular}{|l| ll| ll|}
 \hline & \multicolumn{2}{l|}{Mush} & \multicolumn{2}{l|}{cf. SWA} \\ \hline
 indc. present 1SG `I like' & kə siɾ-i-m & \armenian{կը սիրիմ} & ɡə siɾ-e-m & \armenian{կը սիրեմ} \\
 indc. present 1SG `he likes' & kə siɾ-i-$\emptyset$ & \armenian{կը սիրի} & ɡə siɾ-e-$\emptyset$ & \armenian{կը սիրէ} \\
 indc. present 3PL `they like' & kə siɾ-i-n & \armenian{կը սիրին} & ɡə siɾ-e-$n$ & \armenian{կը սիրեն} \\
 & \multicolumn{2}{l|}{{\ind} $\sqrt{}$-{\thgloss}-{\agr}} & \multicolumn{2}{l|}{{\ind} $\sqrt{}$-{\thgloss}-{\agr}} \\
 \hline indc. present 1SG `I see' & kə tes-n-i-m & \armenian{կը տէսնիմ} & ɡə des-n-e-m & \armenian{կը տեսնեմ} \\
 & \multicolumn{2}{l|}{{\ind} $\sqrt{}$-{\vx}-{\thgloss}-{\agr}} & \multicolumn{2}{l|}{{\ind} $\sqrt{}$-{\vx}-{\thgloss}-{\agr}} \\
 \hline subj. present 1SG `I say' & ɑs-i-m & \armenian{ասիմ} & əs-e-m & \armenian{ըսեմ} \\
 subj. present 1SG `I eat' & ut-i-m & \armenian{ուտիմ} & ud-e-m & \armenian{ուտեմ} \\
 & \multicolumn{2}{l|}{$\sqrt{}$-{\thgloss}-{\agr}} & \multicolumn{2}{l|}{$\sqrt{}$-{\thgloss}-{\agr}} \\
 \hline future 1SG `I will like' & piti siɾ-i-m & \armenian{պիտի սիրիմ} & bidi siɾ-e-m & \armenian{պիտի սիրեմ} \\
 future 1SG `I will bring' & piti bʰeɾ-i-m & \armenian{պիտի բՙէրիմ} & bidi pʰeɾ-e-m & \armenian{պիտի բերեմ} \\
 & \multicolumn{2}{l|}{{\fut} $\sqrt{}$-{\thgloss}-{\agr}} & \multicolumn{2}{l|}{{\fut} $\sqrt{}$-{\thgloss}-{\agr}} \\
\hline present perfect 1SG `I have seen' & tes-i̯eɾ i-m & \armenian{տէսեր իմ} & des-ɑd͡z e-m & \armenian{տեսած եմ} \\
 present perfect 1PL `we have seen' & tes-i̯eɾ i-nkʰ & \armenian{տէսեր ինք} & des-ɑd͡z e-ŋkʰ & \armenian{տեսած ենք} \\
 present perfect 2PL `you have seen' & tes-i̯eɾ i-kʰ & \armenian{տէսեր իք} & des-ɑd͡z e-kʰ & \armenian{տեսած եք} \\
 & \multicolumn{2}{l|}{$\sqrt{}$-{\perfcvb} {\aux}-{\agr}} & \multicolumn{2}{l|}{$\sqrt{}$-{\rptcp} {\aux}-{\agr}} \\
\hline 
 \end{tabular}
\end{table}


 
In the past imperfective, the Classical sound /ē/ <\armenian{է}>, and in some places the sound /i/ <\armenian{ի}>, has been lost (Table \ref{tab:Mush:morpho:verb:pastI}).\footnote{\translatorHD{For the Mush form /kə ɡʰ-i-nkʰ/ <\armenian{կը գՙինք}> `wee were coming', Adjarian reconstructs this from a hypothetical form */kə ɡ-e-i-nkʰ/ *<\armenian{կը գէինք}>.}}


\begin{table}[H]
 \centering
 \caption{Merger of theme vowel /e/ and past marker /i/ the indicative past imperfective in the Mush dialect}
 \label{tab:Mush:morpho:verb:pastI}
 \begin{tabular}{|l| ll| ll|}
 \hline & \multicolumn{2}{l|}{Mush} & \multicolumn{2}{l|}{cf. SWA} \\ \hline
 1SG `I was wanting' & k-uz-i-$\emptyset$ & \armenian{կուզի} & ɡ-uz-ej-i-$\emptyset$ & \armenian{կ՚ուզէի} \\
 1PL `We were coming' & kə ɡʰ-i-nkʰ & \armenian{կը գՙինք} & ɡu kʰ-ɑj-i-ŋkʰ & \armenian{կու գայինք} \\
 & \multicolumn{2}{l|}{{\ind}-$\sqrt{}$-{\thgloss}.{\pst}-{\agr}} & \multicolumn{2}{l|}{{\ind}-$\sqrt{}$-{\thgloss}-{\pst}-{\agr}} \\
\hline 
 \end{tabular}
\end{table}


But because of this change, the present 3SG and imperfective 1SG would have been identical; so to not confuse these words, the plural is used instead of the singular (Table \ref{tab:Mush:morpho:verb:1pl}). \translatorHD{In other words, whereas the suffix /-nkʰ/ is used to mark the only the 1PL in SEA/SWA, this marker is instead used for the both the 1PL and 1SG. }


\begin{table}[H]
 \centering
 \caption{Use of 1PL markers for the past imperfective 1SG in the Mush dialect}
 \label{tab:Mush:morpho:verb:1pl}
 \begin{tabular}{|l| ll| ll|}
 \hline & \multicolumn{2}{l|}{Mush} & \multicolumn{2}{l|}{cf. SWA} \\ \hline
 `I was wanting' & ji̯es k-uz-e-nkʰ & \armenian{յես կուզէնք} & jes ɡ-uz-ej-i-$\emptyset$ & \armenian{ես կ՚ուզէի} \\
 & \multicolumn{2}{l|}{I {\ind}-want-{\thgloss}.{\pst}-1{\sg}} & \multicolumn{2}{l|}{I {\ind}-want-{\thgloss}-{\pst}-1{\sg}} \\
\hline 
`we were wanting' & mənkʰ k-uz-e-nkʰ & \armenian{մընք կուզէնք} & meŋkʰ ɡ-uz-ej-i-ŋkʰ & \armenian{մենք կ՚ուզէինք} \\
 & \multicolumn{2}{l|}{we {\ind}-want-{\thgloss}.{\pst}-1{\pl}} & \multicolumn{2}{l|}{we {\ind}-want-{\thgloss}-{\pst}-1{\pl}} \\
\hline 
 \end{tabular}
\end{table}


In the others, there is a simple distinction between how the vowel of the present is /i/ <\armenian{ի}>, while the vowel of the imperfective is /e/ <\armenian{է}> (Table \ref{tab:Mush:morpho:verb:pastpresEI}). \translatorHD{I would interpret these facts as stating that the theme vowel is /i/ in the present, while the theme vowel and the past marker are fused as /e/ in the past imperfective. Similarly, the auxiliary is /i/ in the present, while /e/ in the past. }


\begin{table}[H]
 \centering
 \caption{Contrast between the present theme vowel /i/ and the past theme vowel /e/ in the Mush dialect}
 \label{tab:Mush:morpho:verb:pastpresEI}
 \begin{tabular}{|l| ll| ll|}
 \hline & \multicolumn{2}{l|}{Mush} & \multicolumn{2}{l|}{cf. SWA} \\ \hline
 `(we) are' & i-nkʰ & \armenian{ինք} & e-ŋkʰ & \armenian{ենք} \\
 & \multicolumn{2}{l|}{{\aux}-1{\pl}} & \multicolumn{2}{l|}{{\aux}-1{\pl}} \\
\hline 
 `(we) were' & e-nkʰ & \armenian{էնք} & ej-i-ŋkʰ & \armenian{էինք} \\
 & \multicolumn{2}{l|}{{\aux}.{\pst}-1{\pl}} & \multicolumn{2}{l|}{{\aux}-{\pst}-1{\pl}} \\
 \hline 
 `they strike' & kə t͡set͡s-i-n & \armenian{կը ծէծին} & ɡə d͡zed͡z-e-n & \armenian{կը ծեծեն} \\
 & \multicolumn{2}{l|}{{\ind} strike-{\thgloss}-3{\pl}} & \multicolumn{2}{l|}{{\ind} strike-{\thgloss}-3{\pl}} \\
\hline 
 `they were striking' & kə t͡set͡s-e-n & \armenian{կը ծէծէն} & ɡə d͡zed͡z-ej-i-n & \armenian{կը ծեծէին} \\
 & \multicolumn{2}{l|}{{\ind} strike-{\thgloss}.{\pst}-3{\pl}} & \multicolumn{2}{l|}{{\ind} strike-{\thgloss}-{\pst}-3{\pl}} \\
\hline 
 `they massacre' & kə d͡ʒʰɑɾdʰ-i-n & \armenian{կը ջՙարդՙին} & ɡə t͡ʃʰɑɾtʰ-e-n & \armenian{կը ջարդեն} \\
 & \multicolumn{2}{l|}{{\ind} massacre-{\thgloss}-3{\pl}} & \multicolumn{2}{l|}{{\ind} massacre-{\thgloss}-3{\pl}} \\
\hline 
 `they were striking' & kə d͡ʒʰɑɾdʰ-e-n & \armenian{կը ջՙարդՙէն} & ɡə t͡ʃʰɑɾtʰ-ej-i-n & \armenian{կը ջարդէին} \\
 & \multicolumn{2}{l|}{{\ind} massacre-{\thgloss}.{\pst}-3{\pl}} & \multicolumn{2}{l|}{{\ind} massacre-{\thgloss}-{\pst}-3{\pl}} \\
\hline \end{tabular}
\end{table}

 \subsubsection{Repetition of the auxiliary in the present perfect}
Oftentimes in the present perfect, the copular verb is repeated (\ref{sent:Mush:morpho:verb:auxRep}).

\begin{exe}
 \ex Mush \label{sent:Mush:morpho:verb:auxRep}
 \begin{xlist}
 \ex \gll ɦoɾi i-s dɾ-i̯eɾ i-s \\
 why {\aux}-2{\sg} put?-{\perfcvb} {\aux}-2{\sg} \\
 \trans Adjarian didn't provide a translation, but my guess is: `Why have you put?' \\
 \armenian{յ}̵\armenian{օրի՞ իս դրեր իս}
 \ex \gll jeɾpʰ i-s ek-i̯eɾ i-s \\
 why {\aux}-2{\sg} come-{\perfcvb} {\aux}-2{\sg} \\
 \trans `Why have you come?' \\
 \armenian{յէ՞րփ իս էկեր իս}
 \end{xlist}
\end{exe}

This is the same as the followings forms in the Bayazit subdialect (\ref{sent:Mush:morpho:verb:repAuxBayazit}). \translatorHD{Note that Adjarian doesn't transcribe these Bayazit sentences. I instead transcribe them with an SEA accent}. 

\begin{exe}
 \ex Bayazit dialect with SEA pronunciation \label{sent:Mush:morpho:verb:repAuxBayazit}
 \begin{xlist}
 \ex \gll t͡ʃʰ-e-s ɡ-ɑ-l-um e-s \\
 {\neggloss}-{\aux}-2{\sg} come-{\thgloss}-{\infgloss}-{\impfcvb} {\aux}-2{\sg} \\
 \armenian{չե՞ս գալում ես}
 \ex uɾ e-s ɡən-um e-s \\
 where {\aux}-2{\sg} go-{\impfcvb} {\aux}-2{\sg} \\
 \armenian{ու՞ր ես գնում ես}
 \end{xlist}
\end{exe}

\begin{adjarianpage}\label{page:120}\end{adjarianpage}% should be 120

\subsubsection{Hortative or imperative marker}

The imperative form of Classical /tʰoɬ/ <\armenian{թող}> `leave/let' (\translatorHD{SEA: /tʰoʁ/} has shortened to /tʰəχ/ <\armenian{թըխ}>, and sometimes simply as /tʰ/ <\armenian{թ}> (\ref{sent:Mush:morpho:verb:togh}). 

\begin{exe}
 \ex Mush \label{sent:Mush:morpho:verb:togh}
\begin{xlist}
 \ex \gll tʰχ mn-ɑ-$\emptyset$ \\
 let stay-{\thgloss}-3{\sg} \\
 \trans `let him stay!' \\
 \armenian{թխ մնա}
 \ex \gll tʰχ ɑs-\'ɑ-$\emptyset$ \\
 let say-{\thgloss}-3{\sg} \\
 \trans `let him say!' \\
 \armenian{թխ ասա՛}
 \ex \gll tʰχ ɑr-n-\'e-$\emptyset$ \\
 let take-{\vx}-{\thgloss}-3{\sg} \\
 \trans `let him take!' \\
 \armenian{թխ առնէ}
 \ex \gll tʰ-ɑr-n-\'e-$\emptyset$ \\
 let-take-{\vx}-{\thgloss}-3{\sg} \\
 \trans `let him take!' \\
 \armenian{թառնէ՛}
 \ex \gll tʰ-eɾtʰ-\'ɑ-$\emptyset$ \\
 let-go-{\thgloss}-3{\sg} \\
 \trans `let him go!' \\
 \armenian{թէրթա՛}
\end{xlist}
\end{exe}

\subsubsection{Monosyllabic verbs}

The Classical monosyllabic verbs /ɡɑl, tɑl, lɑl/ <\armenian{գալ, տալ, լալ}> have changed (Table \ref{tab:Mush:morpho:verb:monoverb}). 


\begin{table}[H]
 \centering
 \caption{Monosyllabic verbs in the Mush dialect}
 \label{tab:Mush:morpho:verb:monoverb}
 \begin{tabular}{|l| ll| ll|}
 \hline & \multicolumn{2}{l|}{Mush} & \multicolumn{2}{l|}{cf. SWA} \\ \hline
 `to come' & iɡʰ-ɑ-l & \armenian{իգՙալ} & kʰ-ɑ-l & \armenian{գալ} \\
 `to give' & it-ɑ-l & \armenian{իտալ} & d-ɑ-l & \armenian{տալ} \\
 `to cry' & il-ɑ-l & \armenian{իլալ} & l-ɑ-l & \armenian{լալ} \\
 & \multicolumn{2}{l|}{$\sqrt{}$-{\thgloss}-{\infgloss}} & \multicolumn{2}{l|}{$\sqrt{}$-{\thgloss}-{\infgloss}} \\
\hline 
 `I come' & k-iɡʰ-ɑ-m & \armenian{կիգՙամ} & ɡu-kʰ-ɑ-m & \armenian{կու գամ} \\
 `I give' & k-iɡʰ-ɑ-m & \armenian{կիտամ} & ɡu-d-ɑ-m & \armenian{կու տամ} \\
 `I cry' & k-iɡʰ-ɑ-m & \armenian{կիլամ} & ɡu-l-ɑ-m & \armenian{կու լամ} \\
 & \multicolumn{2}{l|}{{\ind}-$\sqrt{}$-{\thgloss}-1{\sg}} & \multicolumn{2}{l|}{{\ind}-$\sqrt{}$-{\thgloss}-1{\sg}} \\
 \hline 
\end{tabular}
\end{table}

\subsubsection{Causative suffix}


The Classical causative (\armenian{անցողական}) formative /-et͡sʰut͡sʰɑnel/ <-\armenian{եցուցանել}> form has been shortened to /-t͡sʰu/ <\armenian{ցու}> or /-u/ <\armenian{ու}> (Table \ref{tab:Mush:morpho:verb:caus}, sentences \ref{sent:Mush:morpho:verb:caus}). It is conjugated as the fourth conjugation class. \translatorHD{To clarify, this causative suffix is /-t͡sʰən/ in SEA/SWA, and it takes the theme vowel /e/. But in Mush, this suffix is /-(t͡sʰ)u/ and is conjugated with a theme vowel /u/.}

\begin{table}[H]
 \centering
 \caption{Causative suffix in the Mush dialect}
 \label{tab:Mush:morpho:verb:caus}
 \begin{tabular}{|l| ll| ll|}
 \hline & \multicolumn{2}{l|}{Mush} & \multicolumn{2}{l|}{cf. SWA} \\ \hline
 `I ask' & kə hɑɾ-t͡sʰ-u-s & \armenian{կը հարցում} & ɡə hɑɾ-t͡sʰən-e-m & \armenian{կը հարցնեմ} \\
 `you.{\sg} ask' & kə hɑɾ-t͡sʰ-u-s & \armenian{կը հարցուս} & ɡə hɑɾ-t͡sʰən-e-s & \armenian{կը հարցնես} \\
 `he asks' & kə hɑɾ-t͡sʰ-u-$\emptyset$ & \armenian{կը հարցու} & ɡə hɑɾ-t͡sʰən-e-$\emptyset$ & \armenian{կը հարցնէ} \\
 `we ask' & kə hɑɾ-t͡sʰ-u-nkʰ & \armenian{կը հարցունք} & ɡə hɑɾ-t͡sʰən-e-ŋkʰ & \armenian{կը հարցնենք} \\
 `you.{\pl} ask' & kə hɑɾ-t͡sʰ-u-kʰ & \armenian{կը հարցուք} & ɡə hɑɾ-t͡sʰən-e-kʰ & \armenian{կը հարցնէք} \\
 `we ask' & kə hɑɾ-t͡sʰ-u-n & \armenian{կը հարցուն} & ɡə hɑɾ-t͡sʰən-e-n & \armenian{կը հարցնեն} \\
 & \multicolumn{2}{l|}{{\ind} ask-{\caus}-{\thgloss}-{\agr}} & \multicolumn{2}{l|}{{\ind} ask-{\caus}-{\thgloss}-{\agr}} \\
 \hline `I make swear' & kə heɾtʰəv-t͡sʰ-u-m & \armenian{կը հէրթըվցում} & ɡ-eɾtʰv-e-t͡sʰən-e-m & \armenian{կը երդուեցնեմ} \\
 & \multicolumn{2}{l|}{{\ind} swear-{\caus}-{\thgloss}-{\agr}} & \multicolumn{2}{l|}{{\ind}-swear-{\thgloss}-{\caus}-{\thgloss}-{\agr}} \\
\hline 
 `I remove' & kə veɾ-u-m & \armenian{կը վէրում} & ɡə veɾ-t͡sʰən-e-m & \armenian{կը վերցնեմ} \\
 `you.{\sg} remove' & kə veɾ-u-s & \armenian{կը վէրուս} & ɡə veɾ-t͡sʰən-e-s & \armenian{կը վերցնես} \\
 `he removes' & kə veɾ-u-$\emptyset$ & \armenian{կը վէրու} & ɡə veɾ-t͡sʰən-e-$\emptyset$ & \armenian{կը վերցնէ} \\
 & \multicolumn{2}{l|}{{\ind} remove-{\caus}.{\thgloss}-{\agr}} & \multicolumn{2}{l|}{{\ind}-remove-{\caus}-{\thgloss}-{\agr}} \\
 \hline 
\end{tabular}
\end{table}

\begin{exe}
 \ex Mush \label{sent:Mush:morpho:verb:caus}
 \begin{xlist}
 \ex \gll t͡ʃʰ-ə-n h\'ɑs-u \\
 {\neggloss}-{\aux}?-3{\pl} bring-{\caus}.{\thgloss}\\
 \trans `They won't deliver'\\
 \armenian{չըն հա՛սու}
 \ex \gll pit mi̯eɾ mɑl pʰit-t͡sʰ-u-n, eɾtʰ-ɑ-n pʰit-t͡sʰ-u-n \\
 {\fut} our ox kill-{\caus}-{\thgloss}-3{\pl}, go-{\thgloss}-{3\pl} kill-{\caus}-{\thgloss}-3{\pl} \\
 \trans `They will kill our ox, let them kill.'\\
 \armenian{պտի մեր մալ փիտցուն, էրթան փիտցուն}
 \end{xlist}
\end{exe}

\subsubsection{Past participle}

The past participle is formed with the formative /-eɾ/ <\armenian{եր}>. But for passive (\armenian{կրաւորակերպ}) verbs, the formative /-uk/ <\armenian{ուկ}> is also used. \translatorHD{I think he means that intransitives in general use /-uk/, not just verbs with passive voice.}\footnote{\translatorHD{Note that based on Adjarian's translations to SWA, these past participles seem to function like resultative participles (/-ɑd͡z/ in SWA), and not perfective converbs (/-el/ in SEA). But we can't be sure because he doesn't provide complete paradigms or sentences.} }



\begin{table}[H]
 \centering
 \caption{Past participle suffix /-eɾ, -uk/ in the Mush dialect}
 \label{tab:Mush:morpho:verb:pastPart}
 \begin{tabular}{|l| ll| ll|}
 \hline & \multicolumn{2}{l|}{Mush} & \multicolumn{2}{l|}{cf. SWA} \\ \hline
 `died' & mer-uk & \armenian{մէռուկ} & meɾ-ɑd͡z & \armenian{մեռած} \\
 `died' & pʰit͡sʰ-uk & \armenian{փիցուկ} & meɾ-ɑd͡z & \armenian{սատկած} \\
 & \multicolumn{2}{l|}{$\sqrt{}$-{\perfcvb}} & \multicolumn{2}{l|}{$\sqrt{}$-{\rptcp}} \\
\hline 
`broken' & kotɾ-uk & \armenian{կոտրուկ} & kodəɾ-v-ɑd͡z & \armenian{կոտրուած} \\
`written' & ɡʰɾ-uk & \armenian{գՙրուկ} & kʰəɾ-v-ɑd͡z & \armenian{գրուած} \\
 & \multicolumn{2}{l|}{$\sqrt{}$-{\perfcvb}} & \multicolumn{2}{l|}{$\sqrt{}$-{\pass}-{\rptcp}} \\
 \hline 
\end{tabular}
\end{table}
 
\section{Literature}

For the Mush dialect, there is an extensive study by Mseriants (Мсеріанцъ; \translatorHD{modern Russian: Мсеріанцъ}) \citep{Mseriants-1897-Part1,Mseriants-1901-Part2}\footnote{\translatorHD{The original title that Adjarian gives is in the pre-revolution Russian alphabet: Этюды по армянской діалектологіи.}} A summary of the same work is published in French by the same author \citep{Mserianz-1899-Mush}, and a second in German by L. Patrubàny in his periodical Sprachwissenschaftliche Abhandlungen, volume 1, page 271-288

It is surprising that in these works, the two types of groups for voiced consonants is not considered. This is something that a very common ear would have been able to distinguish with little intention.

There are various works that are published in the Mush dialect: 

{\litoverview}


\begin{itemize}
 \item Literature with the Mush dialect
 \begin{itemize}
 \item General Mush dialect 
 \begin{itemize}
 \item [hd cyrillic]. \armenian{Բեդրսպոււրկ} 1875. The same was published in European transcriptions by L. Patrubàny in his periodical Sprachwissenschaftliche Abhandlungen, 1.241-271
\item \armenian{Գ. վ. Սրուանձտեանց} 
\begin{itemize}
 \item – \armenian{Գրոց ու բրոց. Պօլիս} 1874
 \item – \armenian{Մանանայ. Պօլիս} 1876

 \begin{adjarianpage}\label{page:121}\end{adjarianpage}% should be 121

\item – \armenian{Համով հոտով. Պօլիս}
 \item – \armenian{Հանդէս թռչնոց. Արեւ. Մամուլ} 1884, page 389-392
\end{itemize}
\item \armenian{Արիստ. վ. Սեդրակեան – Քնար Մշեցոց եւ Վանեցոց. Վղրշպտ}. 1874
 \item \armenian{Յ. Ամրիկեանց – Մշու թռչուն օտար երկրում. Կռունկ} 1862, page 386-390
 \item \armenian{Մ. Դանիէլեան} 
 \begin{itemize}
 \item – \armenian{Պարերգ, խաղ, պառաւներու աղօթք. Բիւրակն}, 1898, page 313-4
 \item – \armenian{Յակօի վախենակ կարելը. անդ}, 1899, page 329-330
 \end{itemize}
 \item \armenian{Վ. Արտակ – Պարերգ. անդ}, 1900, page 122-3
\item \armenian{Գ. Տ. Զ}. – \armenian{Կաղ եզը, անդ}, page 618

 \end{itemize}
 \item Bulanık vernacular
 \begin{itemize}
 \item \armenian{Բենսէ – Բուլանըխ կամ Հարք գաւառ. Ազգ. հանդ. Ե. էջ} 9-184, \armenian{Զ} page 7-108
\item \armenian{Ս. Հայկունի} 
\begin{itemize}
 \item – \armenian{Ժողովրդական գրականութ. րեկորն. Արրա}. 1896, 566-7
\item – \armenian{Սօյլամազ խանըմ, Արրա}. 1896, 557-560
\end{itemize}

 \end{itemize}
 \item Alashkert vernacular
 \begin{itemize}
 \item \armenian{Ս. Հայկունի – Ալաշկերտի հայոց առածները. Արտտ}. 1894, page 200, 263-4
\item \armenian{Գ. քհ. Նժդեհեանց – Ալաշկերտի բանաւոր գրականութիւնից. Ազգ. Հանդ. Ե}. 185-199, \armenian{Է} 437-505

 \end{itemize}
 \item Aparan vernacular
 \begin{itemize}
 \item \armenian{Գարեգին Սարկաւագ – Սասմայ ծռեր. Թիֆլիս} 1892
\item \armenian{Բ. Խալաթեանց – Իրանի հերոսները հայ ժողովրդի մէջ. Բարիզ} 1901, page 24-44, 74-76

 \end{itemize}
\item Vernacular of New Bayazet villages
\begin{itemize}
 \item \armenian{Սէնէքէրիմ Արծրունի – Նոր Պայազիտու գաղթական Մշեցւոց նշանդրէքն ու հարսանիքը. Կռունկ} 1863, page 385-400

\end{itemize}
\item Sasun vernacular
\begin{itemize}
 \item \armenian{Մ. Մուրատեան} 
 \begin{itemize}
 \item – \armenian{Սասնցոց պարերգ. Բիւրակն} 1900, page 121-2
\item – \armenian{Հանելուկներ եւ պարերգ. անդ}, page 470-1
\end{itemize}
 
\end{itemize}
\item Paghesh subdialect
\begin{itemize}
 \item \armenian{Թուխ-Կռպօ}
 \begin{itemize}
 \item – \armenian{Պարերգ եւն. Բիւրակն}, 1898, page 300-301
 \item – \armenian{Սիրաբանութիւն. անդ}, page 651-2
 \end{itemize}

\end{itemize}
\item Khouyt vernacular
\begin{itemize}
 \item \armenian{Զ. Կէնճեան – Հարսանեկան պարերգ. Բիւրակն} 1898, page 739-741

\end{itemize}
\end{itemize}
\end{itemize}

Besides these, Sarkis Haykuni (\armenian{Ս. Հայկունի}) has published 34 fables in the vernaculars of Arjesh, Artske, Bulanık, Aparan, Bitlis, Alashkert, Khlat, Hınıs, in the \citeauthor{Eminian}, volumes 2, 4, 5 (\armenian{Բ. Դ. Ե}.; 1901-4), a folk song from Hınıs (ibid., volume 6, \armenian{Զ}., page 101), Manazkert (volume 6, \armenian{Զ}., page 139). Unfortunately, these don't use scientific orthography. 

\begin{adjarianpage}\label{page:122}\end{adjarianpage}% should be 122


\section{Text samples}

{\sampleoverview}

\subsection{Mush dialect}
\subsubsection{City of Mush}

Adjarian's source: See \citet[6-7]{Patkanian-1875-RussianDialects}.




\armenian{Սանասուր նստոկ էր Սասուն, ուր պապու կըռքեր չըն թօղի օր ըդի սըթրէր ու յէլավ խըստ էրէց ուր պապուն ու մամուն ու գՙնաց Բաղդադ։ Ուր պապ նըստուկ էր փանջարէն, տէսավ օր ուր տղէն Սանասար կիգՙէր. ու ճանչցավ ու ասեց։}

– \armenian{Է՜յ, մէռնիմ քզի մեծ կուռք. ի՛մալ զքո մատաղ քաշիր բՙէրիր. յ}̵\armenian{ար յ}̵\armenian{էբՙ է մանցը՝ զմըկէլ լէ կը քաշես բՙերես։}

\armenian{Մամ, չընքի խաչապաշտ էր, նստավ ուր տղէկներն ապով արսունք թափեց։}

\armenian{Պապ առեց թուր ու սուր ու գՙնաց, կանչեց ու ասեց}.

– \armenian{Արի յէրթանք, վորթի, յէրկըրպաքութեն արա մեծ կըռքին, օր զքըզի մատղեմ։}
\armenian{Ասեց տղէն}.

– \armenian{Աբՙօ՛, քո ճոչ կուռք շատ զօրավոր կուռք է. գՙիշեր լէ չըր թողնե, օր մենք ընտեղ սըթըրվենք. յ}̵\armenian{ար յ}̵\armenian{ե՞բՙ է մանցե՝ չուր մէկէլ մատաղ լէ կը քաշէ ու կը բՙէրէ։}

\armenian{Առավ զտղէն ու մտան կռքատուն։}

\armenian{Տղէն պապուն ասեց}.

– \armenian{Աբՙօ՛, չէ՞ դՙու գՙինաս օր մենք գՙացինք՝ մենք պստիկ էինք. մենք զքո կռքի զօրութեն չընք գՙինե. դէ՛, դՙու կզի քո կըռքին յէրկրպաքութեն տուր, իշեմ իմալ կիտաս, ուսնեմ։}

\armenian{Պապն ասեց}. – \armenian{Հմլա, լա՛օ, ու կզավ յէրկրպաքութեն տըվեց։ Տղէն ասեց}.

– \armenian{Աբՙօ, քո կուռք ի՜նչ զօրավոր կուռք էր. օր դՙու յ}̵\armenian{եբՙոր կզար, իմ աչքեր մթնեց, թտէսա իՙմալ էրէցիր։ (Զընքի չհասավ օր առչի դրբին զարկէր. մալա̈թի կօճկըներ չարձՙըկվավ)։ Ասեց. Ա՛բՙօ, աբՙօ, իդա հաղ լէ յէրկրպաքուլեն էրէ, տէսնենք ի՞մալ կէնես, օր յես լէ էնեմ։}

\armenian{Ու հեղմ՚ լէ յ}̵\armenian{եբՙոր կճաւ պապ, տղէն ասեց. «Յա՛ հացն ու գՙինի. տէրն կէնթանի». ու գուրզ մի իջՙավ, ու զուր պապ խալիֆէն յօթն գՙապ գՙէտին վե իջՙուց։ Առեց զգուրզն ու ինգավ մէջ}... 



\begin{adjarianpage}\label{page:123}\end{adjarianpage}% should be 123

... \armenian{կըռքէրուն, զեմէն լէ ջՙարդՙեց, ու առավ զարծըթներ լցեց ուր մալա̈թթի փէշ ու բերեց տըվեց ուր մամուն ոււ ասեց}.

– \armenian{Մա՛մօ, իդոնք էրէ քըզի զէնաթ։}

\armenian{Մամ լէ կզաւ վըր քիթ ու բՙէրնին, յէրկրպաքութեն էրեց, ու ասեց}.

– \armenian{Գՙօհանամ քէնէ յէրկնի ու յէրկրի ստէղծօծ. գՙա օր զմը զի ազզատ էրէցիր էն զալըմի ձՙէռնէն։}

\armenian{Բՙէրեց զՍանասար փսակեց, ու պապու տեղ դՙրեց վըր թախթին։ Ընի ընդեղ մնաց. դՙառնանք Աբՙամէլիքի վրէն։}

\subsubsection{Village of Karnen in Mush}

Adjarian's source: This story was told by my friend Tigran Dimaksian (\armenian{Տիգրան Դիմաքսեան}; \translatorHD{SEA: /tiɡɾɑn dimɑkʰsjɑn/, SWA: /dikʰɾɑn tʰimɑkʰsjɑn/}) when I was in Paris, and I wrote it down. He is from the village of Karnen, which is half an hour away from Mush. He was a previous student at the Istanbul Getronagan Armenian High School (\armenian{Կեդրոնական վարժարան}). He escaped to Paris from the massacres. The dialect is very close, and the narrator is aware of the scientific method; thus he presents the story with a very exact scientific orthography. However when I was in Etchmiadzin, I learned from many people from Mush that the people of Karnen differ from the city for the pronunciation of the sounds /b ɡ d/ <\armenian{բ գ դ}> and so on. Because of this, perhaps we have the sounds /b ɡ d/ <\armenian{բ գ դ}> against some cases of /p t k/ <\armenian{պ կ տ}>. 

– \armenian{Բՙարի լո՛ւս քի, ախպէր Թօ՛րօ։}

– \armenian{Վոյ Ասսու խէրն ու բՙարին, Ըռքօ ջան։}

– \armenian{Ի՞մալ իս, ի՞նչ խէր հարցում (հարցուցանեմ) վրէտ, վըր ճժէրուտ։}

– \armenian{Սախ (ողջ) մնաս. Ասվաձ բՙաշխէ ըզքու զավկըներ. ըմմէն լէ սախ ին. ըզքու ձՙեռք կը պագՙին. նստի, ա՛խպէր, նստի. քիչ մը ժըղլիք (խօսիլ}), \armenian{բՙան մ՚ըսէ մժուլինք (մտիկ ընել}), \armenian{Ասվաձըտ սիրիս. ըզքու էն մէգ գՙըլխու գՙալիք նախլ էրէ։}

– \armenian{Հա տօ աղէկ միտկըս բՙէրիր. նստի ըսիմ։}

– \armenian{Արաբ ասկրի տարին էր. իշօվ քարվընօվ զախիրա (պարէն) տարեր էնք ասկրին. էն դՙիէն օր էգանք, իմ յ}̵\armenian{ընկէրներ Բուլանըխցի էն. ուրանց տուն գՙացին. ես մինադ յօլ ու ռէվան (ճանապարհուել) էզա ի Մուշ. յ}̵\armenian{իրգուն էր հասա Սրէ-Սիփանա տագ. էփէյի քէլէցի. լուսնյակ թամամ էլաձ էր. աստղըներ լէ գը փէլգըդէն. հազ մ՚ լէ տէսնամ օր քուրթ մը յ}̵\armenian{առջՙեվս յ}̵\armenian{էլավ ու ջղարէ (սիկառ) մ՚ յ}̵\armenian{ուզէեց. ես լէ, դՙու գինաս օր ջղարա չըմ}... 

\begin{adjarianpage}\label{page:124}\end{adjarianpage}% should be 124

... \armenian{քա՛շի. ըսի օր չըկա. քուրթ քաշլեց զօր ընել ու ձՙեռք թալլեց ջէրս օր ջղարա իշէ. ես լէ ակցի մը սրդին զրգի, էրսի վրա գՙէդին յ}̵\armenian{ընգավ, ու հըմալ ճըվոցըմ հանեց օր սար ու ձՙոր ձՙէն ավեց։ Հեղմ՚ լէ ի՞նչ՝ տէսնամ օր հինգ հօկի սիլալըխօվ քրու տըգէն (քարի տակէն) դ՚ուրս յ}̵\armenian{էլան ու վրէս վզէցին։ Էլ գիցա օր մէռնէլու յա աբրէլու սըհաթն է։ Աստըձու զօրուտէնօվ ժանգռոդ խանջալըմ կէր վրէս, զթեվս քշթեցի ու միջ՚վընին յ}̵\armenian{ընգա, մէգ էրգու գՙէդին շըռճեցի, հըմա ղօղորան ըսիմ, ես լէ քնիմ տէղօվ յարալու էղա. հմա ախըր ի՛նչ էնիմ. մէգ մարտ հինգ մարտու ինչ կըռնա էնէ. վօր հասըլ (վերջապէս) զիս բՙռնէցին, ձՙէռներս յ}̵\armenian{էտեվս կապէցին, ու սար տարան։ Էլ ինչ օր իմ վրէն էգավ, քու վրէն յ}̵\armenian{իկա Ասսու խէրն ու բՙարին։ Հըրի լուց չարչըրգէցին ու յ}̵\armenian{ըմնուց սօրա յ}̵\armenian{ուզէցին օր զիս սպանին։ Հմա Սատըձուց էր, դըհա մէռնէլուս վախտ չըր է՛գի։ Զիս կանգըցին ուրընցնէ իրեք շեք (քայլ) հէռուն ու զխանջըլներ հանէցին ու ուրանց նստաձ տէղէն սըրըթմանի (շեշտակի) իմ ճըռնէրուն բաշլէցին զըրկել. ու ես ուրանց զօրօվ մսիս միճէն խանջալ հանէլօվ ուրանց կիդէնք} (< \armenian{կուտային, իմա՛ կուտայի)։ Էն սօնը մէգ լէ զթուր քաշեց ու վրէս էկավ օր ըզգՙլուխս կըդրէ. հըմա Աստըձուց էր, դՙիմացէն քընի մ՚ հօկի քըրթէրէն բօռացին. «Տօ՛, դՙուք վօ՞րն իք յ}̵\armenian{օդ. ի՞նչ կէնիք»։ Քըրթեր չէ «Տօ՛, էդէք նէճիր է ընգի ձՙէռվըներս». ու մըր դՙին էգան։ յ}̵\armenian{իրարու հետ խօսաձ վախտ մէգն գ՚ըլուխս լուսնյակին դՙառցուց ու քըրթէրէն հարցուց. «Կո՛ւռօ տու ֆլա՞ ի կուրմանջի (Տղա՛յ, դու հա՞յ ես թէ քուրդ)»։}

– \armenian{Էզ ֆլա մէ (ես հայ եմ)։}

– \armenian{Տօ ախպէր դՙու հա՞յ իս։}

– \armenian{Հա՛յ իմ խուրբան։}

– \armenian{Տօ գիդի բՙըռնէք ըտոնք ըզհայ (կամ արագ խօսած ժամանակ՝ սայ) գը չարչըրին։}

\armenian{Հրաման տըվեց յ}̵\armenian{ընկէրնէրուն ու ըզքըրթեր մէզիգ մէգիգ կարէցիպ, զթըվընքներ առան, ու ծէձէլօվ չարչըրկէլօվ ուրանց գՙէղ տարան, մուդուռին թասլիմ էրին։ յ}̵\armenian{էտկէն ընձի պատմէցին օր ուրանց կօվեր կօրեր էն, փընդըռնէլու յ}̵\armenian{էլաձ էն, ու ուրանց ռաստ էգաձ։}

\begin{adjarianpage}\label{page:125}\end{adjarianpage}% should be 125

\subsubsection{Bulanık}

Adjarian's source: See \citeauthor{AzgagrakanHandes}, volume 6 (\armenian{Զ}.), page 12


\armenian{Ջՙախցպան Աղէկ աղլապ (միշտ) խսա կէնէր (պատմել}), \armenian{կըսէր յէս ջՙաղացն էնք, էն մէկ օր իսկի մարթ չը մնաց մօտսի. հարքէս (ամէն ոք) զաղուն թողեց իմ ումուդով (յուսով}), \armenian{գՙնաց ուր տուն։ յ}̵\armenian{անգՙան գահ էր (ուշ ատեն էր}). \armenian{մէնակ նըստուկ էնք, յէս տէսա դՙօլ ու զուռնի ձՙէն էկավ, դՙառնամ աչքիմ օր} 10-20 \armenian{կնիկ, հաքուկ-խփուկ պար բՙռնած էն ու կը խաղան։ Յէս տէսա կնիկ մ՚էկավ մըտ ընձի, ըսեց. «Աղէկ, յ}̵\armenian{օրի՞ն իս նստի, յ}̵\armenian{էլի խախցի»։ Զարէց կտրավ, յ}̵\armenian{էլա բՙռնի պար ու խախցա. աչքիմ աչքիմ՝ տօ յա, իգա կնկա վրէն իմ քավօրքուր Ա-ի դալմէն է (զգեստ}). \armenian{յ}̵\armenian{ուշիկ մէ զչախուն հանէցի ջէբէս ու զդալմի մէկ փէշ կտրէցի. հըտ իմ կտրէլուն՝ կնկտիք աներեւութ էղան։}

\armenian{Էն լուսուն գՙացի քավօրօչչ տուն. ըսի ըշտէ իդա գՙիշեր հըմլա հըմլա բՙան մէ պատահրավ. մարթ չավտըցավ. ըսի «Ջա՛նըմ, յ}̵\armenian{օրի՞ն չըք ավտընա. բՙէրէք զքավօր քրօչ դալմէն սանք (տեսնանք)». դալմէն օր բՙէրին՝ օղորդ օր մէկ փէշ կտրուկ էր. իմ մօտու կտոր լէ այնի (ճիշտ) էդ դալմի կտօրէն էր։}

\subsubsection{Alashkert}

Adjarian's source: See \citeauthor{Eminian}, volume 2 (\armenian{Բ}.), page 337

\armenian{Քախկի մի մէչ իրեք հատ քօսա կէղնին. էտոնց սօվօրուտեն լէ էն էր օր յ}̵\armenian{ըմնօր գՙիկէն, ճամպնէրու վրէն կը կայնէն օր գՙէղածի – մէղածի ըռաստ յ}̵\armenian{իկէր, խապէն. էտոնք մախսուս յ}̵\armenian{իրարուց հէռու կը կայնէն՝ օր գՙէղածիք գՙինէն թէ ջօկ ջօկ մարթիկ ին։}

\armenian{Ավուր մէկ գՙէղածի մի կօվ մի կիտա ուր տղին ու կըսա}. – \armenian{Լա՛օ, տար իտա կօվ քաղաք՝ ծախա. հըմա իրեք օսկուց պակաս չէղնի տաս. յ}̵\armenian{ընճի օր (ինչու որ) մեր կօվ համ՝ խօրօտ ա ու համ կաթնօվ։}

\armenian{Էտ տղէն էր, լուսուն շուտ առավ կօվն ու գՙնաց. էտոնց իդարէն լէ զատի մած (մնացած) էր էտ կօվ. խէլ մի գՙնաց, ըռուստ էկավ յ}̵\armenian{առճի քօսին, օր քախքից դՙուս կայներ էր։}

– \armenian{Օղուր էղնի, դՙո՞ր կէրտաս. հարծուց քօսէն։}

\begin{adjarianpage}\label{page:126}\end{adjarianpage}% should be 126

– \armenian{Սաղ էղնիս. կէրտամ քաղաք, ըսեց. իմ պապ ընծի ճամպեր ա օր իտա կօվ տանիմ ծախիմ։}

\armenian{Քօսէն կըսա}. – \armenian{Տօ՛, ի՞մալ կօվ. էտի հօրտիկ ա. հա՛յ, մըսի՛ (մի՛ ըսեր) կօվ, խըլղ կը ծիծղան քու վրէն. թէ քզի միտք կա ծախէլու՝ յ}̵\armenian{էռսուն խուռուշ կիտամ։}

– \armenian{Ախպէր, դը՜ գէն գՙնա. չը գՙինամ՝ դ՚ո՞ւ իս ծուռ, չէ յէս իմ ծուռ՝ իմ վրէն կը ծիծղաս, ըսեց ու քէլեց։}

– \armenian{Հա՛, հա՛, ըսեց քօսէն. չուրի դՙու չը տանիս քաղաք, չավտընաս թէ էտի հօրտիկ ա։}



\subsubsection{Aparan}

Adjarian's source: See \armenian{Գարեգին Սարկաւագի Սասմայ ծռեր, էջ} 14-15.


\armenian{Ժամանակօվ մէկ թաքավոր կէղնի, անուն Սէնէքէրիմ։ Սէնէքէրիմին էրկու տղա կէղնի, մէկի անուն Սանասար, մէկին Ասլիմէլիք։ Սէնէքէրիմ ինք կռապաշտ էր, տղէկներ ասվածապաշտ։}

\armenian{Խօշուն էրեց ու գՙնաց Էրուսաղէմա վրէն կռիվ։}

\armenian{Յօթ տարի քախքի բՙօլոր խօշուն չափըռնեց նստավ։}

\armenian{Թանգուտեն ընկավ քախքի մէչ. Թաքավորն ուր վազիր։ դավրէշ խըլըղի էղան ու ընգան քաղքի մէջ. էրկու պառվու ռաստ էկան, տէսան օր իրարու հետ կռիվ կէնին. հարցուցին թէ – յ}̵\armenian{օրի՞ կը կռվիք։}

\armenian{Մէկ պառավ վէրցուց թէ՝ Դավրէշ բաբա, թանգուտեն ընգեր ա էրկիր, հացներսի խըլըսեր ա. ընձի տղէ մ՚ ունէնք (ունէի}), \armenian{բՙէրէցի մօրթինք կէրանք. մկա էնի ուր տղէն չբՙէրա մօրթինք ուտինք։}

\armenian{Վէրցուց թաքավոր ուր վազիրին ըսեց}. – \armenian{Մեր թաքավօրուտեն իսկի մէկ թաքավօրուտենի չէ։}

– \armenian{Ըբա. ըսեց, ի՞նչ էնինք, վա՛զիր։}

– \armenian{Յէտ դՙառնանք. նստինք մեր թախտի վրէն։}

\armenian{Ու սկսէցին խէր ու խէրյաթ էնել ու պատարաք էնել. պատարաքն օր էրէցին պրծան, հրէշտակներ սրօվ, թրօվ իջՙան Սէնէքէրիմի ասքարի մէջ, ու ջՙարթէցին, ու սպանէցի ու կօտօրէցին։}






\begin{adjarianpage}\label{page:127}\end{adjarianpage}% should be 127

\subsubsection{Manazkert}

Adjarian's source: See \citeauthor{Eminian}, volume 7 (\armenian{Զ}.), page 139.

\armenian{Հառավել, հառավել, յէշ},

\armenian{Հառավել, Մուսլօ գոշէշ}.

\armenian{Ձեռսի թալեմ Աստծու փէշ}

\armenian{Մզի ցորեն տա փէշքէշ},

\armenian{Ծախենք տանք մեր ռէսի բէշ},

\armenian{Չէղնի ծախին բօզ գոմէշ։}

\armenian{Հէլէլ – մէլէլ շէկ կռտան},

\armenian{Հօտղներ, քշէք զէդ գութան}.

\armenian{Օրթա հասնինք բրէտան},

\armenian{Հա՛ մռնիմ քզի, Շէկօ, Կռտան։}

\armenian{Տղա՛ Մա՛նուկ քշա ղէդ եզ},

\armenian{Գութան բանի քանդի սէզ}.

\armenian{Ցորեն էղնի դէզ դէզ},

\armenian{Հա մէռնիմ Մանուկին ես։}

\armenian{Քշա՛ Լաւանդն ու Խնձօրօն},

\armenian{Հանդա փոխդ բէրեց Կարօն},

\armenian{Մկա լծինք Շապազն ու Խէրօն։}

\armenian{Օր դմանհնա մզի ռէս Միրօն։}

\armenian{Տղէք, ձէն հանէք գութնէն},

\armenian{Տաս տուն ա մեր բնատէն}.

\armenian{Գութնի ակեր ճըռվըռան},

\armenian{Ճըռվըռա, ձէնիդ խուրբան}.

\armenian{Հան, Խաղօն, Զմօն էկան},

\armenian{Բէրին մածնախառ թան},

\armenian{Իդ հաղսի կարձկինք գութան}.

\armenian{Հա բավէ մըն, հաժավել։}

\armenian{Մատաղ ձզի գմըշտան},

\armenian{Կըսեմ հօտղներ էրթան}

\armenian{Պաղպաղ ջրով տղէկ հօվցան},

\armenian{Լվան, ղըվըռցուն, արծան։}

\armenian{Տավար էկավ, քնուց յ}̵\armenian{էլան},

\armenian{Գութան լծին, խառզան առան։}


\begin{adjarianpage}\label{page:128}\end{adjarianpage}% should be 128


\subsubsection{Hınıs}

\armenian{Կեղնի չեղնի իրիկմ ու կնիկ մի, ըգրանց լէ կեղնի տղէ մի. էդ տղէն լէ իսկի բան չգինա. նա կարդալ, նա սանաթ։}

\armenian{Էդ տղէն կը ծռի թաքավորի աղջկա վրէն, կըսէ ուր մամուն}.

– \armenian{Նա՛նէ, գնա ընձի թաքավորի աղջիկ յ}̵\armenian{ուզէ։}
 
– \armenian{Տօ տղա, լա՛օ, քու պապ աղքատ, ավուր հացի կարօտ, մենք յ}̵\armenian{ի՞մալ էրթանք թաքավորի աղջիկ յ}̵\armenian{ուզինք. չէ՛, զմզի կը մօրթէն, դու լէ սանաթ մի չգինաս օր ըսիմ, հան տղէս սանաթ մի գինա։}

– \armenian{Չէ՛ նա՛նէ, իլաի օր պտի էրթաս յ}̵\armenian{ուզիս։}

\armenian{յ}̵\armenian{ըրկուն օր տղի պապ տուն էկավ, տղի մամ ըսեց}. – \armenian{Հմլա բան կա. քու տղէն կըսէ գացէք թաքավորի աղջիկ յ}̵\armenian{ուզէք։}

\armenian{Պապ կըսէ}. – \armenian{Տղա լա՛օ, խե՛լքտի թռուցե՞ր իս։}

– \armenian{Չէ, կեաքէ, կըսէ, պտի էրթաս յ}̵\armenian{ուզիս։}

\armenian{Չեղնիր տղի վրէն, կէրթա տղի պապ թաքավորի մօտ, թաքավոր կըսէ}. – \armenian{Ընչի՞ համար իս էկի։}

– \armenian{Թաքավոր ապրած, քու աղջիկ պտի տաս իմ տղին։}

– \armenian{Իմ աղջիկ ի՞մալ տամ քու տղին. քու տղէն սանաթ գինա՞։}

– \armenian{Չէ՛, վոլա (արաբ. Վալլահ}), \armenian{չգինա։}

– \armenian{Գնա, զքու տղէն բի տսնամ։}

\subsubsection{Sason}
Adjarian's source: See \citeauthor{Byurakn}, 1900, page 121-122.


\armenian{Խորոտիկ, օսկի գնտիկ},

\armenian{Դու զաղջիկ իտաս ինծիկ։}

\armenian{Բարձր Մարաթկի սարեր},

\armenian{Ամուր կուլէն իւր քարեր},

\armenian{Խըսմէթ էնէր իւր եարեր։}

\armenian{Հընչի ե՞րբ ըսինք զէտ բան},

\armenian{Ինջնենք պաղչան ու սայրան},

\armenian{Քաղինք զէտ մանտրիկ ռեհան},

\armenian{Տարինք դրինք խորըսթան},

\armenian{Եղաւ սեւ օձի նըման},


\begin{adjarianpage}\label{page:129}\end{adjarianpage}% should be 129


\armenian{Խաբերց զԱդամ եւ զԵւան},

\armenian{Հանեց դրախտէն անարժան։}

\armenian{Ելաւ, ուըր Հըլպու սաբուն},

\armenian{Կուլէր զաշուն ու զգարուն։}

\armenian{Բարձր Մարաթկի խաչեր},

\armenian{Շուրջ ու բոլոր կանաչ էր}.

\armenian{Աշխարհք երկիր կ՚աղաչէր},

\armenian{Խըսմէթ ըներ սեւ աչեր։}

\armenian{Շունշանորդի քօլոսով},

\armenian{Քու բան ի՞նչ էր մեր դռնով}.

\armenian{Կ՚ելեմ ըսեմ մեծօրաց},

\armenian{Ինջնին զարկուն խանջարով։}

\armenian{Զարկին խորոնւկ ու զարկին},

\armenian{Սարէց ելաւ սեւ արուն։}

\armenian{Եարն էր գնաց ջուր մերուն},

\armenian{Լցես մէջ իւր կժերուն},

\armenian{Թալեց ուըր իւր թեւերուն},

\armenian{Թափաւ ուըր իւր փօթերուն},

\armenian{Լցուաւ մէջ իւր սօլերուն},

\armenian{Շարի շամամ ծծերուն։}

\armenian{Երցու աղջիկ մեր դռկից}.

\armenian{Բուռ մը չամիչ կրկմից},

\armenian{Պագ մը խաբեց չտուեց։}

\armenian{Ես գացի Մշու դիմաց},

\armenian{Տեսայ դռներ կիսաբաց}.

\armenian{Մտայ կ՚առնէք թաժէ հաց}.

\armenian{Անտէր շունէն մնացած}

\armenian{Բերան բացեց զիս խածնէր}.

\armenian{Ընծի ի՞նչ խածիլ պիտէր։}

\armenian{Ընծի գիրկ ծոցքը պիտէր։}

\armenian{Դիման Մշու գացեր իմ},

\armenian{Լաշփէտ Մշէն բերեր իմ},

\armenian{Չարկամ վրամ թալեր իմ},

\armenian{Եարի դռնէն ընցեր իմ},

\armenian{Փէշտիմալ գօտէն փրցեր իմ},

\armenian{Սեւ աչուըներ սրբեր իմ։}


\begin{adjarianpage}\label{page:130}\end{adjarianpage}% should be 130

\subsubsection{Khouyt}

Adjarian's source: See \citeauthor{Byurakn}, 1898, page 739. 


\armenian{Ամպն էր երկինք հովն անուշ},

\armenian{Սիրականիս քունն անուշ}.

\armenian{Տարին տասներկու ամիս՝}

\armenian{Թօռմաւ խնծորն ի գօտիս}.

\armenian{Խնծորի կէս խածուկ էր},

\armenian{Չորս քէօշէն արծթուկ էր}.

\armenian{Տանիմ իտամ ոսկերիչ},

\armenian{Շինէ մատնիք ապրճան},

\armenian{Տամ եարոջ՝}

\armenian{Իր քրոջ։}

\armenian{Ամպն էր երկինք հովն անուշ՝}

\armenian{Եար խորոտիկ, պաքն անուշ։}

\armenian{Տօ՛, տղա՛յ, տղա՛յ, քօլոսով՝}

\armenian{Մինչ ե՞րբ ընցնիս մեր դրնով}.

\armenian{Զարկինք քեզ խանջար խորուն},

\armenian{Ելնի քու կարմիր արուն}.

\armenian{Ամպն էր երկինք եւ այլն}.

\armenian{Աղջիկ քու անուն ի՞նչ ա}.

\armenian{Աղջիկ քու անուն Շուշան}.

\armenian{Ելիր երթանք Սուրբ Նշան՝}

\armenian{Օսկի մատնի քեզ նշան}.

\armenian{Հարիր ուզես՝ հազար կիտամ}.

\armenian{Ամպն էր երկինք եւ այլն։}

\armenian{Կէս գիշերուն դուրս ելայ}.

\armenian{Մատղաշ ամպիկ մ՚էր ելէր},


\armenian{Դանդաղ ձնիկ մ՚էր թալեր},

\armenian{Բօկիկ հետիկ մ՚է գացէ}.

\armenian{Առա զհետիկն ու գացի},

\armenian{Գացի կայնա գըլխընուն։}

\armenian{Վարդեր փըռուկ էր երըսնուն},

\armenian{Երկու ծծի մէջ նշան կէր},

\armenian{Չոգայ թէ զնչան պաքէք},

\begin{adjarianpage}\label{page:131}\end{adjarianpage}% should be 131

\armenian{Գլորա զընտան ինկայ},

\armenian{Կանչի Սուրբ ու Սրբօրէք},

\armenian{Մէկիկ չեկաւ երեւան},

\armenian{Սուրբ Սարգիսն էր Խորուսան։}


\subsection{Paghesh subdialect}
\subsubsection{Paghesh}

Adjarian's source: See \citeauthor{Eminian}, volume 4 (\armenian{Դ}.), page 93. 


– \armenian{Այ տղա, ես քե, դու ձի։}

\armenian{Աղջիկն ու տղէն սիրեցին իրուր։ Աղջկա աչք ճամբախն է օր ուր ախբէր իգա, չտէսնա խավշիկ. իշկեց օր ախբէր զատենց էկավ, ախջիկ յ}̵\armenian{իմցուց խավշկա։ Դողուն առաւ աղվըզու ջան, էլավ, գնաց, էլման մտավ ջաղչի քարի տակ։ Աղջիկէ ախբէրն էկավ, զատ էփէցին, կէրան, քնան չուր ի լուս։ Ախբէր կու կէլլի, էլի գնաց նէջիր. աղվէզ էլման էկաւ աղջիկէ մօտ. ուտեն, խմեն, քէյֆ էրէցին ուրանց։ Մէ ամիս, էրկու իրեք ընցավ մէջտեղ, աղբէր իշկեց օր քրոջ փոր օր զօր ՝ օր զօր կուռի, օր պը զօր միզար կու բանձրնա։}

– \armenian{Քո՛ւրօ, ասաց տղէն. էս տեղ մարդ չկա. էդ յ}̵\armenian{ը՞մալ բան ա. կիշկիմ քո մէզար օր պը զօր կու բանձրնա. թէ մարդ ունիս, գաղտուտ կու պէհիս, բե՛ր էս տեղ, աշկարա պսակիմ քու վէրէն. հալալ իրիկ կնիկ էղէք, իսան օր կա՝ մէղաց վորդի ա։}

– \armenian{Հավա՛ր, աղբէր, դու էդա խօսք յ}̵\armenian{ի՞նչ խօսք ա ձի ասէցիր. ես իմ խօր անվան մօտէն չե՞մ ամչնար օր դու ձի էդ խօսք կասիս. ես էդ բան էրո՞ղն իմ։}

– \armenian{Հըպա, քո՛ւրօ, էդ քո միզար յ}̵\armenian{օրի՞ն կը բանձրնա։}

– \armenian{Տախտ կավիլի, նռան խատ մի գտա, թալէցի բէրանսի. էն տէղացէն փորս ուռաւ։}

\subsubsection{Arjesh}

Adjarian's source: See \citeauthor{Eminian}, volume 2 (\armenian{Բ}.), page 323. 

\armenian{Ժամանակաց մէկին Հարճէշու մէջ Զիլանաձոր մէ մարդ մ՚կեղի։} 

\armenian{Էտ մարդ յ}̵\armenian{էլավ առավ ուր բեռ, գնաց սարի միջու ջաղաջ աղալու. հալա ջաղջի մօտ չը խասեր էր, ջաղջըպան դուս յ}̵\armenian{էլավ, չուան եզնից պրծուց, բեռն խուրճու պէս թալեց ուր շալակ, տարավ ներս։}


\begin{adjarianpage}\label{page:132}\end{adjarianpage}% should be 132


\armenian{Էդ մարդ օր էդ բան տեսավ, շատ վախէցավ, ասեց}.

– \armenian{Վալա, էսիկ ինձ էլ կսպանա, բեռն էլ կուտա, եզն էլ խետ. ապա ի՛նչ էնիմ, Աստված, օր էնպէս ա, ես պտի փախիմ։}

\armenian{Ջաղջպան բեռան բերանը քակեց, ցօրեն լցեց օղունի մէջ, իրիշկեց տէսավ օր բեռան տէր չէկավ, յ}̵\armenian{էլավ դուս, տէսավ օր էն մարդ կը փախի, բօռաց}.

– \armenian{Տօ՛ աղբէր, մի՛ փախի, արի՛ արի՛։}

\armenian{Էն մարդն էլ ասաց}. – \armenian{Տօ վալլա, ես քո ղուվաթ տէսա, քեզնէ վախէցա, դու քու Աստված. բեռն էլ քեզի, եզն էլ քեզի, ինձի բան մ՚ ասեր, ես թօղեմ էրթամ։} 

\subsubsection{Village of Arinjkus in Artske}

Adjarian's source: See \citeauthor{Eminian}, volume 4 (\armenian{Դ}.), page 201. 

\armenian{Դավրիշ ձուկ մ՚ կը բերա՛ կուտա պառվուն պախ։}

\armenian{Պառավին էլ իրեք խատ աղջիկ կեզի։}

– \armenian{Պա՛ռավ, կասա, առ զիմ ձուկ, ամանաթ պախա։}

\armenian{Պառավ կասա}. – \armenian{Խա, կը պախիմ, ամանաթ օր կա՝ ղրյամաթ ա։}

– \armenian{Պա՛ռավ, ասաց. յան էրկու, յան իրեք օրէն կուգյամ։}

– \armenian{Շուտ արի, օր անգյան գյաս, կը նեխի։}

\armenian{Դախրիշ տուեց ձուկ մըտ պառավ պախ, գնաց։}

\armenian{Պառավ ասաց}. – \armenian{Վէրցէք ձուկ, պախէք, չէղի զայ էնէք։}

\armenian{Էդա աղջիկներն ա, վէրցին էդ ձուկ պախէցին։}

\armenian{Աղջիկներ իշկէցին օր մ՚ էրկու օր դավրիշ չէկավ։}

\armenian{Քշեց չանք ամիս մ՚ էդ դավրիշ չէկավ։}

\armenian{Պառվու ջոջ աղջիկն ա, վէրցրուց ուր քուրվըտոց}.

– \armenian{Ա՛ղչի, յ}̵\armenian{էլի ձուկ բեր, ուտենք, պաս-ցամաք մէռանք։}

– \armenian{Ա՛ղջի, էսի ամանաթ ա, ամանաթ օր կա՝ ղըյամաթ ա։}

– \armenian{Էլի՛, բե՛ր, ուտենք, էնի մօռցեր ա։}

\armenian{Էլան ձուկ բէրին, իրեքով էլ կէրան։}

\armenian{Պառավ խաբար չէ աղջիկներ ձուկ կէրած էն։}

– \armenian{Մա՛րէ, քու աղջիկ դարվիշի ձուկ կէրավ, ասաց պզտի աղջիկ։}

– \armenian{Աղջի, ասաց, դու յ}̵\armenian{օրի՞ն կէրար, մենք ի՛նչ ջուղաբ պտի տանք։}



\begin{adjarianpage}\label{page:133}\end{adjarianpage}% should be 133

\subsubsection{Village of Dapavank in Khlat}

Adjarian's source: See \citeauthor{Eminian}, volume 2 (\armenian{Բ}.), page 376. 

\armenian{Թաքյավոր մ՚ կեղի, խետ մէկ լալէ մ՚. գէլին կէրթան կը պտօտեն. էդա թաքյավոր լէ ըսկի ավլադ ու թավլադ (որդիք) չունէր։ Կէրթան մէկ դավրէշի մ՚ ըռաստ կիւգյան։ Դավրէշ կասա}.

– \armenian{Թաքյավոր ապրած. ես գիտեմ դու ինչի շուռ կիւգյաս. տղէ մ՚ ապօվ շուռ կիւգյաս. արի քը մի խնձոր կուտամ, կէս դու կեր, կէս սուլթան թ՚ուտա. Աստված քը էրկու տղա կուտա. մէկ տղէն լէ խավլ էրա օր տաս ձի։}

\armenian{Թաքյավոր խավլ էրաց, էլավ գնաց ուր տուն. ինչ որ կէս ինք կէրավ, կէսն էտուր Սուլթանին։ Ին ամիս, ին դան, ին դագըգա մնաց, թաքյավորի կնիկ պարկյավ, էբեր ջուխթ մ՚ տղա։ Տղէք ջօջցան, էղան տաս տարէկան։ Աւուր մէկ դավրէշն էկավ, տէսավ օր էրկու տղէն հոլ կը խաին. ասաց}.

– \armenian{Կա ու չկա, էսոնք իմ խնձորի տղէկներն են։}

\armenian{Կանչեց զէն էրկու տղին. պստիկն չէկավ. ջոջն էբեր, ասաց}.

– \armenian{Արի, էրթանք, քյօ խօր տուն տո՛ւր ձի շանց։}

\armenian{Տղէն ընգյավ առջէվ, տարավ, ասաց}. – \armenian{Էսա իմ խօր տուն ա։} 

\armenian{Դավրէշ մտավ ինե, տղի խօր տուն սալամլըղ կապէցին առէչ, ասաց}. – \armenian{Թաքյավոր ապրած կէնա. դու քյօ խօսաց տէ՞րն ես։}

\armenian{Ասաց}. – \armenian{Խա, ես իմ խօսաց տէրն եմ։ Բէրեց զէրկու տղէն կայնէցուց դարվէշի առէջ, ասաց}. – \armenian{Վօ՛ր մէկ կը վէրուս, վէրցու։}

\subsubsection{Villages of New Bayazet}
Adjarian's source: I personally wrote down the samples from the New Bayazet villages during my summer travels in 1907. 

\paragraph{Yeranos village}

Adjarian's source: Migrated from the village of Khastur in Alashkert. 

\armenian{Արի էթանք մեր արա ջՙրինք. Խա՛չօ, գՙնա ջՙուր բՙի, արտ ջՙըրինք, արտն ըռթընա՛։ Էթանք չայիր, տէսնանք քաղէլու չէ՞։ Գՙացինք տէսանք քաղէլու չէր։ Անձրեվ գՙա, թըխ էրկընցու չայիր, էն վախտ կը քաղինք։ Լա՛օ, դՙո՞ւ իս գՙացէր իս կէրցուցէ չայիր։ Հա, յէս իմ գՙացէր իմ կէրցուցէ։ Օչխար քում, չայիր}... 


\begin{adjarianpage}\label{page:134}\end{adjarianpage}% should be 134

... \armenian{քում. տո՛ւրի, տանիմ մէկ լէ կէրցում։ Օր դու չկէրցուս, օ՞վ կէրցու։ Հէրու շատ անձրեվ էկավ, արտերն շատ էրկընցան, դա̈ն լէ չբՙըռնից, հաց լէ չէղավ, շախտէն տարավ}

\begin{center}
 * *
\end{center}


\armenian{Նիկօլա թաքավոր, Ճավճավաց յէրանալ առավ էլլըկ էկավ Գյօքջէն, հազար ութը հա̈րիր քսանը ութ թվնին. մզի բՙէրեց էստեղ. Ալաշկէրտու Խաստուրու բՙէրեց. իցա գՙեղ լէ տրվից մզի. օխտ տարի թարխնութեն տըվից. ձՙզի ասիմ՝ օխտ տարով յէտ կապեց մզի խարջ ու խարաջ։ Տալիզա մօվրօվ կար. էկավ էլլըք պահից. էտ խարջ կապեց մզիկ, յ}̵\armenian{ավալ կապեց մէ մանէթ, էրկու էրեց, իրեք, էտեվ չորս, էտեվ հինգ, էլավ չանքի հիմի տան գՙլօխ տարէկան տըսնըչորս մանէթ վացցուն կապէկ. էս սահաթէս մընք արքունական խարջ կիտանք։ Դՙառնանք մէշէք. մէշէքն էրից պօշլի (ռուս. մաքս}), \armenian{առավ միննէ (կամ նաեւ մզնէ՝ ի մէնջ) առաբին հիցցուն. հիմի դՙարձՙավ փէտին, էրից մի մանէթ. ծովէրէն լէ պօշլի կառնէ հիմի մզնէ։}

\paragraph{Adamxan} 


– \armenian{Քզի օր հօքի կէր՝ մկա դՙոպ շուտ էր մէռէ։}

– \armenian{Օր քօ հօքի էղնի՝ ընչի՞ մզի կը չարչըրես էդ ղդար. մզի ջՙուր չս ի՛տա, մեր արտեր ջՙրինք։}

– \armenian{Մէռնիմ Ասծու դիվնին. օր ախպէր լավ ըլնէր՝ մէկ լէ ուրին կըստէղծէր. մկա դՙու քու ախպօր քանդող իս։}

– \armenian{Դՙու փիս մարթ իս. քօ հօրօխբՙըրտիք զըկեր իս, կօղորդեր իս. կայներ իս կօրշնըվիս՝ գՙէղօվ էլ խաբուլ չինք։}

– \armenian{Օր դՙու լավ էղնէր քու ավպօրտոց հետ՝ քու բՙէրան կերյարա չէր ընգէ։}

– \armenian{Իմ չընգեր ա, քօն կ՚ընգնի։}

– \armenian{Իսա գեղ, ինա գեղ, հմէն լէ ընձմէ հա̈զ կէնին. քի պէս դՙէվ մարթ հա̈զ չէնա։}

\paragraph{Dzoragegh} 

\armenian{Յէրկրէն օր էկանք առաչ քառսունուչորս տուն էնք. հիմի դաֆթրին հարուր քսանը ութ տուն ինք. էտ հօղով չինք գյառնա կառավօրվինքյ. էղէր ինք մկա էրկու հարուր իծծուն}... 


\begin{adjarianpage}\label{page:135}\end{adjarianpage}% should be 135

... \armenian{տուն. ապրուստ մի չէղնի. խարջ շատըցե, խարաջ լէ շատըցե. թաքավոր լէ ծով խլեր ա ձՙէրնէ մզի. մզի փայ չկա. հարուր սէլ խոտ լէ կը կէրցու ձՙուկ առնող մուշտարի. բիրադի (բոլոր) Յէրէվանու, Քյավառա կիգ՚ան, յօթն ավուր ճամբախ կիգՙան, խոտ կը կէրցուն, էն հօղի խարջ լէ մենք կիտանք. մացեր ինք հէսիր։ Մէկ մէկ մարթ լէ թուրքեր կը գՙօղնան տանին սպանին։ Մկլօրսի սպանին մէ մարթ, էլ մի՛ հարցու։ Թաքավոր զմմէն լէ քրթէրուն, հայէրուն փարա տըվեր է, մզի չի՛ իտա։ Հէրու անուն մեր գեղ հաց չէ էկե. տեղ սար է. մրսեր, ցուրտ տարեր է. ժանգ լէ զարկեր է. սկի պտուղ չինք ստացե՛ր ի։}

\paragraph{Tsakkar} 

\armenian{Մզի Ճավճավ յէրանալը բէրեց էս գեղը. օխտը տարի խարջը չառավ. յէտօ դՙարձՙավ առավ տարին մէ մանէթ էրկու մանէթ, իրեք, չորս, խինքգ մանէթ. մկա լէ կառնի} 14 \armenian{մանէթ տնական։ Ծօվէրէն մզի զրկիր ի. ջՙախջՙընէրէն լէ կառնի փող, էրկու անգՙամ, համ գըլդին, համ դախօդ։ Էլման մէշէնէրէն մզի զրկիր ի. մէկ փէտ մզի վրա էրիր ի խինգ մանէթ. խաց լէ չէկավ. քամին քաշեց. հիմի չինք կա̈ռնա մըր աղէկներ պահինք. անձրեվ չի գՙա. չօրութեն տվիր ի մր վրէն։ Արազա ափնէրու թուրքեր էկած ին մր չայիրներ, մըր արտեր կէրցուցած ին. մզի նէղութեն կիտան. կիգՙան մզի վրա, մըր չայիր կը կէրցուն, մզի զիւլլօվ կը զանին։}

\paragraph{Gölköy} 

\armenian{Ահմէտ աղի գՙեղ մեր քյաֆշնի կից, Աթաշ՝ էլման քուրթի գՙեղ ա. մեր սար կը կէրցուն, մեր մալ կը գՙօղնան տանին։ Հո՜ւ կը վախէնանք. թվանք չունինք օր էրթանք կռիվ էնին, զընդոնց սպանինք, հո՜ւ հո՜ւ։ Թուրքն օխտ ավուր ճամբախ կիգՙա մեր մալի արօտ կը կէրցու. մեր պապական մուլք՝ օր Ալաշկէրտու էկէր ինք, մեր ձՙէրնէն ղլած ա. մկա մեր մալ կը փիտնա. մեր մալի խաթեր գՙացեր ինք, առեր ինք. յ}̵\armenian{օրէն հինգ կռիվ կէնինք. չընք իշխընա մօտէնանք։ Թվանք օր էղնի՝ մենք ընդոնց սկի չընք հաշվի մարթ. հմա օր չկա}… \armenian{կը վախէնանք կը փախնինք։ Տարէկան մզնէ տասը մալ ղլին տանին. քսան հատ լէ օչխար}... 


\begin{adjarianpage}\label{page:136}\end{adjarianpage}% should be 136

... \armenian{կը տանին։ Հաց օր չունինք. ճժեր լէ անօթնէ է օր մէկ կը մէռնին. լայլաջ ինք։ Մեր ծով լէ մեր ձՙէռնէն առած ա. մզի լէ փարա չկա օր գՙանգատ էնինք։ Գՙանգատ լէ էրեր ինք, չըն հա՛սու (ոչ հասուցանեն)։ յ}̵\armenian{էրէվընցին կիգՙա մեր արտեր կը կէրցու. օր էրթանք ընդոնց տավրին լէ մօտէնանք՝ մզի կը զարկին կսպանին. ծովը մերն ա կըսին. էրկու վէրստ տեղ ծօվու բՙէրնէն չուր էլնի վեր՝ մզի կը հասնի։}

\paragraph{Upper Adyaman} 

\armenian{Յէս իդա տէղաց ախջՙիկն իմ. իդա տէղաց լէ հարսն իմ. իդա տեղ լէ կարքըվեր իմ. իմ անուն լէ Սանամ. օր էղեր իմ՝ Չաչանա տարվա ճիժն իմ. ս՞կտ լէ ծէրացեր իմ. էլ մկա չմ կառնա բՙանի գՙօրծի էղնի. մարթ պտի օր ընձի պահա. էլ ի՛մալ ապրիմ օր ինձ ապրուստ չէղնի՝ չուր օրիկն ի մահուն։ Մեր լէզուն էդմալ ա։ Իմ մարթ վըր ընձի տըսնըհինգ տարի կէղնի օր մեծ ա։ Դէ, յէս ի՛նչ գՙինամ օր գՙալու ժամանակն չուրի Չաչանա կռիվ՝ ինչ խդար կը քաշա. չմ գՙինա։ Իմալ օր ռուսն էկե Ալաշկերտ, հօնգուց օր բՙարձՙած էկած իդա տեղ, յէս ի՛նչ գՙինամ ի՛նչ խդար ժամանակ կը քաշա օր Չաչանա կռիվ էղնի. մըտ (մօտ) ընձի յ}̵\armenian{ատնի չէ։ Խօ իմ գՙլօխ լէ չմ կա՛պի էդա սուն. բէլքի լուսուն լէ մէռա։ թըխ իմ անուն լէ մնա ախշըրքի էրես, թէ լավ թէ վատ իմ անուն լէ թըխ մնա մըչ ախշըրքին։}

\begin{center}
 * *
\end{center}


– \armenian{Դՙո՞ր կէրթաս։}

– \armenian{Կէրթամ յ}̵\armenian{արտ։}

– \armenian{Էն մարթ էն տղին կը խանչա. յ}̵\armenian{օրի՞ կը խանչա. ի՜նչ կըսա՝ ընձի թըխ ասա. էստէղէն յէս կը լսիմ։}

\paragraph{Lower Karanlug} 

Adjarian's source: Migrated from the village of Mangasar of Nahen. 



– \armenian{Օ՛նօ, յ}̵\armenian{ո՞ւստ գՙուգՙաս, դՙո՞ր կէրթաս։}

– \armenian{Տնէն կիգՙամ։}

– \armenian{Օ՛նօ, դՙու կյա̈ռնա՞ս մեր խին խօսքէրօվ ասես. թօխ էս պարօն գՙրի։}

\begin{adjarianpage}\label{page:137}\end{adjarianpage}% should be 137

– \armenian{Գօ յէս խալիվօրսեր իմ. էլ չեմ կյա̈ռնա բՙան ասեմ։}

– \armenian{Ընչի՞ չես կյա̈ռնա. հալա դՙու շատ կապրիս։}

– \armenian{Խօսքըս չընցնի. խարսի կուշտ խօսքըս չընցի. աղի կուշտ չընցի. ժամանակս անց կէցեր ա. գօ խօսքս օր տուն չը}X\armenian{ցավ՝ գՙէղի մէչ էլ սկի չընցի։ Մէ վախտ օր կէլնինք տանիս կը բօռինք «տղա՛, արի դՙաս», մէկի տեղ սաաղ գՙեղ ժօղնըվին կը գՙին։ Մկա իմ տղին էլ կանչեմ, կ՚ասեմ մէ թաս ջՙուր տու, էլ չի ուզա գՙալ մըտ ընձի։}

\paragraph{Avdalaghalu} 

Adjarian's source: Migrated from the village of Kopıgran that is near Mangasar. 

\armenian{Մենք Քօփղռանա էկեր էնք էստեղ. նէղութէնի խամար էնք էկը. գօ էստեղ շէնլըք էնք շիքե։ Մընք էկանք՝ Ավդալ աղան էր էս գՙէղացի. քուրթ էր ինք. էնի քօչեց գՙնաց. մենք իկանք նստանք էստեղ, առու խանէցինք։ Իմալ օր Մանկասարցու լէզուն ա, էդմալ էլ գօ մեր լէզուն ա։ Մենք ու Մանկասարցիք խնութ դՙըրկից էնք։ Գօ ինոնք էմալ օր էնդեղ ին նստած, մենք էլ էնդոնց կշտի խետ էստեղն էնք նստած։}

\paragraph{Alikrykh} 

\armenian{Մեր գՙէղացիք գՙնացինք նաչալնիկի մօտ. ասաց. դիվան կուզեմ։ Էլավ էկավ թալեց ճպօտի տակ, էտոր լավ ջՙարդՙեց. խօնջան կըտրավ. օր կտրավ՝ խելքը գՙնաց. քթէն խէղէղի պէս արուն պրծավ, լէզես. տէսանք օր խելքը գՙացեր ա, խօղաթըթախ ուր առանք գՙացինք յ}̵\armenian{ախբՙուր. էդ յ}̵\armenian{ախբՙրի դՙէմ ջՙուր թալինք վրէն. ջՙուր գՙնաց գՙնաց, աչքեր բՙացեց, սեվ սիվտակը դՙէղնավ։ Նոր վէրցինք դՙրէցինք ձՙիյանքնէրու վրէն, առանք էկանք գօ մեր տուն։ Տէր Մարգՙարն օրէնքեց, մնաց յօթն օր՝ ութն օր, գՙէրընդՙին դՙըրեց վըր թէվին, գՙնաց բիար (քաղելու}). \armenian{խզլարմամասի քաղեց, էրկու օր քաշեց, մըչ բիարին մէռավ։ Մկա քէզնէ ու ձՙէզնէ կը խարցում. սուչ վի՞ր կէղի։}


\begin{center}
 * *
\end{center}


– \armenian{Է՜, դՙաս արի, դՙո՞ր կէթաս. կայնէ յէս էլ գՙամ։}

– \armenian{Վռազցեր իմ. յէս գօ բՙան ունիմ, կը վազիմ կէթամ. էն տղէյներ գՙնացին. անգաջ չեն էնէ. էտոնց պիտի խասնիմ, գՙինա՞ս։}

\begin{adjarianpage}\label{page:138}\end{adjarianpage}% should be 138

\paragraph{Zolakhach} 

Adjarian's source: Migrated lagely from the village of Ziro and from Hamur. 


– \armenian{Մէլքոն, խին բՙանէրուց զրուցա, վարժապետ թօղ գՙրա։}

– \armenian{Տօ չմ գՙինա ինչ պատմեմ. մեր պապեր էկած ին էստեղ. թուրք էր նստուկ. թուրքէրուն խանէցին տարան Մազրի մահալ, մենք մացինք էստեղ։}

– \armenian{յ}̵\armenian{ո՞ւստ գՙաս։}

– \armenian{Տնէն։}

– \armenian{Էդ օծլուշաղի խետ ի՛նչ կը զրուցիր։}

– \armenian{Սկի. կասի քի մէ փարչ Քոլաղռանա բՙէրած Մարգՙարի տուն. մկա կուզեմ՝ չի իտան. կասեմ յ}̵\armenian{օրի՞ չս իտա. կասեն ձՙէզի պէտք ա, մէզի էլ պէտք ա։ Դէ, էստոք գՙրա. էսքան բօ՞լ ա։ Կուզես ուրիշ բՙան էլ ասեմ։ Մեր կանամփ խամող (անջուր) ա. ջՙուր կուզեմ, չն ի տա. Խաչօի մօտէն գՙնացի ուզէցի՝ չտվեց, քֆրեց, ասաց չմ ի տա. յ}̵\armenian{օրի՞ էս էկե վըր իմ ջՙրին։}

\paragraph{Lower Gyuzeldara} 

Adjarian's source: Migrated from Nahen, Gulasor, Ulikend, Kumlubucak, and Leter. 


– \armenian{Տղա՛, դՙո՞ր կէրթաս. դՙնա ղանչա խսամի Ավօյին գՙա̈։}

– \armenian{Բՙարիրիկուն, խէր ա. ընչի՞ էս ղանչէ, խնամի։}

– \armenian{Էրթանք մեր տղին ախջՙիկ ուզենք։}

– \armenian{Ասված աջՙօղա։}

\armenian{Էլան գՙնացին հարէվանի տուն։}

– \armenian{Բՙարի յ}̵\armenian{իրիկուն. բՙարօվ էկաք. նստէք. խէր ա էս վախտ ձՙեր գՙալ։}

– \armenian{Խէր ա, փառք Աստուծու. դՙու ընձնից խարցու. էկեր էնք բՙարէկամութեն կը խնդՙրենք. ախջՙիկդ տու մեր տղին։ Քէզնէ լավ մարդՙ չենք կառնա դՙըտնի։}

– \armenian{Դէ վօր էդմալ ա, ձՙեռդ բՙե պաչեմ։}

– \armenian{Տղա՛, բուտուլկէք դՙրէք, քէֆ անենք։}


\paragraph{Upper Gyuzeldara} 

Adjarian's source: Migrated from the village of Iritsu, from the village of Vanki, from Korun, from Musun, and from Ardzap. 

\armenian{Ժօղվեր ինք, ըսէցինք. արէք էրթանք զօզան. կէս մ՚ըսին չենք իգՙա, կէս մ՚ըսին կիգՙանք. յէս լէ ինադ էրէցի, ըսի կէրթամ։} ... 


\begin{adjarianpage}\label{page:139}\end{adjarianpage}% should be 139

\armenian{Մկա կըսեմ թէ չէրթամ. մեր կընկըտիք լէ խայիլ չեն. կըսին մենք չկրնանք էրթանք, չընք է՛րթա սար բէր. չընք կառնա, հէռուն ա ճամբախ. օխչար կթինք ու բՙէրինք հա էդա տե՞ղ։ Մկա կըսեն. բՙարձՙենք մեր տներ՝ էրթանք մեր զօզանատեղ. էն վախտ մզիկ ըռահաթ կէղնի. մէ ամիս էրկու կը մնանք էնդեխ, էրկու ամսով յէտօ կիջՙնենք կիգՙանք մեր գՙեղ։}


\paragraph{Zaghalu} 

Adjarian's source: Migrated from the village of Yoncalı in Mush. 

\armenian{Յէս ըսէցի. տէրտէր, արի էրթանք սար, քօլիկըմ ծածկա, մանինք մէչ։ Վո՞ր տեղ էրթանք. խէրու գՙացինք Նէրքի-Խարանլուղի սար. էս տարի՞ դՙո՛ր էրթանք։ Տէրտէրն ասաց. արի, ըս տարի լէ էրթանք Վէրի-Գօզալդարա. մէզի կըսեն էնտեղ խօվ ա. էն տէղաց մարթ լավ ա, աղէկ ա. մեր յէրկրի մարթ ա. էնդոք լէ մեր յէրկրէն էլ էկե. ինչքան չէղի մեր պատիվ կը պախեն. մէզի լավ աչքօվ կիշկեն։ Խոտ լէ ատնօվ ա. կըսեն յէղ լէ շատ կէղնի։ Մկա տէրտէր ընձի բՙէրեր ա էստեղ, ինք լէ թօրկէ գՙացե։}

\subsection{Note on migration}\label{section:Mush:text:migrationNote}

Note: Of the remaining villages of New Bayazet, the Mush dialect also contains Upper and Lower Kyolaghran, Lower Aluchalu and Gedakbulag. Kyolaghran migrated from Nahen, Yoncalı and Krakom; Aluchalu migrated from the Bayazit village of Çakırbey, from Van and Maku; while the people of Gedakbulag from Leter, Mush and Khlat.


The three have the sound /χ/ <\armenian{խ}> instead of /h/ <\armenian{հ}>, and the conjunction /le/ <\armenian{լէ}> instead of /ɑl/ <\armenian{ալ}> `also'. They use the present forms like in Table \ref{tab:Mush:sampleNote:mono}. 


\begin{table}[H]
 \centering
 \caption{Monosyllabic verbs in villages of the Mush dialect}
 \label{tab:Mush:sampleNote:mono}
 \begin{tabular}{|l| ll| ll|}
 \hline & \multicolumn{2}{l|}{Mush villages} & \multicolumn{2}{l|}{cf. SWA} \\ 
 `I come' & k-iɡʰ-ɑ-m & \armenian{կիգՙամ} & ɡu-kʰ-ɑ-m & \armenian{կու գամ} \\
 `you.{\sg} come' & k-iɡʰ-ɑ-s & \armenian{կիգՙաս} & ɡu-d-ɑ-s & \armenian{կու գաս} \\
 `he comes' & k-iɡʰ-ɑ-n & \armenian{կիգՙան} & ɡu-l-ɑ-n & \armenian{կու գան} \\
 `I give' & k-it-ɑ-m & \armenian{կիտամ} & ɡu-d-ɑ-m & \armenian{կու տամ} \\
 & \multicolumn{2}{l|}{{\ind}-$\sqrt{}$-{\thgloss}-{\agr}} & \multicolumn{2}{l|}{{\ind}-$\sqrt{}$-{\thgloss}-{\agr}} \\
 \hline 
\end{tabular}
\end{table} 


The copular verb in the present 3SG is /ɑ/ <\armenian{ա}> `is'. The first conjugation class ends in /il/ <\armenian{իլ}> (Table \ref{tab:Mush:sampleNote:il}).


\begin{table}[H]
 \centering
 \caption{Verbs with /-il/ in villages of the Mush dialect}
 \label{tab:Mush:sampleNote:il}
 \begin{tabular}{|l| ll| ll|}
 \hline & \multicolumn{2}{l|}{Mush villages} & \multicolumn{2}{l|}{cf. SWA} \\ 
 `they drink' & kə χm-i-n & \armenian{կը խմին} & ɡə χəm-e-n & \armenian{կը խմեն} \\
 & \multicolumn{2}{l|}{{\ind}-$\sqrt{}$-{\thgloss}-3{\pl}} & \multicolumn{2}{l|}{{\ind}-$\sqrt{}$-{\thgloss}-3{\pl}} \\
 \hline 
\end{tabular}
\end{table} 



The ablative case uses the formative /-en/ <\armenian{էն}>, but the formative /-it͡sʰ/ <\armenian{ից}> is also used.

The village of Tyuskyulyu migrated from the Arjesh villages of Gandzak, Zirekli, and the Mush villages of Hadgon, Lez, Malakand; it sufficiently differs from the others because, like Julfa, it uses the copular verb with the vowel /ɑ/ <\armenian{ա}>.

\begin{exe}
 \ex Mush villages \label{sent:Mush:textSample:village:Aux}
 \begin{xlist}
 \ex \gll bʰeɾ-i̯eɾ ɑ \\
 bring-{\perfcvb} {\aux} \\
 \trans `He has brought.'\\
 \armenian{բՙէրէր ա}
 \ex \gll ɡʰɑt͡sʰ-ɑt͡s ɑ-n \\
 go-{\rptcp} {\aux}-3{\pl} \\
 \trans `They have gone.'\\
 \armenian{գՙացած ան}
 \ex \gll ɑɾtʰ-ɑ-nkʰ \\
 go-{\thgloss}-1{\pl} \\
 \trans `We go (subjunctive).'\\
 \armenian{արթանք}

 \end{xlist}
\end{exe}


Besides these, it has the sound /h/ <\armenian{հ}> instead of /χ/ <\armenian{խ}>; while its ablative is /-en/ <\armenian{էն}> and /-it͡sʰ/ <\armenian{ից}>. For the remaining points, they are the same as the last branch of Mush. 




\chapter{Van}

\section{Background}

\begin{adjarianpage}\label{page:140}\end{adjarianpage}% should be 140

The dialect of Van is spoken on the entire Eastern shore of the Van sea.\footnote{\translatorHD{Adjarian tends to call Lake Van and Lake Sevan as seas <\armenian{ծով}> instead of lakes <\armenian{լիճ}>. }} Its center is the great city of Van and its multiple surrounding Armenian villages. It spreads from the north until Diyadin, the western side of Bayazit, from the south to Moks, Vozim, Shatakh and Aghbak or Başkale, from the east until the border of Persia, and from the west it scratches the borders of the Mush dialect. As of now, the dialect of Van (especially the Moks subdialect) is the western border-line of the Armenian language, beyond which the Armenians are Kurdish-speaking or Arabic-speaking. During the time of the last Russo-Turkish war, a large number of Armenians from Diyadin migrated to the Caucasus, where they built the village of Basargechar on the south-eastern banks of the Sevan sea. Now, in that same region, it is a great and rich town (\armenian{աւան}), and has essentially taken the image of a city. 

\section{Phonology}

\subsection{Segment inventory}\label{sec:Van:phono:segment}
The sound system of the Van dialect contains 46 sounds, which are divided as follows.

There are 12 vowels (Table \ref{tab:Van:phono:segment:vowels}). 


\begin{table}[H]
 \centering
 \caption{Vowels of the Van dialect}
 \label{tab:Van:phono:segment:vowels}
 \begin{tabular}{|ll l l|}
 \hline 
/i/ <\armenian{ի}> & /ʏ/ <\armenian{իւ}>& & /u/ <\armenian{ու}> 
\\
 /i̯e/ <\armenian{ե}> & /œə̯/ <\armenian{է}\`\armenian{օ}> & & /u̯o/ <\armenian{ո}>
 \\
/e/ <\armenian{է}> & /œ/ <\armenian{էօ}> & /ə/ <\armenian{ը}> & /o/ <\armenian{օ}>
 \\
/æ/ <\armenian{ա}̈> & & & /ɑ/ <\armenian{ա}> 
 \\ \hline 
 \end{tabular}
\end{table}


It has 34 consonants (Table \ref{tab:Van:phono:segment:cons}). 


\begin{table}[H]
 \centering
 \caption{Consonants of the Van dialect}
 \label{tab:Van:phono:segment:cons}
 \begin{tabular}{|l|lll|llll|lll|}
 \hline 
 & \multicolumn{3}{l|}{Labial}& \multicolumn{4}{l|}{Coronal}& \multicolumn{3}{l|}{Dorsal/Back}\\
 Stops& /b/ & /p/ & /pʰ/ & /d/ & /t/ & /tʰ/& & /ɡ/ & /k/ & /kʰ/ 
 \\
 & <\armenian{բ}> &<\armenian{պ}>& <\armenian{փ}> &<\armenian{դ}>& <\armenian{տ}> &<\armenian{թ}>&& <\armenian{գ}>& <\armenian{կ}>& <\armenian{ք}>\\
 & & & & & & && /ɡʲ/ & /kʲ/ & /kʰʲ/ \\
 & & & && & && <\armenian{գյ}>& <\armenian{կյ}>& <\armenian{քյ}>\\
 \hline 
 Affricates & && & /d͡z/ & /t͡s/ & /t͡sʰ/ & && & \\
 & && &<\armenian{ձ}>& <\armenian{ծ}>& <\armenian{ց}> & & & & \\
 & && & /d͡ʒ/ & /t͡ʃ/ & / t͡ʃʰ/ && & & \\
 & & & &<\armenian{ջ}>& <\armenian{ճ}>& <\armenian{չ}> & & && \\
 \hline 
 Fricatives& /f/&/v/& &/s/& /z/& /ʃ/& /ʒ/& /χ/ & /ʁ/ & /h/ \\
 & <\armenian{ֆ}>&<\armenian{վ}>& & <\armenian{ս}>& <\armenian{զ}>& <\armenian{շ}>& <\armenian{ժ}>& <\armenian{խ}> & <\armenian{ղ}> & <\armenian{հ}> \\
 && & & & & & & & & /hʲ/ \\
 && & & & & & & & & <\armenian{հյ}>
\\ \hline 
 Sonorants & /m/ & /n/& & /ɾ/ & /r/& /l/ & /j/ && & \\
& <\armenian{մ}> & <\armenian{ն}> && <\armenian{ր}>& <\armenian{ռ}>& <\armenian{լ}>& <\armenian{յ}> && & 
\\ \hline 
 \end{tabular}
\end{table}

Among vowels, the sounds /i̯e, u̯o, œə̯/ <\armenian{ե ո է}\`\armenian{օ}> are notable. The first two are same as the sounds /i̯e, u̯o/ <\armenian{ե,ո}> for the people of Mush or rural Karin; but they are not as heavy and slow as them, but are pronounced faster. The sound <\armenian{է}\`\armenian{օ}> is pronounced also like the sequence \armenian{էօը} (Adjarian: <öə>, IPA: /œə̯/), but faster and it can be considered a diphthong. 

\subsection{Sound changes}
For sound changes, the following are the most significant conditions. 


\begin{adjarianpage}\label{page:141}\end{adjarianpage}% should be 141

\subsubsection{Monophthong vowels}
\paragraph{Classical Armenian /ɑ/ <\armenian{ա}> }\label{sec:Van:phono:soundchange:monovowel:a}


Classical Armenian /ɑ/ <\armenian{ա}> has changed to /ɑ/ <\armenian{ա}>. In the Van dialect, especially under stress, this vowel is pronounced closed, almost like the <a> vowel of the English word <all>.\footnote{\translatorHD{The prose is unclear, but I think he means that that this vowel is rounded.}} In many places it is changed to /æ/ <\armenian{ա}̈>, but there is no general rule for this. What's clear is only that after the sound /v/ <\armenian{վ}>, it always changes to /æ/ <\armenian{ա}̈>, and even the name of the city `Van' (Table \ref{tab:Van:phono:change:vowel:a:ae}).

\begin{table}[H]
 \centering
 \caption{Change from Classical Armenian /ɑ/ <\armenian{ա}> to /æ/ <\armenian{ա}̈> in the Van dialect}
 \label{tab:Van:phono:change:vowel:a:ae}
 \begin{tabular}{|l|ll|ll|ll|}
 \hline & \multicolumn{2}{l|}{Classical Armenian}& \multicolumn{2}{l|}{> Van }& \multicolumn{2}{l|}{cf. SEA }
 \\
 `fear' & vɑχ& \armenian{վախ} & væχ & \armenian{վա̈խ} & vɑχ & \armenian{վախ} \\
 `to wash' & lu̯ɑnɑl& \armenian{լուանալ} & lvæl, vlæl & \armenian{լվա̈լ, վլա̈լ} & ləvɑnɑl, ləvɑl & \armenian{լվանալ, լվալ} \\
 `wheat-meal' & d͡zɑ{wɑ}ɾ& \armenian{ձաւար} & t͡sævæɾ & \armenian{ծա̈վա̈ր} & d͡zɑvɑɾ & \armenian{ձավար} \\
 `to run' & vɑzel& \armenian{վազել} & væzi̯el & \armenian{վա̈զել} & vɑzel & \armenian{վազել} \\
 `tomorrow' & vɑɬ& \armenian{վաղ} & væʁ & \armenian{վա̈ղ} & vɑʁ-ə (-{\defgloss}) & \armenian{վաղը} \\
 `curtain' & vɑɾɑɡoi̯ɾ& \armenian{վարագոյր} & væɾækʲur & \armenian{վա̈րա̈կյուռ} & vɑɾɑkʰujɾ & \armenian{վարագույր} \\
 `vardapet (archimandrite)' & vɑɾdɑpet& \armenian{վարդապետ} & væɾtæpi̯et & \armenian{վա̈րտա̈պետ} & vɑɾtʰɑpet & \armenian{վարդապետ} \\
 `fee' & vɑɾd͡z-(ə?)kʰ (-{\pl})& \armenian{վարձք} & vært͡sʰkʰ & \armenian{վա̈ռցք} & vɑɾt͡sʰkʰ & \armenian{վարձք} \\
 `Van' & vɑn & \armenian{Վան} & væn & \armenian{Վա̈ն} & vɑn & \armenian{Վան} \\
 \hline
 \end{tabular}
 
\end{table}

In very few cases, the Classical vowel /ɑ/ <\armenian{ա}> also changes to /e, i̯e, je, i, œ, ʏ, o, ə/ <\armenian{է, ե, յէ, ի, էօ, իւ, օ, ը}>. Such words are one or a handful; they are the result of exceptional phonetic rules. 

\paragraph{Classical Armenian /e/ <\armenian{ե}> }


In Van, Classical Armenian /e/ <\armenian{ե}> has changed to /ji̯e/ <\armenian{յե}> in the beginning of monosyllabic words. At the beginning of polysyllabic words, it can turn to /ji̯e/ <\armenian{յե}> or /e/ <\armenian{է}> (Table \ref{tab:Van:phono:change:vowel:e:alot}).

\begin{table}[H]
 \centering
 \caption{Change from Classical Armenian /e/ <\armenian{ե}> to /ji̯e, e/ <\armenian{յե, է}> in the Van dialect}
 \label{tab:Van:phono:change:vowel:e:alot}
 \begin{tabular}{|l|ll|ll|ll|}
 \hline & \multicolumn{2}{l|}{Classical Armenian}& \multicolumn{2}{l|}{> Van }& \multicolumn{2}{l|}{cf. SEA }
 \\
`thirty' &eɾesun& \armenian{երեսուն} & ji̯ersun &\armenian{յեռսուն} & jeɾesun& \armenian{երեսուն} \\
 \armenian{՝}iron' & eɾkɑtʰ & \armenian{երկաթ} & eɾkɑtʰ & \armenian{էրկաթ} & jeɾkɑtʰ & \armenian{երկաթ} \\
 \hline
 \end{tabular}
 
\end{table} 


For words with a repeated Classical /e/ <\armenian{ե}>, some of changed them to /i/ <\armenian{ի}> (Table \ref{tab:Van:phono:change:vowel:e:rep}).

\begin{table}[H]
 \centering
 \caption{Change from repeated Classical Armenian /e/ <\armenian{ե}> to /i/ <\armenian{ի}> in the Van dialect}
 \label{tab:Van:phono:change:vowel:e:rep}
 \begin{tabular}{|l|ll|ll|ll|}
 \hline & \multicolumn{2}{l|}{Classical Armenian}& \multicolumn{2}{l|}{> Van }& \multicolumn{2}{l|}{cf. SEA }
 \\
 `three' &eɾekʰ & \armenian{երեք} & iɾi̯ekʰʲ & \armenian{իրեքյ} &jeɾekʰ & \armenian{երեք} \\
 `to cook' & epʰel & \armenian{եփել} & ipʰi̯el & \armenian{իփել} & jepʰel & \armenian{եփել} \\
 \armenian{՝}face' & eɾes & \armenian{երես} & iɾi̯es & \armenian{իրես}& jeɾes & \armenian{երես} \\
 `evening'& eɾekoi̯ & \armenian{երեկոյ} & iɾikun & \armenian{իրիկուն} & jeɾeko & \armenian{երեկո} \\
\hline
 \end{tabular}
 
\end{table} 

In the final syllable, the Classical sound /e/ <\armenian{ե}> becomes /i̯e/ <\armenian{ե}> (Table \ref{tab:Van:phono:change:vowel:e:ie}).

\begin{table}[H]
 \centering
 \caption{Change from final Classical Armenian /e/ <\armenian{ե}> to /i̯e/ <\armenian{ե}> in the Van dialect}
 \label{tab:Van:phono:change:vowel:e:ie}
 \begin{tabular}{|l|ll|ll|ll|}
 \hline & \multicolumn{2}{l|}{Classical Armenian}& \multicolumn{2}{l|}{> Van }& \multicolumn{2}{l|}{cf. SEA }
 \\
`friend' & ənkeɾ & \armenian{ընկեր} & inɡʲi̯eɾ & \armenian{ինգյեր} & əŋkeɾ & \armenian{ընկեր} \\ 
\hline
 \end{tabular}
 
\end{table} 

Word-medial /e/ <\armenian{ե}> changes to /e/ <\armenian{է}> or /i̯e/ <\armenian{ե}> (Table \ref{tab:Van:phono:change:vowel:e:mid}).

\begin{table}[H]
 \centering
 \caption{Change from word-medial Classical Armenian /e/ <\armenian{ե}> to /e, i̯e/ <\armenian{է, ե}> in the Van dialect}
 \label{tab:Van:phono:change:vowel:e:mid}
 \begin{tabular}{|l|ll|ll|ll|}
 \hline & \multicolumn{2}{l|}{Classical Armenian}& \multicolumn{2}{l|}{> Van }& \multicolumn{2}{l|}{cf. SEA }
 \\
`mouth' &beɾɑn & \armenian{բերան} & peɾɑn & \armenian{պէրան} &beɾɑn & \armenian{բերան} \\
 `ground' & ɡetin & \armenian{գետին}& kʲi̯etin &\armenian{կյետին} & ɡetin& \armenian{գետին} \\
\hline
 \end{tabular}
 
\end{table} 

The latter sound change in specific is the opposite from the Mush dialect, where the sounds /i̯e/ <\armenian{ե}> or /u̯o/ <\armenian{ո}> can exist only in the final syllable according to \citeauthor{Mserianz-1899-Mush} (\armenian{Մսերեան}). 

\paragraph{Classical Armenian /ē/ <\armenian{է}>} 

The Classical sound /ē/ <\armenian{է}> always changes to /e/ <\armenian{է}>. It changes to /i/ <\armenian{ի}> only in the words in Table \ref{tab:Van:phono:change:vowel:ee:i}.

\begin{table}[H]
 \centering
 \caption{Change from Classical Armenian /ē/ <\armenian{է}> to /e/ <\armenian{է}> in the Van dialect}
 \label{tab:Van:phono:change:vowel:ee:i}
 \begin{tabular}{|l|ll|ll|ll|}
 \hline & \multicolumn{2}{l|}{Classical Armenian}& \multicolumn{2}{l|}{> Van }& \multicolumn{2}{l|}{cf. SEA }
 \\
`gum' &χēʒ & \armenian{խէժ} & χiʒ & \armenian{խիժ} &χeʒ & \armenian{խեժ} \\
`Sunday' & kiɾɑkē, kiu̯ɾɑkē & \armenian{կիրակէ, կիւրակէ}& kiɾɑki & \armenian{կիրակի} & kiɾɑki & \armenian{կիրակի} \\ 
\hline
 \end{tabular}
 
\end{table} 


\paragraph{Classical Armenian /i/ <\armenian{ի}>} 

The Classical sound /i/ <\armenian{ի}> is usually preserved, but there it has become /e/ <\armenian{է}> in a few words, as well as /i̯e/ <\armenian{ե}> or /ʏ/ <\armenian{իւ}> (Table \ref{tab:Van:phono:change:vowel:i:stuff}).

\begin{table}[H]
 \centering
 \caption{Change from Classical Armenian /i/ <\armenian{ի}> to /i, e, i̯e, ʏ/ <\armenian{ի, է, ե, իւ}> in the Van dialect}
 \label{tab:Van:phono:change:vowel:i:stuff}
 \begin{tabular}{|l|ll|ll|ll|}
 \hline & \multicolumn{2}{l|}{Classical Armenian}& \multicolumn{2}{l|}{> Van }& \multicolumn{2}{l|}{cf. SEA }
 \\
`nine' & inən & \armenian{ինն} & in & \armenian{ին} & inən & \armenian{ինն} \\
`fifty' & jisun & \armenian{յիսուն} & isun & \armenian{իսուն} & hisun & \armenian{հիսուն} \\
`bed' & ɑnkoɬin & \armenian{անկողին} & ɡʲoʁvenkʰʲ & \armenian{գյօղվէնքյ} & ɑŋkoʁin & \armenian{անկողին} \\
`poop' & t͡siɾt & \armenian{ծիրտ} & t͡si̯ert & \armenian{ծեռտ} & t͡siɾt, t͡seɾt & \armenian{ծիրտ, ծեռտ} \\
`balance' & kəʃir-kʰ (-{\pl}) & \armenian{կշիռք} & kəʃi̯erkʰʲ & \armenian{կըշեռքյ} & kəʃirkʰ & \armenian{կշիռք} \\
`other' & uɾiʃ & \armenian{ուրիշ} & ʏɾʏʃ & \armenian{իւրիւշ} & uɾiʃ & \armenian{ուրիշ} \\ 
\hline
 \end{tabular}
 
\end{table} 

\paragraph{Classical Armenian /o/ <\armenian{ո}>} 

The Classical sound /o/ <\armenian{ո}> changes to /vu̯o/ <\armenian{վո}> in the beginning of monosyllabic words, to /vu̯o/ <\armenian{վո}> in the beginning of a large number of polysyllabic words, and in some places to /o/ <\armenian{օ}> (Table \ref{tab:Van:phono:change:vowel:o:stuff}).\footnote{\translatorHD{The reconstructed ancestor for `widow' is my own.}}

\begin{table}[H]
 \centering
 \caption{Change from Classical Armenian /o/ <\armenian{ո}> to /vu̯o, o/ <\armenian{վո, օ}> in the Van dialect}
 \label{tab:Van:phono:change:vowel:o:stuff}
 \begin{tabular}{|l|ll|ll|ll|}
 \hline & \multicolumn{2}{l|}{Classical Armenian}& \multicolumn{2}{l|}{> Van }& \multicolumn{2}{l|}{cf. SEA }
 \\
`resentment' & oχ & \armenian{ոխ} & voχ & \armenian{վոխ} & voχ & \armenian{ոխ} \\
`who' & ov & \armenian{ով} & vu̯ov & \armenian{վով} & ov & \armenian{ով} \\ 
`male' &oɾd͡z & \armenian{որձ} & vu̯ort͡s & \armenian{վոռց} &voɾt͡sʰ & \armenian{որձ} \\
`gold' & oski & \armenian{ոսկի} & vu̯oski & \armenian{վոսկի} & voski& \armenian{ոսկի} \\
`widow' & *oɾbe{wɑ}i̯ɾi & *\armenian{որբեւայրի} & vu̯orpœveɾi & \armenian{վոռպէօվէրի} & voɾpʰevɑjɾi& \armenian{որբեւայրի} \\
`gold' & oski & \armenian{ոսկի} & vu̯oski & \armenian{վոսկի} & voski& \armenian{ոսկի} \\
`to inundate' & oɬoɬel& \armenian{օղօղել} & oʁoʁi̯el & \armenian{օղօղել} & voʁoʁel& \armenian{ողողել} \\
 \armenian{՝}to take pity on' & oɬoɾmil & \armenian{ողորմիլ} & oʁoɾmi̯el & \armenian{օղօրմել} & voʁoɾmel & \armenian{ողորմել} \\
\hline
 \end{tabular}
 
\end{table} 

We individual examples of changes to /u, œ/ <\armenian{ու, էօ}> (Table \ref{tab:Van:phono:change:vowel:o:other}). 

\begin{table}[H]
 \centering
 \caption{Change from Classical Armenian /o/ <\armenian{ո}> to /u, œ/ <\armenian{ու, էօ}> in the Van dialect}
 \label{tab:Van:phono:change:vowel:o:other}
 \begin{tabular}{|l|ll|ll|ll|}
 \hline & \multicolumn{2}{l|}{Classical Armenian}& \multicolumn{2}{l|}{> Van }& \multicolumn{2}{l|}{cf. SEA }
 \\
`bone' &oskəɾ & \armenian{ոսկր} & usku̯or & \armenian{ուսկոռ} & voskoɾ & \armenian{ոսկոր} \\
`to twist' &oloɾel & \armenian{ոլորել} & œɾlœɾi̯el & \armenian{էօլէօրել} & voloɾel& \armenian{ոլորել} \\
\hline
 \end{tabular}
 
\end{table} 

Word-medially, most of the time, the Classical sound /o/ <\armenian{ո}> becomes /u̯o/ <\armenian{ո}>, in both the final and pre-final syllables. But there are many examples where it has also changed to /o, œə̯, œ, u/ <\armenian{օ, է}\`\armenian{օ, էօ, ու}> (Table \ref{tab:Van:phono:change:vowel:o:eo}). The sound /œə/ <\armenian{է}\`\armenian{օ}> exists only in the final syllable. 



\begin{adjarianpage}\label{page:142}\end{adjarianpage}% should be 142


\begin{table}[H]
 \centering
 \caption{Change from Classical Armenian /o/ <\armenian{ո}> to /o, œ, œə̯, u/ <\armenian{օ, է}\`\armenian{օ, էօ, ու}> in the Van dialect}
 \label{tab:Van:phono:change:vowel:o:eo}
 \begin{tabular}{|l|ll|ll|ll|}
 \hline & \multicolumn{2}{l|}{Classical Armenian}& \multicolumn{2}{l|}{> Van }& \multicolumn{2}{l|}{cf. SEA }
 \\
`gray-haired' & ɑle{wo}ɾ & \armenian{ալեւոր} & χɑlivu̯oɾ & \armenian{խալիվոր} & ɑlevoɾ & \armenian{ալևոր} \\ 
`leaven (CA); dough (SEA)' & χəmoɾ & \armenian{խմոր} & χəmu̯oɾ & \armenian{խըմոր} & χəmoɾ & \armenian{խմոր} \\ 
`earth' &hoɬ & \armenian{հող} & χu̯oʁ & \armenian{խող} & hoʁ & \armenian{հող} \\
`wheat' &t͡sʰoɾe̯ɑn & \armenian{ցորեան} & t͡sʰu̯oɾen & \armenian{ցորէն} & t͡sʰoɾen& \armenian{ցորեն} \\
`flour of parched corn' &pʰoχind & \armenian{փոխինդ} & pʰu̯oχind & \armenian{փոխինդ} & pʰoχind& \armenian{փոխինդ} \\
`prostitute' &boz & \armenian{բոզ} & pœə̯z & \armenian{պէ}\`\armenian{օզ} & boz& \armenian{բոզ} \\
 \armenian{՝}work' & ɡoɾt͡s & \armenian{գործ}& kʲœə̯rt͡s & \armenian{կյէ}\`\armenian{օռծ} & ɡoɾt͡s & \armenian{գործ} \\
`frog' &ɡoɾt & \armenian{գորտ} & kʲœə̯rt & \armenian{կյէ}\`\armenian{օռտ} & ɡoɾt& \armenian{գորտ} \\
`to assemble' &ʒoɬovel & \armenian{ժողովել} & ʒoʁvi̯el & \armenian{ժօղվել} & ʒoʁovel, ʒoʁvel& \armenian{ժողովել, ժողվել} \\
`to bathe' &loɡɑnɑl & \armenian{լոգանալ} & loχknɑl & \armenian{լօխկնալ} & loɡɑnɑl& \armenian{լոգանալ} \\
`shepherd' &hoviu̯ & \armenian{հովիւ} & χoviv & \armenian{խօվիվ} & hoviv& \armenian{հովիվ} \\
`rooster' &ɑkʰɑɬɑɬ & \armenian{աքաղաղ} & ɑhlœɾ & \armenian{ահլէօր} & ɑkʰloɾ& \armenian{աքլոր} \\
`buffalo' &ɡomēʃ & \armenian{գոմէշ} & ɡʲœmeʃ & \armenian{գյէօմէշ} & ɡomeʃ& \armenian{գոմեշ} \\
`madder' &toɾ\'on & \armenian{տորոն} & tuɾun & \armenian{տուրուն} & toɾ\'on & \armenian{տորոն} \\
\hline
 \end{tabular}
 
\end{table} 

\paragraph{Classical Armenian /u/ <\armenian{ու}> }


Before a consonant, the Classical sound /u/ <\armenian{ու}> changed to /u/ <\armenian{ու}> at the beginning of words, to /u/ <\armenian{ու}> or /ʏ/ <\armenian{իւ}> in the middle or end of words. While before a vowel, it is always /v/ <\armenian{վ}> (Table \ref{tab:Van:phono:change:vowel:u:stuff}).



\begin{table}[H]
 \centering
 \caption{Change from Classical Armenian /u/ <\armenian{ու}> to /u, ʏ, v/ <\armenian{ու, իւ, վ}> in the Van dialect}
 \label{tab:Van:phono:change:vowel:u:stuff}
 \begin{tabular}{|l|ll|ll|ll|}
 \hline & \multicolumn{2}{l|}{Classical Armenian}& \multicolumn{2}{l|}{> Van }& \multicolumn{2}{l|}{cf. SEA }
 \\
`camel' & uɬt & \armenian{ուղտ} & uχt & \armenian{ուղտ} & uχt & \armenian{ուղտ} \\ 
`cold' & t͡sʰuɾt & \armenian{ցուրտ} & t͡sʰurt & \armenian{ցուռտ} & t͡sʰuɾt & \armenian{ցուրտ} \\ 
`snow-storm' & bukʰ & \armenian{բուք} & pʏkʰʲ & \armenian{պիւքյ} & bukʰ & \armenian{բուք} \\ 
`spring' &ɡɑɾun & \armenian{գարուն} & kʲæɾuʏ & \armenian{կյա̈րիւն} & ɡɑɾun & \armenian{գարուն} \\
`you.{\sg}.{\nom}' &du & \armenian{դու} & tʏ & \armenian{տիւ} & du & \armenian{դու} \\
`mulberry' &tʰutʰ & \armenian{թութ} & tʰʏtʰ & \armenian{թիւթ} & tʰutʰ & \armenian{թութ} \\ 
`tongue' & lezu & \armenian{լեզու} & lezʏ & \armenian{լէզիւ} & lezu & \armenian{լեզու} \\ 
`flea' &lu & \armenian{լու} & lʏ & \armenian{լիւ} & lu & \armenian{լու} \\ 
 `to wash' & lu̯ɑnɑl& \armenian{լուանալ} & ləvæl, vəlæl & \armenian{լըվա̈լ, վըլա̈լ} & ləvɑnɑl, ləvɑl & \armenian{լվանալ, լվալ} \\
\hline
 \end{tabular}
 
\end{table}

\subsubsection{Diphthong changes}
In diphthongs, we note the following sound changes. 

\paragraph{Classical Armenian /ɑi̯/ <\armenian{այ}>}
 

The Classical sound /ɑi̯/ <\armenian{այ}> changes to /e/ <\armenian{է}> next to a consonant; but when the next syllable has the vowel /i/ <\armenian{ի}>, the diphthong */ɑi̯/ <\armenian{այ}> also becomes /i/ <\armenian{ի}>. Before vowels, /ɑi̯/ <\armenian{այ}> stays /ɑj/ <\armenian{այ}>. At the end of words, it becomes /ɑ/ <\armenian{ա}>; it is deleted when declined (Table \ref{tab:Van:phono:change:vowel:aj:stuff}).



\begin{table}[H]
 \centering
 \caption{Change from Classical Armenian /ɑi̯/ <\armenian{այ}> to /e, i, ɑj, ɑ/ <\armenian{է, ի, այ, ա}> in the Van dialect}
 \label{tab:Van:phono:change:vowel:aj:stuff}
 \begin{tabular}{|l|ll|ll|ll|}
 \hline & \multicolumn{2}{l|}{Classical Armenian}& \multicolumn{2}{l|}{> Van }& \multicolumn{2}{l|}{cf. SEA }
 \\
`father' & hɑi̯ɾ & \armenian{հայր} & χeɾ & \armenian{խէր} & hɑjɾ & \armenian{հայր} \\ 
`shine' & pʰɑi̯l & \armenian{փայլ} & pʰelkʰ & \armenian{փէլք} & pʰɑjl & \armenian{փայլ} \\ 
`to walk' & kʰɑi̯lel & \armenian{քայլել} & kʰʲeli̯el & \armenian{քյէլել} & kʰɑjlel & \armenian{քայլել} \\ 
`wide' & lɑi̯n & \armenian{լայն} & len & \armenian{լէն} & lɑjn & \armenian{լայն} \\ 
 `vineyard' &ɑi̯ɡi& \armenian{այգի} & ikʲi & \armenian{իկյի} &ɑjɡi& \armenian{այգի} \\
 `man' &ɑi̯ɾ& \armenian{այր} & iɾik & \armenian{իրիկ}&ɑjɾ, eɾik& \armenian{այր, էրիկ} \\
`to burn' & ɑi̯ɾel & \armenian{այրել} & iɾit͡sʰi̯el & \armenian{իրիցել} & ɑjɾel & \armenian{այրել} \\ 
`mirror' & hɑ{je}li & \armenian{հայելի} & χɑjlik & \armenian{խայլիկ} & hɑjeli & \armenian{հայելի} \\ 
`tin' & kəlɑ{je}k & \armenian{կլայեկ} & kəlɑji̯ek & \armenian{կըլայեկ} & kəlɑjek & \armenian{կլայեկ} \\ 
`trivet' & kɑskɑɾɑi̯ & \armenian{կասկարայ} & kɑskɑɾɑ & \armenian{կասկարա} & kɑskɑɾɑ & \armenian{կասկարա} \\ 
\hline
 \end{tabular}
 
\end{table}

\paragraph{Classical Armenian /iu̯/ <\armenian{իւ}>}
 


The Classical sound /iu̯/ <\armenian{իւ}> became /iv/ <\armenian{իվ}> before vowels or at the end of words (Table \ref{tab:Van:phono:change:vowel:iu:stuff}a). For Before consonants, we find the sounds /ʏ, u/ <\armenian{իւ, ու}>, and in some places /i, i̯e, o/ <\armenian{ի, ե, օ}> (Table \ref{tab:Van:phono:change:vowel:iu:stuff}b).



\begin{table}[H]
 \centering
 \caption{Change from Classical Armenian /iu̯/ <\armenian{իւ}> to /iv, ʏ, u, i, i̯e, o/ <\armenian{իվ, իւ, ու, ի, ե, օ}> in the Van dialect}
 \label{tab:Van:phono:change:vowel:iu:stuff}
 \begin{tabular}{|ll|ll|ll|ll|}
 \hline && \multicolumn{2}{l|}{Classical Armenian}& \multicolumn{2}{l|}{> Van }& \multicolumn{2}{l|}{cf. SEA }
 \\
a. & `sick' & hi{wɑ}nd & \armenian{հիւանդ} & χivɑnd & \armenian{խիվանդ} & hivɑnd & \armenian{հիվանդ} \\ 
& `shepherd' &hoviu̯ & \armenian{հովիւ} & χoviv & \armenian{խօվիվ} & hoviv& \armenian{հովիվ} \\
b& `flour' & ɑliu̯ɾ & \armenian{ալիւր} & ælʏɾ & \armenian{ա̈լիւր} & ɑljuɾ & \armenian{ալյուր} \\ 
& `fountain' & ɑɬbiu̯ɾ & \armenian{աղբիւր} & æχpʏɾ & \armenian{ա̈խպիւր} & ɑχpjuɾ & \armenian{աղբյուր} \\ 
& `self' & iu̯ɾ & \armenian{իւր} & uɾ & \armenian{ուր} & juɾ & \armenian{յուր} \\ 
& `bodkin' & heɾiu̯n & \armenian{հերիւն} & χiɾun & \armenian{խիրուն} & heɾjun & \armenian{հերյուն} \\ 
& `carpenter' & hiu̯sən & \armenian{հիւսն} & χus & \armenian{խուս} & hjusən & \armenian{հյուսն} \\ 
& \armenian{՝}branch' & t͡ʃiu̯ɬ & \armenian{ճիւղ}& t͡ʃʏʁ & \armenian{ճօղ} & t͡ʃjuʁ & \armenian{ճյուղ} \\
& \armenian{՝}village' & ɡiu̯ɬ & \armenian{գիւղ}& kʲi̯eʁ & \armenian{կյեղ} & ɡjuʁ & \armenian{գյուղ} \\
& \armenian{՝}brick' & ɑɬiu̯s & \armenian{աղիւս}& oʁis & \armenian{օղիս} & ɑʁjus & \armenian{աղյուս} \\
\hline
 \end{tabular}
 
\end{table}


\paragraph{Classical Armenian /oi̯/ <\armenian{ոյ}>}
 


The Classical sound /oi̯/ <\armenian{ոյ}> changed to /u̯o, œə̯, u/ <\armenian{ո, է}\`\armenian{օ, ու}>, and sometimes to /o, vi, vu̯o/ <\armenian{օ, վի, վո}>. There are only individual examples of the latter group (Table \ref{tab:Van:phono:change:vowel:iu:stuff}).\footnote{\translatorHD{For `alfalfa', Adjarian provides an ancestor form <\armenian{առւոյտ}>, but I couldn't find it elsewhere. }}



\begin{table}[H]
 \centering
 \caption{Change from Classical Armenian /oi̯̯/ <\armenian{ոյ}> to /u̯o, œə̯, u, o, vi, vu̯o/ <\armenian{ո, է}\`\armenian{օ, ու, օ, վի, վո}> in the Van dialect}
 \label{tab:Van:phono:change:vowel:oi:stuff}
 \begin{tabular}{|l|ll|ll|ll|}
 \hline & \multicolumn{2}{l|}{Classical Armenian}& \multicolumn{2}{l|}{> Van }& \multicolumn{2}{l|}{cf. SEA }
 \\
`blue' & kɑpoi̯t & \armenian{կապոյտ} & kɑpu̯ot & \armenian{կապոտ} & kɑpujt & \armenian{կապույտ} \\ 
`pinky' & t͡ʃəkoi̯tʰ & \armenian{ճկոյթ} & t͡ʃku̜otʰ & \armenian{ճկոթ} & t͡ʃəkujtʰ & \armenian{ճկույթ} \\ 
`nest' & boi̯n & \armenian{բոյն} &pœə̯n & \armenian{պէ}\`\armenian{օն} & bujn & \armenian{բույն} \\ 
`alfalfa' & ɑru.oi̯t & \armenian{առուոյտ} &ɑrvœə̯t & \armenian{առվէ}\`\armenian{օտ} & ɑrvujt & \armenian{առվույտ} \\ 
 \armenian{՝}color' & ɡoi̯n & \armenian{գոյնք}& kʲœə̯n & \armenian{կյէ}\`\armenian{օն} & ɡujn & \armenian{գույն} \\
`light' & loi̯s & \armenian{լոյս} & lœə̯s & \armenian{լէ}\`\armenian{օս} & lujs & \armenian{լույս} \\ 
`wick' & pɑtɾoi̯ɡ & \armenian{պատրոյգ} & pɑtɾukʰʲ & \armenian{պատրուքյ}& pɑtɾujkʰ & \armenian{պատրույգ} \\ 
 `curtain' & vɑɾɑɡoi̯ɾ& \armenian{վարագոյր} & væɾækʲur & \armenian{վա̈րա̈կյուռ} & vɑɾɑkʰujɾ & \armenian{վարագույր} \\
 `knot' & hɑnɡoi̯t͡sʰ& \armenian{հանգոյց} & χɑnɡʲort͡sʰ & \armenian{խանգյօռց} & hɑŋɡujt͡sʰ & \armenian{հանգույց} \\
 `who.{\gen}.{\sg}' & oi̯ɾ & \armenian{ոյր} & viɾ & \armenian{վիր} & voɾi &\armenian{որի} \\ 
 `strength' & oi̯ʒ & \armenian{ոյժ} & vu̯oʒ & \armenian{վոժ} & uʒ &\armenian{ուժ} \\ 
\hline
 \end{tabular}
 
\end{table}


\subsubsection{Consonant changes}

Consonant sound changes in the Van dialect are the same as in the Karabakh dialect. Here as well, the voiced sounds of Old Armenian have changed to voiceless unaspirates. The voiceless unaspirates stay the same; the voiceless aspirates stay the same. After nasals, voiced sounds and voiceless unaspirated sounds become voiced. After the Classical sound /ɾ/ <\armenian{ր}>, voiced consonants become voiceless aspirates. 

Besides these, the sound /h/ <\armenian{հ}> of the Van dialect becomes /χ/ <\armenian{խ}>. The Classical sound /ɾ/ <\armenian{ր}> sound becomes /r/ <\armenian{ռ}> next to the consonants Classical sound /t͡s, d͡z, t͡sʰ, t͡ʃ, d͡ʒ, t͡ʃʰ, t, tʰ, v/ <\armenian{ծ, ձ, ց, ճ, ջ, չ, տ, թ, վ}>. The Classical sound /kʰ/ <\armenian{ք}> becomes /hʲ/ <\armenian{հյ}> before other consonants. We saw examples of these changes in the above data. 


\begin{adjarianpage}\label{page:143}\end{adjarianpage}% should be 143

As we know, voiced aspirated consonants do not exist in the Van dialect. 

\section{Morphology}
\subsection{Noun inflection or declension}

\subsubsection{Case markers}
The declension system of the Van dialect is similar to to that of /um/ <\armenian{ում}> branch. The ablative is built with the formative /-it͡sʰ/ <\armenian{ից}>. The accusative is the same as the dative if the object is animate; while it is the same as the nominative if the object is inanimate. There is no locative. 

\subsubsection{Plural markers and plural declension}

The plural has three different forms. Monosyllabic words take the formative /-i̯eɾ/ <\armenian{եր}>. Consonant-final polysyllabic words take the formative /-ni̯eɾ/ <\armenian{ներ}>, while vowel-final or /n/-final <\armenian{ն}> polysyllabic words take the formative /-hʲti̯eɾ/ <\armenian{հյտեր}>; based on phonetic laws, this formative originates from the previous form /kʰti̯eɾ/ <\armenian{քտեր}> (Table \ref{tab:Van:morpho:noun:pl}).\footnote{\translatorHD{I'm not sure why Adjarian's examples use a form <\armenian{յհտեր}> /jhti̯eɾ/ with a glide in the wrong location of what Adjarian's prose says: <\armenian{հյտեր}> /hʲti̯eɾ/. }}

\begin{table}[H]
 \centering
 \caption{Plural suffixes /-i̯eɾ, -ni̯eɾ, hʲti̯eɾ/ <\armenian{եր, ներ, հյտեր}> in the Van dialect}
 \label{tab:Van:morpho:noun:pl}
 \begin{tabular}{|l|ll|ll|ll|}
 \hline & \multicolumn{2}{l|}{Classical Armenian} & \multicolumn{2}{l|}{> Van }& \multicolumn{2}{l|}{cf. SEA }
 \\
 `bread' & hɑt͡sʰ & \armenian{հաց} & χɑt͡sʰ & \armenian{խաց} & hɑt͡sʰ & \armenian{հաց} \\
 `bread-{\pl}' & & & χɑt͡sʰ-i̯eɾ & \armenian{խացեր} & hɑt͡sʰ-eɾ & \armenian{հացեր} \\
 `bow' & ɑɬeɬən & \armenian{աղեղն} & ɑni̯eʁ & \armenian{անեղ} & ɑʁeʁ & \armenian{աղեղ} \\
 `bow-{\pl}' & & & ɑnəʁ-ni̯eɾ & \armenian{անըղներ} & ɑʁeʁ-neɾ & \armenian{աղեղներ} \\
`wine' &ɡini & \armenian{գինի} & kʲini &\armenian{կյինի} &ɡini & \armenian{գինի} \\
`wine-{\pl}' & & & kʲini-jhti̯eɾ &\armenian{կյինիյհտեր} &ɡini-neɾ & \armenian{գինիներ} \\
`pantry' &mɑrɑn & \armenian{մառան} & mɑrɑn & \armenian{մառան} & mɑrɑn & \armenian{մառան} \\
`pantry-{\pl}' & & & mɑrɑn-jhti̯eɾ & \armenian{մառանյհտեր} & mɑrɑn-neɾ & \armenian{մառաններ} \\
 \hline
 \end{tabular}
 
\end{table}

The cases of the plural are formed in the following way (Table \ref{tab:Van:morpho:noun:plDecl}). 

\begin{table}[H]
 \caption{Declension of plural nouns in the Van dialect}
 \label{tab:Van:morpho:noun:plDecl} \centering
\begin{tabular}{|l|lll|}
\hline & `breads' & `pantries' & `wines'\\
\hline 
{\nom} & χɑt͡sʰ-i̯eɾ & ɑnəʁ-ni̯eɾ & kʲini-jhti̯eɾ \\
 & \armenian{խացեր} & \armenian{անըղներ} & \armenian{կյինիյհտեր} \\\hline 
{\gen}, {\dat}, {\abl} & χɑt͡sʰ-i̯eɾ-ɑt͡sʰ & ɑnəʁ-ni̯eɾ-ɑt͡sʰ & kʲini-jhti̯eɾ-ɑt͡sʰ \\
 & \armenian{խացեր-աց} & \armenian{անըղներ-աց} & \armenian{կյինիյհտեր-աց} \\\hline 
{\ins} & χɑt͡sʰ-i̯eɾ-ov & ɑnəʁ-ni̯eɾ-ov & kʲini-jhti̯eɾ-ov \\
 & \armenian{խացեր-օվ} & \armenian{անըղներ-օվ} & \armenian{կյինիյհտեր-օվ} 
\\ \hline 
\end{tabular}
\end{table}

\subsubsection{Absence of the definite suffix /-ə/ and word-initial/final schwas}

The Van dialect is famous for the absence of the definite article /-ə/ <\armenian{ը}>. Many times we see that even the most educated Van speakers cannot get used to using the Armenian article /-ə/ <\armenian{ը} >. For example, the speaker would say (\ref{sent:Van:morpho:noun:def:swaMush}) or in the native dialect (\ref{sent:Van:morpho:noun:def:Mush}). 

\begin{exe}
 \ex \label{sent:Van:morpho:noun:def:swaMush}
 \begin{xlist}
 \ex Van speaker producing an SWA sentence without the definite suffix /-ə/
 \begin{xlist}
 \ex \gll 
 ɑjs mɑɾtʰ u ɑjn ɡin ɡəɾ-v-e-t͡sʰ-ɑ-n iɾɑɾ het \\
 this man and that woman fight-{\pass}-{\thgloss}-{\aor}-{\pst}-3{\pl} each.other with \\
 \trans `This man and this woman fought with each other.' \\
 \armenian{այս մարդ ու այն կին կռուեցան իրար հետ}
 \ex \gll 
 meɾ dun met͡s e \\
 our house big is \\
 \trans `Our house is big.' \\
 \armenian{մեր տուն մեծ է}
 \ex \gll 
 vɑnɑ kʰɑʁɑkʰ kʰeʁet͡sʰiɡ e \\
 Van city beautiful is \\
 \trans `The city of Van is beautiful.' \\
 \armenian{Վանայ քաղաք գեղեցիկ է} 
 \end{xlist}
 \ex cf. SWA sentence with the expected definite suffix /-ə/
 \begin{xlist}
 \ex \gll 
 ɑjs mɑɾtʰ-ə u ɑjn ɡin-ə ɡəɾ-v-e-t͡sʰ-ɑ-n iɾɑɾ het \\
 this man-{\defgloss} and that woman-{\defgloss} fight-{\pass}-{\thgloss}-{\aor}-{\pst}-3{\pl} each.other with \\
 \trans `This man and this woman fought with each other.' \\
 \armenian{այս մարդը ու այնը կին կռուեցան իրար հետ}
 \ex \gll 
 meɾ dun-ə met͡s e \\
 our house-{\defgloss} big is \\
 \trans `Our house is big.' \\
 \armenian{մեր տունը մեծ է}
 \ex \gll 
 vɑnɑ kʰɑʁɑkʰ-ə kʰeʁet͡sʰiɡ e \\
 Van city-{\defgloss} beautiful is \\
 \trans `The city of Van is beautiful.' \\
 \armenian{Վանայ քաղաքը գեղեցիկ է} 
 \end{xlist}
 

 \end{xlist}
\ex \label{sent:Van:morpho:noun:def:Mush}
\begin{xlist}
 \ex Van sentence without the definite suffix /-ə/ \gll 
 t͡ʃʏɾ χɑmi̯eʁ i \\
 water delicious is \\
 \trans `(The) water is delicious.'
 \armenian{ճիւր խամեղ ի}
 \ex cf. SWA sentence with the definite suffix /-ə/ \gll 
 t͡ʃuɾ-ə hɑmeʁ e \\
 water delicious is \\
 \trans `The water is delicious.'
 \armenian{ջուրը համեղ է}
\end{xlist}

\end{exe}




Because of this, the word /t͡ʃʏɾ/ <\armenian{ճիւր}> can mean either `water' or `the water'. This characteristic of Van speakers is explained only by their incapability of pronouncing the word-final /-ə/ <\armenian{ը}>, which was also the case in Classical Armenian which did not have word-final /ə/ <\armenian{ը}>. Otherwise, Van speakers are aware of the use of the article, because when they need to be exact, they can add the article /-n/ <\armenian{ն}>. 

It appears that the Van dialect also cannot pronounce a word-initial sound /ə/ <\armenian{ը}> (Table \ref{tab:Van:morpho:noun:schwa}).\footnote{\translatorHD{For `baptized', Adjarian postulates a reconstructed intermediate form /ənkʰɑvoɾ/ <\armenian{ընքավոր}> between the Classical and Van forms. }}


\begin{table}[H]
 \centering
 \caption{Absence of word-initial schwa in the Van dialect}
 \label{tab:Van:morpho:noun:schwa}
 \begin{tabular}{|l|ll|ll|ll|}
 \hline & \multicolumn{2}{l|}{Classical Armenian} & \multicolumn{2}{l|}{> Van }& \multicolumn{2}{l|}{cf. SEA }
 \\
`walnut' & ənkoi̯z & \armenian{ընկոյզ} & ɡœə̯z & \armenian{գէ}\`\armenian{օզ} & əŋkujz & \armenian{ընկույզ} \\
`bed' & ɑnkoɬin, ənkoɬin & \armenian{անկողին, ընկողին} & ɡʲoʁvenkʰʲ & \armenian{գյօղվէնքյ} & ɑŋkoʁin & \armenian{անկողին} \\
`by' & ənd oɾ & \armenian{ընդ որ} & dœə̯ɾ & \armenian{դէ}\`\armenian{օր} & ənd voɾ & \armenian{ընդ որ} \\
 `baptized' & kənkʰɑ{wo}ɾ & \armenian{կնքաւոր} & kʰɑvoɾ & \armenian{քավոր} & kəŋkʰɑvoɾ & \armenian{կնքավոր} \\ 
 \armenian{՝}friend' & ənkeɾ & \armenian{ընկեր}& inɡʲi̯eɾ, ɡi̯eɾ & \armenian{ինգյեր, գեր} & əŋkeɾ & \armenian{ընկեր} \\
 \hline
 \end{tabular}
 
\end{table}

\subsection{Pronoun inflection or declension}
\subsubsection{Personal pronouns}
\begin{adjarianpage}\label{page:144}\end{adjarianpage}% should be 144



\translatorHD{Table \ref{tab:Van:morpho:pronoun:personal} lists the personal pronouns. }

\begin{table}[H]
\caption{Inflection paradigm for personal pronouns in the Van dialect }\label{tab:Van:morpho:pronoun:personal}
\centering 
\begin{tabular}{| l| lll|lll| }
 \hline & 1SG & 2SG &3SG & 1PL & 2PL& 3SG \\
 & `I' & `you' & `he' & `we'& `you' & `they' \\\hline 
 {\nom} & ji̯es & tʏ & zinkʰʲ & mi̯enkʰʲ & tʏkʰʲ & uʁɑnkʰʲ \\
 & \armenian{յես} & \armenian{տիւ} & \armenian{զինքյ} & \armenian{մենքյ} & \armenian{տիւքյ} & \armenian{ուրանքյ} \\ \hline
{\gen} & im & kʰʲu̯o & uɾ & mi̯eɾ & t͡si̯eɾ & uʁɑnt͡sʰ \\
 & \armenian{իմ} & \armenian{քյո} & \armenian{ուր} & \armenian{մեր} & \armenian{ծեր} & \armenian{ուրանց} \\\hline
{\dat}, {\acc} & d͡zi, d͡zik & kʰʲi̯e & uɾ & mi̯e & t͡si̯e & uʁɑnt͡sʰ \\
 & \armenian{ձի, ձիկ} & \armenian{քյե} & \armenian{ուր} & \armenian{մե} & \armenian{ծե} & \armenian{ուրանց} \\\hline
{\abl} & d͡ziz-n-it͡sʰ & kʰʲi̯ez-n-it͡sʰ & uɾ-m-it͡sʰ & mi̯ez-n-it͡sʰ & t͡si̯ez-n-it͡sʰ & uʁɑnt͡sʰ-it͡sʰ \\
 & \armenian{ձիզնից} & \armenian{քյեզնից} & \armenian{ուրմից} & \armenian{մեզնից} & \armenian{ծեզնից} & \armenian{ուրանցից} \\\hline
{\ins} & d͡ziz-n-ov & kʰʲi̯ez-n-ov & uɾ-m-ov & mi̯ez-n-ov & t͡si̯ez-n-ov & uʁɑnt͡sʰ-ov \\
 & \armenian{ձիզնօվ} & \armenian{քյեզնօվ} & \armenian{ուրմօվ} & \armenian{մեզնօվ} & \armenian{ծեզնօվ} & \armenian{ուրանցմօվ} \\ \hline
\end{tabular}
\end{table}
 

\subsubsection{Interrogative pronouns}
\translatorHD{Table \ref{tab:Van:morpho:pronoun:inter} lists interrogative pronouns. }

\begin{table}[H]
\caption{Inflection paradigm for interrogative pronouns in the Van dialect }\label{tab:Van:morpho:pronoun:inter}
\centering 
\begin{tabular}{|l| lll|}
\hline & `who' & `what/that' (singular) & `who/what/that' (plural) \\ \hline 
{\nom} 
& vu̯ov & vu̯oɾ & vuɾu̯onkʰʲ \\
 & \armenian{վով} & \armenian{վոր} & \armenian{վուրոնքյ} \\\hline 
{\gen} & viɾ & vuɾu & vuɾu̯ont͡sʰ \\
 & \armenian{վիր} & \armenian{վուրու} & \armenian{վուրոնց} \\ \hline 
{\dat}, {\acc} & viɾ & vu̯oɾ & vuɾu̯onkʰʲ \\
 & \armenian{վիր} & \armenian{վոր} & \armenian{վուրոնքյ} \\ \hline 
{\abl} & viɾ-n-it͡sʰ, viɾ-m-it͡sʰ & vuɾ-ut͡sʰ & vuɾu̯ont͡sʰ-it͡sʰ \\
 & \armenian{վիրնից, վիրմից} & \armenian{վուրուց} & \armenian{վուրոնցից} \\
\hline {\ins} & viɾ-n-ov, viɾ-m-ov & vuɾ-n-ov & vuɾu̯ont͡sʰ-m-ov \\
 & \armenian{վիրնօվ, վիրմօվ} & \armenian{վուրնօվ} & \armenian{վուրոնցմօվ} 
 \\\hline 
\end{tabular}
\end{table}

\subsubsection{Demonstrative pronouns}
\translatorHD{Table \ref{tab:Van:morpho:pronoun:dem} lists demonstrative pronouns. Demonstratives can be proximal, medial, or distal. }

\begin{table}[H]
\caption{Inflection paradigm for demonstrative pronouns in the Van dialect }\label{tab:Van:morpho:pronoun:dem}
\centering 
\begin{tabular}{|l| lll|lll|}
 \hline & \multicolumn{3}{c|}{Singular} & \multicolumn{3}{c|}{Plural} \\
 & proximal & medial & distal & proximal & medial & distal \\
 & `this' & `that' & `yonder' & `these' & `those' & `those yonder' \\ \hline
{\nom}, {\acc} & es, esɑ, esik & et, etɑ, etik & en, enɑ, enik & isonkʰ & itonkʰ & inonkʰ \\
 & \armenian{էս, էսա, էսիկ} & \armenian{էտ, էտա, էտիկ} & \armenian{էն, էնա, էնիկ} & \armenian{իսոնք} & \armenian{իտոնք} & \armenian{ինոնք} \\ \hline
{\gen}, {\dat} & isoɾ & itoɾ & inoɾ & isont͡sʰ & itont͡sʰ & inont͡sʰ \\
 & \armenian{իսոր} & \armenian{իտոր} & \armenian{ինոր} & \armenian{իսոնց} & \armenian{իտոնց} & \armenian{ինոնց} \\ \hline
{\abl} & isoɾ-m-it͡sʰ & itoɾ-m-it͡sʰ & inoɾ-m-it͡sʰ & isont͡sʰ-it͡sʰ & itont͡sʰ-it͡sʰ & inont͡sʰ-it͡sʰ \\
 & \armenian{իսորմից} & \armenian{իտորմից} & \armenian{ինորմից} & \armenian{իսոնցից} & \armenian{իտոնցից} & \armenian{ինոնցից} \\ \hline
{\ins} & isoɾ-m-ov & itoɾ-m-ov & inoɾ-m-ov & isont͡sʰ-m-ov & itont͡sʰ-m-ov & inont͡sʰ-m-ov \\
 & \armenian{իսորմօվ} & \armenian{իտորմօվ} & \armenian{ինորմօվ} & \armenian{իսոնցմօվ} & \armenian{իտոնցմօվ} & \armenian{ինոնցմօվ} \\ \hline
\end{tabular}
\end{table}

\subsection{Verbal inflection or conjugation}

\subsubsection{Theme vowel changes}
Verbal conjugation does not present major form changes. The only ones are phonetic changes. In the present tenses, the Classical sound /e/ <\armenian{ե}> remains; it changes to /i/ <\armenian{ի}> only in the 3SG in the first conjugation class. In the past tenses, whenever the Classical sounds /ē, ɑi̯/ <\armenian{է, այ}> become vowels, they are deleted (Table \ref{tab:Van:morpho:verb:theme}).


\begin{table}[H]
 \centering
 \caption{Changes to theme vowels in the Van dialect}
 \label{tab:Van:morpho:verb:theme}
 \begin{tabular}{|l|ll|ll|}
 \hline & \multicolumn{2}{l|}{Van }& \multicolumn{2}{l|}{cf. SWA }
 \\
`I would want' & k-uz-$\emptyset$-i-$\emptyset$ & \armenian{կուզի} & ɡ-uz-ej-i-$\emptyset$ & \armenian{կ՚ուզէի} 
\\
`I would sneeze' & kə χɑz-$\emptyset$-i-$\emptyset$ & \armenian{կը խազի} & ɡə hɑz-ɑj-i-$\emptyset$ & \armenian{կը հազայի} \\
& \multicolumn{2}{l|}{{\ind}-want-{\thgloss}-{\pst}-1{\sg}}& \multicolumn{2}{l|}{{\ind}-want-{\thgloss}-{\pst}-1{\sg}}
\\
 \hline
 \end{tabular}
 
\end{table}

\translatorHD{To clarify, Adjarian is discussing how the theme vowel manifests in different morphological contexts. Before the past tense suffix /-i-/, SWA and SEA keep the theme vowels /e,ɑ/ and they add a glide. In contrast in Van, the theme vowel is deleted before this past suffix /-i-/.}



\begin{adjarianpage}\label{page:145}\end{adjarianpage}% should be 145

\subsubsection{General paradigms for the reflex of the E-Class}

As an example, we show the conjugation of the Classical verb /uz-e-m/ <\armenian{ուզեմ}> `I want'. 


{\paradigmExplanation}

\paragraph{Subjunctive present and past imperfective}

\translatorHD{In SWA, the subjunctive present is formed by adding agreement markers after the theme vowel. For a verb like /uz-e-l/ `to want', the theme vowel is an invariant /e/. In Van, essentially the same strategy is used with slightly different agreement markers. However in Van, the theme vowel can alternate between /i/ in the 3SG and between /e, i̯e/ in the other cells.}

\begin{table}[H]
 \centering
 \caption{Subjunctive present <\armenian{ստորադասական ներկայ}> of the verb `to want' in the Van dialect}
 \label{tab:Van:morpho:verb:paradigm:subjPresent}
 \begin{tabular}{|l|ll|ll|}
\hline & \multicolumn{2}{l|}{Van} & \multicolumn{2}{l|}{cf. SWA} \\
1SG & uz-i̯e-m & \armenian{ուզեմ} & uz-e-m & \armenian{ուզեմ} \\
2SG & uz-i̯e-s & \armenian{ուզես} & uz-e-s & \armenian{ուզես} \\
3SG & uz-i-$\emptyset$ & \armenian{ուզի} & uz-e-$\emptyset$ & \armenian{ուզէ} \\
1PL & uz-i̯e-nkʰʲ & \armenian{ուզենքյ} & uz-e-ŋkʰ & \armenian{ուզենք} \\
2PL & uz-e-kʰʲ & \armenian{ուզէքյ} & uz-e-kʰ & \armenian{ուզէք} \\
3PL & uz-i̯e-n & \armenian{ուզեն} & uz-e-n & \armenian{ուզեն} \\
& \multicolumn{2}{l|}{$\sqrt{}$-{\thgloss}-{\agr}}& \multicolumn{2}{l|}{$\sqrt{}$-{\thgloss}-{\agr}}\\ 

\hline 
\end{tabular}
\end{table}

\translatorHD{In SWA, the subjunctive past imperfective (Table \ref{tab:Van:morpho:verb:paradigm:subjPast}) is formed by adding the past suffix /i/ and agreement suffixes after the theme vowel. In Van, the theme vowel /e/ is deleted before the past suffix /i/. }



\begin{table}[H]
 \centering
 \caption{Subjunctive past <\armenian{ստորադասական անցեալ}> of the verb `to want' in the Van dialect}
 \label{tab:Van:morpho:verb:paradigm:subjPast}
 \begin{tabular}{|l|ll|ll|}
\hline & \multicolumn{2}{l|}{Van} & \multicolumn{2}{l|}{cf. SWA} \\
1SG & uz-$\emptyset$-i-$\emptyset$ & \armenian{ուզի} & uz-ej-i-$\emptyset$ & \armenian{ուզէի} \\
2SG & uz-$\emptyset$-i-ɾ & \armenian{ուզիր} & uz-ej-i-ɾ & \armenian{ուզէիր} \\
3SG & uz-e-$\emptyset$-ɾ & \armenian{ուզէր} & uz-e-$\emptyset$-ɾ & \armenian{ուզէր} \\
1PL & uz-$\emptyset$-i-nkʰʲ & \armenian{ուզինքյ} & uz-ej-i-ŋkʰ & \armenian{ուզէինք} \\
2PL & uz-$\emptyset$-i-kʰʲ & \armenian{ուզիքյ} & uz-ej-i-kʰ & \armenian{ուզէիք} \\
3PL & uz-$\emptyset$-i-n & \armenian{ուզին} & uz-ej-i-n & \armenian{ուզէին} \\
& \multicolumn{2}{l|}{$\sqrt{}$-{\thgloss}-{\pst}-{\agr}}& \multicolumn{2}{l|}{$\sqrt{}$-{\thgloss}-{\pst}-{\agr}}\\ 

\hline 
\end{tabular}
\end{table}


\paragraph{Tenses built from the subjunctive: Indicative and future }

 \translatorHD{In Van, many other tenses seem to be built off of the subjunctive (Table \ref{tab:Van:morpho:verb:paradigm:complexSubjunctive}). The indicative present and past imperfective are built by adding the prefix /k-/ before the subjunctive present and subjunctive past. The future and future perfect are formed also by adding the proclitic /piti/ before the appropriate subjunctive form. SWA behaves essentially the same and I don't provide its paradigm. }

\begin{table}[H]
 \centering
 \caption{Forms that are built from the subjunctive forms for the verb `to want' in the Van dialect}
 \label{tab:Van:morpho:verb:paradigm:complexSubjunctive}
 \begin{tabular}{|l|ll|ll|}
\hline & 
\multicolumn{2}{l|}{Indicative present <\armenian{ներկայ}> } & \multicolumn{2}{l|}{Indicative past imperfective <\armenian{անկատար}>} \\
1SG & k-uz-i̯e-m & \armenian{կուզեմ} & k-uz-$\emptyset$-i-$\emptyset$ & \armenian{կուզի} \\
2SG & k-uz-i̯e-s & \armenian{կուզես} & k-uz-$\emptyset$-i-ɾ & \armenian{կուզիր} \\
3SG & k-uz-i-$\emptyset$ & \armenian{կուզի} & k-uz-e-$\emptyset$-ɾ & \armenian{կուզէր} \\
1PL & k-uz-i̯e-nkʰʲ & \armenian{կուզենքյ} & k-uz-$\emptyset$-i-nkʰʲ & \armenian{կուզինքյ} \\
2PL & k-uz-e-kʰʲ & \armenian{կուզէքյ} & k-uz-$\emptyset$-i-kʰʲ & \armenian{կուզիքյ} \\
3PL & k-uz-i̯e-n & \armenian{կուզեն} & k-uz-$\emptyset$-i-n & \armenian{կուզին} 
\\
& \multicolumn{2}{l|}{{\ind}-$\sqrt{}$-{\thgloss}-{\agr}}& \multicolumn{2}{l|}{{\ind}-$\sqrt{}$-{\thgloss}-{\pst}-{\agr}}
\\ \hline 
& \multicolumn{2}{l|}{Future <\armenian{ապառնի}>} & \multicolumn{2}{l|}{Future perfect <\armenian{անցեալ ապառնի}> } \\
1SG & piti uz-i̯e-m & \armenian{պիտի ուզեմ} & piti uz-$\emptyset$-i-$\emptyset$ & \armenian{պիտի ուզի} \\
2SG & piti uz-i̯e-s & \armenian{պիտի ուզես} & piti uz-$\emptyset$-i-ɾ & \armenian{պիտի ուզիր} \\
3SG & piti uz-i-$\emptyset$ & \armenian{պիտի ուզի} & piti uz-e-$\emptyset$-ɾ & \armenian{պիտի ուզէր} \\
1PL & piti uz-i̯e-nkʰʲ & \armenian{պիտի ուզենքյ} & piti uz-$\emptyset$-i-nkʰʲ & \armenian{պիտի ուզինքյ} \\
2PL & piti uz-e-kʰʲ & \armenian{պիտի ուզէքյ} & piti uz-$\emptyset$-i-kʰʲ & \armenian{պիտի ուզիքյ} \\
3PL & piti uz-i̯e-n & \armenian{պիտի ուզեն} & piti uz-$\emptyset$-i-n & \armenian{պիտի ուզին} 
\\
& \multicolumn{2}{l|}{{\fut} $\sqrt{}$-{\thgloss}-{\agr}}& \multicolumn{2}{l|}{{\fut} $\sqrt{}$-{\thgloss}-{\pst}-{\agr}}
\\\hline \end{tabular}
\end{table}

\paragraph{Present perfect and past perfect}

\translatorHD{In SWA, the present perfect (Table \ref{tab:Van:morpho:verb:paradigm:presentPerfect}) and past perfect (Table \ref{tab:Van:morpho:verb:paradigm:pastPerfect}) in are formed by combining a special non-finite form with the present/past auxiliary. For SWA, this non-finite verb can be either the resultative participle (verb with suffix /-ɑd͡z/) or the evidential participle (verb with suffix /-eɾ/). Van uses a similar system. The non-finite form is labeled as just a `past participle' by Adjarian (which I suspect is a perfective converb), and this form uses /-i̯eɾ/ <\armenian{եր}> for the present perfect 3SG, and /-iɾ/ elsewhere. }

\begin{table}[H]
 \centering
 \caption{Present perfect <\armenian{յարակատար}> of the verb `to want' in the Van dialect}
 \label{tab:Van:morpho:verb:paradigm:presentPerfect}
 \begin{tabular}{|l|ll|ll|}
\hline & \multicolumn{2}{l|}{Van} & \multicolumn{2}{l|}{cf. SWA} \\
1SG & uz-iɾ i̯e-m & \armenian{ուզիր եմ} & uz-eɾ e-m & \armenian{ուզեր եմ} \\
2SG & uz-iɾ i̯e-s & \armenian{ուզիր ես} & uz-eɾ e-s & \armenian{ուզեր ես} \\
3SG & uz-i̯eɾ i & \armenian{ուզեր ի} & uz-eɾ e & \armenian{ուզեր է} \\
1PL & uz-iɾ i̯e-nkʰʲ & \armenian{ուզիր ինքյ} & uz-eɾ e-ŋkʰ & \armenian{ուզեր ենք} \\
2PL & uz-iɾ i̯e-kʰʲ & \armenian{ուզիր էքյ} & uz-eɾ e-kʰ & \armenian{ուզեր էք} \\
3PL & uz-iɾ i̯e-n & \armenian{ուզիր են} & uz-eɾ e-n & \armenian{ուզեր են} \\
& \multicolumn{2}{l|}{$\sqrt{}$-{\perfcvb} {\aux}-{\agr}}& \multicolumn{2}{l|}{$\sqrt{}$-{\eptcp} {\aux}-{\agr}}\\ 

\hline 
\end{tabular}
\end{table}


\begin{table}[H]
 \centering
 \caption{Past perfect <\armenian{գերակատար}> of the verb `to want' in the Van dialect}
 \label{tab:Van:morpho:verb:paradigm:pastPerfect}
 \begin{tabular}{|l|ll|ll| }
\hline & \multicolumn{2}{l|}{Van} & \multicolumn{2}{l|}{cf. SWA} \\
1SG & uz-iɾ $\emptyset$-i-$\emptyset$ & \armenian{ուզիր ի} & uz-eɾ ej-i-$\emptyset$ & \armenian{ուզեր էի} \\
2SG & uz-iɾ $\emptyset$-i-ɾ & \armenian{ուզիր իր} & uz-eɾ ej-i-ɾ & \armenian{ուզեր էիր} \\
3SG & uz-iɾ e-$\emptyset$-ɾ & \armenian{ուզիր էր} & uz-eɾ e-$\emptyset$-ɾ & \armenian{ուզեր էր} \\
1PL & uz-iɾ $\emptyset$-i-nkʰʲ & \armenian{ուզիր ինքյ} & uz-eɾ ej-i-ŋkʰ & \armenian{ուզեր էինք} \\
2PL & uz-iɾ $\emptyset$-i-kʰʲ & \armenian{ուզիր իքյ} & uz-eɾ ej-i-kʰ & \armenian{ուզեր էիք} \\
3PL & uz-iɾ i̯e-$\emptyset$-n & \armenian{ուզիր են} & uz-eɾ ej-i-n & \armenian{ուզեր էին} \\
& \multicolumn{2}{l|}{$\sqrt{}$-{\perfcvb} {\aux}-{\pst}-{\agr}}& \multicolumn{2}{l|}{$\sqrt{}$-{\eptcp} {\aux}-{\pst}-{\agr}}\\ 

\hline 
\end{tabular}
\end{table}

\translatorHD{For the 3PL past auxiliary, Adjarian lists /i̯en/ <\armenian{են}> but I would have expected /in/ <\armenian{ին}> based on the rest of the paradigms. This may have been an error. Otherwise, if this is not an error, then the present and past 3PL auxiliaries are homophonous /i̯en/ <\armenian{են}>. }

\paragraph{Past perfective or aorist}

\translatorHD{The past perfective (Table \ref{tab:Van:morpho:verb:paradigm:pastperfectiveAorist}) is also called the aorist. In SWA for /uz-e-l/ `to want', the past perfective is formed by taking the root and theme vowel, adding the aorist or perfective suffix /-t͡sʰ-/, and then adding the past suffix /-i/ and the appropriate agreement suffixes. The 3SG uses covert tense and agreement suffixes. The Van dialect behaves the same, though the theme vowel is /i/ in all but the 3SG. }


\begin{table}[H]
 \centering
 \caption{Past perfective or aorist <\armenian{կատարեալ}> of the verb `to want' in the Van dialect}
 \label{tab:Van:morpho:verb:paradigm:pastperfectiveAorist}
 \begin{tabular}{|l|ll|ll|}
\hline & \multicolumn{2}{l|}{Van} & \multicolumn{2}{l|}{cf. SWA} \\
1SG & uz-i-t͡sʰ-i-$\emptyset$ & \armenian{ուզիցի} & uz-e-t͡sʰ-i-$\emptyset$ & \armenian{ուզեցի} \\
2SG & uz-i-t͡sʰ-i-ɾ & \armenian{ուզիցիր} & uz-e-t͡sʰ-i-ɾ & \armenian{ուզեցիր} \\
3SG & uz-i̯e-t͡sʰ-$\emptyset$-$\emptyset$ & \armenian{ուզեց} & uz-e-t͡sʰ-$\emptyset$-$\emptyset$ & \armenian{ուզեց} \\
1PL & uz-i-t͡sʰ-i-ŋkʰʲ & \armenian{ուզիցինքյ} & uz-e-t͡sʰ-i-ŋkʰ & \armenian{ուզեցինք} \\
2PL & uz-i-t͡sʰ-i-kʰʲ & \armenian{ուզիցիքյ} & uz-e-t͡sʰ-i-kʰ & \armenian{ուզեցիք} \\
3PL & uz-i-t͡sʰ-i-n & \armenian{ուզիցին} & uz-e-t͡sʰ-i-n & \armenian{ուզեցին} \\
& \multicolumn{2}{l|}{$\sqrt{}$-{\thgloss}-{\aor}-{\pst}-{\agr}}& \multicolumn{2}{l|}{$\sqrt{}$-{\thgloss}-{\aor}-{\pst}-{\agr}}\\ 

\hline 
\end{tabular}
\end{table}
\paragraph{Imperative and prohibitive}

\translatorHD{For the imperative 2SG, SWA adds a zero morph /-$\emptyset$/ after the theme vowel /e/ for a verb like `to want' (Table \ref{tab:Van:morpho:verb:paradigm:Imp}). For the 2PL, SWA adds the sequence /-e-t͡sʰ-ekʰ/ after the root such that /-e-t͡sʰ/ forms the aorist stem, while /-ekʰ/ is the agreement marker. Van instead adds a vowel /i/ for the 2SG; it's unclear if this /i/ is the theme vowel or an added suffix. For the 2PL, a suffix /ekʰʲ/ is added. }


\begin{table}[H]
 \centering
 \caption{Imperative forms <\armenian{հրամայական}> for the verb `to want' in the Van dialect}
 \label{tab:Van:morpho:verb:paradigm:Imp}
 \begin{tabular}{|l|lll|ll l|}
\hline & \multicolumn{3}{l|}{Van} & \multicolumn{3}{l|}{cf. SWA} \\
2SG & uz-i & \armenian{ուզի} & $\sqrt{}$-?& uz-e-$\emptyset$ & \armenian{ուզէ} & $\sqrt{}$-{\thgloss}-{\imp}.2{\sg}
\\
2PL& uz-ekʰʲ& \armenian{ուզէքյ} & $\sqrt{}$-{\imp}.2{\pl}& uz-e-t͡sʰ-ekʰ& \armenian{ուզեցէք} & $\sqrt{}$-{\thgloss}-{\aor}-{\imp}.2{\pl}
\\\hline \end{tabular}
\end{table}

\translatorHD{For the prohibitive or negative imperative (Table \ref{tab:Van:morpho:verb:paradigm:Proh}), SWA adds the prohibitive formative /mi/ before the verb. The verb takes a suffix /-ɾ/ in the 2SG, and /-kʰ/ in the 2PL. In Van, the verb is a non-finite form with a suffix /-i̯e/. For the 2SG, the prefix /m-/ is added. For the 2PL, the agreement marker /-ekʰʲ/ is added between the prohibitive marker and the verb, thus agreement morphology is mobile. } 


\begin{table}[H]
 \centering
 \caption{Negative imperative or prohibitive forms for the verb `to want' in the Van dialect}
 \label{tab:Van:morpho:verb:paradigm:Proh}
 \begin{tabular}{|l|lll|lll|}
\hline & \multicolumn{3}{l|}{Van} & \multicolumn{3}{l|}{cf. SWA} \\
2SG & m-uz-i̯e & \armenian{մուզե} & {\proh}-$\sqrt{}$-? & mi uz-e-ɾ & \armenian{մի ուզեր} & {\proh} $\sqrt{}$-{\thgloss}-2{\sg} \\
2PL & m-ekʰʲ uz-i̯e & \armenian{մէքյ ուզե} & {\proh}-{\imp}.2{\pl} $\sqrt{}$-? & mi siɾ-ekʰ& \armenian{մի սիրէք} & {\proh} $\sqrt{}$-{\thgloss}-2{\pl} \\
\hline \end{tabular}
\end{table}

\paragraph{Non-finite forms}

\translatorHD{Finally, Adjarian lists the following non-finite forms of this verb (participles or converbs) in Table \ref{tab:Van:morpho:verb:paradigm:participle}. I give SWA forms for just some of them because it's unclear to me what these Van participles mean. Note that Adjarian uses the term `past participle' to mean multiple different types of non-finite forms: resultative participle with /-ɑd͡z/ in SWA, evidential participle /-eɾ/ in SWA. I suspect the Van /-iɾ/ is a perfective converb. } 

\begin{table}[H]
 \centering
 \caption{Participles or converbs <\armenian{դերբայներ}> for the verb `to want' in the Van dialect}
 \label{tab:Van:morpho:verb:paradigm:participle}
 \begin{tabular}{|ll|lll|lll|}
\hline & & \multicolumn{3}{l|}{Van} & \multicolumn{3}{l|}{cf. SWA} \\
 Infinitive & \armenian{անորոշ} & uz-i̯e-l & \armenian{ուզել} & $\sqrt{}$-{\thgloss}-{\infgloss} & uz-e-l & \armenian{ուզել} & $\sqrt{}$-{\thgloss}-{\infgloss} \\
 Past & \armenian{անցեալ} & uz-ɑt͡s & \armenian{ուզած} & $\sqrt{}$-{\rptcp} & uz-ɑd͡z & \armenian{ուզած} & $\sqrt{}$-{\rptcp} \\
& & uz-iɾ & \armenian{ուզիր}& $\sqrt{}$-{\perfcvb} & uz-eɾ & \armenian{ուզեր}& $\sqrt{}$-{\eptcp} \\
& & uz-i̯eɾ & \armenian{ուզեր}& $\sqrt{}$-{\perfcvb} & & & \\
& & uz-i̯e & \armenian{ուզե}& $\sqrt{}$-{\perfcvb} & & & \\
 Future & \armenian{ապառնի} & uz-i̯e-l-ʏ & \armenian{ուզելիւ} & $\sqrt{}$-{\thgloss}-{\infgloss}-{\futcvb} & uz-e-l-u & \armenian{ուզելու} & $\sqrt{}$-{\thgloss}-{\infgloss}-{\futcvb} \\
\hline \end{tabular}
\end{table}
\section{Subdialects}
The Van dialect has three subdialects. These are Diyadin, Moks, and Vozim. 

\subsection{Diyadin}
The subdialect of Diyadin is familiar to me from the village of Basargechar in the province of New Bayazet, where there is a migrant community from Diyadin, and its... 



\begin{adjarianpage}\label{page:146}\end{adjarianpage}% should be 146

... language remains unchanged until now. 

\subsubsection{Similarities to the Van dialect}
This subdialect is the same as the Van dialect in the following points. 



\paragraph{Palatalization of velar stops}



The Classical sounds /ɡ k kʰ/ <\armenian{գ կ ք}> changed to /ɡʲ kʲ kʰʲ/ <\armenian{գյ կյ քյ}> (Table \ref{tab:Van:subdialect:diyadin:same:velarPal}). 

 \begin{table}[H]
 \centering
 \caption{Palatalization of velar stops in the Diyadin subdialect of the Van dialect}
 \label{tab:Van:subdialect:diyadin:same:velarPal}
 \begin{tabular}{|l|ll|ll|ll|}
 \hline & \multicolumn{2}{l|}{Classical Armenian}& \multicolumn{2}{l|}{> Diyadin (Van) }& \multicolumn{2}{l|}{cf. SEA }
 \\
 `your.{\sg}' & kʰo& \armenian{քո} & kʰʲo & \armenian{քյօ} & kʰo & \armenian{քո} \\
 `he went' & ɡənɑt͡sʰ & \armenian{գնաց} & kʰʲənɑt͡sʰ & \armenian{քյընաց} & ɡənɑt͡sʰ & \armenian{գնաց} \\
 `we ({\nom})' & mekʰ & \armenian{մեք} & mi̯enkʰʲ & \armenian{մենքյ} & meŋkʰ & \armenian{մենք} \\
 `they fell' & ɑnkɑn & \armenian{անկան} & ənkʲɑn & \armenian{ընկյան} & əŋkɑn & \armenian{ընկան} \\
 `we take' & arnumkʰ & \armenian{առնումք} & ɑrni̯enkʰʲ & \armenian{առնենքյ} & ɑrneŋkʰ & \armenian{առնենք} \\
 `you.{\sg}.{\dat}' & kʰez& \armenian{քեզ} & kʰʲi̯ezi & \armenian{քյեզի} & kʰez & \armenian{քեզ} \\
 `cattle-shed' & ɡom & \armenian{գոմ} & ɡʲu̯om & \armenian{գյոմ} & ɡom & \armenian{գոմ} \\
 `sweet' & kʰɑɬt͡sʰəɾ& \armenian{քաղցր} & kʰʲɑχt͡sʰɾ & \armenian{քյախցր} & kʰɑχt͡sʰəɾ & \armenian{քաղցր} \\
 `we discuss' & zəɾut͡sʰemkʰ & \armenian{զրուցեմք} & zɾut͡sʰinkʰʲ & \armenian{զրուցինքյ} & zəɾut͡sʰeŋkʰ & \armenian{զրուցենք} \\
 `back of body' & kʰɑmɑk& \armenian{քամակ} & kʰʲɑmɑk & \armenian{քյամակ} & kʰɑmɑk & \armenian{քամակ} \\
 `female' & ēɡ & \armenian{էգ} & ekʰʲ & \armenian{էքյ} & eɡ & \armenian{էգ} \\
 `I come (Van); & & & kuɡʲɑs & \armenian{կուգյաս} & kəɡɑs & \armenian{կգաս} \\
 I will come (SEA)' & & & & & & \\
 `complaint' & ɡɑnɡɑt & \armenian{գանգատ} & ɡʲɑnɡʲɑt & \armenian{գյանգյատ} & ɡɑŋɡɑt & \armenian{գանգատ} \\
 `we took' & & & ɑrɑnkʰʲ & \armenian{առանքյ} & ɑrɑŋkʰ & \armenian{առանք} \\
 `we go' & eɾtʰɑmkʰ & \armenian{երթամք} & etʰɑnkʰʲ & \armenian{էթանքյ} & jeɾtʰɑŋkʰ & \armenian{երթանք} \\
 `we put ({\pst})' & & & dɾinkʰʲ & \armenian{դրինքյ} & dəɾet͡sʰiŋkʰ & \armenian{դրեցինք} \\
 `we gave ({\pst}, {\impf})' & & & tinkʰʲ & \armenian{տինքյ} &tɑjiŋkʰ & \armenian{տայինք} \\
 `wind' & kʰɑmi & \armenian{քամի} & kʰʲɑmi & \armenian{քյամի} &kʰɑmi & \armenian{քամի} \\
\hline
 \end{tabular}
 
\end{table}
 
\paragraph{Change from Classical Armenian /h/ <\armenian{հ}> to /χ/ <\armenian{խ}>} 

 The Classical sound /h/ <\armenian{հ}> has changed to /χ/ <\armenian{խ}> (Table \ref{tab:Van:subdialect:diyadin:same:hkh}). 

 \begin{table}[H]
 \centering
 \caption{Change from Classical /h/ <\armenian{հ}> to /χ/ <\armenian{խ}> in the Diyadin subdialect of the Van dialect}
 \label{tab:Van:subdialect:diyadin:same:hkh}
 \begin{tabular}{|l|ll|ll|ll|}
 \hline & \multicolumn{2}{l|}{Classical Armenian}& \multicolumn{2}{l|}{> Diyadin (Van) }& \multicolumn{2}{l|}{cf. SEA }
 \\
 `to preserve' & pɑhel & \armenian{պահել} & & & pɑhel & \armenian{պահել} \\
 `they preserve (SWA); & & & kə pɑχi̯en & \armenian{կը պախեն}& kəpɑhen & \armenian{կպահեն} \\
 they will preserve (SEA)' & & & & & \\
 `road' & t͡ʃɑnɑpɑɾh & \armenian{ճանապարհ} & t͡ʃɑmpʰɑχ & \armenian{ճամբախ} & t͡ʃɑnɑpɑɾ, t͡ʃɑmpʰɑ & \armenian{ճանապարհ, ճամփա} \\
 `he reached' & hɑsɑu̯ & 	\armenian{հասաւ} & χɑsɑv & \armenian{խասավ} & hɑsɑv & \armenian{հասավ} \\ 
 `with' & het & \armenian{հետ}& χi̯et & \armenian{խետ} & het& \armenian{հետ} \\
 `cool' & hov & \armenian{հով}& χu̯ov & \armenian{խով} & hov& \armenian{հով} \\
\hline
 \end{tabular}
 
\end{table}

\paragraph{Diphthongization of Classical Armenian /e, o/ <\armenian{ե, ո}>} 

 The Classical sounds /i̯e, u̯o/ <\armenian{ե, ո}> have a diphthongal pronunciation (Table \ref{tab:Van:subdialect:diyadin:same:diph}). 

 \begin{table}[H]
 \centering
 \caption{Change from Classical /e, o/ <\armenian{ե, ո}> to /i̯e, u̯o/ <\armenian{ե, ո}> in the Diyadin subdialect of the Van dialect}
 \label{tab:Van:subdialect:diyadin:same:diph}
 \begin{tabular}{|l|ll|ll|ll|}
 \hline & \multicolumn{2}{l|}{Classical Armenian}& \multicolumn{2}{l|}{> Diyadin (Van) }& \multicolumn{2}{l|}{cf. SEA }
 \\
 `our' & meɾ & \armenian{մեր}& mi̯eɾ & \armenian{մեր} & meɾ& \armenian{մեր} \\
`cow' & kov & \armenian{կով} & ku̯ov &\armenian{կով} & kov & \armenian{կով} \\ 
\hline
 \end{tabular}
 
\end{table} 


\paragraph{Change in theme vowels} 

 The Classical sound /ɑi̯/ <\armenian{այ}> becomes /e/ <\armenian{Է}>. The past forms of the second conjugation use the formative /e/ <\armenian{է}> (Table \ref{tab:Van:subdialect:diyadin:same:theme}). 

 \begin{table}[H]
 \centering
 \caption{Changes in theme vowels in the Diyadin subdialect of the Van dialect}
 \label{tab:Van:subdialect:diyadin:same:theme}
 \begin{tabular}{|l|ll|ll|ll|}
 \hline & \multicolumn{2}{l|}{Classical Armenian}& \multicolumn{2}{l|}{> Diyadin (Van) }& \multicolumn{2}{l|}{cf. SEA }
 \\
 `he was coming' & ɡɑi̯ɾ & \armenian{գայր}& kuɡeɾ & \armenian{կուգէր} & ɡukʰɑɾ& \armenian{կու գար} \\
 `he was going' & eɾtʰi̯ɾ & \armenian{երթայր}& ketʰeɾ & \armenian{կէթէր} & ɡeɾtʰɑɾ& \armenian{կ՚երթար} \\
 `he was going to come' & & & pti ɡʲeɾ & \armenian{պտի գյէր} & bidi kʰɑɾ& \armenian{պիտի գար} \\
\hline
 \end{tabular}
 
\end{table} 

\paragraph{Theme vowel deletion before the past suffix} 

 In the past, the Classical vowel /ē/ <\armenian{է}> is deleted next to /i/ <\armenian{ի}> (Table \ref{tab:Van:subdialect:diyadin:same:pst}). 
 \translatorHD{The examples in contrast show that the theme vowel /ɑ/ is deleted before the past /i/. The Classical forms aren't easy to contrast against the modern forms; instead I contrast against SWA, as did Adjarian.} 

 \begin{table}[H]
 \centering
 \caption{Deletion of theme vowels before the past suffix in the Diyadin subdialect of the Van dialect}
 \label{tab:Van:subdialect:diyadin:same:pst}
 \begin{tabular}{|l|ll|ll|l|}
 \hline & \multicolumn{2}{l|}{Diyadin (Van) }& \multicolumn{2}{l|}{cf. SWA } & 
 \\
 `we gave ({\pst}, {\impf}, {\sbjv})' & t-$\emptyset$-i-nkʰʲ & \armenian{տինքյ} & d-ɑj-i-ŋkʰ& \armenian{տայինք} & give-{\thgloss}-{\pst}-1{\pl}\\
 `I would come ({\pst}, {\impf}, {\ind})' & ku-ɡ-$\emptyset$-i-$\emptyset$ & \armenian{կուգի} & ɡu-kʰ-ɑj-i-$\emptyset$& \armenian{կու գայի} & {\ind}-come-{\thgloss}-{\pst}-1{\sg} \\
 `they were' & $\emptyset$-i-n & \armenian{ին} & ej-i-$\emptyset$& \armenian{էին} & {\aux}-{\pst}-3{\pl} \\
 `they would go ({\pst}, {\impf}, {\ind})' & k-etʰ-$\emptyset$-i-n & \armenian{կէթին} & ɡ-eɾtʰ-ɑj-i-n& \armenian{կ՚երթային} & {\ind}-go-{\thgloss}-{\pst}-3{\pl} \\
 
\hline
 \end{tabular}
 
\end{table} 


\paragraph{Ablative suffix /-e, -it͡sʰ/ <\armenian{է, ից}>} 

The ablative uses the form /-it͡sʰ/ <\armenian{ից}>, but the form /-e/ <\armenian{է}> is also used (Table \ref{tab:Van:subdialect:diyadin:same:abl}, sentence \ref{sent:Van:subdialect:diyadin:same:abl}). 


 \begin{table}[H]
 \centering
 \caption{Ablative suffixs in the Diyadin subdialect of the Van dialect}
 \label{tab:Van:subdialect:diyadin:same:abl}
 \begin{tabular}{|l|ll|ll|ll|}
 \hline & \multicolumn{2}{l|}{Diyadin (Van) }& \multicolumn{2}{l|}{cf. SEA } & \multicolumn{2}{l|}{cf. SWA } 
 \\
\armenian{՝}on-{\abl}(-{\defgloss})' &vɾi̯ev-e-n& \armenian{վրեվէն} &vəɾɑj-it͡sʰ & \armenian{վրայից} & vəɾɑj-e-n& \armenian{վրայէն} \\
\armenian{՝}three-{\abl}(-{\defgloss})' &iɾi̯ekʰ-it͡sʰ& \armenian{իրեքից} &jeɾekʰ-it͡sʰ & \armenian{երեքից} & jeɾekʰ-e-n& \armenian{երեքէն} \\
\armenian{՝}thing-{\abl}(-{\defgloss})' &bæn-it͡sʰ& \armenian{բա̈նից} &bɑn-it͡sʰ & \armenian{բանից} & pʰɑn-e-n& \armenian{բանէն} \\
\armenian{՝}city-{\abl}(-{\defgloss})' &kʰʲɑχkʰ-e-n& \armenian{քյախքէն} &kʰɑʁɑkʰ-it͡sʰ & \armenian{քաղաքից} & kʰɑʁɑkʰ-e-n& \armenian{քաղաքէն} \\
\hline
 \end{tabular}
 
\end{table} 

\begin{exe}
 \ex `It has gone from my mind' (= idiomatic for `I forgot about it') \label{sent:Van:subdialect:diyadin:same:abl}
 \begin{xlist}
 \ex Diyadin (Van) dialect \gll mt-e-s kʰʲɑt͡sʰ-i̯eɾ ɑ \\
 mind-{\abl}-{\possFsg} go-{\perfcvb} {\aux} \\
 \trans \armenian{մտէս քյացեր ա} 
 \ex cf. SWA \gll mətk-e-s kʰɑt͡sʰ-eɾ e \\
 mind-{\abl}-{\possFsg} go-{\eptcp} {\aux} \\
 \trans \armenian{մտքէս գացեր է}
 \end{xlist}
\end{exe}

\subsubsection{Differences from the Van dialect}
This subdialect has the following differences from the Van dialect. 


\paragraph{Retention of Classical Armenian /u/ <\armenian{ու}> } 



The Classical vowel /u/ <\armenian{ու}> is preserved, while it changes to /ʏ/ <\armenian{իւ}> in Van (Table \ref{tab:Van:subdialect:diyadin:diff:u}). 

 \begin{table}[H]
 \centering
 \caption{Lack of the change from Classical Armenian /u/ <\armenian{ու}> to /ʏ/ <\armenian{իւ}> the Diyadin subdialect of the Van dialect}
 \label{tab:Van:subdialect:diyadin:diff:u}
 \begin{tabular}{|l|ll|ll|ll|}
 \hline & \multicolumn{2}{l|}{Classical Armenian}& \multicolumn{2}{l|}{> Diyadin (Van) }& \multicolumn{2}{l|}{cf. SEA }
 \\
`flea' &lu & \armenian{լու} & lu & \armenian{լու} & lu & \armenian{լու} \\ 
 `you.{\sg}.{\nom}' &du & \armenian{դու} & du & \armenian{դու} & du & \armenian{դու} \\
`pomegranate' &nurən & \armenian{նուռն} & nur & \armenian{նուռ} & nur & \armenian{նուռ} \\ 
 `egg' & d͡zu& \armenian{ձու} &d͡zu& \armenian{ձու}& d͡zu& \armenian{ձու} \\
\hline
\end{tabular}
 
\end{table} 

\paragraph{Change from Classical Armenian /oi̯/ <\armenian{ոյ}> } 



 The Classical diphthong /oi̯/ <\armenian{ոյ}> becomes /u/ <\armenian{ու}>, and not /ʏ/ <\armenian{իւ}> as in Van (Table \ref{tab:Van:subdialect:diyadin:diff:u}). 
 

 \begin{table}[H]
 \centering
 \caption{Lack of the change from Classical Armenian /oi̯/ <\armenian{ոյ}> to /ʏ/ <\armenian{իւ}> the Diyadin subdialect of the Van dialect}
 \label{tab:Van:subdialect:diyadin:diff:oj}
 \begin{tabular}{|l|ll|ll|ll|}
 \hline & \multicolumn{2}{l|}{Classical Armenian}& \multicolumn{2}{l|}{> Diyadin (Van) }& \multicolumn{2}{l|}{cf. SEA }
 \\
`weak' & tʰoi̯l & \armenian{թոյլ} & tʰul & \armenian{թուլ} & tʰujl & \armenian{թույլ} \\ 
 \armenian{՝}light' & loi̯s & \armenian{լոյս}& lus & \armenian{լուս} & lujs & \armenian{լույս} \\
`until' & & & t͡ʃʰuɾ & \armenian{չուր} & & \\ 
\hline
\end{tabular}

\end{table} 

\paragraph{Voicing changes } 
The voiced consonants are preserved, while they are changed to voiceless unaspirates in Van.

\paragraph{3SG copula as /ɑ/ <\armenian{ա}> changes } 

The present 3SG of the copular verb is \armenian{ա} /ɑ/, while it is \armenian{ի} /i/ in the Van dialect. With this form, the present perfect (\armenian{յարակատար}) and complex tenses are formed (\ref{sent:Van:subdialect:diyadin:diff:aux}). 

\begin{exe}
 \ex Diyadin (Van) dialect \label{sent:Van:subdialect:diyadin:diff:aux}
 \begin{xlist}
 \ex \gll int͡ʃʰ ɑ \\
 what {\aux}.{\prs}.3{\sg} \\
 \trans `What is it?' \\
 \armenian{ի՞նչ ա}
 \ex \gll en ɑ \\
 that {\aux}.{\prs}.3{\sg} \\
 \trans `It is that.'\\
 \armenian{էն ա}
 \ex \gll edɑ ti̯eʁ-n ɑ \\
 that place-{\defgloss} {\aux}.{\prs}.3{\sg} \\
 \trans `It is that place.'\\
 \armenian{էդա տեղն ա}
 \ex \gll ti̯eʁ ɑ ʃin-i̯e\\
 place {\aux}.{\prs}.3{\sg} build-{\perfcvb} \\
 \trans `He has built a place.' \\
 \armenian{տեղ ա շինե}
 \ex \gll pʰis ɑ ənk-i̯e\\
 ? {\aux}.{\prs}.3{\sg} fall-{\perfcvb} \\
 \trans I don't know what the first word is, but the sense would mean `He has fallen into a X.' The unknown word might be a cognate of SEA /pʰos/ <\armenian{փոս}> `hole' \\
 \armenian{փիս ա ընկե}
 \ex \gll tun ɑ kʰɑnd-i̯e\\
 house {\aux}.{\prs}.3{\sg} demolish-{\perfcvb} \\
 \trans `He has demolished a house.'\\ 
 \armenian{տուն ա քանդե}
 \ex \gll kʰʲəɾtn-i̯eɾ ɑ \\
 sweat?-{\perfcvb} {\aux}.{\prs}.3{\sg} \\
 \trans I suspect this is `He has sweated (= he is sweaty).'\\ 
 \ex \gll kʰʲɑt͡sʰ-i̯eɾ ɑ \\
 go-{\perfcvb} {\aux}.{\prs}.3{\sg} \\
 \trans `He has gone.'\\ 
 \armenian{քյացեր ա}
 \end{xlist}
\end{exe}

\subsection{Moks}

The subdialect of Moks is familiar in the literature with various text samples, which are unfortunately not written with scientific exactness. 


\begin{adjarianpage}\label{page:147}\end{adjarianpage}% should be 147

\subsubsection{Existence of the schwa /ə/}
The primary characteristic of this subdialect is the sound /ə/ <\armenian{ը}>, which is in contrast quite rare in the Van dialect. Classical word-final /i/ <\armenian{ի}> and word-medial /e/ <\armenian{ե}> are indiscriminately changed to /ə/ <\armenian{ը}>. Because of this, the schwa /ə/ <\armenian{ը}> is used in the genitive-dative case suffix and in the present tense of verbs (Table \ref{tab:Van:subdialect:Moks:schwa}, sentence \ref{sent:Van:subdialect:Moks:schwa}).

\begin{table}[H]
 \centering
 \caption{Change from Classical Armenian /i, e/ <\armenian{ի, ե}> to /ə/ <\armenian{ը}> in the Moks subdialect of the Van dialect}
 \label{tab:Van:subdialect:Moks:schwa}
 \begin{tabular}{|l|ll|ll|ll|}
 \hline & \multicolumn{2}{l|}{Classical Armenian}& \multicolumn{2}{l|}{> Moks (Van) }& \multicolumn{2}{l|}{cf. SEA }
 \\
 `rose-{\gen}' & vɑɾd-\'i& \armenian{վարդի} & vɑɾd-\'ə & \armenian{վարդը՛} & vɑɾtʰ-\'i & \armenian{վարդի} \\
`year' &tɑɾ\'i & \armenian{տարի} &tɑɾ\'ə & \armenian{տարը՛} &tɑɾ\'i & \armenian{տարի} \\
`(male?) child' & təɬɑi̯-\'i & \armenian{տղայի} & təʁ\'ə & \armenian{տըղը՛} & təʁɑj-\'i & \armenian{տղայի} \\ 
`I want (Van); I will want (SEA)' & & & kuzəm & \armenian{կուզըմ} & kuzem & \armenian{կուզեմ} \\ 
`I said' & ɑsɑt͡sʰ\'i & \armenian{ասացի} & əsət͡sʰ\'ə & \armenian{ըսըցը՛} & ɑsɑt͡sʰ\'i & \armenian{ասացի} \\ 
`they caught' & bərnet͡sʰ\'in & \armenian{բռնեցին} & brnət͡sʰən & \armenian{բռնըցըն} & bərnet͡sʰ\'in & \armenian{բռնեցին} \\ 
 \hline
 \end{tabular}
 
\end{table} 

\begin{exe}
 \ex Moks (Van) dialect \label{sent:Van:subdialect:Moks:schwa}
 \gll
 χənɡ χɑɾəɾ tɑɾə \\
 five hundred year \\
 \trans `five hundred years' \\
 \armenian{խընգ խարըր տարը}
\end{exe}


Analogous to this, in the future tense, the formative /piti/ <\armenian{պիտի}> is shortened and changed to /tə, t/ <\armenian{տը, տ}>, the latter next to a vowel (\ref{sent:Van:subdialect:Moks:piti}). \translatorHD{Note that in all these examples, the SWA cognates would use /bidi/.} 


\begin{exe}
 \ex Moks (Van) dialect \label{sent:Van:subdialect:Moks:piti}
 \begin{xlist}
 \ex\gll tə brn-ə-m \\
 {\fut} catch-{\thgloss}-1{\sg} \\
 \trans `I will catch.'\\
 \armenian{տը բռնըմ}
 \ex\gll t-ɑs-ə-m \\
 {\fut}-say-{\thgloss}-1{\sg} \\
 \trans `I will say.'\\
 \armenian{տասըմ}
 \ex\gll tə t-e-kʰ d͡zə \\
 {\fut} give-{\thgloss}-2{\pl} me.{\dat}\\
 \trans `You.{\pl} will give to me.'\\
 \armenian{տը տէք ձը}
 \end{xlist}
\end{exe}

It is self-explanatory that all these /ə/ <\armenian{ը}> sounds can be stressed. 

\subsubsection{Lack of diphthong /u̯o/ <\armenian{ո}> }

Similarly, the Classical sound /o/ <\armenian{ո}> (or /ɑu̯/ <\armenian{օ}>) is /u/ <\armenian{ու}> here, similar to the Tbilisi dialect, while it is generally /u̯o/ <\armenian{ո}> (or /o, œə̯/ <\armenian{օ, է}\'\armenian{օ}>) in the Van dialect (Table \ref{tab:Van:subdialect:Moks:uo}).

\begin{table}[H]
 \centering
 \caption{Change from Classical Armenian /o, ɑu̯/ <\armenian{ո, աւ}> to /u/ <\armenian{ու}> in the Moks subdialect of the Van dialect}
 \label{tab:Van:subdialect:Moks:uo}
 \begin{tabular}{|l|ll|ll|ll|}
 \hline & \multicolumn{2}{l|}{Classical Armenian}& \multicolumn{2}{l|}{> Moks (Van) }& \multicolumn{2}{l|}{cf. SEA }
 \\
`apple' & χənd͡zoɾ & \armenian{խնձոր} & χənd͡zuɾ & \armenian{խնձուր} & χənd͡zoɾ & \armenian{խնձոր} \\ 
 `speech' &χɑu̯skʰ & \armenian{խաւսք} & χuskʰ &\armenian{խուսք} & χoskʰ & \armenian{խոսք} \\
 `small' &pʰokʰəɾ & \armenian{փոքր} & pʰukʰɾ &\armenian{փուքր} & pʰokʰəɾ & \armenian{փոքր} \\
 \hline
 \end{tabular}
 
\end{table} 

\subsection{Vozim}


The subdialect of Vozim is spoken in the villages of Vozim or Ozum, Ovs, Havındank, Pas, Past, and Makni, which have around 10,000 Armenian residents. Vozim is the largest town (\armenian{աւան}) among this group of villages. 

\subsubsection{Phonology}

The subdialect of Vozim is distinguished from the Van dialect by four new sounds which are the diphthongs /ĕi̯, ou̯, œu̯/ <\armenian{է}ʲ, \armenian{օւ, էօւ}>\footnote{Adjarian used a subscripted <\armenian{յ}>, to create <\armenian{է}\textsubscript{\armenian{յ}}>. Unfortunately, I don't have a font that allows creating such a subscript in a way that it can be read in simple text files. So I use a superscripted <j> instead. } and the uvular sound (\armenian{կոկորդային}) /q/ <\armenian{ղՙ}>. 

\paragraph{Segment inventory}
\subparagraph{Diphthong /ĕi̯/ <\armenian{է}ʲ>}\label{sec:Van:subdialect:Vozim:seg:diphej}

The first is found also in the Karabakh dialect, but it is pronounced much shorter here and it originates from the Classical sound /i/ <\armenian{ի}> (Table \ref{tab:Van:subdialect:Vozim:i}). 

\begin{table}[H]
 \centering
 \caption{Change from Classical Armenian /i/ <\armenian{ի}> to /ĕi̯/ <\armenian{է}ʲ> in the Vozim subdialect of the Van dialect}
 \label{tab:Van:subdialect:Vozim:i}
 \begin{tabular}{|l|ll|ll|ll|}
 \hline & \multicolumn{2}{l|}{Classical Armenian}& \multicolumn{2}{l|}{> Vozim (Van) }& \multicolumn{2}{l|}{cf. SEA }
 \\
 `span of hand' &tʰiz & \armenian{թիզ} & tʰĕi̯z &\armenian{թէ}ʲ\armenian{զ} & tʰiz & \armenian{թիզ} \\
 `saw' &χizɑɾ & \armenian{խիզար} & χĕi̯zɑɾ & \armenian{խէ}ʲ\armenian{զար} & χizɑɾ & \armenian{խիզար} \\
 `in-law' &χənɑmi & \armenian{խնամի} & χənɑmĕi̯ & \armenian{խնամէ}ʲ & χənɑmi & \armenian{խնամի} \\
`fight' & kəriu̯& \armenian{կռիւ} & krĕi̯v & \armenian{կռէ}ʲ\armenian{վ} & kəriv & \armenian{կռիվ} \\ 
`account' & hɑʃiu̯ & \armenian{հաշիւ} & hɑʃĕi̯v & \armenian{հաշէ}ʲ\armenian{վ} & hɑʃiv & \armenian{հաշիվ} \\ 
`mirror' & hɑ{je}li & \armenian{հայելի} & χejlĕi̯ & \armenian{խէյլէ}ʲ & hɑjeli & \armenian{հայելի} \\ 
\hline
 \end{tabular}
 
\end{table} 

\subparagraph{Diphthong /ou̯/ <\armenian{օւ}> }
The sound /ou̯/ <\armenian{օւ}> is pronounced as /\'ou/ <\armenian{օ՛ու}> and it originates from Classical /u/ <\armenian{ու}> (Table \ref{tab:Van:subdialect:Vozim:o}).

\begin{table}[H]
 \centering
 \caption{Change from Classical Armenian /u/ <\armenian{ու}> to /ou̯/ <\armenian{օւ}> in the Vozim subdialect of the Van dialect}
 \label{tab:Van:subdialect:Vozim:o}
 \begin{tabular}{|l|ll|ll|ll|}
 \hline & \multicolumn{2}{l|}{Classical Armenian}& \multicolumn{2}{l|}{> Vozim (Van) }& \multicolumn{2}{l|}{cf. SEA }
 \\
 `egg' & d͡zu& \armenian{ձու} &d͡zʰo̯u & \armenian{ձՙօւ}& d͡zu& \armenian{ձու} \\
 `dark' & mutʰ& \armenian{մութ} &mou̯tʰ & \armenian{մօւթ}& mutʰ& \armenian{մութ} \\
`mouse' &mukən & \armenian{մուկն} & mou̯k & \armenian{մօւկ}& muk & \armenian{մուկ} \\ 
`raw' &hum & \armenian{հում} & χou̯m & \armenian{խօւմ}& hum & \armenian{հում} \\ 
\hline
 \end{tabular}
 
\end{table} 

\subparagraph{Uvular stop /q/ <\armenian{ղՙ}> }

The sound /q/ <\armenian{ղՙ}> is a Georgian sound\footnote{\translatorHD{Adjarian provides a Georgian letter \includegraphics[scale=0.15]{images/georgianLetter.png}. See footnote \ref{footnote:georgian letter}. }} and it is found in the words in Table \ref{tab:Van:subdialect:Vozim:q}. I have not found this sound in other places.









\begin{table}[H]
 \centering
 \caption{Uvular stop /q/ <\armenian{ղՙ}> in the Vozim subdialect of the Van dialect}
 \label{tab:Van:subdialect:Vozim:q}
 \begin{tabular}{|l|ll|ll|ll|}
 \hline & \multicolumn{2}{l|}{Classical Armenian}& \multicolumn{2}{l|}{> Vozim (Van) }& \multicolumn{2}{l|}{cf. SEA }
 \\
`to bathe' &loɡɑnɑl & \armenian{լոգանալ} & loqɑnɑl & \armenian{լօղՙանալ} & loɡɑnɑl, loʁɑnɑl& \armenian{լոգանալ, լողանալ} \\
 `horse-radish' & boɬk & \armenian{բողկ} & bʰʏq & \armenian{բՙիւղ} & boχk & \armenian{բողկ} \\ 
\hline
 \end{tabular}
 
\end{table} 

 \subparagraph{Diphthong /œu̯/ <\armenian{էօւ}> }

 The diphthong /œu̯/ <\armenian{էօւ}> is pronounced as a fast /œu/ <\armenian{էօու}>. I have found this word only in the word in Table \ref{tab:Van:subdialect:Vozim:œ}.

 
\begin{table}[H]
 \centering
 \caption{Words with the sound /œu̯/ <\armenian{էօւ}> in the Vozim subdialect of the Van dialect}
 \label{tab:Van:subdialect:Vozim:œ}
 \begin{tabular}{|l|ll|ll|ll|}
 \hline & \multicolumn{2}{l|}{Classical Armenian}& \multicolumn{2}{l|}{> Vozim (Van) }& \multicolumn{2}{l|}{cf. SEA }
 \\
`fish' &d͡zukən & \armenian{ձուկն} & d͡zʰœu̯k & \armenian{ձՙէօւկ} & d͡zuk & \armenian{ձուկ} \\ 
\hline
 \end{tabular}
 
\end{table} 

 \subparagraph{Voiced aspirates}
 
Besides these, the subdialect of Vozim has the voiced aspirates /bʰ ɡʰ dʰ d͡zʰ d͡ʒʰ/ <\armenian{բՙ գՙ դՙ ձՙ ջՙ}>, which come from the Armenian voiced consonants. 

\begin{adjarianpage}\label{page:148}\end{adjarianpage}% should be 148

\paragraph{Sound changes}

There are many differences in sound changes. 

\subparagraph{Classical Armenian /o/ <\armenian{ո}>}

The Classical Armenian sound /o/ <\armenian{ո}> changes to /u/ <\armenian{ու}>, similar to the Moks subdialect (Table \ref{tab:Van:subdialect:Vozim:change:o:u}).

 
\begin{table}[H]
 \centering
 \caption{Change from Classical Armenian /o/ <\armenian{ո}> to /u/ <\armenian{ու}> in the Vozim subdialect of the Van dialect}
 \label{tab:Van:subdialect:Vozim:change:o:u}
 \begin{tabular}{|l|ll|ll|ll|}
 \hline & \multicolumn{2}{l|}{Classical Armenian}& \multicolumn{2}{l|}{> Vozim (Van) }& \multicolumn{2}{l|}{cf. SEA }
 \\
`ploughshare' &χopʰ & \armenian{խոփ} & χupʰ & \armenian{խուփ} & χopʰ & \armenian{խոփ} \\ 
`leaven (CA); dough (SEA)' & χəmoɾ & \armenian{խմոր} & χmuɾ & \armenian{խմուր} & χəmoɾ & \armenian{խմոր} \\ 
 \armenian{՝}bosom' & t͡sot͡sʰ & \armenian{ծոց}& t͡sut͡sʰ & \armenian{ծուց} & t͡sot͡sʰ & \armenian{ծոց} \\
 \armenian{՝}hell' & dəʒoχkʰ & \armenian{դժոխք}& dʰʒuχkʰʲ & \armenian{դՙժուխքյ} & dəʒoχkʰ & \armenian{դժոխք} \\
 \armenian{՝}frog' & ɡoɾt & \armenian{գորտ}& ɡʰoɾt & \armenian{գյուրտ} & ɡoɾt & \armenian{գորտ} \\
 \armenian{՝}work' & ɡoɾt͡s & \armenian{գործ}& ɡʲuɾt͡s & \armenian{գՅուրծ} & ɡoɾt͡s & \armenian{գործ} \\
\hline
 \end{tabular}
 
\end{table} 
 


But this sound can also take the forms /ou̯, œ, ʏ, o/ <\armenian{օւ, էօ, իւ, օ}> (Table \ref{tab:Van:subdialect:Vozim:change:o:other}).

 
\begin{table}[H]
 \centering
 \caption{Change from Classical Armenian /o/ <\armenian{ո}> to /ou̯, œ, ʏ, o/ <\armenian{օւ, էօ, իւ, օ}> in the Vozim subdialect of the Van dialect}
 \label{tab:Van:subdialect:Vozim:change:o:other}
 \begin{tabular}{|l|ll|ll|ll|}
 \hline & \multicolumn{2}{l|}{Classical Armenian}& \multicolumn{2}{l|}{> Vozim (Van) }& \multicolumn{2}{l|}{cf. SEA }
 \\
 `mold' & boɾbos & \armenian{բորբոս}& bʰœɾbœs & \armenian{բՙէօրբէօս} & boɾbos & \armenian{բորբոս} \\
 \armenian{՝}barefoot' & bokik & \armenian{բոկիկ}& bʰʏpek & \armenian{բՙիւպէկ} & bopik & \armenian{բոպիկ} \\
 \armenian{՝}all' & boloɾ & \armenian{բոլոր}& bʰœlov & \armenian{բՙէօլօվ} & boloɾ & \armenian{բոլոր} \\
 \armenian{՝}garlic' & səχtoɾ & \armenian{սխտոր}& səʁtou̯ɾ & \armenian{սըղտօւր} & səχtoɾ & \armenian{սխտոր} \\
\hline
 \end{tabular}
 
\end{table} 
 
 \subparagraph{Classical Armenian /iu̯/ <\armenian{իւ}>}

The Classical Armenian sound /iu̯/ <\armenian{իւ}> changes to /o, ou̯, e/ <\armenian{օ, օւ, է}> (Table \ref{tab:Van:subdialect:Vozim:change:iu}).

 
\begin{table}[H]
 \centering
 \caption{Change from Classical Armenian /iu̯/ <\armenian{իւ}> to /o, ou̯, e/ <\armenian{օ, օւ, է}> in the Vozim subdialect of the Van dialect}
 \label{tab:Van:subdialect:Vozim:change:iu}
 \begin{tabular}{|l|ll|ll|ll|}
 \hline & \multicolumn{2}{l|}{Classical Armenian}& \multicolumn{2}{l|}{> Vozim (Van) }& \multicolumn{2}{l|}{cf. SEA }
 \\
`ploughshare' &χopʰ & \armenian{խոփ} & χupʰ & \armenian{խուփ} & χopʰ & \armenian{խոփ} \\ 
 `carpenter' & hiu̯sən & \armenian{հիւսն} & χou̯s & \armenian{խօւս} & hjusən & \armenian{հյուսն} \\ 
 `avalanche' & hiu̯s & \armenian{հիւս} & ou̯sĕi̯ & \armenian{օւսէ}ʲ & hjus & \armenian{հյուս} \\ 
`bodkin' & heɾiu̯n & \armenian{հերիւն} & χĕi̯ɾon & \armenian{խէ}ʲ\armenian{րօն} & heɾjun & \armenian{հերյուն} \\ 
`hundred' & hɑɾiu̯ɾ & \armenian{հարիւր} & χɑɾeɾ & \armenian{խարէր} & hɑɾjuɾ & \armenian{հարյուր} \\
 \armenian{՝}brick' & ɑɬiu̯s & \armenian{աղիւս}& oʁes & \armenian{օղէս} & ɑʁjus & \armenian{աղյուս} \\
 `flour' & ɑliu̯ɾ & \armenian{ալիւր} & jeloɾ & \armenian{յէլօր} & ɑljuɾ & \armenian{ալյուր} \\ 
\hline
 \end{tabular}
 
\end{table}
 
\subparagraph{Classical Armenian /ɑi̯/ <\armenian{այ}>}

The Classical Armenian sound /ɑi̯/ <\armenian{այ}> changes not only to /e/ <\armenian{է}>, but also to /i̯e/ <\armenian{ե}> (Table \ref{tab:Van:subdialect:Vozim:change:e}).

\begin{table}[H]
 \centering
 \caption{Change from Classical Armenian /ɑi̯/ <\armenian{այ}> to /e, i̯e/ <\armenian{է, ե}> in the Vozim subdialect of the Van dialect}
 \label{tab:Van:subdialect:Vozim:change:e}
 \begin{tabular}{|l|ll|ll|ll|}
 \hline & \multicolumn{2}{l|}{Classical Armenian}& \multicolumn{2}{l|}{> Vozim (Van) }& \multicolumn{2}{l|}{cf. SEA }
 \\
 `vineyard' &ɑi̯ɡi& \armenian{այգի} & heɡe & \armenian{հէգէ} &ɑjɡi& \armenian{այգի} \\
`goat' & ɑi̯t͡s & \armenian{այծ} & jet͡s & \armenian{յէծ} & ɑjt͡s & \armenian{այծ} \\ 
 `cave' &ɑi̯ɾ& \armenian{այր} & heɾ & \armenian{հէր} &ɑjɾ & \armenian{այր} \\
`wide' & lɑi̯n & \armenian{լայն} & li̯en & \armenian{լեն} & lɑjn & \armenian{լայն} \\ 
`father' & hɑi̯ɾ & \armenian{հայր} & χi̯eɾ & \armenian{խեր} & hɑjɾ & \armenian{հայր} \\ 
`mother' & mɑi̯ɾ & \armenian{մայր} & mi̯eɾ & \armenian{մեր} & mɑjɾ & \armenian{մայր} \\ 
\hline
 \end{tabular}
 
\end{table}
 
 \subparagraph{Word-initial insertion of /h/ <\armenian{հ}> }
 Words that start with a vowel often get an /h/ <\armenian{հ}> (Table \ref{tab:Van:subdialect:Vozim:change:h}).

\begin{table}[H]
 \centering
 \caption{Insertion of word-initial /h/ <\armenian{հ}> before Classical Armenian vowels in the Vozim subdialect of the Van dialect}
 \label{tab:Van:subdialect:Vozim:change:h}
 \begin{tabular}{|l|ll|ll|ll|}
 \hline & \multicolumn{2}{l|}{Classical Armenian}& \multicolumn{2}{l|}{> Vozim (Van) }& \multicolumn{2}{l|}{cf. SEA }
 \\
 `durable' & ɑmuɾ & \armenian{ամուր} &hɑmuɾ & \armenian{համուր} & ɑmuɾ & \armenian{ամուր} \\ 
 `life (CA); goods (SEA)' & ɑbɾɑnkʰ & \armenian{ապրանք} &hɑbɾɑnkʰ & \armenian{հաբրանք} & ɑbɾɑŋkʰ & \armenian{ապրանք} \\ 
 `more' & ɑu̯eli & \armenian{աւելի} &hɑvil & \armenian{հավիլ} & ɑveli & \armenian{ավելի} \\ 
 `shore' & ɑpʰ & \armenian{ափ} &hɑpʰ & \armenian{հափ} & ɑpʰ & \armenian{ափ} \\ 
`cheap' & ɑɾʒɑn & \armenian{արժան}& heʒɑn & \armenian{հէժան}& ɑɾʒɑn & \armenian{արժան} \\
`oath' &eɾdumən & \armenian{երդումն} & heɾtou̯m & \armenian{հէրտօւմ} & jeɾtʰum & \armenian{երդում} \\ 
 `evening'& eɾekoi̯ & \armenian{երեկոյ} & heɾkon & \armenian{հէրկօն} & jeɾeko & \armenian{երեկո} \\
\hline
 \end{tabular}
 
\end{table} 
 
\subsubsection{Morphology}

\paragraph{Noun inflection or declension}

In the declension of Vozim, it is noticeable that the genitive-dative uses the formatives /-ə/, -ĕi̯> <\armenian{ը, է}ʲ\armenian{ն}> . The instrumental uses /-ov, -œv/ <\armenian{ով, էօվ}>. The plural uses /-dʰiɾ/ <\armenian{դՙիր}> (Table \ref{tab:Van:subdialect:Vozim:morpho:pl}).

\begin{table}[H]
 \centering
 \caption{Plural suffix /-dʰiɾ/ <\armenian{դՙիր}> in the Vozim subdialect of the Van dialect}
 \label{tab:Van:subdialect:Vozim:morpho:pl}
 \begin{tabular}{|l|ll|ll|ll|}
 \hline & \multicolumn{2}{l|}{Classical Armenian}& \multicolumn{2}{l|}{> Vozim (Van) }& \multicolumn{2}{l|}{cf. SEA }
 \\
`(male?) child' & təɬɑi̯ & \armenian{տղայ} & & & təʁɑ- & \armenian{տղա} \\
`(male?) children' & təɬɑi̯-kʰ & \armenian{տղայք} & təʁejkʰ-dʰiɾ & \armenian{տղէյքդՙիր}& təʁɑ-kʰ & \armenian{տղաք} \\
\hline
 \end{tabular}
 
\end{table} 


The following is a small depiction of the case system (Table \ref{tab:Van:subdialect:Vozim:morpho:noun:decl}). 

\begin{table}[H]
 \centering
 \caption{Sample declension paradigm for a noun `bread'}\label{tab:Van:subdialect:Vozim:morpho:noun:decl}
\begin{tabular}{|l| ll| ll |}
\hline &\multicolumn{2}{l|}{Singular} &\multicolumn{2}{l|}{Plural} \\
{\nom} & χɑt͡sʰ & \armenian{խաց} & χɑt͡sʰ-iɾ & \armenian{խացիր} \\
{\gen}-{\dat} & χɑt͡sʰ-ə, χɑt͡sʰ-ejn & \armenian{խացը, խացէյն} & χɑt͡sʰ-iɾ-u & \armenian{խացիրու} \\
{\abl} & χɑt͡sʰ-en & \armenian{խացէն} & χɑt͡sʰ-iɾ-en & \armenian{խացիրէն} \\
{\ins} & χɑt͡sʰ-u̯ov & \armenian{խացով} & χɑt͡sʰ-iɾ-u̯ov & \armenian{խացիրով}
\end{tabular}

\end{table}

\paragraph{Pronoun inflection or declension}


The pronouns are the following (Tables \ref{tab:Van:subdialect:Vozim:morpho:pronoun:not3}, \ref{tab:Van:subdialect:Vozim:morpho:pronoun:3}). 

\begin{table}[H]
\caption{Inflection paradigm for some (non-third person) personal pronouns in the Vozim subdialect of the Van dialect }\label{tab:Van:subdialect:Vozim:morpho:pronoun:not3}
\centering 
\begin{tabular}{| l| llll| }
 \hline & 1SG & 2SG & 1PL & 2PL \\
 & `I' & `you' & `we'& `you' \\\hline 
{\nom} & is & minkʰ & dʰu, dʰœ, dʰʏ & dʰœkʰ \\
 & \armenian{իս} & \armenian{մինք} & \armenian{դՙու, դՙէօ, դՙիւ} & \armenian{դՙէօք} \\\hline 
{\gen} & im & mi & kʰʲʏ & d͡zʰ, d͡zʰə \\
 & \armenian{իմ} & \armenian{մի} & \armenian{քյիւ} & \armenian{ձՙի, ձՙը} \\\hline 
{\dat} & d͡zej, ənd͡zej, d͡zi & mi & kʰʲi & d͡zʰi \\
 & \armenian{ձէյ, ընձէյ, ձի} & \armenian{մի} & \armenian{քյի} & \armenian{ձՙի} \\\hline 
{\acc} & d͡zə, əzd͡zə & mi, zmi & kʰʲi, əzkʰʲi & d͡zʰi, əzd͡zʰi \\
 & \armenian{ձը, ըզձը} & \armenian{մի, զմի} & \armenian{քյի, ըզքյի} & \armenian{ձՙի, ըզձՙի} \\
\hline {\abl} & ənd͡zne & mizne, mine & kʰʲine, kʰʲiz & d͡zʰine, d͡zʰizne \\
 & \armenian{ընձնէ} & \armenian{միզնէ, մինէ} & \armenian{քյինէ, քյիզնէ} & \armenian{ձՙինէ, ձՙիզնէ} \\
\hline {\ins} & – & minu̯ov, miznu̯ov & kʰʲinu̯ov, kʰʲiznu̯ov & d͡zʰinu̯ov, d͡zʰiznu̯ov \\
 & – & \armenian{մինով, միզնով} & \armenian{քյինով, քյիզնով} & \armenian{ձՙինով, ձՙիզնով} \\
 & χɑd͡zej & χami & χɑkʰʲi & χɑd͡zʰi \\
 & \armenian{խաձէյ} & \armenian{խամի} & \armenian{խաքյի} & \armenian{խաձՙի} 
 \\ \hline
\end{tabular}
\end{table}


\begin{table}[H]
\caption{Inflection paradigm for some (third person) personal pronouns in the Vozim subdialect of the Van dialect }\label{tab:Van:subdialect:Vozim:morpho:pronoun:3}
\centering 
\begin{tabular}{| l| ll|ll| }
 \hline & 3SG & & 3PL & \\
 & `he' & & `they'& \\\hline 
 {\nom} & ɑn & \armenian{ան} & ɑnou̯nkʰ, nɑɾonkʰ & \armenian{անօւնք, նարօնք} \\
 {\gen}-{\dat} & ɑnou̯ɾ, nɑnou̯ɾ & \armenian{անօւր, նանօւր} & ɑnou̯nk, nɑnou̯nt͡sʰ & \armenian{անօւնք, նանօւնց} \\
{\acc} & zɑnek & \armenian{զանէկ} & zɑnou̯nk & \armenian{զանօւնք} \\
{\abl} & ɑnomne & \armenian{անօմնէ} & ɑnou̯nt͡sʰmne & \armenian{անօւնցմնէ} \\
{\ins} & ɑnu̯of, ɑnu̯oχejt & \armenian{անոֆ, անոխէյտ} & ɑnou̯nt͡sʰ χejt & \armenian{անօւնց խէյտ} 
\\ \hline
\end{tabular}
\end{table}

\begin{adjarianpage}\label{page:149}\end{adjarianpage}% should be 149


At the edge of instrumentals, the form /χɑ/ <\armenian{խա}> is derived from the Classical word /het/ <\armenian{հետ}>, as can be guessed. Analogous to this is the Classical word /mɑu̯t/ <\armenian{մօտ}> \armenian{՝}near', from which are formed the words in Table \ref{tab:Van:subdialect:Vozim:morpho:pron:ins}. 

\begin{table}[H]
 \centering
 \caption{Sample of instrumental pronouns (`near X') in the Vozim subdialect of the Van dialect}
 \label{tab:Van:subdialect:Vozim:morpho:pron:ins}
 \begin{tabular}{|l ll| }
\hline 
personal 1SG `near me' &mɑd͡zĕi̯ & \armenian{մաձէ}ʲ \\
personal 1PL `near us' &mɑmi & \armenian{մամի} \\
personal 2SG `near you' &mɑkʰʲi & \armenian{մաքյի} \\
personal 2PL `near you' &mɑd͡zʰi & \armenian{մաձՙի} \\

\hline 
 \end{tabular}
\end{table}
 
\subsubsection{Verb inflection or conjunction}

\paragraph{Overview and morphological changes}
\subparagraph{Theme vowel changes}
In conjugation, various changes occur, which are in accordance with phonetic rules. The present of the first conjugation takes the vowel /i/ <\armenian{ի}> vowel; while it takes /ĕi̯/ <\armenian{է}ʲ> in the second conjugation. \translatorHD{He means that the Classical theme vowel /e/ became /i/, while the Classical theme vowel /i/ became /ĕi̯/. The original theme vowels are maintained in SWA. }

\subparagraph{Vowel hiatus between the theme vowel and the past suffix}

In the imperfective, the Classical sound sequence /ēi, ɑji/ <\armenian{էի, այի}> are changed to /e/ <\armenian{է}> (Table \ref{tab:Van:subdialect:Vozim:morpho:verb:themepast}). \translatorHD{To elaborate, when the theme vowel is before the past suffix, the two are replaced by a vowel /e/. It seems that this vowel /e/ marks the past tense. In contrast in SWA, the two vowel morphemes are separated by a glide /j/. }

\begin{table}[H]
 \centering
 \caption{Change from Classical Armenian theme vowels and past suffix in the Vozim subdialect of the Van dialect}
 \label{tab:Van:subdialect:Vozim:morpho:verb:themepast}
 \begin{tabular}{|l|ll|ll|ll|}
 \hline & \multicolumn{2}{l|}{Classical Armenian}& \multicolumn{2}{l|}{> Vozim (Van) }& \multicolumn{2}{l|}{cf. SWA }
 \\\hline
`I bring' & beɾ-e-m & \armenian{բերեմ} & kə bʰiɾ-i-m & \armenian{կը բՙիրիմ} & ɡə pʰeɾ-e-m & \armenian{կը բերեմ} \\ 
`I speak' & χɑu̯s-i-m & \armenian{խօսիմ} & kə χos-ĕi̯-m & \armenian{կը խօսէ}ʲ\armenian{մ} & ɡə χos-i-m & \armenian{կը խօսիմ} \\ 
`I take' & tɑn-i-m & \armenian{տանիմ} & kə tɑn-ĕi̯-m & \armenian{կը տանէ}ʲ\armenian{մ} & ɡə tɑn-i-m & \armenian{կը տանիմ} \\ 
& \multicolumn{2}{l|}{$\sqrt{}$-{\thgloss}-1{\sg}}
& \multicolumn{2}{l|}{{\ind} $\sqrt{}$-{\thgloss}-1{\sg}}
& \multicolumn{2}{l|}{{\ind} $\sqrt{}$-{\thgloss}-1{\sg}}
\\
\hline 
`I was crying' & l-ɑj-i-m & \armenian{լայի} & k-il-e-$\emptyset$ & \armenian{կիլէ} & ɡə l-ɑj-i-$\emptyset$ & \armenian{կու լայի} \\ 
`I was bringing' & beɾ-ē-i-$\emptyset$ & \armenian{բերէի} & kə bʰiɾ-$\emptyset$-e-$\emptyset$ & \armenian{կը բՙիրէ} & ɡə pʰeɾ-ej-i-$\emptyset$ & \armenian{կու բերէի} \\ 
& \multicolumn{2}{l|}{$\sqrt{}$-{\thgloss}-{\pst}-1{\sg}}
& \multicolumn{2}{l|}{{\ind} $\sqrt{}$-{\thgloss}-{\pst}-1{\sg}}
& \multicolumn{2}{l|}{{\ind} $\sqrt{}$-{\thgloss}-{\pst}-1{\sg}}
\\
\hline 
`I was' & jes ē-i-$\emptyset$ & \armenian{ես էի} & kə is $\emptyset$-e-$\emptyset$ & \armenian{իս է} & jes ej-i-$\emptyset$ & \armenian{կու բերէի} \\ 
& \multicolumn{2}{l|}{I {\aux}-{\pst}-1{\sg}}
& \multicolumn{2}{l|}{I {\aux}-{\pst}-1{\sg}}
& \multicolumn{2}{l|}{I {\aux}-{\pst}-1{\sg}}
\\
\hline
 \end{tabular}
 
\end{table} 



\subparagraph{Past suffix}

The perfective takes the vowel /ĕi̯/ <\armenian{է}ʲ> (Table \ref{tab:Van:subdialect:Vozim:morpho:verb:pastperfective}). 

\begin{table}[H]
 \centering
 \caption{Change from Classical Armenian past perfective in the Vozim subdialect of the Van dialect}
 \label{tab:Van:subdialect:Vozim:morpho:verb:pastperfective}
 \begin{tabular}{|l|ll|ll|}
 \hline & \multicolumn{2}{l|}{Vozim (Van) }& \multicolumn{2}{l|}{cf. SWA }
 \\\hline
`I called' & kɑnt͡ʃʰ-ə-t͡sʰ-ĕi̯-$\emptyset$ & \armenian{կանչըցէ}ʲ & ɡɑnt͡ʃʰ-e-t͡sʰ-i-$\emptyset$ & \armenian{կանչեցի} \\ 
`I coughed' & χɑz-ɑ-t͡sʰ-ĕi̯-$\emptyset$ & \armenian{խազացէ}ʲ & hɑz-ɑ-t͡sʰ-i-$\emptyset$ & \armenian{հազացի} \\ 
`I discussed' & zɾut͡sʰ-e-t͡sʰ-ĕi̯-$\emptyset$ & \armenian{զրուցէցէ}ʲ & zəɾut͡sʰ-e-t͡sʰ-i-$\emptyset$ & \armenian{զրուցեցի} \\ 
& \multicolumn{2}{l|}{$\sqrt{}$-{\thgloss}-{\aor}-{\pst}-1{\sg}}
& \multicolumn{2}{l|}{$\sqrt{}$-{\thgloss}-{\aor}-{\pst}-1{\sg}}
\\
\hline 
 \end{tabular}
 
\end{table} 

\subparagraph{Future marker}


The marker of the future is /tə/ <\armenian{տը}> (Table \ref{tab:Van:subdialect:Vozim:morpho:verb:fut}).

\begin{table}[H]
 \centering
 \caption{Future marker in the Vozim subdialect of the Van dialect}
 \label{tab:Van:subdialect:Vozim:morpho:verb:fut}
 \begin{tabular}{|l|ll|ll|l|}
 \hline & \multicolumn{2}{l|}{Vozim (Van) }& \multicolumn{2}{l|}{cf. SWA }
 \\\hline
`I will bring' & tə bʰiɾ-i-m & \armenian{տը բՙիրիմ} & bidi pʰeɾ-e-m & \armenian{պիտի բերեմ} & {l|}{{\fut} $\sqrt{}$-{\thgloss}-1{\sg}}
\\ 
`I was going to bring' & tə bʰiɾ-$\emptyset$-e-m & \armenian{տը բՙիրէ} & bidi pʰeɾ-ej-i-m & \armenian{պիտի բերէի} 
& {l|}{{\fut} $\sqrt{}$-{\thgloss}-{\pst}-1{\sg}}
\\
\hline 
 \end{tabular}
 
\end{table} 


\paragraph{General paradigms for the reflex of the E-Class}

 The following is the conjugation of the Classical verb /uz-e-m/ <\armenian{ուզեմ}> `I want'. 



{\paradigmExplanation}

\subparagraph{Subjunctive present and past imperfective}

\translatorHD{In SWA, the subjunctive present is formed by adding agreement markers after the theme vowel. For a verb like /uz-e-l/ `to want', the theme vowel is an invariant /e/. In the Vozim subdialect of the Van dialect, essentially the same strategy is used with slightly different agreement markers. The theme for this verb in this context is /i/.}

\begin{table}[H]
 \centering
 \caption{Subjunctive present <\armenian{ստորադասական ներկայ}> of the verb `to want' in the Vozim subdialect of the Van dialect}
 \label{tab:Van:subdialect:Vozim:morpho:verb:paradigm:subjPresent}
 \begin{tabular}{|l|ll|ll|}
\hline & \multicolumn{2}{l|}{Vozim (Van)} & \multicolumn{2}{l|}{cf. SWA} \\
1SG & ou̯z-i-m & \armenian{օւզիմ} & uz-e-m & \armenian{ուզեմ} \\
2SG & ou̯z-i-s & \armenian{օւզիս} & uz-e-s & \armenian{ուզես} \\
3SG & ou̯z-i-$\emptyset$ & \armenian{ուզի} & uz-e-$\emptyset$ & \armenian{ուզէ} \\
1PL & ou̯z-i-nkʰ & \armenian{օւզինք} & uz-e-ŋkʰ & \armenian{ուզենք} \\
2PL & ou̯z-i-kʰ & \armenian{օւզիք} & uz-e-kʰ & \armenian{ուզէք} \\
3PL & ou̯z-i-n & \armenian{օւզին} & uz-e-n & \armenian{ուզեն} \\
& \multicolumn{2}{l|}{$\sqrt{}$-{\thgloss}-{\agr}}& \multicolumn{2}{l|}{$\sqrt{}$-{\thgloss}-{\agr}}\\ 

\hline 
\end{tabular}
\end{table}

\translatorHD{In SWA, the subjunctive past imperfective (Table \ref{tab:Van:subdialect:Vozim:morpho:verb:paradigm:subjPast}) is formed by adding the past suffix /i/ and agreement suffixes after the theme vowel. The past suffix is absent in the 3SG. In Vozim, the theme vowel is deleted before the past suffix /e/. Note that the 2SG and 3SG are homophonous with a final [eɾ], but the vowel belongs to different morphemes. }



\begin{table}[H]
 \centering
 \caption{Subjunctive past <\armenian{ստորադասական անցեալ}> of the verb `to want' in the Vozim subdialect of the Van dialect}
 \label{tab:Van:subdialect:Vozim:morpho:verb:paradigm:subjPast}
 \begin{tabular}{|l|ll|ll|}
\hline & \multicolumn{2}{l|}{Vozim (Van)} & \multicolumn{2}{l|}{cf. SWA} \\
1SG & ou̯z-$\emptyset$-e-$\emptyset$ & \armenian{օւզէ} & uz-ej-i-$\emptyset$ & \armenian{ուզէի} \\
2SG & ou̯z-$\emptyset$-e-ɾ & \armenian{օւզէր} & uz-ej-i-ɾ & \armenian{ուզէիր} \\
3SG & ou̯z-e-$\emptyset$-ɾ & \armenian{օւզէր} & uz-e-$\emptyset$-ɾ & \armenian{ուզէր} \\
1PL & ou̯z-$\emptyset$-e-nkʰ & \armenian{օւզէնք} & uz-ej-i-ŋkʰ & \armenian{ուզէինք} \\
2PL & ou̯z-$\emptyset$-e-kʰ & \armenian{օւզէք} & uz-ej-i-kʰ & \armenian{ուզէիք} \\
3PL & ou̯z-$\emptyset$-e-n & \armenian{օւզէն} & uz-ej-i-n & \armenian{ուզէին} \\
& \multicolumn{2}{l|}{$\sqrt{}$-{\thgloss}-{\pst}-{\agr}}& \multicolumn{2}{l|}{$\sqrt{}$-{\thgloss}-{\pst}-{\agr}}\\ 

\hline 
\end{tabular}
\end{table}


\subparagraph{Tenses built from the subjunctive: Indicative and future }

 \translatorHD{In Vozim, many other tenses seem to be built off of the subjunctive (Table \ref{tab:Van:subdialect:Vozim:morpho:verb:paradigm:complexSubjunctive}). The indicative present and past imperfective are built by adding the prefix /k-/ before the subjunctive present and subjunctive past. The future and future perfect are formed also by adding the proclitic /piti/ before the appropriate subjunctive form. SWA behaves essentially the same and I don't provide its paradigm. }

\begin{table}[H]
 \centering
 \caption{Forms that are built from the subjunctive forms for the verb `to want' in the Vozim subdialect of the Van dialect}
 \label{tab:Van:subdialect:Vozim:morpho:verb:paradigm:complexSubjunctive}
 \begin{tabular}{|l|ll|ll|}
\hline & 
\multicolumn{2}{l|}{Indicative present <\armenian{ներկայ}> } & \multicolumn{2}{l|}{Indicative past imperfective <\armenian{անկատար}>} \\
1SG & k-ou̯z-i-m & \armenian{կօւզիմ} & k-ou̯z-$\emptyset$-e-$\emptyset$ & \armenian{կօւզէ} \\
2SG & k-ou̯z-i-s & \armenian{կօւզիս} & k-ou̯z-$\emptyset$-e-ɾ & \armenian{կօւզէր} \\
3SG & k-ou̯z-i-$\emptyset$ & \armenian{կօւզի} & k-ou̯z-e-$\emptyset$-ɾ & \armenian{կօւզէր} \\
1PL & k-ou̯z-i-nkʰ & \armenian{կօւզինք} & k-ou̯z-$\emptyset$-e-nkʰ & \armenian{կօւզէնք} \\
2PL & k-ou̯z-i-kʰ & \armenian{կօւզիք} & k-ou̯z-$\emptyset$-e-kʰ & \armenian{կօւզէք} \\
3PL & k-ou̯z-i-n & \armenian{կօւզին} & k-ou̯z-$\emptyset$-e-n & \armenian{կօւզէն} 
\\
& \multicolumn{2}{l|}{{\ind}-$\sqrt{}$-{\thgloss}-{\agr}}& \multicolumn{2}{l|}{{\ind}-$\sqrt{}$-{\thgloss}-{\pst}-{\agr}}
\\ \hline 
& \multicolumn{2}{l|}{Future <\armenian{ապառնի}>} & \multicolumn{2}{l|}{Future perfect <\armenian{անցեալ ապառնի}> } \\
1SG & t-ou̯z-i-m & \armenian{տօւզիմ} & t-ou̯z-$\emptyset$-e-$\emptyset$ & \armenian{տօւզէ} \\
2SG & t-ou̯z-i-s & \armenian{տօւզիս} &t-ou̯z-$\emptyset$-e-ɾ & \armenian{տօւզէր} \\
3SG &t-ou̯z-i-$\emptyset$ & \armenian{տօւզի} & t-ou̯z-e-$\emptyset$-ɾ & \armenian{տօւզէ} \\
1PL & t-ou̯z-i-nkʰ & \armenian{տօւզինք} & t-ou̯z-$\emptyset$-e-nkʰ & \armenian{տօւզէնք} \\
2PL & t-ou̯z-ikʰ & \armenian{տօւզիք} & t-ou̯z-$\emptyset$-e-kʰ & \armenian{տօւզէք} \\
3PL & t-ou̯z-i-n &\armenian{տօւզին} & t-ou̯z-$\emptyset$-e-n & \armenian{տօւզէն} 
\\
& \multicolumn{2}{l|}{{\fut} $\sqrt{}$-{\thgloss}-{\agr}}& \multicolumn{2}{l|}{{\fut} $\sqrt{}$-{\thgloss}-{\pst}-{\agr}}
\\\hline \end{tabular}
\end{table}

\subparagraph{Present perfect and past perfect}

\translatorHD{In SWA, the present perfect (Table \ref{tab:Van:subdialect:Vozim:morpho:verb:paradigm:presentPerfect}) and past perfect (Table \ref{tab:Van:subdialect:Vozim:morpho:verb:paradigm:pastPerfect}) in are formed by combining a special non-finite form with the present/past auxiliary. For SWA, this non-finite verb can be either the resultative participle (verb with suffix /-ɑd͡z/) or the evidential participle (verb with suffix /-eɾ/). Vozim uses a similar system. The non-finite form is labeled as just a `past participle' by Adjarian (which I suspect is a perfective converb), and this form uses /-iɾ/ <\armenian{իր}>. }

\begin{table}[H]
 \centering
 \caption{Present perfect <\armenian{յարակատար}> of the verb `to want' in the Vozim subdialect of the Van dialect}
 \label{tab:Van:subdialect:Vozim:morpho:verb:paradigm:presentPerfect}
 \begin{tabular}{|l|ll|ll|}
\hline & \multicolumn{2}{l|}{Vozim (Van)} & \multicolumn{2}{l|}{cf. SWA} \\
1SG & ou̯z-iɾ i-m & \armenian{օւզիր իմ} & uz-eɾ e-m & \armenian{ուզեր եմ} \\
2SG & ou̯z-iɾ i-s & \armenian{օւզիր իս} & uz-eɾ e-s & \armenian{ուզեր ես} \\
3SG & ou̯z-iɾ i & \armenian{օւզիր ի} & uz-eɾ e & \armenian{ուզեր է} \\
1PL & ou̯z-iɾ i-nkʰ & \armenian{օւզիր ինք} & uz-eɾ e-ŋkʰ & \armenian{ուզեր ենք} \\
2PL & ou̯z-iɾ i-kʰ & \armenian{օւզիր իք} & uz-eɾ e-kʰ & \armenian{ուզեր էք} \\
3PL & ou̯z-iɾ i-n & \armenian{օւզիր ին} & uz-eɾ e-n & \armenian{ուզեր են} \\
& \multicolumn{2}{l|}{$\sqrt{}$-{\perfcvb} {\aux}-{\agr}}& \multicolumn{2}{l|}{$\sqrt{}$-{\eptcp} {\aux}-{\agr}}\\ 

\hline 
\end{tabular}
\end{table}


\begin{table}[H]
 \centering
 \caption{Past perfect <\armenian{գերակատար}> of the verb `to want' in the Vozim subdialect of the Van dialect}
 \label{tab:Van:subdialect:Vozim:morpho:verb:paradigm:pastPerfect}
 \begin{tabular}{|l|ll|ll| }
\hline & \multicolumn{2}{l|}{Vozim (Van)} & \multicolumn{2}{l|}{cf. SWA} \\
1SG & ou̯z-iɾ $\emptyset$-e-$\emptyset$ & \armenian{ուզիր է} & uz-eɾ ej-i-$\emptyset$ & \armenian{ուզեր էի} \\
2SG & ou̯z-iɾ $\emptyset$-e-ɾ & \armenian{ուզիր էր} & uz-eɾ ej-i-ɾ & \armenian{ուզեր էիր} \\
3SG & ou̯z-iɾ e-$\emptyset$-ɾ & \armenian{ուզիր էր} & uz-eɾ e-$\emptyset$-ɾ & \armenian{ուզեր էր} \\
1PL & ou̯z-iɾ $\emptyset$-e-nkʰ & \armenian{ուզիր էնք} & uz-eɾ ej-i-ŋkʰ & \armenian{ուզեր էինք} \\
2PL & ou̯z-iɾ $\emptyset$-e-kʰ & \armenian{ուզիր էք} & uz-eɾ ej-i-kʰ & \armenian{ուզեր էիք} \\
3PL & ou̯z-iɾ $\emptyset$-e-n & \armenian{ուզիր էն} & uz-eɾ ej-i-n & \armenian{ուզեր էին} \\
& \multicolumn{2}{l|}{$\sqrt{}$-{\perfcvb} {\aux}-{\pst}-{\agr}}& \multicolumn{2}{l|}{$\sqrt{}$-{\eptcp} {\aux}-{\pst}-{\agr}}\\ 

\hline 
\end{tabular}
\end{table}


\subparagraph{Past perfective or aorist}\label{sec:Van:subdialect:vozim:verb:paradigm:pastperf}

\translatorHD{The past perfective (Table \ref{tab:Van:subdialect:Vozim:morpho:verb:paradigm:pastperfectiveAorist}) is also called the aorist. In SWA for /uz-e-l/ `to want', the past perfective is formed by taking the root and theme vowel, adding the aorist or perfective suffix /-t͡sʰ-/, and then adding the past suffix /-i/ and the appropriate agreement suffixes. The 3SG uses covert tense and agreement suffixes. The Vozim subdialect behaves the same, though the past suffix is /-ej/ and the theme vowel is /e/ in all but the 3SG. Note that in Adjarian's earlier transcriptions, he said the past suffix is /ĕi̯/ <\armenian{է}ʲ> but his paradigm uses /ej/ <\armenian{էյ}>. }


\begin{table}[H]
 \centering
 \caption{Past perfective or aorist <\armenian{կատարեալ}> of the verb `to want' in the Vozim subdialect of the Van dialect}
 \label{tab:Van:subdialect:Vozim:morpho:verb:paradigm:pastperfectiveAorist}
 \begin{tabular}{|l|ll|ll|}
\hline & \multicolumn{2}{l|}{Vozim (Van)} & \multicolumn{2}{l|}{cf. SWA} \\
1SG & ou̯z-e-t͡sʰ-ej-$\emptyset$ & \armenian{օւզէցէյ} & uz-e-t͡sʰ-i-$\emptyset$ & \armenian{ուզեցի} \\
2SG & ou̯z-e-t͡sʰ-ej-ɾ & \armenian{օւզէցէյր} & uz-e-t͡sʰ-i-ɾ & \armenian{ուզեցիր} \\
3SG & ou̯z-i-t͡sʰ-$\emptyset$-$\emptyset$ & \armenian{օւզից} & uz-e-t͡sʰ-$\emptyset$-$\emptyset$ & \armenian{ուզեց} \\
1PL & ou̯z-e-t͡sʰ-ej-ŋkʰ & \armenian{օւզէցէյնք} & uz-e-t͡sʰ-i-ŋkʰ & \armenian{ուզեցինք} \\
2PL & ou̯z-e-t͡sʰ-ej-kʰ & \armenian{օւզէցէյք} & uz-e-t͡sʰ-i-kʰ & \armenian{ուզեցիք} \\
3PL & ou̯z-e-t͡sʰ-ej-n & \armenian{օւզէցէյն} & uz-e-t͡sʰ-i-n & \armenian{ուզեցին} \\
& \multicolumn{2}{l|}{$\sqrt{}$-{\thgloss}-{\aor}-{\pst}-{\agr}}& \multicolumn{2}{l|}{$\sqrt{}$-{\thgloss}-{\aor}-{\pst}-{\agr}}\\ 

\hline 
\end{tabular}
\end{table}
\subparagraph{Imperative and prohibitive}\label{sec:Van:morphology:verb:paradigm:imp}

\translatorHD{For the imperative 2SG, SWA adds a zero morph /-$\emptyset$/ after the theme vowel /e/ for a verb like `to want' (Table \ref{tab:Van:subdialect:Vozim:morpho:verb:paradigm:Imp}). For the 2PL, SWA adds the sequence /-e-t͡sʰ-ekʰ/ after the root such that /-e-t͡sʰ/ forms the aorist stem, while /-ekʰ/ is the agreement marker. Vozim instead adds a vowel /i/ for the 2SG; it's unclear if this /i/ is the theme vowel or an added suffix. For the 2PL, the suffix /ekʰ/ is added after the aorist stem. }


\begin{table}[H]
 \centering
 \caption{Imperative forms <\armenian{հրամայական}> for the verb `to want' in the Vozim subdialect of the Van dialect}
 \label{tab:Van:subdialect:Vozim:morpho:verb:paradigm:Imp}
 \begin{tabular}{|l|lll|ll l|}
\hline & \multicolumn{3}{l|}{Vozim (Van)} & \multicolumn{3}{l|}{cf. SWA} \\
2SG & ou̯z-\'i & \armenian{օւզի՛} & $\sqrt{}$-?& uz-e-$\emptyset$ & \armenian{ուզէ՛} & $\sqrt{}$-{\thgloss}-{\imp}.2{\sg}
\\
2PL& ou̯z-e-t͡sʰ-ekʰ& \armenian{օւզէցէք} & $\sqrt{}$-{\thgloss}-{\aor}-{\imp}.2{\pl}& uz-e-t͡sʰ-ekʰ& \armenian{ուզեցէք} & $\sqrt{}$-{\thgloss}-{\aor}-{\imp}.2{\pl}
\\\hline \end{tabular}
\end{table}

\translatorHD{For the prohibitive or negative imperative (Table \ref{tab:Van:subdialect:Vozim:morpho:verb:paradigm:Proh}), SWA adds the prohibitive formative /mi/ before the verb. The verb takes a suffix /-ɾ/ in the 2SG, and /-kʰ/ in the 2PL. In Vozim, the prohibitive is made up of the prefix /m-/ plus the imperative verb. } 


\begin{table}[H]
 \centering
 \caption{Negative imperative or prohibitive forms for the verb `to want' in the Vozim subdialect of the Van dialect}
 \label{tab:Van:subdialect:Vozim:morpho:verb:paradigm:Proh}
 \begin{tabular}{|l|lll|lll|}
\hline & \multicolumn{3}{l|}{Vozim (Van)} & \multicolumn{3}{l|}{cf. SWA} \\
2SG & m-\'ou̯z-i̯ & \armenian{մօ՛ւզի} & {\proh}-$\sqrt{}$-? & m\'i uz-e-ɾ & \armenian{մի ուզեր} & {\proh} $\sqrt{}$-{\thgloss}-2{\sg} \\
2PL & m-\'ou̯z-e-t͡sʰ-ekʰ & \armenian{մօ՛ւզէցէք} & {\proh}-$\sqrt{}$-{\thgloss}-{\aor}-{\imp}.2{\pl} & m\'i siɾ-ekʰ& \armenian{մի սիրէք} & {\proh} $\sqrt{}$-{\thgloss}-2{\pl} \\
\hline \end{tabular}
\end{table}

\translatorHD{On page \ref{page:157}, Adjarian left a footnote with examples of imperatives and prohibitives from Vozim (Table \ref{tab:Van:lit:Vozim:imp}), sentence}. 

\begin{table}[H]
 \centering
 \caption{Imperatives and prohibitives in the Vozim subdialect of the Van dialect}
 \label{tab:Van:lit:Vozim:imp}
 \begin{tabular}{|l|ll|ll|}
 \hline & \multicolumn{2}{l|}{Vozim (Van)}& \multicolumn{2}{l|}{cf. SWA }
 \\
 `bring! ({\sg})' & bʰi & \armenian{բՙի} & pʰeɾ& \armenian{բեր} \\
 `put! ({\sg})' & dʰi & \armenian{դՙի} & tʰiɾ& \armenian{դիր} \\
 `eat! ({\sg})' & ki & \armenian{կի} & ɡeɾ& \armenian{կեր} \\
 `don't eat! ({\sg})' & mou̯ti & \armenian{մ՚օւտի} & mi uteɾ& \armenian{մի՛ ուտեր} \\
 \hline
 \end{tabular}
 
\end{table}

\begin{exe}
\ex Vozim (Van) dialect
\gll bʰi dʰi mɑ d͡zi, ki χɑ d͡zi \\
bring.{\imp}.2{\sg} put.{\imp}.2{\sg} near me.{\dat}, eat.{\imp}.2{\sg} with me.{\dat}\\
\trans `Come bring it near me, eat with me.' \\
\armenian{բՙի դՙի մա ձի, կի խա ձի}
\end{exe}

\subparagraph{Non-finite forms}

\translatorHD{Finally, Adjarian lists the following non-finite forms of this verb (participles or converbs) in Table \ref{tab:Van:subdialect:Vozim:morpho:verb:paradigm:participle}. I give SWA forms for just some of them because it's unclear to me what these Vozim participles mean. Note that Adjarian uses the term `past participle' to mean multiple different types of non-finite forms: resultative participle with /-ɑd͡z/ in SWA, evidential participle /-eɾ/ in SWA. I suspect the Vozim /-iɾ/ is a perfective converb. } 

\begin{table}[H]
 \centering
 \caption{Participles or converbs <\armenian{դերբայներ}> for the verb `to want' in the Vozim subdialect of the Van dialect}
 \label{tab:Van:subdialect:Vozim:morpho:verb:paradigm:participle}
 \begin{tabular}{|ll|lll|lll|}
\hline & & \multicolumn{3}{l|}{Vozim (Van)} & \multicolumn{3}{l|}{cf. SWA} \\
 Infinitive & \armenian{անորոշ} & ou̯z-i-l & \armenian{օւզիլ} & $\sqrt{}$-{\thgloss}-{\infgloss} & uz-e-l & \armenian{ուզել} & $\sqrt{}$-{\thgloss}-{\infgloss} \\
 Past & \armenian{անցեալ} & ou̯z-ɑt͡s & \armenian{օւզած} & $\sqrt{}$-{\rptcp} & uz-ɑd͡z & \armenian{ուզած} & $\sqrt{}$-{\rptcp} \\
& & o̯uz-iɾ & \armenian{օւզիր}& $\sqrt{}$-{\perfcvb} & uz-eɾ & \armenian{ուզեր}& $\sqrt{}$-{\eptcp} \\
 Future & \armenian{ապառնի} & o̯uz-i-l-ʏ & \armenian{օւզիլիւ} & $\sqrt{}$-{\thgloss}-{\infgloss}-{\futcvb} & uz-e-l-u & \armenian{ուզելու} & $\sqrt{}$-{\thgloss}-{\infgloss}-{\futcvb} \\
\hline \end{tabular}
\end{table}

\begin{adjarianpage}\label{page:150}\end{adjarianpage}% should be 150

\section{Literature}

The first study on the Van dialect was done by someone named \armenian{Գրիչ} (see \armenian{Փորձ. Ա}. number 2, page 339-358)\footnote{I couldn't track down this reference. The word \armenian{Գրիչ} (SEA: /ɡəɾit͡ʃʰ/) is the Armenian word for `pen', which makes me think this was an anonymous entry.} during a study of `Manna' \citep{manana} by Garegin Srvandztiants (\armenian{Գ. Վ. Սրուանձտեան}; \translatorHD{SEA: /səɾvɑnd͡ztjɑn/, SWA: /səɾvɑnt͡sʰtjɑn/}). The second and last work was my work in German \citep{adjarian-1901-lautlehreVan}. This contains a detailed phonology of the Van dialect, done with European scientific transliteration. 

{\litoverview}
 
\begin{itemize}
 \item Literature with the Van dialect
 \begin{itemize}
 \item General Van dialect:
 \begin{itemize}
 \item \armenian{Արիստ. Վ. Տէր-Սարգսեան – Պանդուխտ Վանցին. Պօլիս}, 1875
\item \armenian{Արիստ. Վ. Սեդրակեան – Քնար Մշեցոց եւ Վանեցոց. Վղրշպտ}. 1874
\item \armenian{Գէորգ Շէրենց – Վանայ Սազ. Թիֆլիս, Ա}. 1886, \armenian{Բ}. 1899
\item \armenian{Գ. Վ. Սրուանձտեանց} 
\begin{itemize}
 \item 
– \armenian{Գրոց ու բրոց. Պօլիս}. 1874
 \item – \armenian{Մանանայ. Պօլիս}. 1876
 \item – \armenian{Համով հոտով. Պօլիս}.
\end{itemize}
\item \armenian{Տիգրան Տէրոյեան – Երգարան. Պոսթոն}. 1901, page 549-592
\item \armenian{Գրիչ – Պանդուխտ Վանցին}. (\armenian{Մատենախ}). \armenian{Փորձ. Ա. թ}. 3, \armenian{էջ} 113-135

 \end{itemize}
 \item Moks subdialect
 \begin{itemize}
 \item \armenian{Գարեգին Սարկաւագ – Սասմայ ծռեր. Թիֆլիս}. 1892, page 61-151
\item \armenian{Գ. Վ. Յովսէփեան – Ռոստամ Զալ. Ազգ. Հանդ. Է. էջ} 205-254

\begin{adjarianpage}\label{page:151}\end{adjarianpage}% should be 151

\item \armenian{Բ. Խալաթեանց – Իրանի հերոսները. Պարիզ}, 1901, \armenian{էջ} 45-56
\item \armenian{Ա. Աբեղեան – Թլուատ Դաւիթ. Թիֆլիս} 1902
\item \armenian{Մ. Աբեղեան – Դաւիթ եւ Մհեր. Շուշի} 1889
\item \armenian{Հայ-Արմէն – Մոկաց երգեր. Արեւել. մամուլ}. 1880, \armenian{էջ} 177-179


 \end{itemize}
 \item Besides these, Sarkis Haykuni (\armenian{Ս. Հայկունի}) has published 34 fables from Van, Moks, Norduz, Shatakh, and Vozim. See \citeauthor{Eminian}, \armenian{Վղրշպտ}. volume 2 and 4-6 (\armenian{Բ, Դ-Զ}). 
 \item There are a number of small manuscripts in the \citeauthor{Byurakn} periodical. 
 \begin{itemize}
 \item From Van – 1898, \armenian{էջ} 183, 459, 558, 583, 1899, \armenian{էջ} 15, 151
\item Shatakh – 1898, \armenian{էջ} 558, 569
\item from Vozim – 1899, \armenian{էջ} 20, 119, 298

 \end{itemize}

\end{itemize}
\end{itemize}


\section{Text samples}

{\sampleoverview}
 
\subsection{Van dialect }

Adjarian's source: I have taken from \armenian{Տէր-Սարգսեանցի Պանդուխտ Վանցի}, page 52-55, changing it to my orthography. 

\armenian{Էս քյանի տարի ի կուկյա̈ ու կընցնի. էս քյանի տարի ի մեր աչք քյո ճա̈մխի վէրէն կը խալի. էս քյանի տարի ի մեր սիռտ քյե խամար կը մաշվի. էս քյանի տարի ի քյո սիրուց կը մյանքյ կարօտ. է՜հ, վո՞վ իմ յարալու սիռտ պիրէր չիւմ քյո մօտ, վոր պա̈նիր տիսնիր ինոր ցավեր ու վէրքեր։ Ա՜հ, չանձ հըմէն կսկծավոր յես եմ վիրավոր, չանձ հըմէն խռօված յես եմ տրօրված, իրիցած ու մրկած։}

\armenian{Թօղ կյա̈րիւն կյա}̈. \armenian{էրկիր, սարերն ու տա̈ղտեր կանաչ, կարմիր ու նարընջի զա̈րտըրի. ա՜խ, յես ի՞նչ անեմ ինոնքյ. յես մնացի ատնէր ու կյերի. յես մնացի վոռպէօվէրի։}

\armenian{Թօղ ամըռվան պտուղներ խասնեն, միլաղներաց պէս շարվին ու կաթիւկ անեն վեր կանաչ խոտին, լղմոր լղմոր լըղպորվին, յես ի՜նչ անեմ ինոնքյ, կարօտ մնացինքյ. տիւ պէտք ես խամ տաս ինոնց ու խոտ տաս, հա̈մ ինոնց, հա̈մ ձիկ։}

\armenian{Թօղ խօջան ժօղվէ առծաթն ու վոսկին, ակն պավական, միւջաֆարներ անգյին, չանձ Վա̈նա̈ ծով լիցուցի, չանձ աշխըրքիս սարեր բա̈րդի, թեղի ու սեխչի, ինոնքյ հըմէն առանց սիրու, առանց սռտի ինչի՞ս խամար ի. ա՜հ, առանց քյե աստըվորիս մալն յես ի՞նչ անեմ։ Ա՜խ, թէ յես քյե խամար մեռած եմ, էլմ՚ կասեմ, աշխերքիս մալն յես ի՞նչ անեմ։}


\begin{adjarianpage}\label{page:152}\end{adjarianpage}% should be 152

\armenian{Մենքյ ծեռքյ տվինքյ ծեռաց, ուխտ արինքյ խտրաց, վոր խտրաց ապրին, խտրաց մեռնինքյ. քյո սէր տվիր, իմ սիռտն առիր ու մեր սիրու խօսք լսեց էրկինքյ, լսեց էրկիր. մենքյ ուխտ արինքյ, ու մէմէկու վէրէն խոկյի կու տինքյ. ա՜հ, ճղակտոր էլա̈վ մեր յէղունիկ սէր. ու քյէօքյախան էլա̈վ մեր սիռտ. տիւ կարիբ կնա̈ցիր ու ինգյա̈ր օտար էրկիր ու զատ մնացիր. տիւ քյըռտինքյ կը թա̈փես, յես առտսունքյ. տիւ մէօլրած ես, յես բէզըրած, օրս աստըվորս կնա̈ց, էլ զադ չմնաց։ Էնչա̈նք կանչեմ, սարեր լացուցեմ, յես առանց քյե սէրն ի՞նչ անեմ, սիռտն ի՞նչ անեմ, ուխտն ի՞նչ անեմ, կյանքն ի՞նչ անեմ։}

\armenian{Քյո ջուխտակ այվընիկ ծա̈քյեր կուց կուց առտսունքյ կը թա̈փեն, կուլա̈ն ու կասեն. «Մեր խէրն ինչի՞ չիրե, դէդէ մարէ, ապա յեփ պիտի կյա̈»։ Ձի կը խառցուցեն, սիռտս կը դաղեն. էլ ինոնց խապելու մա̈ֆա̈ր չմնաց. ասքն ու պարիկամ տիւր տրացին, ձի խառցմունք կ՚անեն ծեր մարթուց ի՞նչ խա̈բա̈ր կա. յե՞փ պիտի կյա}̈. \armenian{էլ խէրիքյ չէլա̈վ կարիբութան մէջ մնա. էլ խոկյիս էլա̈վ շատերաց սուտ խապելուց. յես ծեռքյից կնա̈ցի։ Տիւրր տրացին, ասքն ու խնամին յես ի՞նչ անեմ առանց իմ ծէտկիկ ծա̈քյերու աղին. յես ի՞նչ անեմ վորտին, առանց իմ նաղէլի կարիբին. աշխար ձի մութ ի, վո՞վ կիրիշկի վեր լաճերաց, կարիբիս մեռնեմ ուր ճամխըներաց։}

\armenian{Մեր տուն տեղ մեր ծեռքյից էլա̈վ. օտար խաֆքյու պէս մնացինք վեր չոր խըլի. վո՞վ պիտի մեր նեղութեն տիսնա, մեզ օղօրմի. խեխճ ու անտէր մնացինքյ. քյեղնից տվել մարթ չունինքյ. ի՞նչ կասես, սաղ սաղ մեռնե՞նքյ։}

\armenian{Խէրտ ու մէրտ խալիվորցիր են. յես ինոնց դարդն չեմ կանա քյաշե. յես քյե քիչ կյըրիցի, տիւ շատ իմացի. շոտ թօղ արե, էլ խէրիքյ ի. խէրիքյ ի տա̈ռն տա̈տէք, տա̈ռտակ նստէքյ. իսկի չէ տարին քյանի մ՚ կուռուշ փարա ճամխէքյ. մենքյ էրթա̈նքյ մուրանքյ, պիրենքյ քյո տղէյնե՞ր պախենքյ. էլ չենքյ կանա անել, ինչ վոր արինքյ՝ էն էլ խէրիքյ ի։}

\subsection{Diyadin subdialect}

Adjarian's source: From the village of Basargechar of New Bayazet. 

\armenian{Իմ խէր իմ ախպօր խէտ մէ օր առանց սէլ քյնացին (կամ քյացին) վոր քիւլա̈շ բերեն. քյամին կայնավ. շատ էլ քյամի էր}... 



\begin{adjarianpage}\label{page:153}\end{adjarianpage}% should be 153

... \armenian{ինչքյան սէլ բա̈րցին, քյամին խըրցներ վէր տըվեց, քըցեց քյետին. չուր էնի կառնի կը դնի, քյամին վէր կուտա։ Խէրս յերսօտ էր. քյամու վրէն յէրսօտավ, խըրցներ վէր տըվեց ասեց. էս էլ քյեզ, էս էլ քյեզ։ Փափախն էլ կը խանա կը ղլօդկա կը քըցա գյեսին. շորերն էլ վրէվէն խանեց քցեց. էս էլ քյեզ. խօ չէս ի քյա էս էլ պրծուս տանես։}


\begin{center}
 * *
\end{center}

\armenian{Զատկի խլուսուն էր. էկան իմ ախչկէն ուզելու. խէրն էլ ասաց իմ ախչիկ կուտամ քյօ տղին։ ուրանք լավ ին. ուրանց բնուեն լավ էր, ամա քյասիբ ին. է՛հ, ուր կընկյան կը պախա ասինքյ տըվեցինք. ամա ախչկա սրտօվ չէր։ Մնաց վոր իմ տղէք գիտին (գիտցած ին) թէ իմ մարթ թռանա (հանաք, կատակ) էրեր ա ուրանց խետ, թէ իմ ախչիկ կուտամ քյօ տղին. չին գիտցե վոր սրտանց էր ասեր էր. վոր յետօ իմացան թէ էս բա̈ն օղորթ ա, ուզեցան քրոչ թէ առի (արի) յետ դա̈ռի, մի՛ առնի։ Ախչիկ լէ վէրցրեց թէ յես հարուստ մարթու ախչիկ ըլնեմ, իմ բիւլոր ախբՙրներու անուն ափեմ գյետի՞ն։ Յետօ յեխբՙարներ կայնան թէ արի քյեզի յետ դարցուցենքյ, էլ չենքյ իտա էն տղին, լավ տղի կուտանքյ։ Իւր քիւր վէրցրեց թէ յեխբՙար ջանէ, չե՛ղի դառնալ. իմ խօր անվանի ամօթ ա։ Ուր անուն լէ Սօֆյան ա։}

\armenian{Մէ ամըսվա խարս էլավ. լավ խարսնիս էրէցինքյ. խարսնըսէն մէ ամիս յետօ մախացավ. խինգյ օր խիվընդցավ մէռավ։ Յէս կանիծի կասի. բօխչէդ կապոկ մնա. խինէդ քյամին տանա. սկի արժան չըլնես վոր դու ընէնց խօնար չես էղե. կ՚ասէր. մա՛յրիկ ջանէ, ա՛դէ ջանէ, յես մկա մեռնիմ վոր քի՞չ լա̈ս, էն վախտ կը մեռնիմ վոր օխտը ձեռքի շոր էլնի, վոր դնես հառէչդ իմ խամար լա̈ս։}

\armenian{Մկա իմ տան էրէխէք վրէն խաղ ա կապած}.
 
 \armenian{Յէս Սօփին եմ ծամավոր},
 
 \armenian{Դու Մանուկն իր խամավոր}.

 * * 


\armenian{Ղօրթմա (ճիշտ որ) խամավոր տղա էր. ամա քյասիբ էր. քյասիբին գինաս ի՛նչ ղդար պատվելի էլնի, ինչքյան լավ զրուցա, պատիվ չկա. քյասբի բա̈ն մերժուկ ա։}

\armenian{Մկա կիւլա̈մ. օր իրեքյ խետ կիւլա̈մ. բա չե՞մ իլա}̈. \armenian{էն շորեր վոր կարի, վոչ խաքյավ. ինչ վոր կարի՝ կապուկ մնաց. մկա}... 


\begin{adjarianpage}\label{page:154}\end{adjarianpage}% should be 154

... \armenian{կանիծե՞մ. էն վախտ կանիծի. իմ լիզուն չորնէր. մկա չեմ անիծե, յ}̵\armenian{ախու չեմ անիծե։}

\armenian{Քյօ դախին է՞տ էր. կըսա արթանքյ արթանքյ. մկա պրծա՞ր։}

Adjarian's source: This is narrated by the unfortuate mother of Sopia (\armenian{Սոփիա}), Aslik (\armenian{Ասլիկ}). I transcribed it during my summer travels of 1907 in Basargechar. 

\subsection{Moks subdialect}
Adjarian's source: See \citeauthor{Eminian}, 4, page 57. 

\armenian{Թաքյավուր մ՚ կէր, իրիք տղա ուներ, իրիք ախչըկ. էսաց}. – \armenian{Իս կը միռնիմ, խաֆք գյա, գէլ գյա, առչ գյա, ախչկըտիր կըտէք (կուտաք}), \armenian{էսաց. իս ինչ կը միռնիմ, իմ իտիվէն չգէք՝ էսաց. ինչ կը պսակվէք՝ մա ձի կընկտիք չը քընէք՝ էսաց. ծառ մը կա մի (մենք ծառ մը ունինք}), \armenian{իրիք խնձուր կը բռնը. Հուլիսի տասնըխնգին կը գյան, կը տանը՛ն. չըն թուղնը տըսնանք ինչ խնձուր ի (անմահական խնձուր ի}). \armenian{ան ցածրի խնձուր մինծ տղըն, ան միչի խնձուր միչնիկ տղըն, ան վէրի խնձուրն էլ պզտը տղըն։ Էսաց. յօթ օր իմ գիրիզման կը պախէք. գըշիր ճրաքն էլ չըք թուղնը ընցնը։}

\armenian{Խէր միռավ. տը տանին վիրուցին. էրկու միծ տղէք ասըցըն}. – \armenian{Խիտ էրթանք։}

\armenian{Փուքր էսաց}. – \armenian{Չընք էրթա։}

– \armenian{Ի՞նչըխ, մի խէր մէռնը, ասըցըն, մինք չէրթա՞նք խիտ՝}

\armenian{Պզտիկ ախբէրն էլ առըն. զիւրովէն գնացըն խիտ։}

\armenian{Մինչիվ զխէր պախըցըն, էկան, խրօխբէր նստավ թախտ, թաքյավուրութեն առըց։}

\armenian{Փուքր ախբէր էսաց}. – \armenian{Չէ՞ իս ձի ասըցը «չինք էրթա իտիվ»։}

\armenian{Առչ իրի. նստավ խնամաթոռ}. – \armenian{Ձի մինծ քիւր տը տէք ձը, էսաց։}

\armenian{Էրկու մինծ ախբէր ասըցըն}. – \armenian{Մինք մի քիւր ի՞նչըխ տը տանք առչին. չընք իտա. տանը տ՚ուտէ։}

\armenian{Պզտիկ ախբէրն էսաց}. – \armenian{Իմ խօր խուսք չէրի, կուշտ կիրա. ա՛ս էլ չէնըմ, տը տամ տանը՛ (պիտի տամ տանի)։}

\armenian{Մինջ քիւր առչ առըց տարախ։}

\begin{adjarianpage}\label{page:155}\end{adjarianpage}% should be 155

\subsection{Norduz}


Adjarian's source: See \citeauthor{Eminian}, 4, page 97. 

\armenian{Մէկ լավ տղէ-մ կէր. զինքն էր, ուր մէր. էլավ գնաց մէկ գեղ. ասաց}. – \armenian{Տ՚էհամ մօ ռէս, ըլնեմ վօրթկարած։}

\armenian{Գնաց, խնդրվավ, ասաց}. – \armenian{Ձի անես վօրթկարած, շախվեմ։}

\armenian{Ասաց}. – \armenian{Ա՛յ տղա, դիւ կանա՞ս վօրթիկներ շախես։ Ասաց}. – \armenian{Կանամ։}

\armenian{Ասաց. Դէ՛, գնա՛ յ}̵\armenian{էռչէվ վօրթկներաց. կիրակնեց կիրակի քիւ խաց-մաց ժօղվի յ}̵\armenian{ըմէն վօրթկի մէկ դիտր ցօրեն տամքիւ հախ։}

\armenian{Էլավ, գնաց յ}̵\armenian{էռչէվ վօրթկներաց։ Էն վօրթիկ իշ կը դըռչըկէր, պառաի կէհէր, էն կըհէր. տղէն կըհէր կը տփէր, բիրէր մըչ վօրթկներաց. իշ կը մնէր դումահիք (ետեւէն}), \armenian{կը տփէր, չում կը խասցնէր վօրթկներաց։}

\armenian{Էդա լավնով մէկ շարթվան մէչ խինք, վեց վօրթիկ սպանեց։} 

\armenian{Գէղական էլան, գնացին ռէսին ասին}.

– \armenian{Մենք էն վօրթկարած կապուլ չընք անի. մե վօրթկներ յ}̵\armenian{ըմէն սպանեց։}

\armenian{յ}̵\armenian{էլան մլուցին դիւս, գեղից խանին։}


\subsection{Shatakh}

Adjarian's source: See \citeauthor{Eminian}, 4, page 369-370. 

\armenian{Մհեր Սասուն կը նստէր թաքյավոր},

\armenian{Մըսրա Մէլիք Մըսըր կը նստէր թաքյավոր}.

\armenian{Մըսրա մէլիքի կնիկ ճիժ չունէր։}

\armenian{Մըսրա մէլիքի կնիկ իրան մէչ կը մտածի}.

\armenian{Մէլիքից իրավունք կառնի, կըսի}.

– \armenian{Ձի ճիժ չունեմ, վոր Մըսըր թաքյավոր ըլնը։}

\armenian{Սերմս փոխեմ, տղէմ ըլնը, ըլնը Մըսրա Մէլիք}.

\armenian{Մըսըրա թաքյավորութուն կանգնի։}

\armenian{Ուր գյօդիկ, լաչիկ կօղօրկի Մհեր թաքյաւորի խամար}.

\armenian{Մհեր կը տիսնա գյօդիկ, լաչիկ օղօրկիր ի ուր խամար},

\armenian{Ըսիր ի}. – \armenian{«Վոր յէս գյոդիկ, լաչիկ օղօրդկիր էմ}.

\armenian{Էն չգյա, քըն զիս շատ կնիկ ի»։}

\armenian{Էն տեղ Մհեր ի՛լ կըսի}.




\begin{adjarianpage}\label{page:156}\end{adjarianpage}% should be 156

– \armenian{Վոր նա խաբար էն տվիր ի, յէս տըհամ։}

\armenian{Կնիկ կըսի}. – \armenian{Մա՛րթ, մէ՛հա. տղիկ բան չի քի խամար։}

\armenian{Կըսի}. – \armenian{Կնիկ, յէս տըհամ. յէս չըհամ՝ յէս էլ ինու ցեղ կնիկ էմ}.

\armenian{Ճարն ի՛նչ ի. տհամ. չըհամ՝ չըլնը։ Իլավ գնաց},

\armenian{Էրկու գիշեր, յան իրեք գիշեր մօտը քնախ։}

\armenian{Դարձավ էրի տուն։}

\armenian{Մըսրա-Մէլիք զինք մեռավ։}

\armenian{Ինը ամիս, ինը օր վոր թըմմավ, Աստված ինու տղէմ իտու։}

\subsection{Vozim subdialect}

Adjarian's source: I personally transcribed tihs in Paris 1897, with a recent migrant from Vozim. 

\armenian{Կըլնին չուրս մարթ, կիան լպուտութեն. կիան (կամ կիհան) սարըմ գՙյըլօխ կը տէսնին վոր գՙյիւղ կը գՙյա. խէյնգյ խատ չարջար գՙյըլօվ կը տէսնին վոր գՙյիւղ կը գՙյա. խէյնգյյ խատ չարջար կը գՙյան. մու ջիւջ ընկիր ուր շլաքը գՙէտին կը դՙընի, մօւ կըսի. «Յէրի յէկէք ժօղվէք իսի բՙամ տ՚ըսիմ (բան մը պիտի ըսեմ}). \armenian{ըշկի (նայէ՛) ձՙիւրէն (կարդալ ձՙիուրէն) խէյնգյ չարջաք քէօրթ գՙյիւղեր կը գՙյան. մօւ մինք էլնինք ասօւնցմնէ թալնըվինք, ալ մի շաշխանէք վորէ՞ ինք դՙրիր վեր մի թիվէյն։ Մօւ քյանի վոր իմ սիրտ կը տրախկա՝ շաշխանէն իս անօւնց ձՙեռ չմէ՛յտա (չեմ ի տար}). \armenian{մօւ զա̈րկըցէք, վոր փախնէ՝ ուր կընկյան քյաֆէք վար ուր գՙըլխուն յէղնէ»։}


\begin{center}
 * *
\end{center}

\armenian{Օ՛րըմ ղօլմիւդիւր կկանչի}. – \armenian{Պօ՛զօ, յարի, քյի բՙամ տըսիմ։}

\armenian{Պօզօն կէլնի կիա (կերթայ}), \armenian{կըսի}.

– \armenian{Բՙառկօն (բարի երեկոյ}), \armenian{Կարապիտ աղա. ի՞նչ կը խրամայիս։}

\armenian{Կըսի}. – \armenian{Հասօր քյիւ ջՙուրէյն տըտաս, վորը բինբաշէյն խեծնի իա Կծվակ (գիւղ մ՚է)։}

\armenian{Կըսի}. – \armenian{Չի, Կարապիտ աղա, իմ ջՙուրէյն մարթիւ չմէ՛յտա։} 

– \armenian{Չի, տտաս։}

– \armenian{Չմէ՛յտա. վալլահ, Կարապիտ աղա, մկա իմ վէյզ կտրիս ու իմ ջՙուրէյն չմէյտա։}

– \armenian{Վորէ՞ չսէ՛յտա (չես ի տար}), \armenian{մահռուզ (անիծած) պապ, ի՞նչ անօւն կը դՙնիս օր չսէ՛յտա. տղէյքդՙիր, գՙյացէք անօւր ջՙուրէյն բՙիրէք։}


\begin{adjarianpage}\label{page:157}\end{adjarianpage}% should be 157

\armenian{Տղէք կիան, ջՙուրէյն մըսրը վըրվէն կը հարցըկին, կօւրդՙէյն (կորդին «թամբ») կը դՙնին վրէն, կառնին կը բՙիրին, կկապին վա̈ր դՙռան. կիան կըսին}. – \armenian{Կարապի՛տ աղա, ջՙուրէյն բՙիրիցէյնք։}

\armenian{Պօզօն կըսի}. – \armenian{Կարապի՛տ աղա, ջՙուրէյն տարա՞ր։}

– \armenian{Կը տանէյմ ու քյիւ աչքն էլ կխանիմ։}

– \armenian{Է՛հ, աղէկ, Կարապիտ աղա, թխ քյիւ խաբար յէղնի։}

\armenian{Պօզօն կելնի կիա ցա՛ծր, կընկնի դՙէօս (դուրս}), \armenian{կիա կը կանչի}.

– \armenian{Պո՛ւղուս, յար՚ ըսիմ. գՙյըտի՞ս, Կարապիտ տղէն զէօրէօվ մի (մեր) ջՙուրէյն տարավ յարի մի ճակիր (զէնքեր) կապինք։} 

(\armenian{Շարունակութիւնը Պօզօն կը պատմէ)։}

\armenian{Ճակիրը կապըցէյնք, գՙյացէյնք վա̈ր քէօշքը՛ դՙռան կանգնանք. կանչըցէի. «Կարապի՛տ աղա, քյիւ գՙյըլօխ հէտըսէն բՙի}\footnote{\translatorHD{Adjarian left a lengthy footnote here that is I moved to \S\ref{sec:Van:morphology:verb:paradigm:imp}. }} \armenian{դ՚էօս։ Կարապի՛տ աղա, ջՙուրէյն տը տանէյս. մօւ ասացէլ, թը դՙիւ չտանէյս, քյիւ մէր անըծիմ, քիւ յօթ պորտ անըծիմ՝ թը դՙիւ չտանէյս. դի յէ՛րի տար։ Իս իմ ջՙիւրիւն հաֆսար (սանձ) բՙըռնըցէյ ու տարա կապըցէյ վը մսրէյն։ Կարպիտ աղա, մկա կտրէյճ իս, յէ՛րի տար. ջՙուրէյնս տարա։ Քյիւ բինբաշէյն վո՞րն ի, ասի անօւր, թխ գՙյա՝ ան տանէ։ Քյիւ բինբաշէյն վո՞րն ի, ասի անօւր, թխ գՙյա ՝ ան տանէ։ Դՙէօ՛ չէ, քյիւ բինբաշէյն չէ, ձՙի յօթ խէր գՙյա՝ չկա՛նէ տանէ։ Վալլահ իս մկա փոսուն (փորոտիք) քյիւ փուրէն տխանիմ։ Դՙիւ գՙյըտի՞ս իս վո՞րն իմ. մօւ իս Հզմա Պօզօ տղէն իմ, գՙյըտի՞ս»։}

\armenian{Մօւ իս տարտ իմ ջՙուրէյն, ալ մարթ հիմ յէրէվան չէկավ. չկյախշցան (չհամարձակեցան)։ Մօւ Կարապիտ աղէն էլավ գՙնաց հիքմէթ, ասից. «Անա մարթ չմօ՛ւզի մանչ (մէջ) իմ գՙյեղէյն. նա մարթէյք մարթասպան ին. յա նա մարթէյկ տը մլիս դՙէօս, յա մինք տիանք»։}

\armenian{Մօւ իս էլա, ի՛նչ կէր ձէյ, էլա կէր ձէյ թազէյս մ ու լօփ մ՚ (կապերտ}). \armenian{բՙա̈րցը վր իմ ջՙիւրիւն, ու խեծա իմ ջՙուրէյն, ու շաշխանէն դՙրի վր իմ թիվէյն, ըսի. «Կարապիտ աղա, իս կիամ. թը դՙիւ խարէր խոկյով չգՙաս իմ հէռչիկ սա (թրք. իսէ՝ եթէ}), \armenian{իս քյիւ մեռել անըծիմ. թը դՙիւ վորցը մարթ իս՝ մըչ}... 


\begin{adjarianpage}\label{page:158}\end{adjarianpage}% should be 158

... \armenian{գՙյեղէյն բաբագէթուիւն չվէ՛լի (չի՛ վայելեր}). \armenian{արի իմ հէռչիվ ու քի նշանց տամ»։}

\armenian{Մօւ էլա գՙընացէյ մանչ իմ նայարնէրիւն. մօւ իմ նայարնիր ըսէյն ձէյ. վորէ՞ էկար։ Մօւ իս էլա գՙացէյ Խլաթու յէրկէյր։ Խլաթցէյք ուրանց կնէյկ հա̈լէօվ՝ կօւզէն վր ձէյ զօրբըթեն էնէն։ Մօւ իս Հըզմըցէյ յէղնէյմ ու տառան Խլաթցօց էվալլահ էնի՜մ. մօւ իմ խէրէյն ղաբուլ չէ ըրած. ասը՛ իամ ասլանիր թղ զանըն, ըզձը սպանըն աղէկ ի՝ քա̈նց Խլացէյք վոր վր ձը զօրբըթեն տէնին։}



\chapter{Tigranakert}

\section{Background and literature}

\begin{adjarianpage}\label{page:159}\end{adjarianpage}% should be 159

The central city of this dialect is Tigranakert (Turkish: Diyarbakır). Similar to the Moks subdialect, this is the southern border of Armenian, south of which Kurdish or Arabic are spoken. It spreads from the southwest until Urfa or Edessa; and starting close to that, the Euphrates river takes the dialect's western borders until Arghni, and then with a straight line until Lice. The northern and eastern border forms the Mush dialect. Based on this, the locations where Tigranakert is spoken are the city of Tigranakert, and Hazzo, Khian, Siverek, Edessa, \armenian{եւ} Lice. The latter is originally Kurdish-speaking, but there are many migrants from Tigranakert who have revived the Armenian dialect. 

The dialect of Tigranakert is still not studied at all. Published manuscripts that use this dialect or its other branches are very insignificant pieces. These are small collections of proverbs, riddles, and popular blessings, in the Istanbul \citeauthor{Byurakn} periodical. For example:
\begin{itemize}
 	\item from Tigranakert: 
 	\begin{itemize}
 		\item year 1898, page 332, 337, 413, 445, 470, 569, 654, and 700 
 		\item year 1899, page 545, and 731 
 		\item year 1900, page 330, 450, and 677 
 	\end{itemize}
 	\item from Khian: 
 	\begin{itemize}
 		\item year 1898, page 301, 493, and 701 
 		\item year 1899, page 650 
 	\end{itemize}
 	\item from Hazzo: 
 	\begin{itemize}
 		\item year 1898, page 538 
 		\item year 1899, page 37, 75, and 641 
 	\end{itemize}
 	\item from Hazro: 
 	\begin{itemize}
 		\item year 1899, page 805 
 		\item year 1900, page 263 
 	\end{itemize}
 	\item from Urfa 
 	\begin{itemize}
 		\item year 1900, page 331 
 	\end{itemize}
 	\item from Siverek 
 	\begin{itemize}
 		\item year 1900, page 331 
 	\end{itemize}
 	 
\end{itemize}

There is a sample of the Tigranakert dialect in \citeauthor{ArevelyanMamul} 1884, page 470-472, but it is not `\armenian{հարազատ}'.\footnote{\translatorHD{I'm not sure if the word \armenian{հարազատ} here is meant to say that the text is a secondary source, a translated source, or just not familiar to Adjarian.}}

During my summer travels in 1910, I got acquainted with two newcomers from Tigranakert; one was a teacher, and the other a medical student. With their help, I started... 

\begin{adjarianpage}\label{page:160}\end{adjarianpage}% should be 160

... to organize a study of the Tigranakert dialect, from which I take the following concise outlines. 

\section{Phonology}

\subsection{Segment inventory}

\subsubsection{Vowel inventory}

The Tigranakert dialect occupies a middle ground between the Mush and Malatya dialects.Among the vowels, the vowel /æ/ <\armenian{ա}̈> is extremely common, while the vowels /oe, ʏ/ <\armenian{էօ, իւ}> are rarely sometimes found in foreign words. 

\subsubsection{Consonant inventory}

\paragraph{Laryngeal changes}
In its consonants, the Tigranakert dialect presents a system that is entirely different from all the other dialects that we have seen up till now. From the three degrees of consonants in Old Armenian, only two remain: voiced and voiceless aspirated. The Armenian voiced plosives become voiceless aspirates, the voiceless unaspirated become voiced, while the voiceless aspirated stay the same (Table \ref{tab:Tigranakert:phonology:inventory:cons:larygenal}). 


\begin{table}[H]
 	\centering 
 	\caption{Laryngeal changes from Classical Armenian to the Tigranakert dialect}
 	\label{tab:Tigranakert:phonology:inventory:cons:larygenal}
 	\begin{tabular}{|l| ll|ll| ll|}
 		\hline & \multicolumn{2}{l|}{Classical Armenian} &\multicolumn{2}{l|}{> Tigranakert} & \multicolumn{2}{l|}{cf. SEA} \\ 
 		 
 		`mouth' &beɾɑn & \armenian{բերան} & pʰeɾæn & \armenian{փէրա̈ն} &beɾɑn & \armenian{բերան} \\ 
 		\armenian{՝}barefoot' & bokik & \armenian{բոկիկ}& pʰobiɡ & \armenian{փօբիգ} & bopik & \armenian{բոպիկ} \\ 
 		\armenian{՝}knife' & dɑnɑk & \armenian{դանակ}& tʰænæɡ & \armenian{թա̈նա̈գ} & dɑnɑk & \armenian{դանակ} \\ 
 		\hline 
 	\end{tabular}
\end{table}


(In the Hazzo subdialect, we find the voiced aspirates group, similar to the Mush dialect. But here the phonetic rules have taken a step further; the voiceless unaspirated sound also turned to voiced aspirates (Table \ref{tab:Tigranakert:phonology:inventory:cons:voiceasp})). 


\begin{table}[H]
 	\centering 
 	\caption{Voiced aspirates in the Hazzo subdialect of the Tigranakert dialect}
 	\label{tab:Tigranakert:phonology:inventory:cons:voiceasp}
 	\begin{tabular}{|l| ll|ll| ll|}
 		\hline & \multicolumn{2}{l|}{Classical Armenian} &\multicolumn{2}{l|}{> Hazzo (Tigranakert)} & \multicolumn{2}{l|}{cf. SEA} \\ 
 		 
 		`mouth' &beɾɑn & \armenian{բերան} & pʰeɾæn & \armenian{փէրա̈ն} &beɾɑn & \armenian{բերան} \\ 
 		\armenian{՝}barefoot' & bokik & \armenian{բոկիկ}& pʰobiɡ & \armenian{փօբիգ} & bopik & \armenian{բոպիկ} \\ 
 		\armenian{՝}knife' & dɑnɑk & \armenian{դանակ}& tʰænæɡ & \armenian{թա̈նա̈գ} & dɑnɑk & \armenian{դանակ} \\ 
 		\armenian{՝}he got up' & kɑnɡnet͡sʰɑu̯ & \armenian{կանգնեցաւ}& ɡʰɑnnɑv & \armenian{գՙաննավ} & kɑŋɡnet͡sʰ & \armenian{կանգնեց} \\ 	 
 		\armenian{՝}woman' & & & ɡʰnik & \armenian{գՙնիկ} & kənik & \armenian{կնիկ} \\ 
 		\armenian{՝}place' & teɬ & \armenian{տեղ}& dʰi̯eχ & \armenian{դՙեխ} & teʁ & \armenian{տեղ} \\ 	 
 		\armenian{՝}he would want' & & & ɡʰuzeɾ & \armenian{գՙուզէր} & kuzeɾ & \armenian{կուզեր} \\ 	 
 		\hline 
 	\end{tabular}
\end{table}


\paragraph{Arabic consonants /ʕ, ħ, q/ / <\armenian{ՙ, հՙ, ղՙ}/ and /lʲ/ <\armenian{լՙ}> }


Among the consonants, the following are added: /ʕ, ħ, q, lʲ/ <\armenian{ՙ, հՙ, ղՙ, լՙ}>. The first three are borrowed from Arabic, and they are found only in Arabic words. (The <\armenian{ՙ}> signifies the Arabic sound /ʕɑjn/ <\armenian{ՙայն}> (\translatorHD{<\textarab{ع}>) / ʕ/}), the <\armenian{ղՙ}> signifies the Arabic sound </kʰɑf/> \armenian{քաֆ} (\translatorHD{<\textarab{ق}> /q/}), and the <\armenian{հՙ}> signifies Arabic <\armenian{հա}> /hɑ/ (\translatorHD{<\textarab{ح}> /ħ/}). For example, Table \ref{tab:Tigranakert:phonology:inventory:cons:arabic}. 


\begin{table}[H]
 	\centering 
 	\caption{New consonants /ʕ, ħ, q/ <\armenian{ՙ, հՙ, ղՙ}> from Arabic in the Tigranakert dialect}
 	\label{tab:Tigranakert:phonology:inventory:cons:arabic}
 	\begin{tabular}{|l|ll|ll| }
 		\hline & \multicolumn{2}{l|}{Arabic} &\multicolumn{2}{l|}{> Tigranakert } \\ 
 		 
 		`scorpion' & <ʿaqrab> & \textarab{عقرب}& ʕɑqɾɑb & \armenian{ՙաղՙրաբ} \\ 
 		`life' & <ʿumr> & \textarab{عمر} & ʕœmɾ & \armenian{ՙէօմր} \\ 
 		`zaatar' & <zaʿtar> & \textarab{زعتر} & zɑʕtʰɑɾ & \armenian{զաՙթար} \\ 
 		`heart (Arabic), false (Tigranakert)' & <qalb> & \textarab{qalb} & qælb & \armenian{ղՙա̈լբ} \\ 
 		`halva' & <ḥalwā> & \textarab{حلوى} & ħælvæ & \armenian{հՙա̈լվա}̈ \\ 
 		`jujube' & <ʿunnāb> & \textarab{عناب} & ʕunnɑb & \armenian{ՙուննաբ} \\ 
 		\hline & \multicolumn{2}{l|}{Classical Armenian} &\multicolumn{2}{l|}{> Tigranakert } \\ 
 		 
 		`cuckoo' & /kəku/ & \armenian{կկու}& quqqu & \armenian{ղՙուղՙղՙու} \\ 
 		 
 		\hline 
 	\end{tabular}
\end{table}



The sound <\armenian{լՙ}> is /l/ <\armenian{լ}> with a soft pronunciation (\translatorHD{/lʲ/}), similar to Russian /lʲ/ <ль> and it is found in native Armenian words (Table \ref{tab:Tigranakert:phonology:inventory:cons:lj}). 


\begin{table}[H]
 	\centering 
 	\caption{Sound /lʲ/ <\armenian{լՙ}> in the Tigranakert dialect}
 	\label{tab:Tigranakert:phonology:inventory:cons:lj}
 	\begin{tabular}{|l| ll|ll| ll|}
 		\hline & \multicolumn{2}{l|}{Classical Armenian} &\multicolumn{2}{l|}{> Tigranakert} & \multicolumn{2}{l|}{cf. SEA} \\ 
 		 
 		`fawn' & ul & \armenian{ուլ} & ulʲ & \armenian{ուլՙ} & ul & \armenian{ուլ} \\ 
 		`head' & ɡəluχ& \armenian{գլուխ} & kʰlʲuχ & \armenian{քլՙուխ} & ɡəluχ & \armenian{գլուխ}\\ 
 		`Pleiades' & boi̯lkʰ& \armenian{բոյլք} & pʰulʲkʰ & \armenian{փուլՙք} & bujlkʰ & \armenian{բույլք}\\ 
 		`to wash' & lu̯ɑnɑl& \armenian{լուանալ} & lʲvænæl & \armenian{լՙվա̈նա̈լ} & ləvɑnɑl & \armenian{լվանալ} \\ 
 		`to bathe' &loɡɑnɑl & \armenian{լոգանալ} & lʲoɡnæl & \armenian{լՙօգնա̈լ} & loɡɑnɑl& \armenian{լոգանալ} \\ 
 		\hline 
 	\end{tabular}
\end{table}

\paragraph{Patalalized stops /kʰ, kʰʲ, /dʲ/ <\armenian{կյ, քյ, դյ}>} 

Similar to the Van dialect, here we also find the sounds /kʰ, kʰʲ/ <\armenian{կյ, քյ}> and also the sound /dʲ/ <\armenian{դյ}> (Table \ref{tab:Tigranakert:phonology:inventory:cons:kj}). 


\begin{table}[H]
 	\centering 
 	\caption{Palatalized sounds /kʰ, kʰʲ, dʲ/ <\armenian{կյ, քյ, դյ}> in the Tigranakert dialect}
 	\label{tab:Tigranakert:phonology:inventory:cons:kj}
 	\begin{tabular}{|l| ll|ll| ll|}
 		\hline & \multicolumn{2}{l|}{Classical Armenian} &\multicolumn{2}{l|}{> Tigranakert} & \multicolumn{2}{l|}{cf. SEA} \\ 
 		 
 		`I wore' & hɑɡɑi̯ & \armenian{հագայ} & hækʰʲæ & \armenian{հա̈քյա}̈ & hɑɡɑ & \armenian{հագա} \\ 
 		`to come' & ɡɑl & \armenian{գալ} & ikʰʲælʲ & \armenian{իքյա̈լՙ} & hɑɡɑ & ɡɑl \\ 
 		`godfather' & kənkʰɑhɑi̯ɾ & \armenian{կնքահայր} & inkʰʲævuɾ & \armenian{ինքյա̈վուր} & kəŋkʰɑhɑjɾ & \armenian{կնքահայր} \\ 
 		`pot' & putuk & \armenian{պուտուկ} & budʲuɡ & \armenian{բուդյուգ} & putuk & \armenian{պուտուկ} \\ 
 		\hline 
 	\end{tabular}
\end{table}

\paragraph{Glide /w/ <\armenian{ւ}> } 
The subdialect of Hazzo has created a new half-sound, which except for Maragha, is not found in other dialect. This is the English sound /w/, whose exact correspondent in Old Armenian is the form <\armenian{ւ}> /w/, just as how we transliterate. It is likewise found in words borrowed from foreign languages (Table \ref{tab:Tigranakert:phonology:inventory:cons:w}). 


\begin{table}[H]
 	\centering 
 	\caption{Glide /w/ <\armenian{ւ}> in the Tigranakert dialect}
 	\label{tab:Tigranakert:phonology:inventory:cons:w}
 	\begin{tabular}{|l| ll|ll| ll|}
 		\hline & \multicolumn{2}{l|}{Classical Armenian} &\multicolumn{2}{l|}{> Tigranakert} & \multicolumn{2}{l|}{cf. SEA} \\ 
 		 
 		`on' & i veɾɑi̯ & \armenian{ի վերայ} & wəɾen & \armenian{ւըրէն} & vəɾɑ & \armenian{վրա} \\ 
 		`that' & oɾ & \armenian{որ} & wəɾ & \armenian{ւըր} & voɾ & \armenian{որ} \\ 
 		\hline 
 		\hline & \multicolumn{2}{l|}{Arabic} &\multicolumn{4}{l|}{> Tigranakert} \\ 
 		`time' & <waqt> & \textarab{وقت} & wɑχt & \armenian{ւախտ} & & \\ 
 		\hline 
 	\end{tabular}
\end{table}


\subsection{Sound changes}


Among sound changes, it is worth mentioning the following.

\begin{adjarianpage}\label{page:161}\end{adjarianpage}% should be 161

\subsubsection{Monophthong vowel changes}
\paragraph{Classical Armenian /ɑ/ <\armenian{ա}>}
The Classical sound /ɑ/ <\armenian{ա}> has for the most part changed to /æ/ <\armenian{ա}̈>, such that the dialect is filled with this sound. For a person from Tigranakert, it is difficult to pronounce the sound /ɑ/ <\armenian{ա}>; that sound is preserved only next to the sound /r/ <\armenian{ռ}> and in few other circumstances (Table \ref{tab:Tigranakert:phonology:changes:vowel:a}). 


\begin{table}[H]
 	\centering 
 	\caption{Change from Classical Armenian /ɑ/ <\armenian{ա}> to usually /æ/ <\armenian{ա}̈> but sometimes /ɑ/ <\armenian{ա}> in the Tigranakert dialect}
 	\label{tab:Tigranakert:phonology:changes:vowel:a}
 	\begin{tabular}{|l| ll|ll| ll|}
 		\hline & \multicolumn{2}{l|}{Classical Armenian} &\multicolumn{2}{l|}{> Tigranakert} & \multicolumn{2}{l|}{cf. SEA} \\ 
 		 
 		`pompous' & ɑmbɑɾtɑ{wɑ}n & \armenian{ամբարտաւան} & æmpʰæɾdævæn & \armenian{ա̈մփա̈րդա̈վա̈ն} & ɑmbɑɾtɑvɑn & \armenian{ամբարտավան} \\ 
 		`cane' & ɡɑ{wɑ}zɑn & \armenian{գաւազան} & kʰævæzæn & \armenian{քա̈վա̈զա̈ն} & ɡɑvɑzɑn & \armenian{գավազան} \\ 
 		`deacon' & sɑɾkɑ{wɑ}ɡ & \armenian{սարկաւագ}& sæɾɡævækʰ & \armenian{սա̈րգա̈վա̈ք} & sɑɾkɑvɑɡ & \armenian{սարկավագ} \\ 
 		`water-mill' &d͡ʒəɾɑɬɑt͡sʰ& \armenian{ջրաղաց} & t͡ʃʰæʁɑɾt͡sʰ &\armenian{չա̈ղա̈րց} & d͡ʒəɾɑʁɑt͡sʰ& \armenian{ջրաղաց} \\ 
 		\hline 
 		`stall' &ɑχor & \armenian{ախոռ} & ɑχor &\armenian{ախօռ} &ɑχor & \armenian{ախոռ} \\ 
 		`granary' &ɑmbɑɾ? & \armenian{ամբար}? & ɑmbɑr &\armenian{ամբառ} &ɑmbɑɾ & \armenian{ամբար} \\ 
 		`male cat' & & & ɑrt͡ʃʰ &\armenian{առչ} & & \\ 
 		`censer' &buɾvɑr& \armenian{բուրվառ} & pʰulvɑr &\armenian{փուլվառ} & buɾvɑr& \armenian{բուրվառ} \\ 
 		`to life' & bɑrnɑl & \armenian{բառնալ} & pʰɑrnɑl &\armenian{փառնալ} & bɑrnɑl& \armenian{բառնալ} \\ 
 		\hline 
 	\end{tabular}
\end{table}




\paragraph{Classical Armenian /o/ <\armenian{ո}>}
The Classical sound /o/ <\armenian{ո}> has changed to /u/ <\armenian{ու}> (Table \ref{tab:Tigranakert:phonology:changes:vowel:o}a). But in case declension, we find as in Table \ref{tab:Tigranakert:phonology:changes:vowel:o}b. 




\begin{table}[H]
 	\centering 
 	\caption{Change from Classical Armenian /o/ <\armenian{ո}> to /u/ <\armenian{ու}> in the Tigranakert dialect}
 	\label{tab:Tigranakert:phonology:changes:vowel:o}
 	\begin{tabular}{|ll| ll|ll| ll|}
 		\hline && \multicolumn{2}{l|}{Classical Armenian} &\multicolumn{2}{l|}{> Tigranakert} & \multicolumn{2}{l|}{cf. SEA} \\ 
 		a. & 
 		`pompous' & ɑmbɑɾtɑ{wɑ}n & \armenian{ամբարտաւան} & æmpʰæɾdævæn & \armenian{ա̈մփա̈րդա̈վա̈ն} & ɑmbɑɾtɑvɑn & \armenian{ամբարտավան} \\ 
 		& \armenian{՝}new' & noɾ & \armenian{նոր}& nuɾ & \armenian{նուր} & noɾ & \armenian{նոր} \\ 
 		& \armenian{՝}belly' & pʰoɾ & \armenian{փոր}& pʰuɾ & \armenian{փուր} & pʰoɾ & \armenian{փոր} \\ 
 		&\armenian{՝}pit' & pʰos & \armenian{փոս}& pʰus & \armenian{փուս} & pʰos & \armenian{փոս} \\ 
 		& 	 `earth' & hoɬ & \armenian{հող} & χuʁ & \armenian{խուղ} & hoʁ & \armenian{հող} \\ 
 		& 	 `onion' & soχ & \armenian{սոխ} & suχ & \armenian{սուխ} & soχ & \armenian{սոխ} \\ 
 		& 	 `dry' & t͡ʃʰoɾ & \armenian{չոր} & t͡ʃʰuɾ & \armenian{չուր} & t͡ʃʰoɾ & \armenian{չոր} \\ 
 		& 	 \armenian{՝}four' & t͡ʃʰoɾs & \armenian{չորս}& t͡ʃʰuɾs & \armenian{չուրս} & t͡ʃʰoɾs & \armenian{չորս} \\ 
 		b. & \armenian{՝}belly-{\gen}' & pʰoɾ-i & \armenian{փորոյ}& pʰoɾ-i & \armenian{փօրի} & pʰoɾ-i & \armenian{փորի} \\ 
 		&\armenian{՝}pit-{\gen}' & pʰos-i & \armenian{փոսի}& pʰos-i & \armenian{փօսի} & pʰos & \armenian{փոսի} \\ 
 		& 	 `earth-{\gen}' & hoɬ-oi̯ & \armenian{հողոյ} & χoʁ-u & \armenian{խօղու} & hoʁ-i & \armenian{հողի} \\ 
 		 
 		\hline 
 	\end{tabular}
\end{table}



The same sound at the beginning of monosyllabic words becomes /vo, və/ <\armenian{վօ. վը}>; it becomes /o/ <\armenian{օ}> at the beginning of polysyllabic words (Table \ref{tab:Tigranakert:phonology:changes:vowel:o:other}.\footnote{\translatorHD{For the word `calf', Adjarian provides an Classical ancestor /hoɾtʰ/ <\armenian{հորթ}>. But the most prescriptive Classical form is /oɾtʰ/ <\armenian{որթ}>. I changed his example for accuracy. }}




\begin{table}[H]
 	\centering 
 	\caption{Change from Classical Armenian /o/ <\armenian{ո}> to /vo, və, o/ <\armenian{վօ. վը, օ}> in the Tigranakert dialect}
 	\label{tab:Tigranakert:phonology:changes:vowel:o:other}
 	\begin{tabular}{|l| ll|ll| ll|}
 		\hline & \multicolumn{2}{l|}{Classical Armenian} &\multicolumn{2}{l|}{> Tigranakert} & \multicolumn{2}{l|}{cf. SEA} \\ 
 		`pompous' & ɑmbɑɾtɑ{wɑ}n & \armenian{ամբարտաւան} & æmpʰæɾdævæn & \armenian{ա̈մփա̈րդա̈վա̈ն} & ɑmbɑɾtɑvɑn & \armenian{ամբարտավան} \\ 
 		`who' & ov & \armenian{ով} & vov & \armenian{վօվ} & ov & \armenian{ով} \\ 
 		`buttocks' & or & & vər & \armenian{վըռ} & vor & \armenian{ոռ} \\ 
 		`lentil' & ospən & \armenian{ոսպն} & vəsp & \armenian{վըսպ} & vosp & \armenian{ոսպ} \\ 
 		\armenian{՝}orphan' & oɾb & \armenian{որբ} & vəɾpʰ & \armenian{վըրփ} & voɾpʰ & \armenian{որբ} \\ 
 		`smell' & oɾtʰ & \armenian{որթ} & vəɾtʰ & \armenian{վըրթ} & hoɾt & \armenian{հորթ} \\ 
 		`smell' & ozni & \armenian{ոզնի} & ozniɡ & \armenian{օզնիգ} & vozni & \armenian{ոզնի} \\ 
 		`to twist' &oloɾel & \armenian{ոլորել} & olɾil & \armenian{օլրիլ} & voloɾel& \armenian{ոլորել} \\ 
 		`gold' & oski & \armenian{ոսկի} & ozɡi & \armenian{օզգի} & voski& \armenian{ոսկի} \\ 
 		`alive' & oɬd͡ʒ & \armenian{ողջ} & voχt͡ʃʰ & \armenian{վօխչ} & voχt͡ʃʰ& \armenian{ողջ} \\ 
 		`to be cured' & oɬd͡ʒɑnɑl & \armenian{ողջանալ} & oχt͡ʃʰənnæl & \armenian{օխչըննա̈լ} & voχt͡ʃʰɑnɑl& \armenian{ողջանալ} \\ 
 		`bone' & oskəɾ & \armenian{ոսկր} & oskʰur & \armenian{օսքուռ} & voskoɾ & \armenian{ոսկոր} \\ 
 		\hline 
 	\end{tabular}
\end{table}





\paragraph{Classical Armenian /u/ <\armenian{ու}>}
The Classical sound /u/ <\armenian{ու}> usually stays /u/ <\armenian{ու}> (Table \ref{tab:Tigranakert:phonology:changes:vowel:u}a). \footnote{\translatorHD{For the Tigranakert word /uχd/ <\armenian{ուխդ}>, it's unclear if this word is a reflex of the Classical word for `vow' /uχt/ <\armenian{ուխտ}> or for `camel' /uɬt/ <\armenian{ուղտ}> (SEA: /uχt/). }} But it becomes /o/ <\armenian{օ}> in the word in Table \ref{tab:Tigranakert:phonology:changes:vowel:u}b.




\begin{table}[H]
 	\centering 
 	\caption{Change from Classical Armenian /u/ <\armenian{ու}> to /u/ <\armenian{ու}> in the Tigranakert dialect}
 	\label{tab:Tigranakert:phonology:changes:vowel:u}
 	\begin{tabular}{| ll| ll|ll| ll|}
 		\hline & & \multicolumn{2}{l|}{Classical Armenian} &\multicolumn{2}{l|}{> Tigranakert} & \multicolumn{2}{l|}{cf. SEA} \\ 
 		a. & `dog' &ʃun & \armenian{շուն} & ʃun & \armenian{շուն} & ʃun & \armenian{շուն} \\ 
 		&	`deaf' &χul & \armenian{խուլ} & χul & \armenian{խուլ} & χul & \armenian{խուլ} \\ 
 		& `pomegranate' &nurən & \armenian{նուռն} & nur & \armenian{նուռ} & nur & \armenian{նուռ} \\ 
 		& `camel? vow?' & & & uχd & \armenian{ուխդ} & & \\ 
 		& 	`fawn' & ul & \armenian{ուլ} & ulʲ & \armenian{ուլՙ} & ul & \armenian{ուլ} \\ 
 		& 			`incense' & χunk & \armenian{խունկ} & χunɡ & \armenian{խունգ} & χuŋk & \armenian{խունկ} \\ 
 		b. & `door' & durən & \armenian{դուռն} & tʰor & \armenian{թօռ} & dur & \armenian{դուռ} \\ 
 		\hline 
 	\end{tabular}
\end{table}



\paragraph{Classical Armenian /e/ <\armenian{ե}>}
The Classical sound /e/ <\armenian{ե}> becomes /i/ <\armenian{ի}> in the final syllable (Table \ref{tab:Tigranakert:phonology:changes:vowel:e}a). But it becomes /e/ <\armenian{է}> during case declension (Table \ref{tab:Tigranakert:phonology:changes:vowel:e}b).





\begin{table}[H]
 	\centering 
 	\caption{Change from Classical Armenian /e/ <\armenian{ե}> to /i/ <\armenian{ի}> in the Tigranakert dialect}
 	\label{tab:Tigranakert:phonology:changes:vowel:e}
 	\begin{tabular}{| ll| ll|ll| ll|}
 		\hline & & \multicolumn{2}{l|}{Classical Armenian} &\multicolumn{2}{l|}{> Tigranakert} & \multicolumn{2}{l|}{cf. SEA} \\ 
 		a. & \armenian{՝}face' & eɾes & \armenian{երես} & eɾis & \armenian{էրիս}& jeɾes & \armenian{երես} \\ 
 		& 		\armenian{՝}place' & teɬ & \armenian{տեղ}& diʁ & \armenian{դիղ} & teʁ & \armenian{տեղ} \\ 	 
 		& 		\armenian{՝}medicine' & deɬ & \armenian{դեղ}& tʰiʁ & \armenian{թիղ} & deʁ & \armenian{դեղ} \\ 	 
 		& 	`sun' & ɑɾeu̯& \armenian{արեւ} & æɾiv & \armenian{ա̈րիվ} & ɑɾev & \armenian{արև} \\ 
 		& `needle' & ɑseɬən & \armenian{ասեղն} & æsiʁ & \armenian{ա̈սիղ} &ɑseʁ & \armenian{ասեղ} \\ 
 		b. & \armenian{՝}face-{\gen}' & eɾes-i & \armenian{երեսի} & eɾes-i & \armenian{էրէսի}& jeɾes-i & \armenian{երեսի} \\ 
 		& 		\armenian{՝}medicine-{\gen}' & deɬ-i & \armenian{դեղի}& tʰeʁ-i & \armenian{թէղի} & deʁ-i & \armenian{դեղի} \\ 	 
 		\hline 
 	\end{tabular}
\end{table}



At the beginning of words, this sound becomes /je/ <\armenian{յէ}> for monosyllables, while /e/ <\armenian{է}> for polysyllables (Table \ref{tab:Tigranakert:phonology:changes:vowel:esize}). 



\begin{table}[H]
 	\centering 
 	\caption{Change from Classical Armenian /e/ <\armenian{ե}> to /je, e/ <\armenian{յէ, է}> in the Tigranakert dialect}
 	\label{tab:Tigranakert:phonology:changes:vowel:esize}
 	\begin{tabular}{| l | ll|ll| ll|}
 		\hline & \multicolumn{2}{l|}{Classical Armenian} &\multicolumn{2}{l|}{> Tigranakert} & \multicolumn{2}{l|}{cf. SEA} \\ 
 		`ox' &ezən & \armenian{եզն} & jez & \armenian{յէզ} &jez & \armenian{եզ} \\ 
 		`I' &es & \armenian{ես} & jes & \armenian{յէս} &jes & \armenian{ես} \\ 
 		\armenian{՝}when' & eɾb & \armenian{երբ} & jepʰ & \armenian{յէփ} & jeɾpʰ & \armenian{երբ} \\ 
 		\armenian{՝}yesterday' & eɾēk & \armenian{երէկ} & eɾeɡ & \armenian{էրէգ} & jeɾek & \armenian{երեկ} \\ 
 		\armenian{՝}iron' & eɾkɑtʰ & \armenian{երկաթ} & eɾɡætʰ & \armenian{էրգա̈թ} & jeɾkɑtʰ & \armenian{երկաթ} \\ 
 		\hline 
 	\end{tabular}
\end{table}

\subsubsection{Diphthong changes}
\paragraph{Classical Armenian /ɑi̯/ <\armenian{այ}>}

The Classical diphthong /ɑi̯/ <\armenian{այ}> becomes /e/ <\armenian{է}> (Table \ref{tab:Tigranakert:phonology:changes:vowel:aj}). 



\begin{table}[H]
 	\centering 
 	\caption{Change from Classical Armenian /ɑi̯/ <\armenian{այ}> to /e/ <\armenian{է}> in the Tigranakert dialect}
 	\label{tab:Tigranakert:phonology:changes:vowel:aj}
 	\begin{tabular}{| l | ll|ll| ll|}
 		\hline & \multicolumn{2}{l|}{Classical Armenian} &\multicolumn{2}{l|}{> Tigranakert} & \multicolumn{2}{l|}{cf. SEA} \\ 
 		distal `that yonder' & ɑi̯n &\armenian{այն}& en& \armenian{էն}& ɑjn& \armenian{այն} \\ 
 		proximal `this' & ɑi̯s &\armenian{այս}& es& \armenian{էս}& ɑjs& \armenian{այս} \\ 
 		`wood' & pʰɑi̯t & \armenian{փայտ} & pʰed & \armenian{փէդ} &pʰɑjt & \armenian{փայտ} \\ 
 		`vineyard' &ɑi̯ɡi& \armenian{այգի} & ekʰi & \armenian{էքի} &ɑjɡi& \armenian{այգի} \\ 
 		`to burn' & ɑi̯ɾel & \armenian{այրել} & eɾvil & \armenian{էրվիլ} & ɑjɾel & \armenian{այրել} \\ 
 		`lightning' &kɑi̯t͡sɑkən & \armenian{կայծակն} & ɡed͡zækʰ & \armenian{գէձա̈ք} & kɑjt͡sɑk & \armenian{կայծակ} \\ 
 		\hline 
 	\end{tabular}
\end{table}

\paragraph{Classical Armenian /iu̯/ <\armenian{իւ}>}

The Classical diphthong /iu̯/ <\armenian{իւ}> becomes /i,u/ <\armenian{ի, ու}> (Table \ref{tab:Tigranakert:phonology:changes:vowel:iu}). 



\begin{table}[H]
 	\centering 
 	\caption{Change from Classical Armenian /iu̯/ <\armenian{իւ}> to /i,u/ <\armenian{ի, ու}> in the Tigranakert dialect}
 	\label{tab:Tigranakert:phonology:changes:vowel:iu}
 	\begin{tabular}{| l | ll|ll| ll|}
 		\hline & \multicolumn{2}{l|}{Classical Armenian} &\multicolumn{2}{l|}{> Tigranakert} & \multicolumn{2}{l|}{cf. SEA} \\ 
 		\armenian{՝}blood' & ɑɾiu̯n & \armenian{արիւն}& æɾin & \armenian{ա̈րին} & ɑɾjun & \armenian{արյուն} \\ 
 		\armenian{՝}hundred' & hɑɾiu̯ɾ & \armenian{հարիւր}& hæɾiɾ & \armenian{հա̈րիր} & hɑɾjuɾ & \armenian{հարյուր} \\ 
 		`flour' & ɑliu̯ɾ & \armenian{ալիւր} &æliɾ & \armenian{ա̈լիր} & ɑljuɾ & \armenian{ալյուր} \\ 
 		`column' & siu̯n & \armenian{սիւն} & sun & \armenian{սուն} & sjun & \armenian{սյուն} \\ 
 		\armenian{՝}snow' & d͡ziu̯n & \armenian{ձիւն}& t͡sʰun & \armenian{ցուն} & d͡zjun & \armenian{ձյուն} \\ 
 		\hline 
 	\end{tabular}
\end{table}


\paragraph{Classical Armenian /oi̯/ <\armenian{ոյ}>}

The Classical diphthong /oi̯/ <\armenian{ոյ}> becomes /u/ <\armenian{ու}> (Table \ref{tab:Tigranakert:phonology:changes:vowel:oi}). 




\begin{table}[H]
 	\centering 
 	\caption{Change from Classical Armenian /oi̯/ <\armenian{ոյ}> to /u/ <\armenian{ու}> in the Tigranakert dialect}
 	\label{tab:Tigranakert:phonology:changes:vowel:oi}
 	\begin{tabular}{| l | ll|ll| ll|}
 		\hline & \multicolumn{2}{l|}{Classical Armenian} &\multicolumn{2}{l|}{> Tigranakert} & \multicolumn{2}{l|}{cf. SEA} \\ 
 		`light' & loi̯s & \armenian{լոյս} & lus & \armenian{լուս} & lujs & \armenian{լույս} \\ 
 		`Pleiades' & boi̯lkʰ& \armenian{բոյլք} & pʰulʲkʰ & \armenian{փուլՙք} & bujlkʰ & \armenian{բույլք}\\ 
 		`nest' & boi̯n & \armenian{բոյն} &pʰun & \armenian{փուն} & bujn & \armenian{բույն} \\ 
 		\hline 
 	\end{tabular}
\end{table}

\subsubsection{Consonant changes}

\paragraph{Classical Armenian /h/ <\armenian{հ}>}

The Classical consonant /h/ <\armenian{հ}> generally stays /h/ <\armenian{հ}>, just as in the Mush dialect; but it changes to /χ/ <\armenian{խ}> in the words in Table \ref{tab:Tigranakert:phonology:changes:cons:h}. 



\begin{table}[H]
 	\centering 
 	\caption{Change from Classical Armenian /h/ <\armenian{հ}> to /χ/ <\armenian{խ}> in the Tigranakert dialect}
 	\label{tab:Tigranakert:phonology:changes:cons:h}
 	\begin{tabular}{| l | ll|ll| ll|}
 		\hline & \multicolumn{2}{l|}{Classical Armenian} &\multicolumn{2}{l|}{> Tigranakert} & \multicolumn{2}{l|}{cf. SEA} \\ 
 		`earth' & hoɬ & \armenian{հող} & χuʁ & \armenian{խուղ} & hoʁ & \armenian{հող} \\ 
 		`far' & heroi̯ & \armenian{հեռոյ} & χoru & \armenian{խօռու} & heru & \armenian{հեռու} \\ 
 		`time' & heɬ & \armenian{հեղ} & χiʁ & \armenian{խիղ} & heʁ & \armenian{հեղ} \\ 
 		\hline 
 	\end{tabular}
\end{table}

\paragraph{Word-medial gemination}

In many places, it is notable that word-medial consonants are repeated (Table \ref{tab:Tigranakert:phonology:changes:cons:gem}). 



\begin{table}[H]
 	\centering 
 	\caption{Gemination from Classical Armenian to the Tigranakert dialect}
 	\label{tab:Tigranakert:phonology:changes:cons:gem}
 	\begin{tabular}{| l | ll|ll| ll|}
 		\hline & \multicolumn{2}{l|}{Classical Armenian} &\multicolumn{2}{l|}{> Tigranakert} & \multicolumn{2}{l|}{cf. SEA} \\ 
 		`cheap' & ɑɾʒɑn & \armenian{արժան}& eʒʒæn & \armenian{էժժա̈ն}& ɑɾʒɑn & \armenian{արժան} \\ 
 		`tail of sheep' & dəmɑk & \armenian{դմակ}& tʰəmmæɡ & \armenian{թըմմա̈գ}& dəmɑk & \armenian{դմակ} \\ 
 		`seven' & e̯\'ɑu̯tʰən & \armenian{եաւթն}& j\'otʰtʰe & \armenian{յօ՛թթէ}& j\'otʰə & \armenian{յոթը} \\ 
 		`to cleanse' & zətel & \armenian{զտել}& zəddel & \armenian{զըդդէլ}& zətel & \armenian{զտել} \\ 
 		`sour' &tʰətʰu & \armenian{թթու} & tʰotʰtʰu & \armenian{թօթթու} & tʰətʰu & \armenian{թթու} \\ 
 		`manure' & tʰəɾikʰ & \armenian{թրիք} & tʰəɾɾikʰ & \armenian{թըրրիք} & tʰəɾikʰ& \armenian{թրիք} \\ 
 		`nine' & \'inən & \armenian{ինն} & \'innə & \armenian{ի՛ննէ} &\'inə & \armenian{ինը} \\ 
 		`to hear' & ləsel & \armenian{լսել} & ləssel & \armenian{լըսսէլ} & ləsel& \armenian{լսել} \\ 
 		`to smoke' & t͡səχel & \armenian{ծխել} & d͡zəχχæl & \armenian{ձըխխա̈լ} & t͡səχel& \armenian{ծխել} \\ 
 		`to suck' & t͡sət͡sel & \armenian{ծծել} & d͡zəd͡zd͡zi̯el & \armenian{ձըձձել} & t͡sət͡sel& \armenian{ծծել} \\ 
 		`early' & kɑnuχ & \armenian{կանուխ} & ɡənnuχ & \armenian{գըննուխ} & kɑnuχ& \armenian{կանուխ} \\ 
 		`pungent' & kət͡su & \armenian{կծու} & ɡəd͡zd͡zu & \armenian{գըձձու} & kət͡su& \armenian{կծու} \\ 
 		`soul' & hoɡi & \armenian{հոգի} & hokʰkʰi & \armenian{հօքքի} & hokʰi & \armenian{հոգի} \\ 
 		`farmer' & məʃɑk & \armenian{մշակ} & mʃʃæɡ & \armenian{մշշա̈գ} & məʃɑk & \armenian{մշակ} \\ 
 		\hline 
 	\end{tabular}
\end{table} 


Sometimes, the simple form of the word uses one consonant, but the consonant is repeated during case declension. (Table \ref{tab:Tigranakert:phonology:changes:cons:gemderive}).\footnote{\translatorHD{Adjarian doesn't provide a translation or ancestor for the word /vit͡sʰ/ <\armenian{վից}>; I speculate that this word is derived from the Classical word for `six'. }}



\begin{table}[H]
 	\centering 
 	\caption{Gemination in derived forms in the Tigranakert dialect}
 	\label{tab:Tigranakert:phonology:changes:cons:gemderive}
 	\begin{tabular}{| l | ll|ll| ll|}
 		\hline & \multicolumn{2}{l|}{Classical Armenian} &\multicolumn{2}{l|}{> Tigranakert} & \multicolumn{2}{l|}{cf. SEA} \\ 
 		`bread' & hɑt͡sʰ & \armenian{հաց} & hɑt͡sʰ & \armenian{հաց} & hɑt͡sʰ & \armenian{հաց} \\ 
 		`bread-{\pl}' & & & hɑt͡sʰt͡sʰ-iɾ & \armenian{հացցիր} & hɑt͡sʰ-eɾ & \armenian{հացեր} \\ 
 		`six?' & vet͡sʰ & \armenian{վեց} & vit͡sʰ & \armenian{վից} & vet͡sʰ & \armenian{վեց} \\ 
 		`six?-{\gen}' & vet͡sʰ-i & \armenian{վեցի} & vit͡sʰt͡sʰ-i & \armenian{վիցցի} & vet͡sʰ-i & \armenian{վեցի} \\ 
 		\hline 
 	\end{tabular}
\end{table} 


\section{Morphology}
\subsection{Noun inflection or declension}
\subsubsection{Definite article /-e/ <\armenian{է}> }

In the grammar, it is notable that the article /-ə/ <\armenian{ը}> of Civil Armenian has the form /e/ <\armenian{է}> here, and it is of course unstressed (Table \ref{tab:Tigranakert:morpho:noun:def}). 



\begin{table}[H]
 	\centering 
 	\caption{Gemination from Classical Armenian to the Tigranakert dialect}
 	\label{tab:Tigranakert:morpho:noun:def}
 	\begin{tabular}{| l | ll| ll|}
 		\hline & \multicolumn{2}{l|}{Tigranakert} & \multicolumn{2}{l|}{cf. SEA} \\ 
 		`mouth-{\defgloss}' & pʰeɾ\'æn-e & \armenian{փէրա̈՛նէ} &beɾ\'ɑn-ə & \armenian{բերանը} \\ 
 		`dog' & ʃ\'un-e & \armenian{շո՛ւնէ} & ʃ\'un-ə & \armenian{շունը} \\ 
 		`column' & s\'un-e & \armenian{սո՛ւնէ} & sj\'un-ə & \armenian{սյունը} \\ 
 		\hline 
 	\end{tabular}
\end{table} 



\subsubsection{Accusative prefix /z-/ <\armenian{զ}> }
The accusative case... 

\begin{adjarianpage}\label{page:162}\end{adjarianpage}% should be 162

... is formed with the \armenian{զ} /z/ prefix, as in the Mush dialect, or without the prefix. 

\subsubsection{Ablative suffix /-e, -ut͡sʰ/ <\armenian{է, ուց}> }
The ablative is the formative /-e/ <\armenian{է}> , but infinitives take the formative /-ut͡sʰ/ <\armenian{ուց}> (Table \ref{tab:Tigranakert:morpho:noun:abl}). 



\begin{table}[H]
 	\centering 
 	\caption{Ablatives in the Tigranakert dialect}
 	\label{tab:Tigranakert:morpho:noun:abl}
 	\begin{tabular}{| l | ll| ll|}
 		\hline & \multicolumn{2}{l|}{Tigranakert} & \multicolumn{2}{l|}{cf. SEA} \\ 
 		`to.love-{\abl}' & siɾel-ut͡sʰ & \armenian{սիրէլուց} &siɾel-ut͡sʰ & \armenian{սիրելուց} \\ 
 		`to.speak-{\abl}' & χosel-ut͡sʰ & \armenian{խօսէլուց} &χosel-ut͡sʰ & \armenian{խոսելուց} \\ 
 		\hline 
 	\end{tabular}
\end{table} 

\subsubsection{Plural markers /-iɾ, -niɾ, -ni/ <\armenian{իր, նիր, նի}> }


The plural marker is /iɾ, -niɾ, -ni/ <\armenian{իր, նիր, նի}> (Table \ref{tab:Tigranakert:morpho:noun:pl}). 



\begin{table}[H]
 	\centering 
 	\caption{Plurals in the Tigranakert dialect}
 	\label{tab:Tigranakert:morpho:noun:pl}
 	\begin{tabular}{| l | ll| ll|}
 		\hline & \multicolumn{2}{l|}{Tigranakert} & \multicolumn{2}{l|}{cf. SEA} \\ 
 		`bread-{\pl}' & hæt͡sʰt͡sʰ-iɾ & \armenian{հա̈ցցիր} &hɑt͡sʰ-eɾ & \armenian{հացեր} \\ 
 		`angel-{\pl}' & hɾəʃdæɡ-ni & \armenian{հրէշդա̈գնի} &həɾəʃtɑk-neɾ & \armenian{հրեշտակներ} \\ 
 		\hline 
 	\end{tabular}
\end{table} 

\subsection{Pronoun inflection or declension}

For pronouns, there are some noteworthy points. The first among them is /jesi/ <\armenian{յէսի}>, the accusative form of the 1SG pronoun <\armenian{ես}> (\translatorHD{CA: /es/, SEA: /jes/}). The second is the absence of the medial demonstrative <\armenian{այդ}> (\translatorHD{CA: /ɑi̯d/, SEA: /ɑjd/}). The Tigranakert dialect distinguishes only two demonstratives: proximial `this' <\armenian{այս}> and distal `that' <\armenian{այն}> (\translatorHD{CA: /ɑi̯s, ɑi̯n/, SEA: /ɑjs, ɑjn/}). While the <\armenian{այդ}> is explained with the forms <\armenian{այս}> or <\armenian{այն}>. 

These are declined as follows.

\translatorHD{Table \ref{tab:Tigranakert:morpho:pronoun:personal} is for personal pronouns. }

\begin{table}[H]
 	\caption{Inflection paradigm for personal pronouns in the Tigranakert dialect }\label{tab:Tigranakert:morpho:pronoun:personal}
 	\centering 
 	\begin{tabular}{|l|ll|ll|}
 		\hline & 1SG & 2SG & 1PL & 2PL \\ 
 		& `I' & `you' & `we'& `you' \\\hline 
 		{\nom} & jes & tʰun & minkʰ & tʰukʰ \\ 
 		& \armenian{յէս} & \armenian{թուն} & \armenian{մինք} & \armenian{թուք} \\ 
 		{\gen} & im & kʰu & miɾ & t͡sʰeɾ \\ 
 		& \armenian{իմ} & \armenian{քու} & \armenian{միր} & \armenian{ցէր} \\ 
 		{\dat} & ənd͡zi & kʰez(i) & mez(i) & t͡sʰez(i) \\ 
 		& \armenian{ընձի} & \armenian{քէզ(ի}) & \armenian{մէզ(ի}) & \armenian{ցէզ(ի}) \\ 
 		{\acc} & jesi & kʰezi, zkʰi & mezi, zmi & t͡sʰez(i) \\ 
 		& \armenian{յէսի} & \armenian{քէզի, զքի} & \armenian{մէզի, զմի} & \armenian{ցէզ(ի}) \\ 
 		{\abl} & ənd͡zme & kʰezme & mezme & t͡sʰezme \\ 
 		& \armenian{ընձմէ} & \armenian{քէզմէ} & \armenian{մէզմէ} & \armenian{ցէզմէ} \\ 
 		{\ins} & ənd͡zmov & kʰezmov & mezmov & t͡sʰezmov \\ 
 		& \armenian{ընձմօվ} & \armenian{քէզմօվ} & \armenian{մէզմօվ} & \armenian{ցէզմօվ} 
 		\\ \hline 
 	\end{tabular}
\end{table}

\translatorHD{Table \ref{tab:Tigranakert:morpho:pronoun:dem} is for demonstrative pronouns. }

\begin{table}[H]
 	\caption{Inflection paradigm for demonstrative pronouns in the Tigranakert dialect }\label{tab:Tigranakert:morpho:pronoun:dem}
 	\centering \begin{tabular}{|l|ll|ll|}
 		\hline & \multicolumn{2}{c|}{Singular}& \multicolumn{2}{c|}{Plural}
 		\\ 
 		& proximal & distal & proximal & distal \\ 
 		& `this' & `that' & `these' & `those' \\ \hline 
 		{\nom} & æs, əsi, əsiɡi & æn, əni, əniɡi & əsunkʰ & ənunkʰ \\ 
 		& \armenian{ա̈ս, ըսի, ըսիգի} & \armenian{ա̈ն, ընի, ընիգի} & \armenian{ըսունք} & \armenian{ընունք} \\ 
 		{\gen} & əsuɾ & ənuɾ & əsunkʰ & ənunt͡sʰ \\ 
 		& \armenian{ըսուր} & \armenian{ընուր} & \armenian{ըսունց} & \armenian{ընունց} \\ 
 		{\dat} & əsuɾ & ənuɾ & əsunt͡sʰ & ənunt͡sʰ \\ 
 		& \armenian{ըսուր} & \armenian{ընուր} & \armenian{ըսունց} & \armenian{ընունց} \\ 
 		{\acc} & əsuɾ, əsi, əsiɡi & ənuɾ, əni, əniɡi & əsunt͡sʰ & ənunt͡sʰ \\ 
 		& \armenian{ըսուր, ըսի, ըսիգի} & \armenian{ընուր, ընի, ընիգի} & \armenian{ըսունց} & \armenian{ընունց} \\ 
 		{\abl} & əsuɾme, əsuɾmene & ənuɾme, ənuɾmene & əsunt͡sʰme & ənunt͡sʰme \\ 
 		& \armenian{ըսուրմէ, ըսուրմէնէ} & \armenian{ընուրմէ, ընուրմէնէ} & \armenian{ըսունցմէ} & \armenian{ընունցմէ} \\ 
 		{\ins} & əsuɾmov & ənuɾmov & əsunt͡sʰmov & ənunt͡sʰmov \\ 
 		& \armenian{ըսուրմօվ} & \armenian{ընուրմօվ} & \armenian{ըսունցմօվ} & \armenian{ընունցմօվ} 
 		\\ \hline 
 	\end{tabular}
\end{table}

\translatorHD{Table \ref{tab:Tigranakert:morpho:pronoun:who} is for the interrogative pronoun `who'. } 

\begin{table}[H]
 	\centering 
 	\caption{Paradigm of the interrogative pronoun `who' in the Tigranakert dialect}
 	\label{tab:Tigranakert:morpho:pronoun:who}
 	\begin{tabular}{|l| l l|}
 		\hline & Singular& Plural \\ 
 		{\nom} & vov & voviɾ \\ 
 		& \armenian{վօվ} & \armenian{վօվիր} \\ 
 		{\gen}-{\dat} & voɾu & voɾeɾun \\ 
 		& \armenian{վօրու} & \armenian{վօրէրուն} \\ 
 		{\abl} & vorme, voɾmene & voɾont͡sʰmene \\ 
 		& \armenian{վօրմէ, վօրմէնէ} & \armenian{վօրօնցմէնէ} \\ 
 		\hline 
 	\end{tabular}
\end{table}


\subsection{Noun inflection or declension (continued)}

\subsubsection{Possessive articles and the extra suffix /-i/ <\armenian{ի}>}

The possessive articles are directly attached to nouns in the Armenian language; here, just as sometimes in the Mush dialect, they receive the unstressed ending /-i/ <\armenian{ի}> (Table \ref{tab:Tigranakert:morpho:noun:poss}). 



\begin{table}[H]
 	\centering 
 	\caption{Possessives in the Tigranakert dialect}
 	\label{tab:Tigranakert:morpho:noun:poss}
 	\begin{tabular}{| l | ll| ll|}
 		\hline & \multicolumn{2}{l|}{Tigranakert} & \multicolumn{2}{l|}{cf. SEA} \\ 
 		`mouth-{\possFsg}' & pʰeɾ\'æn-si & \armenian{փէրա̈՛նսի} &beɾ\'ɑn-əs & \armenian{բերանս} \\ 
 		`head-{\possFsg}' & kʰlʲ\'uχ-si & \armenian{քլՙո՛ւխսի} &ɡəl\'uχ-əs & \armenian{գլուխս} \\ 
 		`face-{\possSsg}' & eɾ\'is-tʰi & \armenian{էրի՛սթի} &jeɾ\'es-ət & \armenian{երեսդ} \\ 
 		`heart-{\possFsg}' & s\'iɾd-is & \armenian{սի՛րդիս} &s\'iɾt-əs & \armenian{սիրտս} \\ 	 
 		`neck-{\possSsg}' & v\'iz-itʰ & \armenian{վի՛զիթ} &v\'iz-ət & \armenian{վիզդ} \\ 
 		`heart-{\gen}-{\possFsg}' & sɾd-\'i-si & \armenian{սրդի՛սի} &səɾt-\'i-s & \armenian{սրտիս} \\ 	 
 		`heart-{\pl}-{\plposs}-{\gen}-{\possFsg}' & sɾd-eɾ-n-\'u-si & \armenian{սրդէրնո՛ւսի} &səɾd-eɾ-n-\'u-s & \armenian{սրտերնուս} (SWA) \\ 	 
 		`soul-{\gen}-{\possSsg}' & hokʰkʰ-\'u-tʰi & \armenian{հօքքո՛ւթի} & hokʰ-u-t & \armenian{հոգուդ} \\ 
 		`sin.{\gen}-{\possSsg}' & meʁ\'ɑt͡sʰ-is & \armenian{մէղա՛ցիս} & & \\ 
 		`heart-{\ins}-{\possFsg}' & sɾd-\'ov-si & \armenian{սրդօ՛վսի} &səɾt-\'ov-əs & \armenian{սրտովս} \\ 	 
 		`face-{\ins}-{\possSsg}' & eɾes-\'ov-tʰi & \armenian{էրէսօ՛վթի} &jeɾes-\'ov-ət & \armenian{երեսովդ} \\ 
 		\hline 
 	\end{tabular}
\end{table} 


The addition of unstressed /i/ <\armenian{ի}> has combined with the article \armenian{է} /e/, giving the dialect a soft Italian harmony,... 




\begin{adjarianpage}\label{page:163}\end{adjarianpage}% should be 163

... especially when they are sequentially placed after another in a row (\ref{sent:Tigranakert:morpho:noun:i}).

\begin{exe}
 	\ex Tigranakert dialect\label{sent:Tigranakert:morpho:noun:i}
 	\begin{xlist}
 		\ex \gll sɾd-\'i-si s\'un-e \\ 
 		heart-{\gen}-{\possFsg} column-{\defgloss} \\ 
 		\trans `the column of my heart' \\ 
 		\armenian{սրդի՛սի սո՛ւնէ} 
 		\ex \gll hokʰkʰ-\'u-si d\'un-e \\ 
 		soul-{\gen}-{\possFsg} house-{\defgloss} \\ 
 		\trans `the house of my soul' \\ 
 		\armenian{հօքքո՛ւսի դո՛ւնէ} 
 	\end{xlist}
\end{exe}


\subsection{Verb inflection or conjugation}

\subsubsection{Theme vowel changes}
In verbs, the Classical present vowel /e/ <\armenian{ե}> becomes /i/ <\armenian{ի}> when next to nasals and /s/ <\armenian{ս}> (Table \ref{tab:Tigranakert:morpho:verb:paradigm:presentIndc}). 


\begin{table}[H]
 	\centering 
 	\caption{Indicative present <\armenian{ներկայ}> of the verb `to want' in the Tigranakert dialect}
 	\label{tab:Tigranakert:morpho:verb:paradigm:presentIndc}
 	\begin{tabular}{|l|ll|ll|}
 		\hline & \multicolumn{2}{l|}{Tigranakert} & \multicolumn{2}{l|}{cf. SWA} \\ 
 		1SG & ɡ-uz-i-m & \armenian{գուզիմ} & ɡ-uz-e-m &\armenian{կ՚ուզեմ} \\ 
 		2SG & ɡ-uz-i-s & \armenian{գուզիս} & ɡ-uz-e-s &\armenian{կ՚ուզես} \\ 
 		3SG & ɡ-uz-e-$\emptyset$ & \armenian{գուզէ} & ɡ-uz-e-$\emptyset$ &\armenian{կ՚ուզէ} \\ 
 		1PL & ɡ-uz-i-nkʰ & \armenian{գուզինք} & ɡ-uz-e-ŋkʰ &\armenian{կ՚ուզենք} \\ 
 		2PL & ɡ-uz-e-kʰ & \armenian{գուզէք} & ɡ-uz-e-m &\armenian{կ՚ուզէք} \\ 
 		3PL& ɡ-uz-i-n & \armenian{գուզին} & ɡ-uz-e-n &\armenian{կ՚ուզեն} \\ 
& \multicolumn{2}{l|}{{\ind}-$\sqrt{}$-{\thgloss}-{\agr}} & \multicolumn{2}{l|}{{\ind}-$\sqrt{}$-{\thgloss}-{\agr}} \\
 		\hline 
 	\end{tabular}
\end{table}

The imperfective loses its vowel /i/ <\armenian{ի}>, but it also stays unchanged (Table \ref{tab:Tigranakert:morpho:verb:paradigm:pastImpfIndc}). \translatorHD{The data is too limited to know if the surface vowel /e/ is synchronically the theme vowel or the past suffix or both. }

 


\begin{table}[H]
 	\centering 
 	\caption{Indicative past imperfective <\armenian{անկատար}> of the verb `to want' in the Tigranakert dialect}
 	\label{tab:Tigranakert:morpho:verb:paradigm:pastImpfIndc}
 	\begin{tabular}{|l|ll|ll|}
 		\hline & \multicolumn{2}{l|}{Tigranakert} & \multicolumn{2}{l|}{cf. SWA} \\ 
 		1PL & ɡ-uz-e-$\emptyset$-nkʰ & \armenian{գուզէնք} & ɡ-uz-ej-i-ŋkʰ &\armenian{կ՚ուզէինք} \\ 
 		& \multicolumn{2}{l|}{{\ind}-$\sqrt{}$-{\thgloss}-{\pst}-{\agr}} & \multicolumn{2}{l|}{{\ind}-$\sqrt{}$-{\thgloss}-{\pst}-{\agr}} \\ 
 		\hline 
 	\end{tabular}
\end{table}

\subsubsection{Monosyllabic verbs} 

In the monosyllabic verbs, the formative \armenian{ի} /i/ is added (Table \ref{tab:Tigranakert:morpho:verb:paradigm:mono}). But this added sound likewise stays when the verb is conjugated or declined

\begin{table}[H]
 	\centering 
 	\caption{Paradigm of monosyllabic verbs past in the Tigranakert dialect}
 	\label{tab:Tigranakert:morpho:verb:paradigm:mono}
 	\begin{tabular}{|l|ll|ll|l|}
 		\hline & \multicolumn{2}{l|}{Tigranakert} & \multicolumn{2}{l|}{cf. Classical Armenian} & \\ 
 		`to cry' & il-æ-l & \armenian{իլա̈լ} & l-ɑ-l &\armenian{լալ} & $\sqrt{}$-{\thgloss}-{\infgloss}\\ 
 		`to come' & ikʰʲ-æ-l & \armenian{իքյա̈լ} & ɡ-ɑ-l &\armenian{գալ}& \\ 
`to give' & id-æ-l & \armenian{իդա̈լ} & t-ɑ-l &\armenian{տալ} &\\
`to exist' & iɡ-æ-l & \armenian{իգա̈լ} &k-ɑ-l &\armenian{կալ} &\\

 		\hline & \multicolumn{2}{l|}{Tigranakert} & \multicolumn{2}{l|}{cf. SWA } & \\ 
 		`to cry ({\ins})' & il-æ-l-ov & \armenian{իլա̈լօվ} & l-ɑ-l-ov &\armenian{լալով} & $\sqrt{}$-{\thgloss}-{\infgloss}-{\ins}\\ 
`I cried' & il-æ-t͡sʰ-i-$\emptyset$ & \armenian{իլա̈ցի} & l-ɑ-t͡sʰ-i-$\emptyset$ &\armenian{լացի}& $\sqrt{}$-{\thgloss}-{\aor}-{\pst}-1{\sg} \\
`there exists' & iɡ-æ-$\emptyset$ & \armenian{իգա}̈ &k-ɑ-l &\armenian{կայ} & $\sqrt{}$-{\thgloss}-3{\sg}\\
\hline 
 	\end{tabular}
\end{table}

\subsubsection{Future formative /mən/ <\armenian{մըն}> } 

The formation of the future is surprising, because it uses the unfamiliar formative /mən/ <\armenian{մըն}> (\ref{sent:Tigranakert:morpho:verb:future}). 

\begin{exe}
 	\ex Tigranakert dialect \label{sent:Tigranakert:morpho:verb:future}
 	\begin{xlist}
 		\ex \gll mən uz-i-m \\ 
 		{\fut} want-{\thgloss}-1{\sg} \\ 
 		\trans `I will want.' \\ 
 		\armenian{մըն ուզիմ} 
 		\ex \gll mən pʰeɾ-i-m \\ 
 		{\fut} bring-{\thgloss}-1{\sg} \\ 
 		\trans `I will bring.' \\ 
 		 \armenian{մըն փէրիմ} 
 		 \ex \gll mən uz-e-i-$\emptyset$ \\ 
 		 {\fut} want-{\thgloss}-{\pst}-1{\sg} \\ 
 		 \trans `I was going to want.' \\ 
 		 \armenian{մըն ուզէի} 
 		 \ex \gll t͡ʃʰ\'ə-mən uz-i-m \\ 
 		 {\neg}-{\fut} want-{\thgloss}-1{\sg} \\ 
 		 \trans `I will not want.' \\ 
 		 \armenian{չը՛մըն ուզիմ} 
 		 \ex \gll t͡ʃʰ\'ə-mən uz-e-i-$\emptyset$ \\ 
 		 {\neg}-{\fut} want-{\thgloss}-{\pst}-1{\sg} \\ 
 		 \trans `I was not going to want.' \\ 
 		 \armenian{չը՛մըն ուզէի} 
 	\end{xlist}
\end{exe}

\subsubsection{Conjunction `also'} 

The Classical conjunction /ɑi̯l/ <\armenian{այլ}> `also' has the form /le/ <\armenian{լէ}>, like the Mush dialect. But the forms /æl, lə/ <\armenian{ա̈լ, լը}> are also used. 



\section{Text samples}

{\sampleoverview}

\subsection{Tigranakert}
 Adjarian's source: \citeauthor{Byurakn} 1898, page 470, 654, 700, and 1899, page 545. 

\begin{enumerate}
 	\item \armenian{Գլո՛խսի դնեմ բարձին}, 

 	\armenian{Հոգիս իտամ Աստըծուն}. 

 	\armenian{Բարի՛ հըրըշտակ, դուն պըհէ}, 

 	\armenian{Չար սատանան չխարէ։} 
 \item 	\armenian{Էվան կթթեց, Մարիամ մակրդեց}, 

 	\armenian{Քրիստոս էկավ խաչակնքեց}, 

 	\armenian{Կաթն էղավ մակարդ, մակարդ լէ կաթ։} 
 	 
 	 
 	\item \armenian{Ըմմէն մարթ կի տսնա, Աստված չտսնա (երազ)։} 
 	\item \armenian{Սպըտակ չադըր, դօռ չունի (հաւկիթ)։} 
 	\item \armenian{Բուրմա մը խուտ ՝ ցած տէնե՛րէ դրուկ է (յօնք)։} 
 	\item \armenian{Ցորեն չըմ կերի, արտի քովէն անցիր ըմ։} 
 	\item \armenian{Քարիր փէտիր չհիյան (չտեսնեն)։} 
 	\item \armenian{Զքու ցա՞նկն իմ կոտրի, քու էգի՞ն իմ մտի։} 
 	\item \armenian{Անկուշտն է պատի զքի։} 
 	\item \armenian{Դուն Չմնաս ՝ տնունդի մնայ։} 
 	\item \armenian{Սիվ իգա քզի։} 
 	 
 	\begin{adjarianpage}\label{page:164}\end{adjarianpage}% should be 164 
 	 
 	 
 	\item \armenian{Ա՛չքիդ էլլա, լո՛ւսիդ փճի։} 
 	\item \armenian{Քոռ էղնաս՝ դէմիս ընկնիս։} 
 	\item \armenian{Խունկ (եղունգ) չեղնա ՝ լաշիդ քերիս։} 
 	\item \armenian{Հոգո՛ւդի տո՛ւնէ փլի, սրտի՛դի սո՛ւնէ կոտրի։} 
 	\item \armenian{Դուն չիգաս ՝ խաբա՛րդի գա։} 
 	\item \armenian{Լաշփէ՛տիդ դօ՛ռէ իգա (դագաղդ դուռը գայ)։} 
 	\item \armenian{Հուղ չըգնաս մէ՛ջէ պառկիս։} 
 	 
 	 
\end{enumerate}

\subsection{Khian}
Adjarian's source: See \citeauthor{Byurakn}, 1898, page 301, 701, and 1899, page 560. 


\begin{enumerate}

\item \armenian{Լապստակ, փէտի վաստակ, վազէ վազէ փուրը դարտակ (կկոց)։}
\item \armenian{Գում մը ՝ մէջն ըլի (լի) սպիտակ մաքի (բերան)։}
\item \armenian{Տակը հուղ, մէչը շաղ, վրէն օսկի (ցորեն)։}
\item \armenian{Կը կապըն կը քէլա, կարցըկըն կը կէնա (տրեխ)։}
\item \armenian{Էրիկ կնիկ կռվան, աշվար գիցավ բաժնվան։}
\item \armenian{Հավկթէն է էլի, զհավկիթ չհավնի։}
\item \armenian{Գնա էնոր քով որ քեզ կի լացընէ, մի՛ էրթա էնոր որ կի խնդացընէ։}
\item \armenian{Կրակ որ ընկնի տաՙաշ (անտառ)՝ չուր ու դալար մէկտեղ կէրի։}
\item \armenian{Ինչ գար (չափ) իջխեր կա, էնգար ալ իլվեր կեղնի։}
\item \armenian{Սար ու ձուր ՝ տէրտըրու փուր։}
\item \armenian{Նա (ոչ) սուխ է կերի, նա հուտ իգա։}
\item \armenian{Աստված տեսեր է զսար, դրեր է զձուն։}
\item \armenian{թէ տէրտէրը մէկը կը գինա, երիցկին զերկուք կը քինա։}
\item \armenian{Ո՞ր աչք ո՞ր համար կիլա։}

\end{enumerate}

\subsection{Hazzo}

Adjarian's source: Ibid., 1898, \armenian{էջ} 538. The orthography is from \citeauthor{Byurakn}, such that the letter <\armenian{գ}> should be read as /ɡʰ/ <\armenian{գՙ}> , and <\armenian{պ-բ}> and so on. 

\armenian{Ւախթըմ ըգէր մարթըմ՝ ւոր գէց (քան զ) ուր հոգին կը սիրէր զուր գնիկ. ւոչ-պարէգէն (դժբախտաբար) թէրզով մէ կը գորցու զինք։ Տարտով բոլրկած դուշրմիշ կեղնէր զուր գտնելու ջա̈ր. ու բաշին (յետոյ) թողեց զուր երկիր, ընկավ քաղքէ քաղաք, օլըրտավ ուր գնկա յ}̵\armenian{էտեվ։ Շատ ջամբա քալեցուց թաշկած}... 


\begin{adjarianpage}\label{page:165}\end{adjarianpage}% should be 165

... \armenian{ու քրտնքով թրջուկ քաղաքըմ գոնըխեց (իջեւանիլ}). \armenian{էն գիշեր դեղըմ (տեղ մը) քնավ ու լուսմութ աղկեցու (եկեղեցւոյ) դուռ գաննավ ւոր ըրան փարսէ (մուրալ)։ Բարի մարդըմ հեցավ (հայեցաւ) ուր խեղճութին, մեղքունք էկավ ւըրէն ու զի՛նքի դարավ ուր տուն։ Փարսըք մարդ (մուրացկանը) յեփ մօտ խրակին (խարոյկին) կը ռըհաթնէր, նժճըվա (յանկարծ) հիցավ դռա մէչ դնիկըմ ւոր շիտակ ուրէն կը մեզէր (նայիլ}). \armenian{ա՛ձփփա՛ց (իսկոյն ցատկեց) ուր դեխէն (տեղէն) ու փաթվավ գնկա վզին ու շատ գուրգուրացին իլացին։ Բարի մարթն յեփ հիցավ զէտ անշըգ (զարմանալի) բան, շիվրավ գաննավ, պէլի (սակայն) գուզէր էտ պըմրատնու (անբախտ) սէրին բաշին հինէր։ Բաշին հարցուց. «Դօ լա՛վօ, էն ի՞նչ դավա է»։ Փարսըք մարթ իլալօվ պատմեց. «Էտ գնիկ իմուն է. թէրզովըմ գորուկ էր, էլիր ըմ ուր վրէն օլրտըմ ու հա օտան (հոդ) գտա զինք»։ Էտ բարի մարթ զգնիկ լը աղկեցու դուռ դեսեր ուր դուն բերեր էր բղելու (պահելու)։ Յեփ իմցավ, զուր հոգին շատ ուրըխցավ ու զէրիկ գնի հատիայով ջամբեց ուրաց էրկիր։}


\subsection{Hazro}

Adjarian's source: See \citeauthor{Byurakn} 1900, page 263. 

\armenian{Միր տան էտին ծառ սալոր էր},

\armenian{Ձիր տան էտին ծառ սալոր էր},,

\armenian{Ուր (իւր) հատիկը հինց կլոր էր},

\armenian{Ով վըր (որ) ուտէր չը հալվորէր։}

~ 

\armenian{Միր տան էտին քառսուն կարաս},

\armenian{Ձիր տան էտին քառսուն կարաս},

\armenian{Կարսու միչու գինին էր խաս},

\armenian{Օսգիէ դդում արծթէ թաս},

\armenian{Ըմըն թասին ընձի պագ մ՚ իտաս։}

~

\armenian{Միր տան էտին առուն հանած},

\armenian{Ձիր տան էտին առուն հանած},

\armenian{Բոլուր բոլուր բիհան ցանած},

\armenian{Էկավ անցավ նուր նշանած։}


\begin{adjarianpage}\label{page:166}\end{adjarianpage}% should be 166

\armenian{Միր հավշի հավուղը բուզ է},

\armenian{Ձիր հավշի հովուղը բուզ է},

\armenian{Վզի շարան յալտուզ է}.

\armenian{Աչքիս տեսավ սի՛րտիս կուզէ։}


\subsection{Edessa}

Adjarian's source: For this and the following see \citeauthor{Byurakn} 1900, page 331. 

\armenian{Մութ խարաբա. գրողը տանիս. գիտինը մտնաս. էրկու աչքտ կուրնա. պատին տակը մնաս. սի՛ւ Յուդա, լէշիդ ձգեմ. ադ օրը չտիսնաս. կարմիր արիւն շրջիս. ալյեշիլ դարձնես. օր արեւ չտիսնաս. մուրնը գլխուդ. ժառա՛նգիդ կարճ ըլլա. ժամուն դուռը մուրաս. ջիվան էրթաս. գօլող գանատող մնաս. Աստուծօ սրին գաս լաշդ լվան. էրթաս չի գաս. գարա գարա (սեւ սեւ) երրին (երկրին) տակը գնա, յօխ ըննաս։}

\subsection{Siverek}

\armenian{Քէօռ ըլլիս. տունդ ավրի. յօխ բէմուրազ ըլլի՛ս. խողը դըրվիս. փոշին գլխուդ. մուրը գլխուդ. օր արեւ չտեսնես. բօ՛յիդ բէդէ՛նիդ գէտին անցնի. խակ դրուիս. դուման իգա գլխուդ. գետնին յ}̵\armenian{ատակը էջնաս. պթխիս հըլլըսիս թափիս. ֆրանկ զահմաթի հանիս. գանջ ջիվան էրթաս. Աստծէն գտնաս. իշու հրեշտակ}.


\chapter{Kharberd-Yerznka}
\section{Background and literature}
\begin{adjarianpage}\label{page:167}\end{adjarianpage}% should be 167

The two main centers of the dialect are Kharberd and Yerznka (Turkish: Erzincan). The first is the southern edge of the region, while the second is the northern edge. The other primary places where this dialect is spoken are the following: Palu, Chapaghjur, Çemişgezek, Çarsancak, Kiğı, Dersim, and Kamakh. The western borderline of this dialect forms the current of the Euphrates river, in its entire length. From the north, one line of the Pontic mountains, while the other borders are determined by the borderlines of the Karin, Mush, and Tigranakert dialects. 

The language of the southern part of the region is quite well-studied. But for the northern part, there is very little known. For example, there is no information at all about the Kamakh province, and I presumably placed it in the aforementioned region. There is some information on the Yerznka dialect, in the periodical \citeauthor{Byurakn} (1898, page 563), and there is a quite extensive manuscript (see ibid. place, 1899, page 386-388). For the Dersim dialect, we can provide \citet{Antranik-1900-Dersim}, which is a volume of travel memoirs, but in some places he gives dialogues from this dialect. More extensive is article by Sarkis Haykuni (\armenian{Ս. Հայկունի}) called \armenian{Դերսիմ} (see \armenian{Արարատ}, 1896, page 183-5).\footnote{\translatorHD{The name of this supposed periodical is <\armenian{Արարատ}> `Ararat'. But such a name is quite common so I haven't been able to track down the exact source. }} 

{\litoverview}

There are manuscripts written in the Kiğı dialect in \citeauthor{Byurakn} (1898, page 201, 314, 315, 345, 472, 809, and 1899, page 554). There are many more manuscripts written in the Çarsancak dialect, such as:

 
 \begin{itemize}
 \item \armenian{Ս. Հայկունի}
 \begin{itemize}
 \item – \armenian{Հութութիկ եւ Սամէլ Հովիկ. Էջմիածին}, 1895
 \item – \armenian{Մռքոս. Էջմիածին}, 1896
 \item – 11 \armenian{ժողովրդական հեքիաթնըր՝ հրատարակուած Էմինեան Ազգ. Ժող. Բ}. 1901
 \end{itemize}
 \end{itemize}

\begin{adjarianpage}\label{page:168}\end{adjarianpage}% should be 168

There is nothing published with the vernaculars of Palu, Çemişgezek, and Chapaghjur. But there are many manuscripts in the mother dialect of Kharberd. In the \citeauthor{Byurakn} periodical: 
\begin{itemize}
 \item year 1898, page 331, 473, 583-4, 623, 671, 776
\item year 1899, page 18
\item year 1900, page 233, 316, 331, 491, 519, 730

\end{itemize}

There is also a small study on formation of this dialect (see \citeauthor{Byurakn}, 1899, page 777). 

I also have a separate study that I have formed with Dr. Andranik Hakobian (\armenian{բժ. Անդրանիկ Յակոբեան}; \translatorHD{SEA: /ɑndɾɑnik hɑkopjɑn/, SWA: /ɑntʰɾɑniɡ hɑɡopʰjɑn/}), which is still unpublished.\footnote{\translatorHD{For `unpublished', the original word is \armenian{անտիպ} which means `atypical'. I suspect this was a typo for \armenian{անտպուած} `unpublished'.}} 

It appears that a migrant Armenian community from Kharberd has settled in the upper district of Manisa, near Smyrna, who until now have kept their native dialect, with little changes and which is different from the dialect of lower district of Manisa. To establish this idea of mine, I have worked on an article in \citeauthor{Byurakn} (see 1899, page 402-405); on the occasion of this article, a response from Shahinian (\armenian{Շահինեան}; \translatorHD{SEA/SWA: /ʃɑhinjɑn/}) was published and a small study on this district's dialect (see 1899, page 291, 402, 503, 528, 575)

\section{Phonology}

\subsection{Vowels}
\subsubsection{Segment inventory}




The sound system of the dialect of Kharberd and Yerznka is much simpler than the dialects of Karin and Mush. The dialect of Kharberd-Yerznka recognizes the vowels /ɑ, æ, e, ə, i, o, u/ <\armenian{ա, ա}̈, \armenian{է, ը, ի, օ, ու}>, but it lacks the vowels //œ, ʏ, ie, uo/ <\armenian{էօ, իւ, ե, ո}>. 

\subsubsection{Sound changes}

The following are notable sound changes among vowels and diphthongs

\paragraph{Classical Armenian /oi̯/ <\armenian{ոյ}>}

The Classical sound /oi̯/ <\armenian{ոյ}> changed to /o/ <\armenian{օ}> (Table \ref{tab:KharberdYerznka:phonology:changes:vowel:oj}). 


\begin{table}[H]
 	\centering 
 	\caption{Change from Classical Armenian /oi̯/ <\armenian{ոյ}> to /o/ <\armenian{օ}> in the Kharberd-Yerznka dialect}
 	\label{tab:KharberdYerznka:phonology:changes:vowel:oj}
 	\begin{tabular}{|l| ll|ll| ll|}
 		\hline & \multicolumn{2}{l|}{Classical Armenian} &\multicolumn{2}{l|}{> Kharberd-Yerznka} & \multicolumn{2}{l|}{cf. SEA} \\ 
 		`light' & loi̯s & \armenian{լոյս} & los & \armenian{լօս} & lujs & \armenian{լույս} \\ 
 	 `sister' & kʰoi̯ɾ & \armenian{քոյր} & kʰoɾ & \armenian{քօր} & kʰujɾ & \armenian{քույր} \\ 
 	`walnut' & ənkoi̯z & \armenian{ընկոյզ} & ənɡoz & \armenian{ընգօզ} & əŋkujz & \armenian{ընկույզ} \\ 
\hline 
 	\end{tabular}
\end{table}


\paragraph{Classical Armenian /iu̯/ <\armenian{իւ}>}

The Classical sound /iu̯/ <\armenian{իւ}> changed to /i/ <\armenian{ի}> (Table \ref{tab:KharberdYerznka:phonology:changes:vowel:iu}). 

\begin{table}[H]
 	\centering 
 	\caption{Change from Classical Armenian /iu̯/ <\armenian{իւ}> to /i/ <\armenian{ի}> in the Kharberd-Yerznka dialect}
 	\label{tab:KharberdYerznka:phonology:changes:vowel:iu}
 	\begin{tabular}{|l| ll|ll| ll|}
 		\hline & \multicolumn{2}{l|}{Classical Armenian} &\multicolumn{2}{l|}{> Kharberd-Yerznka} & \multicolumn{2}{l|}{cf. SEA} \\ 
 		\armenian{՝}blood' & ɑɾiu̯n & \armenian{արիւն}& ɑɾin & \armenian{արին} & ɑɾjun & \armenian{արյուն} \\ 
 `fountain' & ɑɬbiu̯ɾ & \armenian{աղբիւր} & ɑχbʰiɾ & \armenian{ախբՙիր} & ɑχpjuɾ & \armenian{աղբյուր} \\ 
 `fountain' & eɬd͡ʒiu̯ɾ & \armenian{եղջիւր} & ɑχd͡ʒʰiɾ & \armenian{ախջՙիր} & jeʁd͡ʒjuɾ & \armenian{եղջյուր} \\ 
\hline 
 	\end{tabular}
\end{table}
\paragraph{Classical Armenian /e̯/ <\armenian{ե}>}

The Classical sound /e/ <\armenian{ե}> becomes /je/ <\armenian{յէ}> at the beginning of monosyllabic words, and it becomes /e/ <\armenian{է}> in all other circumstances (Table \ref{tab:KharberdYerznka:phonology:changes:vowel:e}). 


\begin{table}[H]
 	\centering 
 	\caption{Change from Classical Armenian /e/ <\armenian{ե}> to /je, e/ <\armenian{յէ, է}> in the Kharberd-Yerznka dialect}
 	\label{tab:KharberdYerznka:phonology:changes:vowel:e}
 	\begin{tabular}{|l| ll|ll| ll|}
 		\hline & \multicolumn{2}{l|}{Classical Armenian} &\multicolumn{2}{l|}{> Kharberd-Yerznka} & \multicolumn{2}{l|}{cf. SEA} \\ 
`ox' &ezən & \armenian{եզն} & jez & \armenian{յէզ} &jez & \armenian{եզ} \\
 \armenian{՝}when' & eɾb & \armenian{երբ} & jeb & \armenian{յէբ} & jeɾpʰ & \armenian{երբ} \\
`to sway' &eɾeɾɑl & \armenian{երերալ} & eɾeɾɑl & \armenian{էրէրալ} &jeɾeɾɑl & \armenian{երերալ} \\
`to appear' &eɾe{we}l & \armenian{երեւել} & eɾvɑl & \armenian{էրվալ} &jeɾevel & \armenian{երեւալ} \\
\hline 
 	\end{tabular}
\end{table}


\paragraph{Classical Armenian /o/ <\armenian{ո}>}

The Classical sound /o/ <\armenian{ո}> becomes /o/ <\armenian{օ}> everywhere (Table \ref{tab:KharberdYerznka:phonology:changes:vowel:o}). 


\begin{table}[H]
 	\centering 
 	\caption{Change from Classical Armenian /o/ <\armenian{ո}> to /o/ <\armenian{օ}> in the Kharberd-Yerznka dialect}
 	\label{tab:KharberdYerznka:phonology:changes:vowel:o}
 	\begin{tabular}{|l| ll|ll| ll|}
 		\hline & \multicolumn{2}{l|}{Classical Armenian} &\multicolumn{2}{l|}{> Kharberd-Yerznka} & \multicolumn{2}{l|}{cf. SEA} \\ 
`alive' & oɬd͡ʒ & \armenian{ողջ} & oχt͡ʃʰ & \armenian{օխչ} & voχt͡ʃʰ& \armenian{ողջ} \\
`lentil' & ospən & \armenian{ոսպն} & osb & \armenian{օսբ} & vosp & \armenian{ոսպ} \\
 `foot' & ot-(ə?)kʰ (-{\pl}) & \armenian{ոտք} & odʰkʰ & \armenian{օդՙք} & votkʰ & \armenian{ոտք} \\
 \armenian{՝}orphan' & oɾb & \armenian{որբ} & oɾbʰ & \armenian{օրբՙ} & voɾpʰ & \armenian{որբ} \\
`ryegrass' & oɾomən & \armenian{որոմն} & oɾom & \armenian{օրօմ} & voɾom & \armenian{որոմ} \\
`to thunder' & oɾotɑl & \armenian{որոտալ} & oɾotɑl & \armenian{օրօտալ} & voɾotɑl & \armenian{որոտալ} \\
 \armenian{՝}to take pity on' & oɬoɾmil & \armenian{ողորմիլ} & oʁoɾmil & \armenian{օղօրմիլ} & voʁoɾmel & \armenian{ողորմել} \\
\hline 
 	\end{tabular}
\end{table}


\paragraph{Classical Armenian /ɑi̯/ <\armenian{այ}>}


The Classical diphthong /ɑi̯/ <\armenian{այ}> becomes /æ/ <\armenian{ա}̈> (Table \ref{tab:KharberdYerznka:phonology:changes:vowel:aj}). With this sound, the dialect presents a type of middle point for the sound changes of /e/ <\armenian{է}> and /ɑ/ <\armenian{ա}>.

\begin{table}[H]
 	\centering 
 	\caption{Change from Classical Armenian /ɑi̯/ <\armenian{այ}> to /æ/ <\armenian{ա}̈> in the Kharberd-Yerznka dialect}
 	\label{tab:KharberdYerznka:phonology:changes:vowel:aj}
 	\begin{tabular}{|l| ll|ll| ll|}
 		\hline & \multicolumn{2}{l|}{Classical Armenian} &\multicolumn{2}{l|}{> Kharberd-Yerznka} & \multicolumn{2}{l|}{cf. SEA} \\ 
 `mother' & mɑi̯ɾ & \armenian{մայր} & mæɾ & \armenian{մա̈ր} & mɑjɾ & \armenian{մայր} \\
 		`wood' & pʰɑi̯t & \armenian{փայտ} & pʰɑd & \armenian{փա̈դ} &pʰɑjt & \armenian{փայտ} \\ 
`mirror' & hɑ{je}li & \armenian{հայելի} & hælli & \armenian{հա̈լլի} & hɑjeli & \armenian{հայելի} \\ 
\hline 
 	\end{tabular}
\end{table}

\subsection{Consonants}
\subsubsection{Segment inventory}
The consonants have three degrees in the dialect: voiced, voiced aspirate, and voiceless aspirates. The voiceless aspirated series... 

\begin{adjarianpage}\label{page:169}\end{adjarianpage}% should be 169

... does not exist. The voiced sounds of Old Armenian are changed to voiced aspirates, the voiceless unaspirates are changed to voiced, while the voiceless aspirates stay voiceless aspirated. Besides these, the palatalized consonants /ɡʲ/ <\armenian{գյ}> and /kʰʲ/ <\armenian{քյ}> are created. Whenever the sounds /ɡ, kʰ/ and <\armenian{գ, ք}> follow /e, i/ and <\armenian{է, ի}>, they become /ɡʲ, kʰʲ/ <\armenian{գյ, քյ}>. 

\subsubsection{Sound changes}

Among consonant changes, the most famous ones are the following.

\paragraph{Stop-nasal assimilation}


The Classical sound /t/ <\armenian{տ}> before /n/ <\armenian{ն}> assimilates to become /n/ <\armenian{ն}> (Table \ref{tab:KharberdYerznka:phonology:changes:cons:tn}). 


\begin{table}[H]
 	\centering 
 	\caption{Change of /tn/ <\armenian{տն}> to /nn/ <\armenian{նն}> in the Kharberd-Yerznka dialect}
 	\label{tab:KharberdYerznka:phonology:changes:cons:tn}
 	\begin{tabular}{|l| ll|ll| ll|}
 		\hline & \multicolumn{2}{l|}{Classical Armenian} &\multicolumn{2}{l|}{> Kharberd-Yerznka} & \multicolumn{2}{l|}{cf. SEA} \\ 
 `to enter' &mətɑnel & \armenian{մտանել} & mənnel & \armenian{մըննէլ} & mətnel& \armenian{մտնել} \\ 
 \armenian{՝}to find' & ɡətɑnel & \armenian{գտանել}& ɡʰənnel & \armenian{գՙըննէլ} & ɡətnel & \armenian{գտնել} \\

\hline 
 	\end{tabular}
\end{table}

\paragraph{Fricative deletion in word-inital /s/-stop clusters } 


The sound /s/ <\armenian{ս}> at the beginning of words is deleted before the sounds /p, t/ and <\armenian{պ, տ}> (Table \ref{tab:KharberdYerznka:phonology:changes:cons:tn}). 


\begin{table}[H]
 	\centering 
 	\caption{Deletion of initial /s/ <\armenian{ս}> in /sp, st/ <\armenian{սպ, ստ}> clusters in the Kharberd-Yerznka dialect}
 	\label{tab:KharberdYerznka:phonology:changes:cons:st}
 	\begin{tabular}{|l| ll|ll| ll|}
 		\hline & \multicolumn{2}{l|}{Classical Armenian} &\multicolumn{2}{l|}{> Kharberd-Yerznka} & \multicolumn{2}{l|}{cf. SEA} \\ 
 `to kill' &əspɑnɑnel & \armenian{սպանանել} & bɑnnel & \armenian{բաննէլ} & spɑnel& \armenian{սպանել} \\ 
 `white' &əspitɑk & \armenian{սպիտակ} & bidɑɡ & \armenian{բիդագ} & spitɑk& \armenian{սպիտակ} \\ 
 `to create' &əsteɬt͡sɑnel & \armenian{ստեղծանել} & deʁd͡zel & \armenian{դէղձէլ} & steχt͡sel & \armenian{ստեղծել}\\ 
 `carrot' &əstepɬin & \armenian{ստեպղին} & dɑbʁin & \armenian{դաբղին} & stepʁin& \armenian{ստեպղին} \\ 
 `sterile' &əsteɾd͡ʒ & \armenian{ստերջ} & deɾt͡ʃʰ & \armenian{դէրչ} & steɾd͡ʒ& \armenian{ստերջ} \\ 

\hline 
 	\end{tabular}
\end{table}


\paragraph{Consonant cluster reduction for obstruent-rhotics}

In both Yerznka and Kharberd, the words in Table \ref{tab:KharberdYerznka:phonology:changes:cons:cccr} changed.\footnote{\translatorHD{For the words `high' and `thick', I suspect that Adjarian incorrectly switched the Kharberd forms. But unfortunately, I cannot be sure. }}



\begin{table}[H]
 	\centering 
 	\caption{Reduction of consonant clusters with for obstruent-rhotics in the Kharberd-Yerznka dialect}
 	\label{tab:KharberdYerznka:phonology:changes:cons:cccr}
 	\begin{tabular}{|l| ll|ll| ll|}
 		\hline & \multicolumn{2}{l|}{Classical Armenian} &\multicolumn{2}{l|}{> Kharberd-Yerznka} & \multicolumn{2}{l|}{cf. SEA} \\ 
 	`sparse' & nɑu̯səɾ & \armenian{նաւսր} & nosɾ & \armenian{նօսր} & nosəɾ & \armenian{նոսր} \\ 
 	`high' &bɑɾd͡zəɾ & \armenian{բարձր} & tʰɑɾz & \armenian{թարզ} & bɑɾt͡sʰəɾ & \armenian{բարձր} \\ 
 	`thick' & tʰɑnd͡zəɾ & \armenian{թանձր} & bʰɑɾs & \armenian{բՙարս}& tʰɑnd͡zəɾ & \armenian{թանձր} \\ 
 	 `sweet' & kʰɑɬt͡sʰəɾ& \armenian{քաղցր} & kʰʲɑrs & \armenian{քառս} & kʰɑχt͡sʰəɾ & \armenian{քաղցր} \\ 
 	 	\hline 
 	\end{tabular}
\end{table}

\paragraph{Consonant cluster reduction for nasal-rhotics}

The words in Table \ref{tab:KharberdYerznka:phonology:changes:cons:cccnr} changed.


\begin{table}[H]
 	\centering 
 	\caption{Reduction of consonant clusters with for obstruent-rhotics in the Kharberd-Yerznka dialect}
 	\label{tab:KharberdYerznka:phonology:changes:cons:cccnr}
 	\begin{tabular}{|l| ll|ll| ll|}
 		\hline & \multicolumn{2}{l|}{Classical Armenian} &\multicolumn{2}{l|}{> Kharberd-Yerznka} & \multicolumn{2}{l|}{cf. SEA} \\ 
 		`heavy' & t͡sɑnəɾ & \armenian{ծանր} & t͡sɑjɾ & \armenian{ծայր} & t͡sɑnəɾ & \armenian{ծանր} \\ 
 				`small' & mɑnəɾ & \armenian{մանր} & mɑjɾ & \armenian{մայր} & mɑnəɾ & \armenian{մանր} \\ 
 				`comb' & sɑntəɾ & \armenian{սանտր} & sɑjɾ & \armenian{սայր} & sɑnəɾ & \armenian{սանր} \\ 
 		\hline 
 	\end{tabular}
\end{table}

\paragraph{Fronting of post-alveolar obstruents}

The Dersim province also has a surprising innovation. The Classical sounds /t͡ʃ, d͡ʒ, t͡ʃʰ/ <\armenian{ճ, ջ, չ}> become /d͡z, d͡zʰ, t͡sʰ/ < \armenian{ձ, ձՙ, ց}> (after passing through the forms /t͡s, d͡z, t͡sʰ/ <\armenian{ծ ձ ց}. While the sound /ʃ/ <\armenian{շ}> becomes /s/ <\armenian{ս}> (Table \ref{tab:KharberdYerznka:phonology:changes:cons:fronting}, sentence \ref{sent:KharberdYerznka:phonology:changes:cons:fronting}).



\begin{table}[H]
 	\centering 
 	\caption{Fronting of post-alveolar obstruents in the Kharberd-Yerznka dialect}
 	\label{tab:KharberdYerznka:phonology:changes:cons:fronting}
 	\begin{tabular}{|l| ll|ll| ll|}
 		\hline & \multicolumn{2}{l|}{Classical Armenian} &\multicolumn{2}{l|}{> Kharberd-Yerznka} & \multicolumn{2}{l|}{cf. SEA} \\ 
 		`white' & t͡ʃeɾmɑk & \armenian{ճերմակ} & d͡zeɾmɑɡ & \armenian{ձէրմագ} & t͡ʃeɾmɑk & \armenian{ճերմակ} \\ 
`water' &d͡ʒuɾ & \armenian{ջուր} & d͡zʰuɾ & \armenian{ձՙուր} & d͡ʒuɾ & \armenian{ջուր} \\ 
 		`raisin' & t͡ʃʰɑmit͡ʃʰ & \armenian{չամիչ} & t͡sʰɑmit͡sʰ & \armenian{ցամից} & t͡ʃʰɑmit͡ʃʰ & \armenian{չամիչ} \\ 
 		`I pulled' & kʰɑʃet͡sʰi & \armenian{քաշեցի} & kʰɑset͡sʰi & \armenian{քասէցի} & kʰɑʃet͡sʰi & \armenian{քաշեցի} \\ 
 						\hline 
 	\end{tabular}
\end{table}

\begin{exe}
 	\ex Kharberd-Yerznka dialect \label{sent:KharberdYerznka:phonology:changes:cons:fronting}
 	\gll met͡sʰ-ə d͡zʰuɾ t͡sʰ-iɡ-ɑ-$\emptyset$ \\ 
 	in-{\defgloss} water {\neggloss}-exist-{\thgloss}-3{\sg} \\ 
 	\trans `There is no water in it.' \\ 
 	\armenian{մէցը ձՙուր ցիգա} 
\end{exe}

\section{Morphology}

\subsection{Noun inflection or declension}

In the grammar, there are no individual innovations. As in all other typical dialects of its type, the Kharberd-Yerznka dialect has 6 cases: nominative, genitive, dative, accusative, ablative, and instrumental. The dative is always the same as the genitive, while the accusative is the same as the nominative, without a difference between animate and inanimate objects. The ablative formative is /-e/ <\armenian{է}>. The plural markers are /-eɾ, -neɾ/ <\armenian{էր, նէր}>. 
 
\subsection{Pronoun inflection or declension}
For the pronouns, we cite the following (Table \ref{tab:KharberdYerznka:morphology:pronoun:sample}).\footnote{\translatorHD{For the plural nouns, I mark them as synthetic for accusative and dative, but it's possible that Adjarian meant that they are strictly accusative. The prose isn't clear. }}



\begin{table}[H]
 	\centering 
 	\caption{Sample of pronouns in the Kharberd-Yerznka dialect}
 	\label{tab:KharberdYerznka:morphology:pronoun:sample}
 	\begin{tabular}{|l ll|}
 		\hline 
 		personal 1SG {\acc} `me' &ind͡zis, ənd͡zis & \armenian{ինձիս, ընձիս} \\ 
 		personal 1SG {\abl} `from me' &imməne & \armenian{իմմընէ}\\ 
 		personal 1PL {\acc}-{\dat} `to us' &mizi, mzi & \armenian{միզի, մզի}\\ 
 		personal 2PL {\acc}-{\dat} `to you' &d͡zʰizi, d͡zʰzi & \armenian{ձՙիզի, ձՙզի}\\ 
 		personal 2SG {\acc}-{\dat} `to you' &kʰizi, kʰzi & \armenian{քիզի, քզի}\\ 
 		personal 1PL {\abl} `from us' &meɾməne & \armenian{մէրմընէ}\\ 
 		personal 2SG {\abl} `from you' &kʰuməne & \armenian{քումընէ}\\ 
 		personal 2PL {\abl} `from you' &d͡zʰeɾməne & \armenian{ձՙէրմընէ}\\ 
 																\hline 
 	\end{tabular}
\end{table}

 

\subsection{Verb inflection or conjugation}


The verb is very simple. The rule of changing the Classical vowel /e/ <\armenian{ե}> to /i/ <\armenian{ի}> takes places only the in 1SG and 1PL persons. In the third person, it changes to /æ/ <\armenian{ա}̈> (in the first conjugation class).

{\paradigmExplanation}

\translatorHD{For the indicative present, SWA combines the indicative prefix /ɡ(ə)/ <\armenian{կը}> with a finite verb. This finite verb is the subjunctive form. For an E-Class verb like `to like' /siɾ-e-l/, the theme vowel is a constant /e/, and the 3SG marker is covert. In Kharberd-Yerznka, the theme vowel varies between /æ, i, e/ (Table \ref{tab:KharberdYerznka:morpho:verb:paradigm:presentPastIndc}). }


\begin{table}[H]
 	\centering 
 	\caption{Indicative present <\armenian{ներկայ}> of the verb `to like' in the Kharberd-Yerznka dialect}
 	\label{tab:KharberdYerznka:morpho:verb:paradigm:presentPastIndc}
 	 \begin{tabular}{|l| ll| ll|}
 		\hline & \multicolumn{2}{l|}{Kharberd-Yerznka} & \multicolumn{2}{l|}{cf. SWA} \\ \hline 
1SG & ɡə siɾ-i-m & \armenian{գը սիրիմ} & ɡə siɾ-e-m & \armenian{կը սիրեմ} \\
2SG & ɡə siɾ-e-s &\armenian{գը սիրէս} & ɡə siɾ-e-s & \armenian{կը սիրես} \\
3SG & ɡə siɾ-æ-$\emptyset$ & \armenian{գը սիրա}̈ & ɡə siɾ-e-$\emptyset$ & \armenian{կը սիրէ} \\
1PL & ɡə siɾ-i-nkʰʲ & \armenian{գը սիրինքյ} & ɡə siɾ-e-ŋkʰ & \armenian{կը սիրենք} \\
2PL & ɡə siɾ-e-kʰʲ & \armenian{գը սիրէքյ} & ɡə siɾ-e-kʰ & \armenian{կը սիրէք} \\
3PL & ɡə siɾ-e-n& \armenian{գը սիրէն} & ɡə siɾ-e-n & \armenian{կը սիրեն} \\
 		& \multicolumn{2}{l|}{{\ind} $\sqrt{}$-{\thgloss}-{\agr}} & \multicolumn{2}{l|}{{\ind} $\sqrt{}$-{\thgloss}-{\agr}} \\ 
 		\hline 
 		 
 	\end{tabular}
\end{table}

\begin{adjarianpage}\label{page:170}\end{adjarianpage}% should be 170

The imperfective and the perfective are the same as in the old forms. 

The future is formed with the formative /də/ <\armenian{դը}> (/tə/ <\armenian{տը}>), which is a shortened from of /piti/ <\armenian{պիտի}>. In the negative future, this sound /p/ <\armenian{պ}> changes to /v/ <\armenian{վ}> (\ref{sent:KharberdYerznka:morpho:fut}). Forms like in (\ref{sent:KharberdYerznka:morpho:fut:Not}) don't exist. 

\begin{exe}
 	\ex Kharberd-Yerznka dialect \label{sent:KharberdYerznka:morpho:fut}
 	\begin{xlist}
 		\ex \gll t͡ʃʰə-vdi siɾ-i-m \\ 
 	{\neggloss}-{\fut} like-{\thgloss}-1{\sg} \\ 
 	\trans `I will not like.' \\ 
 		\armenian{չըվդի սիրիմ} 
 		\ex \gll t͡ʃʰə-vdi siɾ-e-i-$\emptyset$ \\ 
 		{\neggloss}-{\fut} like-{\pst}-{\thgloss}-1{\sg} \\ 
 		\trans `I was not going to like.' \\ 
 		 \armenian{չըվդի սիրէի} 
 	\end{xlist}
 	\ex Absent in the Kharberd-Yerznka dialect \label{sent:KharberdYerznka:morpho:fut:Not}
\begin{xlist}
 	\ex \gll * t͡ʃʰ-piti siɾ-i-m \\ 
 	{\neggloss}-{\fut} like-{\thgloss}-1{\sg} \\ 
 	\trans Hypothetical but unattested: `I will not like.' \\ 
 	\armenian{չպիտի սիրիմ} 
 	\ex \gll * piti t͡ʃʰ-siɾ-i-m \\ 
{\fut} {\neggloss}-like-{\thgloss}-1{\sg} \\
\trans Hypothetical but unattested: `I will not like.' \\
\armenian{պիտի չսիրիմ}

\end{xlist}
\end{exe}

\section{Text samples}



{\sampleoverview}

\subsection{City of Kharberd }

Adjarian's source: Taken from \citeauthor{Byurakn}, 1900, page 730. The orthography was verified by me. 

\armenian{Գըլլի չըլլի խօռօզ մը գըլլի։ ա̈ս խօռօզին օդքը փուշ մը գը մըննա. ի՛նչ գէնէ չէնէր՝ չի գըրնըր ա̈դ փուշը հանէր։ Գէլլա գէրթա մամիգի մը զըստ քի ա̈ս փուշը հանա̈։ Մամիգն ըլ գը հանա̈ ու թօնիրը գը ցըքա̈։ Մէգ-էրգու օր ասնէլուն բէս՝ ըս խօռօզը գէրթա մամիգին գըստ քի փուշս դուր։ Մամիգն ըլ զըստ քի փուշը վառավ ա̈լ ո՛ւսգաց դամ։ Խօռօզը հըմըն հաց մի գառնա̈ ու գը փախի։ Գէրթա օր չօբան մի նսդէր է գաթին մէչ բդուր (սպտուր) գը բՙրդՙա̈ գուդա̈։ Գըստ քի ըս հացը առ գէր, ըն ըլ գառնա̈։ Քանի մը օր գասնի, գէրթա չօբանին գըստ քի հացս դուր, ըն ըլ չունի օր դա, ըս հէղուն ըլ խօռօզը ըդգից մաքի մի գառնա̈ ու գը փախի։ Խօռօզը գէրթա օր դէղ մի շուն մի մօրթէր էն ու քէշգէգ բիդի (դը) էթէն. ըդօնց գըստ քի ըս մաքին առէք։ Քանի մի օր սօղը գՙուգՙա քի մաքիս դըվէք։ Ընօնք ըլ մդիգ չէն էնէր, ինք ըլ ըդօնց հարս գառնա̈ ու գը փախի։ Շադ գէրթա ՝ քիչ գէրթա, գը տէսնա՝ օր մէգ մարթ մի նսդէր ջըզդըրիգ (ջութակ) կը զէնա}̈. \armenian{ըդօր գըստ քի ջըզդըրիզդ դուր օր ըս հարսը դամ։ Հարսը գուդա՝ ջըզդըրիզը գառնա։ Խօռօզը գը նսդի ձառի մը դագ}\footnote{\translatorHD{The original page had <\armenian{դակ}>. But on page 306, Adjarian provides an erratum that the form is <\armenian{դագ}>; I fixed it.}} \armenian{ու կը բըլըշվի (թրք. բաշլամաք ՝ «սկսիլ») ջըզդըրիգը ջըզդըրցընէլ ու խաղ գանչէլ. «Ջըզդըր, ջըզդըր, ջըզդըրիգ, փուշ մի դվի՝ հաց մի առի, հացը դըվի՝ մաքի մի առի, մաքին դըվի՝ հարս մի առի, հարսը դվի ՝ ջըզդըռրիգ մի առի, ջըզդըր, ջըզդըր ջըզդըրիգ»։}


\begin{adjarianpage}\label{page:171}\end{adjarianpage}% should be 171

\subsection{From one village from Yerznka}

Adjarian's source: Taken from \citeauthor{Byurakn}, 1899 \armenian{էջ} 386.The orthography of the manuscript was preserved, even if inaccurate. 

\armenian{Վախտովը էրիկ մ՚ու կնիկ մը կան ա̈ղեր։ Էրիկը թէնպէլ, կնիկը էտէպսիզ։ ա̈նմէն օր առտու լուսծածին պէս կնիկը էրկանը առջեւը երկու հաց կը նետէ, «դուս էլ, աշքիս մ՚երեւար» կըսէ ա̈ղեր։ Էրիկը կիթա ծովուն քէնարը կը նստի, մէկ հաց ինք կուտէ, մէկալն ա ծովուն ձկներուն կը նետէ, կէսը առտուն, կէսն ա ճաշուն։ Պզտի ձկները հացը տանելու չապալամիշ կ՚ըլլան ըմա, ա̈նմէն օր մեծ ձուկ մը կուգա ընոնց ձեռքէն կառնէ կը տանի ա̈ղեր. ըսանկ կանէ տարի մը ա̈նմէն օր։ Անիսէ ձուկը ըս հացերը իրենց մեծաւորին կը տանի ա̈ղեր. մեծաւորնին ընանկ հիւընտութիւն մը կունենա օ բոլոր հէքիմները չին կանա բռըտցըներ. էն ետքը կըսեն քի, էկէր տարի մը հաց ուտէ նը կըռընանա ըս ձուկն ա ըս հացերը մեծաւորին կը տանի ա̈ղեր։ Տարիէն ետքը մեծաւորը կըռըտնա. իրեն հաց բերող ձուկը ա̈ռջեւը կը կանչէ, կը հարցնէ քի ըս հացերը տարի մըն է ո՞ւսկէ կը բերէ. ան ա կըսէ քի «ծովուն քէնարը մարդ մը նստեր ա̈՝ ա̈նմէն օր ըտ հացերը ծովը կը նետէ, ես ա կառնեմ քեզի կը բերիմ»։ Ըն սրային մեծաւորը հրաման կանէ օ իթա ըն մարդը բերէ օ մուրատ տա իրին արած լավութան տեղը։}

\subsection{Kiğı}

Adjarian's source: See \citeauthor{Byurakn}, 1898, taken from various places.

\armenian{Ածան հաւը կարգճան կ՚ըլլի։}

\armenian{Ասղին ծակով Հինտիստան կը հայնա։}

\armenian{Էրէցն օր թքան, կըսէ «ամպ գՙուգՙա»։}

\armenian{Խեւը գնաց հարսնետունը, ըսաց հոս լաւ է քընծ մեր տունը։}

\armenian{Կուշտն անօթուն մայր (մանր) կը փշէ (կը փշրէ)։}

\armenian{Հարիր մազէ ալիր է կերեր։}

\armenian{Հաւն օր հաւ է, ջուր խմած ատենը Աստված ի վեր կը հայնա։}

\armenian{Շունը կը զինին՝ տիրունմինէ կամըչնան։}

\armenian{Ընի իմ արծած խոզն է։}
 
\begin{adjarianpage}\label{page:172}\end{adjarianpage}% should be 172

\armenian{Օրը հարիր սիրտ կրնաս կօյրեր, ըմը հարիր օր էկէր դատիս ՝ սիրտ մը չես կրնար շինիր։}

\armenian{Վով չուսթ, փորը կուշտ։}

\armenian{Տունը չգա տան տիկին՝ հորթուն կըսեն լոս տիկին։}

\subsection{Çarsancak}

Adjarian's source: See \citeauthor{Eminian}, volume 2 (\armenian{Բ}), page 152. 

\armenian{Կըլլի թաքավօր մի. իրէք հատ աղա կունէնա. կըսը մենծ աղին}. – \armenian{Տղա՛ս, ես ա̈լ ծեր եմ, տասնըհինգ տարի սայմանիս գլօխը չայիրը չիմ գցեր։}

– \armenian{Արքա՛յ հայրիկ, կըսը աղան, միշտ սայմանիդ գլխու չայիրը շատ մէթ կինիս (կը գովես)։ – ա̈ն սայմանի չայիրին հավան է օր էօմիւրս էրկընցաւ. ա̈ս տասնըհինգ տարի օր չիմ գըցեր, էօմիւրս փճացաւ. գնա օ տեսնըս. չայիրին գլօխը աղբիր մը կա. չայիրին տկուն տէ՜ (մինչեւ) աղբիրը սահաթ մի կը քաշէ. աղբիրն ա̈լ լեռան տակ է աղբրէն տէ՜ լեռան գլօխը տասվերկու սահաթ կը քաշէ։}

\armenian{Էլավ մենծ տղան հարիր հատ ձիով առըց. գըցին հասան չայիրը. տեսան օ մէ արաբ մի նստեր է աղբրին վրա։ Մի սեւ անպ էլավ, էրկինքը գօռաց, արզեւն ու կարկուտը առըց։}

\armenian{Թագաւորին տղան ըսըց}. – \armenian{Քշեցէք, ըյնինք աղբիդը։}

\armenian{Չայիրը կէս ըրին-չըրին՝ ջրին ու մլին մէջ մնացին. չկըրցին օ հասնէին։}

\armenian{Արաբը թուրը քաշեց, ընկաւ ա̈ս հարիր հատի մէջ. հարիր հատին ալ գլօխը կըյրեց, ձիանն ա̈լ մորթեց, յէղմիշ ըրըց (դիզեց) չայիրին օրթըլըխը, էջաւ սըրթն ի վա̈ր գնըց։}

\armenian{Մի ամիս թաքաւորին խապար չգնըց։}

\armenian{Թաքաւորը կանչեց օրթանճա տղին։}

– \armenian{Այ ա̈ղա՛, աղբա̈րդ ճանփէցի սայմանի գլօխը. յա (կամ) բռնվան, յա ծեծ (պատերազմ) կինին. յա քէֆի տեղ է՝ իրանց քէֆը կը հային. գնա խապար մի ա՛ռ էկօ՛։}

\subsection{Dersim}

Adjarian's source: See <\armenian{Արրտ}>. 1896, page 183. \translatorHD{It's not clear to me what this <\armenian{Արրտ}> source is; it's likely a periodical called `Ararat' <\armenian{Արարատ}>, but there are many such periodicals. }

\armenian{Անքան դէսա օր Գիրօն ասգրին մէց է. ցօրս դՙին բադի բէս բարաձ է}. – \armenian{Օվա՛նէս, ըսըց- ի՛նց գինէս, ինձի բարութ գիւլլէ հասուր։}


\begin{adjarianpage}\label{page:173}\end{adjarianpage}% should be 173

\armenian{Օվանէսը գՙնըց օ մէր աղան ձՙինը հէձէր, փախէր է. դէսա օր Գիրօն թուրը ցըլբըղցօց, թուրը փառլամիս էղավ, յէս ա̈լ ըս դՙիուն քասէցի, թուրը փառլաթմիս էրի – օ, Գիրօ, մի՛ վախիր։ Ընխդըր (այնչափ) հայա (տեսայ) օ, ցօրս սվարօվ դալիս հէձաձ, ամէքուն թուր մի նէդէցի, ցօրս ձՙին ա̈լ բառգէցուցի. մէգ մարթուն ա̈լ գՙլօխը գըյրէցի, դէսա օր փասան գըցէ}. – \armenian{Հա՛ բաբամ օլասըզ հա։ Ցի գՙիդէր օ յէս իմ, ընի գՙիդէ թէ թուրքէր Ղամբէրն է բՙռնած. ցըսէր թէ հինգ հոքի թրէս ընցուցէր իմ. անքան դէսա օ Օվանէսը գիւլլէն հասօց, ըսըց}. – \armenian{Մի վախէխ, ախբՙրդանք. դՙու մըսէր, թաբուր մի ասգրին գէսն է մնացէր. մէր աղան ա̈լ էլէր թամասա գինէր. իրգուն գՙցիք քառցուն օսգի տվէց, ի՞նց ինիմ։}


\chapter{Şebinkarahisar}


\begin{adjarianpage}\label{page:174}\end{adjarianpage}% should be 174

\section{Background}
At the north side of Kharberd-Yerznka, the city of Şebinkarahisar and the province of Adzbder together form a separate dialect; the dialect is occupies a middle ground between the dialects of Kharberd-Yerznka, Sebastia, and Evdokia. 

\section{Phonology}
\subsection{Consonants}
Like the first two dialects (\translatorHD{Kharberd-Yerznka and Sebastia}), the dialect has three groups of consonants that are missing from Evdokia: voiced, voiced aspirate, and voiceless aspirate. There is also here the glottal /ɦ/ <\armenian{հագագ յ}̵>. 

\subsection{Vowels}
\subsubsection{Segment inventory}

The vowel system is like the Evdokia dialect, which we will later see below. The sound /æ/ <\armenian{ա}̈> is added (Table \ref{tab:Şebinkarahisar:phonology:changes:a}). 


\begin{table}[H]
 	\centering 
 	\caption{Emergence of /æ/ <\armenian{օ}> in the Şebinkarahisar dialect}
 	\label{tab:Şebinkarahisar:phonology:a}
 	\begin{tabular}{|l| ll|ll| ll|}
 		\hline & \multicolumn{2}{l|}{Classical Armenian} &\multicolumn{2}{l|}{> Şebinkarahisar} & \multicolumn{2}{l|}{cf. SEA} \\ 
 		`before' & ɑrɑd͡ʒe̯ɑu̯ & \armenian{առաջեաւ} & ært͡ʃʰev & \armenian{ա̈ռչէվ} & ɑrt͡ʃʰev & \armenian{առջեւ} \\ 
\hline 
 	\end{tabular}
\end{table}


\subsubsection{Sound changes}
\paragraph{Classical Armenian /o/ <\armenian{ո}> }
Like the Sebastia and Evdokia dialects, the sound /o/ <\armenian{ո}> has changed to /œ/ <\armenian{էօ}> (Table \ref{tab:Şebinkarahisar:phonology:changes:o}). 


\begin{table}[H]
 	\centering 
 	\caption{Change from Classical Armenian /o/ <\armenian{ո}> to /œ/ <\armenian{էօ}> in the Şebinkarahisar dialect}
 	\label{tab:Şebinkarahisar:phonology:changes:o}
 	\begin{tabular}{|l| ll|ll| ll|}
 		\hline & \multicolumn{2}{l|}{Classical Armenian} &\multicolumn{2}{l|}{> Şebinkarahisar} & \multicolumn{2}{l|}{cf. SEA} \\ 
 \armenian{՝}work' & ɡoɾt͡s & \armenian{գործ}& ɡʰœɾd͡z & \armenian{գՙէօրձ} & ɡoɾt͡s & \armenian{գործ} \\
name `Peter' &petɾos & \armenian{Պետրոս} & bœrœs & \armenian{Բէօռէօս} & petɾos & \armenian{Պետրոս} \\
\hline 
 	\end{tabular}
\end{table}

\paragraph{Classical Armenian /e/ <\armenian{ե}> }

There is an innovation that is absent in the Kharberd-Yerznka and Evdokia dialect: the sound /e/ <\armenian{ե}> changes to /i/ <\armenian{ի}> when stressed (Table \ref{tab:Şebinkarahisar:phonology:changes:e}). 


\begin{table}[H]
 	\centering 
 	\caption{Change from Classical Armenian /e/ <\armenian{ե}> changes to /i/ <\armenian{ի}> in the Şebinkarahisar dialect}
 	\label{tab:Şebinkarahisar:phonology:changes:e}
 	\begin{tabular}{|l| ll|ll| ll|}
 		\hline & \multicolumn{2}{l|}{Classical Armenian} &\multicolumn{2}{l|}{> Şebinkarahisar} & \multicolumn{2}{l|}{cf. SEA} \\ 
 		\armenian{՝}place' & teɬ & \armenian{տեղ}& diʁ & \armenian{դիղ} & teʁ & \armenian{տեղ} \\ 	 
 `with' & het & \armenian{հետ}& hid & \armenian{հիդ} & het& \armenian{հետ} \\
 `you.{\sg}.{\dat}' & kʰez& \armenian{քեզ} & kʰiz & \armenian{քիզ} & kʰez & \armenian{քեզ} \\
\hline 
 	\end{tabular}
\end{table}

With this sound change, the dialect of Şebinkarahisar approaches the dialect of Hamshen, where this same sound change exists (Table \ref{tab:Şebinkarahisar:morpho:verb:theme}). 


\begin{table}[H]
 	\centering 
 	\caption{Change from Classical Armenian /e/ <\armenian{ե}> changes to /i/ <\armenian{ի}> in the Hamshen dialect}
 	\label{tab:Şebinkarahisar:morpho:verb:theme}
 	\begin{tabular}{|l| ll|ll| ll|}
 		\hline & \multicolumn{2}{l|}{Classical Armenian} &\multicolumn{2}{l|}{> Şebinkarahisar} & \multicolumn{2}{l|}{cf. SEA} \\ 
 		`big' &met͡s & \armenian{մեծ} & mid͡z& \armenian{միձ} &met͡s & \armenian{մեծ} \\ 
 `you.{\sg}.{\dat}' & kʰez& \armenian{քեզ} & kʰiz & \armenian{քիզ} & kʰez & \armenian{քեզ} \\
\hline 
 	\end{tabular}
\end{table}


\section{Morphology}
\subsection{Verb inflection or conjugation}
\subsubsection{Theme vowel changes}
In verbs, the 1SG and 1PL of present tense and of the tenses that are formed from the present have changed the vowel /e/ <\armenian{ե}> to /i/ <\armenian{ի}> (Table \ref{tab:Şebinkarahisar:phonology:changes:eHamshen}).\footnote{\translatorHD{The Classical forms are quite different from the modern forms because of different inflectional suffixes. For illustration, I approximate the change by referencing SWA, which is retains the Classical theme vowels. } }


\begin{table}[H]
 	\centering 
 	\caption{Change from theme vowel /e/ <\armenian{ե}> to /i/ <\armenian{ի}> in the Şebinkarahisar dialect}
 	\label{tab:Şebinkarahisar:phonology:eHamshen}
 	\begin{tabular}{|l|ll| ll|l|}
 		\hline & \multicolumn{2}{l|}{Şebinkarahisar} & \multicolumn{2}{l|}{cf. SWA} \\ 
`I want' &ɡʰ-uz-i-m & \armenian{գՙուզիմ} & ɡ-uz-e-m& \armenian{կ՚ուզեմ} & {\ind}-want-{\thgloss}-1{\sg}\\
`I will say' &bidi əs-i-m & \armenian{բիդի ըսիմ} & bidi əs-e-m& \armenian{պիտի ըսեմ} & {\fut} say-{\thgloss}-1{\sg}\\
`I would write letter' &ɡʰiɾ ɡʰɾ-i-m & \armenian{գՙիր գՙրիմ} & kʰiɾ kʰəɾ-e-m& \armenian{գիր գրեմ} & letter write-{\thgloss}-1{\sg}\\
\hline 
 	\end{tabular}
\end{table}

\subsubsection{Present/past tenses and the progressive}

The indicative present tense has two forms, as is found westward in all the Asia Minor dialects until Rodosto. This is the basic present (\armenian{բուն ներկայ}) and the progressive present (\armenian{շարունական ներկայ}). The first is the usual form of the present, which can also be used for the future: `I say', `I bring'.


The second is used when the action is being done at this exact time and it cannot at all have a future meaning: `I am liking'. The progressive present is found only in very few languages. For example, it is found in Ottoman Turkish /ɑl-əjoɾ-əm/ > `I am taking' (\armenian{ալըյօրըմ}) and /veɾ-ijoɾ-um/ `I am giving' (\armenian{վէրիյօրում}) (\translatorHD{Modern Turkish: <alıyorum>, <veriyorum>; the progressive marker is this suffix <iyor> which Adjarian transcribed as /ijoɾ/}),

\begin{adjarianpage}\label{page:175}\end{adjarianpage}% should be 175


... and also in English `I am living'.\footnote{\translatorHD{Adjarian incorrectly translated the English as `I am liking'.}} Persian also has a progressive present: <bé-xâham> `I want' \textarab{بخواهم} (\armenian{բը խահէմ) ՝ կուզեմ, ՝ կուզեմ կոր}. <mi-xâham> `I am wanting' \textarab{می‌خواهم} (\armenian{մի խահեմ}).\footnote{\translatorHD{The Persian terms are respectively subjunctive and indicative. It seems Adjarian interpreted the distinction in terms of progressiveness. }}


The indicative past imperfective also has simple and progressive forms (\ref{sent:Şebinkarahisar:morpho:verb:pastprog}). Compared with Ottoman Turkish /veɾiɾidim/ `I gave' (\armenian{վէրիրիտիմ}) vs. /veɾijorədəm/ `I was giving' (\armenian{վէրիյօրըտըմ}) (\translatorHD{Modern Turkish: <ver-di-m> and <ver-iyor-du-m>}).



\begin{exe}
 \ex SWA? dialect\label{sent:Şebinkarahisar:morpho:verb:pastprog}
 \begin{xlist}
 \ex\gll k-ut-e-i-$\emptyset$\\
 {\ind}-eat-{\thgloss}-1{\sg} \\
 \trans `I would eat.' \\
 \armenian{կուտէի}
 \ex\gll k-ut-e-i-$\emptyset$ koɾ \\
 {\ind}-eat-{\thgloss}-1{\sg} {\prog} \\
 \trans `I was eating.' \\
 \armenian{կուտէի կոր}
 \end{xlist}
\end{exe}


However, the progressive present and past imperfective aren't formed in the same way across all the dialects; instead, each dialect uses a different formative. For example, the Istanbul dialect uses /koɾ/ <\armenian{կոր}>, the Aslanbeg subdialect uses /h\'ɑje/ <\armenian{հա՛յէ}>, the Trabzon dialect uses /eɾ/ <\armenian{էր}>. The Şebinkarahisar dialect constructs its progressive forms with the formative /daɾ/ <\armenian{դար}> (i.e., /tɑɾ/ <\armenian{տար}> which has an unclear origin) (\ref{sent:Şebinkarahisar:morpho:verb:progReal}). 


\begin{exe}
 \ex Şebinkarahisar dialect\label{sent:Şebinkarahisar:morpho:verb:progReal}
 \begin{xlist}
 \ex\gll ɡ-əs-i-m dɑɾ \\
 {\ind}-say-{\thgloss}-1{\sg} {\prog} \\
 \trans `I am saying.' \\
 \armenian{գըսիմ դար}
 \ex\gll ɡ-əs-i-s dɑɾ \\
 {\ind}-say-{\thgloss}-2{\sg} {\prog} \\
 \trans `You are saying.' \\
 \armenian{գըսիս դար}
 \ex\gll ɡ-əs-e-i-$\emptyset$ dɑɾ \\
 {\ind}-say-{\thgloss}-{\pst}-2{\sg} {\prog} \\
 \trans `I was saying.' \\
 \armenian{գըսէի դար}
 \end{xlist}
\end{exe}

\section{Literature}

There is no other information on the Şebinkarahisar dialect, and there is no published manuscript. In Ani in the summer of 1907 (July 7-8), I got acquainted with an architect who was a native of Şebinkarahisar, Mr. T. Toromanian (\armenian{պր. Թ. Թորոմանեան}; \translatorHD{SEA/SWA: /tʰoɾɑmɑnjɑn/}). I requested from him that he accurately write a sample of this dialect. He gladly undertook my request and he wrote to me the following heartbreaking letter, which was rendered in my orthography, and which I present here completely.

\section{Text samples}

{\sampleoverview}

\subsection{Şebinkarahisar}

\armenian{Սէր իմ աղա ախբարս},

\armenian{Միխգիս դէրն էմ. խօսգիս հասդադ չի դՙըդվա. գՙիրըս շադ յ}̵\armenian{ուշացավ. իմմա̈՝ շիդագը օր բիդի ըսիմ նը ՝ շադ մըն ալ ղա̈բիյա̈թս (թրք}. qabahat \armenian{յանցանք) ձառը (ծանր) չէ. բՙանի գՙէօրձի մարթ էմ. յ}̵\armenian{իրինգուն-յ}̵\armenian{առավօդ դիղ մը դիդիգ արած չունիմ օր մէյ մը ղա̈լա̈մ-դիվիթ ա̈ռչէվս առնիմ հու էրգու սրա գՙիր գՙրիմ. չէ նը ինքիրէնս ամըն օր կըսէի քի՝ արաձըս աֆէգ է, խա̈թէր գօռէլը (կոտրելը) աղէգ բՙան չէ. խօշ դՙուն ան չեշիդը մարթիգնէրէն չէս. միխգս էրէսս չէս դար։ Իշդէ ասօր հըմար ալ է քի յէս ալ քիս հոգՙուս բէս գՙուզիմ. քիզ դիսաձ օրէս սիրդըս բՙացին՝ մէչը դՙրին։ Գՙիդի՞ս ինչու։ Է՜յ գիդի դղայութին}… \armenian{Յէս մէգ}... 



\begin{adjarianpage}\label{page:176}\end{adjarianpage}% should be 176


... \armenian{ընգէր մը ունէի մէր քաղաքը ըղաձ վախդըս. Փուռչուլին Բէօռէօս (Պետրոս) գըսէին. իմ շադ շադ յա̈րա̈նս էր. գՙիշէր ցօրէգ իրար հիդ էինք, չա̈լիգ-չօրթու բարաբար գը խաղայինք. շադ հիղ ալ էրգուսս վարբիդէն գը փախչէին գիրթայինք էքէսդանը թութ ու խնձոր ուդէլու։ Ասանգ աոխադաշէ (ընկեր) մը զադվէր էի քսան դարի է։ Քիզ օր դիսա ՝ ան սահաթը միդքըս ինգավ Բէօռէօսը, սիրդըս դիղէն թռավ. ինքիրէնս ըսի քի Ասվաձ Բէօռէօսըս գՙինա̈ ինձի դըվավ։ Յէս քու վրա ա̈սղըդար սէր դըվի իլլա}̈, \armenian{ըջըբ (արդեօք) դՙուն ալ իմ Բէօռէօսիս դիղը բիդի բՙռնէ՞ս. քանի քի զադվէր էմ՝ աշգիս արցունքը չի գըռրէցավ. միդքըս օր գՙուգՙա զադվաձ օրէրնիս՝ ձուխ ու մուխ գը գռիմ։ Հէչ միդքէս չիլլէր. օր մը, շափաթ էր դէք գիրա̈գի, յ}̵\armenian{առօդուն գանուխ էլէր դՙուռը նսդէր էր. գՙնացի քօվը, նայէցա օր դՙիդա̈արը վրա արէր՝ հիդըս խօրաթէլ բիլա̈ չուզէր դար. ըսի քի «Բէօ՛ռէօս, քալ՚ էրթանք քիշ մը խաղանք»։}

– \armenian{Չիմ իգՙար – ըսաց։}

– \armenian{Ինչո՞ւ, ի՞շ գա քի։}

– \armenian{Հէչ բՙան մ՚ալ չիգա։}

– \armenian{Հըբը ինչո՞ւ ադանգ դՙիդա̈րըդՙ գախէր էս ու հիդս ալ չէս խօրաթէր։}

– \armenian{Ի՞շ բիդի խօրաթիմ. զաթը քանի մը օր յէդքը ա̈միս զիս բիդի ղրգէ Սդամբօլ. ալ յա գը դէսնինք զիրար՝ յա չինք դէսնէր։}

– \armenian{Օղօրթմ՞ն գըսէս դար։}

– \armenian{Հըբը սուդմէ՞ն։}

– \armenian{Չիմ ավդար։}

– \armenian{Օր աշգօվըդՙ դէսնէս ՝ անվախդը գաւդաս։}

– \armenian{Ի՞նչ ըղավ օր ադանգ արավ քիզի ա̈միդՙ. մինչէվ հիմի հէչ ադանգ ձՙան մը չիգար։}

– \armenian{Ի՞նչ բիդի ըլլա. յ}̵\armenian{էրէգ յ}̵\armenian{իրինգուն վաժաբէդը մինդա̈արս ղօթլուխս դվավ, գՙնա վաժադան փարա բՙէր, չէ նը ա՛լ մի՛ գՙար ըսաց. ա̈միս ալ փարա չունէր օր դար, զաթը քանի մը հիղ ալ մարս գուլագէն օսգը գռից դվավ՝ դարի վաժադան փարա դվի. հիմի մարս ալ չունի. հարս ալ ըսաց քի շադ գՙարթա դէրա վարթաբէդ չի՛բդի ըլլա. ղրգիմ Սդամբօլ թօղ էրթա ախբօրը քօվը փարա վասդըդի։}


\begin{adjarianpage}\label{page:177}\end{adjarianpage}% should be 177


– \armenian{Է դՙո՞ւն ինչ ըսիր. գՙուղէ՞ս քի էրթաս։}

– \armenian{Ի՜նչ անիմ չէրթամ. ղաթրջին վաղը գՙուգՙա. ա̈միս աս գՙիշէր թա̈մբա̈լիթս (պայուսակ) բիդի գաբէ. վաղը ջին-զաբախդան ջՙամփա բիդի էլլինք։}

\armenian{Սուդ չէր Բէօռէօսը. յէս ալ գանուխ էլա. սիրդէս չէգավ օր էրթայի դէսնէի. հէռուն գայնէցա հու նա̈յէցա. օրն ալ ըսէս նը շադ գՙէշ օր էր. թաթավը վէրէն գՙուգՙա սիջիմի բէս. ջըխանք-ջըխջըխանք}… \armenian{բաղ բաղ փօրյազը գը փչէր։ Ջՙօրիին վրա բՙառցան թա̈մբա̈լիթը. զինքն ալ վրան նսդէցուցին ու դարին. մարը յ}̵\armenian{էդէվէն լացավ. «Յա՛վրում, ֆէղ բՙռնէս օսգի գըռի» ըսաց, հու նօրշըբա մը ջՙուր նէդէց յ}̵\armenian{էդէվէն։ Հարն ալ ջՙօրին հէդը գՙնաց ջՙամփէլու. յէս ալ գՙնացի նէրս, մութ դիղ մը մդա՝ լացի։ Ան է աս է՝ ա՛լ չի դէսա։ Ալ չի գիդիմ դէք սա՞ղ է դէ մէռաձ է. Ասվաձ անօր ալ բՙանին գէօրձին աջՙօղութին դա, իր սիլային հասցընէ. ձՙիզի ալ էրգան օրէր դա։}



\chapter{Trabzon}
\section{Background}
\begin{adjarianpage}\label{page:178}\end{adjarianpage}% should be 178

The dialect of Trabzon is spread across a small region, which contains only the cities of Trabzon, Gümüşhane, and Giresun. The last one is a migrant settlement from Trabzon. The surrounding villages of Trabzon don't speak this dialect, but instead speak the Hamshem dialect. In recent times, a sizable number of Armenians from Trabzon have migrated to the Caucasus and to the shores of the Black Sea. They primarily live in the cities of Batumi, Poti, Kerch, Sevastopol, Yalta. Because these aforementioned cities do not have a native Armenian population, and because the migrants from Trabzon form a sizable number, we have thus included them in the map as part of the region the Trabzon dialect 

\section{Literature}

There is no written study on the Trabzon dialect. There are likewise no manuscripts. In the summer of 1910, I stayed two weeks in Trabzon, and I determined that the Trabzon dialect is quite close to the Istanbul dialect, especially the Crimea dialect. 

\section{Phonology}
\subsection{Vowels}
The vowel system lacks the sounds /æ, i̯e, u̯o/ <\armenian{ա}̈, \armenian{ե, ո}>. Sometimes we find the sounds /œ, ʏ/ <\armenian{էօ, իւ}> (Table \ref{tab:Trabzon:phonology:ʏ}). 


\begin{table}[H]
 	\centering 
 	\caption{Emergence of /\armenian{իւ}/ <ʏ> in the Trabzon dialect}
 	\label{tab:Trabzon:phonology:ʏ}
 	\begin{tabular}{|l| ll|ll| ll|}
 		\hline & \multicolumn{2}{l|}{Classical Armenian} &\multicolumn{2}{l|}{> Trabzon} & \multicolumn{2}{l|}{cf. SEA} \\ 
 		`before' & mɑu̯ɾukʰ & \armenian{մաւրուք} & miɾʏkʰ & \armenian{միրիւք} & moɾukʰ & \armenian{մորուք} \\ 
\hline 
 	\end{tabular}
\end{table}

\subsection{Consonants}

The consonant system has greatly changed. The three degrees of Old Armenian have become two; the voiced and voiceless unaspirated have been confused together and have equally changed to voiced sounds. The voiceless aspirates stayed unchanged. This is the state of all other dialects of Asia Minor, starting from Evdokia until Crimea. In the Trabzon dialect, as well as in the Hamshen dialect, there is however a voiceless unaspirated sound \armenian{կ} /k/, which used instead of the <qaf> sound (\translatorHD{\textarab{ق} /q/}) sound for loanwords from Turkish. 

\subsection{Other sound changes}

We can say that there are no other sound changes in Trabzon, without of course taking into consideration the sound changes /ɑi̯/ to /ɑ/ (\armenian{այ}>\armenian{ա}), /oi̯/ to /u/ (\armenian{ոյ}>\armenian{ու}), /iu̯/ to /u/ (\armenian{իւ}>\armenian{ու}). In this way, the Trabzon dialect is one of the purest Armenian dialects. 


\begin{adjarianpage}\label{page:179}\end{adjarianpage}% should be 179

\section{Morphology}


In the grammar, there are the following notable points. 

\subsection{Noun or pronoun inflection or declension}
The case declensions and pronouns are the same as in the Istanbul dialect. For the latter, it is worth mentioning the words in Table \ref{tab:Trabzon:morphology:pronoun:sample}.

\begin{table}[H]
 	\centering 
 	\caption{Sample of pronouns in the Trabzon dialect}
 	\label{tab:Trabzon:morphology:pronoun:sample}
 	\begin{tabular}{|l ll|}
 		\hline 
 		proximal {\nom} {\sg} `this' &ɑsiviɡ, ɑsiɡ & \armenian{ասիվիգ, ասիգ} \\ 
 		medial {\nom} {\sg} `that' &ɑdiviɡ, ɑdiɡ & \armenian{ադիվիգ, ադիգ} \\ 
 		distal {\nom} {\sg} `that yonder' &ɑniviɡ, ɑniɡ & \armenian{անիվիգ, անիգ} \\ 
\hline 
 	\end{tabular}
\end{table}



\subsection{Verb inflection or conjugation}

{\paradigmExplanation}


\subsubsection{Indicative present and past imperfective}
For verb conjugation, the Classical theme vowels /e,ē/ <\armenian{ե,է}> become /i/ <\armenian{ի}> under stress (Table \ref{tab:Trabzon:morpho:verb:paradigm:presentIndc}).\footnote{\translatorHD{The Classical forms are quite different from the modern forms because of different inflectional suffixes. For illustration, I approximate the change by referencing SWA, which is retains the Classical theme vowels. } }

\translatorHD{Adjarian illustrates this change with the indicative present paradigm (Table \ref{tab:Trabzon:morpho:verb:paradigm:presentIndc}). Morphologically for an E-Class verb like /uz-e-l/ `to want', SWA forms this paradigm by adding the indicative prefix /ɡ(ə)-/ to the finite verb. The finite verb consists of the stem plus agreement suffixes after the theme vowel /e/. In Trabzon, the theme vowel is /e/ for the 3SG, but /i/ elsewhere. }


\begin{table}[H]
 	\centering 
 	\caption{Indicative present <\armenian{ներկայ}> of the verb `to want' in the Trabzon dialect}
 	\label{tab:Trabzon:morpho:verb:paradigm:presentIndc}
 \begin{tabular}{|l|ll| ll| }
 		\hline & \multicolumn{2}{l|}{Trabzon} & \multicolumn{2}{l|}{cf. SWA} \\ \hline 
1SG &ɡ-uz-i-m & \armenian{գուզիմ} & ɡ-uz-e-m& \armenian{կ՚ուզեմ} \\
2SG &ɡ-uz-i-s & \armenian{գուզիս} & ɡ-uz-e-s& \armenian{կ՚ուզես} \\
3SG &ɡ-uz-e-$\emptyset$ & \armenian{գուզէ} & ɡ-uz-e-$\emptyset$& \armenian{կ՚ուզէ} \\
1PL &ɡ-uz-i-nkʰ & \armenian{գուզինք} & ɡ-uz-e-ŋkʰ& \armenian{կ՚ուզենք} \\
2PL &ɡ-uz-i-kʰ & \armenian{գուզիք} & ɡ-uz-e-kʰ& \armenian{կ՚ուզէք} \\
3PL &ɡ-uz-i-n & \armenian{գուզին} & ɡ-uz-e-n& \armenian{կ՚ուզեն} \\
& \multicolumn{2}{l|}{{\ind}-$\sqrt{}$-{\thgloss}-{\agr}}& \multicolumn{2}{l|}{{\ind}-$\sqrt{}$-{\thgloss}-{\agr}}\\
\hline 
\end{tabular}
\end{table}

\translatorHD{In the past imperfective (Table \ref{tab:Trabzon:morpho:verb:paradigm:pastImpfIndc}), SWA adds the past suffix between the theme vowel and the agreement suffix. The past suffix is covert for the 3SG, but /-i-/ elsewhere. The theme vowel is /e/. In Trabzon, the theme vowel is /-i-/ in the 3SG, and /-e-/ elsewhere. The past suffix is covert in the 3SG, and /-i-/ elsewhere. However, the data is rather limited, so it is possible that the 3SG /-i-/ is actually the past suffix, and that the theme vowel is exceptionally covert.}


\begin{table}[H]
 	\centering 
 	 \caption{Indicative past imperfective <\armenian{անկատար}> of the verb `to want' in the Trabzon dialect}\label{tab:Trabzon:morpho:verb:paradigm:pastImpfIndc}
 \begin{tabular}{|l|ll| ll| }
 		\hline & \multicolumn{2}{l|}{Trabzon} & \multicolumn{2}{l|}{cf. SWA} \\ \hline 
1SG&ɡ-uz-e-i-$\emptyset$ & \armenian{գուզէի} & ɡ-uz-ej-i-$\emptyset$& \armenian{կ՚ուզէի} \\
2SG&ɡ-uz-e-i-ɾ & \armenian{գուզէիր} & ɡ-uz-ej-i-ɾ& \armenian{կ՚ուզէիր} \\
3SG&ɡ-uz-i-$\emptyset$-ɾ & \armenian{գուզիր} & ɡ-uz-e-$\emptyset$-ɾ& \armenian{կ՚ուզէր}\\
& \multicolumn{2}{l|}{{\ind}-$\sqrt{}$-{\thgloss}-{\pst}-{\agr}}& \multicolumn{2}{l|}{{\ind}-$\sqrt{}$-{\thgloss}-{\pst}-{\agr}}\\
\hline 
\end{tabular}
\end{table}

\subsubsection{Past perfective or aorist}

\translatorHD{The past perfective (Table \ref{tab:Trabzon:morpho:verb:paradigm:pastperfectiveAorist}) is also called the aorist. In SWA for /jepʰ-e-l/ `to cook', the past perfective is formed by taking the root and theme vowel, adding the aorist or perfective suffix /-t͡sʰ-/, and then adding the past suffix /-i/ and the appropriate agreement suffixes. The 3SG uses covert tense and agreement suffixes. The Trabzon subdialect behaves the same. However, whereas the theme vowel is a constant /-e-/ in SWA, the theme vowel in Trabzon is /-i-/ for the 3SG, and /-e-/ elsewhere. }


\begin{table}[H]
 \centering
 \caption{Past perfective or aorist <\armenian{կատարեալ}> of the verb `to want' in the Vozim subdialect of the Van dialect}
 \label{tab:Trabzon:morpho:verb:paradigm:pastperfectiveAorist}
 \begin{tabular}{|l|ll|ll|}
\hline & \multicolumn{2}{l|}{Trabzon} & \multicolumn{2}{l|}{cf. SWA} \\\hline
1SG&epʰ-e-t͡sʰ-i-$\emptyset$ & \armenian{էփէցի} & jepʰ-e-t͡sʰ-i-$\emptyset$& \armenian{եփեցի} \\
2SG&epʰ-e-t͡sʰ-i-ɾ & \armenian{էփէցիր} & jepʰ-e-t͡sʰ-i-ɾ& \armenian{եփեցիր} \\
3SG &epʰ-i-t͡sʰ-$\emptyset$-$\emptyset$ & \armenian{էփից} & jepʰ-e-t͡sʰ-$\emptyset$-$\emptyset$ & \armenian{եփեց} \\
& \multicolumn{2}{l|}{$\sqrt{}$-{\thgloss}-{\aor}-{\pst}-{\agr}}& \multicolumn{2}{l|}{$\sqrt{}$-{\thgloss}-{\aor}-{\pst}-{\agr}}\\ 

\hline 
\end{tabular}
\end{table}

\subsubsection{Present perfect and past perfect}

\translatorHD{In SWA, the present perfect (Table \ref{tab:Trabzon:morpho:verb:paradigm:presentPerfect}) and past perfect (Table \ref{tab:Trabzon:morpho:verb:paradigm:pastPerfect}) in are formed by combining a special non-finite form with the present/past auxiliary. For SWA, this non-finite verb can be either the resultative participle (verb with suffix /-ɑd͡z/) or the evidential participle (verb with suffix /-eɾ/). Trabzon uses a similar system. The non-finite form is labeled as just a `past participle' by Adjarian (which I suspect is a perfective converb), and this form uses /-iɾ/ <\armenian{իր}>. }

\begin{table}[H]
 \centering
 \caption{Present perfect <\armenian{յարակատար}> of the verb `to like' in the Trabzon dialect}
 \label{tab:Trabzon:morpho:verb:paradigm:presentPerfect}
 \begin{tabular}{|l|ll|ll|}
\hline & \multicolumn{2}{l|}{Trabzon} & \multicolumn{2}{l|}{cf. SWA} \\\hline 
1SG & siɾ-iɾ i-m & \armenian{սիրիր իմ} & siɾ-eɾ e-m & \armenian{սիրեր եմ} \\
& \multicolumn{2}{l|}{$\sqrt{}$-{\perfcvb} {\aux}-{\agr}}& \multicolumn{2}{l|}{$\sqrt{}$-{\eptcp} {\aux}-{\agr}}\\ 

\hline 
\end{tabular}
\end{table}


\begin{table}[H]
 \centering
 \caption{Past perfect <\armenian{գերակատար}> of the verb `to bring' in the Trabzon dialect}
 \label{tab:Trabzon:morpho:verb:paradigm:pastPerfect}
 \begin{tabular}{|l|ll|ll| }
\hline & \multicolumn{2}{l|}{Trabzon} & \multicolumn{2}{l|}{cf. SWA} \\\hline 
1SG & beɾ-iɾ e-i-$\emptyset$ & \armenian{բէրիր էի}& beɾ-eɾ e-i-$\emptyset$ & \armenian{բերեր էի} \\
& \multicolumn{2}{l|}{$\sqrt{}$-{\perfcvb} {\aux}-{\pst}-{\agr}}& \multicolumn{2}{l|}{$\sqrt{}$-{\eptcp} {\aux}-{\pst}-{\agr}}\\ 

\hline 
\end{tabular}
\end{table}

\translatorHD{Based on this small paradigm, it seems that the auxiliary is /i/ in the present, but /e/ in the past. }



\subsubsection{Indicative morpheme as a mobile morpheme}

The present formative is /ɡ/ <\armenian{գ}> for vowel-initial verbs, while a postposed formative /ɡu/ <\armenian{գու}> for consonant-initial verbs. \translatorHD{To clarify, he means that whereas the indicative morpheme is a prefix /ɡ(ə)-/ in SWA, this morpheme is a mobile morpheme in Trabzon (Table \ref{tab:Trabzon:morpho:verb:mobileIndc}).}


\begin{table}[H]
 	\centering 
 	\caption{Mobile indicative morpheme in the Trabzon dialect}
 	\label{tab:Trabzon:morpho:verb:mobileIndc}
 \begin{tabular}{|l|ll| ll|l| }
 		\hline & \multicolumn{2}{l|}{Trabzon} & \multicolumn{2}{l|}{cf. SWA} \\ \hline 
`I want' &ɡ-uz-i-m & \armenian{գուզիմ} & ɡ-uz-e-m& \armenian{կ՚ուզեմ} \\
& \multicolumn{2}{l|}{{\ind}-$\sqrt{}$-{\thgloss}-{\agr}}& \multicolumn{2}{l|}{{\ind}-$\sqrt{}$-{\thgloss}-{\agr}}\\
\hline 
`I like' &siɾ-i-m ɡu & \armenian{սիրիմ գու} & ɡə-siɾ-e-m& \armenian{կը սիրեմ} \\
`he likes' &siɾ-e-$\emptyset$ ɡu & \armenian{սիրէ գու} & ɡə-siɾ-e-$\emptyset$& \armenian{կը սիրէ} \\
& \multicolumn{2}{l|}{$\sqrt{}$-{\thgloss}-{\agr} {\ind}}& \multicolumn{2}{l|}{{\ind}-$\sqrt{}$-{\thgloss}-{\agr}}\\
\hline 
`we would look' &nɑj-e-i-nkʰ ɡu & \armenian{նայէինք գու} & ɡə-nɑj-e-i-ŋkʰ& \armenian{կը նայէինք} \\
& \multicolumn{2}{l|}{$\sqrt{}$-{\thgloss}-{\pst}-{\agr} {\ind}}& \multicolumn{2}{l|}{{\ind}-$\sqrt{}$-{\thgloss}-{\pst}-{\agr}}\\
\hline 
\end{tabular}
\end{table}



\subsubsection{Progressive tenses}

The progressive is built with the postposed formatives /eɾ/ <\armenian{էր}> or /uni/ <\armenian{ունի}>. The present takes /eɾ/ <\armenian{էր}>, the imperfect takes /uni/ <\armenian{ունի}>. Vowel–initial verbs also take the prefix /ɡ/ <\armenian{գ}>. 

\translatorHD{To clarify, whereas spoken SWA uses a progressive marker /ɡoɾ/, Trabzon uses either /eɾ/ or /uni/ based on tense. The indicative morpheme is a fixed prefix in SWA, but this prefix is only used for vowe-initial verbs in Trabzon. }

\translatorHD{For the present progressive, SWA uses both an indicative prefix /ɡ(ə)-/ and a progressive marker /ɡoɾ/. Compared across the paradigms, the present progressive is just the indicative present plus this progressive marker. But for Trabzon, the present progressive is the indicative present plus the progressive marker /eɾ/. The indicative prefix /ɡ/ is retained before vowel-initial verbs (Table \ref{tab:Trabzon:morpho:verb:paradigm:presProgV}), but the indicative suffix /ɡu/ (for consonant-initial verbs) is absent (Table \ref{tab:Trabzon:morpho:verb:paradigm:presProgC}). }


\begin{table}[H]
 	\centering 
 	\caption{Present progressive <\armenian{ներկայ շարոնւական}> of the verb `to take' in the Trabzon dialect}
 	\label{tab:Trabzon:morpho:verb:paradigm:presProgV}
 \begin{tabular}{|l|ll| ll| }
 		\hline & \multicolumn{2}{l|}{Trabzon} & \multicolumn{2}{l|}{cf. SWA} \\ \hline 
1SG &ɡ-ɑr-n-i-m & \armenian{գառնիմ էր} & ɡ-ɑɾ-n-e-m ɡoɾ& \armenian{կ՚առնեմ կոր} \\
2SG &ɡ-ɑr-n-i-s & \armenian{գառնիս էր} & ɡ-ɑɾ-n-e-s ɡoɾ& \armenian{կ՚առնես կոր} \\
3SG &ɡ-ɑr-n-e-$\emptyset$ & \armenian{գառնէ էր} & ɡ-ɑɾ-n-e-$\emptyset$ ɡoɾ& \armenian{կ՚առնէ կոր} \\
1PL &ɡ-ɑr-n-i-nkʰ & \armenian{գառնինք էր} & ɡ-ɑɾ-n-e-ŋkʰ ɡoɾ& \armenian{կ՚առնենք կոր} \\
2PL &ɡ-ɑr-n-i-kʰ & \armenian{գառնիք էր} & ɡ-ɑɾ-n-e-kʰ ɡoɾ& \armenian{կ՚առնէք կոր} \\
3PL &ɡ-ɑr-n-i-n & \armenian{գառնին էր} & ɡ-ɑɾ-n-e-n ɡoɾ& \armenian{կ՚առնեն կոր} \\
& \multicolumn{2}{l|}{{\ind}-$\sqrt{}$-{\vx}-{\thgloss}-{\agr} {\prog}} & \multicolumn{2}{l|}{{\ind}-$\sqrt{}$-{\vx}-{\thgloss}-{\agr} {\prog}}\\
\hline 
\end{tabular}
\end{table}


\begin{table}[H]
 	\centering 
 	\caption{Present progressive <\armenian{ներկայ շարոնւական}> of the verb `to like' in the Trabzon dialect}
 	\label{tab:Trabzon:morpho:verb:paradigm:presProgC}
 \begin{tabular}{|l|ll| ll| }
 		\hline & \multicolumn{2}{l|}{Trabzon} & \multicolumn{2}{l|}{cf. SWA} \\ \hline 
1SG &siɾ-i-m & \armenian{սիրիմ էր} & ɡ-ɑɾ-n-e-m ɡoɾ& \armenian{կը սիրեմ կոր} \\
2SG &siɾ-i-s & \armenian{սիրիս էր} & ɡ-ɑɾ-n-e-s ɡoɾ& \armenian{կը սիրես կոր} \\
3SG &siɾ-e-$\emptyset$ & \armenian{սիրէ էր} & ɡ-ɑɾ-n-e-$\emptyset$ ɡoɾ& \armenian{կը սիրէ կոր} \\
1PL &siɾ-i-nkʰ & \armenian{սիրինք էր} & ɡ-ɑɾ-n-e-ŋkʰ ɡoɾ& \armenian{կը սիրենք կոր} \\
2PL &siɾ-i-kʰ & \armenian{սիրիք էր} & ɡ-ɑɾ-n-e-kʰ ɡoɾ& \armenian{կը սիրէք կոր} \\
3PL &siɾ-i-n & \armenian{սիրին էր} & ɡ-ɑɾ-n-e-n ɡoɾ& \armenian{կը սիրեն կոր} \\
& \multicolumn{2}{l|}{{\ind}-$\sqrt{}$-{\thgloss}-{\agr} {\prog}} & \multicolumn{2}{l|}{{\ind}-$\sqrt{}$-{\thgloss}-{\agr} {\prog}}\\
\hline 
\end{tabular}
\end{table}

\translatorHD{For the past progressive, SWA adds the progressive marker /ɡoɾ/ to the indicative present. The indicative prefix /ɡ(ə)-/ stays. For Trabzon, the progressive marker is instead /uni/. The indicative morpheme is retained as a prefix for vowel-initial verbs (Table \ref{tab:Trabzon:morpho:verb:paradigm:pastProgV}), but absent for consonant-initial verbs (Table \ref{tab:Trabzon:morpho:verb:paradigm:pastProgC}). }



\begin{table}[H]
 	\centering 
 	\caption{Present progressive <\armenian{անկատար շարոնւական}> of the verb `to cook' in the Trabzon dialect}
 	\label{tab:Trabzon:morpho:verb:paradigm:pastProgV}
 \begin{tabular}{|l|ll| ll| }
 		\hline & \multicolumn{2}{l|}{Trabzon} & \multicolumn{2}{l|}{cf. SWA} \\ \hline 
1SG &ɡ-epʰ-e-i-$\emptyset$ uni & \armenian{գէփէի ունի} & ɡ-epʰ-ej-i-$\emptyset$ ɡoɾ& \armenian{կ՚եփէի կոր} \\
2SG &ɡ-epʰ-e-i-ɾ uni& \armenian{գէփէիր ունի} & ɡ-epʰ-ej-i-ɾ ɡoɾ& \armenian{կ՚եփէիր կոր} \\
3SG &ɡ-epʰ-i-$\emptyset$-ɾ uni & \armenian{գէփիր ունի}&ɡ-epʰ-ej-$\emptyset$-ɾ ɡoɾ& \armenian{կ՚եփէր կոր} \\
1PL &ɡ-epʰ-e-i-nkʰ uni & \armenian{գէփէինք ունի} & ɡ-epʰ-ej-i-ŋkʰ ɡoɾ& \armenian{կ՚եփէինք կոր} \\
2PL &ɡ-epʰ-e-i-kʰ uni & \armenian{գէփէիք ունի} & ɡ-epʰ-ej-i-kʰ ɡoɾ& \armenian{կ՚եփէիք կոր} \\
3PL &ɡ-epʰ-e-i-n uni & \armenian{գէփէին ունի} & ɡ-epʰ-ej-i-n ɡoɾ& \armenian{կ՚եփէին կոր} \\
& \multicolumn{2}{l|}{{\ind}-$\sqrt{}$-{\thgloss}-{\pst}-{\agr} {\prog}} & \multicolumn{2}{l|}{{\ind}-$\sqrt{}$-{\thgloss}-{\pst}-{\agr} {\prog}}\\
\hline 
\end{tabular}
\end{table}


\begin{table}[H]
 	\centering 
 	\caption{Present progressive <\armenian{անկատար շարոնւական}> of the verb `to cook' in the Trabzon dialect}
 	\label{tab:Trabzon:morpho:verb:paradigm:pastProgC}
 \begin{tabular}{|l|ll| ll| }
 		\hline & \multicolumn{2}{l|}{Trabzon} & \multicolumn{2}{l|}{cf. SWA} \\ \hline 
1SG &nɑj-e-i-$\emptyset$ uni & \armenian{նայէի ունի} & ɡ-epʰ-ej-i-$\emptyset$ ɡoɾ& \armenian{կը նայէի կոր} \\
2SG &nɑj-e-i-ɾ uni& \armenian{նայէիր ունի} & ɡ-epʰ-ej-i-ɾ ɡoɾ& \armenian{կը նայէիր կոր} \\
3SG &nɑj-i-$\emptyset$-ɾ uni & \armenian{նայիր ունի}&ɡ-epʰ-ej-$\emptyset$-ɾ ɡoɾ& \armenian{կը նայէր կոր} \\
1PL &nɑj-e-i-nkʰ uni & \armenian{նայէինք ունի} & ɡ-epʰ-ej-i-ŋkʰ ɡoɾ& \armenian{կը նայէինք կոր} \\
2PL &nɑj-e-i-kʰ uni & \armenian{նայէիք ունի} & ɡ-epʰ-ej-i-kʰ ɡoɾ& \armenian{կը նայէիք կոր} \\
3PL &nɑj-e-i-n uni & \armenian{նայէին ունի} & ɡ-epʰ-ej-i-n ɡoɾ& \armenian{կը նայէին կոր} \\
& \multicolumn{2}{l|}{$\sqrt{}$-{\thgloss}-{\pst}-{\agr} {\prog}} & \multicolumn{2}{l|}{{\ind}-$\sqrt{}$-{\thgloss}-{\pst}-{\agr} {\prog}}\\
\hline 
\end{tabular}
\end{table}


\subsubsection{Other mobile morphemes}

As can be seen, the verbal formatives ({\ind} /ɡu/ <\armenian{գու}>, {\prog} /eɾ, uni/ <\armenian{էր, ունի}>) are generally postposed. This postponement can also be done in the future and the negative.

\translatorHD{For the future, SWA simply combines the future morpheme /bidi/ with the finite verb. This future morpheme is a proclitic. In Trabzon, the future morpheme can go on either side of the verb (\ref{sent:Trabzon:morpho:verb:fut}). }

\begin{exe}
\ex \begin{xlist}
 \ex Trabzon dialect \label{sent:Trabzon:morpho:verb:fut}
 \begin{xlist}
 \ex \gll bidi uz-i-m \\
 {\fut} want-{\thgloss}-1{\sg}\\
 \trans `I will want.' 
 \armenian{բիդի ուզիմ} 
 \ex \gll uz-im bidi \\
 want-{\thgloss}-1{\sg} {\fut} \\
 \trans `I will want.' 
 \armenian{ուզիմ բիդի}
 \end{xlist}
 \ex cf. SWA \gll bidi uz-e-m \\
 {\fut} want-{\thgloss}-1{\sg}\\
 \trans `I will want.' 
 \armenian{պիտի ուզեմ} 
\end{xlist}
\end{exe}

\translatorHD{For negated verbs, a negated present verb uses a periphrastic construction of the negative auxiliary plus a non-finite verb. The auxiliary carries tense-agreement. The auxiliary is before the verb. The non-finite verb has a suffix /-ɾ/ (the connegative) after the theme vowel. For Trabzon, the auxiliary can go before or after the verb 
(\ref{sent:Trabzon:morpho:verb:neg}). }

\begin{exe}
\ex \begin{xlist}
 \ex Trabzon dialect \label{sent:Trabzon:morpho:verb:neg}
 \begin{xlist}
 \ex \gll t͡ʃʰ-i-s ɡɾ-i-ɾ \\
 {\neggloss}-{\aux}-2{\sg} write-{\thgloss}-{\cn}\\
 \trans `You don't write.'
\armenian{չիս գրիր} 
\ex \gll ɡɾ-i-ɾ t͡ʃʰ-i-s \\
 write-{\thgloss}-{\cn} {\neggloss}-{\aux}-2{\sg} \\
 \trans `You don't write.'
\armenian{գրիր չիս}
 \end{xlist}
 \ex cf. SWA t͡ʃʰ-e-s kʰəɾ-e-ɾ \\
 {\neggloss}-{\aux}-2{\sg} write-{\thgloss}-{\cn}\\
 \trans `You don't write.'
\armenian{չես գրեր} 
\end{xlist}
\end{exe}

\subsubsection{Repetition of agreement in negation}

When forming the negative, the conjugation of the participle is also interesting. \translatorHD{In SWA, negated present verbs are made up of a finite negative auxiliary plus a non-finite verb. Agremeent is strictly on the auxiliary. But in Trabzon, it seems that agreement can be on both the negative auxiliary and the verb 
(\ref{sent:Trabzon:morpho:verb:negRepet}). }

\begin{exe}
\ex \label{sent:Trabzon:morpho:verb:negRepet}
\begin{xlist}
\ex `I don't come' 
\begin{xlist}
 \ex Trabzon dialect 
 \gll t͡ʃʰ-i-m kʰ-ɑ-m \\
 {\neggloss}-{\aux}-1{\sg} come-{\thgloss}-1{\sg} \\
 \trans \armenian{չիմ քամ}
 \ex cf. SWA
 \gll t͡ʃʰ-e-m kʰ-ɑ-ɾ \\
 {\neggloss}-{\aux}-1{\sg} come-{\thgloss}-{\cn} \\
 \trans \armenian{չեմ գար}
\end{xlist}
\ex `I don't want' 
\begin{xlist}
 \ex Trabzon dialect 
 \gll t͡ʃʰ-i-m uz-i-m \\
 {\neggloss}-{\aux}-1{\sg} want-{\thgloss}-1{\sg} \\
 \trans \armenian{չիմ քամ}
 \ex cf. SWA
 \gll t͡ʃʰ-e-m uz-e-ɾ \\
 {\neggloss}-{\aux}-1{\sg} want-{\thgloss}-{\cn} \\
 \trans \armenian{չեմ ուզեր}
\end{xlist} 
\ex `We don't employ' 
\begin{xlist}
 \ex Trabzon dialect 
 \gll t͡ʃʰ-i-nkʰ pʰɑne-t͡sʰun-i-nkʰ \\
 {\neggloss}-{\aux}-1{\pl} work-{\caus}-{\thgloss}-1{\pl} \\
 \trans \armenian{չինք բանէցունինք} 
 \ex cf. SWA
 \gll t͡ʃʰ-e-nkʰ pʰɑne-t͡sʰən-e-ɾ \\
 {\neggloss}-{\aux}-1{\sg} work-{\caus}-{\thgloss}-{\cn} \\
 \trans \armenian{չենք բանեցներ} 
\end{xlist} 
\end{xlist}
\end{exe}

Or the repetition of the copula as in the Bayazit subdialect. \translatorHD{He means that another option is that the verb stays non-finite. The verb is preceded by a finite negative auxiliary, and followed by a finite positive auxiliary (\ref{sent:Trabzon:morpho:verb:negRepetCop}).}


\begin{exe}

\ex `We don't know' \label{sent:Trabzon:morpho:verb:negRepetCop}
\begin{xlist}
 \ex Trabzon dialect 
 \gll t͡ʃʰ-i-nkʰ im-ɑ-t͡sʰ-iɾ i-nkʰ\\
 {\neggloss}-{\aux}-1{\pl} know-{\lvgloss}-{\aor}-{\perfcvb} {\aux}-1{\pl} \\
 \trans \armenian{չինք իմացիր ինք}
 \ex cf. SEA
 \gll t͡ʃʰ-e-ŋkʰ im-ɑ-t͡sʰ-el \\
 {\neggloss}-{\aux}-1{\pl} know-{\lvgloss}-{\aor}-{\perfcvb} \\
 \trans \armenian{չենք իմացել}
\end{xlist}

\end{exe}

\section{Subdialects}
Gümüşhane and Giresun also have the formative /eɾ/ <\armenian{էր}> for forming the progressive. Gümüşhane forms the simple present by using a postposed /ɡə/ <\armenian{գը}> (\ref{sent:Trabzon:morpho:verb:Gümüşhane:prog}), with which it forms a middle zone between Karin and Trabzon. 


\begin{exe}

\ex `He cleans' \label{sent:Trabzon:morpho:verb:Gümüşhane:prog}
\begin{xlist}
 \ex Gümüşhane subdialect of the Trabzon dialect 
 \gll mɑkʰɾ-e-$\emptyset$ ɡə\\
 clean-{\thgloss}-3{\sg} {\ind} \\
 \trans \armenian{մաքրէ գը}
 \ex cf. SWA
 \\
 ɡə-mɑkʰɾ-e-$\emptyset$ \\
 {\ind}-clean-{\thgloss}-3{\sg}
 \armenian{կը մաքրէ}
\end{xlist}
\end{exe}


\begin{adjarianpage}\label{page:180}\end{adjarianpage}% should be 180


\section{Text samples}

{\sampleoverview}



Adjarian's source: Written by my philologist friend, a teacher from Trabzon, Mr. Nshan Khetshian (\armenian{պր. Նշան Խտշեան}; \translatorHD{SEA/SWA: /nəʃɑn χətʃjɑn/}). I have rendered the orthography into the scientific one. 


– \armenian{Բար իրիգուն։}

– \armenian{Ասձու բարին, Լուսիա հանում, բարօվ էգաք. հրամմէցէք։}

– \armenian{Քա՛ Համաս, Օննիզիդ մէնձ օրը շնավօր ըլլա, աբրի մնա, խէրե դէսնիս, Ասվաձ օջախիդ բաշխէ։}

– \armenian{Շինօրագալ իմ, օխջ ըլլաս. Ասվաձ քուգիննէրըդ ալ քէզի բաշխէ. հրամմէցէք վէր։ Ագ թաքու, էգէ Լուսիա հանումին չարսաֆը վա՛ռ առ գուլխէն։}

– \armenian{Քա՛ Համաս, շիդագը գուզիս նա, յէս ասօր դէղէս ժաժվէլու վախըթ չունէի. գիդի՞ս յա, էրդուշաբդի օրը էլա վըլացք էրի, իրէքշաբդի բուղաթայէն հանէցի, չօրէքշաբդի գարէցի զարգըդէցի, ասօր ալ էլա բաղնիգ քնացի քի՝ օսքօռնէրըս քիչ մէ դաքնան. լաքին քու անուշիգ խաթէրըդ համար չի գրցա համփէրիր էի. ըսի լի ի՛նչ գըլլա նա՝ թօ՛ղ ըլլա. մէռնէլըս ալ գիդնամ՝ յէս ասօր Համաս հանումին օրթուն դօնին բիդի էրթամ. չունքի դուն աղէգ գիդիս օր յէս Օննիգը իմ զավգիս բէս սիրիմ գու։}

– \armenian{Օխջ ըլլաս, անիվիգ ալ քէզի մօր բէս սիրէ գու. դունը ձէրն է. հէլբէդ բիդի քայիր. զաթի աչգըս չօրս բացաձ ջանփադ նայէի ունի. ամմա էղէր չի քայիր յա՝ խա՛չ օր իմ ձէռքէս ինդօ՞ր բիդի խալըսէիր։}

– \armenian{Օ՛ֆ, նէֆէսս դդրէցավ. ձունգվընէրուս քօվ հիչ հօքի մնացաձ չէ. դիյ օր հօս էգա նա՝ հօքիս բէրանըս էգավ. քրդինքնէրու մէչ մնացի։ Է՛հ, դահա ի՞նդօր իք նայիմ. աղի՞գ իք։}

– \armenian{Ինդօր բիդի ըլլանք. մէխգօվնիս դանչըվինք էր։}

– \armenian{Սէրքիս աղայէն ի՞նչ խաբար. նամակ՝ բան մը գառնի՞ք էր մի։}

– \armenian{Սէբէ քի ամմէն շարդու գառնինք էր. հարցունօղնէրուն ամմէնքին ալ բարէվ գրած է։}

– \armenian{Բարին ղրգօղ բէրօզին արէվըն ըլլա. դուն ալ նամագ}... 


\begin{adjarianpage}\label{page:181}\end{adjarianpage}% should be 181

... \armenian{գրէլու ըլլաս նա՝ ինձմէն շադ շադ բարէվ գրէ. օրթուն խէրը դէսնէ. Ասվաձ օջախին բաշխէ։}

– \armenian{Գուլխուս վրա. մէղէր ըսէլօվըդ գրէի բիդի։}

– \armenian{Գլօխըդ բարցին վրա}. – \armenian{է՛հ, դահա ի՞նչ գա, ի՞նչ չիգա. դուսէն նէսէն խաբար՝ բան մը գառնի՞ք էր։ Քէզի նօր խաբար մը դամ բիդի ամմա՝ չուքդիմ քի իմացա՞ձ իք մի։}

– \armenian{Ի՞նչ է, քա՜. ըսէ նայիմ. մէնք բանէ մը խաբար չունինք. դունէն դուս էլաձ չունիմ քի բան մալ իմանամ։}

– \armenian{Անցաձն օրդանքը Հաջի Ղասիմէն դէրվէր գէրթայի ունի, նիրօչս մէնձ հարսը դէմըս էլավ. անգից իմացա քի՝ ղազանջի Արութէնին մանչուն Նիշանը յէդ էրիր ին։}

– \armenian{Քա՛, հիմագ խէլքիս քուքամ. յէ՞փ էղավ ադիգ. զահէր մէնք աս քաղքէն չէինք. հիչ բան մալ չինք իմացիր ինք. սէբէբը ի՞նչ է աջաբա. Նիշանը ախջիգան Թարաֆէ՞ն յէդ էրիր ին մի՝ չէ նա մանչուն։}

– \armenian{Ախջիգան թարաֆէն յէդ էրիր ին. սէբէբը գէօյա մանչը խում շադ խմէ գու էղիր. ամմէն իրինդուն քէօռ-գինօվ դուն գէրթա էղիր. վասդըգաձը, դադաձը բիւթիւն խումի գուդա էղիր. ամմէն իրինգուն դանը մէչ ձէձ-փէդ գըլլա էղիր. դահա թախում մը դէդիմ-դէդի խօսգէր. վօ՞ր մէգը ըսիմ։ Ամա խօսգը մէչէրնիս, քէզի բան մը զուրցի՞մ, քա Հա՛մաս. Նիշանը յէդ էնէլնին շադ խաս էղավ. անանգ գինօվի մը դալու իսա՝ ջիդը չուվան մը թօղ ցքին դէն՝ դանին ձօվը նէդին։}

– \armenian{Խօսգըդ մէրղօվ գդրէցի. առաչ խէլքէրնին վո՞ւր դէղն էր. անօր ինչ ձաղիգ ըլլալը չուքդէի՞ն մի. քառսուն դուռ զարգավ. քառսուն գէղէ ախջիգ ուզից, լաքին հիչ մէգն ալ վրան չի թուքավ։ Հէր նէ իսա, յէս խօշլանմիշ էղա աս բանէն. ախջիգը խաս դղա է. գօդէսբանա (յն. տանտիկին}), \armenian{գարօղ գարգըդօղ. դանը մէչ դիյօր ինինգուն ֆըռըլ-ֆըռըլ դառնա գու։ Ասվաձ հէլբէթդէ բաշխա խսմէթ մը հանէ գու դէմը։ Տէլիք բօնջուխ յէրդէ կալմազ։}

– \armenian{Հա՛յ, հա՛յ, դուն ջէնջ ունէցիր, ջանչը Բաղդադէն քուքա։}

– \armenian{Հրամմէ՛, անուշ, կօնյաք առ, Լուսիա հանում։}

– \armenian{Էհ, Օննիգիդ մէնձ օրը շնավօր ըլլա. աբրի մնա. խէրը դէսնիս. ամմէն դարի աս օրէրուն հասնի. թաքն ու բսագը դէսնիմ. մազը-միրիւքը ջէրմըգի։}

\begin{adjarianpage}\label{page:182}\end{adjarianpage}% should be 182


– \armenian{Անո՛ւշ հրամմէցիր։}

– \armenian{Անուշօվ մնաս։ Քա՛, աս ի՞նչ համօվ բան էր. դունէ՞ն մի էրիր իք աս անուշը, չէնա մի գուսէն ձախու առիր իք։}

– \armenian{Քա՛ մէղա, դուսէն ալ ի՞նչէն բիդի առնէի. Թաքուհիձաս էփից։}

– \armenian{Ո՛ւյ, մադվընէրը սիրիմ յէս անօր. էրնէգ քէզի օր ասանգ անգին զավագ ունիս։ Քա, քանի՞ դարու էղավ։}

– \armenian{Սուրբ Սէրքիսին բահօց շարդուն դասնըօխդը բիդի թամնէ՝ դասնըվութը բիդի մդնի։ Անօր աշղարք էգաձ օրը մէր դրացի հաջի Ուսգուն իր օրթին զարքիր ունի. ասօրվան բէս դահա միդգս է. մէզի ալ հարսնիգ գանչիր էր, լաքին ան իրիգունը իմ ցավըս բռնէլուն սէբէբօվը չիգրցի էրթալ։ Ա՛խ, Լուսիա հանում, ինչէ՜ր քաշէցի ան վախթը. թէմամ էռսունխիրէք օր լօխուսա բառգէցա. շադէրը ըսին քի Համասին հալը հալ չէ. հա՛ մէռնի էր, հա՛ մէռնի բիդի. է՜հ, մէռնի՛մ Ասձու աչիչը. դահա խմէլիք ջուրէրնիս չէ հադիր։}

– \armenian{Ի՛նչ ղօլայ է մէռնէլը, հէլէ գէցիր նայիմ. մէյ մը Թաքուհիձադ ամնէ, Օննիգըդ ալ օդգը գլօխ էրէ դէն, անգից յէդգը ի՛նչ գուզիս նա էղիր։ Օղօրմաձ հօքի գէսուրըս գըսէր քի՝ «Մարթուն ըսաձը չըլլար, Ասձու ըսաձը գըլլա»։}

– \armenian{Անանգ է. ջագադնիս ի՛նչ գրվաձ է նա՝ ան գըլլա. հրամմէ, ղայֆէդ ա՛ռ։}

– \armenian{Շինօրագալ իմ. բարէ իդգէց թիւթիւնին ղավանօզը ինձի դուր՝ ձիգար մը փաթթիմ. յէս քիչ մը թիյրաքի ին. ղայֆէյին հէդը մութլախա ձիգառ մը բիդի խմիմ։ Քա՛, աս ի՛նչ սէրթ թիւթիւն է. փաքէ՞թ է մի՝ չէ նա ղաչախ։}

– \armenian{Մէնք ղաչախ չինք բանէցունինք. իձձունօց փաքէթ է. մէր Սէրքիս աղան փաքէթէն զադ բաշխա թիւթիւն չի բանէցունիր։}

– \armenian{Է՛հ, մնագ բարօվ. օր մը դուք ալ ամմէնքօվ մէզի հրամմէցէք. բէդ գէնիմ (կ՚սպասեմ)։}

– \armenian{Էրթաք բարօվ. նօրէն հրամմէցէք. ասիգ չիմ սէբիմ. ախջիգնէրուն բարէվ էրէ։}

* * 

\armenian{Իմ սիրագան էրգու աչգիս լուս զավագըս}.

\armenian{Հէն առաչ անուշիգ խաթրըդ հարցընիմ գու. իշալլահ օխջ}... 


\begin{adjarianpage}\label{page:183}\end{adjarianpage}% should be 183

... \armenian{առօխջ բանիդ-դօրձիդ էդէվէն իս։ Մէզի ալ հարցունէլու ըլլաս նա՝ փառք Ասձու, ամմէնքօվ աղիգ ինք։ Մինագ, անցաձ օրդանքը մարըդ քիչ մը քէյֆը ավըրից, սթմայի բէս էրավ. էրգու իրէք օր ալ բառգէցավ. համա հէքիմէն առաձ դէղէրուս վրա՝ հիմիգ աղիգ է. ասօր հինգ օր է օր օդքի էլիր է։ Ախբարնէրըդ ու քուրվըդիքըդ ամմէնքօվ աղիգ ին. իլլէ բըզդըլիգ ախբարըդ Արութջանը. գուդէ էր, խմէ էրր, դռդիգ գուդա էր. Ասվաձ չար աչգէ բահէ. ասանգ մնալու ըլլա նա՝ շադ աղիգ է։}

\armenian{Քալօվ հիմիջագ, քիչ մ՚ալ քու վրայօքըդ խօրաթինք։}

\armenian{Զավագըս, էրգու դարիէն բէրի ղարիբ-ղուրբէթ Ուռուսսիային չօլէրը քնացիր ընգիր իս. դիյօր հիմա, իրարու վրա հաշիֆ էնիմ նա՝ էռսունըօխդը մանէթէն էվէլի չէ ղրգաձըդ։ Ի՞նչ գէնիս էր. վօրի՞ն քօվը գէնաս էր. ի՞նչ է դադաձըդ, ի՞նչ է վասդըգաձըդ. ի՞նչ է խարջաձըդ. շիդգէ շիդագ բան մը գրիր չիս ինձի։ Օ՛ղուլ, դուն մէր ասդէզի հալը խօշ աղիգ գիդիս. յէս առաչվան բէս վաձձումը անցավ։ Ասդէղի գօրձէրը հարցունէլու ըլլաս նա՝ հիչ բան չիգա. բէրաննիս քամիին բացիր գէսիր ինք։ Հիչ չէ նա, յա՛վրիս, ամիսէ ամիս քսանագան մանէթ խաշլուխ ղրգիս մէզի. յէս քէզի ադ բօյէրը բէրի քի՝ ինձի յարդում էնիս, թէվընգէր ըլլաս. քէզմէն զադ ուրիշ գիւվէնէչէխ մը չունիմ. վէրը Ասվաձ, վարը քէզի գիւվէնմիշ էղիր իմ. էհմալութան չի դաս. գիրըս առնիս չառնիս՝ ինձի փարա ժըմընցընիս. դուն խընդացուր մէզի քի Ասվաձ ալ քէզի խնդացունէ։ Ամէնանփրգիչը բանիդ գօրձիդ աչօղութէն դա ու օխջ առօխջ նօրէն իրար դէսնէլու արժանի էնէ։}

\armenian{Մարըդ իր միդգը փօխիր է. գըսէ էր քի՝ Քիրքօրըս աս անքամ քալու ըլլա նա՝ օդգը գլօխ բիդի էնիմ։ Դէսնի՛մ քէզի, զավագըս, մդահան չէնիս մէզի. նամագիս դարցին բէդ գէնիմ։}

\armenian{Յէս ու մարըդ էրգու աչվընէրըդ բաքնինք գու քուրէրըդ ու ախբարնէրըդ ալ սիրօվ ու գարօդօվ բարէվնէր գէնին քէզի։}




\chapter{Hamshen}
\section{Background and literature}
\begin{adjarianpage}\label{page:184}\end{adjarianpage}% should be 184

This dialect is scattered and spread across many diverse regions. Its principle area and place of origin is east of Trabzon, in the province of Hamshen, in the same-named village-city. A few centuries ago, this province was entirely filled with Armenian residents, but the barbaric and fanatical Muslims have almost entirely erased the Armenians there. Tens of thousands of Armenians were martyred during the invasions of the bandit-preacher \armenian{Ղուռուֆ օղլի Մէհմէտ},\footnote{\translatorHD{I couldn't track down this person, or easily determine a romanization. The problem is that this name sounds. I asked a Turkish linguist who works on Hamshen (Neşe Kaya) on a possible romanization. Adjarian's phonetic transcription suggests <Ğuruf Oğli Mehmet>. But the name <Ğuruf> doesn't clearly mean anything. This name might be have mis-heard from <Yusuf Oğlu Mehmet>, meaning `Yusuf, son of Mehmet'. Despite these possible romanizations, I still couldn't track down this person. Neither Kaya nor other dialectologists (Hratch Martirosyan) recognized this name. }} tens of thousands were forced to convert to Islam, and until now are considered as Turkish (\armenian{տաճիկ}), even though they have preserved their old Armenian customs and native Armenian dialect. The remaining Armenians who were freed from the sword and apostasy were able to escape and save themselves, and they took refuge in the villages near Trabzon, Ünye, Fatsa, Terme, Çarşamba, and even much farther around Samsun, Sinop, and Nicomedia. Near İzmit, above Başiskele, they built a village called Manishag. In recent times, before the latest massacres and after, new large migrant communities of Hamshen Armenians passed through the Caucasus, where they established many small Armenian settlements on the shores of the Black Sea. For example, Sokhumi, Sochi, Mtsara, Tsebelda, Adler, Shapshuga, and so on. 


The Hamshen dialect is still not studied, but many extensive manuscripts have been published. Among these, the principle one is the publication in Ararat (\armenian{Արարատ}) 1892, \armenian{էջ} 428-447, which although it is not signed, is by the known folklorist Sarkis Haykuni (\armenian{Ս. Հայկունի}). When I was in Etchmiadzin, I had the opportunity of converting this same manuscript into the scientific orthography through this person; I bring forth this manuscript later below. Other smaller manuscripts and... 



\begin{adjarianpage}\label{page:185}\end{adjarianpage}% should be 185

... collections of words have been published in various periodicals, for example:

\begin{itemize}
\item \citeauthor{Byurakn}
\begin{itemize}
\item 1899: page 508, 558, 603, 654, 699, 752, 779
1900: page 14, 29, 42, 59, 82, 120
\end{itemize}
\item \citeauthor{HandesAmsorya}
\begin{itemize}
\item 1891: page 116, 300
\item 1892: page 24, 183-4, 382-3
\item 1895: page 13, 183-6

\end{itemize}
\item Ararat (\armenian{Արարատ})
\begin{itemize}
\item 1895: page 54, 83-84, 239-243, 293-297, 396-400
\end{itemize}
\end{itemize}

In the summer of 1910, with the goal of studying this study, I crossed Trabzon, where I stayed for two weeks. I was surrounded by many villages and teachers who were from Malya, Abgion, Küçük Şana, and Joshara; they wholeheartedly offered their help to me. I was able to make a dictionary and grammar of the dialect, to collect manuscripts, and so on. 

Across various villages, the dialect has small differences. The effect of the city is obvious. The villages that are far from Trabzon and hidden in the mountains present the most original form, while the villages that are close to the city have changed. The first group includes the village of Malya, which preserves the purest form of the Hamshen dialect. The second group includes Zefanos which is a village that is almost half an hour away from the city, and it has a very simplified dialect. 

\section{Phonology}

\subsection{Vowels}
\subsubsection{Segment inventory}

The original Hamshen dialect has the vowels /ɑ, æ, i̯e, e, ə, i, u̯o, o, œ, u, ʏ/ <\armenian{ա, ա}̈, \armenian{ե, է, ը, ի, ո, օ, էօ, ու, իւ}>. The villages close to Trabzon do not have the vowels /æ, i̯e, u̯o/ <\armenian{ա}̈, \armenian{ե, ո}>. 

\subsubsection{Sound changes}
\paragraph{Classical Armenian /ɑ/ <\armenian{ա}> }


By a general rule, the Classical Armenian sound /ɑ/ <\armenian{ա}> has changed to /o/ <\armenian{օ}> next to nasals (Table \ref{tab:Hamshen:phono:vowel:a}).



\begin{table}[H]
\centering
\caption{Change from Classical Armenian /ɑ/ <\armenian{ա}> to /o/ <\armenian{օ}> in the Hamshen dialect}
\label{tab:Hamshen:phono:vowel:a}
\begin{tabular}{|l| ll|ll| ll|}
\hline & \multicolumn{2}{l|}{Classical Armenian} &\multicolumn{2}{l|}{> Hamshen} & \multicolumn{2}{l|}{cf. SEA} \\ 
 \armenian{՝}skull' & ɡɑnk, ɡɑnɡ & \armenian{գանկ, գանգ} & ɡʰɑnɡ & \armenian{գՙօնգ} &ɡɑŋk, ɡɑŋɡ & \armenian{գանկ, գանգ} \\
 `to complain' & ɡɑnɡɑtil & \armenian{գանգատիլ} & ɡʰonɡdil & \armenian{գՙօնգդիլ} & ɡɑŋɡɑtel & \armenian{գանգատել} \\
 \armenian{՝}soup (CA); tan drink (SEA)' & tʰɑn & \armenian{թան} & tʰon & \armenian{թօն} &tʰɑn & \armenian{թան} \\
`month' & ɑmis & \armenian{ամիս} & omis & \armenian{օմիս} &ɑmis & \armenian{ամիս} \\ 
 `durable' & ɑmuɾ & \armenian{ամուր} & omuɾ & \armenian{օմուր} & ɑmuɾ & \armenian{ամուր} \\ 
 `mint' & ɑnɑnuχ & \armenian{անանուխ} & onluχkʰ & \armenian{օնլուխք} & ɑnɑnuχ & \armenian{անանուխ} \\ 
\hline 
\end{tabular}
\end{table}


\paragraph{Classical Armenian /e/ <e> }


The Classical sound /e/ <\armenian{ե}> becomes /ji/ <\armenian{յի}> at the beginning of monosyllabic words, while it becomes /e/ <\armenian{է}> at the beginning of polysyllabic words (Table \ref{tab:Hamshen:phono:vowel:e}).\footnote{\trans
For the inflected words for `ox', the final /n/ in Hamshen may be a definite suffix /-n/. }



\begin{table}[H]
\centering
\caption{Change from word-initial Classical Armenian /e/ <\armenian{ե}> to /ji, e/ <\armenian{յի, է}> in the Hamshen dialect}
\label{tab:Hamshen:phono:vowel:e}
\begin{tabular}{|l| ll|ll| ll|}
\hline & \multicolumn{2}{l|}{Classical Armenian} &\multicolumn{2}{l|}{> Hamshen} & \multicolumn{2}{l|}{cf. SEA} \\ 
 \armenian{՝}ox' & ezən & \armenian{եզն} & jiz & \armenian{յիզ} & jez & \armenian{եզ} \\
 \armenian{՝}ox ({\gen})' & ezin & \armenian{եզին} & ez-onə & \armenian{էզօնը} & jez-ɑn & \armenian{եզան} \\
 \armenian{՝}ox ({\pl})' & ezinkʰ & \armenian{եզինք} & ez-nin & \armenian{էզնին} & jez-neɾ & \armenian{եզներ} \\
 \armenian{՝}I' & es & \armenian{ես} & jis & \armenian{յիս} & jes & \armenian{ես} \\
 \armenian{՝}when' & eɾb & \armenian{երբ} & jipʰ & \armenian{յիփ} & jeɾpʰ & \armenian{երբ} \\
`oath' &eɾdumən & \armenian{երդումն} & eʃtʰvunkʰ & \armenian{էշթվունք} & jeɾtʰum & \armenian{երդում} \\ 
`thirty' &eɾesun& \armenian{երեսուն} & ersun &\armenian{էռսուն} & jeɾesun& \armenian{երեսուն} \\
`nail (finger/toe)' &eɬunɡ & \armenian{եղունգ} & eʁunkʰ & \armenian{էղունք} &jeʁuŋɡ& \armenian{եղունգ} \\
\hline 
\end{tabular}
\end{table}

Inside the word, it becomes /i̯e, i, e/ <\armenian{ե, ի, է}> (Table \ref{tab:Hamshen:phono:vowel:eMid}). (The \armenian{ե} is found especially in Malya.) 


\begin{table}[H]
\centering
\caption{Change from word-medial Classical Armenian /e/ <\armenian{ե}> to /i̯e, i, e/ <\armenian{ե, ի, է}> in the Hamshen dialect}
\label{tab:Hamshen:phono:vowel:eMid}
\begin{tabular}{|l| ll|ll| ll|}
\hline & \multicolumn{2}{l|}{Classical Armenian} &\multicolumn{2}{l|}{> Hamshen} & \multicolumn{2}{l|}{cf. SEA} \\ 
 \armenian{՝}ox' & ezən & \armenian{եզն} & jiz & \armenian{յիզ} & jez & \armenian{եզ} \\
 \armenian{՝}night' & ɡiʃeɾ & \armenian{գիշեր} & ɡʰiʃi̯eɾ & \armenian{գՙիշեր} & ɡiʃeɾ & \armenian{գիշեր} \\
 \armenian{՝}we ({\nom})' & mekʰ & \armenian{մեք} & mi̯ekʰ & \armenian{մեք} & meŋkʰ & \armenian{մենք} \\
 \armenian{՝}our ({\gen})' & meɾ & \armenian{մեր} & miɾ& \armenian{միր} & meɾ & \armenian{մեր} \\
 		`big' &met͡s & \armenian{մեծ} & mid͡z& \armenian{միձ} &met͡s & \armenian{մեծ} \\ 
 `place' & teɬ & \armenian{տեղ}& deʁ & \armenian{դէղ}&teʁ & \armenian{տեղ} \\
\hline 
\end{tabular}
\end{table}
\paragraph{Classical Armenian /o/ <\armenian{ո}>}


The sound /o/ <\armenian{ո}> changes everywhere to /u̯o, o, œ, ʏ/ <\armenian{ո, օ, էօ, իւ}> (Table \ref{tab:Hamshen:phono:vowel:o}a) except for the words in Table \ref{tab:Hamshen:phono:vowel:o}b. 


\begin{table}[H]
\centering
\caption{Change from word-medial Classical Armenian /o/ <\armenian{ո}> to /u̯o, o, œ, ʏ/ <\armenian{ո, օ, էօ, իւ}> with exceptions in the Hamshen dialect}
\label{tab:Hamshen:phono:vowel:o}
\begin{tabular}{|ll| ll|ll| ll|}
\hline & & \multicolumn{2}{l|}{Classical Armenian} &\multicolumn{2}{l|}{> Hamshen} & \multicolumn{2}{l|}{cf. SEA} \\ 
a. &\armenian{՝}villager' & & &&& ɡjuʁɑt͡sʰi & \armenian{գյուղացի} \\
 & \armenian{՝}villager ({\dat},{\pl})' & & &ɡʰeʁɑt͡sʰ-ot͡sʰ&\armenian{գՙէղացոց}& ɡjuʁɑt͡sʰ-ot͡sʰ & \armenian{գյուղացոց} \\
& `louse' &od͡ʒil & \armenian{ոջիլ} & ot͡ʃʰil & \armenian{օչիլ} &vot͡ʃʰil& \armenian{ոջիլ} \\
& `apple' & χənd͡zoɾ & \armenian{խնձոր} & χənd͡zʏɾ, χənd͡zœj & \armenian{խընձիւր, խընձէօյ} & χənd͡zoɾ & \armenian{խնձոր} \\ 
& `valley' &d͡zoɾ & \armenian{ձոր} & d͡zʰœɾ, d͡zʏɾ & \armenian{ձՙէօր, ձիւր} & d͡zoɾ & \armenian{ձոր} \\
& `four' &t͡ʃʰoɾs & \armenian{չորս} &t͡ʃʰœjs &\armenian{չէօյս} & t͡ʃʰoɾs & \armenian{չորս} \\
b. & \armenian{՝}who' &ov &\armenian{ով} & vov, vœv & \armenian{վօվ, վէօվ} & ov & \armenian{ով} \\
&\armenian{՝}which' &oɾ &\armenian{որ} & vœɾ & \armenian{վէ՞օր} & voɾ & \armenian{որ} \\
\hline 
\end{tabular}
\end{table}

\paragraph{Classical Armenian /ɑi̯/ <\armenian{այ}>}

Among the diphthongs, Classical /ɑi̯/ <\armenian{այ}> changes usually to /e/ <\armenian{է}>, ... 

\begin{adjarianpage}\label{page:186}\end{adjarianpage}% should be 186

... to /æ/ <\armenian{ա}̈> in Malya (Table \ref{tab:Hamshen:phono:vowel:aj}). 


\begin{table}[H]
\centering
\caption{Change from Classical Armenian /ɑi̯/ <\armenian{այ}> to /e, æ/ <\armenian{է, ա}̈> in the Hamshen dialect}
\label{tab:Hamshen:phono:vowel:aj}
\begin{tabular}{|l| ll|ll| ll|}
\hline & \multicolumn{2}{l|}{Classical Armenian} &\multicolumn{2}{l|}{> Hamshen} & \multicolumn{2}{l|}{cf. SEA} \\ 
`goat' & ɑi̯t͡s & \armenian{այծ} & æd͡z, ed͡z & \armenian{ա̈ձ, էձ} & ɑjt͡s & \armenian{այծ} \\ 
`this' & ɑi̯s & \armenian{այս} & æs, es & \armenian{ա̈ս, էս} & ɑjs & \armenian{այս} \\ 
 `other' &ɑi̯l& \armenian{այլ} & æl, el & \armenian{ա̈լ, էլ} &ɑjl& \armenian{այլ} \\
\hline 
\end{tabular}
\end{table}

\paragraph{Classical Armenian /oi̯, iu̯/ <\armenian{ոյ, իւ}> }

For the others, Classical /oi̯, iu̯/ <\armenian{ոյ, իւ}> become /u/ <\armenian{ու}> (Table \ref{tab:Hamshen:phono:vowel:aj}). 


\begin{table}[H]
\centering
\caption{Change from Classical Armenian /oi̯, iu̯/ <\armenian{ոյ, իւ}> to /u/ <\armenian{ու}> in the Hamshen dialect}
\label{tab:Hamshen:phono:vowel:otherDiph}
\begin{tabular}{|l| ll|ll| ll|}
\hline & \multicolumn{2}{l|}{Classical Armenian} &\multicolumn{2}{l|}{> Hamshen} & \multicolumn{2}{l|}{cf. SEA} \\ 
`light' & loi̯s & \armenian{լոյս} & lus & \armenian{լուս} & lujs & \armenian{լույս} \\ 
 \armenian{՝}snow' & d͡ziu̯n & \armenian{ձիւն}& d͡zun & \armenian{ձուն} & d͡zjun & \armenian{ձյուն} \\
\hline 
\end{tabular}
\end{table}

\subsection{Consonants}

\subsubsection{Voice quality or laryngeal changes}
The consonant group has three degrees: voiced, voiced aspirate, and voiceless unaspirate. It must be noted however that the voiced sounds are also not fully voiced here, but are very close to the voiceless unaspirated. The villages that are close to the city have only two degrees, missing the voiced aspirates. The Old Armenian voiced consonants are usually changed to voiced aspirates, and the voiceless unaspirates are changed to voiced, while the voiceless aspirates stay the same.\footnote{\translatorHD{This contradicts the first sentence; he might just mean that the dialect only has voiceless consonants, without being sure of aspiration.}}

\subsubsection{Reflexes of Classical /ɾ/ <\armenian{ր}>} 

What is interesting is the changes for the Classical sound /ɾ/ <\armenian{ր}>. Next to dentals, it becomes /ʃ/ <\armenian{շ}>, it becomes /j/ <\armenian{յ}> next to other consonants, while it stays the same next to vowels (Table \ref{tab:Hamshen:phono:cons:r}). 


\begin{table}[H]
\centering
\caption{Change from Classical Armenian /ɾ/ <\armenian{ր}> to /ʃ, j, ɾ/ <\armenian{շ, յ, ր}> in the Hamshen dialect}
\label{tab:Hamshen:phono:cons:r}
\begin{tabular}{|l| ll|ll| ll|}
\hline & \multicolumn{2}{l|}{Classical Armenian} &\multicolumn{2}{l|}{> Hamshen} & \multicolumn{2}{l|}{cf. SEA} \\ 
`man' &mɑɾd & \armenian{մարդ} & mɑʃd & \armenian{մաշդ} &mɑɾtʰ & \armenian{մարդ} \\
`empty' &dɑtɑɾk& \armenian{դատարկ} & dɑjdɑɡ &\armenian{դայդագ} & dɑtɑɾk & \armenian{դատարկ} \\
`four' &t͡ʃʰoɾs & \armenian{չորս} &t͡ʃʰʏjs & \armenian{չիւյս} & t͡ʃʰoɾs & \armenian{չորս} \\
 `dream' & eɾɑz & \armenian{երազ} & neɾɑz & \armenian{նէրազ} & jeɾɑz & \armenian{երազ} \\
 \armenian{՝}face' & eɾes & \armenian{երես} & eɾis & \armenian{էրիս} & jeɾes & \armenian{երես} \\
 \armenian{՝}face-{\gen}' & eɾes-i & \armenian{երեսի} & ejs-i & \armenian{էյսի} & jeɾes-i & \armenian{երեսի} \\
\hline 
\end{tabular}
\end{table}

\section{Morphology}

\subsection{Noun inflection or declension}
\subsubsection{Plural suffixes}

For declensions, what is noteworthy are the plural formatives /-iɾ, -niɾ, -nin/ <\armenian{իր նիր նին}> \ref{tab:hamshen:morpho:noun:pl}.\footnote{\translatorHD{Adjarian's paradigms have quite ambiguous segmentations. Table \ref{tab:hamshen:morpho:noun:pl} is just my speculation on one possible morpheme segmentation. }} 

\begin{table}[H]
\centering
\caption{Plural suffixes in the Hamshen dialect}
\label{tab:hamshen:morpho:noun:pl}

\begin{tabular}{|l|ll|ll|}
\hline & `bread' & & `apple' &\\
\hline {\nom}& hɑt͡sʰ-iɾ & \armenian{հացիր} & χnd͡zʰoj-niɾ & \armenian{խնձՙօյնիր} \\
 & & & χnd͡zʰoj-nin & \armenian{խնձՙօյնին} \\
{\gen}-{\dat} & hɑt͡sʰ-eɾ-u & \armenian{հացէրու} & χnd͡zʰoj-nun & \armenian{խնձՙօյնուն} \\
{\abl}& hɑt͡sʰ-eɾ-u-n & \armenian{հացէրուն} & χnd͡zʰoj-neɾ-en & \armenian{խնձՙօյնէրէն}\\
 & hɑt͡sʰ-eɾ-u-n-menen & \armenian{հացէրունմէնէն} & χnd͡zʰoj-nun & \armenian{խնձՙօյնուն} \\
 & & & χnd͡zʰoj-nun-mene & \armenian{խնձՙօյնունմէնէ} \\
{\ins}& hɑt͡sʰ-eɾ-ov& \armenian{հացէրօվ} & χnd͡zʰoj-neɾ-ov & \armenian{խնձՙօյնէրօվ}\\ \hline 
\end{tabular}
\end{table}
\subsubsection{Case marking}

The accusative is sometimes the same as the nominative, and sometimes the same as the dative. The singular ablative takes /-en/ <\armenian{էն}>or /-\'ən/ <\armenian{ը՚ն}>. 
\subsection{Pronoun inflection or declension}
\subsubsection{Personal pronouns}
\translatorHD{Table \ref{tab:Hamshen:morpho:pronoun:personal} is for personal pronouns. }

\begin{table}[H]
\caption{Inflection paradigm for personal pronouns in the Hamshen dialect }\label{tab:Hamshen:morpho:pronoun:personal}
\centering
\begin{tabular}{|l|lll|lll|}
 \hline& 1SG & 2SG& 3SG & 1PL& 2PL & 3PL\\
 & `I' & `you' & `he/she' & `we'& `you' & `they'\\\hline 
{\nom} & jes & dʰun & en, inɑ & mekʰ & dʰunkʰ & eniɾ, iniɾ \\
 & \armenian{յէս} & \armenian{դՙուն} & \armenian{էն, ինա} & \armenian{մէք} & \armenian{դՙունք} & \armenian{էնիր, ինիր} \\\hline 
{\gen} & im & kʰu& enu, inu & miɾ, mij & d͡zʰuɾ, d͡zʰij & enut͡sʰ, inut͡sʰ\\
 & \armenian{իմ} & \armenian{քու}& \armenian{էնու, ինու}& \armenian{միր, միյ} & \armenian{ձՙիր, ձՙիյ} & \armenian{էնուց, ինուց}\\\hline 
{\dat} & ind͡zi, ind͡ziɡi & kʰezi, kʰeziɡi & enu & mezi, meziɡi & d͡zʰezi, d͡zʰeziɡi & enut͡sʰ, inut͡sʰ\\
 & \armenian{ինձի, ինձիգի} & \armenian{քէզի, քէզիգի} & \armenian{էնու} & \armenian{մէզի, մէզիգի} & \armenian{ձՙէզի, ձՙէզիգի} & \armenian{էնուց, ինուց}\\\hline 
{\acc} & ind͡z-i, ind͡z-iɡi & kʰez-i, kʰez-iɡi & en, enu, zən & mez-i, mez-iɡi & d͡zʰez-i, d͡zʰez-iɡi & enut͡sʰ, inut͡sʰ, zeniɾ \\
 & \armenian{ինձի, ինձիգի} & \armenian{քէզի, քէզիգի} & \armenian{էն, էնու, զըն} & \armenian{մէզի, մէզիգի} & \armenian{ձՙէզի, ձՙէզիգի} & \armenian{էնուց, ինուց, զէնիր} \\\hline 
{\abl} & ind͡z-men & kʰez-men& endi, indi& mez-men & d͡zʰez-men & enut͡sʰ-men, inut͡sʰ-men \\
 & \armenian{ինձմէն} & \armenian{քէզմէն} & \armenian{էնդի, ինդի}& \armenian{մէզմէն} & \armenian{ձՙէզմէն}& \armenian{էնուցմէն, ինուցմէն} \\\hline 
{\ins} & ind͡z-mov & kʰez-mov& enu hid & mez-mov & d͡zʰez-mov & enut͡sʰ-mov \\
 & \armenian{ինձմօվ} & \armenian{քէզմօվ} & \armenian{էնու հիդ} & \armenian{մէզմօվ} & \armenian{ձՙէզմօվ}& \armenian{էնուցմօվ} 
\\ \hline 
\end{tabular}
\end{table}


\begin{adjarianpage}\label{page:187}\end{adjarianpage}% should be 187

The plural instrumental is very diverse (Table \ref{tab:Hamshen:morphology:pronoun:sampleInst}). But the word /inut͡sʰ-mov/ <\armenian{ինուցմօվ}> is not said.



\begin{table}[H]
 \centering
 \caption{Sample of plural instrumental pronouns in the Hamshen dialect}
 \label{tab:Hamshen:morphology:pronoun:sampleInst}
 \begin{tabular}{|l l| ll|}
\hline 
\multicolumn{2}{| l|}{3PL `with those'} & \multicolumn{2}{l|}{Other pronouns, maybe `with these/those'?} \\\hline 
enut͡sʰ-mov & \armenian{էնուցմօվ} & esiɾ & \armenian{էսիր}\\
ənut͡sʰ-mov & \armenian{ընուցմօվ} &isiɾ & \armenian{իսիր}\\
 enut͡sʰ-meɾ-ov & \armenian{էնուցմէրօվ} &esut͡sʰ-meɾ-ov &\armenian{էսուցմէրօվ} \\
 ənut͡sʰ-meɾ-ov & \armenian{ընուցմէրօվ} & esduut͡sʰ-mov & \armenian{էսդուցմօվ}\\
enit͡sʰ-meɾ-ov & \armenian{էնիցմէրօվ} &esit͡sʰ-meɾ-ov & \armenian{էսիցմէրօվ}\\
ənit͡sʰ-meɾ-ov \armenian{ընիցմէրօվ} & edit͡sʰ-meɾ-ov&\armenian{էդիցմէրօվ} \\
& & edut͡sʰ-meɾ-ov & \armenian{էդուցմէրօվ} \\
\hline 
 \end{tabular}
\end{table}

\translatorHD{Adjarian likewise lists a paradigm (Table \ref{tab:Hamshen:morpho:pronoun:person:logo}) which seems to be for the reflex of the logophoric third person pronoun. }

\begin{table}[H]
\caption{Inflection paradigm for the logophoric 3SG/3PL pronoun in the Hamshen dialect }\label{tab:Hamshen:morpho:pronoun:person:logo}
\centering
\begin{tabular}{|l|l| l|}
 \hline 
 & Singular & Plural \\ \hline 
 {\nom} & inkʰ-ə & uɾinkʰ \\
 & \armenian{ինքը} & \armenian{ուրինք} \\\hline 
{\gen} & uɾ, uj, uɾin & uɾint͡sʰ \\
 & \armenian{ուր, ույ, ուրին} & \armenian{ուրինց} \\\hline 
{\dat}-{\acc} & uɾin-ə & uɾint͡sʰ \\
 & \armenian{ուրինը} & \armenian{ուրինց} \\\hline 
{\abl} & uɾ-men, uɾ-mən, uɾin-men & uɾint͡sʰ-men \\
 & \armenian{ուրմէն, ուրմըն, ուրինմէն} & \armenian{ուրինցմէն} \\\hline 
{\ins} & uɾ hid, uɾin-ə hid & uɾint͡sʰ hid \\
 & \armenian{ուր հիդ, ուրինը հիդ} & \armenian{ուրինց հիդ} 
\\ \hline 
\end{tabular}
\end{table}

\subsubsection{Interrogative pronouns}


\translatorHD{Table \ref{tab:Hamshen:morpho:pronoun:inter:who} is for the interrogative pronoun `who' }

\begin{table}[H]
\caption{Inflection paradigm for interrogatives pronouns `who' in the Hamshen dialect }\label{tab:Hamshen:morpho:pronoun:inter:who}
\centering
\begin{tabular}{|l|l| l|}
 \hline 
 & Singular & Plural \\ \hline 
 {\nom} & vœv & voɾokʰ \\
 & \armenian{վէօվ}& \armenian{վօրօք}\\ \hline 
{\gen}-{\dat}-{\acc} & vum & umint͡sʰ, vumint͡sʰ, voɾint͡sʰ, voɾœt͡sʰ \\
 & \armenian{վում}& \armenian{ումինց, վումինց, վօրինց, վօրէօց} \\\hline 
{\abl} & vum-men& umint͡sʰ-men, voɾot͡sʰ-men\\
 & \armenian{վումմէն} & \armenian{ումինցմէն, վօրօցմէն} \\ 
 & um-men & voɾʏnt͡sʰ-men, vumet͡sʰ-men, umot͡sʰ-men \\
 & \armenian{ումմէն} & \armenian{վօրիւնցմէն, վումէցմէն, ումօցմէն} \\ 
 \hline 
{\ins} & vum hid & umint͡sʰ-mov, vumint͡sʰ-mov, voɾot͡sʰ-mov\\
 & \armenian{վում հիդ}& \armenian{ումինցմօվ, վումինցմօվ, վօրօցմօվ} 
\\ \hline 
\end{tabular}
\end{table}

\translatorHD{Table \ref{tab:Hamshen:morpho:pronoun:inter:what} is for the interrogative pronoun `what/which' }

\begin{table}[H]
\caption{Inflection paradigm for interrogatives pronouns `what/which' in the Hamshen dialect }\label{tab:Hamshen:morpho:pronoun:inter:what}
\centering
\begin{tabular}{|l|l| l|}
 \hline 
 & Singular & Plural \\ \hline 
{\nom} & vœɾ & vuɾokʰ-ə, vuɾonkʰ-ə \\
 & \armenian{վէօր} & \armenian{վուրօքը, վուրօնքը} \\ \hline 
{\gen}-{\dat}-{\acc} & voɾ-in & vuɾot͡sʰ-ə, vuɾont͡sʰ-ə \\
 & \armenian{վօրին} & \armenian{վուրօցը, վուրօնցը} \\\hline 
{\abl} & voɾ-men & vuɾot͡sʰ-men, vuɾont͡sʰ-men \\
 & \armenian{վօրմէն} & \armenian{վուրօցմէն, վուրօնցմէն} \\\hline 
{\ins} & voɾ-ov & vuɾot͡sʰ-mov, vuɾont͡sʰ-mov \\
 & \armenian{վօրօվ} & \armenian{վուրօցմօվ, վուրօնցմօվ} 
\\ \hline 
\end{tabular}
\end{table}

\subsection{Verb inflection or conjugation}


\subsubsection{Indicative present and past imperfective}\label{sec:Hamshen:morphology:verb:indcprespst}
In verbs, there are many interesting innovations. In the first conjugation class, the Classical vowel /e/ <\armenian{ե}> has become /i/ <\armenian{ի}> in the 1SG, 2SG, 1PL, and 3PL persons. In the second conjugation class, the vowel /ɑ/ <\armenian{ա}> has become /o/ <\armenian{օ}> in the 1SG, 1PL, and 3PL. The indicative present and imperfective are formed with the formatives /ɡ/ <\armenian{գ}> or /ɡu/ <\armenian{գու}>. The first is placed at the beginning of vowel-initial verbs, the second is placed after consonant-initial verbs. The progressive is formed with the formative /uni/ <\armenian{ունի}> (and sometimes also /ɡuni/ <\armenian{գունի}>).

\translatorHD{To clarify, in SWA, the indicative present tense is made up of the indicative prefix /ɡ(ə)-/ and the finite verb. The finite verb is made up of the stem, the theme vowel, and agreement suffixes. For a verb like `to eat' and `to bring', the theme vowel is an invariant /e/. For Hamshen, we find the following differences (Table \ref{tab:Hamshen:morpho:verb:paradigm:presentIndc}): 1) the theme vowel of these verbs is /e/ in the 3SG and 2PL, but /i/ elsewhere; 2) the indicative morpheme is a prefix /ɡ-/ for vowel-initial verbs but a suffix /-ɡu/ for consonant-initial verbs; 3) the 1PL and 2PL are homophonous suffixes /-kʰ/ that take different theme vowels. }


\begin{table}[H]
 	\centering 
 	\caption{Indicative present <\armenian{ներկայ}> of the verb `to eat' and `to bring' in the Hamshen dialect}
 	\label{tab:Hamshen:morpho:verb:paradigm:presentIndc}
 \begin{tabular}{|l|ll| ll| }
 		\hline & \multicolumn{2}{l|}{Hamshen `to eat'} & \multicolumn{2}{l|}{cf. SWA} \\ \hline 
1SG &ɡ-ud-i-m & \armenian{գուդիմ} & ɡ-ud-e-m& \armenian{կ՚ուտեմ} \\
2SG &ɡ-ud-i-s & \armenian{գուդիս} & ɡ-ud-e-s& \armenian{կ՚ուտես} \\
3SG &ɡ-ud-e-$\emptyset$ & \armenian{գուդէ} & ɡ-ud-e-$\emptyset$& \armenian{կ՚ուտէ} \\
1PL &ɡ-ud-i-kʰ & \armenian{գուդիք} & ɡ-ud-e-ŋkʰ& \armenian{կ՚ուտենք} \\
2PL &ɡ-ud-e-kʰ & \armenian{գուդէք} & ɡ-ud-e-kʰ& \armenian{կ՚ուտէք} \\
3PL &ɡ-ud-i-n & \armenian{գուդին} & ɡ-ud-e-n& \armenian{կ՚ուտեն} \\
& \multicolumn{2}{l|}{{\ind}-$\sqrt{}$-{\thgloss}-{\agr}}& \multicolumn{2}{l|}{{\ind}-$\sqrt{}$-{\thgloss}-{\agr}}\\
\hline & \multicolumn{2}{l|}{Hamshen `to bring'} & \multicolumn{2}{l|}{cf. SWA} \\ \hline
1SG &bʰeɾ-i-m ɡu& \armenian{բՙէրիմ գու} & ɡə-pʰeɾ-e-m& \armenian{կը բերեմ} \\
2SG &bʰeɾ-i-s ɡu& \armenian{բՙէրիս գու} & ɡə-pʰeɾ-e-s& \armenian{կը բերես} \\
3SG &bʰeɾ-e-$\emptyset$ ɡu& \armenian{բՙէրէ գու} & ɡə-pʰeɾ-e-$\emptyset$& \armenian{կը բերէ} \\
1PL &bʰeɾ-i-kʰ ɡu& \armenian{բՙէրիք գու} & ɡə-pʰeɾ-e-ŋkʰ& \armenian{կը բերենք} \\
2PL &bʰeɾ-e-kʰ ɡu& \armenian{բՙէրէք գու} & ɡə-pʰeɾ-e-kʰ& \armenian{կը բերէք} \\
3PL &bʰeɾ-i-n ɡu & \armenian{բՙէրին գու} & ɡə-pʰeɾ-e-n& \armenian{կը բերեն} \\
& \multicolumn{2}{l|}{$\sqrt{}$-{\thgloss}-{\agr} {\ind}}& \multicolumn{2}{l|}{{\ind}-$\sqrt{}$-{\thgloss}-{\agr}}\\
\hline 
\end{tabular}
\end{table}

\begin{adjarianpage}\label{page:188}\end{adjarianpage}% should be 188

\translatorHD{In SWA, the past imperfective is similar to the present (Table \ref{tab:Hamshen:morpho:verb:paradigm:pastImpfInd}). The main difference is that SWA adds the past suffix /-i-/ between the theme vowel and agreement. The past suffix is zero for the 3SG. For Hamshen, we see many differences against the SWA forms. The past suffix is /-i-/ for 1SG and 2SG but seems covert in all other person-numbers. The past agreement suffixes in Hamshen are different than those of SWA. The theme vowel can vary between /e/, /\'ɑ/, and /\'e/ with non-final stress. }


\begin{table}[H]
 	\centering 
 	\caption{Indicative past imperfective <\armenian{անկատար}> of the verb `to eat' and `to bring' in the Hamshen dialect}
 	\label{tab:Hamshen:morpho:verb:paradigm:pastImpfInd}
 \begin{tabular}{|l|ll| ll| }
 		\hline & \multicolumn{2}{l|}{Hamshen `to eat'} & \multicolumn{2}{l|}{cf. SWA} \\ \hline 
1SG &ɡ-ud-ej-i-$\emptyset$ & \armenian{գուդէյի} & ɡ-ud-ej-i-$\emptyset$& \armenian{կ՚ուտէի} \\
2SG &ɡ-ud-ej-i-ɾ & \armenian{գուդէյիր} & ɡ-ud-ej-i-ɾ& \armenian{կ՚ուտէիր} \\
& ɡ-ud-\'e-$\emptyset$-jdə & \armenian{գուդէ՛յդը} & & \\
3SG &ɡ-ud-e-$\emptyset$-ɾ & \armenian{գուդէր} & ɡ-ud-e-$\emptyset$-ɾ& \armenian{կ՚ուտէր} \\
1PL &ɡ-ud-\'ɑ-$\emptyset$-kʰə & \armenian{գուդա՛քը} & ɡ-ud-ej-\'i-ŋkʰ& \armenian{կ՚ուտէինք} \\
2PL &ɡ-ud-\'e-$\emptyset$-kʰə & \armenian{գուդէ՛քը} & ɡ-ud-ej-\'i-kʰ& \armenian{կ՚ուտէիք} \\
3PL &ɡ-ud-\'e-$\emptyset$-jnə & \armenian{գուդէ՛յնը} & ɡ-ud-ej-\'i-n& \armenian{կ՚ուտէին} \\
& \multicolumn{2}{l|}{{\ind}-$\sqrt{}$-{\thgloss}-{\pst}-{\agr}}& \multicolumn{2}{l|}{{\ind}-$\sqrt{}$-{\thgloss}-{\pst}-{\agr}}\\
\hline & \multicolumn{2}{l|}{Hamshen `to bring'} & \multicolumn{2}{l|}{cf. SWA} \\ \hline
1SG &bʰeɾ-ej-i-$\emptyset$ ɡu& \armenian{բՙէրէյի գու} & ɡə-pʰeɾ-ej-i-$\emptyset$& \armenian{կը բերէի} \\
2SG &bʰeɾ-ej-i-ɾ ɡu & \armenian{բՙէրէյիր գու} & ɡə-pʰeɾ-ej-i-ɾ& \armenian{կը բերէիր} \\
 &bʰeɾ-\'ej-$\emptyset$-də ɡu& \armenian{բՙէրէյ՛դը գու}& & \\
3SG &bʰeɾ-e-$\emptyset$-ɾ ɡu& \armenian{բՙէրէր գու} & ɡə-pʰeɾ-e-$\emptyset$-ɾ& \armenian{կը բերէր} \\
1PL &bʰeɾ-\'ɑ-$\emptyset$-kʰə ɡu& \armenian{բՙէրա՛քը գու} & ɡə-pʰeɾ-ej-\'i-ŋkʰ& \armenian{կը բերէինք} \\
2PL &bʰeɾ-\'e-$\emptyset$-kʰə ɡu& \armenian{բՙէրէ՛քը գու} & ɡə-pʰeɾ-ej-\'i-kʰ& \armenian{կը բերէիք} \\
3PL &bʰeɾ-\'e-$\emptyset$-jnə ɡu & \armenian{բՙէրէ՛յնը գու} & ɡə-pʰeɾ-ej\'-i-n& \armenian{կը բերէին} \\
& \multicolumn{2}{l|}{$\sqrt{}$-{\thgloss}-{\pst}-{\agr} {\ind}}& \multicolumn{2}{l|}{{\ind}-$\sqrt{}$-{\thgloss}-{\pst}-{\agr}}\\
\hline 
\end{tabular}
\end{table}

\subsubsection{Past perfective or aorist}

The perfective is formed in the old way. But in the first conjugation, the vowel of the 3SG becomes /i/ <\armenian{ի}>. 

\translatorHD{Adjarian doesn't provide complete paradigms for the past perfective or aorist (Table \ref{tab:Hamshen:morpho:verb:paradigm:pastperfectiveAorist}). But his implicitness suggests that Hamshen follows SWA in forming the past perfective. For a verb like `to broom', the past perfective is made up of the root + theme vowel /e/ + aorist suffix /t͡sʰ/ + past/agreement marking. In the 3SG, past and agreement are covert. For Hamshen, it seems that the main difference is that the theme vowel is /i/ in the 3SG instead of /e/. Note that Adjarian also lists the verb `to bring' which is irregular in SWA. }



\begin{table}[H]
 \centering
 \caption{Past perfective or aorist <\armenian{կատարեալ}> in the Hamshen dialect}
 \label{tab:Hamshen:morpho:verb:paradigm:pastperfectiveAorist}
 \begin{tabular}{|l|ll|ll|}
\hline & \multicolumn{2}{l|}{Hamshen} & \multicolumn{2}{l|}{cf. SEA} \\
3PL `they broomed' & & & ɑvl-e-t͡sʰ-i-n &\armenian{աւլեցին} \\
3SG `he broomed' & ɑvl-i-t͡sʰ-$\emptyset$-$\emptyset$ & \armenian{ավլից} &ɑvl-e-t͡sʰ-$\emptyset$-$\emptyset$ &\armenian{աւլես} \\
3SG \armenian{՝}?' & ɑʃ-i-t͡sʰ-$\emptyset$-$\emptyset$ & \armenian{աշից} & & \\
3SG \armenian{՝}he threw away' & tʰɑpʰ-i-t͡sʰ-$\emptyset$-$\emptyset$ & \armenian{թափից} & tʰɑpʰ-e-t͡sʰ-$\emptyset$-$\emptyset$ & \armenian{թափեց}\\
3SG \armenian{՝}he brought' & bʰeɾ-i-t͡sʰ-$\emptyset$-$\emptyset$ & \armenian{բՙէրից} & pʰeɾ-x$\emptyset$-$\emptyset$-ɑ-v & \armenian{բերաւ} \\
& \multicolumn{2}{l|}{$\sqrt{}$-{\thgloss}-{\aor}-{\pst}-{\agr}}& \multicolumn{2}{l|}{$\sqrt{}$-{\thgloss}-{\aor}-{\pst}-{\agr}}\\ 

\hline 
\end{tabular}
\end{table}


\translatorHD{He provides a more complete paradigm for the negated past perfective in \S\ref{sec:Hamshen:morphology:verb:neg:pastperfectiveaux}. }


\subsubsection{Future and future perfect}

The future formative is /bidi/ <\armenian{բիդի}>, which is always placed after the verb. In the 1SG of the present future, the formative loses its sound /b/ <\armenian{բ}> when after the sound /m/ <\armenian{մ}>, and of course by first turning into /m/ <\armenian{մ}> and then shortening. In the other persons, the sound /b/ <\armenian{բ}> stays the same. 

\translatorHD{To clarify, the future (Table \ref{tab:Hamshen:morpho:verb:paradigm:fut}) and future perfect (Table \ref{tab:Hamshen:morpho:verb:paradigm:futPerf}) are formed by taking the finite verb from respectively the indicative present and past imperfective. In both SWA and Hamshen, the indicative morpheme is replaced by a future morpheme /bidi/. In SWA, this future morpheme is a proclitic, while it is a proclitic in Hamshen. In Hamshen, the sound /b/ of the future morpheme /bidi/ is deleted after the /m/ of the 1SG suffix. }


\begin{table}[H]
 	\centering 
 	\caption{Future <\armenian{ապառնի}> of the verb `to bring' in the Hamshen dialect}
 	\label{tab:Hamshen:morpho:verb:paradigm:fut}
 \begin{tabular}{|l|ll| ll| }
 		\hline & \multicolumn{2}{l|}{Hamshen} & \multicolumn{2}{l|}{cf. SWA} \\ \hline 
1SG &bʰeɾ-i-m idi& \armenian{բՙէրիմ իդի} & bidi pʰeɾ-e-m& \armenian{պիտի բերեմ} \\
2SG &bʰeɾ-i-s bidi& \armenian{բՙէրիս բիդի} & bidi pʰeɾ-e-s& \armenian{պիտի բերես} \\
3SG &bʰeɾ-e-$\emptyset$ bidi& \armenian{բՙէրէ բիդի} & bidi pʰeɾ-e-$\emptyset$& \armenian{պիտի բերէ} \\
1PL &bʰeɾ-i-kʰ bidi& \armenian{բՙէրիք բիդի} & bidi pʰeɾ-e-ŋkʰ& \armenian{պիտի բերենք} \\
2PL &bʰeɾ-e-kʰ bidi& \armenian{բՙէրէք բիդի} & bidi pʰeɾ-e-kʰ& \armenian{պիտի բերէք} \\
3PL &bʰeɾ-i-n bidi & \armenian{բՙէրին բիդի} & bidi pʰeɾ-e-n& \armenian{պիտի բերեն} \\
& \multicolumn{2}{l|}{$\sqrt{}$-{\thgloss}-{\agr} {\fut}}& \multicolumn{2}{l|}{{\fut} $\sqrt{}$-{\thgloss}-{\agr}}\\
\hline 
\end{tabular}
\end{table}




\begin{table}[H]
 	\centering 
 	\caption{Future perfect <\armenian{անցեալ ապառնի}> of the verb `to bring' in the Hamshen dialect}
 	\label{tab:Hamshen:morpho:verb:paradigm:futPerf}
 \begin{tabular}{|l|ll| ll| }
 		\hline & \multicolumn{2}{l|}{Hamshen } & \multicolumn{2}{l|}{cf. SWA} \\ \hline 
1SG &bʰeɾ-ej-i-$\emptyset$ bidi& \armenian{բՙէրէյի բիդի} & bidi pʰeɾ-ej-i-$\emptyset$& \armenian{պիտի բերէի} \\
2SG &bʰeɾ-\'ej-$\emptyset$-də bidi& \armenian{բՙէրէյ՛դը բիդի}&bidi pʰeɾ-ej-\'i-ɾ & \armenian{պիտի բերէիր} \\
3SG &bʰeɾ-e-$\emptyset$-ɾ bidi& \armenian{բՙէրէր բիդի} & bidi pʰeɾ-e-$\emptyset$-ɾ& \armenian{պիտի բերէր} \\
1PL &bʰeɾ-\'ɑ-$\emptyset$-kʰə bidi& \armenian{բՙէրա՛քը բիդի} & bidi pʰeɾ-ej-\'i-ŋkʰ& \armenian{պիտի բերէինք} \\
2PL &bʰeɾ-\'e-$\emptyset$-kʰə bidi& \armenian{բՙէրէ՛քը բիդի} & bidi pʰeɾ-ej-\'i-kʰ& \armenian{պիտի բերէիք} \\
3PL &bʰeɾ-\'e-$\emptyset$-jnə bidi & \armenian{բՙէրէ՛յնը բիդի} & bidi pʰeɾ-ej-\'i-n& \armenian{պիտի բերէին} \\
& \multicolumn{2}{l|}{$\sqrt{}$-{\thgloss}-{\pst}-{\agr} {\fut}}& \multicolumn{2}{l|}{{\fut}-$\sqrt{}$-{\thgloss}-{\pst}-{\agr}}\\
\hline 
\end{tabular}
\end{table}

\subsubsection{Subjunctive present and past imperfective with marker /nɑ/ <\armenian{նա}>}

\translatorHD{In SWA, the subjunctive present/past is the finite verb form that is found in the indicative present/past (\ref{sent:SWA:subj:if}). In fact, the indicative is built from the subjunctive by adding the indicative morpheme /ɡ(ə)-/ (\ref{sent:SWA:subj:indc}). The subjunctive can be found in conditional clauses. In colloquial speech, such conditional clauses can be optionally accompanied by a clitic /-ne/ (\ref{sent:SWA:subj:ne}). }


\begin{exe}
\ex SWA \label{sent:SWA:subj}
\begin{xlist}
 \ex \gll jetʰe ləs-e-n, jetʰe ləs-ej-i-n\\
 if listen-{\thgloss}-3{\pl}, if listen-{\thgloss}-{\pst}-3{\pl} \\
 \trans `If they listen; if they listened.' \label{sent:SWA:subj:if}\\
 \armenian{Եթէ լսեն, եթէ լսէին։}
 \ex \gll ɡə-ləs-e-n. ɡə-ləs-ej-i-n\\
 listen-{\thgloss}-3{\pl}. listen-{\thgloss}-{\pst}-3{\pl} \\
 \trans `They listen. They were listening'\label{sent:SWA:subj:indc} \\
 \armenian{Կը լսեմ, կը լսեին։}
 \ex \gll jetʰe ləs-e-n ne. jetʰe ləs-ej-i-n ne\\
 if listen-{\thgloss}-1{\sg} {\sbjv}. if listen-{\thgloss}-{\pst}-1{\sg} {\sbjv} \\
 \trans `If they listen; If they listened.'\label{sent:SWA:subj:ne}\\
 \armenian{Եթէ լսեմ նէ, եթէ լսէին նէ։}
 
\end{xlist}
\end{exe}

\translatorHD{Given this background, we can understand Adjarian's description of Hamshen.}

The subjunctive (\armenian{ստորադասական}) is formed with the formative /nɑ/ <\armenian{նա}> (\ref{sent:Hamshen:morpho:verb:subj}). 

\begin{exe}
 \ex Hamshen \label{sent:Hamshen:morpho:verb:subj}
 \begin{xlist}
 \ex \gll eɡeɾem bʰeɾ-i-m nɑ \\
 if? bring-{\thgloss}-1{\sg} {\sbjv} \\
 \trans `If I bring.'\\
 \armenian{էգէրէմ բՙէրիմ նա}
 \ex \gll eɡeɾem bʰeɾ-\'e-$\emptyset$-jdə nɑ \\
 if? bring-{\thgloss}-{\pst}-2{\sg} {\sbjv} \\
 \trans `If you brought.'\\
 \armenian{էգէրէմ բՙէրէ՛յդը նա} 
 \end{xlist}
\end{exe}

This formative is also used to form a type of hortative or soft imperative (\ref{sent:Hamshen:morpho:verb:subjimp}).

\begin{exe}
 \ex Hamshen \label{sent:Hamshen:morpho:verb:subjimp}
 \gll bʰeɾ-i-s nɑ \\
 bring-{\thgloss}-2{\sg} {\sbjv} \\
 \trans `If it's possible, bring it!.' \\
 \armenian{բՙէրիս նա}
 
 
\end{exe}

\subsubsection{Present perfect and past perfect}

\translatorHD{In SWA, the present perfect and past perfect are formed periphrastically. In the default case, the verb is in the resultative participle form with the suffix /-ɑd͡z/. This participle is combined with either the present auxiliary (to mark the present perfect) or the past auxiliary (to mark the past perfect). To mark evidentiality, the verb can instead use the evidential participle with suffix /-eɾ/. Given this information, we can now better understand how Hamshen works. }

The present perfect and past perfect (\armenian{յարակատար ու գերակատար}) are formed with either the verb /em/ <\armenian{եմ}> `{\aux}; to be' or /unim/ <\armenian{ունիմ}> `to have', and with the participle suffixes /-ɑd͡z, -eɾ/ <-\armenian{ած}, -\armenian{եր}> (\ref{Hamshen:morpho:verb:presPerfect}). \translatorHD{Note that the morphemes he lists are with SWA pronunciation, not Hamshen. The sentences are Hamshen. Sentence (\ref{Hamshen:morpho:verb:presPerfect:incompelte}) is incompletely suggested by Adjarian} 

\begin{exe}
 \ex \label{Hamshen:morpho:verb:presPerfect}
 \begin{xlist}
 \ex Hamshen 
 \begin{xlist}
 \ex \gll ɡʰn-ɑ-t͡sʰ-iɾ i-m \\
 go-{\thgloss}-{\aor}-{\eptcp} {\aux}-1{\sg} \\
 \trans `I have gone.' \\
 \armenian{գՙնացիր իմ}
 \ex \gll ɡʰn-ɑ-t͡sʰ-ɑd͡z i-m \\
 go-{\thgloss}-{\aor}-{\rptcp} {\aux}-1{\sg} \\
 \trans `I have gone.' \\
 \armenian{գՙնացաձ իմ}
 \ex \gll ɡʰn-ɑ-t͡sʰ-ɑd͡z un-i-m \\
 go-{\thgloss}-{\aor}-{\rptcp} have-{\thgloss}-1{\sg} \\
 \trans `I have gone.' \label{Hamshen:morpho:verb:presPerfect:incompelte}\\
 \armenian{գՙնացաձ ունիմ}
 \ex \gll ɑsd-ɑd͡z un-ej-i-$\emptyset$ \\
 say-{\rptcp} have-{\thgloss}-{\pst}-1{\sg} \\
 \trans `I had said.' \\
 \armenian{ասդաձ ունէյի} 
 \end{xlist}
 \ex cf. SWA 
 \begin{xlist}
 \ex \gll kʰɑt͡sʰ-eɾ e-m \\
 go-{\thgloss}-{\aor}-{\eptcp} {\aux}-1{\sg} \\
 \trans `I have gone.' \\
 \armenian{գացեր իմ}
 \ex \gll kɑt͡sʰ-ɑd͡z i-m \\
 go-{\rptcp} {\aux}-1{\sg} \\
 \trans `I have gone.' \\
 \armenian{գացած եմ}
 \ex \gll əs-ɑd͡z ej-i-$\emptyset$ \\
 say-{\rptcp} {\aux}-{\pst}-1{\sg} \\
 \trans `I had said.' \\
 \armenian{ըսած էի} 
 \end{xlist}
 \end{xlist}
\end{exe}

\subsubsection{Infinitives with /-uʃ/ <\armenian{ուշ}>}

\translatorHD{In SWA, the infinitive form of the verb ends in the theme vowel plus the infinitive suffix /-l/. }

For the infinitive, the Classical endings /-e-l, -i-l, -ɑ-l, -u-l/ <\armenian{ել, իլ, ալ, ուլ}> have been lost; in their place, there is a new formative /-uʃ/ <\armenian{ուշ}> that is general for all verbs (Table \ref{tab:Hamshen:morpho:verb:inf}).

\translatorHD{To clarify, in CA and SWA/SEA, the infinitive of a verb is formed by adding the infinitive suffix \textit{-l} after the theme vowel. In Hamshen however, the infinitive uses the suffix /-uʃ/ without a theme vowel.}


\begin{table}[H]
\centering
\caption{Replacement of the infinitive suffix with /-uʃ/ <\armenian{ուշ}> in the Hamshen dialect}
\label{tab:Hamshen:morpho:verb:inf}
\begin{tabular}{|l| ll|ll| ll|ll|}
\hline & \multicolumn{2}{l|}{Classical Armenian} &\multicolumn{2}{l|}{> Hamshen} & \multicolumn{2}{l|}{cf. SEA} & \multicolumn{2}{l|}{cf. SWA} \\ 
`to speak' &χɑu̯s-i-l & \armenian{խաւսիլ} & χos-uʃ & \armenian{խօսուշ} &χos-e-l & \armenian{խոսել}&χos-i-l & \armenian{խօսիլ} \\
 `to go' & eɾtʰ-ɑ-l & \armenian{երթալ} & eʃd-uʃ & \armenian{էշդուշ} & jeɾtʰ-ɑ-l & \armenian{երթալ}& jeɾtʰ-ɑ-l & \armenian{երթալ} \\
 \armenian{՝}to bring' & beɾ-e-l & \armenian{բերել} & bʰeɾ-uʃ &\armenian{բՙէրուշ} & beɾ-e-l & \armenian{բերել} & pʰeɾ-e-l & \armenian{բերել} \\
& \multicolumn{2}{l|}{$\sqrt{}$-{\thgloss}-{\infgloss}}
& \multicolumn{2}{l|}{$\sqrt{}$-{\infgloss}}
& \multicolumn{2}{l|}{$\sqrt{}$-{\thgloss}-{\infgloss}}
& \multicolumn{2}{l|}{$\sqrt{}$-{\thgloss}-{\infgloss}}
\\
\hline 
\end{tabular}
\end{table}



\begin{adjarianpage}\label{page:189}\end{adjarianpage}% should be 189

In declension, this same formative is used (Table \ref{tab:Hamshen:morpho:verb:infDec}). 


\begin{table}[H]
\centering
\caption{Declension of infinitives like `to bring' in the Hamshen dialect}
\label{tab:Hamshen:morpho:verb:infDec}
\begin{tabular}{|l| ll|ll| }
\hline & \multicolumn{2}{l|}{Singular } &\multicolumn{2}{l|}{Plural} \\
\hline 
{\nom}-{\acc} & bʰeɾ-uʃ & \armenian{բՙէրուշ} & bʰeɾ-uʃ-nin & \armenian{բՙէրուշնին} \\
{\gen}-{\dat} & bʰeɾ-uʃ-i & \armenian{բՙէրուշի} & bʰeɾ-uʃ-nun & \armenian{բՙէրուշնուն} \\
{\abl} & bʰeɾ-uʃ-ən & \armenian{բՙէրուշըն} & bʰeɾ-uʃ-nun & \armenian{բՙէրուշնուն} \\
{\ins} & bʰeɾ-uʃ-ov & \armenian{բՙէրուշօվ} & bʰeɾ-uʃ-neɾ-ov & \armenian{բՙէրուշնէրօվ}
\\ 

\hline 
\end{tabular}
\end{table}

It appears to me that this formative is borrowed from Turkish formatives /-iʃ, -əʃ, -uʃ/ <\armenian{իշ, ըշ, ուշ}> (\translatorHD{deverbal nominal suffix <-iş> in modern Turkish orthography}) and from the Persian formatives /-iʃ/ <\armenian{իշ}> (\translatorHD{deverbal nominal suffix <-eš> \textarab{ـش} in modern Persian orthography}); these likewise form participles. For example Turkish /ɑləʃ-veɾiʃ/ <\armenian{ալըշ-վէրիշ}> (\translatorHD{<alışveriş>}) meaning `trade' (that is `to take and to give'), Persian \armenian{առեւտուն} (that is, \armenian{առումն եւ տալն}), Persian /ɑsɑjiʃ/ <\armenian{ասայիշ}> (\translatorHD{<âsâyeš> \textarab{اسايش}}) meaning `rest'. 

\subsubsection{Negation or negative forms}

\paragraph{Indicative present and past imperfective}
Negative forms are built by placing the formatives /t͡ʃʰ-, t͡ʃʰi/ <\armenian{չ, չի}> before the verb, or by placing the formative /ut͡ʃʰ/ <\armenian{ուչ}> after the verb.

\translatorHD{To clarify, the first method resembles how SWA does negative forms, while the second method is about using the reflex of CA `no' /ot͡ʃʰ/ <\armenian{ոչ}> as a post-verbal marker.}

\translatorHD{Consider a verb like `to bring' to illustrate the formation of the negative indicative present (Table \ref{tab:Hamshen:morpho:verb:paradigm:negPres}). For the first method, SWA combines the negative present auxiliary with a non-finite form called the connegative. The connegative is formed by adding the suffix /-ɾ/ after the theme vowel. Hamshen also uses a negative present auxiliary, while the verb's non-finite form uses the suffix \textit{-il}. This \textit{-il} seems to be decomposable to a theme vowel /-i/ plus a suffix /-l/; such that this Hamshen /-l/ suffix is a reflex of the CA infinitive suffix /-l/ <\armenian{լ}>. }


\begin{table}[H]
 	\centering 
 	\caption{Negative <\armenian{բացասական}> of the indicative present <\armenian{ներկայ}> of the verb `to bring' in the Hamshen dialect}
 	\label{tab:Hamshen:morpho:verb:paradigm:negPres}
 \begin{tabular}{|l|ll| ll| }
 		\hline & \multicolumn{2}{l|}{Hamshen} & \multicolumn{2}{l|}{cf. SWA} \\ \hline 
1SG &t͡ʃʰ-i-m bʰeɾ-i-l & \armenian{չիմ բՙէրիլ} & t͡ʃʰ-e-m pʰeɾ-e-ɾ& \armenian{չեմ բերեր} \\
2SG &t͡ʃʰ-i-s bʰeɾ-i-l & \armenian{չիս բՙէրիլ} & t͡ʃʰ-e-s pʰeɾ-e-ɾ& \armenian{չես բերեր}\\
3SG &t͡ʃʰ-i-$\emptyset$ bʰeɾ-i-l & \armenian{չի բՙէրիլ}& t͡ʃʰ-i-$\emptyset$ pʰeɾ-e-ɾ& \armenian{չի բերեր} \\
1PL &t͡ʃʰ-i-kʰ bʰeɾ-i-l & \armenian{չիք բՙէրիլ} & t͡ʃʰ-e-ŋkʰ pʰeɾ-e-ɾ& \armenian{չենք բերեր}\\
2PL &t͡ʃʰ-e-kʰ bʰeɾ-e-l & \armenian{չէք բՙէրիլ} & t͡ʃʰ-e-kʰ pʰeɾ-e-ɾ & \armenian{չէք բերեր} \\
3PL &t͡ʃʰ-i-n bʰeɾ-i-l & \armenian{չին բՙէրիլ} & t͡ʃʰ-e-n pʰeɾ-e-ɾ& \armenian{չեն բերեր} \\
& \multicolumn{2}{l|}{{\neggloss}-{\aux}-{\agr} $\sqrt{}$-{\thgloss}-{\cn}}& \multicolumn{2}{l|}{{\neggloss}-{\aux}-{\agr} $\sqrt{}$-{\thgloss}-{\cn}}\\
\hline 
\end{tabular}
\end{table}

\translatorHD{Similarly for the negation of the indicative past imperfective (Table \ref{tab:Hamshen:morpho:verb:paradigm:negPastImpf}), SWA combines the negative past auxiliary with the above non-finite form. Hamshen behaves the same. }



\begin{table}[H]
 	\centering 
 	\caption{Negative <\armenian{բացասական}> of the indicative past imperfective <\armenian{անկատար}> of the verb `to bring' in the Hamshen dialect}
 	\label{tab:Hamshen:morpho:verb:paradigm:negPastImpf}
 \begin{tabular}{|l|ll| ll| }
 		\hline & \multicolumn{2}{l|}{Hamshen} & \multicolumn{2}{l|}{cf. SWA} \\ \hline 
1SG &t͡ʃʰ-\'ej-ə-$\emptyset$ bʰeɾ-i-l & \armenian{չէ՛յը բՙէրիլ} & t͡ʃʰ-ej-\'i-$\emptyset$ pʰeɾ-e-ɾ& \armenian{չէի բերեր} \\
2SG &t͡ʃʰ-\'ej-ə-ɾ bʰeɾ-i-l & \armenian{չէ՛յըր բՙէրիլ} & t͡ʃʰ-ej-\'i-ɾ pʰeɾ-e-ɾ& \armenian{չէիր բերեր}\\
3SG &t͡ʃʰ-i-$\emptyset$-ɾ bʰeɾ-i-l & \armenian{չիր բՙէրիլ}& t͡ʃʰ-e-$\emptyset$-ɾ pʰeɾ-e-ɾ& \armenian{չէր բերեր} \\
1PL &t͡ʃʰ-\'ɑ-$\emptyset$-kʰə bʰeɾ-i-l & \armenian{չա՛քը բՙէրիլ} & t͡ʃʰ-ej-\'i-ŋkʰ pʰeɾ-e-ɾ& \armenian{չէինք բերեր}\\
2PL &t͡ʃʰ-\'e-$\emptyset$-kʰə bʰeɾ-e-l & \armenian{չէ՛քը բՙէրիլ} & t͡ʃʰ-ej-\'i-kʰ pʰeɾ-e-ɾ & \armenian{չէիք բերեր} \\
3PL &t͡ʃʰ-\'e-$\emptyset$-jnə bʰeɾ-i-l & \armenian{չէ՛յնը բՙէրիլ} & t͡ʃʰ-ej-\'i-n pʰeɾ-e-ɾ& \armenian{չէին բերեր} \\
& \multicolumn{2}{l|}{{\neggloss}-{\aux}-{\pst}-{\agr} $\sqrt{}$-{\thgloss}-{\cn}}& \multicolumn{2}{l|}{{\neggloss}-{\aux}-{\pst}-{\agr} $\sqrt{}$-{\thgloss}-{\cn}}\\
\hline 
\end{tabular}
\end{table}

\translatorHD{Note how the two dialects use different formatives to form the negated auxiliary. The segmentation is difficult to verify; see similar problems for the indicative past imperfective (\S\ref{sec:Hamshen:morphology:verb:indcprespst}). }

\paragraph{Past perfective and auxiliary}\label{sec:Hamshen:morphology:verb:neg:pastperfectiveaux}

The perfective has two forms. 


\translatorHD{To negate the past perfective, SWA places the negation prefix /t͡ʃʰə-/ before the verb. Hamshen in contrast has two methods. To first method (Table \ref{tab:Hamshen:morpho:verb:paradigm:negPastPerf:1}) is like in SWA, but the negation prefix is /t͡ʃʰi-/.\footnote{\translatorHD{Note that in colloquial SWA, the negation prefix /t͡ʃʰə-/ can be pronounced as /t͡ʃʰi-/ as well. }}.}







\begin{table}[H]
 	\centering 
 	\caption{Negative <\armenian{բացասական}> of the past perfective <\armenian{կատարեալ}> of the verb `to bring' in the Hamshen dialect (Method 1)}
 	\label{tab:Hamshen:morpho:verb:paradigm:negPastPerf:1}
 \begin{tabular}{|l|ll| ll| }
 		\hline & \multicolumn{2}{l|}{Hamshen} & \multicolumn{2}{l|}{cf. SWA} \\ \hline 
1SG &t͡ʃʰi bʰeɾ-i-$\emptyset$ & \armenian{չի բՙէրի} & t͡ʃʰə-pʰeɾ-i-$\emptyset$ & \armenian{չբերի}\\
2SG &t͡ʃʰi bʰeɾ-i-ɾ &\armenian{չի բՙէրիր} &t͡ʃʰə-pʰeɾ-i-ɾ& \armenian{չբերիր}\\
3SG &t͡ʃʰi bʰeɾ-ɑ-v & \armenian{չի բՙէրավ}& t͡ʃʰə-pʰeɾ-ɑ-v & \armenian{չէր բերաւ} \\
1PL &t͡ʃʰi bʰeɾ-ɑ-kʰ & \armenian{չի բՙէրաք} & t͡ʃʰə-pʰeɾ-i-i-ŋkʰ & \armenian{չբերինք}\\
2PL &t͡ʃʰi bʰeɾ-i-kʰ & \armenian{չի բՙէրիք} & t͡ʃʰə-pʰeɾ-i-kʰ & \armenian{չբերիք} \\
3PL &t͡ʃʰi bʰeɾ-i-n & \armenian{չի բՙէրին} & t͡ʃʰə-pʰeɾ-i-n & \armenian{չբերին} \\
& \multicolumn{2}{l|}{{\neggloss} $\sqrt{}$-{\pst}-{\agr}}& \multicolumn{2}{l|}{{\neggloss}-$\sqrt{}$-{\pst}-{\agr}}\\
\hline 
\end{tabular}
\end{table}










\translatorHD{The second method is to place the reflex of the CA word `no' /ot͡ʃʰ/ <\armenian{ոչ}> after the verb (Table \ref{tab:Hamshen:morpho:verb:paradigm:negPastPerf:2}). }




\begin{table}[H]
 	\centering 
 	\caption{Negative <\armenian{բացասական}> of the past perfective <\armenian{կատարեալ}> of the verb `to bring' in the Hamshen dialect (Method 2)}
 	\label{tab:Hamshen:morpho:verb:paradigm:negPastPerf:2}
 \begin{tabular}{|l|ll| ll| }
 		\hline & \multicolumn{2}{l|}{Hamshen} & \multicolumn{2}{l|}{cf. SWA} \\ \hline 
1SG & bʰeɾ-i-$\emptyset$ ut͡ʃʰ & \armenian{բՙէրի ուչ} & t͡ʃʰə-pʰeɾ-i-$\emptyset$ & \armenian{չբերի}\\
2SG & bʰeɾ-i-ɾ ut͡ʃʰ& \armenian{բՙէրիր ուչ}&t͡ʃʰə-pʰeɾ-i-ɾ& \armenian{չբերիր}\\
3SG & bʰeɾ-ɑ-v ut͡ʃʰ& \armenian{բՙէրավ ուչ}& t͡ʃʰə-pʰeɾ-ɑ-v & \armenian{չէր բերաւ} \\
1PL & bʰeɾ-ɑ-kʰ ut͡ʃʰ& \armenian{բՙէրաք ուչ} & t͡ʃʰə-pʰeɾ-i-i-ŋkʰ & \armenian{չբերինք}\\
2PL & bʰeɾ-i-kʰ ut͡ʃʰ & \armenian{բՙէրիք ուչ} & t͡ʃʰə-pʰeɾ-i-kʰ & \armenian{չբերիք} \\
3PL & bʰeɾ-i-n ut͡ʃʰ & \armenian{բՙէրին ուչ} & t͡ʃʰə-pʰeɾ-i-n & \armenian{չբերին} \\
& \multicolumn{2}{l|}{{\neggloss} $\sqrt{}$-{\pst}-{\agr}}& \multicolumn{2}{l|}{{\neggloss}-$\sqrt{}$-{\pst}-{\agr}}\\
\hline 
\end{tabular}
\end{table}
\paragraph{Future}

The future has three forms. 

\translatorHD{In SWA, the future is negated by placing the negation prefix /t͡ʃʰə-/ between the future morpheme /bidi/ and the finite verb (\ref{sent:Hamshen:morpho:verb:negfut:SWA:bidineg}). Colloquial SWA also allows placing the negation prefix before the future morpheme (\ref{sent:Hamshen:morpho:verb:negfut:SWA:negbidi}). }


\begin{exe}
 \ex SWA\label{sent:Hamshen:morpho:verb:negfut:SWA}
 \begin{xlist}
 \ex \gll bidi pʰeɾ-e-m \\
 {\fut} bring-{\thgloss}-1{\sg} \\
 \trans `I will bring.' \\
 \armenian{պիտի բերեմ}
 \ex \gll bidi t͡ʃʰə-pʰeɾ-e-m \\
 {\fut} {\neggloss}-bring-{\thgloss}-1{\sg} \\
 \trans `I will not bring.' \label{sent:Hamshen:morpho:verb:negfut:SWA:bidineg}\\
 \armenian{պիտի չբերեմ}
 \ex \gll t͡ʃʰə-bidi pʰeɾ-e-m \\
 {\neggloss}-{\fut} bring-{\thgloss}-1{\sg} \\
 \trans `I will not bring.'\label{sent:Hamshen:morpho:verb:negfut:SWA:negbidi}\\
 \armenian{չպիտի բերեմ}
 \end{xlist}
\end{exe}

\translatorHD{In contrast, Hamshen seems to have three possible strategies. The first is to place the negation morpheme /t͡ʃʰi/ before the future morpheme, and then add the verb (\ref{sent:Hamshen:morpho:verb:negfut:Hamshen:1}).} 


\begin{exe}
 \ex Hamshen\label{sent:Hamshen:morpho:verb:negfut:Hamshen:1}
 \begin{xlist}
 \ex \gll t͡ʃʰ\'i bidi bʰeɾ-i-m \\
 {\neggloss} {\fut} bring-{\thgloss}-1{\sg} \\
 \trans `I will not bring.' \\
 \armenian{չի՛ բիդի բՙէրիմ}
 \ex \gll t͡ʃʰ\'i bidi bʰeɾ-i-s \\
 {\neggloss} {\fut} bring-{\thgloss}-2{\sg} \\
 \trans `You will not bring.' \\
 \armenian{չի՛ բիդի բՙէրիս}
 \end{xlist}
\end{exe}






\translatorHD{The second is to place the reflex of `no' between the verb and the future morpheme (\ref{sent:Hamshen:morpho:verb:negfut:Hamshen:2}).}


\begin{exe}
 \ex Hamshen\label{sent:Hamshen:morpho:verb:negfut:Hamshen:2}
 \begin{xlist}
 \ex \gll bʰeɾ-i-m \'ut͡ʃʰ bidi\\
 bring-{\thgloss}-1{\sg} {\neggloss} {\fut} \\
 \trans `I will not bring.' \\
\armenian{բՙէրիմ ո՛ւչ բիդի}
 \ex \gll bʰeɾ-i-s \'ut͡ʃʰ bidi \\
 bring-{\thgloss}-2{\sg} {\neggloss} {\fut} \\
 \trans `You will not bring.' \\
\armenian{բՙէրիս ո՛ւչ բիդի}
 \end{xlist}
\end{exe}




\translatorHD{The third is to place the reflex of `no' after the verb and future morpheme (\ref{sent:Hamshen:morpho:verb:negfut:Hamshen:3}. }


\begin{exe}
 \ex Hamshen\label{sent:Hamshen:morpho:verb:negfut:Hamshen:3}
 \begin{xlist}
 \ex \gll bʰeɾ-i-m idi \'ut͡ʃʰ \\
 bring-{\thgloss}-1{\sg} {\fut} {\neggloss} \\
 \trans `I will not bring.' \\
\armenian{բՙէրիմ իդի ո՛ւչ}
 \ex \gll bʰeɾ-i-s bidi \'ut͡ʃʰ \\
 bring-{\thgloss}-2{\sg} {\fut} {\neggloss} \\
 \trans `You will not bring.' \\
\armenian{բՙէրիս բիդի ո՛ւչ}
 \end{xlist}
\end{exe}


\section{Miscellaneous}

\subsection{Question formation}

The interrogative is built with the formative /tʰe/ <\armenian{թէ}>, which can take various positions. For example, all the sentences in (\ref{sent:Hamshen:miscell:question}) all equally mean `Aren't they coming?'. 

\begin{exe}
 \ex Hamshen \label{sent:Hamshen:miscell:question}
 \begin{xlist}
 \ex \gll t͡ʃʰ-i-n tʰe ɡʰ-ɑ-l \\
 {\neggloss}-{\aux}-3{\pl} {\q} come-{\thgloss}-{\cn} \\
 \trans `Aren't they coming?' \\
 \armenian{չի՞ն թէ գՙալ}
 \ex \gll ɡʰ-ɑ-l t͡ʃʰ-i-n tʰe \\
 come-{\thgloss}-{\cn} {\neggloss}-{\aux}-3{\pl} {\q} \\
 \trans `Aren't they coming?' \\
 \armenian{գՙալ չի՞ն թէ}
 \ex \gll t͡ʃʰ-i-n ɡʰ-ɑ-l tʰe \\
 {\neggloss}-{\aux}-3{\pl} come-{\thgloss}-{\cn} {\q} \\
 \trans `Aren't they coming?' \\
 \armenian{չի՞ն գՙալ թէ} 
 \end{xlist}
\end{exe}



But, if before the verb there is an interrogative pronoun (\armenian{միջակ անուն}) or adverb, then the formative /tʰe/ <\armenian{թէ}>... 


\begin{adjarianpage}\label{page:190}\end{adjarianpage}% should be 190

... is not used (\ref{sent:Hamshen:miscell:whquestion}). 

\begin{exe}
 \ex Hamshen \label{sent:Hamshen:miscell:whquestion}
\begin{xlist}
 \ex \gll kʰoni hokʰi ji-kʰ \\
 how.,many person {\aux}-2{\pl} \\
 \trans `How many people are you?' \\
 \armenian{քօնի՞ հօքի յիք} 
 \ex \gll int͡ʃʰo eɾ-i-kʰ \\
 how do-{\pst}-2{\pl} \\
 \trans `How did you do?' \\
 \armenian{ի՞նչօ էրիք} 
 \ex \gll vœɾ mɑʃd-ə eɡ-ɑ-v \\
 which man-{\defgloss} come-{\pst}-3{\sg} \\
\trans `Which man came?' \\
\armenian{վէօ՞ր մաշդը էգավ} 
\ex \gll h\'obʰoɾ oɾ ɡʰ-ɑ-l t͡ʃʰ-\'e-$\emptyset$-jdə, oɾi χoskʰ ɡu-d-\'e-$\emptyset$-jdə \\
when that? come-{\thgloss}-{\cn} {\neggloss}-{\aux}-{\pst}-2{\sg}, why speech {\ind}-give-{\thgloss}-{\pst}-2{\sg} \\
\trans `When you weren't coming, why did you promise it?' \\
\armenian{հօ՛բՙօր օր գՙալ չէ՛յդը, օրի՞ խօսք գուդէ՛յդը} 

\end{xlist}
\end{exe}





In contrast, what is said is (\ref{sent:Hamshen:miscell:notwhquestion}). 

\begin{exe}
 \ex Hamshen\label{sent:Hamshen:miscell:notwhquestion}
 \begin{xlist}
 \ex \gll esɑ d-o-m tʰe \\
 this give-{\thgloss}-1{\sg} {\q} \\
 \trans Do I give \underline{this}? (as opposed to something else)' \\
 \armenian{էսա՞ դօմ թէ}
 \ex \gll mekʰ tʰe \\
 we {\q} \\
 \trans `Us?' \\
 \armenian{մէ՞ք թէ}
 \ex \gll dun ɡʰɑt͡sʰ-i-ɾ tʰə \\
 house go-{\pst}-2{\sg} {\q} \\
 \trans `Did you go \underline{home}? (as opposed to someone else)'\\


 \end{xlist}
\end{exe}


\subsection{Stress}

In the dialect, another famous phenomenon is stress. Just as in Trabzon, likewise in Hamshen, stress... 


\begin{adjarianpage}\label{page:191}\end{adjarianpage}% should be 191

... is by rule on the final syllable. But in fast speech, it happens many times that in the two dialects, stress has moved to the first syllable of the word. This is due to the influence of the language of the Laz people who are a native Pontic populace. The Laz language places stress on the first syllable. Although in many places the Laz have lost their mother language and speak Turkish, but they stress their Turkish with their previous stress rule. In this word, the stress of the Laz language has passed to Turkish and from this into Armenian. 


\subsection{Borrowing Turkish morphology}

The Hamshen dialect also has a strange characteristic which does not exist in any other Armenian dialect, nor do I think in any other language.\footnote{\translatorHD{My debt to Tabita Toparlak for providing the modern Turkish translation (and diachronic sources) for the Hamshen data here. }}


As we know, every language has foreign borrowed words. But these borrowings are taken with such a form, that the borrowing language considers them as roots and can subject them to grammatical rules. If the borrowings are nouns or adjectives, then they are taken in the simplest nominative case-form; if they are verbs, they are taken in the form of participles; if it is any other unchanging form, then they are likewise taken in their simplest root form. All of these can be declined or conjugated. For example, the following sentence is made up of purely Turkish borrowings (\ref{sent:Hamshen:misc:borrowing:sentenceBig}). 

\begin{exe}
 	\ex \begin{xlist}
 		\ex Hamshen \label{sent:Hamshen:misc:borrowing:sentenceBig}
 		\gll sɑ jenit͡ʃʰeɾi-i-n χɑlpʰɑχ-i-n tʰekʰme mə jeɾləʃdiɾ-miʃ ən-e-m kʰi tʰekʰeɾ-mekʰeɾ ɡɑ \\ 
 		this janissary-{\gen}-{\defgloss} calpack-{\dat}-{\defgloss} kick {\indf} place-? do-{\thgloss}-1{\sg} so wheel-{\echo} come-{\thgloss}-3{\sg}\\ 
 		\trans `I kick this janissary's calpack so that it comes all rolling.' \\ 
 		\armenian{Սա յէնիչէրիին խալփախին թէքմէ մը յէրլէշդիրմիշ ընէմ քի թէքէր-մէքէր գա} 
 		\ex cf. Turkish 
 		\gll Şu yeniçeri-nin kalpağ-ın-a bir tekme yerleştir-eyim, ki teker-meker gel-sin \\ 
 		this janissary-{\gen} calpack-{\poss}-{\dat} a kick place-{\opt}.1{\sg}, so wheel-{\echo} come-{\imp}.3{\sg} \\ 
 		\trans `I kick this janissary's calpack so that it comes all rolling.' 
 	\end{xlist}
\end{exe}



Here, the words are Turkish, but they are declined or conjugated as Armenian words. In Hamshen, it often happens that the borrowed words are conjugated according to Turkish grammar, and they are imported in this way into Armenian sentences (\ref{sent:Hamshen:misc:borrowing:sentence}). \translatorHD{I placed the borrowed words in bold.}

\begin{exe}
 \ex Hamshen \label{sent:Hamshen:misc:borrowing:sentence}


\begin{xlist}
 \ex \gll eniɾ χɑt͡ʃʰo-ɡi-n kʰit͡ʃʰ mə \textbf{jɑɾɑlɑ-di-leɾ} \\
 they Khacho-{\dat}-{\defgloss} little {\indf} injure-{\pst}-3{\pl} \\
 \trans `They injured Khacho a bit'\\
 \translatorHD{The verb is borrowed from Ottoman Turkish. Its Modern Turkish form is <yarala-dı-lar> `they injured'. } \\
 \armenian{Էնիր Խաչօգին քիչ մը յարալադիլէր}
 \ex \gll iɾɑt͡sʰ hed \textbf{uj-ɑlum} \\
 each.other with agree-{\opt}.1{\pl} \\
 \trans `Let's get along with each other.' \\
 \translatorHD{The verb is borrowed from Ottoman Turkish. Its Modern Turkish form is <uy-alım> `(optative) they get along'. } \\
 \armenian{իրաց հէդ ույալում} 
\ex eʁɑr mezi hed \textbf{koʃ-di} \\
take? we.{\dat} with combine-{\pst} \\
\trans `He combined us' (?) \label{sent:Hamshen:misc:borrowing:sentence:kosdi} \translatorHD{I didn't understood Adjarian's translation well <\armenian{առաւ մեզ հետ միացուց}>. One possible translation is `He went and combined us'. } \\
\translatorHD{The verb is from Ottoman Turkish. Its Modern Turkish form is <koş-tu> `he attached/combined'. } \\
\armenian{էղառ մէզի հէդ կօշդի} 
\ex \gll ɡʰo̯eʁ tʰe ɡʰ-ɑ-s, iɾɑt͡sʰ hed \textbf{doʁuʃ-\'uɾ-ukʰ} \\
village {\q} come-{\thgloss}-2{\sg}, each.other with fight-{\aor}-1{\pl} \\
\trans `If you come to the village, we will fight each other.'\\
\translatorHD{The verb is from Ottoman Turkish. Its Modern Turkish form is <dövüş-ür-uz> `we fight' (aorist). } \\
\armenian{գՙեղ թէ գՙաս, իրաց հէդ դօղուշո՛ւրուք} 
\ex \gll dʰunkʰ ʃɑd pʰɑɾɑ \textbf{kɑzon-\'uɾ-sunuz} \\
you.{\pl} much money earn-{\aor}-2{\pl} \\ 
\trans `Do you earn much money?'\\
\translatorHD{The verb is from Ottoman Turkish. Its Modern Turkish form is <kazan-ır-sınız> `you ear' (aorist). The noun `money' is also from Turkish <para> `money'. } \\
\armenian{դՙունք շա՞դ փարա կազօնո՛ւրսունուզ} 
\end{xlist}
\end{exe}

In these sayings, the following words are conjugated with purely Turkish rules: 
\begin{itemize}
 \item /jɑɾɑlɑdileɾ/ <\armenian{յարալադիլէր}> with past perfective 3PL
 \item /ujɑlum/ <\armenian{ույալում}> imperative 1PL 
 \item /koʃdi/ <\armenian{կօշդի}> past perfective 3SG 
 \item /doʁuʃ\'uɾukʰ/ <\armenian{դօղուշո՛ւրուք}> present 1PL 
 \item /kɑzon\'uɾsunuz/ <\armenian{կազօնո՛ւրսունուզ}> present 2PL
\end{itemize}

These would become everywhere else as in (\ref{sent:Hamshen:misc:borrow:SWA}). \translatorHD{He means that in other Armenian varieties like SWA, Turkish verbs are borrowed in some participle form ending in /-miʃ/. The suffix corresponds to the Modern Turkish suffix /-mIş/. This suffix is used to mark evidentiality in Turkish. But when used as borrowings in SWA, the suffix has no evidential meaning; the suffix is used to create a generic non-finite form (= essentially an infinitive or meaningless participle -{\ptcp}) that can be used in Armenian sentences, usually alongside a light verb like `to be' or `to do.' }

\begin{exe}
 	\ex SWA with Turkish borrowings or codeswitching \label{sent:Hamshen:misc:borrow:SWA}
 	\begin{xlist}
 		\ex \gll \textbf{jɑɾɑlɑ-miʃ} əɾ-i-n \\ 
 		injure-{\ptcp} do-{\pst}-3{\pl} \\ 
 		\trans `They injured.' \\
 		\translatorHD{The borrowed word is from Ottoman Turkish. The Modern Turkish form is <yarala-mış>.} \\
 		\armenian{յարալամիշ ըրին} 
 		\ex \gll \textbf{uj-miʃ} əll-ɑ-ŋkʰ \\ 
 		agree-{\ptcp} be-{\thgloss}-1{\pl} \\ 
 		\trans `Let us agree.' \\
 		\translatorHD{The borrowed word is from Ottoman Turkish. The Modern Turkish form is <uy-muş>.} \\
		\armenian{ույմիշ ըլլանք} 
 		\ex \gll \textbf{doʁuʃ-miʃ} ɡ-əll-ɑ-ŋkʰ \\ 
fight-{\ptcp} {\ind}-be-{\thgloss}-1{\pl} \\ 
 		\trans `We will fight.' \\
 		\translatorHD{The borrowed word is from Ottoman Turkish. The Modern Turkish form is <dövüş-müş>.} \\
		\armenian{դօղուշմիշ կըլլանք} 
 		\ex \gll \textbf{koʃmiʃ} əɾ-ɑ-v \\ 
combine-{\ptcp} do-{\pst}-3{\sg} \\ 
 		\trans `He combined/attached.' \\
 		\translatorHD{The borrowed word is from Ottoman Turkish. The Modern Turkish form is <koş-müş>.} \\
 		\armenian{կօշմիշ ըրաւ} 
 		\ex \gll \textbf{kɑzɑnmiʃ} ɡ-əll-ɑ-kʰ \\ 
earn-{\ptcp} {\ind}-be-{\thgloss}-2{\pl} \\ 
 		\trans `We earn.' \\
 		\translatorHD{The borrowed word is from Ottoman Turkish. The Modern Turkish form is <kazan-mış>.} \\
 		\armenian{կազանմիշ կըլլաք։} 
 		 
 	\end{xlist}
\end{exe}

\section{Text samples}

{\sampleoverview}

 \subsection{Zefanos village}

Adjarian's source: See Ararat (\armenian{Արարատ}), 1892, page 428. 



\armenian{Էքուց կալանդար (կաղանդ) է. էս օր էմէն մաշդիգ վիր-վէր, նէս-նէն գէշդօն գուքօն. էմէն մաշդ իսթուս ինթուս թռչի գու։ Էմօն դադին գու օր՝ քըյդինքի մէչ գօյսըվին գու։}

\armenian{Էմէն մաշդ ժաժվի գու ասդի (ասացի) ՝ օր օնցաձ դարվօնէ ինչիգ մը բագաս թօղուն ուչ. յիս էլ բարաբ դօղնիլ չիմ. նէդվիմ գու նէս նէն, ինձիգի ինչ բօն օր ասդաձ ին՝ զէն ընիմ իդի, օր էշդօմ միր դընվօյնուն (տնւօրներուն) հիդ կալանդար ընիմ։}

\armenian{Կալանդարը բէդքը բօն է. ինցօ՞ սիյդըս ֆըռֆըռա գու թէ յի՞փ հասնիմ իդի իրիգվօն. ինցօ՞ բէդ գընիմ (կ՚սպասեմ) թէ մէգ մը իրիգվօն հասնէի։}

\armenian{Գիդի՞ս ինչ բօնի հօմար գուզիմ կալանդարը. իրիգվօն չէրէզ շադ ուդինք բիդի. կալանդարի ձառ զայթարինք բիդի. խընձիւրի մէչ փարա դընինք բիդի. միր բէդքը (լաւ) յիզը գումէն դուն բէրինք աշինք բիդի սա՞ղ թէ սօլ օդքը նիյս դընէ բիդի. գօդօշվընուն ձէրը լուցաձ մէղրէ մում գըբցընինք բիդի. էրգու գօդօշնուն վրէն էլ սիմիթ օնցընինք բիդի։}

\armenian{Էքվօն կալանդար է ասդաձ ունիմ. հիմիջաք միր բօնն է կալանդար էնուշի հօմար էմէն ինչիգ հազր ընուշ. միր բօնը շադ}... 



\begin{adjarianpage}\label{page:192}\end{adjarianpage}% should be 192

... \armenian{չէթին էր, էնուր հօմար օր միր դընվօյնին քիչվօր էին, ու մէզիգի բօն շադ գէր (կար)։}

\armenian{Միր դօնը շադ մաշդ բի գէր. յիս ունէի ախբէր մը ու մէյըս. հէյըս իմ չիւյս դարէգօն էղաձ վախդըս մէռած էր. մէյս՝ հօյս մէռնուշէն էրգու դարի յիդ ՝ գընաց Վէրանա գարքըվէցավ. մէզիգի Ագօփ հօխբէյըս բէհից, աշից ինչաք միձըցօնք։}

\armenian{Շադ բիջիլիգ էի. միդքըս գուքա օր հօխբէյըս գարքըվէցավ. հօղօփգինց Գուխլա գէղէն հայս բէրաձ ունէին. յիփ ձիէն վէր առին ՝ քօխքը գլխուն նսդավ, ինձիգի գօքը նըսդէցուցին. հօղօփգինս ինձիգի օնթից ու բաքնից։}

\armenian{Հօխբէյս մէզիգի շադ սիրէր գու. գասէր թէ իմ ախբօր դէղն ին. ամմա հօղօփգինս էմէն դարբա մէզիգի քէօթգէր գու. էմէն դարբա միր բօնը լացուշ էր։}

\subsection{Küçük Şana village}

\armenian{Այս եւ յաջորդները իմ հաւաքածներս են տեղացի ուսուցիչներէն եւ գիւղացիներէն։}

– \armenian{Բՙարիվս քէ, Աթօմ, ո՞ւսդի (վո՞ւյ դէղէն) գՙուգՙաս։}

– \armenian{Քախքը՛ն։}

– \armenian{Ի՞նցօ իս, բէ՞դք իս թէ։}

– \armenian{Բէդք իմ, ի՞նչ ընիմ իդի։}

– \armenian{Բՙօնի՞յդ ինցօ ին։}

– \armenian{Գէշ չին. դՙո՞ւնք ինցօ էք. աս դարի ի՞նցօ օնցուցիք։}

– \armenian{Ֆուխարէ մաշդը ի՞նչ գայնա ընիլ. գիդիս օր խէօղ չունիմ. մէգ գդէօր մը խէօզ ունէյի, էն էլ բօրջիս դէղ ձՙէռնէս առին. գՙնացի միր աղայէն մարաբալուղի հումար խաձ (քիչ, կտոր) մը խէօղ առի. էնու վրա էլ ը՛նղըդար էմէղ էրի օր, հիչ հսաբի չի գՙալ. փօրէցի, քօքը փէդէցի, իսդգէցի, թէմիզ մը զիբլէցի, մէկ փարչը՛ն լազուդ ցօնէցի, վէօյն էլ լոբգյէ ցօնէցի}, X\armenian{ըյնէգն էլ դնթում-մնթում սադրէցի. դնթմընին ու լօբգէնին էփէյի էղօն ամա, լազդ հիչ չէղավ. էղաձ իրադն էլ դարի քաղաքը- ձախէցի, անջաք ընիցմէրօվ բօրջիս գէսը դըվի. մէգէլ գէսն էլ բՙաց մնաց։}

– \armenian{Դունդՙ քօ՞նի հօքի յիք։}


\begin{adjarianpage}\label{page:193}\end{adjarianpage}% should be 193

– \armenian{Վից հօքի յինք, յէս, դղօցը մէրը, չիւյս էլ դղաքը։}

– \armenian{Դղաքդՙ մօ՞նչ ին թէ ախչիգ։}

– \armenian{Էրգու մօնչ, էրգու ախչիգ ին. ախչգընիս իրէք էինը ամա, մէգը մէռավ։}

– \armenian{Դղաքդ, գաղդօ՞ն գու թէ։}

– \armenian{Հա, հէլբէթ. մօնչիյն էլ գաշդօն գու, բուլգընին էլ (պուլիկ «աղջիկ»}). \armenian{միձ մօնչըս դասվիրէք դարէգօն է, բզդիգը դասնըմէգ. առաչմէն բուլգընիս գաշդում դվուշի չուզէցի (ուզէցի էօչ գաշդում գալ}). \armenian{«բուլգօն գաշդուշը ի՞նչ բէդք է» գասէյը՛. ամա քաղաքէն էգաձ վարջաբէդը (վարբէդը) շադ ասաց, շադ թէքլիֆ էրավ, աքըր աքբէթ յէս էլ կանմիշ էղա, դահա ինչիգ չա՛սդի։}

\subsection{Malya village}


– \armenian{Գՙիրքէօր, արի, քիչ մը նսդիք, ինչի մը հալլաշալում։}

– \armenian{Ի՞նչ հալլաշաջա՛ղուք, մէք ինչիգ չիյդիք օր. մէզի հէդ ի՞նչ հալլաշաջա՛քսուն։}

– \armenian{Բադմէ աշիք, ի՞նչօ էղավ Դալդաբօնի բՙօնը։}

– \armenian{Մա̈յիսին գիսուն էր՝ գՙնացաք էլաք Դալդաբօնը. մեր ա̈ձէ՛ն ու օխչըյնին օնցընաքը բիդի. Թուրքըրը չի՛ք թօղուլ ասդին. ընդէղէն դՙարձՙուցին յէդ. խէլ մը յէդ էգաք. Քիւրդալօղլի գՙեղը էգաք. ընդէղը զարգին մէզիգի. մէք ա̈լ գբուցաք, էնիր էլ զբուցին։}

– \armenian{Վո՞ւմ շադ քյէօթգէցին։}

– \armenian{Կօքիս ա̈լ քյէօթգէցին. կալդի քի էնիր Խաչօգին քիչ մը յարալադիլէր. մէք ա̈լ էնից քյէօթգէցաք։}

– \armenian{Ընդէղէն նի՞ւր օնցա̈ք։}

– \armenian{Գՙիշերը փախաք օնցաք սարը էլաք խալսէցաք։}

– \armenian{Թուրքերը ինչիգ գայցի՞ն թէ իմօնալ}

– \armenian{Իսգի ինչիգ ա̈լ չիմացին}

– \armenian{Էդեվ ի՞նչօ էրիք։}

– \armenian{Հիւքիւմէթին գՙօնգդաքը բիդի. Թուրքերը մէզի դէսօն, ասդին հըն թէ մե՛ք էշդալ. դՙունք մէզի քյէօթգէցիք, մէք ա̈լ ձՙէզի քյէօքգէցաք, էրաք խալափօթ (ռուս. խառնակութիւն})... 

\begin{adjarianpage}\label{page:194}\end{adjarianpage}% should be 194

... \armenian{մը. իրաց հէդ ույալում հը՛ն ու հիւքիւմէթը չըղնիք (չիյնանք}). \armenian{միյ գՙօղցաձ մալիյն ա̈լ յէդ առէք։ Էնից խօսաձը օնգօջ դՙրաք էօչ. ինչաք Գիւմիւշխանա. հուն Մըգըրդիչ էֆընդի մը գյա̈ր, գՙնացաք գՙդաք զա̈ն. հալիյս էնու հասգընցուցաք. օնցավ այչի դՙիյս էղառ մէզիգի, ու իրաց հէդ գՙնացաք հիւքիւմէթը. ընդեղ մաշնագ} (< \armenian{մարդնակ «լաւ») արզուհալ մը դվաք. արզուհալն ա̈լ Մըզըրդիչ էֆընդին գՙրեց. բէդքը (լաւ) մաշթ էր. քէօլէ՚լիմ էնու միրվացը։ Փօլիս մը, էրգու զաֆթիյա Մըգըրդիչ էֆընդին էղառ մէզի հէդ կօշդի, ինք էլ հէդվընիս, ինչաք Քիւրդալուղուն գՙեղը. դասը հադիգ մալ էնից գՙօղցաձ մալերուն յէդ էղառ, մէզի էրէդ, էբՙեր քախքի գՙլխէն, ջօմփա էդՙիր օնցաք գՙնացաք էյլըն (թրք. արօտատեղի}). \armenian{շադ օյ դէսնու բօլա քի։ Ընդէղ դէղ բՙռնէցաք, դուն շինէցաք. էդէք (կամ ընչաք) հիմի հուն իք։}

\subsection{Abgion village}

\armenian{Յէս Արգյօնցի իմ. մէք ունիք մէգ վարջադուն. խաչ (եկեղեցի) մէլ ունիք. բիթուն գՙէղը չիւյս մահալա է. վարջադունը բՙաց է. ունիք իրէք վարժա՛բեդ. միյ դէղի իրադն է լա՛զուդ, գա՛}X\armenian{ին խօղիրը շադ իրադսուզ ին. էդու հումար էղաձ լա՛զդն էլ չի՛ հէրքիր։ Այդէրուն մէչ դահա գը՛լլի լո՛բգէ, բՙօնջՙար (կաղամբ}), \armenian{ոգրիշ յէ՛շիլլուղ։ Հէն միձ իրադը գաղինն է։ Ձՙմռօն բարաբ մնացաձ վաքըթը գՙէղացուց բազին ցախուդիրը գՙօյձէլի գընէն, խօդիյն էլ քէսադ (քիչ) ըլլուշին սէբաբի։ օմրօն գօվիրը էյլէն դօնին գու։ Էյլըն մինըն էրգու օր հէռու է. գօվիրը չաբուխ չին գարի էշդալ. էդու հումար էրգու գՙիշիր դՙուսը մընօն գու, ուչինջի իրիգունը դեղ հասնին գու. իրէք օմիս գինուշըն (կենալէն) էդիվ՝ էլի վէր գուգՙօն։ Միր գՙէղացիք գօվ շադ բէհին գու. չունքի զիբիլը (աղբ) շադ բիդու է. սադէ գաթէն բաշքա էլ մաշդը ուզէ չուզէ զիբիլին հումար բիդի դիրէ հօշդիր (հորթեր) ու գաթ չընօղ գօվիր։ Միր գՙէղացիք շադ ֆուխարէ ին. էդու հումար շադիրը կուրբէթ գէշդօն, թարա կազօնմիշ ըլլուշի հումար. շադիյն էլ կուրբէթին մէչ զէրուր (թշուառ) ըլլուշէն մէ՛ռնին գու՝ թօղէլով չօլուխ-չօջուխ էրիսի վրա։ Է՜յ գիդէ հէ՜յ, էդմօն քօնի՞ օջախ մէրաձ (մարած) է։}

\armenian{Մէ մէլ օր (մէկ մ՚ալ որ) միր գՙէղացիք շադ ուղուզ (տը}-... 

\begin{adjarianpage}\label{page:195}\end{adjarianpage}% should be 195

...\armenian{գէտ) ին, ինչիգէ (բանէ մը) խաբար չունին. էրգուս մը (մէկ-երկու, մի քան) դարի անջաք գա օր քիչ նէհրաձ ին առաչ էշդալ։ Միր էխդիայնին ուղուզ ըլլուշին սէբաբի շադ ժամասիր ին. քօռ հավու բէս գրօնավօյնուն ասդաձին (ըսածին) հավդօն գու. նօրվէ գՙալիք (նորելուկ) խէլաց մօդիգ բՙօն մը ասիս նա, չին հավդալ. օնդան սօրա էլ դահա բՙարիվդ էլ չին առնուլ դէ (թէ) էս դըղըն մաշթ չըլլի բիդի, յախօդ օնասդվաձ է։ Էդ սէբէբին ըմըն սըրա հինի ու նօրի գռիվ գըլլի։}




\chapter{Malatya}
\section{Background}
\begin{adjarianpage}\label{page:196}\end{adjarianpage}% should be 196


This dialect is spoken in the city of Malatya and in its surrounding villages until Adıyaman or Hisn-Mansur. Its region occupies a middle ground between the dialects of Tigranakert, Kharberd, Arapgir, and Cilicia. This is one of the southern borders (\armenian{սահմանապահ}) of Armenian, because Armenian is no longer spoken south of Hisn-Mansur. Kurdish, Turkish, and Arabic have taken its area or sphere. 

For the Malatya dialect, we have a small sketch of its phonetic system in the periodical \citeauthor{Byurakn} (1900, page 118) and in two small insufficient excerpts (\citeauthor{Byurakn} 1898, page 620; 1899, page 772). There is a smaller manuscript from Hisn-Mansur (ibid., 1900, page 331).

Based on all of this, we can follow up by saying that the Malatya dialect occupies a middle position between the dialects of Kharberd, Tigranakert, and Cilicia. If we compare with the first two, we see that the Malatya dialect has changed a lot; while if we compare with the Cilicia dialect, especially with the Marash subidialect, then the Malatya dialect has a sufficiently clear picture. 

\section{Phonology}
\subsection{Segment inventory}
\subsubsection{Laryngeal quality}

The consonant system is the same as the Tigranakert dialect. From the three degrees of sounds from Old Armenian, only two remain (voiced and voiceless aspirated). The voiced and voiceless aspirated sounds became voiceless aspirated, while the voiceless unaspirated become voiced (Table \ref{tab:Malatya:phono:cos:voicing}).




\begin{table}[H]
\centering
\caption{Laryngeal quality of stops and affricates in the Malatya dialect}
\label{tab:Malatya:phono:cos:voicing}
\begin{tabular}{|l| ll|ll| ll|}
\hline & \multicolumn{2}{l|}{Classical Armenian} &\multicolumn{2}{l|}{> Malatya} & \multicolumn{2}{l|}{cf. SEA} \\ 
 `good' & bɑɾi & \armenian{բարի} & pʰɑɾi & \armenian{փարի} & bɑɾi & \armenian{բարի} \\ 
`pillow' &bɑɾd͡z & \armenian{բարձ} & pʰɑɾt͡sʰ & \armenian{փարց} &bɑɾt͡sʰ & \armenian{բարձ} \\
 \armenian{՝}to bring' & beɾel & \armenian{բերել} & pʰeɾel & \armenian{փէրէլ} & beɾel & \armenian{բերել} \\
 	`high' &bɑɾd͡zəɾ & \armenian{բարձր} & pʰɑnt͡sʰəɾ & \armenian{փանցըր} & bɑɾt͡sʰəɾ & \armenian{բարձր} \\ 
 	`book' &ɡiɾ-kʰ (-{\pl}) & \armenian{գիրք} & kʰiɾkʰ & \armenian{քիրք՝} & ɡiɾkʰ & \armenian{գիրք} \\ 
`door' & durən & \armenian{դուռն} & tʰor & \armenian{թօռ} & dur & \armenian{դուռ} \\ 
 		\armenian{՝}knife' & dɑnɑk & \armenian{դանակ}& tʰɑnɑɡ & \armenian{թանագ} & dɑnɑk & \armenian{դանակ} \\ 
\hline 
\end{tabular}
\end{table}

\subsection{Sound changes}

For vowels and consonants, the Malatya dialect provides the following sound changes. 

\begin{adjarianpage}\label{page:197}\end{adjarianpage}% should be 197


\subsubsection{Classical Armenian /e/ <\armenian{ե}>}

The Classical sound /e/ <\armenian{ե}> changed to /ɑ/ <\armenian{ա}> (Table \ref{tab:Malatya:phonology:changes:vowel:e:a}). 


\begin{table}[H]
 	\centering 
 	\caption{Change from Classical Armenian /e/ <\armenian{ե}> to /ɑ/ <\armenian{ա}> in the Malatya dialect}
 	\label{tab:Malatya:phonology:changes:vowel:e:a}
 	\begin{tabular}{|l| ll|ll| ll|}
 		\hline & \multicolumn{2}{l|}{Classical Armenian} &\multicolumn{2}{l|}{> Malatya} & \multicolumn{2}{l|}{cf. SEA} \\ 
`big' &met͡s & \armenian{մեծ} & mɑnd͡z& \armenian{մանձ} &met͡s & \armenian{մեծ} \\
`burden' &berən & \armenian{բեռն} & pʰɑr & \armenian{փառ} &ber & \armenian{բեռ} \\ 
`chickpea' &siserən & \armenian{սիսեռն} & səsɑr & \armenian{սըսառ}& siser & \armenian{սիսեռ} \\
`mountain' &le̯ɑrən & \armenian{լեառն} & lɑr & \armenian{լառ}& ler & \armenian{լեռ} \\
 \armenian{՝}when' & eɾb & \armenian{երբ} & jɑpʰ & \armenian{յափ} & jeɾpʰ & \armenian{երբ} \\
\hline 
 	\end{tabular}
\end{table}

The Classical sound /e/ <\armenian{ե}> changed to /i/ <\armenian{ի}> (Table \ref{tab:Malatya:phonology:changes:vowel:e:i}). 


\begin{table}[H]
 	\centering 
 	\caption{Change from Classical Armenian /e/ <\armenian{ե}> to /i/ <\armenian{ի}> in the Malatya dialect}
 	\label{tab:Malatya:phonology:changes:vowel:e:i}
 	\begin{tabular}{|l| ll|ll| ll|}
 		\hline & \multicolumn{2}{l|}{Classical Armenian} &\multicolumn{2}{l|}{> Malatya} & \multicolumn{2}{l|}{cf. SEA} \\ 
`wheat' &t͡sʰoɾe̯ɑn & \armenian{ցորեան} & t͡sʰoɾin & \armenian{ցօրին} & t͡sʰoɾen& \armenian{ցորեն} \\
`brains, mind' &χelkʰ & \armenian{խելք} & χilkʰ & \armenian{խիլք} & χelkʰ& \armenian{խելք} \\
 `gospel'& ɑ{we}tɑɾɑn & \armenian{աւետարան} & ɑvidiɾɑn & \armenian{ավիդիրան} & ɑvetɑɾɑn & \armenian{ավետարան} \\
 `black'& se̯ɑu̯ & \armenian{սեաւ} & siv & \armenian{սիվ} & sev & \armenian{սև} \\
\hline 
 	\end{tabular}
\end{table}


\subsubsection{Classical Armenian /u/ <\armenian{ու}>}

The Classical sound /u/ <\armenian{ու}> changed to /o/ <\armenian{օ}> (Table \ref{tab:Malatya:phonology:changes:vowel:u:o}). 

\begin{table}[H]
 	\centering 
 	\caption{Change from Classical Armenian /u/ <\armenian{ու}> to /o/ <\armenian{օ}> in the Malatya dialect}
 	\label{tab:Malatya:phonology:changes:vowel:u:o}
 	\begin{tabular}{|l| ll|ll| ll|}
 		\hline & \multicolumn{2}{l|}{Classical Armenian} &\multicolumn{2}{l|}{> Malatya} & \multicolumn{2}{l|}{cf. SEA} \\ 
`door' & durən & \armenian{դուռն} & tʰor & \armenian{թօռ} & dur & \armenian{դուռ} \\ 
`water' &d͡ʒuɾ & \armenian{ջուր} & t͡ʃʰoɾ & \armenian{չօռ} & d͡ʒuɾ & \armenian{ջուր} \\ 
`to who' ({\dat}) & \armenian{ո՞ւմ}& hom & \armenian{հօմ} &um & \armenian{ո՞ւմ} \\ 
\hline 
 	\end{tabular}
\end{table}

\subsubsection{Classical Armenian /ɑi̯/ <\armenian{այ}>}

The Classical sound /ɑi̯/ <\armenian{այ}> changed to /e/ <\armenian{է}> (Table \ref{tab:Malatya:phonology:changes:vowel:ɑi:e}). 

\begin{table}[H]
 	\centering 
 	\caption{Change from Classical Armenian /ɑi̯/ <\armenian{այ}> to /e/ <\armenian{է}> in the Malatya dialect}
 	\label{tab:Malatya:phonology:changes:vowel:ɑi:e}
 	\begin{tabular}{|l| ll|ll| ll|}
 		\hline & \multicolumn{2}{l|}{Classical Armenian} &\multicolumn{2}{l|}{> Malatya} & \multicolumn{2}{l|}{cf. SEA} \\ 
 `mother' & mɑi̯ɾ & \armenian{մայր} & meɾ & \armenian{մէր} & mɑjɾ & \armenian{մայր} \\
`this' & ɑi̯s & \armenian{այս} & es & \armenian{էս} & ɑjs & \armenian{այս} \\ 
`wood' & pʰɑi̯t & \armenian{փայտ} & pʰed & \armenian{փէդ} &pʰɑjt & \armenian{փայտ} \\
 `vineyard' &ɑi̯ɡi& \armenian{այգի} & ekʰi & \armenian{էքի} &ɑjɡi& \armenian{այգի} \\
`to burn' & ɑi̯ɾel & \armenian{այրել} & eɾil & \armenian{էրիլ} & ɑjɾel & \armenian{այրել} \\ 
\hline 
 	\end{tabular}
\end{table}


The Classical sound /ɑi̯/ <\armenian{այ}> changed to /ɑ/ <\armenian{ա}> (Table \ref{tab:Malatya:phonology:changes:vowel:ɑi:ɑ}). 

\begin{table}[H]
 	\centering 
 	\caption{Change from Classical Armenian /ɑi̯/ <\armenian{այ}> to /ɑ/ <\armenian{ա}> in the Malatya dialect}
 	\label{tab:Malatya:phonology:changes:vowel:ɑi:ɑ}
 	\begin{tabular}{|l| ll|ll| ll|}
 		\hline & \multicolumn{2}{l|}{Classical Armenian} &\multicolumn{2}{l|}{> Malatya} & \multicolumn{2}{l|}{cf. SEA} \\ 
`sound' & d͡zɑi̯n & \armenian{ձայն} & t͡sʰɑn & \armenian{ցան} & d͡zɑjn & \armenian{ձայն} \\ 
`wide' & lɑi̯n & \armenian{լայն} & lɑn & \armenian{լան} & lɑjn & \armenian{լայն} \\ 
\hline 
 	\end{tabular}
\end{table}

\subsubsection{Classical Armenian /oi̯/ <\armenian{ոյ}>}

The Classical sound /oi̯/ <\armenian{ոյ}> changed to /o/ <\armenian{օ}> (Table \ref{tab:Malatya:phonology:changes:vowel:oi:o}). 
 
\begin{table}[H]
 	\centering 
 	\caption{Change from Classical Armenian /oi̯/ <\armenian{ոյ}> to /o/ <\armenian{օ}> in the Malatya dialect}
 	\label{tab:Malatya:phonology:changes:vowel:oi:o}
 	\begin{tabular}{|l| ll|ll| ll|}
 		\hline & \multicolumn{2}{l|}{Classical Armenian} &\multicolumn{2}{l|}{> Malatya} & \multicolumn{2}{l|}{cf. SEA} \\ 
`sister' & kʰoi̯ɾ & \armenian{քոյր} & kʰoɾ & \armenian{քօր} & kʰujɾ & \armenian{քույր} \\
`light' & loi̯s & \armenian{լոյս} & los & \armenian{լօս} & lujs & \armenian{լույս} \\ 
`nest' & boi̯n & \armenian{բոյն} &pʰon & \armenian{փօն} & bujn & \armenian{բույն} \\ 
\hline 
 	\end{tabular}
\end{table}

\subsubsection{Classical Armenian /iu̯/ <\armenian{իւ}>}

The Classical sound /iu̯/ <\armenian{իւ}> changed to /i/ <\armenian{ի}> (Table \ref{tab:Malatya:phonology:changes:vowel:iu:i}). 

\begin{table}[H]
 	\centering 
 	\caption{Change from Classical Armenian /iu̯/ <\armenian{իւ}> to /i/ <\armenian{ի}> in the Malatya dialect}
 	\label{tab:Malatya:phonology:changes:vowel:iu:i}
 	\begin{tabular}{|l| ll|ll| ll|}
 		\hline & \multicolumn{2}{l|}{Classical Armenian} &\multicolumn{2}{l|}{> Malatya} & \multicolumn{2}{l|}{cf. SEA} \\ 
 `fountain' & ɑɬbiu̯ɾ & \armenian{աղբիւր} & ɑχpʰiɾ & \armenian{ախփիր} & ɑχpjuɾ & \armenian{աղբյուր} \\ 
`hundred' & hɑɾiu̯ɾ & \armenian{հարիւր} & heɾiɾ & \armenian{հէրիր} & hɑɾjuɾ & \armenian{հարյուր} \\
 \armenian{՝}blood' & ɑɾiu̯n & \armenian{արիւն}& eɾin & \armenian{էրին} & ɑɾjun & \armenian{արյուն} \\
\hline 
 	\end{tabular}
\end{table}

(These two sound changes are characteristic of also the Kharberd-Yerznka dialect, but they do not exist in the Tigranakert dialect.) \translatorHD{It seems Adjarian is referring to the following two sound changes.}


The Classical sound /iu̯/ <\armenian{իւ}> changed to /o/ <\armenian{օ}> (Table \ref{tab:Malatya:phonology:changes:vowel:iu:o}). 
 
\begin{table}[H]
 	\centering 
 	\caption{Change from Classical Armenian /iu̯/ <\armenian{իւ}> to /o/ <\armenian{օ}> in the Malatya dialect}
 	\label{tab:Malatya:phonology:changes:vowel:iu:o}
 	\begin{tabular}{|l| ll|ll| ll|}
 		\hline & \multicolumn{2}{l|}{Classical Armenian} &\multicolumn{2}{l|}{> Malatya} & \multicolumn{2}{l|}{cf. SEA} \\ 
 		`column' & siu̯n & \armenian{սիւն} & son & \armenian{սօն} & sjun & \armenian{սյուն} \\ 
\hline 
 	\end{tabular}
\end{table}

The Classical sound /iu̯/ <\armenian{իւ}> changed to /œ/ <\armenian{էօ}> (Table \ref{tab:Malatya:phonology:changes:vowel:iu:œ}). 
 
\begin{table}[H]
 	\centering 
 	\caption{Change from Classical Armenian /iu̯/ <\armenian{իւ}> to /œ/ <\armenian{էօ}> in the Malatya dialect}
 	\label{tab:Malatya:phonology:changes:vowel:iu:œ}
 	\begin{tabular}{|l| ll|ll| ll|}
 		\hline & \multicolumn{2}{l|}{Classical Armenian} &\multicolumn{2}{l|}{> Malatya} & \multicolumn{2}{l|}{cf. SEA} \\ 
 		\armenian{՝}snow' & d͡ziu̯n & \armenian{ձիւն}& t͡sʰœn & \armenian{ցէօն} & d͡zjun & \armenian{ձյուն} \\ 
\hline 
 	\end{tabular}
\end{table}

\section{Background (continued)}
In the grammar, we couldn't find separate characteristic forms; and if the published excerpts are correct, we can say that the grammar of the Malatya dialect does not have separate innovations.\footnote{\translatorHD{The word `separate' here <\armenian{առանձին}> may also be possibly translated as unique.}}


\section{Text samples}

{\sampleoverview}

 \subsection{Malatya}

Adjarian's source: See \citeauthor{Byurakn} 1899, page 772. 


\armenian{Դէվէէն ինգէր՝ հօփ հօփը ցառքէ թօղ չիդար։}

\armenian{Օչիլօդը կը քէրվի՝ անօթին գիւման գըյնի։}

\armenian{Հարսնէդունը չը քըդէ, շէրէփն առիր գը վազէ։}

\armenian{Յարա չիւնիս նէ ինչո՞ւ գուջունմիշ գըլլիս։}

\armenian{Գադունէրը քացին, մուգէրուն ջանփա փացվէցավ։}

\armenian{Մէղավօրը ժամ չէ գէցիր, գայնիր է նէ մադը աչքն է մդիր։}

\armenian{Չօռը (ջուրը) սանդը թիր՝ ձէձէ ձէձէ՝ գինէ չօռ։}

\armenian{Դանձը քէնց ձառը ձանդր է։}

\armenian{Չէմ ուդիր՝ ջէբս թրէք, չիմ գարքըվիր՝ ձոցս դվէք։}




\begin{adjarianpage}\label{page:198}\end{adjarianpage}% should be 198

\armenian{Չօռը բարդաղը դէսնէս, դէրդէրը խուցը։}

\armenian{Իշուն չի հասնիր, փալանը գը ձէձէ։}

\armenian{Շանը գը զէնէն դէ դիրունմնէ գամչնան։}

\armenian{Իս գուզիմ շալգօղ՝ Ասվաձ գուդա շալգէլիք։}


\subsection{Adıyaman}

Turkish: Hisn-Mansur

Adjarian's source: See \citeauthor{Byurakn} 1900, page 331. The orthography is preserved unchanged. 

\armenian{Մուդ դդում. էրին սրջէս. քէօռ ըննաս. խանադ խարապ ըննա. պատին տակը մնաս. էրէսիդ հայող չննա. պէմուրատ էրթաս. պապուդ գանկը կողը չհանգչի. Աստուծոր խշմին էրթաս օղուլ ուշաղի տէր չննաս. կէտնին եօթը յատակը անցնիս. տունիդ պայխուշ խօսա աչվըներդ փաթր փաթր փաթլամիշ ըննա։}



\chapter{Cilicia}\label{chapter:Cilicia}
\section{Background}
\begin{adjarianpage}\label{page:199}\end{adjarianpage}% should be 199

Under this general name we want to include the Armenian spoken in Zeytun, Hadjin, Marash, and further south Kilis, Payas, Alexandretta, Antioch, Svedia. Although they show sufficient differences among themselves, but because their general characteristics are much larger and more common, then we can consider them as subdialects. 

\section{Literature}
From the aforementioned areas, only the vernaculars of Zeytun and Marash are satisfactorily studied. For the first one, we have the work by Allahverdian \citep{Allahverdian-1884-UlniaZeytun} for Zeytun. This book contains many manuscripts with the Zeytun dialect, and it ends with an succinct dictionary. Some small manuscripts in this dialect have been published in \citeauthor{Byurakn} (1898, page 744\footnote{\translatorHD{The original page had 144. But on page 306, Adjarian provides an erratum that the page number is 744; I fixed it.}} ; 1899, page 18, 137, 443, 545; 1900, page 74, 228). In \citeauthor{Bazmaveb} (1897, page 467-73), my one published fable is taken from \citeauthor{Allahverdian-1884-UlniaZeytun}'s book.


Beside these, I have a detailed study of the Zeytun dialect which I prepared during my summer travels of 1910 in Istanbul, with help from Zeytun native and student at Istanbul Getronagan Armenian High School (\armenian{Կեդրոնական}): Mr. Onnik Mahtesian (\armenian{պր. Օննիկ Մահտեսեան}; \translatorHD{SEA: /onnik mɑhtesjɑn/, SWA: /onniɡ mɑhdesjɑn/}) and a prince's son H. Yaghoupian (\armenian{Յ. Եաղուպեան}; \translatorHD{SEA: /jɑʁupjɑn/, SWA: /jɑʁubjɑn/}). 




The Marash subdialect was previously succinctly studded by Melik S. Davit-Bek (\armenian{Մէլիք Ս. Դաւիթ բէկ}) in \citeauthor{HandesAmsorya} 1896, page 43-45, 113-114, 229-232, and 354-357. This study was prepared over text samples that were published in Araks (\armenian{Արաքս}) 1889, volume 2 (\armenian{Բ}.) page 21-27;\footnote{\translatorHD{There are many periodicals with this name Araks so I haven't been able to track down the right citation yet.}} this study was later published in a shorter form in a French translation in the periodcal Mélanges de Harlez.\footnote{\translatorHD{On page 306, Adjarian provides an erratum that I think means that this publication is from 1896, pages 204-211.}} Another more complete study... 

\begin{adjarianpage}\label{page:200}\end{adjarianpage}% should be 200

... and a few manuscripts have been published by a native from Marash, H. Varzhapetian (\armenian{Յ. Վարժապետեան}; \translatorHD{SEA: /vɑɾʒɑpetjɑn/, SWA: /vɑɾʒɑbedjɑn/}), in the periodical \citeauthor{Byurakn} (1898, page 179, 360, 386, 425, 452, 465, 481, 535, 570, 585, 597, 693, 860, 888; 1899, page 101, 314, 349, 405, 425; 1900, page 185 and 363). 

For the Hadjin subdialect, we have first a sufficiently extensive article by Hayganoush Boyajian (\armenian{Հայկանուշ Պօյաճեան}; \translatorHD{SEA: /hɑjkɑnuʃ pojɑt͡ʃjɑn/, SWA: /hɑjɡɑnuʃ bojɑd͡ʒjɑn/}) in Araks (\armenian{Արաքս}; 1889, volume 1 (\armenian{Ա}.), page 47-51), and a few small writings in \citeauthor{Byurakn} (1898, page 779; 1899, page 41; 1900, page 331). For the language of Kesab and other villages that surround Antioch, see \citeauthor{Byurakn} 1899, page 443, and 1900, page 731). There is no information on the language of other places. 

\section{Phonology}

\subsection{Segment inventory}
\subsubsection{Vowels}
The Cilicia dialect, whose most chief representative is Zeytun, has the vowels in Table \ref{tab:Cilicia:vowels}. 

\begin{table}[H]
 \centering
 \caption{Vowels of the Cilicia dialect}
 \label{tab:Cilicia:vowels}
 \begin{tabular}{|lllll|}
\hline 
/i/ <\armenian{ի}> & /ʏ/ <\armenian{իւ}> & && /u/ <\armenian{ու}> \\
/i̯e/ <\armenian{ե}> & & & & /u̯o/ <\armenian{ո}> \\ 
/e/ <\armenian{է}> & /œ/ <\armenian{էօ}> & /ə/ <\armenian{ը}> & /ə̞/ <\armenian{ը}°> & /o/ <\armenian{օ}> \\ 
/æ/ <\armenian{ա}̈> & & & &/ɑ/ <\armenian{ա}>
\\
 \hline 
 % Orthography & \armenian{ա} & \armenian{է} & \armenian{ը}& \armenian{ի}& \armenian{օ}& \armenian{ու} \\
 % IPA transcription & ɑ & e & ə & i & o & u 
 % \\ \hline
 \end{tabular}
\end{table}


Among these, the sound /ə̞/ <\armenian{ը}°> is a new sound which represented a middle degree between between the vowels /ə, ɑ/ <\armenian{ը ա}>.

\translatorHD{Note that for the sound that I transcribe as /ə̞/, Adjarian uses the upside-down version of the letter <\armenian{ը}>. However, my text processor couldn't type this letter. So for Adjarian's transcriptions, I use the symbol <\armenian{ը}°>.} 

\subsubsection{Consonants}
The consonants have three degrees in the Zeytun dialect and in the Hadjin subidalect (voiced, voiced aspirate, and voiceless aspirated). In the southern regions, meaning in the Marash subidalect, the voiced aspirates are lost. In the Shorvayian (\armenian{Շորվայեան}; \translatorHD{SEA/SWA: /ʃorvɑjɑn/}) district of Zeytun, I also found the voiceless sounds /t͡s, t͡ʃ, p, k, t/ <\armenian{ծ, ճ, պ, կ, տ}>.

\subsubsection{Subdialectal diphthongs}\label{sec:Cilicia:phono:segment:subdialect:diph}
In the villages of Antioch, there are also the diphthongs /oə̯, ei̯, ii̯, ɑi̯/ <\armenian{օը, էյ, ի}ʲ, \armenian{ա}ʲ>,\footnote{\translatorHD{For the sounds that I transcribe as /ii̯, ɑi̯/, Adjarian used the superscript of the Armenian letter <\armenian{յ}> as in <\armenian{ի}\textsuperscript{\armenian{յ}}, \armenian{ա}\textsuperscript{\armenian{յ}}>. But for my text-processor, I cannot render such superscripts easily for <\armenian{յ}> so I instead used a superscript ʲ. }} which do not exist in other places. 


\subsection{Sound changes}

\subsubsection{Vowel changes}

\paragraph{Classical Armenian /ɑ/ <\armenian{ա}> }
Among the sound changes, the most characteristic one that is spread across the entirety of Cilicia is that the Classical sound /ɑ/ <\armenian{ա}> changes to /o/ <\armenian{օ}> under stress (Table \ref{tab:Cilicia:phonology:soundChange:monoph:a:o}). 




\begin{table}[H]
 \centering
 \caption{Change from Classical Armenian /ɑ/ <\armenian{ա}> to /o/ <\armenian{օ}> in the Cilicia dialect}
 \label{tab:Cilicia:phonology:soundChange:monoph:a:o}
 \begin{tabular}{|l| ll|ll| lll|}
 \hline & \multicolumn{2}{l|}{Classical Armenian} &\multicolumn{2}{l|}{> Cilicia } & \multicolumn{3}{l|}{cf. SEA or SWA} \\ 
 \hline 
 & & & \multicolumn{2}{l|}{Zeytun subdialect}& & & \\
`star' & ɑstəɬ& \armenian{աստղ} & osʁ & \armenian{օսղ} & ɑstəʁ & \armenian{աստղ} & SEA \\
`ceiling' & ɑrɑstɑɬ& \armenian{առաստաղ} & ɑjəsdoχ & \armenian{այըսդօխ} & ɑrɑstɑʁ & \armenian{առաստաղ} & SEA \\
`to open' & bɑnɑl& \armenian{բանալ} & bʰɑnol & \armenian{բՙանօլ} &pʰɑnɑl & \armenian{բանալ} & SWA\\
 `God' & ɑstu̯ɑt͡s & \armenian{Աստուած} & ɑsbʰod͡z & \armenian{Ասբՙօձ} &ɑstvɑt͡s & \armenian{Աստված} & SEA \\
 `late' & ɑnɑɡɑn & \armenian{անագան} & ɑnɡʰon & \armenian{անգՙօն} & ɑnɑɡɑn & \armenian{անագան} & SEA\\ 
 `to descend' & id͡ʒɑnel & \armenian{իջանել} & it͡ʃʰnol & \armenian{իչնօլ} & it͡ʃʰnɑl & \armenian{իջնալ} & SWA \\ 
 \hline 
 & & & \multicolumn{2}{l|}{Marash subdialect}& & & \\
`idle' & pɑɾɑp & \armenian{պարապ} & bɑɾob & \armenian{բարօբ} & pɑɾɑp & \armenian{պարապ} & SEA\\ 
`city' & kʰɑɬɑkʰ & \armenian{քաղաք} & kʰɑʁokʰ & \armenian{քաղօք} & kʰɑʁɑkʰ & \armenian{քաղաք} & SEA \\ 
`it is' & & & ɡənno & \armenian{գըննօ} & ɡəllɑ & \armenian{ըլլայ} & S\armenian{Վ}A \\ 
`man' &mɑɾd & \armenian{մարդ} & moɾtʰ & \armenian{մօրթ}&mɑɾtʰ & \armenian{մարդ} & SEA \\
`rock' &kʰɑɾ & \armenian{քար} & kʰoɾ & \armenian{քօր} &kʰɑɾ & \armenian{քար} & SEA\\
\hline 
 & & & \multicolumn{2}{l|}{Hadjin subdialect}& & & \\
`I go' & eɾtʰɑm & \armenian{երթամ} & ɡɑʃdom & \armenian{գաշդօմ} & ɡeɾtʰɑm & \armenian{կ՚երթամ} & SWA\\ 
`I will come' & & & biɡʰɡom & \armenian{բիգՙգօմ} & bidi ɡɑm & \armenian{պիտի գամ} &SWA \\ 
`thousand' & hɑzɑɾ & \armenian{հազար} & hɑzoɾ & \armenian{հազօր} & hɑzɑɾ & \armenian{հազար} & SEA \\ 
`piece' & hɑt & \armenian{հատ} & hod & \armenian{հօդ} & hɑt & \armenian{հատ} & SEA \\ 
`debt' & pɑɾt-(ə?)kʰ (-{\pl}) & \armenian{պարտք} & boɾdkʰ & \armenian{բօրդք} & pɑɾtkʰ & \armenian{պարտք} & SEA \\ 
\hline 
 & & & \multicolumn{2}{l|}{Antioch subdialect}& & & \\
`dad (Ant.); & & & dod & \armenian{դօդ} & & & \\ 
grandma (SEA)' & & & & & tɑt & \armenian{տատ} & SEA\\ 
`world' & ɑʃχɑɾh & \armenian{աշխարհ} & eχʃoɾ & \armenian{էխշօր} & ɑʃχɑɾ & \armenian{աշխարհ} & SEA\\ 
`I stand' & kɑjɑnɑm & \armenian{կայանամ} & ɡo ɡinom & \armenian{գօ գինօմ} & ɡə ɡenɑm & \armenian{կը կենամ} & SWA\\ 
`it be' & lini & \armenian{լինի} & ənno & \armenian{ըննօ} & əllɑ & \armenian{ըլլայ} & SWA \\ 
interjection & & & əɾo & \armenian{ըրօ՛} & ɑɾɑ & \armenian{արա} & SEA \\ 
`debt' & pɑɾt-(ə?)kʰ (-{\pl}) & \armenian{պարտք} & boɾdkʰ & \armenian{բօրդք} & pɑɾtkʰ & \armenian{պարտք} & SEA\\ 

 \hline 
 \end{tabular}
\end{table}




The Yaghoubian (\armenian{Եաղուբեան}; \translatorHD{SEA: /jɑʁubjɑn, jɑʁupjɑn/, SWA: /jɑʁupʰjɑn/}) district of Zeytun always replaces this /o/ <\armenian{օ}> with /u̯o/ <\armenian{ո}>.

When the Classical /ɑ/ <\armenian{ա}> vowel is not under stress, it stays /ɑ/ <\armenian{ա}> or becomes /æ/ <\armenian{ա}̈ >, and also /u, œ/ <\armenian{ու, էօ}>, according to... 



\begin{adjarianpage}\label{page:201}\end{adjarianpage}% should be 201

various phonological conditions (Table \ref{tab:Cilicia:phonology:soundChange:monoph:a:other})




\begin{table}[H]
 \centering
 \caption{Change from Classical Armenian /ɑ/ <\armenian{ա}> to /ɑ, æ, u, œ/ <\armenian{ա, ա}̈, \armenian{ու, էօ}> in the of the Cilicia dialect}
 \label{tab:Cilicia:phonology:soundChange:monoph:a:other}
 \begin{tabular}{|l| ll|ll| ll|}
 \hline & \multicolumn{2}{l|}{Classical Armenian} &\multicolumn{2}{l| }{Cilicia } & \multicolumn{2}{l|}{cf. SEA } \\ 
 \hline 
 & & & \multicolumn{2}{l|}{Zeytun subdialect}& & \\
`king' & tʰɑɡɑ{wo}ɾ & \armenian{թագաւոր} & tʰækʰævʏɾ & \armenian{թա̈քա̈վիւր} &tʰɑkʰɑvoɾ & \armenian{թագավոր}\\
`plough' & ɑɾɑu̯ɾ & \armenian{արաւր} & hæjœj & \armenian{հա̈յէօյ} &ɑɾoɾ & \armenian{արօր}\\
`thin' &bɑɾɑk & \armenian{բարակ} & bejoɡ & \armenian{բՙայոգ} & bɑɾɑk & \armenian{բարակ} \\
`melody' &ɑ{wɑ}t͡ʃʰ & \armenian{աւաչ} & evæt͡ʃʰkʰ & \armenian{էվա̈չք} & ɑvɑt͡ʃʰ & \armenian{ավաչ} \\
`student' & ɑʃɑkeɾt & \armenian{աշակերտ} & eʃɡijd & \armenian{էշգիյդ}& ɑʃɑkeɾt & \armenian{աշակերտ} \\ 

 \hline 
 \end{tabular}
\end{table}



 

\paragraph{Classical Armenian /e, ē/ <\armenian{ե, է}> }
The Classical sounds /e, ē/ <\armenian{ե, է}> change to /e/ <\armenian{է}> or /i/ <\armenian{ի}> in both monosyllabic and polysyllabic words 
(Table \ref{tab:Cilicia:phonology:soundChange:monoph:e}). 




\begin{table}[H]
 \centering
 \caption{Change from Classical Armenian /e, ē/ <\armenian{ե, է}> to /e, i/ <\armenian{է, ի}> in the Cilicia dialect}
 \label{tab:Cilicia:phonology:soundChange:monoph:e}
 \begin{tabular}{|l| ll|ll| ll|}
 \hline & \multicolumn{2}{l|}{Classical Armenian} &\multicolumn{2}{l|}{> Cilicia } & \multicolumn{2}{l|}{cf. SEA } \\ 
 \hline 
 & & & \multicolumn{2}{l|}{Zeytun subdialect}& & \\
`evening'& eɾekoi̯ & \armenian{երեկոյ} & ijɡon & \armenian{իյգօն} & jeɾeko & \armenian{երեկո} \\
 `I'& es & \armenian{ես} & is & \armenian{իս} & jes & \armenian{ես} \\
`thirty' &eɾesun& \armenian{երեսուն} & ersun &\armenian{էռսուն} & jeɾesun& \armenian{երեսուն} \\
 \armenian{՝}face' & eɾes & \armenian{երես} & ijis & \armenian{իյիս} & jeɾes & \armenian{երես} \\
 \armenian{՝}kidneys' & eɾikɑmunkʰ & \armenian{երիկամունք} & ijɡom & \armenian{իյգօմ} & jeɾikɑmuŋkʰ & \armenian{երիկամունք} \\
 \armenian{՝}millstone' & eɾkɑnɑkʰɑɾ & \armenian{երկանաքար} & ijɡonkʰ-kʰoj & \armenian{իյգօնք-քօյ} & jeɾkɑnɑkʰɑɾ & \armenian{երկանաքար} \\
`happy! (interjection)' & eɾɑni & \armenian{երանի} & ijɑni & \armenian{իյանի} &jeɾɑni & \armenian{երանի} \\
 \armenian{՝}border' & ezəɾ & \armenian{եզր} & izijkʰ & \armenian{իզիյք} & jezəɾ & \armenian{եզր} \\
 \hline 
 & & & \multicolumn{2}{l|}{Hadjin subdialect}& & \\
 `three' &eɾekʰ & \armenian{երեք} & d͡zikʰ & \armenian{ձիք} &jeɾekʰ & \armenian{երեք} \\
`light (weight)' &tʰetʰeu̯ & \armenian{թեթեւ} & tʰitʰiv & \armenian{թիթիվ} &tʰetʰev & \armenian{թեթև} \\ 
\hline 
 & & & \multicolumn{2}{l|}{Marash subdialect}& & \\
 \armenian{՝}night' & ɡiʃeɾ & \armenian{գիշեր} & ɡiʃiɾ & \armenian{գիշիր} & ɡiʃeɾ & \armenian{գիշեր} \\
 `three' &eɾekʰ & \armenian{երեք} & iɾikʰ & \armenian{իրիք} &jeɾekʰ & \armenian{երեք} \\
 \hline \end{tabular}
\end{table}

In Zeytun, it can also stay as /i̯e/ <\armenian{ե}> 
(Table \ref{tab:Cilicia:phonology:soundChange:monoph:ie}). 




\begin{table}[H]
 \centering
 \caption{Change from Classical Armenian /e, ē/ <\armenian{ե, է}> to /i̯e/ <\armenian{ե}> in the Cilicia dialect}
 \label{tab:Cilicia:phonology:soundChange:monoph:ie}
 \begin{tabular}{|l| ll|ll| ll|}
 \hline & \multicolumn{2}{l|}{Classical Armenian} &\multicolumn{2}{l|}{> Cilicia } & \multicolumn{2}{l|}{cf. SEA } \\ 
 \hline 
 & & & \multicolumn{2}{l|}{Zeytun subdialect}& & \\
`chickpea' &siserən & \armenian{սիսեռն} & sisi̯er & \armenian{սիսեռ}& siser & \armenian{սիսեռ} \\
`beauty' &ɡeɬ & \armenian{գեղ} & ɡi̯eʁ & \armenian{գեղ}& ɡeʁ & \armenian{գեղ} \\
 \hline \end{tabular}
\end{table}

\paragraph{Classical Armenian /ə/ <\armenian{ը}> }

The Classical sound /ə/ <\armenian{ը}> often becomes /ə̞/ <\armenian{ը}°> in Zeytun.

\paragraph{Classical Armenian /i/ <\armenian{ի}> }

The Classical sound /i/ <\armenian{ի}> usually stays /i/ <\armenian{ի}>, but it has a tendency to get opened. In the Zeytun dialect, it has changed in various places to /e, ə, ə̞, ɑ/ <\armenian{է, ը, ը}°, \armenian{ա}>. In Marash, it became /ɑ/ <\armenian{ա}> 
(Table \ref{tab:Cilicia:phonology:soundChange:monoph:i}). 




\begin{table}[H]
 \centering
 \caption{Change from Classical Armenian /i/ <\armenian{ի}> to /i, e, ə, ə̞, ɑ/ <\armenian{ի, է, ը, ը}°, \armenian{ա}> in the Cilicia dialect}
 \label{tab:Cilicia:phonology:soundChange:monoph:i}
 \begin{tabular}{|l| ll|ll| ll|}
 \hline & \multicolumn{2}{l|}{Classical Armenian} &\multicolumn{2}{l|}{> Cilicia } & \multicolumn{2}{l|}{cf. SEA } \\ 
 \hline 
 & & & \multicolumn{2}{l|}{Zeytun subdialect}& & \\
`meat' & mis & \armenian{միս} & mə̞s & \armenian{մը°ս} &mis & \armenian{միս} \\ 
 \armenian{՝}woman' & & & ɡə̞nɑɡ & \armenian{գը°նագ} & kənik & \armenian{կնիկ} \\
 \hline 
 & & & \multicolumn{2}{l|}{Marash subdialect}& & \\
`meat' & mis & \armenian{միս} & mɑs & \armenian{մըմասս} &mis & \armenian{միս} \\ 
 \armenian{՝}woman' & & & ɡənɑɡ & \armenian{գընագ} & kənik & \armenian{կնիկ} \\
 \hline \end{tabular}
\end{table}

\paragraph{Other vowels}

 The other vowels have the following changes:
 
 \begin{itemize}
 \item CA /o/ > /o, ʏ, œ/ (\armenian{ո} > \armenian{օ, իւ, էօ})
 \item CA /u/ > /o, ʏ/ (\armenian{ու} > \armenian{օ, իւ})
 \item CA /iu̯/ > /i, ə̞, e/ (\armenian{իւ} > \armenian{ի, ը}°, \armenian{է})
 \item CA /oi̯/ > /ʏ, i/ (\armenian{ոյ} > \armenian{իւ, ի})
 \item CA /ɑi̯/ > /æ/ (\armenian{այ} > \armenian{ա}̈) 
 \end{itemize}
 
 \subsubsection{Consonant changes}
 \paragraph{Laryngeal changes}
 In the Zeytun dialect and Hadjin subdialect, the Armenian voiced consonants became voiced aspirates, the voiceless unaspirated became voiced, while the voiceless aspirates stayed the same. In the Marash subdialect, where as we said there are no voiced aspirates, both the voiced and voiceless unaspirated consonants became voiced. 
 \paragraph{Classical Armenian rhotic /ɾ/ <\armenian{ր}> }

 The Classical Armenian consonant /ɾ/ <\armenian{ր}> became /j/ <\armenian{յ}> in many cases for the area surrounding Zeytun. In the main town (\armenian{աւան}) of Zeytun, this sound change is found in the Shorvoyian (\armenian{Շորվոյեան}) district, which is considered a migrant settlement. The other districts use the sound /ɾ/ <\armenian{ր}>. In Hadjin, This same consonant becomes /ʃ/ <\armenian{շ}> when next to /t/ <\armenian{տ}>, and it in some places it becomes /j/ <\armenian{յ}>. 

\subsubsection{Vowel harmony}

Another general characteristic of the Cilicia dialect is the tendency for all vowels to assimilate in a word (Table \ref{tab:Cilicia:phonology:soundChange:vowelharmony}). 








\begin{table}[H]
 \centering
 \caption{Vowel harmony in the Cilicia dialect}
 \label{tab:Cilicia:phonology:soundChange:vowelharmony}
 \begin{tabular}{|l| ll|ll| lll|}
 \hline & \multicolumn{2}{l|}{Classical Armenian} &\multicolumn{2}{l|}{> Cilicia } & \multicolumn{3}{l|}{cf. SEA or SWA } \\ 
 \hline 
 & & & \multicolumn{2}{l|}{Zeytun subdialect}& & \\
`he went' & ɡənɑt͡sʰ & \armenian{գնաց} & ɡʰonot͡sʰ & \armenian{գՙօնօց} &ɡənɑt͡sʰ & \armenian{գնաց}& SEA \\ 
`twenty' & kʰəsɑn & \armenian{քսան} & kʰoson & \armenian{քօսօն} &kʰəsɑn & \armenian{քսան}& SEA \\ 
 \armenian{՝}woman (genitive)' & & & ɡonɡon & \armenian{գօնգօն} & kəŋkɑn & \armenian{կնկան} & SEA \\
`gold' & oski & \armenian{ոսկի} & isɡi & \armenian{իսգի} & voski& \armenian{ոսկի} & SEA\\
`bone' &oskəɾ & \armenian{ոսկր} & ʏsɡʏj & \armenian{իւսգիւյ} & voskoɾ & \armenian{ոսկոր} & SEA\\
`bone-{\gen}' &oskeɾ & \armenian{ոսկեր} & isɡij-i & \armenian{իսգիյի} & voskoɾ-i & \armenian{ոսկորի} & SEA\\
`I go' & eɾtʰɑm & \armenian{երթամ} & ɡoɾtʰom & \armenian{գօրթօմ} & ɡeɾtʰɑm & \armenian{կ՚երթամ} & SWA\\ 
 \hline 
 & & & \multicolumn{2}{l|}{Marash subdialect}& & \\
`this.{\gen}' & & & ʏsʏɾ & \armenian{իւսիւր} & ɑsoɾ & \armenian{ասոր} & SWA\\ 
`Jesus Christ' & jisus kʰəɾistos & \armenian{Յիսուս Քրիստոս} & ʏsʏs kʰʏɾʏsdʏs & \armenian{Իւսիւս Քիւրիւսդիւս} & hisus kʰəɾistos & \armenian{Հիսուս Քրիստոս} & SEA\\ 
 \hline 
 \end{tabular}
\end{table}


Here, we see that the Classical vowels /ə, o, e, ɑ, i/ <\armenian{ը, ո, ե, ա, ի}> left their real forms and became like the vowel of the subsequent syllable. 

\section{Morphology}

\subsection{Noun inflectional or declension}
In the grammar, there are a few innovations; while the phonological rules have brought off many unusual...



\begin{adjarianpage}\label{page:202}\end{adjarianpage}% should be 202

... forms. 

\subsubsection{Vowel harmony in definite and indefinite articles}

For example, in the Marash subdialect, the definite suffix /ə/ <\armenian{ը}> and the indefinite article /mə/ <\armenian{մը}> have changed: the first to /ə, i, u, ʏ/ <\armenian{ը, ի, ու, իւ}>, and the second to /mə, mi, mu, mʏ/ <\armenian{մը, մի, մու, միւ}>, in accordance with the vowel of the word-final syllable (Table \ref{tab:Cilicia:morpoh:noun:defin}).

\translatorHD{Note that the definite and indefinite articles are found in modern SWA as /-ə,-n/ and /-mə/ respectively. The modern definite article descends from the Classical distal suffix /-(ə)n/ <\armenian{ն}>, while the indefinite descends from the Classical numeral /mi/ <\armenian{մի}> `one'. }



\begin{table}[H]
 \centering
 \caption{Vowel harmony in the definite article in the Marash subdialect of the Cilicia dialect}
 \label{tab:Cilicia:morpoh:noun:defin}
 \begin{tabular}{|l| ll| ll|}
 \hline &\multicolumn{2}{l|}{Marash (Cilicia) } & \multicolumn{2}{l|}{cf. SWA } \\ 
 \hline 
 `shirt-{\defgloss}' & ʃɑb\'ɑɡ-ə & \armenian{շաբա՛գը} & ʃɑb\'iɡ-ə & \armenian{շապիկը} \\ 
 `wood-{\defgloss}' & pʰ\'ed-i & \armenian{փէ՛դի} & pʰ\'ɑjd-ə & \armenian{փայտը} \\ 
 `girl-{\defgloss}' & ɑχʃin mi &\armenian{ախշին մի} & ɑχt͡ʃʰ\'iɡ-ə & \armenian{աղջիկը} \\ 
 `knife-{\defgloss}' & dɑn\'oɡ-u & \armenian{դանօ՛գու} & tʰɑn\'ɑɡ-ə & \armenian{դանակը} \\ 
 `mouse-{\defgloss}' & m\'uɡ-u & \armenian{մո՛ւգու} & m\'uɡ-ə & \armenian{մուկը} \\ 
 `grass-{\defgloss}' & χ\'ʏd-ʏ & \armenian{խի՛ւդիւ} & χ\'od-ə & \armenian{խոտը} \\ 
`shirt {\indf}' & ʃɑbɑɡ mə & \armenian{շաբագ մը} & ʃɑbiɡ mə & \armenian{շապիկ մը} \\ 
 `wood {\indf}' & pʰed mi & \armenian{փէդ մի} & pʰɑjd mə & \armenian{փայտ մը} \\ 
 `girl {\indf}' & ɑχʃ\'in-i & \armenian{ախշի՛նի} & ɑχt͡ʃʰiɡ mə & \armenian{աղջիկ մը} \\ 
 `knife {\indf}' & ɑnoɡ mu & \armenian{դանօգ մու} & tʰɑnɑɡ mə & \armenian{դանակ մը} \\ 
 `mouse {\indf}' & muɡ mu & \armenian{մուգ մու} & muɡ mə & \armenian{մուկ մը} \\ 
 `grass {\indf}' & χʏd mʏ & \armenian{խիւդ միւ} & χod mə & \armenian{խոտ մը} \\ 
 `day {\indf}' & œɾ mʏ & \armenian{էօր միւ} & oɾ mə & \armenian{օր մը} \\ 
 
 \hline 
 \end{tabular}
\end{table}
\subsubsection{Plural formation}


The number of declension is the same as that of Kharberd and with more Western dialects. The plural is formed with the formatives /-iɾ, -niɾ, (-ij, -nij), -nɑ, -nə, -dækʰ/ <\armenian{իր, նիր}, (\armenian{իյ, նիյ}), \armenian{նա, նը, դա̈ք}> (Table \ref{tab:Cilicia:morpoh:noun:pl}). 


\begin{table}[H]
 \centering
 \caption{Plural suffixes in the Cilicia dialect}
 \label{tab:Cilicia:morpoh:noun:pl}
 \begin{tabular}{|l| ll| ll|}
 \hline &\multicolumn{2}{l|}{Cilicia } & \multicolumn{2}{l|}{cf. SEA } \\ 
 \hline 
`wheat-{\pl}' & t͡sʰijin-niɾ & \armenian{ցիյիննիր} & t͡sʰoɾen-neɾ& \armenian{ցորեններ} \\
`chickpea-{\pl}' & sə̞sə̞r-nə & \armenian{սը°սը°ռնը} & siser-neɾ& \armenian{սիսեռներ} \\
`garlic-{\pl}' & sœχdij-nə & \armenian{սէօխդիյնա} & səχtoɾ-neɾ& \armenian{սխտորներ} \\
`flower-{\pl}' & d͡zɑʁəɡ-nɑ & \armenian{ձաղըգնա}, & t͡sɑʁik-neɾ& \armenian{ծաղիկներ} \\
 & d͡zɑʁəɡ-nij & \armenian{ձաղըգնիյ}, & & \\
 & d͡zɑʁɡə-nij & \armenian{ձաղգընիյ} & & \\
 \hline 
 \end{tabular}
\end{table}

\subsection{Pronoun inflection or declension}

The pronouns are declined in the following way.

\subsubsection{Pronouns in the Zeytun subdialects}
 

\translatorHD{Table \ref{tab:Cilicia:morpho:pronoun:zeytun:personal} lists the personal pronouns. }

\translatorHD{It's unclear if the instances of <\armenian{էյ}> here should be transcribed as a diphthong /ei̯/ or a vowel-glide sequence /ej/. Adjarian earlier said that the Antioch subdialect uses a diphthong (\ref{sec:Cilicia:phono:segment:subdialect:diph}), but then it's unclear if the following data also use a diphthong. }

\begin{table}[H]
\caption{Inflection paradigm for personal pronouns in the Zeytun subdialect of the Cilicia dialect }\label{tab:Cilicia:morpho:pronoun:zeytun:personal}
\centering 
\begin{tabular}{| l| llllll| }
 \hline & 1SG & 2SG & 3SG & 1PL & 2PL & 3PL\\
 & `I' & `you' &`he' & `we'& `you' & `they' \\\hline 
{\nom} & is & dʰon & æn & minkʰ & dʰokʰ & ænij \\
 & \armenian{իս} & \armenian{դՙօն} & \armenian{ա̈ն} & \armenian{մինք} & \armenian{դՙօք} & \armenian{ա̈նիյ} \\
{\gen} & im & kʰu & œnʏj & mej, mij & d͡zʰij, d͡zʰej & œnʏnt͡sʰ \\
 & \armenian{իմ} & \armenian{քու} & \armenian{էօնիւյ} & \armenian{մէյ, միյ} & \armenian{ձՙիյ, ձՙէյ} & \armenian{էօնիւնց} \\
{\dat} & ind͡zi̯e & kʰiz & œnʏj & miz & d͡zʰiz & œnʏnt͡sʰ \\
 & \armenian{ինձե} & \armenian{քիզ} & \armenian{էօնիւյ} & \armenian{միզ} & \armenian{ձՙիզ} & \armenian{էօնիւնց} \\
{\acc} & ə̞sə̞ɡ & ə̞zkʰiz & zæn & ə̞zmiz & ə̞zd͡zʰiz & zenij \\
 & \armenian{ը°սը°գ} & \armenian{ը°զքիզ} & \armenian{զա̈ն} & \armenian{ը°զմիզ} & \armenian{ը°զձՙիզ} & \armenian{զէնիյ} \\
{\abl} & imni̯et͡s & kʰinni̯et͡sʰ & eniɡet͡sʰ & mijni̯et͡sʰ & d͡zʰijni̯et͡sʰ & œnʏt͡sʰni̯e \\
 & \armenian{իմնեց} & \armenian{քիննեց} & \armenian{էնիգէց} & \armenian{միյնեց} & \armenian{ձՙիյնեց} & \armenian{էօնիւցնե} \\
{\ins} & imnœv & kʰizmœv & œnʏvœkʰ & mijnœv & d͡zʰizmœv & œnʏnt͡sʰmœv \\
 & \armenian{իմնէօվ} & \armenian{քիզմէօվ} & \armenian{էօնիւվէօք} & \armenian{միյնէօվ} & \armenian{ձՙիզմէօվ} & \armenian{էօնիւնցմէօվ} 
 \\ \hline 
\end{tabular}
\end{table}

\translatorHD{Table \ref{tab:Cilicia:morpho:pronoun:zeytun:inter} lists interrogative pronouns. }

\begin{table}[H]
\caption{Inflection paradigm for interrogative pronouns `who' and `what' in the Zeytun subdialect of the Cilicia dialect }\label{tab:Cilicia:morpho:pronoun:zeytun:inter}
\centering 
\begin{tabular}{|l| ll|}
\hline & `who' & `what' \\ \hline 
{\nom} & ʏv & t͡ʃʰijkʰ, ʏnt͡ʃʰ \\
 & \armenian{իւվ} & \armenian{չիյք, իւնչ} \\
{\gen}-{\dat} & om & t͡ʃʰʏkʰ-u, ʏnt͡ʃʰ-i, int͡ʃʰ-i \\
 & \armenian{օմ} & \armenian{չիյքու, իւնչի, ինչի} \\
{\acc} & zʏv & t͡ʃʰijkʰ, ʏnt͡ʃʰ \\
 & \armenian{զիւվ} & \armenian{չիյք, իւնչ} \\
{\abl} & omn-i̯et͡sʰ & t͡ʃʰijkʰ-i̯en, int͡ʃʰ-i̯en \\
 & \armenian{օմնեց} & \armenian{չիյքեն, ինչեն} \\
{\ins} & om hid & t͡ʃʰʏjkʰ-œv \\
 & \armenian{օմ հիդ} & \armenian{չիւյքէօվ} 
 \\\hline 
\end{tabular}
\end{table}

We also find /z-əz-kʰiz/ `you.{\sg}.{\acc}' with two prepositions. \translatorHD{He means we see two accusative prepositions /z-/. }

\translatorHD{Adjarian likewise lists the following other pronouns (Table \ref{tab:Cilicia:morpho:pron:zeytun:sample}). }


\begin{table}[H]
 \centering
 \caption{Sample of other pronouns in the Zeytun subdialect of the Cilicia dialect}
 \label{tab:Cilicia:morpho:pron:zeytun:sample}
 \begin{tabular}{|l ll| }
\hline 
demonstrative proximal {\nom} {\sg} `this' &oso & \armenian{օսօ}\\
demonstrative medial {\nom} {\sg} `that' &odo & \armenian{օսօ}\\
demonstrative distal {\nom} {\sg} `that yonder' &ono & \armenian{օնօ}\\
demonstrative distal {\gen} {\sg} `of that yonder' &əniɾ & \armenian{ընիր}\\
 logophoric 3PL {\nom} `they' &iɾinkʰ & \armenian{իրինք}\\
 interrogative {\nom} {\sg} `which' &joɾ & \armenian{յօր}\\
 interrogative {\nom} {\sg} `wherever' &jo\'ɾəɾ & \armenian{յօ՛րըր}\\
 interrogative {\sg} `was?' &t͡ʃʰuɾ\'u & \armenian{չուրո՞ւ}\\

\hline 
 \end{tabular}
\end{table}

\translatorHD{For the word /t͡ʃʰuɾ\'u/ <\armenian{չուրո՞ւ}>, Adjarian translates this with SWA /eɾ/ <\armenian{է՞ր}>, but it's unclear to me how this is a pronoun. }

\begin{adjarianpage}\label{page:203}\end{adjarianpage}% should be 203

\subsubsection{Pronouns in the Marash subdialects}


\translatorHD{In the Marash subdialect, there are pronouns such as in Table \ref{tab:Cilicia:morpho:pron:marash:sample}.}


\begin{table}[H]
 \centering
 \caption{Sample of pronouns in the Marash subdialect of the Cilicia dialect}
 \label{tab:Cilicia:morpho:pron:marash:sample}
 \begin{tabular}{|l ll| }
\hline 
demonstrative proximal {\nom} {\sg} `this' &es, əso & \armenian{էս, ըսօ} \\
demonstrative proximal {\gen} {\sg} `of this' &əsʏɾ & \armenian{ըսիւր}\\
demonstrative proximal {\abl} {\sg} `from this' &əsiɡem & \armenian{ըսիգէմ}\\
demonstrative proximal {\ins} {\sg} `with this' &əsiɡʏ & \armenian{ըսիգիւ}\\
demonstrative proximal {\nom} {\pl} `these' &əsinkʰ, ʏsʏnkʰ & \armenian{ըսինք, իւսիւնք}\\
demonstrative proximal {\gen} {\pl} `of these' &əsʏnt͡sʰ, ʏsʏnt͡sʰ & \armenian{ըսիւնց, իւսիւնց}\\
demonstrative proximal {\abl} {\pl} `from these' &əsʏnt͡sʰ-me, ʏsʏnt͡sʰ-me & \armenian{ըսիւնցմէ, իւսիւնցմէ}\\
demonstrative proximal {\ins} {\pl} `with these' &əsʏnt͡sʰ-mʏ, ʏsʏnt͡sʰ-mʏ & \armenian{ըսիւնցմիւ, իւսիւնցմիւ}\\
demonstrative medial {\nom} {\sg} `that' &ed, ədo & \armenian{էդ, ըդօ}\\
demonstrative distal {\nom} {\sg} `that yonder' &en, əno & \armenian{էն, ընօ}\\

\hline 
 \end{tabular}
\end{table}

The following is the complete declension of the pronouns `I (1{\sg})', `you.{\sg}' and `which' in the Marash dialect (Table \ref{tab:Cilicia:morpho:pronoun:marash:infl}). 


\begin{table}[H]
\caption{Inflection paradigm for various pronouns in the Marash subdialect of the Cilicia dialect }\label{tab:Cilicia:morpho:pronoun:marash:infl}
\centering 
\begin{tabular}{|l| llllll|}
\hline & 1SG & 2SG & `which' ({\sg})' &1PL & 2PL & `which' ({\pl}) \\ \hline 
{\nom} & is & don & ʏɾʏ & minkʰ & dekʰ & ʏɾiɾi \\
 & \armenian{իս} & \armenian{դօն} & \armenian{իւրիւ} & \armenian{մինք} & \armenian{դէք} & \armenian{իւրիրի} \\\hline 
{\gen} & im & kʰin & uɾumɑn & miɾ & d͡zʰiɾ & ʏɾuɾun \\
 & \armenian{իմ} & \armenian{քին} & \armenian{ուրուման} & \armenian{միր} & \armenian{ձիր} & \armenian{իւրուրուն} \\\hline 
{\dat} & ies & kʰez & uɾumɑn & miz & d͡ziz & ʏɾuɾun \\
 & \armenian{իէս} & \armenian{քէզ} & \armenian{ուրուման} & \armenian{միզ} & \armenian{ձիզ} & \armenian{իւրուրուն} \\\hline 
{\acc} & jɑs & əsɡi & ʏɾʏ & mizni & d͡zizni & ʏɾiɾi \\
 & \armenian{յաս} & \armenian{ըսգի} & \armenian{իւրիւ} & \armenian{միզնի} & \armenian{ձիզնի} & \armenian{իւրիրի} \\\hline 
{\abl} & imne & kʰinne & uɾumen & miɾne & d͡zizne & ʏɾuɾune \\
 & \armenian{իմնէ} & \armenian{քիննէ} & \armenian{ուրումէն} & \armenian{միրնէ} & \armenian{ձիրնէ} & \armenian{իւրուրունէ} \\\hline 
{\ins} & imʏ & kʰinnʏ & uɾumʏ & miɾnʏ & d͡ziɾnʏ & ʏɾuɾumi \\
 & \armenian{իմիւ} & \armenian{քիննիւ} & \armenian{ուրումիւ} & \armenian{միրնիւ} & \armenian{ձիրնիւ} & \armenian{իւրուրումի}
 \\\hline 
\end{tabular}
\end{table}

\subsection{Verb inflection or conjugation}

\subsubsection{Indicative present and past imperfective}

\translatorHD{In SWA, the indicative present and past imperfective are formed by adding indicative prefix before the finite verb: /ɡu-/ for monosyllabic verb stems, /ɡ-/ before vowel-initial verbs, and /ɡə-/ elsewhere. And based on Adjarian's following descriptions, Cilicia uses essentially the same strategy but with a) different prefix forms, b) vowel harmony, and c) repeating the indicative prefix in some phonological contexts. }

In verbal conjugation (Table \ref{tab:Cilicia:morpho:verb:Indcpres}), the indicative present and imperfective forms in the Zeytun dialect are formed by using the formative /ɡo/ <\armenian{գօ}>; before vowel-initial verbs and monosyllabic verbs, it is repeated and becomes a progressive marker.\footnote{\translatorHD{The original sentence syntax is quite complicated and I'm not 100\% what it means.}} The Marash subdialect uses the formatives /kə, ki, ku/ <\armenian{կը, կի, կու}>. The first is used when the verb's first vowel is /ɑ, e, o/ <\armenian{ա, է, օ}>. The second is used when that vowel is /i/ <\armenian{ի}>. The third is vowel when that vowel is /u/ <\armenian{ու}>. In both the Marash and Hadjin subdialects, the indicative formative is not repeated. 


\begin{table}[H]
 \caption{Indicative present verbs in the Cilicia dialect }\label{tab:Cilicia:morpho:verb:Indcpres}
\centering
 \begin{tabular}{|l| ll| ll|}
 \hline &\multicolumn{2}{l|}{Cilicia } & \multicolumn{2}{l|}{cf. SWA} \\ 
 \hline 
 & \multicolumn{2}{l|}{Zeytun subdialect}& & \\
 `they sell' & ɡo d͡zɑχ-i-n & \armenian{գօ ձախին} & ɡə-d͡zɑχ-e-n & \armenian{կը ծախեն} \\
 & \multicolumn{2}{l|}{{\ind} $\sqrt{}$-{\thgloss}-{\agr}}& \multicolumn{2}{l|}{{\ind} $\sqrt{}$-{\thgloss}-{\agr}}\\
 `I cook' & ɡo ɡ-ipʰ-i-m & \armenian{գօ գիփիմ} & ɡ-epʰ-e-m & \armenian{կ՚եփեմ} \\
 `I cook' & ɡo ɡ-uz-i-m & \armenian{գօ գուզիմ} & ɡ-uz-e-m & \armenian{կ՚ուզեմ} \\
 `you give' & ɡo ɡu-d-o-s & \armenian{գօ գուդօս} & ɡu-d-ɑ-s & \armenian{կու տաս} \\
 & \multicolumn{2}{l|}{{\ind} {\ind}-$\sqrt{}$-{\thgloss}-{\agr}}& \multicolumn{2}{l|}{{\ind} $\sqrt{}$-{\thgloss}-{\agr}}\\
 \hline 
 & \multicolumn{2}{l|}{Marash subdialect}& & \\
 `I read' & ɡə ɡɑɾtʰ-o-m & \armenian{գը գարթօմ} & ɡə-ɡɑɾtʰ-ɑ-m & \armenian{կը կարդամ} \\
 `I hit' & ɡə zen-i-m & \armenian{գը զէնիմ} & ɡə-zɑɾn-e-m & \armenian{կը զարնեմ} \\
 `I like' & ɡə siɾ-i-m & \armenian{գը սիրիմ} & ɡə-siɾ-e-m & \armenian{կը սիրեմ} \\
 `I drink' & ɡu χum-i-m & \armenian{գու խումիմ} & ɡə-χəm-e-m & \armenian{կը խմեմ} \\
 `he rises' & ɡ-ill-e-$\emptyset$ & \armenian{գիլլէ} & ɡ-ell-e-m & \armenian{կ՚ելլէ} \\
 & \multicolumn{2}{l|}{{\ind} $\sqrt{}$-{\thgloss}-{\agr}}& \multicolumn{2}{l|}{{\ind}-$\sqrt{}$-{\thgloss}-{\agr}}\\
 `he takes' & ɡ-ɑɾ-n-u-$\emptyset$ & \armenian{գառնու} & ɡ-ɑɾ-n-e-m & \armenian{կ՚առնէ} \\
 & \multicolumn{2}{l|}{{\ind}-$\sqrt{}$-{\vx}-{\thgloss}-{\agr}}& \multicolumn{2}{l|}{{\ind}-$\sqrt{}$-{\vx}-{\thgloss}-{\agr}}\\
\hline 
& \multicolumn{2}{l|}{Hadjin subdialect}& & \\
\hline 
 `I go' & ɡ-ɑʃd-o-m & \armenian{գաշդօմ} & ɡ-eɾtʰ-ɑ-m & \armenian{կ՚երթամ} \\
 `I come' & ɡɑ-ɡʰɡ-o-m & \armenian{գագՙգօմ} & ɡu-kʰ-ɑ-m & \armenian{կու գամ} \\
 & \multicolumn{2}{l|}{{\ind}-$\sqrt{}$-{\thgloss}-{\agr}}& \multicolumn{2}{l|}{{\ind}-$\sqrt{}$-{\thgloss}-{\agr}}\\
\hline 
\end{tabular}

\end{table}


\begin{adjarianpage}\label{page:204}\end{adjarianpage}% should be 204

\translatorHD{As said for SWA, both the indicative present and past imperfective use a finite verb. In SWA, the finite verb lacks the past suffix in the present, while it has the past suffix /-i-/ in the past imperfective. This past suffix is added after the theme vowel. But for Cilicia, the theme vowel and past suffix seem to not co-occur based on Adjarian's paradigms. I'm not sure how to gloss Adjarian's vowels, but the simplest situation seems to be that the theme vowel and past suffix are fused into one morph, and then there's a separate past proclitic /idi/. }

In the Hadjin subdialect, the imperfective has two forms. The first is a simple form that originates from the Armenian imperfective; while the second is a complex form that is formed by adding the formative /idi/ <\armenian{իդի}> (the Turkish imperfective-forming formative <idi>). As an example, the following are the imperfectives of `to go' and `to come'. 


\translatorHD{It's unclear if the instances of <\armenian{էյ}> here should be transcribed as a diphthong /ei̯/ or a vowel-glide sequence /ej/. Adjarian earlier said that the Antioch subdialect uses a diphthong (\ref{sec:Cilicia:phono:segment:subdialect:diph}), but then it's unclear if the following data also use a diphthong. }



\begin{table}[H]
 \centering
 \caption{Indicative past imperfective <\armenian{անկատար}> of the verbs `to go' and `to come' in the Hadjin subdialect of the Cilicia dialect}
 






\label{tab:Cilicia:morpho:verb:paradigm:pastImpfIndcA}
 \begin{tabular}{|l|ll| ll| ll| }
 		\hline & \multicolumn{2}{l|}{Form 1 (without /idi/)} & \multicolumn{2}{l|}{Form 2 (with /idi/)} & & \\ \hline 
 & \multicolumn{2}{l|}{Marash `to go'} & \multicolumn{2}{l|}{Marash `to go'} & \multicolumn{2}{l|}{cf. SWA} \\ \hline
1SG &ɡ-ɑʃd-$\emptyset$-i-$\emptyset$ & \armenian{գաշդի} &ɡ-ɑʃd--i-$\emptyset$ idi & \armenian{գաշդի իդի} & ɡ-eɾtʰ-ɑj-i-$\emptyset$& \armenian{կ՚երթայի} \\
2SG &ɡ-ɑʃd--i-j & \armenian{գաշդիյ} &ɡ-ɑʃd-$\emptyset$-i-j idi & \armenian{գաշդիյ իդի} & ɡ-eɾtʰ-ɑj-i-ɾ& \armenian{կ՚երթէիր} \\
3SG &ɡ-ɑʃd-e-j & \armenian{գաշդէյ} &ɡ-ɑʃd-e-$\emptyset$-j idi & \armenian{գաշդէյ իդի} & ɡ-eɾtʰ-ɑ-$\emptyset$-ɾ& \armenian{կ՚երթար} \\
1PL &ɡ-ɑʃd-i-nkʰ & \armenian{գաշդինք} &ɡ-ɑʃd-$\emptyset$-i-nkʰ idi & \armenian{գաշդինք իդի} & ɡ-eɾtʰ-ɑj-i-ŋkʰ& \armenian{կ՚երթայինք} \\
2PL &ɡ-ɑʃd-i-kʰ& \armenian{գաշդիք} &ɡ-ɑʃd-$\emptyset$-i-kʰ idi & \armenian{գաշդիք իդի} & ɡ-eɾtʰ-ɑj-i-kʰ& \armenian{կ՚երթայիք} \\
3PL &ɡ-ɑʃd-i-n & \armenian{գաշդին} &ɡ-ɑʃd-$\emptyset$-i-n idi & \armenian{գաշդին իդի} & ɡ-eɾtʰ-ɑj-i-n& \armenian{կ՚երթային} \\
\hline 
 & \multicolumn{2}{l|}{Marash `to come'} & \multicolumn{2}{l|}{Marash `to come'} & \multicolumn{2}{l|}{cf. SWA} \\ \hline
1SG &ɡɑ-ɡʰɡ-i-$\emptyset$ & \armenian{գագՙգի} &ɡɑ-ɡʰɡ-$\emptyset$-i-$\emptyset$ idi & \armenian{գագՙգի իդի} & ɡ-eɾtʰ-ɑj-i-$\emptyset$& \armenian{կու գայի} \\
2SG &ɡɑ-ɡʰɡ-i-j & \armenian{գագՙգիյ} &ɡɑ-ɡʰɡ-$\emptyset$-i-j idi & \armenian{գագՙգիյ իդի} & ɡ-eɾtʰ-ɑj-i-ɾ& \armenian{կու գայիր} \\
3SG &ɡɑ-ɡʰɡ-e-j & \armenian{գագՙգէյ} &ɡɑ-ɡʰɡ-e-$\emptyset$-j idi & \armenian{գագՙգէյ իդի} & ɡ-eɾtʰ-ɑ-$\emptyset$-ɾ& \armenian{կու գար} \\
1PL &ɡɑ-ɡʰɡ-i-nkʰ & \armenian{գագՙգինք} &ɡɑ-ɡʰɡ-$\emptyset$-i-nkʰ idi & \armenian{գագՙգինք իդի} & ɡ-eɾtʰ-ɑj-i-ŋkʰ&\armenian{կու գայինք}\\
2PL &ɡɑ-ɡʰɡ-i-kʰ& \armenian{գագՙգիք} &ɡɑ-ɡʰɡ-$\emptyset$-i-kʰ idi & \armenian{գագՙգիք իդի} & ɡ-eɾtʰ-ɑj-i-kʰ& \armenian{կու գայիք} \\
3PL &ɡɑ-ɡʰɡ-i-n & \armenian{գագՙգին} &ɡɑ-ɡʰɡ-$\emptyset$-i-n idi & \armenian{գագՙգին իդի} & ɡ-eɾtʰ-ɑj-i-n& \armenian{կու գային} \\
\hline 
& \multicolumn{2}{l|}{{\ind}-$\sqrt{}$-{\thgloss}.{\pst}-{\agr}}& \multicolumn{2}{l|}{{\ind}-$\sqrt{}$-{\thgloss}.{\pst}-{\agr} {\pst} }& \multicolumn{2}{l|}{{\ind}-$\sqrt{}$-{\thgloss}-{\pst}-{\agr}}\\
\hline 
\end{tabular}\end{table}

\subsubsection{Progressive forms}

\translatorHD{In SWA, the indicative present and past imperfective are made progressive by adding the progressive enclitic /ɡoɾ/. Though this marker is banned in formal writing. }
The progressive forms are absent in Hadjin. But the Marash subdialect has them, and they're formed with the formative /ɡo/ <\armenian{գօ}>. This formative is not shortened next to vowels (Table \ref{tab:Cilicia:morpho:verb:marash:prog}). 


\begin{table}[H]
 \caption{Progressive forms in the Marash subdialect of the Cilicia dialect }\label{tab:Cilicia:morpho:verb:marash:prog}
\centering
 \begin{tabular}{|l| l l| ll|}
 \hline &\multicolumn{2}{l|}{Marash (Cilicia) } & \multicolumn{2}{l|}{cf. SWA} \\ 
 \hline 
 `I am liking' & ɡo siɾ-i-m & \armenian{գօ սիրիմ} & ɡə-siɾ-e-m ɡoɾ & \armenian{կը սիրեմ կոր} \\
 `he is rising' & ɡo ill-e-$\emptyset$ & \armenian{գօ իլլէ} & ɡ-ell-e-$\emptyset$ ɡoɾ & \armenian{կ՚ելլէ կոր} \\\hline 
&\multicolumn{2}{l|}{{\ind} $\sqrt{}$-{\thgloss}-{\agr}}&\multicolumn{2}{l|}{{\ind}-$\sqrt{}$-{\thgloss}-{\agr} {\prog}}\\
 `he is taking' & ɡo ɑr-n-u-$\emptyset$ & \armenian{գօ առնու} & ɡ-ɑɾ-n-e-$\emptyset$ ɡoɾ & \armenian{կ՚առնէ կոր} \\
&\multicolumn{2}{l|}{{\ind} $\sqrt{}$-{\vx}-{\thgloss}-{\agr}}&\multicolumn{2}{l|}{{\ind}-$\sqrt{}$-{\vx}-{\thgloss}-{\agr} {\prog}}\\
\hline 
 `I was liking' & ɡo siɾ-ɑ-$\emptyset$& \armenian{գօ սիրա} & ɡə-siɾ-ej-i-$\emptyset$ ɡoɾ & \armenian{կը սիրէի կոր} \\
&\multicolumn{2}{l|}{{\ind} $\sqrt{}$-{\thgloss}.{\pst}-{\agr}}&\multicolumn{2}{l|}{{\ind}-$\sqrt{}$-{\thgloss}-{\pst}-{\agr} {\prog}}\\
\hline 
\end{tabular}

\end{table}

\subsubsection{Future and future perfect}

\translatorHD{In SWA, the future is formed by taking the finite verb form of the indicative present (= minus the indicative prefix), and then adding the future proclitic /bidi/. The future perfect is similarly formed by taking the finite verb form of the indicative past imperfective and then adding this future proclitic. Adjarian describes the Cilicia dialect as doing a similar strategy.}

In the Marash subdialect, the future has two forms. The `ordinary future' (\armenian{հասարակ ապառնի}) is formed with the typical formative /bide/ <\armenian{բիդէ}> (related to SWA /bidi/ <\armenian{պիտի}>), and the `immediate future' (\armenian{անմիջական ապառնի}) which is formed with the verb /izil/ `to want' (related to SWA /uzel/ <\armenian{ուզել}> `to want').

\begin{exe}
\ex\label{sent:Cilicia:morpho:verb:fut:marash} \begin{xlist}
 \ex Marash (Cilicia) 
 \begin{xlist}
 \ex \gll bide biɾ-i-m \\
 {\fut} bring-{\thgloss}-1{\sg} \\
 \trans `I will bring' \\ 
 \armenian{բիդէ բիրիմ}
 \ex \gll ɡ-iz-i-m biɾ-i \\
 {\ind}-want-{\thgloss}-1{\sg} bring-{\thgloss}? \\
 \trans `I will immediately bring' \\ 
 \armenian{գիզիմ բիրի}
 
 \end{xlist}
 \ex cf. SWA
 \ex \gll bidi beɾ-i-m \\
 {\fut} bring-{\thgloss}-1{\sg} \\
 \trans `I will bring' \\ 
\armenian{պիտի բերեմ} 
\end{xlist} 
\end{exe}


In Hadjin, the future formative is shortened to /b/ (from CA /p/ <\armenian{պ}>), while the future perfect (\armenian{անցեալ ապառնի}) is formed with the aforementioned Turkish formative /idi/ <\armenian{իդի}>. The following are the repeated futures of the verbs `to go' and `to come' (Table \ref{tab:Cilicia:morpho:verb:paradigm:fut:Hadjin}, \ref{tab:Cilicia:morpho:verb:paradigm:futPerf:Hadjin}).




\begin{table}[H]
 	\centering 
 	\caption{Future <\armenian{ապառնի}> of the verbs `to go' and `to in the Hadjin subdialect of the Cilicia dialect}
 	\label{tab:Cilicia:morpho:verb:paradigm:fut:Hadjin}
 \begin{tabular}{|l|ll| ll| }
 		\hline & \multicolumn{2}{l|}{Hadjin `to go'} & \multicolumn{2}{l|}{cf. SWA} \\ \hline 
1SG &b-iʃd-o-m & \armenian{բիշդօմ} & bidi jeɾtʰ-ɑ-m& \armenian{պիտի երթամ} \\
2SG &b-iʃd-o-s & \armenian{բիշդօս} & bidi jeɾtʰ-ɑ-s& \armenian{պիտի երթաս} \\
3SG &b-iʃd-o-$\emptyset$ & \armenian{բիշդօ} & bidi jeɾtʰ-ɑ-$\emptyset$& \armenian{պիտի երթայ} \\
1PL & b-iʃd-o-nkʰ & \armenian{բիշդօնք} & bidi jeɾtʰ-ɑ-ŋkʰ& \armenian{պիտի երթանք} \\
2PL &b-iʃd-e-kʰ & \armenian{բիշդէք} & bidi jeɾtʰ-ɑ-kʰ& \armenian{պիտի երթաք} \\
3PL &b-iʃd-o-n & \armenian{բիշդօն} & bidi jeɾtʰ-ɑ-n& \armenian{պիտի երթան} \\
 		\hline & \multicolumn{2}{l|}{Hadjin `to come'} & \multicolumn{2}{l|}{cf. SWA} \\ \hline 
1SG &b-iɡʰɡ-o-m & \armenian{բիգՙգօմ} & bidi kʰ-ɑ-m& \armenian{պիտի գամ} \\
2SG &b-iɡʰɡ-o-s & \armenian{բիգՙգօս} & bidi kʰ-ɑ-s& \armenian{պիտի գաս} \\
3SG &b-iɡʰɡ-o-$\emptyset$ & \armenian{բիգՙգօ} & bidi kʰ-ɑ-$\emptyset$& \armenian{պիտի գայ} \\
1PL & b-iɡʰɡ-o-nkʰ & \armenian{բիգՙգօնք} & bidi kʰ-ɑ-ŋkʰ& \armenian{պիտի գանք} \\
2PL &b-iɡʰɡ-e-kʰ & \armenian{բիգՙգէք} & bidi kʰ-ɑ-kʰ& \armenian{պիտի գաք} \\
3PL &b-iɡʰɡ-o-n & \armenian{բիգՙգօն} & bidi kʰ-ɑ-n& \armenian{պիտի գան} \\
\hline 
& \multicolumn{2}{l|}{{\fut}-$\sqrt{}$-{\thgloss}-{\agr}}& \multicolumn{2}{l|}{{\fut} $\sqrt{}$-{\thgloss}-{\agr}}\\
\hline 
\end{tabular}
\end{table}


\begin{adjarianpage}\label{page:205}\end{adjarianpage}% should be 205


\begin{table}[H]
 	\centering 
\caption{Future perfect <\armenian{անցեալ ապառնի}> of the verbs `to go' and `to go' in the Hadjin subdialect of the Cilicia dialect}
 	\label{tab:Cilicia:morpho:verb:paradigm:futPerf:Hadjin}
 \begin{tabular}{|l|ll| ll| }
 		\hline & \multicolumn{2}{l|}{Hadjin } & \multicolumn{2}{l|}{cf. SWA} \\ \hline 
1SG & b-iʃd-i-$\emptyset$ idi & \armenian{բիշդի իդի} & bidi jeɾtʰ-ɑj-i-$\emptyset$& \armenian{պիտի երթայի} \\
2SG & b-iʃd-i-j idi&\armenian{բիշդիյ իդի} &bidi jeɾtʰ-ɑj-i-ɾ & \armenian{պիտի երթայիր} \\
3SG & b-iʃd-e-j idi & \armenian{բիշդէյ իդի} & bidi jeɾtʰ-ɑ-$\emptyset$-ɾ& \armenian{պիտի երթար} \\
1PL & b-iʃd-o-nkʰ idi& \armenian{բիշդօնք իդի} & bidi jeɾtʰ-ɑj-i-ŋkʰ& \armenian{պիտի երթայինք} \\
2PL &b-iʃd-e-kʰ idi& \armenian{բիշդէք իդի} & bidi jeɾtʰ-ɑj-i-kʰ& \armenian{պիտի երթայիք} \\
3PL &b-iʃd-i-n idi& \armenian{բիշդին իդի} & bidi jeɾtʰ-ɑj-i-n& \armenian{պիտի երթային} \\
 	\hline & \multicolumn{2}{l|}{Hadjin } & \multicolumn{2}{l|}{cf. SWA} \\ \hline 
1SG & b-iɡʰɡ-i-$\emptyset$ idi & \armenian{բիգՙգի իդի} & bidi jeɾtʰ-ɑj-i-$\emptyset$& \armenian{պիտի երթայի} \\
2SG & b-iɡʰɡ-i-j idi&\armenian{բիգՙգիյ իդի} &bidi jeɾtʰ-ɑj-i-ɾ & \armenian{պիտի երթայիր} \\
3SG & b-iɡʰɡ-e-j idi & \armenian{բիգՙգէյ իդի} & bidi jeɾtʰ-ɑ-$\emptyset$-ɾ& \armenian{պիտի երթար} \\
1PL & b-iɡʰɡ-i-nkʰ idi& \armenian{բիգՙգինք իդի} & bidi jeɾtʰ-ɑj-i-ŋkʰ& \armenian{պիտի երթայինք} \\
2PL &b-iɡʰɡ-e-kʰ idi& \armenian{բիգՙգէք իդի} & bidi jeɾtʰ-ɑj-i-kʰ& \armenian{պիտի երթայիք} \\
3PL &b-iɡʰɡ-i-n idi& \armenian{բիգՙգին իդի} & bidi jeɾtʰ-ɑj-i-n& \armenian{պիտի երթային} \\
\hline 
& \multicolumn{2}{l|}{{\fut}-$\sqrt{}$-{\thgloss}.{\pst}-{\agr} {\pst}}& \multicolumn{2}{l|}{{\fut} $\sqrt{}$-{\thgloss}-{\pst}-{\agr}}\\
\hline 
\end{tabular}
\end{table}

\subsubsection{Non-finite forms}

\translatorHD{What is often called the `past participle' has different meanings and functions per dialect. In SWA, the `past participle' is either the resultative participle with suffix /-ɑd͡z/, or the evidential participle with suffix /-eɾ/. These participles are both are used to form the present perfect or past perfect; the resultative has non-evidential connotation while the evidential has an evidential connotation. For SEA, there is a resultative participle with /-ɑt͡s/ and a perfective converb with the suffix /-el/. The perfective converb is used for the present perfect and past perfect. }



The past participle (\ref{sent:Cilicia:morpho:verb:pastparticiple}) has the form /-iɾ/ <\armenian{իր}> in Marash, and /-ij/ <\armenian{իյ}> in Hadjin and Zeytun, based on the regional pronunciation. The form /-od͡z/ <\armenian{օձ}> (from CA /-ɑt͡s/ <\armenian{ած}>) is more commonly used. But there is also the formative /-mon/ <\armenian{մօն}> (Greek <ménos>), for passive (\armenian{կրաւորակերպ}) verbs. 

 
\begin{exe}
 \ex Cilicia \label{sent:Cilicia:morpho:verb:pastparticiple}
 \begin{xlist}
 \ex ɡiɾ-iɾ e \\
 eat-{\eptcp}? {\aux} \\
 \trans `He has eaten' \\
 \armenian{գիրիր է}
 \ex ɡiɾ-ij e \\
 eat-{\eptcp} {\aux} \\
 \trans `He has eaten' \\
 \armenian{գիրիյ է}
 \ex ɡiɾ-od͡z e \\
 eat-{\rptcp} {\aux} \\
 \trans `He has eaten' \\
 \armenian{գիրօձ է}
 \ex \gll ipʰ-mon e \\
 eat-{\pass}.{\rptcp} {\aux} \\
 \trans `It is cooked.' \\
 \armenian{իփմօն է}
 \ex \gll pʰor-mon e \\
 spread-{\pass}.{\rptcp} {\aux} \\
 \trans `It is spread.' \\
 \armenian{փօռմօն է}
 \end{xlist}
\end{exe}
\section{Background (continued) and literature}

It is clear that starting from the west regions of Cilicia until the borders of Smyrna and Nicomedia, there is not Armenian language. The local language, Turkish, has turned into the native language. But Armenian is still preserved in some villages. These are Stanoz (Yenikent) (western side of Ankara), Sivrihisar (south-west of it), Nallıhan (the north-west side of Stanoz), and a few villages next to Yozgat. Information is lacking on these places. For the language of Stanoz (Yenikent), there is some information and a small manuscript in \citeauthor{Byurakn} (1899, page 670; 1900, page 233). Although these pieces are not entirely sufficient for studying the language of these areas, but they appear to show that they as well form subdialects of Cilicia. 


\begin{adjarianpage}\label{page:206}\end{adjarianpage}% should be 206

\section{Text samples}

{\sampleoverview}

 \subsection{Zeytun dialect}
 
Adjarian's source: Taken from \citet[159]{Allahverdian-1884-UlniaZeytun}. In its new form, it was narrated to me by Mr. Onnik Mahmtesian (\armenian{պր. Օննիկ Մահտեսեան}) and I wrote it in scientific orthography. Instead of the voiceless sounds /p, k, t, t͡s, t͡ʃ/ <\armenian{պ, կ, տ, ծ, ճ}>, he would sometimes use voiced aspirates; and Mr. Yaghoupian (\armenian{պր. Եաղուպեան}; \translatorHD{SEA: /jɑʁupjɑn/, SWA: /jɑʁubjɑn/}) would always do that. The latter would not change the sound /ə/ <\armenian{ը}> to /j/ <\armenian{յ}>. 


\armenian{Թա̈քա̈վիւյ մը ու վէզիյ մը թէփդիլ բՙիլլիլ գՙացըն ու գՙէղան (կամ տճկ. քէօյան) ջիւթը ախգՙոդ միգի մը դունը իճօն. ա̈ն գՙիշիյը դօնդիւյիւչը գը°նը°գը բօլուզ մը ունցօվ. իյինք ա̈ հէօն իչնալնըն փուշմօն էղօն. ա̈ն սահօթը նիքսէն (մառան) դէլըղանլը մը, մօդով, դՙօյս իլլիլն իքէն թա̈քա̈վիւյը քը°շպիլ գիմի ասօց. «Ջա՛նըմ, տօն չօ՞ց մօյդՙ իս. մինք հէօս դըսդիլո բիլո գօ գամըչնօնք ու դՙօն իգիյ նիքսէն մօդօյ»։ Դէլըղանլըն ասօց «Ջա՛նըմ, ինծօն ինծօն թա̈քա̈վիյէ մը գը°նը°գը աշգին բիդի մ՚ունցօվ ու էօսիւյ ջագօդը կըյիցի թը ա̈ս բօլուզը ա̈ն աշգինը բիդի առնու»։ ա̈ն սահօթը թա̈քա̈վիւյը շաշմըշ էղօվ թը «հէչ միւքի՞ւն է յույ (որ) մէյ աշգինը էօսիւյ դօնք»։ Սօնղրո թա̈քա̈վիւյը ու վէզիյը դանուշուխ էյան թը «չօ՞ց թէվույ (կերպ) ա̈դՙ բօլուզը գանընք ըսպանի օլա (արդեօք)»։ Վէզէյը ասօց. «Էյգօնը (էրկանը) ասինք թը մէ՛գիս թա̈քա̈վիւյ ինք, մէգս ա̈ վէզէյ. ա̈ս դէղվանքը թէփդիլ գօ բՙիլլինք թը իշտի (արդեօք) բօլուզ մը գանը՞նք գՙը°նի (գնել}). \armenian{չիւնքիւ մէ՛գս ա̈ բօլուզ չունանք. ը°ռանդ էղօվ. տօն չիւյ հինգ բօլուզ ունաս. կուհանօմ Ասպօձ. հումմօյ ա}̈ (\armenian{հիմայ ալ) տօն բօլուզ մը ունցօյ. ու գըյցընիլո բՙօն չունաս. քէլե օդո նիւյ էղօձ բօլուզը միզ ձախե. գօշօռէօքը իսկի գուդօնք»։}

\armenian{Իփիյ բօլուզը առըն ու քէօյէն տօյս կացըն, թա̈քա̈վիւյը ասոց վէզիյն թը «հուդո զէնինք քայինք (ձգենք)»։ Վէզիյն ա̈ ասոց թը «զէնիլօն՝ օսո ույմունան (թրք. օրման «անտառ») մեչը քայինք, ինքիյէն գու մէռնա»։ ա̈ն դէղվը°նը°ն չուբօն մը դավօյ գօգայձէյ. էձուն մէգը կօնօց, ա̈ն բօլուզը ձը°ը°ցուց։ Քէլէ՛ յույ (արի տես որ) ա̈ձը բառբուգօն մ՚էյ. իփըյ իյգօն էղօվ, չուբօնը դունը գՙօնօց. բառբուգը ջէնչից չուբունան վիյա̈ն թը «Ասըձե չի՞ս վախի}.



\begin{adjarianpage}\label{page:207}\end{adjarianpage}% should be 207

\armenian{չօ՞ց իս մէգ էձուգ ունամ նը՝ ա̈՛ն գօ գը°թիս»։ Չուբօնը ասոց. «Մինձ անա, իս խաբօյ չունամ»։ Ու իփըյ դանգնամազը էղօվ՝ ա̈վա̈ չուբօնը այձիլ գՙօնօց. ը°յը°գվը°նե իգիք չուբօնը նը՝ ա̈վա̈ բառբուգը ջէնչից. «Ջա՛նըմ, ա̈վա̈՞ գը°թիցիյ. չօ՞ց Ասըձե չը վախոձ մէգ մ՚իս էղի դՙօն»։ ա̈ն միգա̈լ էօյը բառբուգը դառընմըշ (նեղանալ) նա̈լէօվ՝ իլօվ էձգօնը իդիվը գՙօնօց. ա̈ձն ա̈ այձիլէն այձիլէն իլօվ իւղիւյթ բը°լը°զան խէչը գՙօնօց ու գօ ձը°ձցը°նէյ։ Բառբուգը ույախնիլէօվ առօվ բօլուզը՝ դունը գՙօնօց. ջօվօց դՙըյացնիյը, ասոց թը «Ամա՛ն, չուբօնը չօ գը°թե էղիյ է էձուգիս. իլլա}̈ (\armenian{այլ) իմ էձուգը բը°դիգ մը ունցիյ է»։ ա̈ս բօլուզը ձառիցին (խնամել}), \armenian{չիւյ մինձձՙօվ։ Քօսոն քըսանը հինգ դայվօն սօնղրո, ան թա̈քա̈վիւյը ու վէզիյը պիլլիլէն իգին աս բը°լը°զան քէօյը. գու դիսնուն յույ բօլուզ մը գո, ու էօնիւյ էձու դօղո գօջօվօն։ Թա̈քա̈վիւյը ասոց թը «Ջա՛նըմ, չույո՞ւ էօդիւյ էձու դօղո գօջը°վա̈ք»։ Էնիյ ա̈ էօնիւյ ասըն թը «Բառօբ մը էձուգ մ՚ունէյ. ա̈ս ա̈ձը դայիմ բայօբ գուգՙէյ. ձը°ձը° գոթ չէյ գինէյ. էօյ մը բառօբը գՙօնոց դիսօվ յույ ա̈ձը բըդիգօն մը էղեղը (վրայ) չուքիյ է՝ գօ ձը°ձցը°նե. առօվ դունը բՙիյօվ ու ձառից, մօնձձՙուց դէհի, զօ ջօվօն էձու դօղո։» ա̈ն ադինը թա̈քա̈վիւյը ու վէզիյը դանուշուխ էյան թը ա̈դ մէյ չալուն մէչ քայօձ բօլուզն է. դՙեռ սող գիցիյ է։ Մէգ-միգի ասըն թը էօդիւյ ծառը (ձՙա̈ռը) թուխթ մը դօնք, խափինք ու գՙօլօխը դօնք գը°դրի։ Ծը}° (\armenian{ձՙը}°) \armenian{մը դը°վը°ն, թուխթ մա̈ թը օսո դէ ֆօլօն թա̈քա̈վիյէն սէյա̈ն դօյ. ա̈ն ա̈ իլօվ ծը°ն հէձօվ, թուխթն ա առօվ դայօվ։ Բօլուզը իւղիւյթ սէյա̈ն գՙօնօց. գօյթէյ նը՝ սային քէօվը բաշջո մը գէյ. էձուն դէղա̈ն մը°դքէօվը ասոց թը օսո բէշջա̈ն բառգիմ}X\armenian{գու դանըմ զահայ (հարկաւ)։ Հէօն բառգիլն իքէն, թա̈քա̈վիյէն բը°դիգ աշգինը դիսօվ յույ բաշջին մէչը գը°դռը°ջ մը բառգիյ է. աշգինը գՙօնօց, գը°դռը°ջը°ն քէօվը գանիցօվ. ու զա̈ն հավնիցօվ. դիսօվ յույ ձիւցը թուխթ մը գո զա̈ն բՙացօվ, գայդՙօց, դիսօվ յույ մեչը կօյմօն է թը ա̈ս բօլուզը չիւյ իգՙո նը (հազիւ եկած)՝ չուխուսուցունիլէօվ քէլլա̈ն գը°դրիցէք։ Աշգինը ա̈ն գՙօյվօձը անցուց ու ինք ա̈ գՙը°յից թը սայիս (պալատիս) դՙէմը սայո մը շինիցէք ու քառսուն էօյ հայսնըք էյէք, բը°դիգ աշգինս ա̈ էօսիւյ դը°վէք. խը°փնից (գոցեց}), \armenian{ա̈վա̈ ձիւցը դՙօյօվ։ Չիւյ ա̈ն ադինը իլօվ, բօլուզը թուխթը սէյա̈ն դայով։ Թուխթը բՙացըն, գայդՙացըն, դիսօն յույ թա̈քա̈վիյէէն ու վէզիյէն մէօհիւյը}... 


\begin{adjarianpage}\label{page:208}\end{adjarianpage}% should be 208

... \armenian{մէչը գո. գՙօյվօձը գիմի էյը°ն։ Շինիցին սէյա̈ն, քառսուն էօյ ու քառսուն գՙիշիյ հայսնըք էյան, ա̈ն աշգընան հիդը փսագիցին։ Մէք-քանը՛ դայը՛ սօնղռո՝ թա̈քա̈վիւյը ու վէզիյը իփըյ էգին, թիւմ (ամբողջ) քախքըցը°ք դՙէմ գՙացըն. իյ էնիշդա̈ն ա̈ քէօմէօլու (շատ) սանսալաթէօվ տէմ իլօվ։ Թա̈քա̈վիւյը իփըյ ա̈ն դէղա̈ն ջէնչօց, շաշմըշ էղօվ մօնօց։ Իդքը ջօվօց իյ վէքիլխօյջը ու ասօց. «ա̈ս պռնաձնըդՙ չի՞յք է. ա̈ս բօլուզը չույո՞ւ ինդա̈ս էյը°ք»։ Չիւյ ա̈ն ադինը վէքիլխօյջը թուխթը ձուցէն հանից, թա̈քա̈վիյէն ու վէղիյէն ցօցուց. էնիյ ա̈ իփըյ թուխթը դիսօն, զայմացօն։ Չիւյ ա̈ն ադինը թա̈քա̈վիւյը թուխթը բաքօվ, դՙօլխօն դՙօյօվ ու դը°ղան ասօց թը «Դՙօն ա̈լ իմ դէղան իս. գՙօյվօձը չա՛նցնը էղիյ է»։}

\subsection{Hadjin subdialect}

Adjarian's source: See \citeauthor{Byurakn} 1899, page 41 and 1900, page 331. 



– \armenian{Բՙարէվ բաբա։}

– \armenian{Ասձօ բՙարին։}

– \armenian{Ընչօ՞ց իս։}

– \armenian{Ըռինդ իմ։}

– \armenian{Ուսդէ՞ գագՙօս։}

– \armenian{Սէհէլէն։}

– \armenian{Ի՛նչ գօ, ի՛նչ չօգՙօ։}

– \armenian{Ըռընդութին։}

– \armenian{Բՙանվընիդ ընչօ՞ց էր։}

– \armenian{Է՜, զարօր չունէր։}

– \armenian{Բօբի՞դ ընչօց էր։}

– \armenian{Ա դարի քիչ մը քէֆլու է։}

– \armenian{Ասէլ է էփէյի բօրդք դըվիք։}

– \armenian{Փօռք Ասդըձու. Էրէջէբին դղին հազօր ղրուշ բօրդք բարդօնք իդի, վեց հէրիրը դըվօնք՝ չօրս հէրիրը մնօց. ֆայիզն ա վրան գՙրօվ ՝ավա ինը հէրիր ղրուշի սէնէդ մը դըվօնք։ Թըռսարգՙիսին գըգին ա ձիք (երեք. հմմտ. Ննխջ. ժէք) հէրիր ղրուշ բօրդք բարդօնք, էշվընիս էրգու հօդ էր՝ մէգը դըվօնք, մնացաձին ա ֆայիզէվի էրգու հէրիր իսուն ղրուշի սէնէդ մը դըվօնք. վէրգունիս ա քիչ մը թիթիֆցընէնք նէ՝ մնէղ բուլաշըղէվէն ա դընվընիս էշունքը գու լէցնինք, բըլըզդէքէ րահօթ գէնէ։}

\begin{adjarianpage}\label{page:209}\end{adjarianpage}% should be 209

\armenian{Փէօյիդ խանձի. հէգՙգիդ խանձի. Ասդվօձ հէգՙգիդ առնօ. չդինօս գէնջութօնիդ խայիյը (թրք. խայիր}). \armenian{խօգ մէռնիս. խօգ սադգիս. մույը բէլէվիս (մուրը պլլուիս}). \armenian{չհասնիս գէնջութօն. սիյդիդ դՙուռը թէգինօ (թող կենայ}). \armenian{օջախդ անցնի. էչքիդ գույնօ (աչքդ կուրանայ}). \armenian{թիվիդ խանձի։}

\subsection{Marash subdialect}

Adjarian's source: See \citeauthor{Byurakn} 1899, page 405 and 101.

\armenian{Գէղցա մը քաղօք իգիլ է, բէզօրու բիլլիլլ-իքէն՝ դօլանդըրըջը (խաբեբայ) մը գօ հասգընօ քի իւսիւր քէօվիւ. փարօ գօ, ույունմուշ գօննօ իւսիւր։}

\armenian{Յօր քի դէսնա «քէռա՛, բարիվ իգիր, չօ՞ց իս, ըռընդ իս}…\armenian{» գօսէ։}

\armenian{Գէղցան է գըսէ քի «լա՛ դղօ, իս քին քէռադ չիմ»։}

– \armenian{Չէ՛, իլլէ դօն իմ քէռաս իս. իս քիննէ չիմ անցնա։ Ինդէս գօսէ դօլանդըրըջըն։}

\armenian{Էօր մ՚է գօսէ քի – Քէ՛ռա, իգօ իսգի (զքեզ) քէբաջուվա խանութ մու դանամ դէ՝ քիղ քէբօբ մու դըրցնիմ։}

\armenian{Գէղցան էս լսօձու գիմի քէֆ գօ էնէ. առչիվի գօ ընգնա՝ բարաբօր գօրթօն։}

\armenian{Խանութու գօ մըդնան, իվիրի գօ իլլին, քէբօբու ուտուլէն սօնրօ՝ դօլանդըրըջըն գօսէ քի. «Քէռա, իվէր ըրթօմ դէ այրօն մու բիրըցնիմ (բերել տամ}), \armenian{խումինք»։}

\armenian{Իվէր գօ իչնա՝ քէբաբջըվան գօսէ քի «ահա իս գօրթօմ, իվիրի գինօղ մօրթէն փարէն օռ (առ) դէ փռցու»։}

\armenian{Ինդէս գօսէ՝ հէրիֆի դօրս գօ իլլէ գօրթօ։ Գէղցան մէգ բէքլէմիշ գօ էնէ, իրգու բէքլէմիշ գօ էնէ՝ սօնսօնու իվէր գօ իչնա, գօ} X\armenian{իսնա՛ քի հէրիֆի գացիլ ա. քէբաբջուն է ինքիրնէ փարօ գօ ուզէ։}

\armenian{Գէղցան մէգ իրիսի լացօձ, մէգ իրիսի ձըձաղօձ՝ «հազօր դաբու իս քին քէռադ չիմ ըսա դէ, չէ իլլէ դօն իմ քէռաս իս ըսօձ» ըսիլէն փարին ձրօրու գօ աձգէ։}


\begin{adjarianpage}\label{page:210}\end{adjarianpage}% should be 210



\begin{center}
 * *
\end{center}

\armenian{Էօր միւ գէղցա մը ի ժամու դէրբաբուն խուսդուվանիլլիքէն գօսէ քի. «Դէրբօբ, մօրթ մու սբանըցի, էն մէխքի չիմ համրի. մօրթ մու գէղվըցի, էն է չիմ համրի. դուն մու բադռըցի, էն է չիմ համրի. ամմա բօք (պահք) էօր միւ իչյաղը միմէն չըբուխիս վառըցի, էս մէխքաս հազօ՜ր մէղօ. թուղութուն դուր, դէ՛րբօբ»։}

\armenian{Դէրբօբն է գօսէ քի. «Ի՛լ ի՛լ, էդ մէխքն է իս չիմ համրի»։}

\subsection{Kessab village of Antioch}

Adjarian's source: See \citeauthor{Byurakn} 1900, page 731. 

\armenian{Իս ձՙի զիւղիւրթ զիւղիւրթ գէսիմ թը ցէրինը հէօդ չիննէյրը գՙիյդէյն ու միռնայր, էնք բէս գօ. ըմմը թը գը միռնա, շիւդ մահսօլ գօ դու։ Ան ըր զէյր վիւջիւդը գը սիրիյ՝ զան գը սիբի. օ ան ըր զէյր վիւջըդդը գը գՙօդդի (կ՚ատէ}), \armenian{յաս էխշօր էբէդի էօմըրէն գօ բըհի զան. ան ըր զէյս գը բաշդի՝ յէմ իդդը դէյ գօ գՙու. օ իս յէօր գօ գՙինօմ, էյմ խըզմէթջին է հօն գը գինօ. թէ զէյս գը բաշդի, էյմ դօդն է էնիւր իքրամ գինի։}

\subsection{Stanoz (Yenikent) subdialect }

Adjarian's source: See \citeauthor{Byurakn} 1899, page 443. For this, see this and the next subsection. 

\armenian{Մօյրամ, Մօյրամ, մօյր Ասդուծօ},

\armenian{Քո՞ւն էս մը ղարթուն էս մը}.



\begin{adjarianpage}\label{page:211}\end{adjarianpage}% should be 211


\armenian{Օչ օր քուն էմ, օչ օրր զարթուն էմ}.

\armenian{Գՙըշէրին գէսը մէգ էլաձ (երազ) դէսօ},

\armenian{Գըրագը ընգօ չօրէցօ}.

\armenian{Դէնիզ ընգօ չը խըդդըվէցօ},

\armenian{Անդի դիւնյօն գնացի՝ չը գօրսնըվէցօ։}

~


\armenian{Օհօն Օհօն Էսգիբիրօն},

\armenian{Գաբէս ամէն չարուս բիրօն},

\armenian{Գՙավազանը դուռիս վիրօն},

\armenian{Խաչը գուսգիս վիրօն},

\armenian{Փիլօնը օրթիգիս վիրօն},


\armenian{Թէօօթ հայրաբէդին աղօթքը}

\armenian{Ամէնին վիրօն ամէն։}


\chapter{Syria}\label{chapter:Syria}
\section{Background}

\begin{adjarianpage}\label{page:212}\end{adjarianpage}% should be 212


In the \citeauthor{HandesAmsorya} 1907, page 27, there was a small section in the dialect of the village of Aramo. Based on the article, this small Armenian-populated village is found near the village-city of Jisr al-Shughu in Syria. The language of the section was so far from Old Armenian, that a linguist would have had a difficult time understanding it. Because this manuscript didn't give much more extensive information, we thus cannot say if this dialect is special to only the village of Aramo, or if it is spoken also in other surrounding areas. The latter situation appears quite probable. The vernacular of Svedia, which is absolutely unfamiliar in the literature, could also belong to the same branch. There is event an excerpt of the language of a few Armenian villages of Antioch (see page 210), which has a lot of similarities with this language; and if we had a larger manuscript section, then we could perhaps say if the aforementioned villages of Antioch belong to the dialect of Aramo. Because of this, and because we did not consider the name ``Aramo'' to be sufficient enough; we wanted to a more general name and called it the Syrian dialect. 

\translatorHD{As a disclaimer, what Adjarian describes as Syrian Armenian is \textit{not} the variety of Armenian that most modern Armenians from Syria speak. The majority of Armenian-speaking Armenians in Syria, i.e., those in Aleppo or Damascus (and refugees abroad), speak (Standard) Western Armenian as their native language. The dialect that Adjarian describes here is a non-standardized variety of Armenian that developed in various villages in Syria (close towards the modern Turkish border) that is quite divergent from the (Standard) Western Armenian that is spoken by Armenians in urban areas. It is unfortunate that Adjarian gives this dialect a generic ambiguous name like `Syrian Armenian`.}

\translatorHD{As my own speculation, what I think Adjarian intended to say was that the dialect he describes here is a non-standard variety that developed \textbf{indigenously} by various villages near the Turkish border. During 1911, the majority of Syrian Armenians outside of these villages spoke Arabic (\S\ref{sec:Langauges:Arabic}). But after the Armenian Genocide of 1915, many Armenian refugees migrated to major Syrian cities (like Aleppo and Damascus), and those emerging communities spoke (or learned to speak) Standard Western Armenian, not this indigenous variety. } 
\section{Phonology}
\subsection{Vowels}
\subsubsection{Inventory}
By judging the aforementioned section from \citeauthor{HandesAmsorya}, which we will see later below, we can deduce that the Syrian dialect recognizes the vowels /ɑ, e, ə, i, o, u/ <\armenian{ա, է, ը, ի, օ, ու}>. But the vowels /æ, œ, ʏ/ <\armenian{ա}̈, \armenian{էօ, իւ}> are missing. The latter point is quite natural, because the Arabic language, which is in the native language of this area, does not have those sounds. In contrast to this, the Syrian dialect has a few diphthongs, which in other places are either rare or don't exist. These are /ɑi̯, ɑə̯, ɑu̯, ĕi̯, u̯ɑ/ <\armenian{ա}ʲ, \armenian{աը, աւ, է}ʲ, \armenian{ուա}>.\footnote{\translatorHD{For the diphthongs that I re-transcribe as /ɑi̯/ and /ĕi̯/, Adjarian uses the Armenian symbols <\armenian{ա}\textsuperscript{\armenian{յ}}, \armenian{է}\textsuperscript{\armenian{յ}}> with a superscript <\armenian{յ}>. But my text-processor cannot easily create such superscripts, so I instead use <\armenian{ա}ʲ, \armenian{է}ʲ> with a superscript <j>.}}

\begin{adjarianpage}\label{page:213}\end{adjarianpage}% should be 213


\subsubsection{Sound changes}
The following phonetic changes caught our eyes: 


\paragraph{Classical Armenian /ɑ/ <\armenian{ա}> }
 

Classical Armenian /ɑ/ <\armenian{ա}> has changed to /u/ <\armenian{ու}> (Table \ref{tab:Syria:phono:change:vowel:a:u}).

\begin{table}[H]
 \centering
 \caption{Change from Classical Armenian /ɑ/ <\armenian{ա}> to /u/ <\armenian{ու}> in the Syria dialect}
 \label{tab:Syria:phono:change:vowel:a:u}
 \begin{tabular}{|l|ll|ll|ll|}
 \hline & \multicolumn{2}{l|}{Classical Armenian}& \multicolumn{2}{l|}{> Syria }& \multicolumn{2}{l|}{cf. SEA }
 \\
 `bread' & hɑt͡sʰ & \armenian{հաց} & hut͡sʰ & \armenian{հուց} & hɑt͡sʰ & \armenian{հաց} \\
 `debt' & pɑɾt-(ə?)kʰ (-{\pl}) & \armenian{պարտք} & buɾkʰ & \armenian{բուրք} & pɑɾtkʰ & \armenian{պարտք} \\ 
 		`mouth' &beɾɑn & \armenian{բերան} & beɾun & \armenian{բէրուն} &beɾɑn & \armenian{բերան} \\ 
 `bad' & t͡ʃʰɑɾ & \armenian{չար} & t͡ʃʰuɾ & \armenian{չուր} & t͡ʃʰɑɾ & \armenian{չար} \\
\hline
 \end{tabular}
 
\end{table}
\paragraph{Classical Armenian /e/ <\armenian{ե}> }
 
 
Classical Armenian /e/ <\armenian{ե}> has changed to /i/ <\armenian{ի}> (Table \ref{tab:Syria:phono:change:vowel:e:i}).

\begin{table}[H]
 \centering
 \caption{Change from Classical Armenian /e/ <\armenian{ե}> to /i/ <\armenian{ի}> in the Syria dialect}
 \label{tab:Syria:phono:change:vowel:e:i}
 \begin{tabular}{|l|ll|ll|ll|}
 \hline & \multicolumn{2}{l|}{Classical Armenian}& \multicolumn{2}{l|}{> Syria }& \multicolumn{2}{l|}{cf. SEA }
 \\
 `our (we.{\gen})' & meɾ & \armenian{մեր} & miɾ & \armenian{միր} & meɾ & \armenian{մեր} \\
 `we (we.{\nom})' & mekʰ & \armenian{մեք} & mikʰ & \armenian{միք} & meŋkʰ & \armenian{մենք} \\
 `us (we.{\acc})' & əzmez & \armenian{զմեզ} & əzmi & \armenian{ըզմի} & mez & \armenian{մեզ} \\
\hline
 \end{tabular}
 
\end{table}


\paragraph{Classical Armenian /ə/ <\armenian{ը}> }

 
Classical Armenian /ə/ <\armenian{ը}> has changed to /ɑ/ <\armenian{ա}> (Table \ref{tab:Syria:phono:change:vowel:ə:ɑ}).

\translatorHD{Note that the forms he give all involve the definite suffix. This suffix is /-ə/ in SEA. But in Classical Armenian, there was no actual definite suffix. The ancestor of the modern SEA definite suffix /-ə/ is the CA distal suffix /-ən/. Thus the glossing is -{\defgloss} for SEA but -{\dist} for CA. }


\begin{table}[H]
 \centering
 \caption{Change from Classical Armenian /ə/ <\armenian{ը}> to /ɑ/ <\armenian{ա}> in the Syria dialect}
 \label{tab:Syria:phono:change:vowel:ə:ɑ}
 \begin{tabular}{|lll|ll|lll|}
 \hline & \multicolumn{2}{l|}{Classical Armenian}& \multicolumn{2}{l|}{> Syria }& \multicolumn{3}{l|}{cf. SEA }
 \\
 `bread-{\dist}' & h\'ɑt͡sʰ-ən & \armenian{հացն} & h\'ut͡sʰ-ə & \armenian{հո՛ւցա} & `bread-{\defgloss} (SEA)' & h\'ɑt͡sʰ-ə & \armenian{հացը} \\
 		`mouth-{\dist}' &beɾ\'ɑn-ən & \armenian{բերանն} & beɾ\'un-ɑ & \armenian{բէրո՛ւնա} &` mouth-{\defgloss}' & beɾ\'ɑn-ə & \armenian{բերանը} \\ 
 `debt-{\pl}-{\dist}' & pɑɾt-(ə?)kʰ-ən & \armenian{պարտքն} & buɾkʰkʰ-eɾ-ɑ & \armenian{բուրքքէրա} & `debt-{\pl}-{\defgloss}' & pɑɾtkʰ-eɾ-ə & \armenian{պարտքերը} \\ 
\hline
 \end{tabular}
 
\end{table}

\paragraph{Classical Armenian /i/ <\armenian{ի}> }

 
Classical Armenian /i/ <\armenian{ի}> has changed to /ɑ/ <\armenian{է}ʲ> (Table \ref{tab:Syria:phono:change:vowel:i:ĕi̯}).




\begin{table}[H]
 \centering
 \caption{Change from Classical Armenian /i/ <\armenian{ի}> to /ɑ/ <\armenian{է}ʲ> in the Syria dialect}
 \label{tab:Syria:phono:change:vowel:i:ĕi̯}
 \begin{tabular}{|l|ll|ll| lll|}
 \hline & \multicolumn{2}{l|}{Classical Armenian}& \multicolumn{2}{l|}{> Syria }& \multicolumn{3}{l|}{cf. SEA or SWA }
 \\
 \armenian{՝}sky' & eɾkin & \armenian{երկին} & jeɾɡĕi̯nkʰ & \armenian{յէրգէ}ʲ\armenian{նք} & jeɾkiŋkʰ & \armenian{երկինք} & SEA \\
 `soul-{\dist} (CA); soul-{\defgloss} (SEA)' & hoɡi-n & \armenian{հոգին} & hokʰĕi̯-n & \armenian{հօքէ}ʲ\armenian{ն} & hokʰi-n & \armenian{հոգին} & SEA \\
 `me (I.{\acc})' & zis & \armenian{զիս} & jĕi̯s & \armenian{յէ}ʲ\armenian{ս} & zis & \armenian{զիս} & SWA \\
\hline
 \end{tabular}

\end{table}


\paragraph{Classical Armenian /u/ <\armenian{ու}> }
 
 
Classical Armenian /u/ <\armenian{ու}> has changed to /ɑu̯/ <\armenian{աւ}> (Table \ref{tab:Syria:phono:change:vowel:u:ɑu̯}).

\begin{table}[H]
 \centering
 \caption{Change from Classical Armenian /u/ <\armenian{ու}> to /ɑu̯/ <\armenian{աւ}> in the Syria dialect}
 \label{tab:Syria:phono:change:vowel:u:ɑu̯}
 \begin{tabular}{|l|ll|ll|ll|}
 \hline & \multicolumn{2}{l|}{Classical Armenian}& \multicolumn{2}{l|}{> Syria }& \multicolumn{2}{l|}{cf. SEA }
 \\
 `holy' & suɾb & \armenian{սուրբ} & sɑu̯ɾpʰ & \armenian{սաւրփ} & suɾpʰ & \armenian{սուրբ} \\
 `name' & ɑnun & \armenian{անուն} & ɑnɑu̯n & \armenian{անաւն} & ɑnun & \armenian{անուն} \\
\hline
 \end{tabular}
 
\end{table}


\section{Morphology}
\subsection{Noun inflection or declension}


 In the text, we see that the accusative always uses the prefix /z/ <\armenian{զ}>, while the ablative and locative use the prefix /i/ <\armenian{ի}>. The latter is a deep archaism and it is found in no other dialect (Table \ref{tab:Syria:morpho:case:prefix}). 

 \translatorHD{As examples, Adjarian provides dialectal forms alongside a Classical Armenian gloss. Both varieties involve case-marking prefixes. Modern SWA/SEA lacks such prefixes. I found the dialectal forms sufficiently hard to segment and gloss. }


\begin{table}[H]
 \centering
 \caption{Case-marking prefixes in Syrian Armenian and Classical Armenian, but not Modern Standard Armenian}
 \label{tab:Syria:morpho:case:prefix}
 \begin{tabular}{|l|l |l |l |}
 \hline & Classical Armenian & > Syria & cf. SWA 
 \\\hline
 `at ground ({\locgloss})' & i ɡetin & i ɡedĕi̯nkʰ-ɑ &kʰedin-ə \\
 & {\locgloss} ground& {\locgloss} ground-{\defgloss} & ground-{\defgloss}\\
 & \armenian{ի գետին} & \armenian{ի գէդէ}ʲ\armenian{նքա}& \armenian{գետինը}
\\\hline `our bread ({\acc})' & əz-hɑt͡sʰ meɾ & əz-miɾ hut͡sʰ-ɑ & meɾ hɑt͡sʰ-ə \\
 & {\acc}-bread our & {\acc}-our bread-{\defgloss} & our bread-{\defgloss}
 \\
 & \armenian{զհաց մեր}& \armenian{ըզմիր հուցա} & \armenian{մեր հացը} 
 \\\hline
 `at proof ({\pl}?, {\locgloss})' & i pʰoɾd͡zɑn-(ə?)s & i pʰuɾt͡sʰunkʰ-ɑ & \\
 & {\locgloss} proof-{\pl}.{\acc} & {\locgloss} proof-{\defgloss} & 
 \\
 & \armenian{ի փորձանս}& \armenian{ի փուրցունքա} & 
 \\\hline
 `from evil ({\abl})' & i t͡ʃʰɑɾ-e-n & i t͡ʃʰuɾkʰ-ĕi̯-n &t͡ʃʰɑɾ-e-n \\
 & {\locgloss} evil-{\abl}-{\dist} & {\locgloss} evil-{\abl}-{\defgloss} & evil-{\abl}-{\defgloss}
 \\
 & \armenian{ի չարէն}& \armenian{ի չուրքէ}ʲ\armenian{ն} & \armenian{չարէն} 
 \\\hline
 `my mouth ({\acc})' & əz-beɾɑn im & z-im beɾ\'un-ɑ &im pʰeɾɑn-əs \\
 & {\acc}-mouth my & {\acc}-my mouth-{\defgloss} & mouth-{\possSsg}
 \\
 & \armenian{զբերան իմ}& \armenian{զիմ բէրո՛ւնա} & \armenian{իմ բերանս}
 \\\hline
 `my enemy ({\acc})' & əz-tʰəʃnɑmi-n im & z-im t͡ʃɑɾɡ\'um-ɑ & \\
 & {\acc}-enemy-{\dist} my & {\acc}-my enemy-{\defgloss} & 
 \\
 & \armenian{զթշնամին իմ}& \armenian{զիմ չարգո՛ւմա} & 
 \\\hline
 \end{tabular}
 
\end{table}

 

\section{Text samples}

{\sampleoverview}

\subsection{Aramo village}

Adjarian's source: See \citeauthor{HandesAmsorya}, 1907, page 27.


\armenian{Յա միր Դուդա իլ գուս յէրգէյնքա. սաւրփ ըննու քու անաւնա։ Ու ուքու քու ըրքայութէյնա ու ըննու քու րադադ չուսմա գու յէրգէյնքա հանց ըննու ի գՙէդէնքա։ Զմիր հուցա ի ամէն ջՙուք դաւղա մի՛ ըսցօր։ Ու դղշէ զմիր բուրքա չուսմա միք դղշինք միգ միգա բուրքքէրա։ Ու չը սալմէս ըզմի ի փուրցուն-քա. լաքին նաջջի ըզմի ի չուրքէյն։ Լաըն քու է ըրքայութէյնա ու քու ւաթա ու ըձզահամակա լքի յէդդայնքա ամեն։}

\armenian{Յա Ասդուձ զիմ բռունգունգա բՙուց ու զիմ բՙէրունա նաղնի քու զօրշհնութէյնա։ Օրշհնալ համաքում ու բըթթը հադաչ}... 



\begin{adjarianpage}\label{page:214}\end{adjarianpage}% should be 214


... \armenian{ընբՙըժնվազ սաւրփ Էրօրթօթէյնա զԴուդա ու զՋըթթա ու զՀօքէյն սաւրփա համագալքի յէդդայն յէդդայնքա ամեն։}

\armenian{Յա Ասդուձ շադէրա էղէյն նէղէյն էյէս, ու շադէրա էլայն էր վաս։ Շադէրա ասէցայն զիմ զունձայն, թի չքագյէր խալաս էր Ասդունձայն։ Լաքին դՙուն, յա Դիր, նացրն իս ու զիմ օձն իս ու զիմ գՙըլուխա բՙանցրացնուղա դՙուն իս։}

\armenian{Ղրմությամփ քուվ, Դիր, սադդէցօ զիմ չարգումա ու ղամնւ նէղի զիմ զուձա լաըն իս քում աբդն իմ։}

\armenian{Մաջդ Դուդա ու Ջըթթա ու Հէքէն սաւրփա։}




\chapter{Arapgir}
\section{Background}
\begin{adjarianpage}\label{page:215}\end{adjarianpage}% should be 215

In some of the villages of Arapgir, Divriği, Gürün, and Darende, and Kayseri, the Armenian language has quite common borders, so I combined them under one name; I call it the Arapgir dialect, because Arapgir is the largest center of this area. Divriği, Gürün, Darende, and Kayseri can form its subdialects.

\section{Literature}

The dialect of exactly the city of Arapgir has been studied in dialect by a local linguist, Melik S. Davit-Bek (\armenian{Մէլիք Ս. Դաւիթ բէկ}) (See \citeauthor{HandesAmsorya}, years 1900-1906). There, we find a few manuscripts, which were written with sufficient precision. There is a collection of riddles from Arapgir in \citeauthor{Byurakn}, 1900, page 135. For the other subdialects, there are the following text samples:



{\litoverview}

\begin{itemize}
 \item Literature with the Arapgir dialect
 \begin{itemize}
 \item Gürün subdialect: \citeauthor{Byurakn}, 
 \begin{itemize}
 \item 1898, page 839 
 \item 1899, page 410, 425, 478 
 \item 1900, page 331, 634

 \end{itemize}
 
\item Darende subdialect: \citeauthor{Byurakn} 1899, page 295, 498, 572. 
\item 
Subdialect of Kayseri villages:\footnote{The Armenians of the city of Kayseri of all westward areas are Turkish-speaking. But there are a few villages which speak Armenian. These villages are Efkere, Everek, Tomarza, Munjusun, Nize, Palahesi, Fenese. Similarly, the city of Yozgat is Turkish-speaking, but it has a few Armenian-speaking villages.} 
\begin{itemize}
 \item \citeauthor{Byurakn},
 \begin{itemize}
 \item 1898, page 331, 406, 454, 580, 647 
 \item 1899, page 74, 200 
 \item 1900, page 469, 636
 \end{itemize}
 \item \armenian{Բանասէր} 1902, \armenian{էջ} 174-5
\end{itemize}

\item Divriği subdialect: See \citeauthor{Eminian}, volume 6 (\armenian{Զ}.), page 206, 312, 327, 364, 378

 \end{itemize}
\end{itemize}



\begin{adjarianpage}\label{page:216}\end{adjarianpage}% should be 216


\section{Phonology}

The dialect of Arapgir has 7 vowels: /ɑ, æ, e, ə, i, o, u/ <\armenian{ա, ա}̈, \armenian{է, ը, ի, օ, ու}>. The consonants have three degrees (voiced, voiced aspirated, and voiceless aspirated). But from this angle, the area of the dialect of Arapgir can be divided into two major branches. The first branch has the dialect of Arapgir and the subdialect of Divriği; while the second branch has the subdialects of Gürün, Darende, and Kayseri. The primary difference of the latter branch is that there are no voiced aspirates among the consonants. Similarly whereas the Arapgir dialect has turned the CA diphthong /ɑi̯/ \armenian{այ} to /ɑ/ <\armenian{ա}>, Gürün and the other subdialects and others have turned it to /e/ <\armenian{է}>. 

\section{Morphology}

The general characteristics of the two branches and their subdialects are the following. 

\subsection{Instrumental marking with /-okʰ/ <\armenian{օք}> }


The instrumental formative is /-okʰ/ <\armenian{օք}> (instead of /-ov/ <\armenian{ով}>). This formative is the Classical Armenian plural instrumental formative for the ɑ-stems, which has taken here a singular meaning (Table \ref{tab:Arapgir:morpho:noun:inst}).


 \begin{table}[H] \centering
 \caption{Instrumental marking in the Arapgir dialect}
 \label{tab:Arapgir:morpho:noun:inst}
 \begin{tabular}{|l|ll|ll|l| }
 \hline & \multicolumn{2}{l|}{Arapgir} & \multicolumn{2}{l|}{cf. SEA} & \\ \hline 
 `with hand' & d͡zʰerkʰ-okʰ, & \armenian{ձՙէռքօք}, & d͡zerkʰ-ov & \armenian{ձեռքով} &$\sqrt{}$-{\ins} \\
 & d͡zerkʰ-okʰ & \armenian{ձէռքօք} & & &\\
 `with speaking' & χos-e-l-okʰ, & \armenian{խօսէլօք} & χos-e-l-ov & \armenian{խօսելով}&$\sqrt{}$-{\thgloss}-{\infgloss}-{\ins} \\
 \hline 
 \end{tabular}
\end{table}

\subsection{Mobile indicative marking}


The formatives for the indicative present and the imperfective are /ɡɑ, ɡo, ɡu/ <\armenian{գա, գօ, գու}>. These are placed before the verb or after it, and they are repeated for vowel-initial verbs (Table \ref{tab:Arapgir:morpho:verb:ind}).


 \begin{table}[H] \centering
 \caption{Indicative marking in the Arapgir dialect}
 \label{tab:Arapgir:morpho:verb:ind} 
 \begin{tabular}{|l|ll|ll| }
 \hline & \multicolumn{2}{l|}{Arapgir} & \multicolumn{2}{l|}{cf. SWA} \\ \hline
 `he sends' & ʁɾɡ-æ-$\emptyset$ ɡu & \armenian{ղրգա̈ գու} &ɡə ʁəɾɡ-e-$\emptyset$ & \armenian{կը ղրկէ}\\
 & \multicolumn{2}{l|}{send-{\thgloss}-3{\sg} {\ind}} & \multicolumn{2}{l|}{{\ind} send-{\thgloss}-3{\sg}}\\
 `he says' & ɡ-əs-æ-$\emptyset$ ɡu & \armenian{գըսա̈ գու} &ɡ-əs-e-$\emptyset$ & \armenian{կ՚ըսէ}\\
 & \multicolumn{2}{l|}{{\ind}-say-{\thgloss}-3{\sg} {\ind}} & \multicolumn{2}{l|}{{\ind}-say-{\thgloss}-3{\sg}}\\
 `he goes' & ɡ-etʰ-ɑ-$\emptyset$ ɡu & \armenian{գէթա գու} &ɡ-eɾtʰ-ɑ-$\emptyset$ & \armenian{կ՚երթայ}\\
 & \multicolumn{2}{l|}{{\ind}-go-{\thgloss}-3{\sg} {\ind}} & \multicolumn{2}{l|}{{\ind}-go-{\thgloss}-3{\sg}}\\
 \hline 
 & \multicolumn{2}{l|}{Gürün subdialect}& & \\
 `you strangle' & ɡo χəχd-e-s & \armenian{գօ խըխդէս} &ɡə χeχtʰ-e-s & \armenian{կը խեղդէս}\\
 & \multicolumn{2}{l|}{{\ind} send-{\thgloss}-2{\sg} } & \multicolumn{2}{l|}{{\ind} send-{\thgloss}-2{\sg}}\\
 `I go' & ɡo ɡ-eɾtʰ-ɑ-m & \armenian{գօ գէրթամ} &ɡ-eɾtʰ-ɑ-m & \armenian{կ՚երթամ}\\
 & \multicolumn{2}{l|}{{\ind} {\ind}-go-{\thgloss}-1{\sg} } & \multicolumn{2}{l|}{{\ind}-go-{\thgloss}-1{\sg}}\\
 `he eats' & ɡ-ud-e-$\emptyset$ ɡo & \armenian{գուդէ գօ} &ɡ-ud-e-$\emptyset$ & \armenian{կ՚ուտէ}\\
 & \multicolumn{2}{l|}{{\ind}-eat-{\thgloss}-3{\sg} {\ind}} & \multicolumn{2}{l|}{{\ind}-eat-{\thgloss}-3{\sg}}\\
\hline 
 & \multicolumn{2}{l|}{Kayseri subdialect}& & \\
 `I go' & ɡɑ ɡ-eɾtʰ-ɑ-m & \armenian{գա գէրթամ} &ɡ-eɾtʰ-ɑ-m & \armenian{կ՚երթամ}\\
 & \multicolumn{2}{l|}{{\ind} {\ind}-go-{\thgloss}-1{\sg} } & \multicolumn{2}{l|}{{\ind}-go-{\thgloss}-1{\sg}}\\
 `they eat' & ɡɑ ɡ-ud-e-n & \armenian{գա գուդէն} &ɡ-ud-e-n & \armenian{կ՚ուտեն}\\
 & \multicolumn{2}{l|}{{\ind} {\ind}-eat-{\thgloss}-3{\pl}} & \multicolumn{2}{l|}{{\ind}-go-{\thgloss}-3{\pl}}\\
 \hline 
 & \multicolumn{2}{l|}{Darende subdialect}& & \\
 `I bring' & beɾ-e-m ɡɑ & \armenian{բէրէմ գա} &ɡə pʰeɾ-e-m & \armenian{կը բերեմ}\\
 & \multicolumn{2}{l|}{bring-{\thgloss}-1{\sg} {\ind}} & \multicolumn{2}{l|}{{\ind} bring-{\thgloss}-1{\sg}}\\
 \hline \end{tabular}
\end{table} 

\subsection{Faithfulness to Classical Armenian}

In the Arapgir dialect, the phonetic changes and the grammatical formations are not new phenomena; instead we can say that the dialect is in general faithful to the Old Armenian, especially when we compare with the Cilicia dialect. 

\subsection{Genitive marking}
The only form that we can consider as more or less interesting is that the genitive of the infinitive in the Kayseri subdialect (Table \ref{tab:Arapgir:morpho:verb:gen}).


 \begin{table}[H] \centering
 \caption{Repeated genitive marking in the Kayseri subdialect of the Arapgir dialect}
 \label{tab:Arapgir:morpho:verb:gen} 
 \begin{tabular}{|l|ll|ll| }
 \hline & \multicolumn{2}{l|}{Kayseri (Arapgir)} & \multicolumn{2}{l|}{cf. SEA} \\ \hline
 `of staying' & mən-ɑ-l-uj-i & \armenian{մընալույի} &mən-ɑ-l-u & \armenian{մնալու}\\ 
 `of speaking' & χos-e-l-uj-i & \armenian{խօսէլույի} &χos-e-l-u & \armenian{խոսելու} \\ 
 `of giving' & d-ɑ-l-uj-i & \armenian{դալույի} &t-ɑ-l-u & \armenian{տալու} \\ 
 `of going' & eɾtʰ-ɑ-l-uj-i & \armenian{էրթալույի} &jeɾtʰ-ɑ-l-u & \armenian{երթալու} \\ 
 & \multicolumn{2}{l|}{$\sqrt{}$-{\thgloss}-{\infgloss}-{\gen}-{\gen}}& \multicolumn{2}{l|}{$\sqrt{}$-{\thgloss}-{\infgloss}-{\gen}}\\
 \hline 
 \end{tabular}
\end{table} 


As can be seen, these forms have a repeated genitive mark (/-u/ <\armenian{ու}> and /-i/ <\armenian{ի}>); this is something that is not found in any other dialect. \translatorHD{To clarify, in SEA/SWA, the suffix /-i/ is the regular genitive marker, while the suffix /-u/ is an irregular genitive marker used for some declension classes such as for verbal infinitives. }

\subsection{Progressive marking with /nə/ <\armenian{նը}>}
In Arapgir, the progressive is formed with the formative /nə/ <\armenian{նը}> (Table \ref{tab:Arapgir:morpho:verb:subj}).


 \begin{table}[H] \centering
 \caption{Subjunctive marking in the Arapgir dialect}
 \label{tab:Arapgir:morpho:verb:subj} 
 \begin{tabular}{|l|ll|ll| }
 \hline & \multicolumn{2}{l|}{Arapgir} & \multicolumn{2}{l|}{cf. SWA} \\ \hline
 `I am going' & ɡ-eɾtʰ-ɑ-m nə & \armenian{գէրթամ նը} &ɡ-eɾtʰ-ɑ-m ɡoɾ & \armenian{կ՚երթամ կոր}\\ 
& \multicolumn{2}{l|}{{\ind}-go-{\thgloss}-1{\sg} {\prog}} & \multicolumn{2}{l|}{{\ind}-go-{\thgloss}-1{\sg} {\prog}}\\
 `I am going' & χm-i-m nə & \armenian{խմիմ նը} &ɡə χəm-e-m ɡoɾ & \armenian{կը խմեմ կոր}\\ 
& \multicolumn{2}{l|}{drink-{\thgloss}-1{\sg} {\prog}} & \multicolumn{2}{l|}{{\ind}-drink-{\thgloss}-1{\sg} {\prog}} \\\hline \end{tabular}
\end{table} 

\begin{adjarianpage}\label{page:217}\end{adjarianpage}% should be 217

\section{Text samples}

{\sampleoverview}

\subsection{Arapgir dialect}

Adjarian's source: See \citeauthor{HandesAmsorya} 1900, page 251. In accordance with the author's exposition or arguments, I have rendered to the scientific orthography. 



\armenian{Նասրէդդին խօջան յ}̵\armenian{իր դան մէչը աղօթաձ վախդը ըսաց Ասձու. «Ըման Ասվաձ բաբա, ի՞շ գըլի գու, յ}̵\armenian{ընձի հարիր լիրա ղըրգա. շագ բէքք ունիմ փարայի։ Ըմմա թա̈մամ հարիր լիրա յ}̵\armenian{ըլլէլու ա}̈, \armenian{ա̈գուր մէգ հադ մը բագաս՝ դօխսանը ինը հադ յ}̵\armenian{ըլլին նա̈՝ չօթթէր. հա̈մ դու չէմ առնար»։}

\armenian{Խօջան յահուդի դՙրացին մը գունէնա} X\armenian{ու վօր ան դախքա̈յին դէնէրը նստէր ա̈ էղէր. խօջային ձՙա̈նը լսաձին գիբի գէթա յ}̵\armenian{ընգաջ դՙնա գու յու յ}̵\armenian{էքքէն յ}̵\armenian{ինք յ}̵\armenian{իրէնը գըստ գու}.

– \armenian{յ}̵\armenian{ա̈ջա̈փա դօխսանը ինը օսդի դէսնա նա̈ իրա̈վցընէ՞ չառնա դի} (< \armenian{պիտի) խօջան. էս լիրանէրը իսա բաջայէն վար ձՙըքիմ դա̈ դէսնամ ի՞շ դէնա գի իդա խօջան։}

\armenian{յ}̵\armenian{էքքէն դօխսանը ինը լիրա քէսա̈յի մը մէչը դՙնա գու յու գուգՙա գու բաջայէն փը՜րթ}… \armenian{վար նէդա գու. յ}̵\armenian{ինգ ալ հօնիգ բա̈քլամիշ գէնա̈ գու քի դէսնա դէ խօջան ի՞շ դէնա̈ դի։}

\armenian{Խօջան հըմըն օջախը վա̈զա̈ գու յու դէսնա գու քի գՙունդ բօխջա մը յ}̵\armenian{ընգէր ա̈ ֆօն. թէզ մը դառնա գու, դէշա̈գի մը վրան նսդի գու, քէսա̈ն բՙանա̈ գու վօր՝ ի՜շ դէսնա}… \armenian{օսգի՜։ յ}̵\armenian{էքքէն բաշլայտ գու մէգիգ մէգիգ հարմէլ. դէռնա գու քի դօխսանը ինի հադ էն օսգինէրը. մէյ մալ գու հարմէ, էրգուք գու հարմէ՝ բա̈լքի յաղըլմիշ էղա ըսէլօք. յ}̵\armenian{ամմէն հէղ հըսաբ էնէլուն նօրէն դօխսանը ինը հադ գՙըդնա գու. յ}̵\armenian{էքքէն գըսա̈ գու Ասձու «Դՙուն օր դօխսանը ինը լիրանէրը ղրգէցիր յ}̵\armenian{ընձի՝ հա̈լբա̈թ մէգալ մէդ հադ-նալ ղրգէս գու»։}


\subsection{Gürün subdialect}

Adjarian's source: See \citeauthor{Byurakn}, 1899, page, 410, 425, 478. 

\armenian{Օրը շադ է քանց գորէգը։}

\armenian{Ասձուձու բըյհաձը գէլը չուդէր։}

\armenian{Ուդօդղը չի գիդէ՝ փշօղը գիդէ։}

\armenian{Էրգըրի աշք է հանէր՝ դիրուն հօքին։}

\armenian{Ա̈շքը դէսածը դիրուն խէր չէներ։}


\begin{adjarianpage}\label{page:218}\end{adjarianpage}% should be 218


\armenian{Փըջաձ օջախը ջուր չուզեր։}

\armenian{Խրադ խրադ լէռն ի վէր, յէս գօ գէրթամ ձօրն ի վէր։}

\armenian{Էրթ դաս, գարի չուզէր։}

\armenian{Մէգը մէգուն էքի մը դվեր է, էնի էնօր գիզ (ողկոյզ) մը հաղօղ չէ դըվեր։}

\armenian{Մայը փշէլօվ (մանր փշրելով) փօր չի գըշդանար։}

\armenian{Ուզօղուն մէգ էրեսը սեվ, չի դըվօղուն էրգու էրեսը։}

\armenian{Գէդէն անցար դէ գաւառը գօ խըղդիս։}

\armenian{Գօվիս մէռէր է, խաբիս գըյէր է։}

\armenian{Դուռիդ գօց բռնէ՝ դրացիդ ղօրթ բռնէ։}

\armenian{Բագը ձուռ է, գօվը գաթ դօ չի դար։}

\armenian{Սուդ խէնթ էղէր է, վանքին հավէրը գուդէ գօ։}

\armenian{Ջաղարջը ջուրն է դարի՝ չախչար գօ բռես։}

\armenian{Գադու չէղած մուգ գօ բռնէս։}

\armenian{Թէմբէլուէնէն իշուն «քէռի» գօ գըսէ։}

\armenian{Սօխ գէրաձ չիմ ռր սիրդիս էրի։}

\armenian{Դանձ էիր նը հասար, խնձօր էիր նը գէյմըրէցար։}

\armenian{Ձայը (ծանր) նադիր օր լիր գաս։}

\armenian{Թէք ինգիզին (ընկոյզին) քար չի նէդէր։}

\armenian{Մէյմէգի միս գուդէն գօ։}

\armenian{Թըքալին բէրնօվը գը գէրցնէ, գօթօվը էշքը գը հանէ։}

\armenian{Սադգած էշ կը բռ՚ օր նալը քէշէ։}

\armenian{Սօխին քարսը (քաղցրը) չիլլար։}

\armenian{Մէյմէգի իչին (համար) ղուշան (գնդակ) ղսմին գօ (կը սեղմեն)։}


\begin{adjarianpage}\label{page:219}\end{adjarianpage}% should be 219

\subsection{Darende subdialect}

Adjarian's source: See \citeauthor{Byurakn} 1899, page 498. 


\armenian{Չի հավնաձ մարդդ շօրէր զուրցէ գա օր փօրդքօռօգը նէդէ գա։}

\armenian{Ասվաձ լէռը գը հայի՝ ձունն անօր գէօրէ գուդա։}

\armenian{Վար թուքէս մօրուք, վէր թուքէս բըյըգ։}

\armenian{Հէյ սիրդդ սիրէմ Ասվաձ, գօմէշը գօդօշն ի՞նչ ընէ։}

\armenian{Բանին մէչ բան գա, մաձունին մէչ թան գա։}

\armenian{Հարսնիք գէրթամ, գաթա գուտէմ թէ դան նէ։}


\armenian{Չէրթային ադ ջաղարչը՝ չուդէիր ադ բաղարչը։}

\armenian{Գօյաձ (կոտրած) ընգուզը հազարն անցավ։}

\armenian{Ջուրը տէսնա ձուգ գըլլա, գադուն դէսնա մուգ գըլլա։}


 
\begin{center}
 * *
\end{center}

\armenian{Գնդուկ գնդուկ մաղարա, մէնք հարս մ՚ունինք գը խաղա, ջուրը ցըքէքնք կը լօղա, ջուրէն հանէնք գը դօղա, ձէռքը չամիչ մը դանք նէ, մինչէվ իրիգուն գը խաղա։}

\armenian{Ադվընի, իսգի՞ց գուքաս}. – \armenian{Արիւնօդ ձօվէն}. – \armenian{Վըրադ օրի՞ արիւն չէ էղէր}. – \armenian{Ասդուձու հրամանօվ։}

\armenian{Արաբը փօսն ընգավ, գըլօխը դուրսն ընգավ։}

\armenian{Քէօքը Ֆէօղին մէչ, միչուգն էշին մէչ, գըլօխը փօրուդ մէչ,։}

\armenian{Աննէրկ գէրմուր՝ անդանագ քէրթուք (աքլորի կատար)։}

\subsection{Kayseri subdialect}

\subsubsection{Munjusun village}

Adjarian's source: See \citeauthor{Byurakn} 1898, page 407. 

\armenian{Մարթ մը մէգ հադիդ աչգին մը ունէ ղըլէմ, օր բաշխա գէղի գիզիր դղայի հէդ գը գարքէ։ Օր մը էս մարթը գէլլէ իր աչգանը դունը էրթալու, փէսին հէդ, աչգինին հէդ դէսնըվէլույի։ Էրթալ իքէն ջամփան, ի՞նչ գըսէս, ղայայի մը արալըխը՝ ասլանի մը ձաքէրը՝ առչէվմնին իրէք դարվան բըլուզ դղա մը առէլ գա գուդէն։ Էս մարթը հէմէն գը վազէ, շադ մը էզիյէթնէրօք գը խըլասէ։ Ինքը մէգ աչգինէն բաշխա զավագ չուննալույին համար}. – \armenian{«Իմ սօն օրիս Ասդվաձ ինձի դղա մը դվավ» դէյի շադ գը խընդա։ Էնգիւց ղօմղօրթ էդ խըլասաձ դղան գուջախը առաձ, աչգանը գունը գէրթա։}

\armenian{Աչգինը խընդալէն, «հէ՛ր, էս ավը ո՞ւրդէղէն ավլամիշ էրիր» գըսէ։}

\subsubsection{Palahesi village}

Adjarian's source: See \citeauthor{Byurakn} 1898, page 580.


\armenian{Աղէնօք Ուղուզէլի անուն գէղի մը հարուսդին մէգը հէչ չօջուխ չուննար էղէր, մէգ օր դիւշիւնմիւշ գէնէ օր, աջաբա իմ}... 

\begin{adjarianpage}\label{page:220}\end{adjarianpage}% should be 220

... \armenian{դունին ջօմաաթը ըզմէն էզիյէթ քէշէլմընըյին համա՞ր մը Ասձվաձ ընձը չօջուխ գա չի դա։}

\armenian{Էս դիւշիւնգէյօք ինգէր աղային հէդը՝ թաբդիլի ղըյաֆէթ, բէլլէլու գէլլէն։ Բէլլէն իքէն դավրիշի մը րասդ դուքան։ Հարուսդը բադիվ գուդա դարվիշին. դարվիշն է – Ասձուձու բարին ինեն, ա՛ղա գըսէ։}

\armenian{Հարուսդը դարվիշին՝}

– \armenian{Դուն իմ իշխան ըլլալիս՝ էնօր է տղա ըլլալը ինչէ՞ն գա գիդէնաս, ըսաձը գիբի, դարվիշն է՝}

– \armenian{Սէ գի գիդէմ օր, գըսէ, ինչ օր էս դարը չօջուխ չուննալուդ համար, աջաբա դունիս ջօմաաթը ըզմէն էզիյէ՞թ մը գա քէշէ դէ՛ի թաբդիլի ղըյաֆէթ էղէր էք. ամմա քու դունիդ մէչը գդնվօղ փիթին ջօմաաթըդ քէզմէն խօշնուդ է. էգէր քիչուգ մը առաչ էրթաք նա, ջուրը մը րասդ բըդը գուքաք, ջուրը խնձօր մը բըդը պէրէ, էդ խնձօրը առ ՝ էրէսը ձիուդ գէրցուր, մէչն է գնըգիդ։}

\subsubsection{Everek}

Adjarian's source: See \armenian{Բանասէր}, 1902, page 174. \translatorHD{Note that there are multiple publications called \armenian{Բանասէր} /bɑnɑseɾ/ `Philologist'. As of writing, I haven't tracked it down.}




\armenian{Ժամէրնիս փըլցըվաձ},

\armenian{Սըրդէրնիս խըռօվաձ},

\armenian{Ամբարը չըքա հաց}…

\armenian{Օրդնյալ էս, Դէ՜ր Ասվաձ։}

~

\armenian{Դուրավիքը ձիւն ձըմէր},

\armenian{Փագվէր է ջամփա լէր},

\armenian{Նէ էգօղ, նէ էրթօղ},

\armenian{Չօրս դընիս վախ ու դօղ։}

~

\armenian{Ահանց դէ՜ չօրս ամիս}

\armenian{Չը դըրինք բէրաննիս}

\armenian{Բադառ մը միս, բուլղուր},

\armenian{Համփիրթին մէզի դուր։}

~

\armenian{Ձիձ ձըձօղ մէսիւմնին}

\armenian{Ուդէլու գաթ չունին},


\begin{adjarianpage}\label{page:221}\end{adjarianpage}% should be 221



\armenian{Գա գուլան ինչու՚ րգուն}…

\armenian{Էրէսնին հայիս դուն։}

~

\armenian{Էրգըթէ սիրդ բէդգ է՜},

\armenian{Ձէռքիդ է (ալ) չի ադգէ}…

\armenian{Ձօցվօրը դուռնէ դուռ՝}

\armenian{Գա մուրա հաց աբուր։}

~

\armenian{Էդ ադէն գօռէլէն}

\armenian{Դէվ գիբի գօռալէն}

\armenian{Ալլաղգնէր գա թափնէն}…

\armenian{«Քըշդէցէք գյավուրին»։}

~

\armenian{Գէղէն դուրս, քիշէրը},

\armenian{Անքրիստօս թուրքէրը}

\armenian{Հէրիւր մարթ քշդէր էն}…\armenian{։}

\armenian{Վա՜յ անխիղջ անօրէ՜ն։}




\chapter{Akn}
\section{Background and literature}

\begin{adjarianpage}\label{page:222}\end{adjarianpage}% should be 222

The dialect of Akn is spoken only in the city of Akn and in a few of its surrounding Armenian villages. Texts are written with this dialect are found in the rich ethnographic collection of \citet{Janigian-1895-Akn} and \citeauthor{Gabrielian-1912-Akn}'s extensive study that was not written with a scientific method (\citealt{Gabrielian-1912-Akn}; \citeauthor{HandesAmsorya}, 1908-1911, and continuous). Other succinct manuscripts are in \citeauthor{Byurakn} (1898, page 101, 330, 360, 393, 429, 557, 565, 601, 827, 895; 1900, page 388, 695). There are also succinct dialogues in the Akn dialect in \armenian{Տարեցոյց նշանՊէրպէրեանի} (1897, page 67-62; 1898, page 23-24, 147; 1899, page 54-71; 1900, page 254-266; 1903, page 145-168), and the comedic writings of \armenian{«Երանոս Աղբար կամ Թապլաքեար վարժապետը»} and \armenian{«Թապլաքեար Փիլիկ աղպօր աղջիկտեսը»}. 

\section{Phonology}

\subsection{Segment inventory}
 The Akn dialect has 8 vowels: /ɑ, e, ə, i, o, œ, u/ <\armenian{ա, է, ը, ի, օ, էօ, ու}>, and three series of consonants, like the Arapgir dialect. 

 \subsection{Sound changes}
 
 For its sound changes, the characteristic situations are the following. 


\subsubsection{Monophthong vowel changes }
 \paragraph{Classical Armenian /ɑ/ <\armenian{ա}> }
 


The Old Armenian sound /ɑ/ <\armenian{ա}> becomes /o/ <\armenian{օ}> when immediately before a nasal, such as also in the Hamshen dialect (Table \ref{tab:Akn:phonology:soundChange:monoph:a:o}). 

\begin{table}[H]
 \centering
 \caption{Change from Classical Armenian /ɑ/ <\armenian{ա}> to /o/ <\armenian{օ}> in the Akn dialect}
 \label{tab:Akn:phonology:soundChange:monoph:a:o}
 \begin{tabular}{|l| ll|ll| ll|}
 \hline & \multicolumn{2}{l|}{Classical Armenian} &\multicolumn{2}{l|}{> Akn } & \multicolumn{2}{l|}{cf. SEA } \\ 
`fly (bug)' & t͡ʃɑnt͡ʃ& \armenian{ճանճ} & d͡ʒond͡ʒ & \armenian{ջօնջ} & t͡ʃɑnt͡ʃ & \armenian{ճանճ} \\ 
`unsalted' & ɑnɑli & \armenian{անալի}& olli & \armenian{օլլի} & ɑnɑli & \armenian{անալի} \\ 
`rain' & ɑnd͡zɾeu̯ & \armenian{անձրեւ} & oɾzev & \armenian{օրզէվ} & ɑnd͡zɾev & \armenian{անձրեւ} \\ 

 \hline 
 \end{tabular}
\end{table}


 \paragraph{Classical Armenian /u/ <\armenian{ու}> }
 



The Old Armenian sound /u/ <\armenian{ու}> becomes /ʏ/ <\armenian{իւ}> (Table \ref{tab:Akn:phonology:soundChange:monoph:u}). 

\begin{table}[H]
 \centering
 \caption{Change from Classical Armenian /u/ <\armenian{ու}> to /ʏ/ <\armenian{իւ}> in the Akn dialect}
 \label{tab:Akn:phonology:soundChange:monoph:u}
 \begin{tabular}{|l| ll|ll| ll|}
 \hline & \multicolumn{2}{l|}{Classical Armenian} &\multicolumn{2}{l|}{> Akn } & \multicolumn{2}{l|}{cf. SEA } \\ 
`you.{\sg} have' & unis& \armenian{ունիս} & ʏnis & \armenian{իւնիս} & unes & \armenian{ունես} \\ 
`eight' & utʰ& \armenian{ութ} & ʏtʰ & \armenian{իւթ} & utʰ & \armenian{ութ} \\ 

 \hline 
 \end{tabular}
\end{table}


\paragraph{Classical Armenian /o/ <\armenian{ո}> and /ɑu̯/ <\armenian{օ, աւ}> }
 




The Old Armenian sounds /o, ɑu̯/ <\armenian{ու, օ}> become /œ/ <\armenian{էօ}> (Table \ref{tab:Akn:phonology:soundChange:monoph:o}). 

\begin{table}[H]
 \centering
 \caption{Change from Classical Armenian /o, ɑu̯/ <\armenian{ու, օ}> to /œ/ <\armenian{էօ}> in the Akn dialect}
 \label{tab:Akn:phonology:soundChange:monoph:o}
 \begin{tabular}{|l| ll|ll| ll|}
 \hline & \multicolumn{2}{l|}{Classical Armenian} &\multicolumn{2}{l|}{> Akn } & \multicolumn{2}{l|}{cf. SEA } \\ 
 	 \armenian{՝}four' & t͡ʃʰoɾs & \armenian{չորս}& t͡ʃʰœɾs & \armenian{չէօրս} & t͡ʃʰoɾs & \armenian{չորս} \\ 
 	`door' & durən & \armenian{դուռն} & dʰœr & \armenian{դՙէօռ} & dur & \armenian{դուռ} \\ 
 `today'& ɑi̯sɑu̯ɾ & \armenian{այսաւր} & ɑsœɾ & \armenian{ասէօր} & ɑjsoɾ & \armenian{այսօր} \\

 \hline 
 \end{tabular}
\end{table}


\subsubsection{Glide insertion before post-vocalic /h/ <\armenian{հ}> }

The only unique property of the Akn dialect is that if a vowel is immediately before the CA sound /h/ <\armenian{հ}>, then the semivowel /j/ <\armenian{յ}> is added next to the vowel (Table \ref{tab:Akn:phonology:soundChange:h}). 

\begin{table}[H]
 \centering
 \caption{Glide insertion before post-vocalic /h/ <\armenian{հ}> in the Akn dialect}
 \label{tab:Akn:phonology:soundChange:h}
 \begin{tabular}{|l| ll|ll| ll|}
 \hline & \multicolumn{2}{l|}{Classical Armenian} &\multicolumn{2}{l|}{> Akn } & \multicolumn{2}{l|}{cf. SEA } \\ 
 \armenian{՝}death' & mɑh & \armenian{մահ} & mɑjh & \armenian{մայհ} & mɑh & \armenian{մահ} \\
 \armenian{՝}satisfied' & ɡoh & \armenian{գոհ} & ɡʰojh & \armenian{գՙօյհ} & ɡoh & \armenian{գոհ} \\
 \armenian{՝}gain' & ʃɑh & \armenian{շահ} & ʃɑjh & \armenian{շայհ} & ʃɑh & \armenian{շահ} \\
 `fear' & ɑh& \armenian{ահ} & ɑjh & \armenian{այհ} & ɑh & \armenian{ահ} \\

 \hline 
 \end{tabular}
\end{table}


 
This characteristic is also unavoidable among educated Akn speakers. 

\subsubsection{Diphthongal vowel changes }

For CA diphthong changes, the notables ones are the following.



\paragraph{Classical Armenian /ɑi̯/ <\armenian{այ}> }

The Classical sound /ɑi̯/ <\armenian{այ}> changed to /ɑ/ <\armenian{ա}> (Table \ref{tab:Akn:phonology:soundChange:ai}). 

\begin{table}[H]
 \centering
 \caption{Change from Classical Armenian /ɑi̯/ <\armenian{այ}> to /ɑ/ <\armenian{ա}> in the Akn dialect}
 \label{tab:Akn:phonology:soundChange:ai}
 \begin{tabular}{|l| ll|ll| ll|}
 \hline & \multicolumn{2}{l|}{Classical Armenian} &\multicolumn{2}{l|}{> Akn } & \multicolumn{2}{l|}{cf. SEA } \\ 
`mother' & mɑi̯ɾ & \armenian{մայր} & mɑɾ & \armenian{մար} & mɑjɾ & \armenian{մայր} \\ 
`father' & hɑi̯ɾ & \armenian{հայր} & hɑɾ & \armenian{հար} & hɑjɾ & \armenian{հայր} \\ 
 \hline 
 \end{tabular}
\end{table}

\paragraph{Classical Armenian /oi̯/ <\armenian{ոյ}> }

The Classical sound /oi̯/ <\armenian{ոյ}> changed to /u/ <\armenian{ու}> (Table \ref{tab:Akn:phonology:soundChange:oi̯}). 

\begin{table}[H]
 \centering
 \caption{Change from Classical Armenian /oi̯/ <\armenian{ոյ}> to /u/ <\armenian{ու}> in the Akn dialect}
 \label{tab:Akn:phonology:soundChange:oi̯}
 \begin{tabular}{|l| ll|ll| ll|}
 \hline & \multicolumn{2}{l|}{Classical Armenian} &\multicolumn{2}{l|}{> Akn } & \multicolumn{2}{l|}{cf. SEA } \\ 
 \armenian{՝}light' & loi̯s & \armenian{լոյս}& lus & \armenian{լուս} & lujs & \armenian{լույս} \\
 \hline 
 \end{tabular}
\end{table}



\paragraph{Classical Armenian /iu̯/ <\armenian{իւ}> }

The Classical sound /iu̯/ <\armenian{իւ}> changed to /u/ <\armenian{ու}> (Table \ref{tab:Akn:phonology:soundChange:iu̯}). 

\begin{table}[H]
 \centering
 \caption{Change from Classical Armenian /iu̯/ <\armenian{իւ}> to /u/ <\armenian{ու}> in the Akn dialect}
 \label{tab:Akn:phonology:soundChange:iu̯}
 \begin{tabular}{|l| ll|ll| ll|}
 \hline & \multicolumn{2}{l|}{Classical Armenian} &\multicolumn{2}{l|}{> Akn } & \multicolumn{2}{l|}{cf. SEA } \\ 
 \armenian{՝}blood' & ɑɾiu̯n & \armenian{արիւն}& ɑɾun & \armenian{արուն} & ɑɾjun & \armenian{արյուն} \\
 \armenian{՝}snow' & d͡ziu̯n & \armenian{ձիւն}& d͡zʰun & \armenian{ձՙուն} & d͡zjun & \armenian{ձյուն} \\
 \hline 
 \end{tabular}
\end{table}



\begin{adjarianpage}\label{page:223}\end{adjarianpage}% should be 223

\subsubsection{Consonant changes}

The consonant changes are exactly same as in Kharberd, Arapgir, and Sebastia



\section{Morphology}
The grammar of the Akn dialect does not present individual characteristic properties. For whatever differences that are present, these originated from the effect of general phonological rules. 

\subsection{Noun inflection or declension}
\subsubsection{Genitive marking}
For example, the genitive formative is /-ʏ/ <\armenian{իւ}> (Table \ref{tab:Akn:morpho:gen:y}). 

\begin{table}[H]
 \centering
 \caption{Genitive marking in the Akn dialect}
 \label{tab:Akn:morpho:gen:y}
 \begin{tabular}{|l| ll| ll|}
 \hline & \multicolumn{2}{l|}{Akn } & \multicolumn{2}{l|}{cf. SWA } \\ 
 \armenian{՝}God-{\gen}' & ɑsdʏd͡z-ʏ & \armenian{Ասդիւձիւ} & ɑstud͡z-o & \armenian{Աստուծո} \\
 \armenian{՝}soul-{\gen}-{\defgloss}' & hokʰ-ʏ-n & \armenian{հօքիւն} & hokʰij-i-n & \armenian{հոգուն} \\
 \armenian{՝}dead-{\pl}-{\gen}-{\defgloss} ' & meɾel-neɾ-ʏ-n & \armenian{մէռէլնէրիւն} & meɾel-neɾ-u-n & \armenian{մեռելներուն} \\
 (= of the dead)& & & & \\
 \hline 
 \end{tabular}
\end{table}


\subsection{Verb inflection or conjugation}
\subsubsection{Indicative marking with /ɡʏ/ <\armenian{գիւ}> }

Similarly, the indicative present and imperfective use the formative /ɡʏ/ (\armenian{գիւ}) (cf. SWA /ɡu/ <\armenian{կու}>) (Table \ref{tab:Akn:morpho:indc:g}). 

\begin{table}[H]
 \centering
 \caption{Indicative marking in the Akn dialect}
 \label{tab:Akn:morpho:indc:g}
 \begin{tabular}{|l| ll| ll|}
 \hline & \multicolumn{2}{l|}{Akn } & \multicolumn{2}{l|}{cf. SWA } \\ 
 \armenian{՝}I give' & ɡʏ d-ɑ-m & \armenian{գիւ դամ} & ɡu d-ɑ-m & \armenian{կու տամ} \\
 \armenian{՝}I cry' & ɡʏ l-ɑ-m & \armenian{գիւ լամ} & ɡu l-ɑ-m & \armenian{կու լամ} \\
 & \multicolumn{2}{l|}{{\ind} $\sqrt{}$-{\thgloss}-1{\sg}}& \multicolumn{2}{l|}{{\ind} $\sqrt{}$-{\thgloss}-1{\sg}}\\
 \hline 
 \end{tabular}
\end{table}


\subsubsection{Theme vowel changes and the indicative present}
 
 In the verbal endings, the vowel \armenian{ե} becomes \armenian{ի} when next to nasals, while it stays unchanged in other places. 
 
 \translatorHD{To clarify, he's talking about theme vowels in verbs, before agreement suffixes. He provides examples from the indicative present. In SWA, the indicative present is formed by adding the indicative prefix /ɡ(ə)-/ before the finite verb. The theme vowel stays constant in the indicative present. Akn behaves differently with respect to theme vowel uses. }


\begin{table}[H]
 \centering
 \caption{Indicative present <\armenian{ներկայ}> of the verb `to send' in the Akn dialect}
\label{tab:Akn:morpho:verb:paradigm:presentPastIndc}
 \begin{tabular}{|l|ll|ll|}
 \hline & \multicolumn{2}{l|}{Akn} & \multicolumn{2}{l|}{cf. SWA} \\ \hline 
1SG & ɡʏ χəɾɡ-i-m & \armenian{գիւ խըրգիմ} & ɡə χəɾɡ-e-m & \armenian{կը խրկեմ} \\
2SG & ɡʏ χəɾɡ-e-s & \armenian{գիւ խըրգէս} & ɡə χəɾɡ-e-s & \armenian{կը խրկես} \\
3SG & ɡʏ χəɾɡ-e-$\emptyset$ & \armenian{գիւ խըրգէ} & χəɾɡ-e-$\emptyset$ & \armenian{կը խրկէ} \\
1PL & ɡʏ χəɾɡ-i-nkʰ & \armenian{գիւ խըրգինք} & ɡə χəɾɡ-e-ŋkʰ & \armenian{կը խրկենք} \\
2PL & ɡʏ χəɾɡ-e-kʰ & \armenian{գիւ խըրգէք} & ɡə χəɾɡ-e-kʰ & \armenian{կը խրկէք} \\
3PL & ɡʏ χəɾɡ-e-n & \armenian{գիւ խըրգին} & ɡə χəɾɡ-e-n & \armenian{կը խրկեն} \\
& \multicolumn{2}{l|}{{\ind}} $\sqrt{}$-{\thgloss}-{\agr} & \multicolumn{2}{l|}{{\ind} $\sqrt{}$-{\thgloss}-{\agr}}
 \\ \hline 
\end{tabular}
\end{table}

\subsubsection{Archaism in past 1PL suffix /-ɑ-nkʰ/ <\armenian{անք}>}\label{sec:Akn:morphology:verb:archaicPstPl}

\translatorHD{In SWA and SEA, the 1PL suffix is [-ŋkʰ]. This same formative is used for the present, past imperfective, and past perfective. In the past, this plural suffix follows the past suffix /-i/ or /-ɑ/, thus creating the sequence [-i-ŋkʰ] or [-ɑ-ŋkʰ]. In contrast, Classical Armenian used the suffix /-mkʰ/ for the present 1PL, and /-ɑkʰ/ for the past 1PL; the /-ɑ/ in this form could be separately segmented as a past suffix. Adjarian reports that Akn aligns with Classical Armenian. }

Like the Sebastia dialect, the ending of the imperfective and perfective 1PL is /-ɑ-nkʰ/ <\armenian{անք}> (here, the sound change of CA /ɑn/ <\armenian{ան}> to /on/ <\armenian{օն}> does not happen), or which is more similar to the Classical Armenian ending /-ɑkʰ/ <\armenian{աք}>, than with the /-i-nkʰ/ <\armenian{ինք}> form of other dialects (Table \ref{tab:Akn:morpho:verb:pl}). 

\begin{table}[H]
 \centering
 \caption{Archaisms in the 1PL suffix in the Akn dialect}
 \label{tab:Akn:morpho:verb:pl}
 \begin{tabular}{|l| ll|ll| ll|}
 \hline & \multicolumn{2}{l|}{Classical Armenian} &\multicolumn{2}{l|}{> Akn } & \multicolumn{2}{l|}{cf. SWA } \\ 
 \armenian{՝}we were eating' (past impf.) & ud-ē-ɑ-kʰ & \armenian{ուտէաք}& ɡ-ʏd-e-ɑ-nkʰ & \armenian{գիւդէանք} & ɡ-ud-ej-i-ŋkʰ & \armenian{կ՚ուտէինք} \\
 & \multicolumn{2}{l|}{eat-{\thgloss}-{\pst}-1{\pl}}& \multicolumn{2}{l|}{{\ind}-eat-{\thgloss}-{\pst}-1{\pl}} & \multicolumn{2}{l|}{{\ind}-eat-{\thgloss}-{\pst}-1{\pl}}
 \\
 \armenian{՝}we brought' (past perf.) & beɾ-ɑ-kʰ & \armenian{բերաք}& bʰeɾ-ɑ-nkʰ & \armenian{բՙէրանք} & pʰeɾ-i-ŋkʰ & \armenian{բերինք} \\
 & \multicolumn{2}{l|}{bring-{\pst}-1{\pl}}& \multicolumn{2}{l|}{bring-{\pst}-1{\pl}} & \multicolumn{2}{l|}{bring-{\pst}-1{\pl}}\\
 \hline 
 \end{tabular}
\end{table}

\subsubsection{Future marking with /di, d/ <\armenian{դի, դ}>}

\translatorHD{In SWA, the future is formed by combining the future proclitic /bidi/ with the finite verb. If this proclitic is added to the present form of the verb, then the meaning is the simple future (Table \ref{tab:Akn:morpho:verb:paradigm:fut}). If the proclitic is added to the past imperfective form (which includes a past suffix /-i, -$\emptyset$/), then the meaning is the future perfect (Table \ref{tab:Akn:morpho:verb:paradigm:futPerf}). Akn behaves similarly with different formatives. }

The future formative is /di/ <\armenian{դի}>, which is shortened to /d/ <\armenian{դ}> when next to a vowel; this is a shortened form of the CA formative /piti/ <\armenian{պիտի}> `must'). 


\begin{table}[H]
 	\centering 
 	\caption{Future <\armenian{ապառնի}> of the verb `to bring' in the Akn dialect}
 	\label{tab:Akn:morpho:verb:paradigm:fut}
 \begin{tabular}{|l|ll| ll| }
 		\hline & \multicolumn{2}{l|}{Akn} & \multicolumn{2}{l|}{cf. SWA} \\ \hline 
1SG &di bʰeɾ-i-m & \armenian{դի բՙէրիմ} & bidi pʰeɾ-e-m& \armenian{պիտի բերեմ} \\
2SG &di bʰeɾ-e-s & \armenian{դի բՙէրէս}& bidi pʰeɾ-e-s& \armenian{պիտի բերես} \\
3SG &di bʰeɾ-e-$\emptyset$ & \armenian{դի բՙէրէ} & bidi pʰeɾ-e-$\emptyset$& \armenian{պիտի բերէ} \\
1PL &di bʰeɾ-i-nkʰ & \armenian{դի բՙէրինք}& bidi pʰeɾ-e-ŋkʰ& \armenian{պիտի բերենք} \\
2PL &di bʰeɾ-e-kʰ & \armenian{դի բՙէրէք} & bidi pʰeɾ-e-kʰ& \armenian{պիտի բերէք} \\
3PL &di bʰeɾ-i-n & \armenian{դի բՙէրին} & bidi pʰeɾ-e-n& \armenian{պիտի բերեն} \\
& \multicolumn{2}{l|}{{\fut} $\sqrt{}$-{\thgloss}-{\agr}}& \multicolumn{2}{l|}{{\fut} $\sqrt{}$-{\thgloss}-{\agr}}\\
\hline 
\end{tabular}
\end{table}




\begin{table}[H]
 	\centering 
 	\caption{Future perfect <\armenian{անցեալ ապառնի}> of the verb `to eat' in the Akn dialect}
 	\label{tab:Akn:morpho:verb:paradigm:futPerf}
 \begin{tabular}{|l|ll| ll| }
 		\hline & \multicolumn{2}{l|}{Akn } & \multicolumn{2}{l|}{cf. SWA} \\ \hline 
1SG &d-ʏd-e-i-$\emptyset$ & \armenian{դիւդէի} & bidi ud-ej-i-$\emptyset$& \armenian{պիտի ուտէի} \\
2SG &d-ʏd-e-i-ɾ & \armenian{դիւդէիր} &bidi ud-ej-i-ɾ & \armenian{պիտի ուտէիր} \\
3SG &d-ʏd-e-$\emptyset$-ɾ & \armenian{դիւդէր} & bidi ud-e-$\emptyset$-ɾ& \armenian{պիտի ուտէր} \\
1PL &d-ʏd-e-ɑ-nkʰ & \armenian{դիւդէանք} & bidi ud-ej-i-ŋkʰ& \armenian{պիտի ուտէինք} \\
2PL &d-ʏd-e-i-kʰ & \armenian{դիւդէիք} & bidi ud-ej-i-kʰ& \armenian{պիտի ուտէիք} \\
3PL &d-ʏd-e-i-n & \armenian{դիւդէին} & bidi ud-ej-i-n& \armenian{պիտի ուտէին} \\
& \multicolumn{2}{l|}{{\fut}-$\sqrt{}$-{\thgloss}-{\pst}-{\agr} }& \multicolumn{2}{l|}{{\fut}-$\sqrt{}$-{\thgloss}-{\pst}-{\agr}}\\
\hline 
\end{tabular}
\end{table}



\begin{adjarianpage}\label{page:224}\end{adjarianpage}% should be 224

\section{Text samples}

{\sampleoverview}


Adjarian's source: See \citeauthor{Janigian-1895-Akn}, page 292.


– \armenian{Նըսդէ նայիմ, յէգէն Թօրօս, ինդէ՞օր էս։}

– \armenian{Ձառա իմ ա՛ղա, Ասվաձ գէնք գա։}

– \armenian{Ի՞շ գա իշ չի գա, յէ՛գէն։}

– \armenian{Բադվագան գէնթանությանդ դուվաջի ինք։ Հրամանքէդ իրիջայէ մի իմ էգէր. ըմմա չի գՙիդիմ քի խաբուլ գանէ՞ս։}

– \armenian{Ըսէ՛ նայիմ, խաբիլլիւ գՙօրձ է իսէ, փէ՛ք աղէգ։}

– \armenian{Վէր Ասդվաձ, վար հրամանքըդ. գՙլէօխս նէղի գՙա նը ո՞ւր դէրթամ. հէլբէթ հէօս դի գՙամ։ Թաջիզութիւն չի դանք, ա՛ղա. խընթիրքս աս է թը՝ առիւդիւր մի գա՝ դի անիմ. հազար ղրուշ բագաս է. քէրէմ արէ դի՛ւր, ֆայիզօվը գիւդամ։}

– \armenian{Թահվիլ մի գՙրէ, Գա՛րաբէդ, առ աս բՙալլին, գՙնա նէրսի դօլաբէս հազար ղրուշ բՙէր, յէգէնին դիւր, թահվիլը առ։}

– \armenian{Շինօյհրագալ իմ աղա, Ասդվաձ իշախանության բահէ։}

– \armenian{Չարսի՞ւն իշ գա իշ չի գա. առիւդիւրնէրը ինդէօ՞ր է։}

– \armenian{Ասդվաձ բէրէքէթ դա, ըմմա առիւդիւրնէրը քէսադ է. ցօրէն, բանիր շա՛դ գիւգՙա, լաքին թիւքէնչինէրը էռէչքը գէրթան, դիւ բՙէրին։ Յասախ գանին քի էռէչքը մարթ չի դէրթա. մդիգ չին անէր գինէ գէրթան, էսնէֆը նէղը գիւ մնա, անիւնց քէր մի գիւղա գառնէ։ Բՙէրօղը գուզէ իւր (որ) դընվօրին դա, ըմմա մդիգ անօ՞ղն օվ է. ջՙօրին բՙէռնօվ ձՙէռքէն գիւ քաշին գառնին. քիչ մալ դՙիմանա նը՝ գիւ ձէձին։ Հէդդա չէ ըմա, չի՞լլիր իւր ժամը ձանիւցիւմ անին քի անիւնցմէ բՙան մ՚ա՛ռէք. Հայն իւր անիւնցմէ չառնէ նը առաձնին չին գրնար ձախէր, ալ չին առնէր։}

– \armenian{Ադ ըսաձդ էռէչ էր. ան վախթը խասթին մէգը գամ իւրիշի գՙիշիւթիւն մի անէր նը՝ ժամը գանիձէին թը ան մարթէն միս գամ իւրիշ բՙան մ՚ա՛ռնէք, չէյին առնէր. մինչիւգ իւր գՙար մէղա ըսէր, նէօրէն ձանիւցիւմ անէյին քի, առէ՛ք։ Հըմա էռչի միյափանութիւնը չիգա. ադէնգ բՙանէր ձանիւցիւմ չանվիր. անին ալ նը՝ վօրը մդիգ գանէ, վօրը չանէր։ Դՙիւն քիւ գՙօրձդ դէսար նը ա՛դ նայէ։}


\chapter{Sebastia }
\section{Background}

\begin{adjarianpage}\label{page:225}\end{adjarianpage}% should be 225


This dialect is specific to the Armenian-heavy city of Sebastia and its many surrounding Armenian villages, which occupy the valley of Alis, starting from Sebastia and eastern until Zara. The southern border is Ulaş. Starting from this village until Mandjilik, and north-west from that, the Tonus province, until Gemerek, we find the subdialect of Gürün subdialect. The villages of the Alis valley differ a little from the dialect of the city. The Pirknik village forms its own subdialect; the village is found one hour away from the city towards the northeast. 

\section{Literature}




The dialect of Sebastia has still not been studied. It is also surprising that there are no published manuscripts. For the first time in Paris, I had the opportunity to study the pronunciation of plosives using recording devices of Abbé Rousselot (Jean-Pierre Rousselot, \armenian{Ռուսլօ աբբա}). The result was published in my ``Les explosives'' work \citep{Adjarian-1899-ArmenianExplosives}. For this dialect, Mr. Karapet Gabikian (\armenian{պր. Կ. Գաբիկեան}; \translatorHD{SEA: /kɑɾɑpet ɡɑbikjɑn/, SWA: /ɡɑɾɑbed kʰɑpʰiɡjɑn/}) has an extensive study that was funded by an Izmirian (\armenian{Իզմիրեան}) award, but it's unfortunately unpublished. Based on this work, Mr. Gabikian (\armenian{պր. Գաբիկեան}) was kind enough to send me a succinct note on this study of the dialect, and a manuscript with a few pages, which I will provide a bit later. 


\section{Phonology}
\subsection{Segment inventory}
\subsubsection{Vowels}
\paragraph{Vowel inventory}
The sound system of the Sebastia dialect is similar to the dialects of Karin and Kharberd-Yerznka. For vowels, besides the sounds /ɑ, e, ə, i, o, u/ <\armenian{ա, է, ը, ի, օ, ու}>, we also find /æ, i̯e, œ, ʏ, u̯œ/ <\armenian{ա}̈, \armenian{ե, էօ, իւ, օ}̂>. 

\paragraph{Vowel /u̯œ/ <\armenian{օ}̂>}
The last one is a sound that's uniquely characteristic to Sebastia; its pronunciation is approximately like a fast pronunciation of the sequence /uœ/ <\armenian{ուէօ}>. It is found word-initially and word-medially, but always under stress. When it is unstressed, it becomes a simple /o/ <\armenian{օ}> (Table \ref{tab:Sebastia:phonology:vowel:uœ}). 


\begin{table}[H]
 	\centering 
 	\caption{Emergence of /u̯œ/ <\armenian{օ}̂> in the Sebastia dialect}
 	\label{tab:Sebastia:phonology:vowel:uœ}
 	\begin{tabular}{|l| ll|ll| ll|}
 		\hline & \multicolumn{2}{l|}{Classical Armenian} &\multicolumn{2}{l|}{> Sebastia} & \multicolumn{2}{l|}{cf. SEA} \\ 
`frog' &ɡoɾt & \armenian{գորտ} & ɡʰu̯œɾd & \armenian{գՙօ}̂\armenian{րդ} & ɡoɾt& \armenian{գորտ} \\
`frog-{\gen}' &ɡoɾt-i & \armenian{գորտի} & ɡʰoɾd-ɑn & \armenian{գՙօրդան} & ɡoɾt-i& \armenian{գորտի} \\
 `horse-radish' & boɬk & \armenian{բողկ} & bʰu̯œʁɡ & \armenian{բՙօ}̂\armenian{ղգ} & boχk & \armenian{բողկ} \\ 
 `horse.radish-{\gen}/{\dat}' & boɬk-i & \armenian{բողկի} & bʰoʁɡ-i & \armenian{բՙօղգի} & boχk-i & \armenian{բողկի} \\ 
 `horse.radish-{\abl}' & boɬk-e & \armenian{բողկէ} & bʰoʁɡ-e & \armenian{բՙօղգէ} & boχk-it͡sʰ & \armenian{բողկից} \\ 
 `horse.radish-{\ins}' & boɬk-iu̯ & \armenian{բողկիւ} & bʰoʁɡ-ov & \armenian{բՙօղգօվ} & boχk-it͡sʰ & \armenian{բողկով} \\ 
\hline 
 	\end{tabular}
\end{table}

\paragraph{Vowel /i̯e/ <\armenian{ե}>}

The sound /i̯e/ <\armenian{ե}> (pronounced as a heavy /ie/ <\armenian{իէ}>) is more common in the villages of the Alis valley (Table \ref{tab:Sebastia:phonology:vowel:i̯e}). 


\begin{table}[H]
 	\centering 
 	\caption{Emergence of /i̯e/ <\armenian{ե}> in the Sebastia dialect}
 	\label{tab:Sebastia:phonology:vowel:i̯e}
 	\begin{tabular}{|l| ll|ll| ll|}
 		\hline & \multicolumn{2}{l|}{Classical Armenian} &\multicolumn{2}{l|}{> Sebastia} & \multicolumn{2}{l|}{cf. SEA} \\ 
`mother' & mɑi̯ɾ & \armenian{մայր} & mi̯eɾ & \armenian{մեր} & mɑjɾ & \armenian{մայր} \\ 
`Karapet (a given name)' & kɑɾɑpet & \armenian{Կարապետ} & ɡɑɾɑbed & \armenian{Գարաբեդ} & kɑɾɑpet & \armenian{Կարապետ} \\ 
`you went' & ɡənɑt͡sʰeɾ & \armenian{գնացեր} & ɡʰnɑt͡sʰeɾ & \armenian{գՙնացեր} & ɡənɑt͡sʰiɾ & \armenian{գնացիր} \\ 
`mouth' &beɾɑn & \armenian{բերան} & bʰeɾɑn & \armenian{բՙերան} &beɾɑn & \armenian{բերան} \\
\hline 
 	\end{tabular}
\end{table}



\paragraph{Vowel /æ/ <\armenian{ա}̈>}

Sometimes we find the sound /æ/ <\armenian{ա}̈> (Table \ref{tab:Sebastia:phonology:vowel:æ}). 


\begin{table}[H]
 	\centering 
 	\caption{Emergence of /æ/ <\armenian{ա}̈> in the Sebastia dialect}
 	\label{tab:Sebastia:phonology:vowel:æ}
 	\begin{tabular}{|l| ll|ll| ll|}
 		\hline & \multicolumn{2}{l|}{Classical Armenian} &\multicolumn{2}{l|}{> Sebastia} & \multicolumn{2}{l|}{cf. SEA} \\ 
 `hand' &d͡zer-kʰ (-{\pl}) & \armenian{ձեռք} & d͡zʰærkʰ & \armenian{ձՙա̈ռք} & d͡zerkʰ & \armenian{ձեռք} \\ 
 `corpse' & mere̯ɑl & \armenian{մեռեալ} & mærel & \armenian{մա̈ռէլ} & merjɑl & \armenian{մեռյալ} \\
\hline 
 	\end{tabular}
\end{table}

\subsubsection{Consonants}
\paragraph{Laryngeal values}

The consonants have three degrees: voiced, voiced aspirated, and voiceless aspirates. Their changes are exactly as in the Karin and Kharberd dialects. 
\paragraph{Emergence of word-initial /j/ <\armenian{յ}̵> }

Here we have the sound /ɦ/ <\armenian{յ}̵> which is often added before vowel-initial words (Table \ref{tab:Sebastia:phonology:cons:ɦ}). 


\begin{table}[H]
 	\centering 
 	\caption{Emergence of word-initial /ɦ/ <\armenian{յ}̵> before vowels in the Sebastia dialect}
 	\label{tab:Sebastia:phonology:cons:ɦ}
 	\begin{tabular}{|l| ll|ll| ll|}
 		\hline & \multicolumn{2}{l|}{Classical Armenian} &\multicolumn{2}{l|}{> Sebastia} & \multicolumn{2}{l|}{cf. SEA} \\ 
`tail' &ɑɡi& \armenian{ագի} & ɦɑɡʰi & \armenian{յ}̵\armenian{ագՙի} &ɑɡi& \armenian{ագի} \\
`tinder' &ɑbetʰ& \armenian{աբեթ} & ɦɑbʰetʰ & \armenian{յ}̵\armenian{աբՙէթ} &ɑbetʰ& \armenian{աբեթ} \\
`Alis river' & & & ɦɑlis & \armenian{յ}̵\armenian{ալիս} &ɑlis& \armenian{Ալիս} \\
\hline 
 	\end{tabular}
\end{table}



\begin{adjarianpage}\label{page:226}\end{adjarianpage}% should be 226

\subsection{Sound changes}
For the sound changes, the important ones are the following.

\subsubsection{Monopthonal vowel changes}

\paragraph{Classical Armenian /e/ <\armenian{ե}> }

The Classical sound /e/ <\armenian{ե}> (Table \ref{tab:Sebastia:phonology:change:e}a), in the beginning of both monosyllabic and polysyllabic words, is sometimes /je/ <\armenian{յէ}> and sometimes /e/ <\armenian{է}> (Table \ref{tab:Sebastia:phonology:change:e}a). Inside the word, it becomes /e, æ, i̯e/ <\armenian{է, ա}̈, \armenian{ե}> (Table \ref{tab:Sebastia:phonology:change:e}b). 



\begin{table}[H]
 	\centering 
 	\caption{Change from Classical Armenian /e/ <\armenian{ե}> to /je, e, æ, i̯e/ <\armenian{յէ, է, ա}̈, \armenian{ե}> in the Sebastia dialect}
 	\label{tab:Sebastia:phonology:change:e}
 	\begin{tabular}{|ll| ll|ll| ll|}
 		\hline && \multicolumn{2}{l|}{Classical Armenian} &\multicolumn{2}{l|}{> Sebastia} & \multicolumn{2}{l|}{cf. SEA} \\ 
 a. & \armenian{՝}when' & eɾb & \armenian{երբ} & jepʰ & \armenian{յէփ} & jeɾpʰ & \armenian{երբ} \\
 & \armenian{՝}to delimit' & ezeɾel & \armenian{եզերել} & jezeɾel & \armenian{յէզէրէլ} & jezeɾel & \armenian{եզերել} \\
 & \armenian{՝}face' & eɾes & \armenian{երես} & eɾes & \armenian{էրէս} & jeɾes & \armenian{երես} \\
 & \armenian{՝}iron' & eɾkɑtʰ & \armenian{երկաթ} & eɾɡɑtʰ & \armenian{էրգաթ} & jeɾkɑtʰ & \armenian{երկաթ} \\
 b. & \armenian{՝}last year' & heɾu & \armenian{հերու} & hi̯eɾu & \armenian{հերու} & heɾu & \armenian{հերու} \\
& `hand' &d͡zer-kʰ (-{\pl}) & \armenian{ձեռք} & d͡zʰærkʰ & \armenian{ձՙա̈ռք} & d͡zerkʰ & \armenian{ձեռք} \\ 
&`mouth' &beɾɑn & \armenian{բերան} & bʰeɾɑn & \armenian{բՙերան} &beɾɑn & \armenian{բերան} \\
\hline 
 	\end{tabular}
\end{table}


\paragraph{Classical Armenian /e/ <\armenian{ո}> }

 

The Classical sound /o/ <\armenian{ո}> becomes /u̯œ/ <\armenian{օ}̂> at at the beginning of monosyllabic words, and it becomes /o/ <\armenian{օ}> at the beginning of polysyllabic words (Table \ref{tab:Sebastia:phonology:change:o}a). Under stress, it becomes /u̯œ/ <\armenian{օ}̂>; when unstressed, it becomes /o/ <\armenian{օ}>. An exception is in Table \ref{tab:Sebastia:phonology:change:o}b. 




\begin{table}[H]
 	\centering 
 	\caption{Change from Classical Armenian /o/ <\armenian{ո}> to /u̯œ, / <\armenian{օ}̂, \armenian{օ}> in the Sebastia dialect}
 	\label{tab:Sebastia:phonology:change:o}
 	\begin{tabular}{|l l| ll|ll| ll|}
 		\hline && \multicolumn{2}{l|}{Classical Armenian} &\multicolumn{2}{l|}{> Sebastia} & \multicolumn{2}{l|}{cf. SEA} \\ 
 a. \armenian{՝}who' & ov & \armenian{ով} & u̯œv & \armenian{օ}̂\armenian{վ} & ov & \armenian{ով} \\ 
& \armenian{՝}no' & ot͡ʃʰ & \armenian{ոչ} & u̯œt͡ʃʰ & \armenian{օ}̂\armenian{չ} & vot͡ʃʰ & \armenian{ոչ} \\ 
& `male' &oɾd͡z & \armenian{որձ} & u̯œɾt͡sʰ & \armenian{օ}̂\armenian{րց} &voɾt͡sʰ & \armenian{որձ} \\
& `worm' & oɾdən & \armenian{որդն} & u̯œɾtʰ & \armenian{օ}̂\armenian{րթ} & voɾtʰ & \armenian{որդ} \\
&\armenian{՝}orphan' & oɾb & \armenian{որբ} & u̯œɾpʰ & \armenian{օ}̂\armenian{րփ} & voɾpʰ & \armenian{որբ} \\
&`frog' &ɡoɾt & \armenian{գորտ} & ɡʰu̯œɾd & \armenian{գՙօ}̂\armenian{րդ} & ɡoɾt& \armenian{գորտ} \\
&`frog-{\gen}' &ɡoɾt-i & \armenian{գորտի} & ɡʰoɾd-ɑn & \armenian{գՙօրդան} & ɡoɾt-i& \armenian{գորտի} \\
& \armenian{՝}belly' & pʰoɾ & \armenian{փոր}& pʰu̯œɾ & \armenian{փօ}̂\armenian{ր} & pʰoɾ & \armenian{փոր} \\
& \armenian{՝}charcoal' & ɡoɾt͡seli & \armenian{գործելի}& ɡʰoɾd͡zeli & \armenian{գՙօրձէլի} & ɡoɾt͡seli & \armenian{գործելի} \\
&`bone' & oskəɾ & \armenian{ոսկր} & osɡo. & \armenian{օսգօր} & voskoɾ & \armenian{ոսկոր} \\
& `sheep' & ot͡ʃʰχɑɾ & \armenian{ոչխար} & oʃχɑɾ & \armenian{օշխար} & vot͡ʃʰχɑɾ & \armenian{ոչխար} \\
b. & `buttocks' & or & \armenian{ոռ} & vu̯œr, ɦu̯œr & \armenian{վօ}̂\armenian{ռ, յ}̵\armenian{օ}̂\armenian{ռ} & vor & \armenian{ոռ} \\

 \hline 	\end{tabular}
\end{table}


\subsubsection{Dipthonal vowel changes}


\paragraph{Classical Armenian /ɑi̯/ <\armenian{այ}> }

The Classical diphthong /ɑi̯/ <\armenian{այ}> changed to /ɑ/ <\armenian{ա}> (Table \ref{tab:Sebastia:phonology:change:aj}). 



\begin{table}[H]
 	\centering 
 	\caption{Change from Classical Armenian /ɑi̯/ <\armenian{այ}> to /ɑ/ <\armenian{ա}> in the Sebastia dialect}
 	\label{tab:Sebastia:phonology:change:aj}
 	\begin{tabular}{|l| ll|ll| ll|}
 		\hline & \multicolumn{2}{l|}{Classical Armenian} &\multicolumn{2}{l|}{> Sebastia} & \multicolumn{2}{l|}{cf. SEA} \\ 
`mother' & mɑi̯ɾ & \armenian{մայր} & mɑɾ & \armenian{մար} & mɑjɾ & \armenian{մայր} \\ 
`wolf' & ɡɑi̯l & \armenian{գայլ} & ɡʰɑl & \armenian{գՙալ} & ɡɑjl & \armenian{գայլ} \\ 

\hline 
 	\end{tabular}
\end{table}

\paragraph{Classical Armenian /iu̯/ <\armenian{իւ}> }

The Classical diphthong /iu̯/ <\armenian{իւ}> changed to /u/ <\armenian{ու}> (Table \ref{tab:Sebastia:phonology:change:iu̯}). 



\begin{table}[H]
 	\centering 
 	\caption{Change from Classical Armenian /iu̯/ <\armenian{իւ}> to /u/ <\armenian{ու}> in the Sebastia dialect}
 	\label{tab:Sebastia:phonology:change:iu̯}
 	\begin{tabular}{|l| ll|ll| ll|}
 		\hline & \multicolumn{2}{l|}{Classical Armenian} &\multicolumn{2}{l|}{> Sebastia} & \multicolumn{2}{l|}{cf. SEA} \\ 
 `flour' & ɑliu̯ɾ & \armenian{ալիւր} & ɑluɾ & \armenian{ալուր} & ɑljuɾ & \armenian{ալյուր} \\ 		 
\armenian{՝}blood' & ɑɾiu̯n & \armenian{արիւն}& ɑɾun & \armenian{արուն} & ɑɾjun & \armenian{արյուն} \\
 

\hline 
 	\end{tabular}
\end{table}


\paragraph{Classical Armenian /oi̯/ <\armenian{ոյ}> }

The Classical diphthong /oi̯/ <\armenian{ոյ}> changed to /u/ <\armenian{ու}> (Table \ref{tab:Sebastia:phonology:change:oi̯}). 



\begin{table}[H]
 	\centering 
 	\caption{Change from Classical Armenian /oi̯/ <\armenian{ոյ}> to /u/ <\armenian{ու}> in the Sebastia dialect}
 	\label{tab:Sebastia:phonology:change:oi̯}
 	\begin{tabular}{|l| ll|ll| ll|}
 		\hline & \multicolumn{2}{l|}{Classical Armenian} &\multicolumn{2}{l|}{> Sebastia} & \multicolumn{2}{l|}{cf. SEA} \\ 
 		`light' & loi̯s & \armenian{լոյս} & lus & \armenian{լուս} & lujs & \armenian{լույս} \\ 
 
 	`walnut' & ənkoi̯z & \armenian{ընկոյզ} & ənɡuz & \armenian{ընգուզ} & əŋkujz & \armenian{ընկույզ} \\ 

\hline 
 	\end{tabular}
\end{table}

\subsubsection{Consonant changes}

For consonants, the following are notable. 

\paragraph{Weakening of stops to glides}
The Classical sound /kʰ/ <\armenian{ք}> became /jh/ <\armenian{յհ}>, which happens in the villages of Alis (Table \ref{tab:Sebastia:phonology:change:kʰj}). 



\begin{table}[H]
 	\centering 
 	\caption{Change from Classical Armenian /kʰ/ <\armenian{ք}> to /jh/ <\armenian{յհ}> in the Sebastia dialect}
 	\label{tab:Sebastia:phonology:change:kʰj}
 	\begin{tabular}{|l| ll|ll| ll|}
 		\hline & \multicolumn{2}{l|}{Classical Armenian} &\multicolumn{2}{l|}{> Sebastia} & \multicolumn{2}{l|}{cf. SEA} \\ 
`monastery' & vɑnkʰ & \armenian{վանք} & vɑjh & \armenian{վայհ} &vɑŋkʰ & \armenian{վանք} \\
 `three' &eɾekʰ & \armenian{երեք} & iɾejh & \armenian{իրէյհ} &jeɾekʰ & \armenian{երեք} \\
 `desire' &pʰɑpʰɑkʰ, pʰɑpʰɑɡ & \armenian{փափաք, փափագ} & pʰɑpʰɑjh & \armenian{փափայհ} &pʰɑpʰɑkʰ, pʰɑpʰɑɡ & \armenian{փափաք, փափագ}\\

\hline 
 	\end{tabular}
\end{table}

The Classical sound /k/ <\armenian{կ}> became /j/ <\armenian{յ}> before a consonant (Table \ref{tab:Sebastia:phonology:change:kj}). 



\begin{table}[H]
 	\centering 
 	\caption{Change from Classical Armenian /k/ <\armenian{կ}> to /j/ <\armenian{յ}> in the Sebastia dialect}
 	\label{tab:Sebastia:phonology:change:kj}
 	\begin{tabular}{|l| ll|ll| lll|}
 		\hline & \multicolumn{2}{l|}{Classical Armenian} &\multicolumn{2}{l|}{> Sebastia} & \multicolumn{3}{l|}{cf. SEA or SWA} \\ 
`monastery' & vɑnkʰ & \armenian{վանք} & vɑjh & \armenian{վայհ} &vɑŋkʰ & \armenian{վանք} & SEA\\
 `three' &eɾekʰ & \armenian{երեք} & iɾejh & \armenian{իրէյհ} &jeɾekʰ & \armenian{երեք} & SEA\\
 `desire' &pʰɑpʰɑkʰ & \armenian{փափաք} & pʰɑpʰɑjh & \armenian{փափայհ} &pʰɑpʰɑkʰ & \armenian{փափաք} & SWA\\
 `trap?, eyeless?' & ɑknɑt & \armenian{ակնատ} & ɑjnad & \armenian{այնադ} &ɑknɑt & \armenian{ակնատ} & SEA\\
 `angel-{\pl}' &həɾeʃtɑk-(ə?)kʰ & \armenian{հրեշտակք} & hɾəʃdɑj-nin & \armenian{հրըշդայնին} &həɾeʃtɑk-neɾ & \armenian{հրեշտակներ} & SEA\\
 `claw {\indf}' &t᷂͡ʃɑnk & \armenian{ճանկ} & d͡ʒɑj mə & \armenian{ջայ մը} &d͡ʒɑŋɡ mə & \armenian{ճանկ մը} & SWA\\
 `boy-girl' &ɑɬd͡ʒik-təɬɑi̯ & \armenian{աղջիկ-տղայ} & ɑχd͡ʒʰij dʁɑ & \armenian{ախջՙիյ դղա} &ɑχt͡ʃʰik-təʁɑ & \armenian{աղջիկ-տղա} & SEA\\
`broad bean' & bɑklɑi̯ & \armenian{բակլայ} & bʰɑjlɑ & \armenian{բՙայլա} & bɑklɑ & \armenian{բակլա} & SEA\\ 

\hline 
 	\end{tabular}
\end{table}


The Classical sound /ɡ/ <\armenian{գ}>, which wherever it took the form /kʰ/ <\armenian{ք}>, became /j/ <\armenian{յ}> (Table \ref{tab:Sebastia:phonology:change:gj}). 



\begin{table}[H]
 	\centering 
 	\caption{Change from Classical Armenian /ɡ/ <\armenian{գ}> to /j/ <\armenian{յ}> in the Sebastia dialect}
 	\label{tab:Sebastia:phonology:change:gj}
 	\begin{tabular}{|l| ll|ll| ll|}
 		\hline & \multicolumn{2}{l|}{Classical Armenian} &\multicolumn{2}{l|}{> Sebastia} & \multicolumn{2}{l|}{cf. SEA } \\ 
`king' & tʰɑɡɑ{wo}ɾ & \armenian{թագաւոր} & tʰɑjhvu̯œɾ & \armenian{թայհվօ}̂\armenian{ր} &tʰɑkʰɑvoɾ & \armenian{թագավոր}\\
`to bathe (trans.)' & loɡɑt͡sʰut͡sʰɑnel & \armenian{լոգացուցանել} & lojt͡sʰənel & \armenian{լօյցընէլ} &loɡɑt͡sʰnel & \armenian{լոգացնել}\\
`lap' &ɡoɡ & \armenian{գոգ} & ɡu̯œjh & \armenian{գօ}̂\armenian{յհ} & ɡokʰ & \armenian{գոգ} \\
`five' &hinɡ & \armenian{հինգ} &hij & \armenian{հիյ} &hiŋɡ & \armenian{հինգ} \\
`apron' &ɡoɡnot͡sʰ & \armenian{գոգնոց} &ɡʰojhnot͡sʰ & \armenian{գՙօյհնօց} &ɡokʰnot͡sʰ & \armenian{գոգնոց} \\
`to bathe' &loɡɑnɑl & \armenian{լոգանալ} & lojnɑl & \armenian{լօյնալ} & loɡɑnɑl& \armenian{լոգանալ} \\
`five points/denomination' & & & hijnot͡sʰ & \armenian{հիյնօց} & hiŋɡnot͡sʰ& \armenian{հինգնոց} \\
\hline 
 	\end{tabular}
\end{table}

 

Analogous to this, we have the Classical form /ʃɑpik/ <\armenian{շապիկ}> `shirt' became /ʃɑbijh/ <\armenian{շաբիյհ}> (which passed through the form /ʃɑpikʰ/ <\armenian{շապիք}>), cf. SEA /ʃɑpik/. 

\paragraph{Classical Armenian /h/ <\armenian{հ}> to /f/ <\armenian{ֆ}>}

The Classical sound /h/ <\armenian{հ}> in monosyllabic words, next to a stressed sound /o/ <\armenian{ո}> ... 


\begin{adjarianpage}\label{page:227}\end{adjarianpage}% should be 227

became /f/ <\armenian{ֆ}> (Table \ref{tab:Sebastia:phonology:change:f}a, but Table \ref{tab:Sebastia:phonology:change:f}b). 



\begin{table}[H]
 	\centering 
 	\caption{Change from Classical Armenian /h/ <\armenian{հ}> to /f/ <\armenian{ֆ}> in the Sebastia dialect}
 	\label{tab:Sebastia:phonology:change:f}
 	\begin{tabular}{|ll| ll|ll| ll|}
 		\hline& & \multicolumn{2}{l|}{Classical Armenian} &\multicolumn{2}{l|}{> Sebastia} & \multicolumn{2}{l|}{cf. SEA } \\ 
a. & `hole (CA); pit (SEA)' &hoɾ & \armenian{հոր} & fu̯œɾ & \armenian{ֆօ}̂\armenian{ր} & hoɾ & \armenian{հոր} \\ 
& `smell' & oɾtʰ & \armenian{որթ} & fu̯œɾtʰ &\armenian{ֆօ}̂\armenian{րթ} & hoɾt & \armenian{հորթ} \\
&`earth' &hoɬ & \armenian{հող} & fu̯œʁ & \armenian{ֆօ}̂\armenian{ղ} & hoʁ & \armenian{հող} \\
& `care' &hoɡ & \armenian{հոգ} & fu̯œkʰ & \armenian{ֆօ}̂\armenian{ք} & hokʰ & \armenian{հոգ} \\
b. & `to care' &hoɡɑl & \armenian{հոգալ} & hokʰɑl & \armenian{հօքալ} & hokʰɑl & \armenian{հոգալ} \\
& `edge of pit' & & & hoɾezeɾ & \armenian{հօրէզէր} & hoɾi jezeɾkʰ & \armenian{հորի եզերք} \\

\hline 
 	\end{tabular}
\end{table}

\paragraph{Consonant cluster lenition}

The Classical Armenian sound sequence /tɾ/ <\armenian{տր}> becomes /jj/ <\armenian{յյ}>, and it can delete if there's a nasal before it (Table \ref{tab:Sebastia:phonology:change:tr}). 



\begin{table}[H]
 	\centering 
 	\caption{Change from Classical Armenian /tɾ/ <\armenian{տր}> to /jj/ in the Sebastia dialect}
 	\label{tab:Sebastia:phonology:change:tr}
 	\begin{tabular}{|l| ll|ll| ll|}
 		\hline & \multicolumn{2}{l|}{Classical Armenian} &\multicolumn{2}{l|}{> Sebastia} & \multicolumn{2}{l|}{cf. SEA } \\ 
`to divide' &kətɾel & \armenian{կտրել} & ɡəjjel & \armenian{գըյյէլ} & kətɾel & \armenian{կտրել} \\
`sharp' & *kətɾuk & *\armenian{կտրուկ} & ɡəjjuɡ & \armenian{գըյյուգ} &kətɾuk &\armenian{կտրուկ} \\
`brave' & *kətɾit͡ʃ & *\armenian{կտրիճ} & ɡəjid͡ʒ & \armenian{գըյիջ} &kətɾit͡ʃ &\armenian{կտրիճ} \\
`to choose' &əntɾel & \armenian{ընտրել} & hənjel & \armenian{հընյէլ} & əntɾel & \armenian{ընտրել} \\
`small' & mɑnəɾ, mɑntəɾ & \armenian{մանր, մանտր} & mɑjjə & \armenian{մայյը} & mɑnəɾ, mɑndəɾ & \armenian{մանր, մանդր} \\
`heavy' & t͡sɑnəɾ, *t͡sɑntəɾ & \armenian{ծանր}, *t͡sɑntəɾ & d͡zɑnjə & \armenian{ձանյը} & t͡sɑnəɾ & \armenian{ծանր} \\
`to break' &kotoɾel & \armenian{կոտորել} & ɡojjel & \armenian{գօյյէլ} & kotoɾel, kotɾel & \armenian{կոտորել, կոտրել} \\
name `Peter' &petɾos & \armenian{Պետրոս} & bejjœs & \armenian{Բէյյէօս} & petɾos & \armenian{Պետրոս} \\

\hline 
 	\end{tabular}
\end{table}


\subsubsection{Subdialectal changes in Pirknik}

 
\paragraph{Classical Armenian /e/ <\armenian{ե}> }

In the Pirknik subdialect, we can find the sound change /ɑ/ <\armenian{ա}> to /u̯œ, o/ <\armenian{օ}̂, \armenian{օ}> which don't exist in the Sebastia dialect (Table \ref{tab:Sebastia:phonology:change:Pirknik:a}). 



\begin{table}[H]
 	\centering 
 	\caption{Change from Classical Armenian /ɑ/ <\armenian{ա}> to /u̯œ, o/ <\armenian{օ}̂, \armenian{օ}> in the Pirknik subdialect of the Sebastia dialect}
 	\label{tab:Sebastia:phonology:change:Pirknik:a}
 	\begin{tabular}{| l| ll|ll| lll|}
 		\hline & \multicolumn{2}{l|}{Classical Armenian} &\multicolumn{2}{l|}{> Pirknik (Sebastia)} & \multicolumn{3}{l|}{cf. SEA or SWA} \\ 
`cheese' &pɑniɾ & \armenian{պանիր} & bu̯œniɾ & \armenian{բօ}̂\armenian{նիր} &pɑniɾ & \armenian{պանիր} & SEA\\
`Mary' &mɑɾiɑm & \armenian{Մարիամ} & mojjɑm & \armenian{Մօյյամ} &mɑɾjɑm & \armenian{Մարիամ} & SEA \\
`like that' & & & ɑdonɡ & \armenian{ադօնգ} &ɑdɑŋɡ & \armenian{ատանկ} & SWA \\
`cross' &χɑt͡ʃʰ & \armenian{խաչ} & χu̯œt͡ʃʰ & \armenian{խօ}̂\armenian{չ} & χɑt͡ʃʰ & \armenian{խաչ} & SEA\\
`time-{\ins}' & & & ʒomɑnɡ-u & \armenian{ժօմանգու} & ʒɑmɑnɑk-ov & \armenian{ժամանակով} & SEA\\
\hline 
 	\end{tabular}
\end{table}

 
\paragraph{Classical Armenian /o/ <\armenian{ո}>}

The Classical sound /o/ <\armenian{ո}> becomes /ɑ/ <\armenian{ա}> next to the sounds /v, m, n/ <\armenian{վ, մ, ն}> (Table \ref{tab:Sebastia:phonology:change:Pirknik:o}), and even /ɑnbiɾ/ <\armenian{անբիր}> `eleven' 11, borrowed from Turkish <on bir> `eleven'. 



\begin{table}[H]
 	\centering 
 	\caption{Change from Classical Armenian /o/ <\armenian{ո}> becomes /ɑ/ <\armenian{ա}> in the Pirknik subdialect of the Sebastia dialect}
 	\label{tab:Sebastia:phonology:change:Pirknik:o}
 	\begin{tabular}{| l| ll|ll| ll|}
 		\hline & \multicolumn{2}{l|}{Classical Armenian} &\multicolumn{2}{l|}{> Pirknik (Sebastia)} & \multicolumn{2}{l|}{cf. SEA } \\ 
a village's name & & & ɡɑvdun & \armenian{Գավդուն} & koftun & \armenian{Կովտուն} \\
`buffalo' &ɡomēʃ & \armenian{գոմէշ} & ɡʰɑmeʃ, ɡʰɑvmeʃ & \armenian{գՙամէշ, գՙավմէշ} & ɡomeʃ& \armenian{գոմեշ} \\
\hline 
 	\end{tabular}
\end{table}


\section{Morphology}

\subsection{Noun inflection or declension}
The grammar is very similar to the Istanbul dialect. The case declensions are the same. 

\translatorHD{On page 306, Adjarian provides an erratum where he adds the following information about this dialect's nominal morphology:} 
 	In the villages of Sebastia, there are many cases of the plural suffix /-ni/ <\armenian{նի}> (Table \ref{tab:Sebastia:morpho:noun:pl}). 
 

\begin{table}[H]
 	\centering 
 	\caption{Plural marking with /-ni/ <\armenian{նի}> in the Sebastia dialect}
 	\label{tab:Sebastia:morpho:noun:pl}
 	\begin{tabular}{| l| ll| ll|}
 		\hline & \multicolumn{2}{l|}{Sebastia} & \multicolumn{2}{l|}{cf. SEA } \\ 
 \armenian{՝}ox' & & & jez & \armenian{եզ} \\
 \armenian{՝}ox-{\pl}' & jez-ni & \armenian{յէզնի} & jez-neɾ & \armenian{եզներ} \\
 
 		\hline 
 	\end{tabular}
\end{table}


\subsection{Pronoun inflection or declension}

For the pronouns, the following forms are notable (Table \ref{tab:Sebastia:morpho:pron:samp}). 

\begin{table}[H]
 \centering
 \caption{Sample of pronouns in the Sebastia dialect}
 \label{tab:Sebastia:morpho:pron:samp}
 \begin{tabular}{|l ll| }
\hline 
demonstrative proximal {\sg} {\nom} `this' &ɑsi & \armenian{ասի} \\
demonstrative medial {\sg} {\nom} `that' &ɑdi & \armenian{ադի} \\
demonstrative distal {\sg} {\nom} `that yonder' &ɑni & \armenian{անի} \\
demonstrative proximal {\sg} {\nom} `this' &zəviɡɑɡ & \armenian{զըվիգագ} \\
demonstrative proximal {\sg} {\abl} `from this' &zəɡe & \armenian{զըգէ} \\
 `like this' &z\'ɑnɡəs & \armenian{զա՛նգըս} \\
 `like this' &zəvizɑnɡ & \armenian{զըվիզանգ} \\

\hline 
 \end{tabular}
\end{table}


\subsection{Veb inflection or conjugation}

\subsubsection{Mobile indicative marking}
\translatorHD{In SWA, the indicative mood is formed by combining the indicative prefix with the finite verb. For example, the indicative present is formed by combining this prefix with the present form of verbs. This prefix is /ɡu-/ for monosyllabic verb stems, /ɡ-/ for vowel-initial stems, and /ɡə-/ elsewhere for polysyllabic consonant-initial stems. In Sebastia however, Adjarian reports that the shape and position of this affix can vary. }


The simple present of verbs is formed similarly to the Karin dialect, with a postposed formative /ɡə/ <\armenian{գը}> (Table \ref{tab:Sebastia:morpho:verb:indc}a), which sometimes can also be placed first (b). Before vowel-initial verbs, the formative /ɡ/ <\armenian{գ}> is preposed, but in many cases the postposed formative /ɡə/ <\armenian{գը}> is also added (c). Monosyllabic verbs take /ɡu/ <\armenian{գու}> (d), while the form `to come' takes /ɡʰu/ <\armenian{գՙու}> (e). 




\begin{table}[H]
 	\centering 
 	\caption{Mobile indicative marking in the indicative present <\armenian{ներկայ}> in the Sebastia dialect}
 	\label{tab:Sebastia:morpho:verb:indc}
 	\begin{tabular}{| ll| ll| ll|}
 		\hline & &\multicolumn{2}{l|}{Sebastia} & \multicolumn{2}{l|}{cf. SWA } \\ 
a. & `he asks' &hɑɾ-t͡sʰn-e-$\emptyset$ ɡə & \armenian{հարցնէ գը}& ɡə hɑɾ-t͡sʰən-e-$\emptyset$& \armenian{կը հարցնէ}\\
& &\multicolumn{2}{l|}{ask-{\caus}-{\thgloss}-3{\sg} {\ind}} &\multicolumn{2}{l|}{{\ind} ask-{\caus}-{\thgloss}-3{\sg}}\\
b. & `he leaves' &t͡sʰkʰ-e-$\emptyset$ ɡə & \armenian{ցքէ գը}, & ɡə t͡sʰəkʰ-e-$\emptyset$& \armenian{կը ձքէ}\\
& &\multicolumn{2}{l|}{leave-{\thgloss}-3{\sg} {\ind}} &\multicolumn{2}{l|}{{\ind} leave-{\thgloss}-3{\sg} }\\
 & &ɡə t͡sʰkʰ-e-$\emptyset$ & \armenian{գը ցքէ} & & \\
& & \multicolumn{2}{l|}{{\ind} leave-{\thgloss}-3{\sg}} & & \\
c. & `he rises' &ɡ-ell-e-$\emptyset$ (ɡə) & \armenian{գէլլէ (գը}) & ɡ-ell-e-$\emptyset$& \armenian{կ՚ելլէ} \\
& &\multicolumn{2}{l|}{{\ind}-rise-{\thgloss}-3{\sg} ({\ind})} &\multicolumn{2}{l|}{{\ind}-rise-{\thgloss}-3{\sg}}\\
 & `he goes' &ɡ-eɾtʰ-ɑ-$\emptyset$ (ɡə) & \armenian{գէրթա (գը}) & ɡ-eɾtʰ-ɑ-$\emptyset$& \armenian{կ՚երթայ} \\
& &\multicolumn{2}{l|}{{\ind}-go-{\thgloss}-3{\sg} ({\ind})} &\multicolumn{2}{l|}{{\ind}-go-{\thgloss}-3{\sg}}\\
 & `they say' &ɡ-əs-e-n & \armenian{գըսէն գը}& ɡ-əs-e-$\emptyset$& \armenian{կ՚ըսեն} \\
& &\multicolumn{2}{l|}{{\ind}-say-{\thgloss}-3{\pl}} &\multicolumn{2}{l|}{{\ind}-say-{\thgloss}-3{\pl}}\\
d. & `he gives' &ɡu-d-ɑ-$\emptyset$ & \armenian{գուդա} & ɡu-d-ɑ-$\emptyset$& \armenian{կու տայ} \\
& &\multicolumn{2}{l|}{{\ind}-give-{\thgloss}-3{\sg}} &\multicolumn{2}{l|}{{\ind}-give-{\thgloss}-3{\sg}}\\
e. & `he comes' &ɡʰu-ɡʰ-ɑ-$\emptyset$ & \armenian{գՙուգՙա} & ɡu-kʰ-ɑ-$\emptyset$& \armenian{կու գայ} \\
& &\multicolumn{2}{l|}{{\ind}-come-{\thgloss}-3{\sg}} &\multicolumn{2}{l|}{{\ind}-come-{\thgloss}-3{\sg}}\\
\hline 
 	\end{tabular}
\end{table}

\subsubsection{Progressive marking}

\translatorHD{In SWA, the progressive is formed by adding the enclitic /ɡoɾ/ after the indicative mood forms. This enclitic is used in spoken speech, not in writing. }

The progressive, which does not exist in Karin, is formed with the formative /ɡoɾ/ <\armenian{գօր}<, similar to the Istanbul dialect; but here, the formative /ɡə/ <\armenian{գը}> is added only for vowel-initial verbs (Table \ref{tab:Sebastia:morpho:verb:prog}).



\begin{table}[H]
 	\centering 
 	\caption{Progressive marking in the Sebastia dialect}
 	\label{tab:Sebastia:morpho:verb:prog}
 	\begin{tabular}{| l| ll| ll|}
 		\hline & \multicolumn{2}{l|}{Sebastia} & \multicolumn{2}{l|}{cf. SWA } \\ 
 `I am bringing' &bʰeɾ-e-m ɡoɾ & \armenian{բՙէրէմ գօր} & ɡə pʰeɾ-e-m ɡoɾ & \armenian{կը բերեմ կոր}\\
 & \multicolumn{2}{l|}{bring-{\thgloss}-1{\sg} {\prog}}& \multicolumn{2}{l|}{{\ind} bring-{\thgloss}-1{\sg} {\prog}} \\ 
 `you are doing' &ɡ-ən-e-kʰ ɡoɾ & \armenian{գընէք գօր} & ɡ-ən-e-kʰ ɡoɾ & \armenian{կ՚ընէք կոր}\\
 & \multicolumn{2}{l|}{{\ind}-do-{\thgloss}-2{\pl} {\prog}}& \multicolumn{2}{l|}{{\ind}-do-{\thgloss}-2{\pl} {\prog}} \\ 
 `I was eating' &ɡ-ud-ej-i-$\emptyset$ ɡoɾ & \armenian{գուդէյի գօր} & ɡ-ud-ej-i-$\emptyset$ ɡoɾ & \armenian{կ՚ուտէի կոր}\\
 & \multicolumn{2}{l|}{{\ind}-eat-{\thgloss}-{\pst}-1{\sg} {\prog}}& \multicolumn{2}{l|}{{\ind}-do-{\thgloss}-{\pst}-1{\sg} {\prog}} \\ 
\hline 
 	\end{tabular}
\end{table}

\subsubsection{Mobile future marking}
\translatorHD{In SWA, the future is formed by combining the future proclitic /bidi/ with the finite verb. As Adjarian explains however, this future formative can vary its position in Sebastia. }


The future takes the formative /bidi/ <\armenian{բիդի}>, which can be placed also after the verb; and next to a vowel it becomes /bi/ <\armenian{բի}> (Table \ref{tab:Sebastia:morpho:verb:fut}).



\begin{table}[H]
 	\centering 
 	\caption{Mobile future marking in the Sebastia dialect}
 	\label{tab:Sebastia:morpho:verb:fut}
 	\begin{tabular}{| l| ll| ll|}
 		\hline & \multicolumn{2}{l|}{Sebastia} & \multicolumn{2}{l|}{cf. SWA } \\ 
 `I will give' &bidi d-ɑ-m & \armenian{բիդի դամ} & bidi d-ɑ-m & \armenian{պիտի տամ}\\
 & \multicolumn{2}{l|}{{\fut} give-{\thgloss}-1{\sg} }& \multicolumn{2}{l|}{{\fut} give-{\thgloss}-1{\sg} }\\ 
 & d-ɑ-m bidi & \armenian{դամ բիդի}& & \\
 & \multicolumn{2}{l|}{{\fut} give-{\thgloss}-1{\sg} }& & \\ 
 `we will do' &bi ən-i̯e-nkʰ & \armenian{բի ընենք} & bidi ən-e-ŋkʰ & \armenian{պիտի ընենք}\\
 & \multicolumn{2}{l|}{{\fut} do-{\thgloss}-1{\pl} {\prog}}& \multicolumn{2}{l|}{{\fut} do-{\thgloss}-1{\pl}} \\ 
\hline 
 	\end{tabular}
\end{table}

\subsubsection{Archaism in the past plural suffix}\label{sec:Sebastia:morphology:verb:archaicPlPst}
\translatorHD{In SWA and SEA, the 1PL suffix is [-ŋkʰ]. This same formative is used for the present, past imperfective, and past perfective. In the past, this plural suffix follows the past suffix /-i/ or /-ɑ/, thus creating the sequence [-i-ŋkʰ] or [-ɑ-ŋkʰ]. In contrast, Classical Armenian used the suffix /-mkʰ/ for the present 1PL, and /-ɑkʰ/ for the past 1PL; the /-ɑ/ in this form could be separately segmented as a past suffix. Adjarian reports that Sebastia aligns with Classical Armenian. }


In verb conjugation, there are no vowel changes. Only that the perfective uses the 1PL suffix /ɑ-nkʰ/ <\armenian{անք}>, which is in accordance with Old Armenian (Table \ref{tab:Sebastia:morpho:verb:pl}).



\begin{table}[H]
 \centering
 \caption{Archaisms in the 1PL suffix in the Sebastia dialect}
 \label{tab:Sebastia:morpho:verb:pl}
 \begin{tabular}{|l| ll|ll| ll|}
 \hline & \multicolumn{2}{l|}{Classical Armenian} &\multicolumn{2}{l|}{> Sebastia } & \multicolumn{2}{l|}{cf. SEA } \\ 
 \armenian{՝}we wrote' & ɡəɾ-e-t͡sʰ-ɑ-kʰ & \armenian{գրեցաք}& ɡʰɾ-e-t͡sʰ-ɑ-nkʰ & \armenian{գՙրէցանք} & ɡəɾ-e-t͡sʰ-i-ŋkʰ & \armenian{գրեցինք} \\
 & \multicolumn{2}{l|}{write-{\thgloss}-{\aor}-{\pst}-1{\pl}}& \multicolumn{2}{l|}{write-{\thgloss}-{\aor}-{\pst}-1{\pl}}& \multicolumn{2}{l|}{write-{\thgloss}-{\aor}-{\pst}-1{\pl}}
 \\
 \hline \end{tabular}
\end{table}

\translatorHD{Note that the above description that there are no vowel changes is contradicted by Adjarian's subsequent discussion of vowel changes.}

\subsubsection{Theme vowel and auxiliary changes}

The villages of the Alis valley, and the subdialect of Pirknik use the ending /-i-m/ <\armenian{իմ}> instead of the ending /-e-m/ <\armenian{եմ}> (\ref{sent:Sebastia:verb:theme:1sg}). \translatorHD{He means the theme vowel changes its shape in the present 1SG.}

\begin{exe}
 \ex \label{sent:Sebastia:verb:theme:1sg}
 \begin{xlist}
 \ex Sebastia \gll 
 bʰeɾ-i-m ɡə \\
 bring-{\thgloss}-1{\sg} {\ind} \\
 \trans `I bring.' \\
 \armenian{բՙէրիմ գը}
 \ex cf. SWA \gll 
 ɡə pʰeɾ-e-m \\
 {\ind} bring-{\thgloss}-1{\sg} \\
 \trans `I bring.' \\
 \armenian{կը բերեմ}
 \end{xlist}
\end{exe}

In the negative, the /e/ <\armenian{ե}> becomes /u/ <\armenian{ու}> (\ref{sent:Sebastia:verb:theme:neg}). \translatorHD{He means that the vowel of the negative auxiliary is /u/ instead of CA/SEA /e/.}

\begin{exe}
 \ex \label{sent:Sebastia:verb:theme:neg}
 \begin{xlist}
 \ex Sebastia \gll 
 t͡ʃʰ-u-m d͡ʒoʃ-n-ɑ-ɾ \\
 {\neggloss}-{\aux}-1{\sg} recognize-{\inch}-{\thgloss}-{\cn} \\
 \trans `I don't recognize.' \\
 \armenian{չում ջօշնար}
 \ex cf. SWA \gll 
 t͡ʃʰ-e-m d͡ʒɑnt͡ʃ-n-ɑ-ɾ \\
{\neggloss}-{\aux}-1{\sg} recognize-{\inch}-{\thgloss}-{\cn} \\
 \trans `I don't recognize.' \\
 \armenian{չեմ ճանչնար}
 \end{xlist}
\end{exe}


\translatorHD{In addition to the above data, Adjarian provides an erratum on page 306. He states the following about the reflexes of the Classical theme vowel /e/ and of the auxiliary /e/. Note that I could not easily determine the meaning, ancestor, or cognate of his example verb.} In verbs, the Classical theme vowel /e/ <\armenian{ե}> changes to <\armenian{ի}> (Table \ref{tab:Sebastia:morpho:change:theme:More}). 


\begin{table}[H]
 	\centering 
 	\caption{Vowel changes in verbs in the Sebastia dialect}
 	\label{tab:Sebastia:morpho:change:theme:More}
 	\begin{tabular}{|l | ll| ll|l|}
 		\hline & \multicolumn{2}{l|}{Sebastia} & \multicolumn{2}{l|}{cf. SWA} & \\ 
 		1SG ? & ɡn-i-m & \armenian{գնիմ} & && $\sqrt{}$-{\thgloss}-1{\sg} \\ 
 		2SG ? & ɡn-i-s & \armenian{գնիս} & & & $\sqrt{}$-{\thgloss}-2{\sg} \\ 
 		3SG ? & ɡn-i-$\emptyset$ & \armenian{գնի} && & $\sqrt{}$-{\thgloss}-3{\sg} \\ 
 		1SG \armenian{՝}I am not' & t͡ʃʰ-i-m & \armenian{չիմ} & t͡ʃʰ-e-m & \armenian{չեմ}& {\aux}-{\thgloss}-1{\sg} \\ 
 		2SG `you are not' & t͡ʃʰ-i-s & \armenian{չիս} & t͡ʃʰ-e-s & \armenian{չես} & {\aux}-{\thgloss}-2{\sg} \\ 
 		3SG `he is not' & t͡ʃʰ-i-$\emptyset$ & \armenian{չի} &t͡ʃʰ-e-$\emptyset$ & \armenian{չէ} & {\aux}-{\thgloss}-3{\sg} \\ 
 		 
 		\hline 	\end{tabular}
\end{table}


\begin{adjarianpage}\label{page:228}\end{adjarianpage}% should be 228


\section{Text samples}

{\sampleoverview}

Adjarian's source: Written by Mr. Karapet Gabikian (\armenian{պր. Կ. Գաբիկեան}). I have rendered it to the scientific orthography. 

\subsection{\armenian{Գՙրվաձը չավրըվիր}}



\armenian{Ժամանագին թաքավօրին մէգը թէբդիլ գէլլէ։ Գՙնալն իքէն էլման գՙէդ մը գէլլէ յ}̵\armenian{էռջՙէվը։ Գՙէդը բՙռնէ դՙարվէր մինչէվ ագը գէրթա օր էրգու մարթ թուղթ մը գՙրէն ՝ ջՙուրը ցքէն գօր։}

– \armenian{Ի՞շ գընէք գօր – դէյի հարցնէ գը նէ՝}

– \armenian{Ի՞շ բի ընէնք, սա ինչին ախչիգը աս ինչին դղին գՙրինք գօր. անինչին ախչիգը նաինչին դղին – գըսէն գը։}

– \armenian{Իմ ախչի՞գս օրու գՙրէք բիդի – հարցնէ գը թաքավէօրը նէ՝}

– \armenian{Քու ախչիյդ ալ անիշ դէղը չօբան մը գա, անէօր դղին գՙրէցանք – գըսէն։}

– \armenian{Վա՜յ, ի՞նչ ըսէլ ըլլա. յէս թաքավէօր մ՚ըլլամ դէ, ախչիգս չօբնի մը դղի՛ն գՙրէք. ախշարք յ}̵\armenian{ախշրքի գՙա նէ ըլլալիք բՙան չէ ադի}. – \armenian{գըսէ, հէրսօդի, շիդագ գէրթա չօբանը գՙդնէ, դունը միսաֆիր գըլլա գը։ Նա̈յի գը օր մանչը օրէօսքը մըշը՜ր մըշը՜ր քնանա գօր։ Գընէ չինէր ՝ չօբնին մէդէն շինէ, դՙիմօքը (ծարութեամբը) օսգի գիշառէ, մանչը գՙնէ գը։ Գառնէ ձՙօ}̂\armenian{ր մը դանի, «դէ՛, ախչիգս չօբնին դղին թօղ գՙրէն նէյիմ» գըսէ, մէշքէն խա̈նչա̈րը հանէ, մանչուն սիրդը գը խօթէ, հօ}̂\armenian{ն գքէ, գասնի գէրթա գը։ Մանչը մա̈ռավ գՙիդնա գը։}

\armenian{Ադդէղվանք չօբնի մը սիւրիւյէն ազ մը յ}̵\armenian{ամմէն օր զադվի, գՙուգՙա մանչը ձըձցընէ} 
\armenian{գա̈շթա էղէր։ Աձը բառավի մըն է էղէր բառավը յ}̵\armenian{ամմէն յ}̵\armenian{իրինգուն նա̈յի գը օր աձուն ձձէրը բարբաձ է. բուդ մը գաթ օր ըսէս չիգա։ Անբաջջառ չօբանը աս աձը գթէ գօր գըսէ, գէրթա հէդը ձէզգըվի գը։}

\armenian{Չօբանը յ}̵\armenian{էրթում-բադառ գըլլա, «Շան արուն-թարախ ըլլա, թա̈ օր աձդ գթէմ գօր նէ» գըսէ. բառավը չավդընար։}

\armenian{Ձէզգըվէլօվ թօղ ըլլան, հէղ մը նա̈յին գը օր, աձը սիւրիւյէն զադվէր, գՙլօխն առէր գէրթա գօր։ Աձուն յ}̵\armenian{էղէվէ հէդքիշուք}...


\begin{adjarianpage}\label{page:229}\end{adjarianpage}% should be 229



... \armenian{գէրթան. նէյին գը օր աղվէօր, բՙշգՙառ դղա մը՝ աձը վրան ձռէր ձձցնէ գօր։ Խէնչէրն ալ մանչուն սիրդն է. հանէն գը օր, բարաբին էգէր, մանչուն հէշ բՙան մըն ա չէ էղէր։ Զարմանք կը մնան։}

\armenian{Չօբանը ՝ «բՙարի աշքօվդ դէսսա՞ր հիմա օր աձդ յէս չէմ գթէր գօր էղէր, յ}̵\armenian{էրթում-բադառ գըլլայի գօր, չէիր ավդընար գօր» գըսէ։}

\armenian{Բառավը՝ «Յէս քէզ մէղա. մէղքդ մդնէի գօր, գըսէ, նա̈յէ դէ՛ս Ասսու բՙանը օր՝ աձս գՙուգՙա դէ մէխսումը ձըձցնէ գը էղէր. ի՛շ գՙիդնամ. գՙիդէմ թէ գթէս գոր. օրո՞ւ միք (միտք) գՙուգՙա. Ասվազ բահէր դէ, գՙալագէր չէ էղէր» գըսէ։}

\armenian{Աս հէղու չօբանը ինէն բառավը բաշլիյէն գը յ}̵\armenian{իրարու հէդ հաջագըռվիլ (հակաճառուիլ}). \armenian{ան թէ մանչը ընձի, ան թէ՝ ընձի. էնգ յ}̵\armenian{էղքը բառավը խօսքը գընէ. «Ա՛սղըդէր ադէն է աձս ձըձցնէ գօր էղէր, ընձի գըյնի» գըսէ, մանչը գառնէ, դուն գը բՙէրէ, դղա գընէ գը։ Անունն ալ Բուլդուխ (թրք. գտանք) դՙնէ գը։ Մանչը մէզնա, ուռուօլէօր (կայտառ) գըյիջ մը գըլլա։}

\armenian{Բառավը դէղանը (անկողինը) գը ցքէ, թաքավէօրը բառգի գը. առդըվանց գըլլա, թաքավէօրը գարթննա, մանչուն թուխթ մը գուդա, «շիդագ թաքավօրին ղօնախը դար, գըսէ, քէզ յ}̵\armenian{անդէղ էհյա գընէն»։}

\armenian{Մանչը թուխթը գառնէ, իրէք օր, իրէք քիշէր գէրթա}, ... 


\begin{adjarianpage}\label{page:230}\end{adjarianpage}% should be 230


... \armenian{ղօնախին յ}̵\armenian{էռչէվ հասնի գը։ «Հէլէ քիշ մը հօնքութիւնս առնէմ» գըսէ, բադդագը նսդի գը։ Հէ}X \armenian{մալ նա̈յիս ՝ քունը դանի, քնանա գը։}

\armenian{Թաքավօրին ախչիգը յ}̵\armenian{օդան նսդէր ՝ քէրգէֆ նաշխէ գօր էղէր. նսեազ դէղը խէլքը բարս գընէ (յանկարծ այնպէս մտածել}), \armenian{փէնջիրէն գէլլէ, դՙարվար նէյէ գը օր, ի՜շ գը նէյիս, աղվէօր աղվէօր դղա մը քնանա գօր. քէզի մէգ սարէն գըյյաձ գըյիջ. մանչուն զարնըվի գը. «Անբաջջառ յէս աս դղան գառնէմ» գըսէ. դառնէ նա̈յի գը օր թուխթը մանչուն ձօցուն դՙուս է ընգէր. դառնէ նա̈յի գը օր հօրը գՙիրն է. բՙանա գարթա գը օր ՝ հարը գՙրէր է թէ ՝ «Աս դղան էգաձին բէս ջէլլադ ընէք»։}

\armenian{Մանչը գարթննա, նա̈յի գը օր գՙխլուն վէրէվ ախչիյ մը գայնէր է ըմմա՜, քէզի մէգ հրըշդաք մը, օխդը բՙէրդՙ արէգՙագ աղվէօր։ Մանչուն խէլքը գՙխլէն թռի գը. ախչիգը գըսէ գը քի «Մի՛ գէնար, ձօցուդ թուխթը հօրս վէզիրին դար»։}

\armenian{Մանչը թուխթը գառնէ՝ վէզիրին դանի գը. վէզիրը նէյի գը օր թաքավօրին գՙիրն է, բաքնէ ջայդին դՙնէ, բՙանա գը գարթա օր՝ «Աս դղան էգաձին բէս ախչիգանս հեդ նիքյահ ընէք» գՙրէր է։}

\armenian{Վէզիրը թէզ մը հաշնիքը գը բՙռնէ. թէլլալ գանչէլ գուդա. քառսուն օր հաշնիք, քառսուն օր բՙաղնիք գընէն, մանչն ու ախչիգը բսագէն։ Անէօնք գը հասնին իրէնց մուրադին, դՙուք ալ հասնիք ձՙեր մուրադին։}

\armenian{Օր յ}̵\armenian{էղք, թաքավէօրը խաբար գը ղրգէ քի գՙուգՙամ գօր. դուն-դունօրթով դՙէմ գէրթամ. թաքավէօրը հա̈ռունէն նա̈յի գը օր էգօղնէրուն մէչ մէգը գա, ջանշցաձը չէ. գը հասնին, վէզիրին հարցընէ գը քի, «Ադ չէ ըմմա, ա՞ս օ}̂\armenian{վ է» գըսէ մանչուն համար։ Վէզիրը թուխթը հանէ՝ թաքավօրին ձՙա̈ռքը գուդա. «Թաքավէօ՛ր, աբրազ գէնաս, փէսադ է, գըսէ. ասանգ ասանգ գՙրէր էիր թէ էգաձին բէս ախչիգանս հեդ գարքէք. մէնք ալ հրամանդ դէղը դարանք»։}

\armenian{Թաքավէօրը հասգնա գը օր ան դղան է. ճէղ մը մյօրուքը}... 

\begin{adjarianpage}\label{page:231}\end{adjarianpage}% should be 231

... \armenian{ձՙա̈ռքը գառնէ, գՙլօխը գը փարդէ, ձՙան չի հանէր}. – \armenian{ի՞նչ ընէ, էդաձը էղէր է. «Իրավօր, գըսէ, գՙրվաձը չավրըվիր էղէր»։}

\armenian{յ}̵\armenian{էրգընուցը էրգու խնձօ}̂\armenian{ր յ}̵\armenian{ընգավ, մէգը ըսօղին, մէյն ալ լսօղին։}

\subsection{\armenian{Փէսա Ղազար}}

\armenian{Գնգանը մէգը Ղազար անունօվ դղա մու թէսա մը ունի էղէր։Աս գնիգը յ}̵\armenian{ամմէն օր ժամ գէրթա, «Ա՛սվազ, դՙուն Ղազարիս օղօրմիս» ըսէլէն ՝ սրդին ձէձէ գուլա գաղօթէ էղէր։ Դղան մէրախը մնա գը քի ըջաբ աս մարս օ}̂\armenian{ր Ղազարին համար գաղօթէ գօր. առդըվանց մը գանուխ քըզ մարն յ}̵\armenian{ա̈ռաչ գէլլէ ժամ գէրթա, սէղանին յ}̵\armenian{էդէվ բաբբըդի գը (կը պահվտի)։ Նէյիս մարը գՙուգՙա, «Ա՛սվաձ, դՙուն Ղազարիս օղօրմիա» ըսէլէն՝ գուլա գաղօթէ նէ, դղան սէղանին յ}̵\armenian{էդէվէն գամացէն մը՝ «օ}̂\armenian{ր Ղազարիգ օղօրմիմ» գըսէ։ Գնիգը գՙդէ քի Ասվազ ձՙանը լսէց։ «Փէսսա՛ Ղազարիս, փէսսա՛ Ղազարիս» գըսէ։ Դղան հասգնա գը օր փէսա Ղազարին համար գաղօթէ գօր էղէր, սէղանին յ}̵\armenian{ՙէդէվէն գէլլէ, աղվէօր մը մարը ձէձէ գը։}


\chapter{Evdokia}
\section{Background, literature, and subdialects}

\begin{adjarianpage}\label{page:232}\end{adjarianpage}% should be 232

The dialect of Evdokia is spoken primarily in the city of Evdokia or Tokat. It is spread until Amasia, Merzifon, Ordu, Samsun, and Sinop, and their surrounding villages. For the last three cities, their Armenian populations are still recent migrant settlements, so they cannot naturally have their own proper dialect. But because the majority of the migrant settlements came from Evdokia, thus we consider them as part of this dialect. 

The Evdokia dialect is studied by \citet{KazandjianBook} (Hovhannes Kazandjian) in a sufficiently extensive work. Besides this, he has an article on the study of this dialect, in \citeauthor{Byurakn} 1898, page 317. There are manuscripts with the Evdokia dialect in Kazandjian's work, page 5-8, 95, and so on. For the subdialects, there is only a text that is written in the Merzifon subdialect (\citeauthor{Byurakn} 1900, page 427) and some information on the Ordu subdialect (ibid., page 73). 

Near Evdokia, there is the village of Girgores, which speaks its own separate subdialect. 

\section{Phonology}

\subsection{Segment inventory}

The sound system of the Evdokia dialect has in total 31 sounds: vowels in Table \ref{tab:Evdokia:phono:segment:vowels} and consonants in Table \ref{tab:Evdokia:phono:segment:cons}. 




\begin{table}[H]
 \centering
 \caption{Vowels of the Evdokia dialect}
 \label{tab:Evdokia:phono:segment:vowels}
 \begin{tabular}{|ll l|}
 \hline 
/i/ <\armenian{ի}> & /u̯e/ <\armenian{ուէ}> & /u/ <\armenian{ու}> 
\\
/e/ <\armenian{է}> & /ə/ <\armenian{ը}> & /o/ <\armenian{օ}>
 \\
 & & /ɑ/ <\armenian{ա}> 
 \\ \hline 
 \end{tabular}
\end{table}


 

\begin{table}[H]
 \centering
 \caption{Consonants of the Evdokia dialect}
 \label{tab:Evdokia:phono:segment:cons}
 \begin{tabular}{|l|lll|llll|lll|}
 \hline 
 & \multicolumn{3}{l|}{Labial}& \multicolumn{4}{l|}{Coronal}& \multicolumn{3}{l|}{Dorsal/Back}\\
 Stops& /b/ & /pʰ/ & & /d/ & /tʰ/ & & & /ɡ/ & /kʰ/ & 
 \\
 & <\armenian{բ}> & <\armenian{փ}> &&<\armenian{դ}>& <\armenian{թ}>& && <\armenian{գ}>& <\armenian{ք}> & \\

 \hline 
 Affricates & && & /d͡z/ & /t͡sʰ/ & /d͡ʒ/ & /t͡ʃʰ/ && & \\
 & && &<\armenian{ձ}>& <\armenian{ց}> & <\armenian{ջ}>& <\armenian{չ}> & & & \\
 \hline 
 Fricatives& /f/&/v/& &/s/& /z/& /ʃ/& /ʒ/& /χ/ & /ʁ/ & /h/ \\
 & <\armenian{ֆ}>&<\armenian{վ}>& & <\armenian{ս}>& <\armenian{զ}>& <\armenian{շ}>& <\armenian{ժ}>& <\armenian{խ}> & <\armenian{ղ}> & <\armenian{հ}> 
\\ \hline 
 Sonorants & /m/ & /n/& & /ɾ/ & /r/& /l/ & /j/ && & \\
& <\armenian{մ}> & <\armenian{ն}> && <\armenian{ր}>& <\armenian{ռ}>& <\armenian{լ}>& <\armenian{յ}> && & 
\\ \hline 
 \end{tabular}
\end{table}



\subsection{Sound changes}

For the sound changes, the following are notable. 

\subsubsection{Monophthong vowel changes }
\paragraph{Classical Armenian /e/ <\armenian{ե}> }

The Classical sound /e/ <\armenian{ե}> becomes /je/ <\armenian{յէ}> at the beginning of monosyllabic words, while it becomes /e/ <\armenian{է}> everywhere else (Table \ref{tab:Evdokia:phonology:change:e}a). But see words in Table \ref{tab:Evdokia:phonology:change:e}b because they are monosyllabic. 





\begin{table}[H]
 	\centering 
 	\caption{Change from Classical Armenian /e/ <\armenian{ե}> to /je, e/ <\armenian{յէ, է}> in the Evdokia dialect}
 	\label{tab:Evdokia:phonology:change:e}
 	\begin{tabular}{|ll| ll|ll| ll|}
 		\hline && \multicolumn{2}{l|}{Classical Armenian} &\multicolumn{2}{l|}{> Evdokia} & \multicolumn{2}{l|}{cf. SEA} \\ 
 a. & \armenian{՝}I' & es & \armenian{ես} & jes & \armenian{յէս} & jes & \armenian{ես} \\
 & \armenian{՝}two' & eɾku & \armenian{երկու} & eɾɡu & \armenian{էրգու} & jeɾku & \armenian{երկու} \\
& `to cook' & epʰel & \armenian{եփել} & epʰel & \armenian{էփէլ} & jepʰel & \armenian{եփել} \\
 & `to rise' & elɑnel & \armenian{ելանել} & ellel & \armenian{էլլէլ} & jellel & \armenian{ելլել} \\ 
b. \armenian{՝}when' & eɾb & \armenian{երբ} & jepʰ & \armenian{յէփ} & jeɾpʰ & \armenian{երբ} \\
 `rise! ({\imp}.2{\sg})' & el & \armenian{ե՛լ} & jel & \armenian{յէ՛լ} & jel & \armenian{ե՛լ} \\ 

\hline 
 	\end{tabular}
\end{table}


\paragraph{Classical Armenian /o/ <\armenian{ո}> }

 The Classical sound /o/ <\armenian{ո}>, in both word-initial and word-medial positions, often... 



\begin{adjarianpage}\label{page:233}\end{adjarianpage}% should be 233

... gets pronounced as a diphthong /u̯e/ <\armenian{ուէ}> (Table \ref{tab:Evdokia:phonology:change:o}). 



\begin{table}[H]
 	\centering 
 	\caption{Change from Classical Armenian /o/ <\armenian{ո}> to /u̯e/ <\armenian{ուէ}> in the Evdokia dialect}
 	\label{tab:Evdokia:phonology:change:o}
 	\begin{tabular}{|l | ll|ll| ll|}
 		\hline & \multicolumn{2}{l|}{Classical Armenian} &\multicolumn{2}{l|}{> Evdokia} & \multicolumn{2}{l|}{cf. SEA} \\ 
`onion' & soχ & \armenian{սոխ} & su̯e & \armenian{սուէ} & soχ & \armenian{սոխ} \\
 `horse-radish' & boɬk & \armenian{բողկ} & pʰu̯eʁkʰ & \armenian{փուէղք} & boχk & \armenian{բողկ} \\ 
 `horse-radish' & boɬk & \armenian{բողկ} & hu̯ed & \armenian{հուէդ} & boχk & \armenian{բողկ} \\ 
 \armenian{՝}orphan' & oɾb & \armenian{որբ} & vu̯eɾpʰ & \armenian{վուէրփ} & voɾpʰ & \armenian{որբ} \\
`male' &oɾd͡z & \armenian{որձ} & u̯eɾt͡sʰ & \armenian{ուէրց} &voɾt͡sʰ & \armenian{որձ} \\
 		`lentil' & ospən & \armenian{ոսպն} & vu̯esb & \armenian{ուէսբ} & vosp & \armenian{ոսպ} \\ 
`to try' & ɡoɾt͡sel & \armenian{գործել} & ɡu̯eɾd͡zel & \armenian{գուէրձէլ} & ɡoɾt͡sel & \armenian{գործել} \\ 
\hline 
 	\end{tabular}
\end{table}


\subsubsection{Diphthongal vowel changes }


\paragraph{Classical Armenian /ɑi̯/ <\armenian{այ}> }


The Classical diphthong /ɑi̯/ <\armenian{այ}> becomes /ɑ/ <\armenian{ա}> (Table \ref{tab:Evdokia:phonology:change:aj}). 



\begin{table}[H]
 	\centering 
 	\caption{Change from Classical Armenian /ɑi̯/ <\armenian{այ}> to /ɑ/ <\armenian{ա}> in the Evdokia dialect}
 	\label{tab:Evdokia:phonology:change:aj}
 	\begin{tabular}{|l | ll|ll| ll|}
 		\hline & \multicolumn{2}{l|}{Classical Armenian} &\multicolumn{2}{l|}{> Evdokia} & \multicolumn{2}{l|}{cf. SEA} \\ 
proximal `this' & ɑi̯s & \armenian{այս} & ɑs & \armenian{աս} & ɑjs & \armenian{այս} \\
 `other' &ɑi̯l& \armenian{այլ} & ɑl & \armenian{ալ} &ɑjl& \armenian{այլ} \\
\hline 
 	\end{tabular}
\end{table}


\paragraph{Classical Armenian /o̯i/ <\armenian{ոյ}> }


The Classical diphthong /o̯i/ <\armenian{ոյ}> becomes /u/ <\armenian{ու}> (Table \ref{tab:Evdokia:phonology:change:oj}). 



\begin{table}[H]
 	\centering 
 	\caption{Change from Classical Armenian /o̯i/ <\armenian{ոյ}> to /u/ <\armenian{ու}> in the Evdokia dialect}
 	\label{tab:Evdokia:phonology:change:oj}
 	\begin{tabular}{|l | ll|ll| ll|}
 		\hline & \multicolumn{2}{l|}{Classical Armenian} &\multicolumn{2}{l|}{> Evdokia} & \multicolumn{2}{l|}{cf. SEA} \\ 
`light' & loi̯s & \armenian{լոյս} & lus & \armenian{լուս} & lujs & \armenian{լույս} \\ 
\hline 
 	\end{tabular}
\end{table}


\paragraph{Classical Armenian /iu̯/ <\armenian{իւ}> }


The Classical diphthong /iu̯/ <\armenian{իւ}> becomes /u/ <\armenian{ու}> (Table \ref{tab:Evdokia:phonology:change:iu}). 



\begin{table}[H]
 	\centering 
 	\caption{Change from Classical Armenian /iu̯/ <\armenian{իւ}> to /u/ <\armenian{ու}> in the Evdokia dialect}
 	\label{tab:Evdokia:phonology:change:iu}
 	\begin{tabular}{|l | ll|ll| ll|}
 		\hline & \multicolumn{2}{l|}{Classical Armenian} &\multicolumn{2}{l|}{> Evdokia} & \multicolumn{2}{l|}{cf. SEA} \\ 
 \armenian{՝}snow' & d͡ziu̯n & \armenian{ձիւն}& d͡zun & \armenian{ձուն} & d͡zjun & \armenian{ձյուն} \\
\hline 
 	\end{tabular}
\end{table}



\subsubsection{Consonant changes}

\paragraph{Laryngeal changes}

The consonant changes are very big. Like the dialects of Trabzon, Istanbul, Smyrna, and Crimea, the dialect of Evdokia has changed the three degrees of consonants from Old Armenian into two; the voiceless unaspirates are lost, only the voiced and the voiceless aspirates are preserved. There are no voiced aspirates. Based on this, the Armenian voiced and voiceless unaspirate sounds have equally changed into voiced, while the voiceless aspirates remain the same. 

\paragraph{Consonant deletion around sonorants} 


Dentals that are before the Classical sound /ɾ/ <\armenian{ր}> are lost, while the following /ɾ/ <\armenian{ր}> becomes /r/ <\armenian{ռ}> (Table \ref{tab:Evdokia:phonology:change:dentalR}). 



\begin{table}[H]
 	\centering 
 	\caption{Loss of dentals before Classical Armenian /ɾ/ <\armenian{ր}> and subsequent trilling in the Evdokia dialect}
 	\label{tab:Evdokia:phonology:change:dentalR}
 	\begin{tabular}{|l | ll|ll| ll|}
 		\hline & \multicolumn{2}{l|}{Classical Armenian} &\multicolumn{2}{l|}{> Evdokia} & \multicolumn{2}{l|}{cf. SEA} \\ 
`to divide' &kətɾel & \armenian{կտրել} & ɡərel & \armenian{գըռէլ} & kətɾel & \armenian{կտրել} \\
`to break' &kotoɾel & \armenian{կոտորել} & ɡorel & \armenian{գօռէլ} & kotoɾel, kotɾel & \armenian{կոտորել, կոտրել} \\
\hline 
 	\end{tabular}
\end{table}


If there is a /n/ <\armenian{ն}> before the dentals, then it is also lost (Table \ref{tab:Evdokia:phonology:change:NasdentalR}). 



\begin{table}[H]
 	\centering 
 	\caption{Loss of nasal /n/ before Classical Armenian dental plus /ɾ/ <\armenian{ր}> in the Evdokia dialect}
 	\label{tab:Evdokia:phonology:change:NasdentalR}
 	\begin{tabular}{|l | ll|ll| ll|}
 		\hline & \multicolumn{2}{l|}{Classical Armenian} &\multicolumn{2}{l|}{> Evdokia} & \multicolumn{2}{l|}{cf. SEA} \\ 
`to divide' &kətɾel & \armenian{կտրել} & ɡərel & \armenian{գըռէլ} & kətɾel & \armenian{կտրել} \\
`to break' &kotoɾel & \armenian{կոտորել} & ɡorel & \armenian{գօռէլ} & kotoɾel, kotɾel & \armenian{կոտորել, կոտրել} \\
 				`comb' & sɑntəɾ & \armenian{սանտր} & sɑr & \armenian{սառ} & sɑnəɾ & \armenian{սանր} \\ 
\hline 
 	\end{tabular}
\end{table}



In accordance with the latter, the Armenian word-final sequence /nɾ/ <\armenian{նր}>, which became /ndɾ/ <\armenian{նդր}> via the addition of a dental, is simply /r/ <\armenian{ռ}> in the Evdokia dialect (Table \ref{tab:Evdokia:phonology:change:cluter}). 



\begin{table}[H]
 	\centering 
 	\caption{Cluster reduction of nasal-dental-rhotic to a trill in the Evdokia dialect}
 	\label{tab:Evdokia:phonology:change:cluter}
 	\begin{tabular}{|l | ll|ll| ll|}
 		\hline & \multicolumn{2}{l|}{Classical Armenian} &\multicolumn{2}{l|}{> Evdokia} & \multicolumn{2}{l|}{cf. SEA} \\ 
`heavy' & t͡sɑnəɾ & \armenian{ծանր} & d͡zɑr & \armenian{ձառ} & t͡sɑnəɾ & \armenian{ծանր} \\
`small' & mɑnəɾ & \armenian{մանր} & & & mɑnəɾ & \armenian{մանր} \\
`small (reduplicated)' & & & mɑr mur & \armenian{մառ մուռ} & mɑnəɾ munəɾ & \armenian{մանր մունր} \\
\hline 
 	\end{tabular}
\end{table}


\section{Morphology}

The grammar of the Evdokia dialect does not have innovations, and it is entirely in agreement with the Istanbul dialect and with the literary Western language. There are only a few rare differences. 

\subsection{Verb inflection or conjugation}
\subsubsection{Theme vowel changes}

In the 1SG and 1PL of verbs, the rime /e/ <\armenian{ե}> becomes /i/ <\armenian{ի}>, while the other persons do not change it. 



 \translatorHD{To clarify, he's talking about theme vowels in verbs, before agreement suffixes. He provides examples from the indicative present. In SWA, the indicative present is formed by adding the indicative prefix /ɡ(ə)-/ before the finite verb. The theme vowel /e/ stays constant in the indicative present. In Evdokia, the theme vowel /e/ is replaced by /i/ in the 1SG and 1PL, before nasal suffixes. }


\begin{table}[H]
 \centering
 \caption{Theme vowel changes in the indicative present <\armenian{ներկայ}> of the verb `to like' in the Evdokia dialect}
\label{tab:Evdokia:morpho:verb:paradigm:presentPastIndc}
 \begin{tabular}{|l|ll|ll|}
 \hline & \multicolumn{2}{l|}{Evdokia} & \multicolumn{2}{l|}{cf. SWA} \\ \hline 
 
1SG & ɡə siɾ-i-m & \armenian{գը սիրիմ} & ɡə siɾ-e-m & \armenian{կը սիրեմ} \\
2SG & ɡə siɾ-e-s &\armenian{գը սիրէս} & ɡə siɾ-e-s & \armenian{կը սիրես} \\
3SG & ɡə siɾ-e-$\emptyset$ & \armenian{գը սիրէ} & ɡə siɾ-e-$\emptyset$ & \armenian{կը սիրէ} \\
1PL & ɡə siɾ-i-nkʰ & \armenian{գը սիրինք} & ɡə siɾ-e-ŋkʰ & \armenian{կը սիրենք} \\
2PL & ɡə siɾ-e-kʰ & \armenian{գը սիրէք} & ɡə siɾ-e-kʰ & \armenian{կը սիրէք} \\
3PL & ɡə siɾ-e-n& \armenian{գը սիրէն} & ɡə siɾ-e-n & \armenian{կը սիրեն} \\
 		& \multicolumn{2}{l|}{{\ind} $\sqrt{}$-{\thgloss}-{\agr}} & \multicolumn{2}{l|}{{\ind} $\sqrt{}$-{\thgloss}-{\agr}} \\ 
 \hline 
\end{tabular}
\end{table}

\subsubsection{Progressive marker /ɡoɾ/ <\armenian{կոր}> }
The Evdokia dialect has a progressive present and imperfective, which are formed with (\translatorHD{the cognate of}) the formative \armenian{կոր} (\translatorHD{SWA: /ɡoɾ/}), as in the Sebastia and Istanbul dialects (\ref{sent:Evdokia:morpho:verb:prog}). 

\translatorHD{To clarify, in SWA, the progressive marker /ɡoɾ/ <\armenian{կոր}> is added after the indicative present or indicative past imperfective to give them a progressive meaning. }

\begin{exe}
 \ex\label{sent:Evdokia:morpho:verb:prog} \begin{xlist}
 \ex Evdokia
 \begin{xlist}
 \ex \gll ɡ-ud-i-m, ɡ-ud-i-m ɡoɾ \\
 {\ind}-eat-{\thgloss}-1{\sg}, {\ind}-eat-{\thgloss}-1{\sg} {\prog} \\
 \trans `I eat; I am eating.' \\
 \armenian{գուդիմ, գուդիմ գօր}
 \ex \gll ɡə beɾ-ej-i-$\emptyset$, ɡə beɾ-ej-i-$\emptyset$ ɡoɾ \\
 {\ind} bring-{\thgloss}-{\pst}-1{\sg}, {\ind} bring-{\thgloss}-{\pst}-1{\sg} {\prog} \\
 \trans `I would bring; I was bringing.' \\
 \armenian{գը բէրէյի, գը բէրէյի գօր} 
 \end{xlist}
 \ex cf. SWA
 \begin{xlist}
 \ex \gll ɡ-ud-e-m, ɡ-ud-e-m ɡoɾ \\
 {\ind}-eat-{\thgloss}-1{\sg}, {\ind}-eat-{\thgloss}-1{\sg} {\prog} \\
 \trans `I eat; I am eating.' \\
 \armenian{կ՚ուտեմ, կ՚ուտեմ կոր} 
 \ex \gll ɡə pʰeɾ-ej-i-$\emptyset$, ɡə pʰeɾ-ej-i-$\emptyset$ ɡoɾ \\
 {\ind} bring-{\thgloss}-{\pst}-1{\sg}, {\ind} bring-{\thgloss}-{\pst}-1{\sg} {\prog} \\
 \trans `I would bring; I was bringing.' \\
 \armenian{կը բերէի, կը բերէի կոր} 
 \end{xlist}
 
 \end{xlist}
\end{exe}


It is thought that the aforementioned formative <\armenian{կոր}> (\translatorHD{SWA: /ɡoɾ/}) originates from the synonymous Turkish form <yor>...

\begin{adjarianpage}\label{page:234}\end{adjarianpage}% should be 234


Compare <getiri-yor-əm> `I am bringing' (\translatorHD{Modern Turkish spelling <getiɾi-yor-um>}), <getiri-yor-ədəm> `I was bringing' (\translatorHD{Modern Turkish spelling <getiɾi-yor-dum>}). 

\translatorHD{Note that although Adjarian treats this progressive marker as borrowed from Turkish, it might also have a language-internal or native source \citep{donabedian-2001-tabouLinguisticArmenianOccidentalGorProgressive}. }



In Amasia and Merzifon, instead of /koɾ/, the formative /ɡɑ/ <\armenian{գա}> is used (\ref{sent:Evdokia:morpho:verb:prog:sub}).

\begin{exe}
 \ex Amasia and Merzifon (Evdokia) \label{sent:Evdokia:morpho:verb:prog:sub} 
\begin{xlist}
 \ex \gll ɡɾ-e-m ɡɑ \\
 write-{\thgloss}-1{\sg} {\prog} \\
 \trans `I am writing.'\\
 \armenian{գրէմ գա}
 \ex \gll ɡ-eɾt-ɑ-m ɡɑ \\
 {\ind}-go-{\thgloss}-1{\sg} {\prog} \\
 \trans `I am going.'\\
 \armenian{գէրթամ գա}

\end{xlist}
\end{exe}

\subsubsection{Future marking with /bidi/ <\armenian{բիդի}> }
The future is formed with the form /bidi/ <\armenian{բիդի}>, which becomes /bid/ <\armenian{բիդ}> when next to a vowel. In the latter condition, the Ordu subdialect uses the simple form /b/ <\armenian{բ}> (\ref{sent:Evdokia:morpho:verb:fut}). 


\translatorHD{To clarify, in SWA, the future is formed by adding the proclitic /bidi/ <\armenian{պիտի}> before the finite present-form of the verb. }




\begin{exe}
 \ex\label{sent:Evdokia:morpho:verb:fut} \begin{xlist}
 \ex Ordu (Evdokia)
\gll b-eɾtʰ-ɑ-m \\
 {\fut}-go-{\thgloss}-1{\sg} \\
 \trans `I will go.'\\
 \armenian{բէրթամ}
 \ex cf. SWA
 \begin{xlist}
 \ex \gll bidi jeɾtʰ-ɑ-m \\
 {\fut} go-{\thgloss}-1{\sg} \\
 \trans `I will go.'\\
 \armenian{պիտի երթամ}
 
 \end{xlist}
 
 \end{xlist}
\end{exe}
\subsubsection{Interrogative marking with /mə/ <\armenian{մը}>}

\translatorHD{In written or formal SWA, there is no special morphology used for interrogatives or question. The only difference between a declarative statement (\ref{sent:Evdokia:morpho:verb:q:SWA:dec}) vs. an interrogative yes-no question (\ref{sent:Evdokia:morpho:verb:q:SWA:q}) is the use of a final-rise in the question. But colloquial or spoken SWA borrowed the Turkish interrogative particle <mi> as /mə/ and can optionally add it to a yes-no question (\ref{sent:Evdokia:morpho:verb:q:SWA:me}). }

\begin{exe}
 \ex SWA (formal and informal) \label{sent:Evdokia:morpho:verb:q:SWA}
 \begin{xlist}
 \ex \gll ɑn nɑmɑɡ-neɾ un-i-$\emptyset$ $\searrow$\\
 he letter-{\pl} have-{\thgloss}-3{\sg} \\
 \trans `He has letters.' \label{sent:Evdokia:morpho:verb:q:SWA:dec} \\
 \armenian{Ան նամակներ ունի։}
 \ex \gll ɑn nɑmɑɡ-neɾ un-i-$\emptyset$ $\nearrow$\\
 he letter-{\pl} have-{\thgloss}-3{\sg} \\
 \trans `Does he have letters?' \label{sent:Evdokia:morpho:verb:q:SWA:q}\\
 \armenian{Ան նամակներ ունի՞։}
 \ex \gll ɑn nɑmɑɡ-neɾ un-i-$\emptyset$ mə$\nearrow$\\
 he letter-{\pl} have-{\thgloss}-3{\sg} {\q}\\
 \trans `Does he have letters?' \label{sent:Evdokia:morpho:verb:q:SWA:me} \\
 \armenian{Ան նամակներ ունի՞ մը։}
 \end{xlist}
\end{exe}

\translatorHD{As Adjarian explains, Evdokia follows colloquial SWA in having a question particle. }


Interrogative verbs takes the formative /mə/ <\armenian{մը}>, which is borrowed from the Turkish form /mi, mə/ <\armenian{մի, մը}> (\ref{sent:Evdokia:morpho:verb:q:ev}).

\begin{exe}
\ex\label{sent:Evdokia:morpho:verb:q:ev} \begin{xlist}
\ex Evdokia
\begin{xlist}
 \ex \gll ɡu-d-ɑ-s mə \\
 {\ind}-give-{\thgloss}-2{\sg} {\q} \\
 \trans `Will/do you give it?' \\
 \armenian{գուդա՞ս մը} 
\ex \gll ɡ-ɑr-n-e-$\emptyset$ mə \\
 {\ind}-take-{\vx}-{\thgloss}-3{\sg} {\q} \\
 \trans `Will/does he take? \\
 \armenian{գառնէ՞ մը}


\end{xlist}
\ex cf. Turkish
\begin{xlist}
 \ex \gll ver-ir mi-sin \\
 give-{\aor} {\q}-2{\sg}\\
 \trans `Will/do you give it?' \\
 \translatorHD{Adjarian provides an Ottoman version: /veɾiɾ mi sin/ <\armenian{վէրի՞ր մի սին}>}
\ex \gll al-ır mı \\
take-{\aor} {\q} \\
\trans `Will/does he take? \\
 \translatorHD{Adjarian provides an Ottoman version: /ɑləɾ mə/ <\armenian{ալը՞ր մը}> }
\end{xlist}
\end{xlist}
\end{exe}



In this same condition, the Istanbul dialect uses /mi/ <\armenian{մի}>. 


\begin{exe}
\ex Istanbul \label{sent:Evdokia:morpho:verb:q:istanbul} 
 
\begin{xlist}
 \ex \gll ɡu-d-ɑ-s mi \\
 {\ind}-give-{\thgloss}-2{\sg} {\q} \\
 \trans `Will/do you give it?' \\
 \armenian{գուդա՞ս մի}
\ex \gll ɡ-ɑr-n-e-$\emptyset$ mi \\
 {\ind}-take-{\vx}-{\thgloss}-3{\sg} {\q} \\
 \trans `Will/does he take? \\
 \armenian{գառնէ՞ մի}

\end{xlist}
\end{exe}

\section{Text samples}

{\sampleoverview}

\subsection{Evdokia}

Adjarian's source: This story was communicated to me by a resident of Evdokia, Mr. Hovhannes Kazandjian (\armenian{պր. Յովհ. Գազանճեանի}; \translatorHD{SEA: /hovhɑnnes ɡɑzɑnt͡ʃjɑn/, SWA: /kʰɑzɑnd͡ʒjɑn/}), an ardent follower of Armenian dialectology, in his one extensive letter (October 8, 1897, Evdokia). The orthography is with scientific accuracy. 


\armenian{Վախթին-ժամանագին էրիգ-գնիգ մը գան էղէր, ութ-դասը դարվան ախչիգ մը, էրգու-իրէք դարվան ալ մանչ մը ունին էղէր։ Ասուէնք շադ ախքադ, օրէ բանօղ, օրէ ուդօղ մարթիգ էն էղէր։ Էրիգը դուրսը ըռղըդութին գանէ, գնիգն ալ դունը դէզգէհ գը գուէրձէ էղէր. ասանգօվ էրիգ-գնիգ վասդըդաձօվնին անջահ նէղ-նըվազ, ցամաք-հաց, գըձու-սուէխ աբրուսդ մը գը ջարէն էղէր։}

\armenian{Իրինգվանը մէգը էրիգը բանէն էլլէլօվ դուն քալու ադէնը գը նէլի քի չարսուն աղվուրիգ նախշունիգ հավ մը գը ձախէն գօր։}

\armenian{Մարթը հավուն աղվուրգութանը գը հավասի, մըդքէն գըսէ քի՝ յէս աս իրինգուն դուն հաց չէմ դանիր. թէք աս իրինգուն անօթի գը գէնանք, իլլէ սի հավը գառնիմ։ Ասանգ ըսէլօվ ան օրվան առաձ օրչէքը գուդա, հավը գառնէ, դուն գը դանի, օդային ըռաֆիգը գը դընէ, գէր գը թափէ էռչէվը։ Նաշխունիգ հավը գըդ-գըդ գըդ-դըդ անէլօվ գէրը գուդէ, ըռաֆին վրա գը բըդըդի։ Գնիգը գըսէ քի «Քա՛, աս հավը ինչո՞ւ առիր»}. – \armenian{«Իշթէ բան մըն էր արի. աղւորգութանը հավասէցա դէ առի», գըսէ էրիգը։ Ի՛նչ է նէ՝ ան իրինգունը էռչի օրվան հացի էվէլցուք գըդըրդուքնէրօվ էօյիւն գանցունէն, անգի վէրչը դղաքը հավը գը սիրէն, անգի ալ գը}... 


\begin{adjarianpage}\label{page:235}\end{adjarianpage}% should be 235



... \armenian{բառգին գը քնանան։ Մէ մըն ալ քիշէրը գարթննան քի օդան լուսավօրվէր է. աս ի՞նչ էջէյիբ լուս է՝ ըսէլօվ՝ գէլլէն գը նէյին քի հավը հավգիթ մըն է աձէր, ա՛նդար ջէրմագ, ա՛նդար փառլախ հավգիթ մը քի ՝էլմասի բէս փառ-փառ գը վառի, օդան լուսավօրէ գոր։ Էրիգ-գնիգ շադ գը զարմանան քի աս ի՛նչ թէվիւր հավգիթ է։ Անգի վէրչ ամմէն օր հավը ադանգ մէ մէգ հավգիթ գաձէ։ Ավուրը մէգը մարթը գըսէ քի «Գնիգ, էգու սի հավգիթնէրէն քանի մը հադ չարշուն դանիմ դէ ձախիմ. բէքի քանի մը փարա բըռնէ»։ Գնիգն ալ «դար դէ ձախէ» գըսէ։ Էրիգը գառնէ քանի մը հավգիթ, չարշուն դանէլու ադէնը ղույումջի մը գը դէսնէ, հէռույէն գը գանչէ զինքը։ Մարթը գէրթա ղույումջուն քօվ. ղույումջին գըսէ քի՝ «ադ հավգիթնէրը ձախէ՞ս մը գօր»։ – «Հա, ձախիմ գօր. քանի՞ փարա գուդաս», գըսէ մարթը։ Ղույումջին հավգիթը ձէռքը գառնէ, գը նէյի քի խալիս էլմաս է. մարթուն գըսէ քի՝ «հազար ղուրուշի գուդա՞ս մը»։ Մարթը ի՛նչ քիդէ հավգիթէն էլմաս ըննալը, գըսէ քի «Ա՛խբար, ընձի զէքլէնմի՞շ մը գանէս գօր»}. – \armenian{«Չէ՛, ի՞նչ զէքլէնմէ է. քիչ է նէ էրգու հազար դամ»։ – «Ախբար, ի՞նչ գըսէս գոր, ընձի զէքլէնմի՞շ մը գանէս գօր»։ «Է՛, իրէք հազար դամ անանգ է նէ»։}

\armenian{Մարթը գը մդաձէ քի էջէբ ղույումջին իրա՞վ մը գըսէ գօր՝ շախա՞ մը. հէմէն գըսէ քի «դո՛ւր փարան»։ Ղույումջին իրէք հազարը գը հանէ գուդա, հավգիթը գառնէ։ «Աս հավգիթէն դահա գա՞ մը» գըսէ։ – Հաբա՛, գա։ – «Անանգ է նէ՝ ի՛նչդար ունիս նէ ընձի բէր. յէս հադը իրէք հազարագանի գառնիմ»։}

\armenian{Մարթը խնդումէն ձաղիգը բառէլօվ (իմա ծաղիկը պատռիլ «չափազանց ուրախանալ») դուն գէրթա. «Գնիգ, մէնք էհյալըխը գդանք» ՝ ըսէլօվ գնգանն ալ բանը գիմացնէ. գնիգն ալ շադ գուրախանա։ Ալ գը հասգընան քի հավէրնին էլմաս աձօղ հավ է էղէր. ալ անգի վէրչը էրիգը ըռղըդութինը, գնիգն ալ դէզգէհ գուէրձիլը վար գը ցքէ. հավուն հավգիթնէրը ձախէլօվ գուդէն գը խմէն, գյանք-գէնթանութին ժամանագ գանցունէն։ Վախիթ վէրչը ղօնախի բէս սիւսլիւ դուն մըն ալ շինէլ գուդան, մէչը գը նսդին։}

\armenian{Զադգըվան օր մը քախքին վարթաբէդը փօքրավօրին հէտ մէգդէղ դուն-օրհնէնքի գուքա մարթուն դունը։ Օրհնէլու ադէննին փօքրավօրը ըռաֆին վրայի հավը գը դէսնէ, գը նէյի քի հավուն ղանադին վրա գիր գա. գը գարթա քի սըվիսանգ գրված}... 


\begin{adjarianpage}\label{page:236}\end{adjarianpage}% should be 236

... \armenian{է. «Աս հավուն ուէդքն ուդօղը վազան (լաւ վազօղ) գըննա, սիրդն ուդօղըը իմասդուն գըննա, գլօվն ուդօղը թաքավուէր գըննա»։ Փօքրավօրը գը մդաձէ քի սի հավը ինչբէ՞ս անիմ դէ ձէռք ցքիմ։ Մէր (թրք. մէգէր) մարթուն գնիգն ալ անթիէն փօքրավօրին գէնջութանը զէրնըվէր է (սիրահարուիլ}). \armenian{աչքօվ ունքօվ նիշաննէր գանէ էղէր. փօքրավօրը բանը հասգնալօվ գամաց մը գնգանը քօվը գէրթա. գնիգը գամացուգ մը գըսէ քի «Վաղը մէզի էգու». փօքրավօրն ալ գըսէ քի «Էղէր (եթէ) դի (այդ) հավը գը մօրթէս գէթէս նէ գուքամ». «Փէք աղէգ, գը մօրթիմ, գէփիմ» գըսէ գնիգը։ Էրթէսի վանգուցը (առաւօտ) էրգանը էրթալէն վէրչը գնիգը գը բռնէ հավը գը մօրթէ, թէնգիրէն գը դընէ գէփէ։ Մէմն ալ փօքրավօրը գուքա. գնիգը դղաքը գլխէն ջամփէլու համար՝ ախջիգանը գըսէ քի «սի ախբարըդդ ա՛ռ դէ բըդըդցուր»։ Ախջիգն ալ ախբարը գըրգաձ օդայէն դուրս գէլլէ, դանը մէչ վէր վար դառնալու ադէնը օջախին գլօխը գէրթա. դղան օջխին վրայի թէնգիրէն դէսնէլօվ գը նէղէ քուրը քի անգից բան դա դէ ուդէ. ախջիգն ալ թէնգիրէն գը բանա, հավուն գլօխը ախբօրը գը գէրցնէ, սիրդը ինէն ուէդքէրն ալ ինքը գուդէ. մէ մն ալ վրան իմասդութին քալօվ գը մդաձէ քի «յէս ինչո՞ւ սա հավէն գէրա. հիմա մարըս գը հէրսօդի, զիս գը ձէձէ». ըսէլօվ մօրը վախուն՝ դղան գիրդը՝ դունէն դուրս գընգնի, վազէլօվ գը փախչի։ Ախջիգը հավուն սիրդը ինէն ուէդքը ուդէլօվ հէմ իմասդուն էղէր էր, հէմ վազան։ Գը վազէ գը վազէ, շադ շադ դէղ վազէլէն էրթալէն վէրչը՝ լէոթը գը բառի (լեարդը պատռիլ «չափազանց յօգնիլ») գը մնա. գը նէյի քի մէյդան դէղ մը ղալաբալըխ մը գա, մէգ ձաք (ձագ ՝ Եւդոկիոյ բարբառով կը նշանակէ «թռչուն») մը թռցնէն գօր. ի՞նչ է դէյի քօվէրնին գէրթա. մէ մըն ալ ձաքը գուքա գըրդի ախրօրը գլխուն գը նսդի։ Մարթիգը ասի չէղավ, ասի չէղավ ըսէլօվ ձաքը գը բըռնէն, նօրէն գը թռցնէն, նօրէն գուքա դղուն գլխուն գը նսդի. «նօրէն չէղավ, նօրէն չէղավ» ըսէլօվ նօրէն ձաքը գը թռցնէն, գինէ գուքա դղուն գլխուն ղօնմիշ գանէ։ Մարթիգը գը նէյին քի ըննալիք չունի, «Էյ, թաքավուէրնիս զահիր ասի է էղէր» ըսէլօվ՝ դղան ախջիգանը հէդ գառնէն իրէնց քաղաքը սէրայը գը դանին, թաքավուէր գը նսդէցնէն։ Մէր ադ քախքին թաքավուէրը մէռաձ է էղէր, ադ ձաքն ալ դէօվլէթ ղուշի է էղէր քի վօրու գլօխ նսդի նէ՝ անի}... 


\begin{adjarianpage}\label{page:237}\end{adjarianpage}% should be 237


... \armenian{թաքավուէր ըննա- ադ քախքին էդէթը ադանգ է էղէր։ Ադ բըդիդիգ դղան թաքավուէրութին գայնա՞ մը անէր. ըմմա քուրը իմասդութինօվը ախբօրը դէղը թաքավուէրութինը գանէ, անդար աղէգ գանէ գի վախիթ անցնէլօվ աս ախջգան իմասդութինը ամմէն թարաֆ շան գուդա, մէնձ անուն գը հանէ։}

\armenian{Քանք հիմա հօրը մօրը։}

\armenian{Հարը ան իրինգունը դունը գուքա քի հավը չի գա. «Գնիգ, հավը վո՞ւր է», գըսէ նէ, «Ի՞նչ գիդնամ» գըսէ գնիգը». մէ մն ալ մարթը գը նայի քի դղաքն ալ չի գան. «Քա դղաքը վո՞ւր էն», գըսէ նէ՝ գնիգը անօր ալ «չիմ գիդէր» գըսէ։ Մարթը խէվի բէս սօխգընէրը գընգնի դղա փնդռէլու, հավ փնդռէլու, ըմմա նէ՛ դղա գը գըդնէ, նէ՝ հավ։ Մարթը շադ մէրաք գանէ, քիշէր ցօրէգ օ՛ֆ փի՜ւֆ անէլօվ մդաձէլ գէնալն իքէն՝ մէ մն ալ գիմանա քի հէռու քաղաք մը իմասդուն ախջիգ մը գա էղէր, թաքավուէրի քուր, ամմէն բան գիդէ էղէր, ի՞նչ հարցընէս նէ ջուղաբը գուդա էղէր։ Մարթը քանի մը հավգիթ ձօցը դնէլօվ՝ գէլլէ ադ ախջիգը փնդռէլու գէրթա. գըսէ քի «Էրթամ ադ ախջիգանը սի հավգիթնէրը հէդիյէ դամ, ցավըս բադմիմ, բէքի ընձի ջար մը գը ցըցընէ»։ Ասանգ մդաձէլօվ շադ ջամփա էրթալէն վէրչը ադ քաղաքը գը հասնի, սէրայը գէրթա։ Ախջիգը ախբօրը հէդ նսդաձ դէղը գը հասգընա քի հարը գուքա գօր, ախբօրը խաբար գուդա. էմիր գանէն, նէրս գուքա. հարէրնին խօրաթձընէլէն վէրչը՝ «մէնք քու փնդռաձ զավգընէրդ էնք» ըսէլօվ գէրթան վիղը գը բըլլըվին. հարն ալ խնդումէն լալ գը բաշլըյէ։ Էն վէրչը իրարու հէրսէթ առնէլէն յէդքը՝ հօրէրնուն գըսէն քի «Գնա դուն, դնօվ դէղօվ աս քաղաքը մէր քօվն էգու, ֆէս (հոս) գէնանք»։ Մարթն ալ գէրթա, դուն դէղ գը ձախէ գը ձախվօրի, գընգանը հէդ գէլլէ դղօցը քօվ գուքա։ Ան վախթը ախջիգը մօրը գըսէ քի «Մա՛րիգ, հավը ի՛նչ էղավ, մէզի բիդ՚ ըսէս». անի ալ գըսէ քի «Ի՛նչ գիդնամ, գօրավ»։ Ան ադէնը ախջիգը հօրը էռչէվը մէգիգ մէգիգ գը բադմէ մօրը արուրքը («արարք»}), \armenian{հավը մօրթէլը, փօքրավօրին քալը, հավուն գլօխը, սիրդը, ուէդքէրը ուդէլնին, վախէրնուն դնէն փախչէլնին, մինչի թաքավուէր ըննալնին։}

\armenian{Մարը ասուէնք լսաձին բէս գաս-գաբուդ գըննա գընգնի գը մէռնի. դղաքը հարէրնին մէնձ բադիվնէրօվ գը բահէն, օրէրնին էրչանիգ գանցընէն։}


\begin{adjarianpage}\label{page:238}\end{adjarianpage}% should be 238

\subsection{Merzifon subdialect}

Adjarian's source: See \citeauthor{Byurakn} 1900, page 427. 

\armenian{Իրիգունը դրի մաղ մը հավգիթ՝ վանգուց չգար հիչ մէդ հադիգ (աստղեր)։}

\armenian{Ինքը սէվ, էրէսը ջէրմագ (սուրճ)։}

\armenian{Էվէլ բարին աչք չի հանիր։}

\armenian{Էշը չգէրաձ խօդը ուդէ նէ՝ փօրը գը ցավի։}

\armenian{Ցօրէգին ձութ գը ձամէ՝ քիշէրը ձէթ գը վառէ։}

\armenian{Էրէսին զէօրէ սիլլէ գը զարնէն։}

\armenian{Ձանրը նսդիր օր լըռ գաս։}

\chapter{Smyrna}
\section{Background}

\begin{adjarianpage}\label{page:239}\end{adjarianpage}% should be 239

Beyond the region of Evdokia, Sebastia, and Cilicia, towards the west, there is a Turkish-speaking Armenian population, as we know. But two large settlements form an exception in the general area of Asia Minor, and they speak a separate Armenian dialect. These are Smyrna and Nicomedia. 

The dialect of Smyrna is spoken not only in Smyrna, which is the largest and most famous center of the area, but also in a few of its surrounding cities, which are Manisa, Kasaba, Menemen, Bayındır, Kırkağaç, and also a few other villages. 

The dialect of Smyrna is still not at all studied. There is only a short manuscript on this dialect \citep[300]{Kosian-smyrna}. We use this text as a sample. 

From this text, it seems that the Smyrna dialect is extremely similar to the Istanbul dialect, and especially the Evdokia dialect; we find differences in some points. 

\section{Text samples}

{\sampleoverview}

– \armenian{Քա եավրում՝ Կիւլիցա, ինտո՞ր իս։}

– \armenian{Վիրէդ բարի, Համաս ղատըն. հայես նէ հազկէկ կը խօսամ. մէկամ օրվան անձրեւը չէրչիֆնէնէրուն արալըխըն ներս վազէր օտին քանափէն պիւս պիւթիւն թրջեր էր. չարշին ըլած էի մախսուս զատկին հէմար ալաճա գնելու, տուն դառնալքէն խապէրսիզ քանափէին վիրէն են եկայ քիչ մը հանգչելու. ի՛նչ}... 

\begin{adjarianpage}\label{page:240}\end{adjarianpage}% should be 240

... \armenian{հայիս, Նէմլախը զէհիրի պէս թախ օսկրներս անցեր է, երկու օրէ վեր հալ չունիմ, երէկ երկու հէղ պայլմիշ եղայ. եավրում, այս ի՛նչ ծանտր բան է. չօճօղները օրթան մնացեր, հայող չունին։}

– \armenian{Քա ատ ի՛նչ լախըրտը, վա՜յ գլխուս, աղջիկդ Հռիփսիմէն ո՞ւր է.անոր խելքը հիմա պիւթիւն է, թօղ չօճօղները ան հայի. դուն հիչ տիւշիւնշիշ մըլլար, քեզի պաշխա իլաճ չկայ. րահաթ տեղդ նստէ, զէնճէֆիլի քէօքը ղայնաթմըշ ընել տուր աղկէկ մը խմէ, ատոր գուվէթը պինդ պաշվա բան է տիւշիւնմիշ մըլլար, քէֆսըզլըղդ կանցնի. հիմա Իզմիր աղկէկ է. կըսեն. կելլենք տէ մենք ալ մէկ աղկէ՛կ փարլաղ զատիկ մը կընենք։}




\chapter{Nicomedia}
\section{Background, subdialects, and literature}
\begin{adjarianpage}\label{page:241}\end{adjarianpage}% should be 241



This heavily Armenian-populated region, which still unbreakably keeps the Armenian language at the northwest of Asia Minor, has two primary cities. Nicomedia (Turkish İzmit) and Adapazar. Around them, there are many large Armenian villages, of which we mention Yalova, Aslanbeg, Bardızağ (Ottoman: Bahçecik), Pazarköy, Geyve, Ortaköy, Sölöz, Benli, Iznik (old Nicaea), and so on. With this diverse vernaculars, there are some manuscripts that are published in \citeauthor{Byurakn}; these are:

\begin{itemize}
 \item Geyve: 1900, page 563, 579, 598, 618
\item Bardızağ: 1898, page 396, 471
\item Ovacık: 1898, page 473, 540
\item Adapazar: 1898, page 597, 887; 1900, page 676
\item Benli: 1898, page 120
\end{itemize}


It is accurate to say these dialects display many differences among themselves, but it appears that we should unite them into one group, and then divide into some subdialects. Based on the manuscripts that we have at hand, their unsatisfactory condition and their scientific inexactness do not allow us to do this division, nor to decide on the borders of these subdialects. 

For the subdialects in this region, the Aslanbeg subdialect has the most genuine and characteristic phenomena. And it is because of this that in Paris, I conducted a study on this subdialect, by working with a young person from Aslanbeg, Mr. Aleksan Nalbandian (\armenian{պր. Ալեքսան Նալբանդեան}). My study was published in \citeauthor{Bazmaveb} \citep{Adjarian-Aslanbeg}, and then published in a separate... 


\begin{adjarianpage}\label{page:242}\end{adjarianpage}% should be 242


... volume in Venice. Besides this, I also studied the sounds of the aforementioned young person, by using the the recording machines of phonetic machines (\armenian{ձայնախօսական մեքենավով}) of Abbé Rousselot (Jean-Pierre Rousselot, Armenian: \armenian{Աբբա Ռուսլօ}), and the results were published in \citet{Adjarian-1899-ArmenianExplosives}. 

\section{Phonology}
\subsection{Segment inventory}

The sound system of the Aslanbeg subdialect has the following sounds: vowels (Table \ref{tab:Nicomedia:phono:segment:vowels}) and consonants. 



\begin{table}[H]
 \centering
 \caption{Vowels of the Aslanbeg subdialect of the Nicomedia dialect}
 \label{tab:Nicomedia:phono:segment:vowels}
 \begin{tabular}{|ll ll l|}
 \hline 
/i/ <\armenian{ի}> & /ʏ/ <\armenian{իւ}>&& & 
\\
/e̞/ <\armenian{է} ̀>& /e/ <\armenian{է}> & /œ/ <\armenian{էօ}> & /ə/ <\armenian{ը}> & /o/ <\armenian{օ}>
 \\
/æ/ <\armenian{ա}̈> & && /ɑ̃/ <\armenian{ա}̄> & /ɑ/ <\armenian{ա}> 
 \\ \hline 
 \end{tabular}
\end{table}


 


 

\begin{table}[H]
 \centering
 \caption{Consonants of the Aslanbeg subdialect of the Nicomedia dialect}
 \label{tab:Nicomedia:phono:segment:cons}
 
 \begin{tabular}{|l|lll|llll|lll|}
 \hline 
 & \multicolumn{3}{l|}{Labial}& \multicolumn{4}{l|}{Coronal}& \multicolumn{3}{l|}{Dorsal/Back}\\
 Stops& /b/ & /pʰ/ & & /d/ & /tʰ/ & & & /ɡ/ & /kʰ/ & 
 \\
 & <\armenian{բ}> & <\armenian{փ}> &&<\armenian{դ}>& <\armenian{թ}>& && <\armenian{գ}>& <\armenian{ք}> & \\

 \hline 
 Affricates & && & /d͡z/ & /t͡sʰ/ & /d͡ʒ/ & /t͡ʃʰ/ && & \\
 & && &<\armenian{ձ}>& <\armenian{ց}> & <\armenian{ջ}>& <\armenian{չ}> & & & \\
 \hline 
 Fricatives& /f/&/v/& &/s/& /z/& /ʃ/& /ʒ/& /χ/ & /ʁ/ & /h/ \\
 & <\armenian{ֆ}>&<\armenian{վ}>& & <\armenian{ս}>& <\armenian{զ}>& <\armenian{շ}>& <\armenian{ժ}>& <\armenian{խ}> & <\armenian{ղ}> & <\armenian{հ}> 
\\ \hline 
 Sonorants & /m/ & /n/& & /ɾ/ & /r/& /l/ & /j/ && & \\
& <\armenian{մ}> & <\armenian{ն}> && <\armenian{ր}>& <\armenian{ռ}>& <\armenian{լ}>& <\armenian{յ}> && & 
\\
 & && & && /lʲ/ & & & & \\
& &&& && <\armenian{լՙ}>& && & 
\\\hline 
 \end{tabular}
 

\end{table}


Among these, the sound /ɑ̃/ <\armenian{ա}̄> represents a nasalized /ɑ/ <\armenian{ա}> sound. The /e̞/ <\armenian{է} ̀> represents a very open /e/ <\armenian{է}> sound. The sounds /œ, ʏ/ <\armenian{էօ, իւ}> have their usual closedness when before stress, but they are pronounced as very open when stressed, like /œɑ, ʏə/ <\armenian{էօա, իւը}>. 

\subsection{Sound changes}

For the sound changes, the following are notable. 


\subsubsection{Vowel changes}

\paragraph{Classical Armenian /ɑ/ <\armenian{ա}>}

The Classical sound /ɑ/ \armenian{ա} became /ɑ̃/ <\armenian{ա}̈> without a definitive rule. It becomes /ɑ̃/ <\armenian{ա}̄> next to nasal. When there a sound /u, o/ <\armenian{ու, օ}> after the nasal, the /ɑ/ <\armenian{ա}> becomes /e/ <\armenian{է}> (Table \ref{tab:Nicomedia:phonology:change:a:an}). 




\begin{table}[H]
 	\centering 
 	\caption{Change from Classical Armenian /ɑ/ <\armenian{ա}> to /e/ <\armenian{է}> in the Nicomedia dialect}
 	\label{tab:Nicomedia:phonology:change:a:an}
 	\begin{tabular}{|l | ll|ll| ll|}
 		\hline & \multicolumn{2}{l|}{Classical Armenian} &\multicolumn{2}{l|}{> Nicomedia} & \multicolumn{2}{l|}{cf. SEA} \\ 
 \armenian{՝}sweet' & ɑnoi̯ʃ & \armenian{անոյշ}& enʏʃ & \armenian{էնիւշ} & ɑnujʃ (dated) & \armenian{անույշ} \\
 & & & & & ɑnuʃ & \armenian{անուշ} \\
`name' & ɑnun & \armenian{անուն} & enʏn & \armenian{էնիւն} &ɑnun & \armenian{անուն} \\ 
`hungry' & nɑu̯tʰi, ɑnɑu̯tʰi & \armenian{նաւթի, անաւթի} & enœtʰi & \armenian{էնէօթի} &ɑnotʰi & \armenian{անոթի} \\ 
 `durable' & ɑmuɾ & \armenian{ամուր} & emʏɾ & \armenian{էմիւր} & ɑmuɾ & \armenian{ամուր} \\ 
 \hline 
 	\end{tabular}
\end{table}



When there is a sound /ɾ/ <\armenian{ր}> after the nasal, the /ɑ/ <\armenian{ա}> becomes /o/ <\armenian{օ}> (Table \ref{tab:Nicomedia:phonology:change:a:o}). 




\begin{table}[H]
 	\centering 
 	\caption{Change from Classical Armenian /ɑ/ <\armenian{ա}> to /o/ <\armenian{օ}> in the Nicomedia dialect}
 	\label{tab:Nicomedia:phonology:change:a:o}
 	\begin{tabular}{|l | ll|ll| ll|}
 		\hline & \multicolumn{2}{l|}{Classical Armenian} &\multicolumn{2}{l|}{> Nicomedia} & \multicolumn{2}{l|}{cf. SEA} \\ 
 				`small' & mɑnəɾ & \armenian{մանր} & moɾjə & \armenian{մօրյը} & mɑnəɾ & \armenian{մանր} \\ 
`heavy' & t͡sɑnəɾ & \armenian{ծանր} & d͡zoɾjə & \armenian{ձօրյը} & t͡sɑnəɾ & \armenian{ծանր} \\
 				`comb' & sɑntəɾ, *sɑnəɾ & \armenian{սանտր}, *\armenian{սանր} & soɾjə & \armenian{սօրյը}& sɑnəɾ & \armenian{սանր} \\ 
\hline 
 	\end{tabular}
\end{table}


When there are two consonants after the nasal, the /ɑ/ <\armenian{ա}> becomes /œ/ <\armenian{էօ}>, while the nasal is lost (Table \ref{tab:Nicomedia:phonology:change:a:o}). 




\begin{table}[H]
 	\centering 
 	\caption{Change from Classical Armenian /ɑ/ <\armenian{ա}> to /œ/ <\armenian{էօ}> in the Nicomedia dialect}
 	\label{tab:Nicomedia:phonology:change:a:oe}
 	\begin{tabular}{|l | ll|ll| ll|}
 		\hline & \multicolumn{2}{l|}{Classical Armenian} &\multicolumn{2}{l|}{> Nicomedia} & \multicolumn{2}{l|}{cf. SEA} \\ 
 `to recognize' & t͡ʃɑnɑt͡ʃʰel & \armenian{ճանաչել} & ɡœʃnɑl & \armenian{գէօշնալ} & t͡ʃɑnɑt͡ʃʰel & \armenian{ճանաչել} \\
`rain' & ɑnd͡zɾeu̯ & \armenian{անձրեւ} & œɾzæv & \armenian{էօրզա̈վ} & ɑnd͡zɾev & \armenian{անձրեւ} \\ 
`thick' &tʰɑnd͡zəɾ& \armenian{թանձր} & tʰœɾzə & \armenian{թէօրզը} &tʰɑnd͡zəɾ& \armenian{թանձր} \\
\hline 
 	\end{tabular}
\end{table}

\paragraph{Classical Armenian /e/ <\armenian{ե}>}

The Classical sound /e/ <\armenian{ե}> becomes /e/ <\armenian{է}> at the beginning of words, while it is /e̞/ <\armenian{է} ̀> in other places. 

\paragraph{Classical Armenian /o/ <\armenian{ո}>}


The sound /o/ is usually /œ/ <\armenian{էօ}>, but it becomes /ɑ/ <\armenian{ա}> next to nasals (Table \ref{tab:Nicomedia:phonology:change:o}). \translatorHD{Note that Adjarian actually writes <\armenian{օ}> which is CA /ɑu̯/; but his example is about CA /o/ <\armenian{ո}>; it seems he made a typo. } 




\begin{table}[H]
 	\centering 
 	\caption{Change from Classical Armenian /o/ <\armenian{ո}> to /ɑ/ <\armenian{ա}> in the Nicomedia dialect}
 	\label{tab:Nicomedia:phonology:change:o}
 	\begin{tabular}{|l | ll|ll| ll|}
 		\hline & \multicolumn{2}{l|}{Classical Armenian} &\multicolumn{2}{l|}{> Nicomedia} & \multicolumn{2}{l|}{cf. SEA} \\ 
`buffalo' &ɡomēʃ & \armenian{գոմէշ} & ɡɑmeʃ & \armenian{գամէշ}& ɡomeʃ& \armenian{գոմեշ} \\
\hline 
 	\end{tabular}
\end{table}

\paragraph{Other vowel changes}

In others, we see the following changes:
\begin{itemize}
 \item CA /u/ <\armenian{ու}> $\rightarrow$ /ʏ/ <\armenian{իւ}> 
 \item CA /oi̯/ <\armenian{ոյ}> $\rightarrow$ /ʏ/ <\armenian{իւ}> 
 \item CA /iu̯/ <\armenian{իւ}> $\rightarrow$ /ʏ/ <\armenian{իւ}> 
 \item CA /ɑi̯/ <\armenian{այ}> $\rightarrow$ /ɑ/ <\armenian{ա}> (under stress)
 \item CA /ɑi̯/ <\armenian{այ}> $\rightarrow$ /e/ <\armenian{է}> (without stress)
 
\end{itemize}


For example, see Table \ref{tab:Nicomedia:phonology:change:other}. 

\begin{table}[H]
 	\centering 
 	\caption{Miscellaneous vowel changes from Classical Armenian to the Nicomedia dialect}
 	\label{tab:Nicomedia:phonology:change:other}
 	\begin{tabular}{|l | ll|ll| ll|}
 		\hline & \multicolumn{2}{l|}{Classical Armenian} &\multicolumn{2}{l|}{> Nicomedia} & \multicolumn{2}{l|}{cf. SEA} \\ 
 `father' & hɑi̯ɾ & \armenian{հայր}& hɑɾ & \armenian{հար} & hɑjɾ & \armenian{հայր} \\ 
`to burn' & ɑi̯ɾel & \armenian{այրել} & eɾel & \armenian{էրէլ} & ɑjɾel & \armenian{այրել} \\ 
\hline 
 	\end{tabular}
\end{table}

\subsubsection{Consonant changes: cluster reduction}

The consonant sound changes are very interesting. Speaking in general, the sequence plosive+consonant is unacceptable in the Aslanbeg subdialect. When such a sequence occurs in a word, whether originally, or in connected speech when a plosive-final word precedes a consonant-initial word (such as ).\footnote{\translatorHD{Includes a set of words and phrases from SEA/SWA as examples for words with clusters. I don't include them in the translation because I don't think they're useful for a non-Armenian reader, and including them into the prose seems confusing. For those Armenian readers who are interested, Adjarian lists the following words as having word-internal clusters: \armenian{ոտք, ձեռք, մարդ, կոճկել, մեծնալ, կանգնիլ, ծնկվըներ}. He lists the following simple phrases that have a cluster across a word-boundary: \armenian{հաց տուր, դուք գացէք, մենք քեզի ըսինք, կրակ վառէ}, where the sequences <\armenian{հց, քգ, քք, կվ}> appear. }} In this... 


\begin{adjarianpage}\label{page:243}\end{adjarianpage}% should be 243


... situation, the first member of the sequence (the plosive) undergoes the following changes. 

\paragraph{Lenition of /ɡ/ <\armenian{գ}> to a glide /j/ <\armenian{յ}>}



The sound /ɡ/ <\armenian{գ}> becomes the semivowel /j/ <\armenian{յ}> (Table \ref{sent:Nicomedia:phonology:change:cons:g}). 


\begin{exe}
 \ex \label{sent:Nicomedia:phonology:change:cons:g}
 \begin{xlist}
 \ex 
 \glll mʏj mə (Nicomedia) \\ 
 muɡ mə (SWA) \\
 mouse {\indf} \\
 \trans `a mouse' \\
 \armenian{միւյ մը, մուկ մը} 
 \ex 
 \glll hʏ dɑsə (Nicomedia) \\ 
 hiŋkʰ dɑsə (SWA) \\
 five ten \\
 \trans \translatorHD{I'm unsure of the translation, perhaps `fifteen'.} \\
 \armenian{հիւդասը, հինգ-տասը}
 \end{xlist}
\end{exe}




\paragraph{Deletion of other stops} 

The sounds /kʰ, b, pʰ, d, tʰ/ <\armenian{ք, բ, փ, դ, թ}> are deleted; but in its place we find a sudden cessation of breathing and constriction of the throat, which we present with the symbol \armenian{※}. This form change is very interesting, and from a general phonetic perspective, it shows the path that consonants take before they are completely lost (Table \ref{sent:Nicomedia:phonology:change:cons:stops}). 
 

\translatorHD{Without recordings, it's difficult to know exactly what Adjarian interpreted as this cessation. It could be a glottal stop, or the impression of an unreleased stop. I thus cannot give it a believable IPA symbol. }



\begin{exe}
 \ex \label{sent:Nicomedia:phonology:change:cons:stops}
 \begin{xlist}
\ex \glll ʃɑ\armenian{※} mɑɾtʰ (Nicomedia) \\ 
ʃɑd mɑɾtʰ (SWA) \\
many person\\
 \trans `many people' \\
 \armenian{շա※ մարթ, շատ մարդ}
\ex \glll pʰɑ\armenian{※} ɡ-ɑ-$\emptyset$ (Nicomedia) \\
 pʰɑjd ɡ-ɑ-$\emptyset$ (SWA) \\
 wood exist-{\thgloss}-3{\sg} \\
 \trans `Is there wood?' \\
 \armenian{փա※ գա՞}, 
 \armenian{փայտ կա՞յ}
 \ex \gll œ\armenian{※}kʰ-ə (Nicomedia) \\
 votk-ə (SWA)\\
 foot-{\defgloss} \\
 \trans `the foot'\\ 
 \armenian{էօ※քը, ոտքը}
 \end{xlist}
\end{exe}

\paragraph{Deaffrication of affricates} 

The Classical affricates (\armenian{շչական}) /t͡ʃ, d͡ʒ, t͡ʃʰ, t͡s, d͡z, t͡sʰ/ <\armenian{ճ, ջ, չ, ծ, ձ, ց}> lose their dental plosive part and become the simpler sounds /ʒ, ʃ, z, s/ <\armenian{ժ, շ, զ, ս}>. 


\begin{exe}
 \ex \label{sent:Nicomedia:phonology:change:cons:affr}
 \begin{xlist}
\ex \glll məz mɑɾtʰ (Nicomedia) \\
 med͡z mɑɾtʰ (SWA) \\
 big person\\
 \trans `big man/person'\\
 \armenian{մըզ մարթ, մեծ մարդ}
\ex \glll ve̞s dʁɑ (Nicomedia) \\
 vet͡sʰ dəʁɑ (SWA) \\
 six boy\\
 \trans `six boys' \\
 \armenian{վէ ̀ս դղա, վեց տղայ}
 \ex \glll dɑʒɡ-ə-n-ɑ-l dʁɑ (Nicomedia) \\
 dɑd͡ʒɡ-ɑ-n-ɑ-l (SWA) \\
 Turk-{\lvgloss}-{\inch}-{\thgloss}-{\infgloss}\\
 \trans `to become a Turk' \\
 \armenian{դաժգընալ, տաճկանալ}
 
 \end{xlist}
\end{exe}


\paragraph{Deletion of nasals}

And also for these three conditions, if there is a nasal sound before the plosive sound, then it is lost (Table \ref{tab:Nicomedia:phonology:change:cons:ndel}). 

\begin{table}[H]
 	\centering 
 	\caption{Deletion of nasals in cluster reduction in the Nicomedia dialect}
 	\label{tab:Nicomedia:phonology:change:cons:ndel}
 	\begin{tabular}{|l | ll|ll| lll|}
 		\hline & \multicolumn{2}{l|}{Classical Armenian} &\multicolumn{2}{l|}{> Nicomedia} & \multicolumn{3}{l|}{cf. SEA or SWA} \\ 
 \armenian{՝}to fall' & ɑnkɑnil & \armenian{անկանիլ}& ijnɑl & \armenian{իյնալ} & ənknel & \armenian{ընկնել} & SEA \\
`thick' &tʰɑnd͡zəɾ& \armenian{թանձր} & tʰœɾzə & \armenian{թէօրզը} &tʰɑnd͡zəɾ& \armenian{թանձր} & SEA \\
 	`high' &bɑɾd͡zəɾ & \armenian{բարձր} & bɑɾsə & \armenian{բարսը} & bɑɾt͡sʰəɾ & \armenian{բարձր} & SEA \\ 
`fly (bug)' & t͡ʃɑnt͡ʃ& \armenian{ճանճ} & & & d͡ʒɑnd͡ʒ & \armenian{ճանճ} & SWA\\ 
`a fly' & & & ʃɑʒ mə & \armenian{շաժ մը} & d͡ʒɑnd͡ʒ mə & \armenian{ճանճ մը} & SWA\\ 
 `to pass' & ɑnt͡sʰɑnel & \armenian{անցանել} & ɑsnil & \armenian{ասնիլ}& ɑnt͡sʰnil & \armenian{անցնիլ} & SWA \\
\hline 
 	\end{tabular}
\end{table}


\paragraph{Voicing assimilation in deaffrication}


In itself, it is understandable that for the third condition, if the affricates are lost next to plosives of a disagreeing degree, then the affricate takes the degree of the plosive: the voiceless becomes voiced, and the voiced become voiceless. \translatorHD{Note that in Adjarian's examples, the affricate is voiced in SWA, but it would not have been voiced in the original CA form.}

\begin{exe}
 \ex \label{sent:Nicomedia:phonology:change:cons:voicing}
 \begin{xlist}
 \ex \glll ɡɑɾʃ kʰitʰ (Nicomedia) \\
 ɡɑɾd͡ʒ kʰitʰ (SWA)\\
 short nose \\
 \trans `a short nose'\\
 \armenian{գարշ քիթ, կարճ քիթ} 
 \ex \glll əs-ɑs-s e (Nicomedia) \\
 əs-ɑd͡z-əs e (SWA)\\
 say-{\rptcp}-{\possFsg} {\aux} \\
 \trans `it is what I said'\\
 \armenian{ըսասս է, ըսածս է}

 \end{xlist}
\end{exe}
\section{Morphology}
\subsection{Verb inflection or conjugation}
The grammatical forms are like Istanbul. But in verbal conjugation, the Classical ending /e/ <\armenian{ե}> becomes /i/ <\armenian{ի}> next to nasals. Like the Evdokia dialect, the imperfective and perfective changed the Old Armenian ending /-ɑ-kʰ/ <\armenian{աք}> (New Armenian /-i-nkʰ/ <\armenian{ինք}>) to /-ɑ̃-nkʰ/ <\armenian{ա}̄\armenian{նք}>. The progressive is always made with formative /h\'ɑje/ <\armenian{հա՛յէ}>. The following are the mentioned forms of the verb `to like'. 

{\paradigmExplanation}

\subsubsection{Indicative present and past imperfective}



\translatorHD{For the indicative present, SWA combines the indicative prefix /ɡ(ə)/ <\armenian{կը}> with a finite verb. This finite verb is the subjunctive form. For an E-Class verb like `to like' /siɾ-e-l/, the theme vowel is a constant /e/, and the 3SG marker is covert. In Nicomedia, the theme vowel varies between /i/ <\armenian{ի}> and /e̞/ <\armenian{է} ̀> (Table \ref{tab:Nicomedia:morpho:verb:paradigm:presentPastIndc}). }


\begin{table}[H]
 	\centering 
 	\caption{Indicative present <\armenian{ներկայ}> of the verb `to like' in the Nicomedia dialect}
 	\label{tab:Nicomedia:morpho:verb:paradigm:presentPastIndc}
 	 \begin{tabular}{|l| ll| ll|}
 		\hline & \multicolumn{2}{l|}{Nicomedia } & \multicolumn{2}{l|}{cf. SWA} \\ \hline 
1SG & ɡə siɾ-i-m & \armenian{գը սիրիմ} & ɡə siɾ-e-m & \armenian{կը սիրեմ} \\
2SG & ɡə siɾ-e̞-s &\armenian{գը սիրէ ̀ս} & ɡə siɾ-e-s & \armenian{կը սիրես} \\
3SG & ɡə siɾ-e̞-$\emptyset$ & \armenian{գը սիրէ} ̀ & ɡə siɾ-e-$\emptyset$ & \armenian{կը սիրէ} \\
1PL & ɡə siɾ-i-nkʰ & \armenian{գը սիրինք} & ɡə siɾ-e-ŋkʰ & \armenian{կը սիրենք} \\
2PL & ɡə siɾ-e̞-kʰ & \armenian{գը սիրէ ̀ք} & ɡə siɾ-e-kʰ & \armenian{կը սիրէք} \\
3PL & ɡə siɾ-i-n& \armenian{գը սիրին} & ɡə siɾ-e-n & \armenian{կը սիրեն} \\
 		& \multicolumn{2}{l|}{{\ind} $\sqrt{}$-{\thgloss}-{\agr}} & \multicolumn{2}{l|}{{\ind} $\sqrt{}$-{\thgloss}-{\agr}} \\ 
 		\hline 
 		 
 	\end{tabular}
\end{table}

\translatorHD{For the indicative past imperfective, SWA combines the indicative prefix with a finite verb (the past imperfective). This finite form includes the past suffix /-i/ after the theme vowel, such as the past 1PL sequence /-i-nkʰ/ (Table \ref{tab:Nicomedia:morpho:verb:paradigm:PastIndc}). This past suffix is however covert in the 3SG, along with a covert agreement suffix. This is in contrast to CA where the past 1PL was the sequence of morphs /-ɑ-kʰ/ where /ɑ/ was likely a past marker. Nicomedia is more conservative and uses the past suffix /-ɑ/ for the past 1PL. Note how the theme vowel varies in form. }


\begin{table}[H]
 \centering
 \caption{Indicative past imperfective <\armenian{անկատար}> of the verb `to like' in the Nicomedia dialect}
\label{tab:Nicomedia:morpho:verb:paradigm:PastIndc}
 \begin{tabular}{|l|ll|ll|}
\hline 
& \multicolumn{2}{l|}{Nicomedia} & \multicolumn{2}{l|}{cf. SWA} \\
1SG & ɡə siɾ-e-i-$\emptyset$ & \armenian{գը սիրէի} & ɡə siɾ-ej-i-$\emptyset$ & \armenian{կը սիրէի} \\
2SG & ɡə siɾ-e-i-ɾ & \armenian{գը սիրէիր} & ɡə siɾ-ej-i-ɾ & \armenian{կը սիրէիր} \\
3SG &ɡə siɾ-e̞-$\emptyset$-ɾ & \armenian{գը սիրէ ̀ր} & ɡə siɾ-e-$\emptyset$-ɾ & \armenian{կը սիրէր} \\
1PL & ɡə siɾ-e-ɑ̃-nkʰ & \armenian{գը սիրէա}̄\armenian{նք} & ɡə siɾ-ej-i-ŋkʰ & \armenian{կը սիրէինք} \\
2PL & ɡə siɾ-e-i-kʰ & \armenian{գը սիրէիք} & ɡə siɾ-ej-i-kʰ & \armenian{կը սիրէիք} \\
3PL & ɡə siɾ-e-i-n & \armenian{գը սիրէին} & ɡə siɾ-ej-i-n & \armenian{կը սիրէին} \\
& \multicolumn{2}{l|}{{\ind} $\sqrt{}$-{\thgloss}-{\pst}-{\agr} }& \multicolumn{2}{l|}{{\ind} $\sqrt{}$-{\thgloss}-{\pst}-{\agr} } \\
\hline 
\end{tabular}
\end{table}

 

\subsubsection{Progressive marking}

\translatorHD{In SWA, the indiciatve present and the indicative past imperfective are rendered progressive by simply adding the progressive enclitic /ɡoɾ/. In Nicomedia, the progressive marker is instead /h\'ɑje/ <\armenian{հա՛յէ}> (Table \ref{tab:Nicomedia:morpho:verb:paradigm:prog}). }



\begin{table}[H]
 	\centering 
 	\caption{Progressive <\armenian{շարունական}> of the present and past imperfective of the verb `to like' in the Nicomedia dialect}
 	\label{tab:Nicomedia:morpho:verb:paradigm:prog}
 	 \begin{tabular}{|l| ll| ll|}
 		\hline & \multicolumn{4}{l|}{Progressive present <\armenian{շարունական ներկայ}> }\\ 
 & \multicolumn{2}{l|}{Nicomedia } & \multicolumn{2}{l|}{cf. SWA} \\ \hline
1SG & ɡə siɾ-i-m h\'ɑje & \armenian{գը սիրիմ հա՛յէ} & ɡə siɾ-\'e-m ɡoɾ & \armenian{կը սիրեմ կոր} \\
2SG & ɡə siɾ-e̞-s h\'ɑje &\armenian{գը սիրէ ̀ս հա՛յէ}& ɡə siɾ-\'e-s ɡoɾ & \armenian{կը սիրես կոր} \\
3SG & ɡə siɾ-e̞-$\emptyset$ h\'ɑje & \armenian{գը սիրէ} ̀ \armenian{հա՛յէ} & ɡə siɾ-\'e-$\emptyset$ ɡoɾ & \armenian{կը սիրէ կոր} \\
1PL & ɡə siɾ-i-nkʰ h\'ɑje & \armenian{գը սիրինք հա՛յէ} & ɡə siɾ-\'e-ŋkʰ ɡoɾ & \armenian{կը սիրենք կոր}\\
2PL & ɡə siɾ-e̞-kʰ h\'ɑje & \armenian{գը սիրէ ̀ք հա՛յէ}& ɡə siɾ-\'e-kʰ ɡoɾ& \armenian{կը սիրէք կոր} \\
3PL & ɡə siɾ-i-n h\'ɑje& \armenian{գը սիրին հա՛յէ}& ɡə siɾ-\'e-n ɡoɾ & \armenian{կը սիրեն կոր}\\
 		& \multicolumn{2}{l|}{{\ind} $\sqrt{}$-{\thgloss}-{\agr} {\prog}} & \multicolumn{2}{l|}{{\ind} $\sqrt{}$-{\thgloss}-{\agr} {\prog}} \\ 
\hline & \multicolumn{4}{l|}{Progressive past imperfective <\armenian{շարունական անկատար}> }\\
& \multicolumn{2}{l|}{Nicomedia} & \multicolumn{2}{l|}{cf. SWA} \\
1SG & ɡə siɾ-e-i-$\emptyset$ h\'ɑje & \armenian{գը սիրէի հա՛յէ}& ɡə siɾ-ej-\'i-$\emptyset$ ɡoɾ& \armenian{կը սիրէի կոր} \\
2SG & ɡə siɾ-e-i-ɾ h\'ɑje& \armenian{գը սիրէիր հա՛յէ}& ɡə siɾ-ej-i-ɾ ɡoɾ& \armenian{կը սիրէիր կոր} \\
3SG &ɡə siɾ-e̞-$\emptyset$-ɾ h\'ɑje & \armenian{գը սիրէ ̀ր հա՛յէ} & ɡə siɾ-\'e-$\emptyset$-ɾ ɡoɾ& \armenian{կը սիրէր կոր} \\
1PL & ɡə siɾ-e-ɑ̃-nkʰ h\'ɑje& \armenian{գը սիրէա}̄\armenian{նք հա՛յէ} & ɡə siɾ-ej-\'i-ŋkʰ ɡoɾ& \armenian{կը սիրէինք կոր} \\
2PL & ɡə siɾ-e-i-kʰ h\'ɑje & \armenian{գը սիրէիք հա՛յէ}& ɡə siɾ-ej-\'i-kʰ ɡoɾ & \armenian{կը սիրէիք կոր} \\
3PL & ɡə siɾ-e-i-n h\'ɑje& \armenian{գը սիրէին հա՛յէ} & ɡə siɾ-ej-\'i-n ɡoɾ & \armenian{կը սիրէին կոր} \\
& \multicolumn{2}{l|}{{\ind} $\sqrt{}$-{\thgloss}-{\pst}-{\agr} {\prog} }& \multicolumn{2}{l|}{{\ind} $\sqrt{}$-{\thgloss}-{\pst}-{\agr} {\prog} } \\
\hline 
\end{tabular}
\end{table}



\subsubsection{Past perfective or aorist}

\translatorHD{The past perfective (Table \ref{tab:Nicomedia:morpho:verb:paradigm:pastperfectiveAorist}) is also called the aorist. In SWA for /siɾ-e-l/ `to like', the past perfective is formed by taking the root and theme vowel, adding the aorist or perfective suffix /-t͡sʰ-/, and then adding the past suffix /-i/ and the appropriate agreement suffixes. The 3SG uses covert tense and agreement suffixes. The Nicomedia dialect behaves essemntially the same with two major differences: the theme vowel can vary, and the past suffix is /ɑ̃/ <\armenian{ա}̄> for the 1PL. }


\begin{table}[H]
 \centering
 \caption{Past perfective or aorist <\armenian{կատարեալ}> of the verb `to like' in the Nicomedia dialect}
 \label{tab:Nicomedia:morpho:verb:paradigm:pastperfectiveAorist}
 \begin{tabular}{|l|ll|ll|}
\hline & \multicolumn{2}{l|}{Nicomedia} & \multicolumn{2}{l|}{cf. SWA} \\
1SG & siɾ-e-t͡sʰ-i-$\emptyset$ & \armenian{սիրէցի} & siɾ-e-t͡sʰ-i-$\emptyset$ & \armenian{սիրեցի} \\
2SG & siɾ-e-t͡sʰ-i-ɾ &\armenian{սիրէցիր} & siɾ-e-t͡sʰ-i-ɾ & \armenian{սիրեցիր} \\
3SG & siɾ-e̞-t͡sʰ-$\emptyset$-$\emptyset$ & \armenian{սիրէ ̀ց} & siɾ-e-t͡sʰ-$\emptyset$-$\emptyset$ & \armenian{սիրեց} \\
1PL & siɾ-e-t͡sʰ-ɑ̃-nkʰ & \armenian{սիրէցա}̄\armenian{նք} & siɾ-e-t͡sʰ-i-ŋkʰ & \armenian{սիրեցինք} \\
2PL & siɾ-e-t͡sʰ-i-kʰ & \armenian{սիրէցիք} & siɾ-e-t͡sʰ-i-kʰ & \armenian{սիրեցիք} \\
3PL & siɾ-e-t͡sʰ-i-n & \armenian{սիրէցին} & siɾ-e-t͡sʰ-i-n & \armenian{սիրեցին} \\
& \multicolumn{2}{l|}{$\sqrt{}$-{\thgloss}-{\aor}-{\pst}-{\agr}}& \multicolumn{2}{l|}{$\sqrt{}$-{\thgloss}-{\aor}-{\pst}-{\agr}}\\ 
\hline 
\end{tabular}
\end{table}



\begin{adjarianpage}\label{page:244}\end{adjarianpage}% should be 244

\subsubsection{Other tenses}


The other tenses are formed in accordance to these. 

\translatorHD{Adjarian does not discuss the above forms at all, but his brief examples suggest the following differences between Nicomedia and SWA.}

\translatorHD{The future and future perfect are formed the same across Nicomedia and SWA (\ref{sent:Nicomedia:morpho:verb:other:fut}). In SWA, these forms are created by adding the future proclitic /bidi/ before the finite verb forms that are used for the indicative present and indicative past imperfective. }

\begin{exe}
 \ex \label{sent:Nicomedia:morpho:verb:other:fut}
 \begin{xlist}
 \ex Future 
 \glll bidi siɾ-i-m (Nicomedia) \\
 bidi siɾ-e-m (SWA) \\
 {\fut} like-{\thgloss}-1{\sg} \\
 \trans `I will like.' \\
 \armenian{բիդի սիրիմ, պիտի սիրեմ}
 \ex Future perfect
 \glll bidi siɾ-e-i-$\emptyset$ (Nicomedia) \\
 bidi siɾ-ej-i-$\emptyset$ (SWA) \\
 {\fut} like-{\thgloss}-{\pst}-1{\sg} \\
 \trans `I was going to like.' \\
 \armenian{բիդի սիրէի, պիտի սիրէի}
 \end{xlist}
\end{exe}

\translatorHD{The imperative 2SG does not have an overt 2SG suffix in either dialect (\ref{sent:Nicomedia:morpho:verb:other:imp}). }


\begin{exe}
 \ex Imperative 2SG \label{sent:Nicomedia:morpho:verb:other:imp}
 \glll siɾ-e-$\emptyset$ (Nicomedia) \\
 siɾ-e-$\emptyset$ (SWA) \\
 like-{\thgloss}-{\imp}.2{\sg} \\
 \trans `Like!' \\
 \armenian{սիրէ, սիրէ}
 
\end{exe}

\translatorHD{The subjunctive present and past imperfective in formal SWA are just the finite verb, while spoken informal SWA allows adding a subjunctive enclitic /ne/ after this form. Nicomedia has a subjunctive enclitic /nə/ (\ref{sent:Nicomedia:morpho:verb:other:subj}). }

\begin{exe}
 \ex \label{sent:Nicomedia:morpho:verb:other:subj}
 \begin{xlist}
 \ex Subjunctive present 1SG 
 \glll siɾ-i-m nə (Nicomedia) \\
 siɾ-e-m (ne) (SWA) \\
 like-{\thgloss}-1{\sg} {\sbjv} \\
 \trans `(If) I like'\\
 \armenian{սիրիմ նը, սիրեմ (նէ})
 \ex Subjunctive past imperfective 1SG 
 \glll siɾ-e-i-$\emptyset$ nə (Nicomedia) \\
 siɾ-ej-i-$\emptyset$ (ne) (SWA) \\
 like-{\thgloss}-{\pst}-1{\sg} {\sbjv} \\
 \trans `(If) I liked'\\
 \armenian{սիրէի նը, սիրէի (նէ})
 \end{xlist}
\end{exe}


\section{Miscellaneous}
\subsection{Subdialectal variation on the progressive}

In the city of Nicomedia, the progressive is formed with the formative /joɾ/ <\armenian{յօր}>, which is an exact borrowing from Turkish <yor>.


\subsection{Prosody}

In the area of Nicomedia, conversations generally have a very long stress. The end of every word or speech is lengthened with the singing melody, like for the people of Shamakhi (\ref{sent:Nicomedia:other:stressLong}).



\begin{exe}
 \ex Nicomedia? \label{sent:Nicomedia:other:stressLong}
 \begin{xlist}
 \ex \gll bidi eɾtʰ-ɑː-s \\
 {\fut} go-{\thgloss}-2{\sg} \\
 \trans `You will go.' \\
 \armenian{բիդի էրթա՜՜ս}
 \ex \gll ɑnun-d int͡ʃʰ eː \\
 name-{\possSsg} what {\aux}\\ 
 \trans `What is your name?''
 \armenian{անունդ ի՞նչ է՜՜՜}
 \end{xlist}
\end{exe}

\translatorHD{I suspect that Adjarian's transcriptions are however not in his orthographic system, but are instead written as SWA words. For example, for the word `your name', the spelling is <\armenian{անունդ}> which would be interpreted as [ɑnund] in Adjarian's notation, but the SWA pronunciation is [ɑnunətʰ]. }


\section{Text samples}

{\sampleoverview}

\subsection{Aslanbag subdialect}
Adjarian's source: \citet[35]{Adjarian-Aslanbeg}


– \armenian{Փար իրգիւն, Խաշդիւր ախբա̈ր։}

– \armenian{Խէր ըլՙլՙա։}

– \armenian{Օվադէմ բաբային քլխիւն էգաձը իմացա՞ր, էքիին մալահաթը։}

– \armenian{Իրավ էօր էրէգ իրգիւն ադայ լաֆ մը գար հըմը, ըռի※ մը չիյդիմ։}

– \armenian{Էքին քէօղ էգէ ̀ր է} ̀ \armenian{դը ջիւղէ ̀րը ա̈նընգիւյ գէօդըրդէ ̀ր ին։}

– \armenian{Ձիւ ըբը ի՞շ գայնէ ̀ր է ̀ք, ցիյէ ̀րը քաշէցէք դը իզը իա}̄\armenian{նք. աս փա}̄\armenian{նի ընէ ̀օղը Լազէ ̀րը ըլՙլՙալիւ ին. առչի էօրն ա Վարթա}̄\armenian{ն ամջիւն էքին ըրէ ̀ր ին։ Թիւն քընը Գարըբէդին իլէն Մինա}̄\armenian{նին գանչէ} ̀, \armenian{էս ա էրգիւ հա※ բէգիրջի ջարիմ։}

\begin{adjarianpage}\label{page:245}\end{adjarianpage}% should be 245

\armenian{Մանիւգը գիա Գարըբէդին դիւնը}.

– \armenian{Փարի լիւսս, Բա՛յձառ, Գարըբէդը վէ՞րն է ̀։}

– \armenian{Խէր ըլՙլՙա, Խաշդիւր ախբար, վէրն է ̀։}

– \armenian{Թէօղ չիւֆդա̈ն առնէ} ̀ \armenian{դը քա։}

\armenian{Բայձառը վէր գիա։}

– \armenian{Մար※, գըսէ} ̀, \armenian{էլի չիւֆդա̈ն առ դը վար քնը։}

\armenian{Գարըբէդը գը ցա※գէ} ̀, \armenian{չիւֆդա̈ն ա̈միւզը գը զարնէ} ̀, \armenian{գէնէ ի՞շ դա ըջըբ, գըսէ} ̀, \armenian{վար գիշնա։}

– \armenian{Փարի լիւյս, Մանիւգ ախբար։}

– \armenian{Աս※ձիւ փարին, քընի՞ սահաթէն վար գիշնաս. մնչիգը ղայֆա̈ն ին. Լազէ ̀րիւն բիդի իա}̄\armenian{նք։}

– \armenian{Շա※ զիյանիւթին ըրէ ̀՞ր ին։}

– \armenian{Ախբար, հավէօզնէ ̀րը իւդէլնին հա՛դէ զարար չիւնի ըսինք, գէօջէ ̀րն ա̈նընգիւ※ դէօդրդէ ̀ր ին։}

– \armenian{Անա}̄\armenian{նգ է} ̀ \armenian{նը քընի մը հադ ա ցի առնէլիւ է ̀։}



\subsection{Bardızağ}
Adjarian's source: See \citeauthor{Byurakn} 1898, page 396. 

\armenian{Էլանք գացանք Գալիլիա},

\armenian{Գալիլիան ծով մը կար},

\armenian{Ծովուն մէջ ծառ մը կար},

\armenian{Ծառին վրա բուն մը կար},

\armenian{Բընին մէջ հավկիթ մը կար},

\armenian{Հաւկթին մէջ ձագ մը կար},

\armenian{Ան ձագը անդանակ մօրթեցին},

\armenian{Անկրակ եփեցին},

\armenian{Ով կերաւ զարմացաւ},

\armenian{Ով չկերաւ ճաթեցավ։}

\armenian{Աչք ընողուն աչքը ճաթի։}

\armenian{Աղէկ չօճուխին օրնէն (օրրանէն) կառնին},

\armenian{Աղէկ կրիյճին (կտրիճին) փէրչըմէն կառնին},

\armenian{Աղէկ եզան լիծէն կառնին},

\armenian{Աղէկ գոմշուն կոտոշէն կառնին}.

\armenian{Աչք ընողուն աչքը ճաթի։}

\begin{adjarianpage}\label{page:246}\end{adjarianpage}% should be 246

\armenian{Երսուն երկու ձի ու ջորի},

\armenian{Էկան անցան յիսուս պաղչան},

\armenian{Կարմիր կովը մորթեցին},

\armenian{Վէվ կերաւ զարմացաւ},

\armenian{Վէվ չկերաւ ճաթեցավ։}

\subsection{Ovacık}

Adjarian's source: See \citeauthor{Byurakn} 1898, page 475


\armenian{Հարիկ չունէր, մարիկ ունէր},

\armenian{Չըխտըկուք (զոյգ) քըրվըտանք ունէր։}

\armenian{Նէ հարիկ ունիմ, նէ մարիկ},

\armenian{Ունիմ միայն մէ աղբարիկ։}

\armenian{Մարիկ չունիմ, աղբար չումիմ},

\armenian{Մէկ հատիկ քուրիկ մ՚ունիմ։}

\armenian{Մօրկանս սիրելին էի},

\armenian{Հօրկանս գանձապահն էի։}

\armenian{Մօրկանս մէկ հատն էի},

\armenian{Եղբօրս սրտաշն էի}.

\armenian{Որ շարած մարգրիտ էինք},

\armenian{Շարքուկ շարքուկ քակրվէցանք։}

\armenian{Մենք ջուխթ մի կէօվէրճին էինք},

\armenian{Որըս սար ելանք, որըս ձոր։}

\armenian{Որ ծալած ղումաշիկ էինք},

\armenian{Ծալուկ ծալուկ քակրվեցանք},

\armenian{Չեն արժան տեղվանքներ ինկանք։}

\armenian{Մեր քուրը հարսնիք է բռնէր},

\armenian{Աղբօրը մոմ մը չէ ղրկեր}.

\armenian{Ղրկեր է տեղը հասեր։}

\armenian{Այս աշխարհըս առին տարին},

\armenian{Աղջիկ տղին հացն էր հարամ։}

\subsection{Adapazar}
Adjarian's source: See \citeauthor{Byurakn} 1900, page 676

\armenian{Աղղբար աղբարուկ էանք},

\armenian{Խմելու պաղ ջուր էանք}.


\begin{adjarianpage}\label{page:247}\end{adjarianpage}% should be 247


~ ~ ~ \armenian{Ըստամպօլէն հէքիմ բերէք, հալիկս ցըցուցէք},

~ ~ ~ \armenian{Իմ հալիկս հալիկ չէ, բարձիկս դարձուցէք։}

\armenian{Ես որ մեռնիմ, մայրի՜կ, փոսս խորունկ փորեցէք։}

\armenian{Իմ քաշած չարչուրանքներս վրաս գրեցէք։}

~ ~ ~ \armenian{Խարիպութան բանը եաման դիժար է},

~ ~ ~ \armenian{Աշխարհս լուս արեւ՝ մեզի տուման է։}

\armenian{Մի՛ լար, մա՛յրիկ, մի՛ լար, աչվիդ կաւրի},

\armenian{Մէրտիվէնէն վար իչնամ նէ սդտիկդ կը մարի։}

~ ~ ~ \armenian{Իմ սէրս քու սէրդ դատասան մնաց},

~ ~ ~ \armenian{Շատ մուրազներ ունէանք ՝ կիսկատար մնաց։}


\armenian{Մե՛րթար օղուլ, մե՛րթար, մինակ կը մնանք},

\armenian{Կերթաս ալ չես ի գար՝ կարօտ կը մնան։}

\armenian{Փէշերդ սօթտեցին, գօտիդ խօթեցիր},

\armenian{Քու ղարիպուկ հայրիկդ ու մայրիկդ որի՞ ձգեցիր։}

\armenian{Մի՛ լար, մայրիկ, մի՛ լար, էս կէնէ կուգամ},

\armenian{Ասկէ տասնը հինգ օրէն երազդ կուգամ։}

\armenian{Օ՛ղուլ, երազով կարօտ չառնուիր},

\armenian{Էրկիրմով ալ մուրազ չառնուիր։}

\armenian{Մի, լար, մայրիկ, մի՛ լար դու ինծի համար},

\armenian{Ինչո՞ւ մեզ աշխարհք բերիր, մեռնելու համար։}

\armenian{Մէրտիվէնէն վար իջնամ նէ՝ ետեւէս նայէ}.

\armenian{Դուռնէն դուրս ելլեմ նէ ըմուտդ կըրեէ։}

\subsection{Benli}

Adjarian's source: See \citeauthor{Byurakn}, 1898, page 120

\armenian{Արտերը փուսեր է փուսը},

\armenian{Վէրուսաղում էլավ լուսը}.

\armenian{Հռուփսիմա Մարյամ կուսը}

\armenian{Օրնէ ըս մէր թոգովէօրը։}

\armenian{Քահանանին անցան տասը}

\armenian{Բաժնեցին խօվէրդն (հաղորդ) ու մասը}.

\armenian{Երկինքէն կախվեր է փուլքը},

\armenian{Ի՞նչ ընեմ աշխըրքիս միլքը}.

\armenian{Ըռըսդագէս վարեց ծովը},

\armenian{Տասվերկու աշակերտ քովը},


\begin{adjarianpage}\label{page:248}\end{adjarianpage}% should be 248



\armenian{Օրնէ թոգովէօրն ու թագուհին},

\armenian{Օրնէ անմէնքս ալ միասին։}

\armenian{Ծովին մէչի կարմիր ատան},

\armenian{Աստված փրկէ չարն ու խատան։}

\armenian{Էրզինկավու Սավուղ տատան}

\armenian{Օրնէ ըս մեր հէօրսն ու փեսան։}

\armenian{Օրնէ թոգովէօր, օրնէ սաղտուճ},

\armenian{Օրնէ ամնէքս ալ միարան։}

\armenian{Ան վէօէ էր օր ընկավ հէօրը},

\armenian{Վրան լցին քարն ու փէօտը։}

\armenian{Լուսաւորիչ Գիրգէօր հէօրը}

\armenian{Օրնէ ըս մեր թոգովէօրը։}



\chapter{Istanbul}\label{chapter:istanbul}
\section{Background}

\begin{adjarianpage}\label{page:249}\end{adjarianpage}% should be 249

The Istanbul dialect is spoken in the city of Constantinople, and in the villages that lie between the two shores of the Bosporus and the Golden Horn. Just as Tbilisi is the center of the Eastern literature, so is Istanbul the center of Western literature; the Istanbul dialect has served as a basis for the formation of the Western literary language. Keeping in mind this large role, it is surprising that the Istanbul dialect has still not been studied in detail. However, there are innumerable writings where the Istanbul dialect has been written down, with small or large relevance or authenticity.\footnote{\translatorHD{The original phrase is <\armenian{քիշ կամ շատ հարազատութեամբ}>; I'm not completely sure what this phrase means though because it seems to be some idiomatic use.}} When the Civil language of Armenian (Civil Armenian, \armenian{աշխարհաբար}) was first established, the newspapers and books that were published, whether in Istanbul, Venice, or Smyrna, were written in the colloquial language of the plebeian class (\armenian{ռամիկ դաս}) of Istanbul. Armenian writers bit by bit cleaned it up with Classical Armenian, and they created the new literary language. 


\section{Phonology}
\subsection{Segment inventory}
\subsubsection{Vowels}\label{sec:Istanbul:phono:segment:vowel}
The sound system of the Istanbul dialect has the following 8 vowels: /ɑ, e, ə, i, o, œ, u, ʏ/ <\armenian{ա, է, ը, ի, օ, էօ, ու, իւ}>. 

The sound /æ/ <\armenian{ա}̈>, which is found in many other dialects, does not exist here. 

Similarly, the differences between the sounds /i̯e, e/ <\armenian{ե, է}> and /u̯o, o/ <\armenian{ո, օ}> are missing here. 

The sound /ʏ/ <\armenian{իւ}> is founded in Turkish loanwords, the literary language of Istanbul uses it in place of the Old Armenian /iu̯/ <\armenian{իւ}> diphthong, next to a consonant. 
 For example, the words in Table \ref{tab:Istanbul:phonology:seg:vowel:y} are pronounced with /ʏ/ in the Istanbul literary dialect, while the plebeian (\armenian{ռամիկ}) dialect uses /d͡zun/ `snow' <\armenian{ձուն}>. 


\begin{table}[H]
 	\centering 
 	\caption{Emergence of /ʏ/ <\armenian{իւ}> in the literary Istanbul dialect}
 	\label{tab:Istanbul:phonology:seg:vowel:y}
 	\begin{tabular}{|l | ll|ll| ll|}
 		\hline & \multicolumn{2}{l|}{Classical Armenian} &\multicolumn{2}{l|}{> literary Istanbul} & \multicolumn{2}{l|}{cf. SEA} \\ 
`snow' & d͡ziu̯n & \armenian{ձիւն} & t͡sʰʏn & \armenian{ցիւն} & d͡zjun & \armenian{ձյուն} \\
`column' & siu̯n & \armenian{սիւն} & sʏn & \armenian{սիւն} & sjun & \armenian{սյուն} \\
 \hline 
 	\end{tabular}
\end{table}


In general, the sound /œ/ <\armenian{էօ}> is absent from the literary language, while in the popular language it exists and it is used instead of the sounds /e, o/ <\armenian{է,օ}>, if there is a sound /o/ <\armenian{օ}> and /e/ <\armenian{է}> before or after them (Table \ref{tab:Istanbul:phonology:seg:vowel:oe}). 



\begin{table}[H]
 	\centering 
 	\caption{Emergence of /œ/ <\armenian{էօ}> in the Istanbul dialect}
 	\label{tab:Istanbul:phonology:seg:vowel:oe}
 	\begin{tabular}{|l | ll|ll| ll|}
 		\hline & \multicolumn{2}{l|}{Classical Armenian} &\multicolumn{2}{l|}{> Istanbul} & \multicolumn{2}{l|}{cf. SEA} \\ 
`wheat' &t͡sʰoɾe̯ɑn & \armenian{ցորեան} & t͡sʰœɾen & \armenian{ցէօրէն} & t͡sʰoɾen& \armenian{ցորեն} \\
 `daytime' &t͡sʰeɾek, t͡sʰoɾek & \armenian{ցերեկ, ցորեկ} &t͡sʰœɾeɡ & \armenian{ցէօրէգ} &t͡sʰeɾek & \armenian{ցերեկ} \\
 `cherub' &kʰeɾovbē & \armenian{քերովբէ} &kʰœɾœpʰe & \armenian{քէօրէոփէ} &kʰeɾovbe & \armenian{քերովբե} \\
 `seraph' &seɾovbē & \armenian{սերովբէ} &sœɾœpʰe & \armenian{սէօրէօփէ} &seɾovbe & \armenian{սերովբե}\\
\hline 
 	\end{tabular}
\end{table}
 
\begin{adjarianpage}\label{page:250}\end{adjarianpage}% should be 250


There are no diphthongs in the Istanbul dialect.

\subsubsection{Consonants}

The consonants have two degrees: voiced and voiceless aspirated. However it must be noted that the voiced consonants of Istanbul are like voiced sounds of German, and for example a French listener would perceive them as voiceless unaspirated. When the sound is given emphasis\footnote{\translatorHD{The original phrase is <\armenian{ուղուի ձայնին սաստկութիւն տալ}>; my translation is my best guess on how to interpret the original.}}, then the voicing can increase and approach the degree of French voiced sounds. This is such that there many times when the same person pronounces the same word, sometimes as voiceless unaspirated, and sometimes as very voiced. For details and a study of the pronunciation of these sounds in the gloss, see my work \citep{Adjarian-1899-ArmenianExplosives}. 

\subsection{Sound changes}\label{sec:Istanbul:phono:soundchange}
The sound changes in the Istanbul dialect are not big. Although the Istanbul dialect is very far from the borders of the Armenian country, but it is much more faithful to Old Armenian, then many of the dialects in the Armenian country. 

\subsubsection{Monophthong vowel changes}

The vowels have generally preserved the old pronunciations:
\begin{itemize}
 \item CA /ɑ/ <\armenian{ա}> $\rightarrow$ /ɑ/ <\armenian{ա}> 
 \item CA /e, ē/ <\armenian{ե, է}> $\rightarrow$ /e/ <\armenian{է}> (in every situations)
 \item CA /ə/ <\armenian{ը}> $\rightarrow$ /ə/ <\armenian{ը}> 
 \item CA /i/ <\armenian{ի}> $\rightarrow$ /i/ <\armenian{ի}> 
 \item CA /o, ɑu̯/ <\armenian{ո, օ (աւ})> $\rightarrow$ /o/ <\armenian{օ}> (in every situations)
 \item CA /u/ <\armenian{ու}> $\rightarrow$ /u/ <\armenian{ու}> 
\end{itemize}

\subsubsection{Diphthongal vowel changes}\label{sec:Istanbul:phono:soundchange:diph}

The diphthongs became simple vowels:
\begin{itemize}
 \item CA /ɑi̯/ <\armenian{այ}> $\rightarrow$ /ɑ/ <\armenian{ա}> 
 \item CA /e̯ɑ/ <\armenian{եա}> $\rightarrow$ /e/ <\armenian{է}> 
 \item CA /iu̯/ <\armenian{իւ}> $\rightarrow$ /u/ <\armenian{ու}> 
 \item CA /oi̯̯/ <\armenian{ոյ}> $\rightarrow$ /u/ <\armenian{ու}> 
 \end{itemize}


 
 For example in Table \ref{tab:Istanbul:phonology:change:dipth}. 
 
 
 \begin{table}[H]
 	\centering 
 	\caption{Reduction of Classical Armenian diphthongs in the Istanbul dialect}
 	\label{tab:Istanbul:phonology:change:dipth}
 	\begin{tabular}{|l | ll|ll| ll|}
 		\hline & \multicolumn{2}{l|}{Classical Armenian} &\multicolumn{2}{l|}{> Istanbul} & \multicolumn{2}{l|}{cf. SEA} \\ 
`father' & hɑi̯ɾ & \armenian{հայր}& hɑɾ & \armenian{հար} & hɑjɾ & \armenian{հայր} \\ 
 `black'& se̯ɑu̯ & \armenian{սեաւ} & sev & \armenian{սէվ} & sev & \armenian{սև} \\
`snow' & d͡ziu̯n & \armenian{ձիւն} & d͡zun & \armenian{ձուն} & d͡zjun & \armenian{ձյուն} \\
`light' & loi̯s & \armenian{լոյս} & lus & \armenian{լուս} & lujs & \armenian{լույս} \\ 
 `sister' & kʰoi̯ɾ & \armenian{քոյր} & kʰuɾ & \armenian{քուր} & kʰujɾ & \armenian{քույր} \\
\hline 
 	\end{tabular}
\end{table}

 
 Next to vowels or alone, these become a vowel + consonant (Table \ref{tab:Istanbul:phonology:change:dipthGl}). 
 
 
 \begin{table}[H]
 	\centering 
 	\caption{Splitting of Classical Armenian diphthongs to vowel + glide sequences in the Istanbul dialect}
 	\label{tab:Istanbul:phonology:change:dipthGl}
 	\begin{tabular}{|l | ll|ll| ll|}
 		\hline & \multicolumn{2}{l|}{Classical Armenian} &\multicolumn{2}{l|}{> Istanbul} & \multicolumn{2}{l|}{cf. SEA} \\ 
 \armenian{՝}Armenian-{\gen}' & hɑ{j-oi̯} & \armenian{հայոյ} & hɑj-u & \armenian{հայու} & hɑj-i & \armenian{հայի} \\
 `to look at' & nɑ{ji}l & \armenian{նայիլ} & nɑjil & \armenian{նայիլ} & nɑjel& \armenian{նայել} \\
`sick' & hi{wɑ}nd & \armenian{հիւանդ} & hivɑntʰ & \armenian{հիվանդ} & hivɑnd & \armenian{հիվանդ} \\ 
\hline 
 	\end{tabular}
\end{table}

\translatorHD{Note that Adjarian seems to treat an intervocalic glide in Classical Armenian as part of a diphthong: /hɑ{j-oi̯}/ instead of /hɑ{i̯-oi̯} <\armenian{հայոյ}>. But, I'm very skeptical of such a treatment for Classical Armenian, simply because such an analysis creates unclear syllable boundaries. See discussion in \S\ref{sec:HossepIntro:phono:Classical:Diphthong}. }

 \subsubsection{Consonant changes}
 \paragraph{Laryngeal changes}\label{sec:Istanbul:phono:soundchange:cons:laryngeal}

 For the consonants, the voiced stay voiced, but they became voiceless aspirates after the sound /ɾ/ <\armenian{ր}>. The voiceless unaspirates became voiced everywhere. The voiceless aspirates stay the same. 

 \paragraph{Word-initial uvular fricatives}

The word-initial sound /ʁ/ <\armenian{ղ}> is not known in the Istanbul dialect; and whenever this sound occurs at the beginning of the word, it becomes /χ/ <\armenian{խ}> Even the name of the letter /ʁ/ <\armenian{ղ}> has changed (Table \ref{tab:Istanbul:phonology:change:gh}).\footnote{\translatorHD{For the word `to guide', Adjarian includes a reconstructed intermediate form */ʁɑu̯ɾel/ *<\armenian{ղաւրել}>. } } 


 \begin{table}[H]
 	\centering 
 	\caption{Absence of word-initial Classical Armenian /ʁ/ <\armenian{ղ}> in the Istanbul dialect}
 	\label{tab:Istanbul:phonology:change:gh}
 	\begin{tabular}{|l | ll|ll| lll|}
 		\hline & \multicolumn{2}{l|}{Classical Armenian} &\multicolumn{2}{l|}{> Istanbul} & \multicolumn{3}{l|}{cf. SEA or SWA} \\ 
`to send' & uɬɑɾkel & \armenian{ուղարկել} & χɾkel & \armenian{խրկէլ} & uʁɑɾɡel, ʁəɾɡel & \armenian{ուղարկել, ղրկել} & SWA \\ 
`to guide' & uɬe{we}ɾel & \armenian{ուղեւորել} & χɑχɾel & \armenian{խախրէլ} & uʁevoɾel & \armenian{ուղեւորել} & SEA\\
`Luke' & ɬɑkɑs & \armenian{Ղուկաս} & χuɡɑs & \armenian{Խուգաս} & ʁukɑs & \armenian{Ղուկաս}& SEA \\ 
`Lazaros' & ɬɑzɑɾos & \armenian{Ղազարոս} & χɑzɑɾos & \armenian{Խազարօս} & ʁɑzɑɾos & \armenian{Ղազարոս}& SEA \\ 
`name of letter <\armenian{ղ}>' & ɬɑt & \armenian{ղատ} & χɑzɑɾos & χɑd & \armenian{խադ} & \armenian{ղատ} & SEA \\ 

\hline 
 	\end{tabular}
\end{table}



\section{Morphology}
\subsection{Noun inflection or declension}\label{sec:Istanbul:morpho:noun}
In case declension, a strong simplification has been introduced. There are only four cases: nominative-accusative, genitive-dative, ablative, and instrumental. The plural is formed with the formatives /-eɾ/ <\armenian{էր}> or /-neɾ/ <\armenian{նէր}>. The following is the general picture of declension (Table \ref{tab:Istanbul:morpho:noun:case}). 



\begin{adjarianpage}\label{page:251}\end{adjarianpage}% should be 251



\begin{table}[H]
\centering 
\caption{Case declension for the word `bread' in the Istanbul dialect}
\label{tab:Istanbul:morpho:noun:case}
\begin{tabular}{|l|ll|ll|}
\hline & \multicolumn{2}{l|}{Singular} & \multicolumn{2}{l|}{Plural} \\ \hline 
{\nom}-{\acc} & hɑt͡sʰ & \armenian{հաց} & hɑt͡sʰ-eɾ & \armenian{հաց-էր} \\
{\gen}-{\dat} & hɑt͡sʰ-i & \armenian{հաց-ի} & hɑt͡sʰ-eɾ-u & \armenian{հաց-էրու} \\
{\abl} & hɑt͡sʰ-e & \armenian{հաց-է} & hɑt͡sʰ-eɾ-e & \armenian{հաց-էր-է} \\
{\ins} & hɑt͡sʰ-ov & \armenian{հաց-օվ} & hɑt͡sʰ-eɾ-ov & \armenian{հաց-էր-օվ} \\ \hline
\end{tabular}\end{table}

Except for a few exceptions, which form their own declension classes (among these, a very notable class are the words with suffix /-utʰin/ <-\armenian{ութին}>, from CA /-utʰiu̯n/ <\armenian{ութիւն}>), all the remaining words follow this declension even the words in (Table \ref{tab:Istanbul:morpho:noun:irr}) (\translatorHD{that are irregular in SEA/SWA}), and words with a rime /i/ <\armenian{ի}>. 



 \begin{table}[H]
 	\centering 
 	\caption{Regular declension for various words in the Istanbul dialect}
 	\label{tab:Istanbul:morpho:noun:irr}
 	\begin{tabular}{|l | ll| ll|}
 		\hline &\multicolumn{2}{l|}{Istanbul} & \multicolumn{2}{l|}{cf. SEA} \\ 
`dog' & & & ʃun & \armenian{շուն} \\ 
`dog ({\gen})' & ʃun-i& \armenian{շունի} & ʃɑn & \armenian{շան} \\ 
\armenian{՝}house' & & & tun & \armenian{տուն} \\
\armenian{՝}house ({\gen})' & dun-i &\armenian{դունի} & tɑn & \armenian{տան} \\
\armenian{՝}house ({\abl})' & tun-e &\armenian{տունէ} & tən-it͡sʰ & \armenian{տնից} \\
`mouse' & & & muk & \armenian{մուկ} \\ 
`mouse ({\gen})' & muɡ-i &\armenian{մուգի} & mək-ɑn & \armenian{մկան} \\ 
`fish' & & & d͡zuk & \armenian{ձուկ} \\ 
`fish ({\gen})' &d͡zuɡ-i & \armenian{ձուգի} & d͡zək-ɑn & \armenian{ձկան} \\ 
`fish ({\abl})' & d͡zuɡ-e &\armenian{ձուգէ} & d͡zəkn-it͡sʰ & \armenian{ձկնից} \\ 
`wine' & & & ɡini & \armenian{գինի} \\ 
`wine ({\gen})' & ɡini-i & \armenian{գինիի} & ɡin-u & \armenian{գինու} \\ 
`barley' & & & ɡɑɾi & \armenian{գարի} \\ 
`barley ({\gen})' &ɡɑɾi-i & \armenian{գարիի} & ɡɑɾ-u & \armenian{գարու} \\ 
\hline 
 	\end{tabular}
\end{table}

\subsection{Pronoun inflection or declension}


The following is the general picture of the pronoun declension (Table \ref{tab:Istanbul:morpho:pronoun:personal}). 

 \begin{table}[H]
 	\centering 
 	\caption{Declension paradigm for personal pronouns in the Istanbul dialect}
 	\label{tab:Istanbul:morpho:pronoun:personal}
\begin{tabular}{|l|lll| lll|}
\hline & 1SG & 2SG & 3SG & 1PL & 2PL & 3PL 
\\
& `I' & `you' & `he' & `we' & `you' & `they' \\ 
\hline 
{\nom} & es & dun & ɑn & menkʰ & dukʰ & ɑnonkʰ \\
 & \armenian{էս} & \armenian{դուն} & \armenian{ան} & \armenian{մէնք} & \armenian{դուք} & \armenian{անօնք} \\\hline 
{\gen} & im & kʰu & ɑnoɾ & meɾ & d͡zeɾ & ɑnont͡sʰ \\
 & \armenian{իմ} & \armenian{քու} & \armenian{անօր} & \armenian{մէր} & \armenian{ձէր} & \armenian{անօնց} \\\hline 
{\dat} & ənd͡z-i & kʰez-i & ɑnoɾ & mez-i & d͡zez-i & ɑnont͡sʰ \\
 & \armenian{ընձի} & \armenian{քէզի} & \armenian{անօր} & \armenian{մէզի} & \armenian{ձէզի} & \armenian{անօնց} \\\hline 
{\acc} & is, ənd͡z-i & kʰez-i& ɑn & mez-i & d͡zez-i & ɑnonkʰ \\
 & \armenian{իս, ընձի} & \armenian{քէզի} & \armenian{ան} & \armenian{մէզի} & \armenian{ձէզի} & \armenian{անօնք} \\\hline 
{\abl} & iz-me & kʰez-me & ɑn-ɡe & mez-me & d͡zez-me & ɑnont͡sʰ-me \\
 & \armenian{իզմէ} & \armenian{քէզմէ} & \armenian{անգէ} & \armenian{մէզմէ} & \armenian{ձէզմէ} & \armenian{անօնցմէ} \\\hline 
{\ins} & iz-mov & kʰez-mov & ɑn-ov& mez-mov & d͡zez-mov & ɑnont͡sʰ-mov \\
 & \armenian{իզմօվ} & \armenian{քէզմօվ} & \armenian{անօվ} & \armenian{մէզմօվ} & \armenian{ձէզմօվ} & \armenian{անօնցմօվ} \\ \hline 
\end{tabular}\end{table}

\translatorHD{Adjarian provides the following paradigm for demonstrative proximal pronoun `this' (Table \ref{tab:Istanbul:morpho:pronoun:dem}). }

\begin{table}[H]
 	\centering 
 	\caption{Declension paradigm for the demonstrative proximal pronoun `this' in the Istanbul dialect}
 	\label{tab:Istanbul:morpho:pronoun:dem}
\begin{tabular}{| l| ll| ll| }
\hline & \multicolumn{2}{l|}{Singular `this'} & \multicolumn{2}{l|}{Plural `these'}\\
\hline 
{\nom} & sɑ & \armenian{սա} & s(ɑ)vonkʰ & \armenian{ս(ա)վօնք} \\
{\gen} & səvoɾ & \armenian{սըվօր} & s(ɑ)vont͡sʰ & \armenian{ս(ա)վօնց} \\
{\dat} & səvoɾ & \armenian{սըվօր} & s(ɑ)vont͡sʰ & \armenian{ս(ա)վօնց} \\
{\acc} & səviɡɑ & \armenian{սըվիգա} & s(ɑ)vontkʰ & \armenian{ս(ա)վօնք} \\
{\abl}& sə(v)-ɡə & \armenian{սը(վ)գէ} & s(ɑ)vont͡sʰ-me & \armenian{ս(ա)վօնցմէ} \\
 
{\ins} & sə(v)-ɡov & \armenian{սըվօվ} & s(ɑ)vont͡sʰ-mov & \armenian{ս(ա)վօնցմօվ} \\
 \hline 
\end{tabular}
\end{table}

There are also the forms in Table (\ref{tab:Istanbul:morpho:pronoun:demsample}). These are all declined simply based on the pronoun `this' /sɑ/ <\armenian{սա}>. 


\begin{table}[H]
\centering
 	\caption{Sample of demonstrative nominative singular pronouns in the Istanbul dialect}
 	\label{tab:Istanbul:morpho:pronoun:demsample}
\begin{tabular}{| ll|ll|ll|}
\hline \multicolumn{2}{|l|}{Proximal `this'} &
 \multicolumn{2}{l|}{Medial `that'} & 
 \multicolumn{2}{l|}{Distal `that yonder'}
\\\hline 
ɑs & \armenian{աս} & ɑd & \armenian{ադ} & & \\
 & & dɑ & \armenian{դա} & nɑ & \armenian{նա} \\
ɑsiɡɑ & \armenian{ասիգա} & ɑdiɡɑ & \armenian{ադիգա} & ɑniɡɑ & \armenian{անիգա} \\
ɑsiɡɑɡ & \armenian{ասիգագ} & ɑdiɡɑɡ & \armenian{ադիգագ} & ɑniɡɑɡ & \armenian{անիգագ} \\
sɑviɡɑ & \armenian{սավիգա} & dɑviɡɑ & \armenian{դավիգա} & nɑviɡɑ & \armenian{նավիգա} \\
sɑviɡɑɡ & \armenian{սավիգագ} & dɑviɡɑɡ & \armenian{դավիգագ} & nɑviɡɑɡ & \armenian{նավիգագ} \\
sviɡɑ & \armenian{սվիգա} & dviɡɑ & \armenian{դվիգա} & nviɡɑ & \armenian{նվիգա} \\
sviɡɑɡ & \armenian{սվիգագ} & dviɡɑɡ & \armenian{դվիգագ} & nviɡɑɡ & \armenian{նվիգագ} \\ \hline
\end{tabular}
\end{table}


\subsection{Verb inflection or conjugation}

For verb conjugation classes, what remains is only /-el, -il, -ɑl/ <\armenian{էլ, իլ, ալ}>, and /-nel, -nil, -nɑl/ <\armenian{նէլ, նիլ, նալ}>. We place here the conjugation of the verb `to like', as an example of the first conjugation class. 


{\paradigmExplanation}

\begin{adjarianpage}\label{page:252}\end{adjarianpage}% should be 252

\subsubsection{Subjunctive present and past imperfective}

\translatorHD{In SWA (Table \ref{tab:Istanbul:morpho:verb:paradigm:subjPresent}), the subjunctive present is a finite verb form made up of the verb stem, plus a theme vowel, plus agreement suffixes. For a verb like `to like', the theme vowel is a constant /-e-/. The Istanbul dialect uses an identical morphological strategy. }

\begin{table}[H]
 \centering
 \caption{Subjunctive present <\armenian{ստորադասական ներկայ}> of the verb `to like' in the Istanbul dialect}
 \label{tab:Istanbul:morpho:verb:paradigm:subjPresent}
 \begin{tabular}{|l|ll|ll|}
\hline & \multicolumn{2}{l|}{Istanbul } & \multicolumn{2}{l|}{cf. SWA} \\
1SG & siɾ-e-m & \armenian{սիրէմ} & siɾ-e-m & \armenian{սիրեմ} \\
2SG & siɾ-e-s & \armenian{սիրէս} & siɾ-e-s & \armenian{սիրես} \\
3SG & siɾ-e-$\emptyset$ & \armenian{սիրէ} & siɾ-e-$\emptyset$ & \armenian{սիրէ} \\
1PL & siɾ-e-nkʰ & \armenian{սիրէնք} &siɾ-e-ŋkʰ & \armenian{սիրենք} \\
2PL & siɾ-e-kʰ & \armenian{սիրէք} & siɾ-e-kʰ & \armenian{սիրէք} \\
3PL & siɾ-e-n & \armenian{սիրէն} & siɾ-e-n & \armenian{սիրեն} \\
& \multicolumn{2}{l|}{$\sqrt{}$-{\thgloss}-{\agr}}& \multicolumn{2}{l|}{$\sqrt{}$-{\thgloss}-{\agr}}\\ 

\hline 
\end{tabular}
\end{table}

\translatorHD{In SWA, the subjunctive past imperfective (Table \ref{tab:Istanbul:morpho:verb:paradigm:subjPast}) is formed by adding the past suffix /i/ and agreement suffixes after the theme vowel. The past suffix is absent in the 3SG. Istanbul again uses the same strategy. }



\begin{table}[H]
 \centering
 \caption{Subjunctive past <\armenian{ստորադասական անցեալ}> of the verb `to like' in the Istanbul dialect}
 \label{tab:Istanbul:morpho:verb:paradigm:subjPast}
 \begin{tabular}{|l|ll|ll|}
\hline & \multicolumn{2}{l|}{Istanbul} & \multicolumn{2}{l|}{cf. SWA} \\
1SG & siɾ-e-i-$\emptyset$ & \armenian{սիրէի} & siɾ-ej-i-$\emptyset$ & \armenian{սիրէի} \\
2SG & siɾ-e-i-ɾ & \armenian{սիրէիր} & siɾ-ej-i-ɾ & \armenian{սիրէիր} \\
3SG & siɾ-e-$\emptyset$-ɾ & \armenian{սիրէր} & siɾ-e-$\emptyset$-ɾ & \armenian{սիրէր} \\
1PL & siɾ-e-i-nkʰʲ & \armenian{սիրէինք} & siɾ-ej-i-ŋkʰ & \armenian{սիրէինք} \\
2PL & siɾ-e-i-kʰʲ & \armenian{սիրէիք} & siɾ-ej-i-kʰ & \armenian{սիրէիք} \\
3PL & siɾ-e-i-n & \armenian{սիրէին} &siɾ-ej-i-n & \armenian{սիրէին} \\
& \multicolumn{2}{l|}{$\sqrt{}$-{\thgloss}-{\pst}-{\agr}}& \multicolumn{2}{l|}{$\sqrt{}$-{\thgloss}-{\pst}-{\agr}}\\ 

\hline 
\end{tabular}
\end{table}




\subsubsection{Tenses built from the subjunctive: Indicative, progressive, and future }

 \translatorHD{In Istanbul and SWA, many other tenses seem to be built off of the subjunctive (Table \ref{tab:Istanbul:morpho:verb:paradigm:complexSubjunctive}). The indicative present and past imperfective are built by adding the prefix /ɡə-/ before the subjunctive present and subjunctive past. The progressive is formed by adding the enclitic /ɡoɾ/ after the indicative forms. The future and future perfect are formed also by adding the proclitic /bidi/ before the appropriate subjunctive form. SWA behaves exactly the same and I don't provide its paradigm. }

\begin{table}[H]
 \centering
 \caption{Forms that are built from the subjunctive forms for the verb `to like' in the Istanbul dialect}
 \label{tab:Istanbul:morpho:verb:paradigm:complexSubjunctive}
 \begin{tabular}{|l|ll|ll|}
\hline & 
\multicolumn{2}{l|}{Indicative present <\armenian{ներկայ}> } & \multicolumn{2}{l|}{Indicative past imperfective <\armenian{անկատար}>} \\
1SG & ɡə siɾ-e-m & \armenian{գը սիրէմ} & ɡə siɾ-e-i-$\emptyset$ & \armenian{գը սիրէի} \\
2SG & ɡə siɾ-e-s & \armenian{գը սիրէս} & ɡə siɾ-e-i-ɾ & \armenian{գը սիրէիր} \\
3SG & ɡə siɾ-e-$\emptyset$ & \armenian{գը սիրէ} & ɡə siɾ-e-$\emptyset$-ɾ & \armenian{գը սիրէր} \\
1PL & ɡə siɾ-e-nkʰ & \armenian{գը սիրէնք} & ɡə siɾ-e-i-nkʰ & \armenian{գը սիրէինք} \\
2PL & ɡə siɾ-e-kʰ & \armenian{գը սիրէք} & ɡə siɾ-e-i-kʰ & \armenian{գը սիրէիք} \\
3PL & ɡə siɾ-e-n & \armenian{գը սիրէն} & ɡə siɾ-e-i-n & \armenian{գը սիրէին} 
\\
& \multicolumn{2}{l|}{{\ind}-$\sqrt{}$-{\thgloss}-{\agr}}& \multicolumn{2}{l|}{{\ind}-$\sqrt{}$-{\thgloss}-{\pst}-{\agr}}
\\ \hline & 
\multicolumn{2}{l|}{Present progressive <\armenian{շարունական ներկայ}> } & \multicolumn{2}{l|}{Past imperfective progressive <\armenian{շարունական անկատար}>} \\
1SG & ɡə siɾ-e-m ɡoɾ & \armenian{գը սիրէմ գօր} & ɡə siɾ-e-i-$\emptyset$ ɡoɾ& \armenian{գը սիրէի} \\
2SG & ɡə siɾ-e-s ɡoɾ & \armenian{գը սիրէս գօր} & ɡə siɾ-e-i-ɾ ɡoɾ & \armenian{գը սիրէիր գօր} \\
3SG & ɡə siɾ-e-$\emptyset$ ɡoɾ & \armenian{գը սիրէ գօր} & ɡə siɾ-e-$\emptyset$-ɾ ɡoɾ & \armenian{գը սիրէր գօր} \\
1PL & ɡə siɾ-e-nkʰ ɡoɾ & \armenian{գը սիրէնք գօր} & ɡə siɾ-e-i-nkʰ ɡoɾ & \armenian{գը սիրէինք գօր} \\
2PL & ɡə siɾ-e-kʰ ɡoɾ & \armenian{գը սիրէք գօր} & ɡə siɾ-e-i-kʰ ɡoɾ & \armenian{գը սիրէիք գօր} \\
3PL & ɡə siɾ-e-n ɡoɾ & \armenian{գը սիրէն գօր} & ɡə siɾ-e-i-n ɡoɾ & \armenian{գը սիրէին գօր}
\\
& \multicolumn{2}{l|}{{\ind}-$\sqrt{}$-{\thgloss}-{\agr} {\prog}}& \multicolumn{2}{l|}{{\ind}-$\sqrt{}$-{\thgloss}-{\pst}-{\agr} {\prog}}
\\ \hline 
& \multicolumn{2}{l|}{Future <\armenian{ապառնի}>} & \multicolumn{2}{l|}{Future perfect <\armenian{անցեալ ապառնի}> } \\
1SG & bidi siɾ-e-m & \armenian{բիդի սիրէմ} & bidi siɾ-e-i-$\emptyset$ & \armenian{բիդի սիրէի} \\
2SG & bidi siɾ-e-s & \armenian{բիդի սիրէս} & bidi siɾ-e-i-ɾ & \armenian{բիդի սիրէիր} \\
3SG & bidi siɾ-e-$\emptyset$ & \armenian{բիդի սիրէ} & bidi siɾ-e-$\emptyset$-ɾ & \armenian{բիդի սիրէր} \\
1PL & bidi siɾ-e-nkʰ & \armenian{բիդի սիրէնք} & bidi siɾ-e-i-nkʰ & \armenian{բիդի սիրէինք} \\
2PL & bidi siɾ-e-kʰ & \armenian{բիդի սիրէք} & bidi siɾ-e-i-kʰ & \armenian{բիդի սիրէիք} \\
3PL & bidi siɾ-e-n & \armenian{բիդի սիրէն} & bidi siɾ-e-i-n & \armenian{բիդի սիրէին} 
\\
& \multicolumn{2}{l|}{{\fut} $\sqrt{}$-{\thgloss}-{\agr}}& \multicolumn{2}{l|}{{\fut} $\sqrt{}$-{\thgloss}-{\pst}-{\agr}}
\\\hline \end{tabular}
\end{table}

\subsubsection{Present perfect and past perfect}

\translatorHD{In SWA, the present perfect (Table \ref{tab:Istanbul:morpho:verb:paradigm:presentPerfect}) and past perfect (Table \ref{tab:Istanbul:morpho:verb:paradigm:pastPerfect}) in are formed by combining a special non-finite form with the present/past auxiliary. For SWA, this non-finite verb can be either the resultative participle (verb with suffix /-ɑd͡z/) or the evidential participle (verb with suffix /-eɾ/). Istanbul uses a similar system. Adjarian only provides a participle with the suffix /-eɾ/ <\armenian{էր}>. Adjarian doesn't state if this suffix has evidential meaning or not, but I suspect it does. }

\begin{table}[H]
 \centering
 \caption{Present perfect <\armenian{յարակատար}> of the verb `to like' in the Istanbul dialect}
 \label{tab:Istanbul:morpho:verb:paradigm:presentPerfect}
 \begin{tabular}{|l|ll|ll|}
\hline & \multicolumn{2}{l|}{Istanbul} & \multicolumn{2}{l|}{cf. SWA} \\
1SG & siɾ-eɾ e-m & \armenian{սիրէր էմ} & siɾ-eɾ e-m & \armenian{սիրեր եմ} \\
2SG &siɾ-eɾ e-s & \armenian{սիրէր էս} & siɾ-eɾ e-s & \armenian{սիրեր ես} \\
3SG & siɾ-eɾ e-$\emptyset$ & \armenian{սիրէր է} & siɾ-eɾ e-$\emptyset$ & \armenian{սիրեր է} \\
1PL & siɾ-eɾ e-nkʰ & \armenian{սիրէր էնք} & siɾ-eɾ e-ŋkʰ & \armenian{սիրեր ենք} \\
2PL & siɾ-eɾ e-kʰ & \armenian{սիրէր էք} & siɾ-eɾ e-kʰ & \armenian{սիրեր էք} \\
3PL & siɾ-eɾ e-n & \armenian{սիրէր էն} & siɾ-eɾ e-n & \armenian{սիրեր են} \\
& \multicolumn{2}{l|}{$\sqrt{}$-{\eptcp} {\aux}-{\agr}}& \multicolumn{2}{l|}{$\sqrt{}$-{\eptcp} {\aux}-{\agr}}\\ 

\hline 
\end{tabular}
\end{table}


\begin{table}[H]
 \centering
 \caption{Past perfect <\armenian{գերակատար}> of the verb `to like' in the Istanbul dialect}
 \label{tab:Istanbul:morpho:verb:paradigm:pastPerfect}
 \begin{tabular}{|l|ll|ll| }
\hline & \multicolumn{2}{l|}{Istanbul} & \multicolumn{2}{l|}{cf. SWA} \\
1SG & siɾ-eɾ e-i-$\emptyset$ & \armenian{սիրէր էի} & siɾ-eɾ ej-i-$\emptyset$ & \armenian{սիրեր էի} \\
2SG & siɾ-eɾ e-i-ɾ & \armenian{սիրէր էիր} & siɾ-eɾ ej-i-ɾ & \armenian{սիրեր էիր} \\
3SG & siɾ-eɾ e-$\emptyset$-ɾ & \armenian{սիրէր էր} & siɾ-eɾ e-$\emptyset$-ɾ & \armenian{սիրեր էր} \\
1PL & siɾ-eɾ e-i-nkʰ & \armenian{սիրէր էինք} & siɾ-eɾ ej-i-ŋkʰ & \armenian{սիրեր էինք} \\
2PL & siɾ-eɾ e-i-kʰ & \armenian{սիրէր էիք} & siɾ-eɾ ej-i-kʰ & \armenian{սիրեր էիք} \\
3PL & siɾ-eɾ e-i-n & \armenian{սիրէր էին} & siɾ-eɾ ej-i-n & \armenian{սիրեր էին} \\
& \multicolumn{2}{l|}{$\sqrt{}$-{\eptcp} {\aux}-{\pst}-{\agr}}& \multicolumn{2}{l|}{$\sqrt{}$-{\eptcp} {\aux}-{\pst}-{\agr}}\\ 

\hline 
\end{tabular}
\end{table}


\subsubsection{Past perfective or aorist}

\translatorHD{The past perfective (Table \ref{tab:Istanbul:morpho:verb:paradigm:pastperfectiveAorist}) is also called the aorist. In SWA for /siɾ-e-l/ `to like', the past perfective is formed by taking the root and theme vowel, adding the aorist or perfective suffix /-t͡sʰ-/, and then adding the past suffix /-i/ and the appropriate agreement suffixes. The 3SG uses covert tense and agreement suffixes. The Istanbul dialect again behaves identically. }


\begin{table}[H]
 \centering
 \caption{Past perfective or aorist <\armenian{կատարեալ}> of the verb `to like' in the Istanbul dialect}
 \label{tab:Istanbul:morpho:verb:paradigm:pastperfectiveAorist}
 \begin{tabular}{|l|ll|ll|}
\hline & \multicolumn{2}{l|}{Istanbul} & \multicolumn{2}{l|}{cf. SWA} \\
1SG & siɾ-e-t͡sʰ-i-$\emptyset$ & \armenian{սիրէցի} & siɾ-e-t͡sʰ-i-$\emptyset$ & \armenian{սիրեցի} \\
2SG & siɾ-e-t͡sʰ-i-ɾ & \armenian{սիրէցիր} & siɾ-e-t͡sʰ-i-ɾ & \armenian{սիրեցիր} \\
3SG & siɾ-e-t͡sʰ-$\emptyset$-$\emptyset$ & \armenian{սիրէց} & siɾ-e-t͡sʰ-$\emptyset$-$\emptyset$ & \armenian{սիրեց} \\
1PL & siɾ-e-t͡sʰ-i-nkʰ & \armenian{սիրէցինք} &siɾ-e-t͡sʰ-i-ŋkʰ & \armenian{սիրեցինք} \\
2PL & siɾ-e-t͡sʰ-i-kʰ & \armenian{սիրէցիք} & siɾ-e-t͡sʰ-i-kʰ & \armenian{սիրեցիք} \\
3PL & siɾ-e-t͡sʰ-i-n & \armenian{սիրէցին} &siɾuz-e-t͡sʰ-i-n & \armenian{սիրեցին} \\
& \multicolumn{2}{l|}{$\sqrt{}$-{\thgloss}-{\aor}-{\pst}-{\agr}}& \multicolumn{2}{l|}{$\sqrt{}$-{\thgloss}-{\aor}-{\pst}-{\agr}}\\ 

\hline 
\end{tabular}
\end{table}
\subsubsection{Imperative and prohibitive}

\translatorHD{For the imperative 2SG, SWA adds a zero morph /-$\emptyset$/ after the theme vowel /e/ for a verb like `to like' (Table \ref{tab:Istanbul:morpho:verb:paradigm:Imp}). For the 2PL, SWA adds the sequence /-e-t͡sʰ-ekʰ/ after the root such that /-e-t͡sʰ/ forms the aorist stem, while /-ekʰ/ is the agreement marker. Istanbul again does the exact same strategy. }


\begin{table}[H]
 \centering
 \caption{Imperative forms <\armenian{հրամայական}> for the verb `to like' in the Istanbul dialect}
 \label{tab:Istanbul:morpho:verb:paradigm:Imp}
 \begin{tabular}{|l|ll|ll |l|}
\hline & \multicolumn{2}{l|}{Istanbul} & \multicolumn{2}{l|}{cf. SWA} & \\
2SG & siɾ-e-$\emptyset$ & \armenian{սիրէ} & siɾ-e-$\emptyset$ & \armenian{սիրէ} & $\sqrt{}$-{\thgloss}-{\imp}.2{\sg}
\\
2PL& siɾ-e-t͡sʰ-ekʰ& \armenian{սիրէցէք} & siɾ-e-t͡sʰ-ekʰ& \armenian{սիրեցէք} & $\sqrt{}$-{\thgloss}-{\aor}-{\imp}.2{\pl}
\\\hline \end{tabular}
\end{table}

\translatorHD{For the prohibitive or negative imperative (Table \ref{tab:Istanbul:morpho:verb:paradigm:Proh}), SWA adds the prohibitive formative /mi/ before the verb. The prohibitive marker carries stress. The verb takes a suffix /-ɾ/ in the 2SG, and /-kʰ/ in the 2PL. In Istanbul, the prohibitive marker is either /mi/ or /mə/. } 


\begin{table}[H]
 \centering
 \caption{Negative imperative or prohibitive forms for the verb `to like' in the Istanbul dialect}
 \label{tab:Istanbul:morpho:verb:paradigm:Proh}
 \begin{tabular}{|l|ll|ll|l|}
\hline & \multicolumn{2}{l|}{Istanbul} & \multicolumn{2}{l|}{cf. SWA} & \\
2SG & m\'i siɾ-e-ɾ & \armenian{մի՛ սիրէր} & m\'i siɾ-e-ɾ & \armenian{մի՛ սիրեր} & {\proh} $\sqrt{}$-{\thgloss}-2{\sg} \\
 & m\'ə siɾ-e-ɾ & \armenian{մը՛ սիրէր} & & & \\
2PL & m\'i siɾ-e-kʰ& \armenian{մի՛ սիրէք} & m\'i siɾ-e-kʰ& \armenian{մի՛ սիրէք} & {\proh} $\sqrt{}$-{\thgloss}-2{\pl} \\
 & m\'ə siɾ-e-kʰ& \armenian{մը՛ սիրէք} & & & \\
\hline \end{tabular}
\end{table}



\subsubsection{Non-finite forms}

\translatorHD{Finally, Adjarian lists the following non-finite forms of this verb (participles or converbs) in Table \ref{tab:Istanbul:morpho:verb:paradigm:participle}. SWA and Istanbul have the same forms. Note that Adjarian uses the term `past participle' to mean multiple different types of non-finite forms: resultative participle with /-ɑd͡z/ in SWA, and evidential participle /-eɾ/ in SWA. The future participle is a future converb. } 

\begin{table}[H]
 \centering
 \caption{Participles or converbs <\armenian{դերբայներ}> for the verb `to like' in the Istanbul dialect}
 \label{tab:Istanbul:morpho:verb:paradigm:participle}
 \begin{tabular}{|ll|ll |ll| l|}
\hline & & \multicolumn{2}{l|}{Istanbul } & \multicolumn{2}{l|}{cf. SWA} & \\
 Infinitive & \armenian{անորոշ} & siɾ-e-l & \armenian{սիրէլ} & siɾ-e-l & \armenian{սիրել} & $\sqrt{}$-{\thgloss}-{\infgloss} \\
 Past & \armenian{անցեալ} & siɾ-ɑd͡z & \armenian{սիրաձ} & siɾ-ɑd͡z & \armenian{սիրած} & $\sqrt{}$-{\rptcp} \\
& & siɾ-eɾ & \armenian{սիրէր}& siɾ-eɾ & \armenian{սիրեր}& $\sqrt{}$-{\eptcp} \\
 Future & \armenian{ապառնի} & siɾ-i-l-u & \armenian{սիրէլու} & siɾ-e-l-u & \armenian{սիրելու} & $\sqrt{}$-{\thgloss}-{\infgloss}-{\futcvb} \\
 & & siɾ-i-l-ikʰ & \armenian{սիրէլիք} & siɾ-e-l-ikʰ & \armenian{սիրելիք} & $\sqrt{}$-{\thgloss}-{\infgloss}-{\futcvb} \\
\hline \end{tabular}
\end{table}
\begin{adjarianpage}\label{page:253}\end{adjarianpage}% should be 253

\subsubsection{Other complex tenses}
For the participles, the forms /siɾ-ɑd͡z/ <\armenian{սիրաձ}> and /siɾ-e-l-u/ <\armenian{սիրէլու}> are used to make many complex forms. 

\translatorHD{Throughout this section, Adjarian lists various complex tenses. He Adjarian doesn't translate or explain the meaning of any these complex tenses. For the morphology, I provide a simple rule to describe it. For the semantics, I can somewhat guess their meaning based on similarities to SWA. The (?) marks my uncertainty in my translation, even after consulting with other SWA speakers.\footnote{\translatorHD{I thank George Balabanian for help in this.}} It's generally difficult to find concrete semantic differences between some of these complex forms; see \citet{Boyacioglu-2010-HayPayVerbsArmenianOccidentalWestArmenian} for some pedagogically-oriented paradigms and their explanation. }




\translatorHD{In (\ref{sent:Istanbul:morpho:verb:complex:resultpart}), Adjarian lists complex tenses build over the resultative participle with present tense-marking. }


\begin{exe}
 \ex Istanbul \label{sent:Istanbul:morpho:verb:complex:resultpart} 
 \begin{xlist}
 \ex \gll siɾ-ɑd͡z e-m \\
$\sqrt{}$-{\rptcp} {aux}-1{\sg}\\
\trans `I have liked.'\\
 \armenian{սիրաձ էմ} \\
 Rule: Take the resultative participle, add the present auxiliary.
\ex \gll siɾ-ɑd͡z e-m eʁ-eɾ \\
$\sqrt{}$-{\rptcp} {aux}-1{\sg} be-{\eptcp} \\
\trans `I have liked, apparently.' \\
 \armenian{սիրաձ էմ էղէր} \\
 Rule: Take the resultative participle, add the present auxiliary, add the word /eʁ-eɾ/ `it apparently became' which provides an evidential meaning of uncertainty.
 \ex \gll siɾ-ɑd͡z ɡ-əll-ɑ-m \\
$\sqrt{}$-{\rptcp} {\ind}-be-{\thgloss}-1{\sg} \\
\trans `I (will) have liked.'
 \armenian{սիրաձ գըլլամ} \\
 Rule: Take the resultative participle, add the indicative present form of `to be'.
 \ex \gll siɾ-ɑd͡z ɡ-əll-ɑ-m eʁ-eɾ \\
$\sqrt{}$-{\rptcp} {\ind}-be-{\thgloss}-1{\sg} be-{\eptcp}\\
\trans `I (will) have liked, apparently.' \\
 \armenian{սիրաձ գըլլամ էղէր} \\
 Rule: Take the resultative participle, add the indicative present form of `to be', add the word /eʁ-eɾ/ `it apparently became' which provides an evidential meaning of uncertainty.
\ex \gll siɾ-ɑd͡z ɡ-əll-ɑ-m ɡoɾ \\
$\sqrt{}$-{\rptcp} {\ind}-be-{\thgloss}-1{\sg} {\prog} \\
\trans `I have been liking.' \\
 \armenian{սիրաձ գըլլամ գօր} \\
 Rule: Take the resultative participle, add the indicative present form of `to be', add the progressive marker /ɡoɾ/. 

 \ex \gll 
siɾ-ɑd͡z ɡ-əll-ɑ-m ɡoɾ eʁ-eɾ \\
$\sqrt{}$-{\rptcp} {\ind}-be-{\thgloss}-1{\sg} {\prog} be-{\eptcp} \\
\trans `I have been liking, apparently.' \\
 \armenian{սիրաձ գըլլամ գօր էղէր} \\
 Rule: Take the resultative participle, add the indicative present form of `to be', add the progressive marker /ɡoɾ/, add the word /eʁ-eɾ/ `it apparently became' which provides an evidential meaning of uncertainty.
\ex \gll siɾ-ɑd͡z əll-ɑ-m (ne) \\
$\sqrt{}$-{\rptcp} be-{\thgloss}-1{\sg} {\sbjv} \\
\trans `If I have liked.' \\
 \armenian{սիրաձ ըլլամ (նէ}) \\
 Rule: Take the resultative participle, add the subjuctive present form of `to be', add the subjunctive marker /ne/.

\ex \gll siɾ-ɑd͡z əll-ɑ-m (ne) eʁ-eɾ \\
$\sqrt{}$-{\rptcp} be-{\thgloss}-1{\sg} {\sbjv} be-{\eptcp} \\
\trans `If I have liked, apparently.' \\
 \armenian{սիրաձ ըլլամ (նէ) էղէր} \\
 Rule: Take the resultative participle, add the subjuctive present form of `to be', add the subjunctive marker /ne/, add the word /eʁ-eɾ/ `it apparently became' which provides an evidential meaning of uncertainty.
 \ex \gll siɾ-ɑd͡z eʁ-ɑ-$\emptyset$ \\
$\sqrt{}$-{\rptcp} be-{\pst}-1{\sg}\\
\trans `I was liked.' \\
 \armenian{սիրաձ էղա} \\
 Rule: Take the resultative participle, add the past perfective form of `to be'. 

\ex \gll siɾ-ɑd͡z bidi əll-ɑ-m \\
$\sqrt{}$-{\rptcp} {\fut} be-{\thgloss}-1{\sg} \\
\trans `I will have liked.' \\
 \armenian{սիրաձ բիդի ըլլամ} \\
 Rule: Take the resultative participle, add the future form of `to be' (made up of /pidi/ plus the verb form). 

\ex \gll siɾ-ɑd͡z bidi əll-ɑ-m eʁ-eɾ \\
$\sqrt{}$-{\rptcp} {\fut} be-{\thgloss}-1{\sg} be-{\eptcp}\\
\trans `I will have liked, apparently.' \\
 \armenian{սիրաձ բիդի ըլլամ էղէր} \\
 Rule: Take the resultative participle, add the future form of `to be' (made up of /pidi/ plus the verb form), add the word /eʁ-eɾ/ `it apparently became' which provides an evidential meaning of uncertainty. 

\ex \gll siɾ-ɑd͡z əll-ɑ-l-u e-m \\
$\sqrt{}$-{\rptcp} be-{\thgloss}-{\infgloss}-{\futcvb} {\aux}-1{\sg} \\
\trans `I am to like.' (there's a sense of an obligatory and necessitative future) \\
 \armenian{սիրաձ ըլլալու էմ}\\
 Rule: Take the resultative participle, add the future converb form of `to be' (with suffix /-u/), add the present auxiliary. 

\ex \gll siɾ-ɑd͡z əll-ɑ-l-u e-m eʁ-eɾ \\
$\sqrt{}$-{\rptcp} be-{\thgloss}-{\infgloss}-{\futcvb} {\aux}-1{\sg} be-{\eptcp} \\
\trans `I am be like, apparently.' \\
 \armenian{սիրաձ ըլլալու էմ էղէր} \\
 Rule: Take the resultative participle, add the future converb form of `to be' (with suffix /-u/), add the present auxiliary, add the word /eʁ-eɾ/ `it apparently became' which provides an evidential meaning of uncertainty. 
\ex \gll siɾ-ɑd͡z əll-ɑ-l-u əll-ɑ-m (ne) \\
$\sqrt{}$-{\rptcp} be-{\thgloss}-{\infgloss}-{\futcvb} be-{\thgloss}-1{\sg} {\sbjv} \\
\trans `If I will be liking.' (or perhaps `If I am to be liking.') \\
 \armenian{սիրաձ ըլլալու ըլլամ (նէ}) \\
 Rule: Take the resultative participle, add the future converb form of `to be' (with suffix /-u/), add the present subjunctive form of `to be', add the subjunctive marker /ne/. 

\ex \gll siɾ-ɑd͡z əll-ɑ-l-u əll-ɑ-m (ne) eʁ-eɾ \\
$\sqrt{}$-{\rptcp} be-{\thgloss}-{\infgloss}-{\futcvb} be-{\thgloss}-1{\sg} {\sbjv} be-{\eptcp} \\
\trans `If I will be liking, apparently.' (or perhaps `If I am to be liking, apparently.')\\
 \armenian{սիրաձ ըլլալու ըլլամ (նէ) էղէր}\\
 Rule: Take the resultative participle, add the future converb form of `to be' (with suffix /-u/), add the present subjunctive form of `to be', add the subjunctive marker /ne/, add the word /eʁ-eɾ/ `it apparently became' which provides an evidential meaning of uncertainty. 
 
 \end{xlist}
\end{exe}

\translatorHD{In (\ref{sent:Istanbul:morpho:verb:complex:resultpart:pst}), Adjarian lists complex tenses build over the resultative participle with past tense-marking. }


\begin{exe}
 \ex Istanbul \label{sent:Istanbul:morpho:verb:complex:resultpart:pst}
 \begin{xlist}
 \ex \gll siɾ-ɑd͡z e-i-$\emptyset$ \\
$\sqrt{}$-{\rptcp} {aux}-{\pst}-1{\sg} \\
\trans `I had liked.' \\
\armenian{սիրաձ էի} \\
 Rule: Take the resultative participle, add the past auxiliary.
\ex \gll siɾ-ɑd͡z e-i-$\emptyset$ eʁ-eɾ \\
$\sqrt{}$-{\rptcp} {aux}-{\pst}-1{\sg} be-{\eptcp} \\
\trans `I had liked, apparently.' \\
\armenian{սիրաձ էի էղէր} \\
 Rule: Take the resultative participle, add the past auxiliary, add the word /eʁ-eɾ/ `it apparently became' which provides an evidential meaning of uncertainty.
\ex \gll siɾ-ɑd͡z ɡ-əll-ɑj-i-$\emptyset$ \\
$\sqrt{}$-{\rptcp} {\ind}-be-{\thgloss}-{\pst}-1{\sg} \\
\trans `I would have liked.' \\
\armenian{սիրաձ գըլլայի} \\
 Rule: Take the resultative participle, add the indicative past imperfective form of `to be'.
\ex \gll siɾ-ɑd͡z ɡ-əll-ɑj-i-$\emptyset$ eʁ-eɾ \\
$\sqrt{}$-{\rptcp} {\ind}-be-{\thgloss}-{\pst}-1{\sg} be-{\eptcp} \\
\trans `I would have liked, apparently.' \\
\armenian{սիրաձ գըլլայի էղէր} \\
 Rule: Take the resultative participle, add the indicative past imperfective form of `to be', add the word /eʁ-eɾ/ `it apparently became' which provides an evidential meaning of uncertainty.
\ex \gll siɾ-ɑd͡z ɡ-əll-ɑj-i-$\emptyset$ ɡoɾ \\
$\sqrt{}$-{\rptcp} {\ind}-be-{\thgloss}-{\pst}-1{\sg} {\prog} \\
\trans `I had been liking.' \\
\armenian{սիրաձ գըլլայի գօր} \\
 Rule: Take the resultative participle, add the indicative past imperfective form of `to be', add the progressive marker /ɡoɾ/. 
\ex \gll siɾ-ɑd͡z ɡ-əll-ɑj-i-$\emptyset$ ɡoɾ eʁ-eɾ \\
$\sqrt{}$-{\rptcp} {\ind}-be-{\thgloss}-{\pst}-1{\sg} {\prog} be-{\eptcp} \\
\trans `I had been liking, apparently.' \\
\armenian{սիրաձ գըլլայի գօր էղէր} \\ 
 Rule: Take the resultative participle, add the indicative past imperfective form of `to be', add the progressive marker /ɡoɾ/, add the word /eʁ-eɾ/ `it apparently became' which provides an evidential meaning of uncertainty.
\ex \gll siɾ-ɑd͡z əll-ɑj-i-$\emptyset$ (ne) \\
$\sqrt{}$-{\rptcp} be-{\thgloss}-{\pst}-1{\sg} {\sbjv} \\
\trans `If I had liked.' \\
\armenian{սիրաձ ըլլայի (նէ}) \\
 Rule: Take the resultative participle, add the subjunctive past imperfective form of `to be', add the subjunctive marker /ne/.
\ex \gll siɾ-ɑd͡z əll-ɑj-i-$\emptyset$ (ne) eʁ-eɾ \\
$\sqrt{}$-{\rptcp} be-{\thgloss}-{\pst}-1{\sg} {\sbjv} be-{\eptcp} \\
\trans `If I had liked, apparently.' \\
\armenian{սիրաձ ըլլայի (նէ) էղէր} \\
 Rule: Take the resultative participle, add the subjuctive past imperfective form of `to be', add the subjunctive marker /ne/, add the word /eʁ-eɾ/ `it apparently became' which provides an evidential meaning of uncertainty.
\ex \gll siɾ-ɑd͡z bidi əll-ɑj-i-$\emptyset$ \\
$\sqrt{}$-{\rptcp} {\fut} be-{\thgloss}-{\pst}-1{\sg} \\
\trans `I would have liked.' \\
\armenian{սիրաձ բիդի ըլլայի} \\
 Rule: Take the resultative participle, add the future perfect form of `to be' (made up of /pidi/ plus the verb form). 
\ex \gll siɾ-ɑd͡z bidi əll-ɑj-i-$\emptyset$ eʁ-eɾ \\
$\sqrt{}$-{\rptcp} {\fut} be-{\thgloss}-{\pst}-1{\sg} be-{\eptcp} \\
\trans `I would have liked, apparently.' \\
\armenian{սիրաձ բիդի ըլլայի էղէր} \\
 Rule: Take the resultative participle, add the future perfect form of `to be' (made up of /pidi/ plus the verb form), add the word /eʁ-eɾ/ `it apparently became' which provides an evidential meaning of uncertainty. 
\ex \gll siɾ-ɑd͡z əll-ɑ-l-u e-i-$\emptyset$ \\
$\sqrt{}$-{\rptcp} be-{\thgloss}-{\infgloss}-{\futcvb} {\aux}-1{\sg} \\
\trans `I was (going) to like.' \\
\armenian{սիրաձ ըլլալու էի} \\
 Rule: Take the resultative participle, add the future converb form of `to be' (with suffix /-u/), add the past auxiliary. 
\ex \gll siɾ-ɑd͡z əll-ɑ-l-u e-i-$\emptyset$ eʁ-eɾ \\
$\sqrt{}$-{\rptcp} be-{\thgloss}-{\infgloss}-{\futcvb} {\aux}-1{\sg} be-{\eptcp} \\
\trans `I was (going) to like, apparently.' \\
\armenian{սիրաձ ըլլալու էի էղէր} \\
 Rule: Take the resultative participle, add the future converb form of `to be' (with suffix /-u/), add the past auxiliary, add the word /eʁ-eɾ/ `it apparently became' which provides an evidential meaning of uncertainty. 
\ex \gll siɾ-ɑd͡z əll-ɑ-l-u əll-ɑj-i-$\emptyset$ (ne) \\
$\sqrt{}$-{\rptcp} be-{\thgloss}-{\infgloss}-{\futcvb} be-{\thgloss}-{\pst}-1{\sg} {\sbjv} \\
\trans `If I was going to like.' \\
\armenian{սիրաձ ըլլալու ըլլայի (նէ}) \\
 Rule: Take the resultative participle, add the future converb form of `to be' (with suffix /-u/), add the subjunctive past imperfective form of `to be', add the subjunctive marker /ne/. 
\ex \gll siɾ-ɑd͡z əll-ɑ-l-u əll-ɑj-i-$\emptyset$ (ne) eʁ-eɾ \\
$\sqrt{}$-{\rptcp} be-{\thgloss}-{\infgloss}-{\futcvb} be-{\thgloss}-{\pst}-1{\sg} {\sbjv} be-{\eptcp} \\
\trans `If I was going to like, apparently.' \\
\armenian{սիրաձ ըլլալու ըլլայի (նէ) էղէր} \\ 
 Rule: Take the resultative participle, add the future converb form of `to be' (with suffix /-u/), add the subjunctive past imperfective form of `to be', add the subjunctive marker /ne/, add the word /eʁ-eɾ/ `it apparently became' which provides an evidential meaning of uncertainty. 

 \end{xlist}
\end{exe}


\translatorHD{In (\ref{sent:Istanbul:morpho:verb:complex:futcvb:pres}), Adjarian lists complex tenses build over the future converb with present tense-marking. }


\begin{exe}
 \ex Istanbul \label{sent:Istanbul:morpho:verb:complex:futcvb:pres}
 \begin{xlist}
 \ex \gll siɾ-e-l-u e-m \\
$\sqrt{}$-{\thgloss}-{\infgloss}-{\futcvb} {aux}-1{\sg} \\
\trans `I will like.' \\
\armenian{սիրէլու էմ} \\
 Rule: Take the future converb, add the present auxiliary.
\ex \gll siɾ-e-l-u e-m eʁ-eɾ \\
$\sqrt{}$-{\thgloss}-{\infgloss}-{\futcvb} {aux}-1{\sg} be-{\eptcp} \\
\trans `I will like, apparently.' \\
\armenian{սիրէլու էմ էղէր} \\
 Rule: Take the future converb, add the present auxiliary, add the word /eʁ-eɾ/ `it apparently became' which provides an evidential meaning of uncertainty.
\ex \gll siɾ-e-l-u ɡ-əll-ɑ-m \\
$\sqrt{}$-{\thgloss}-{\infgloss}-{\futcvb} {\ind}-be-{\thgloss}-1{\sg} \\
\trans `I would like.' \\
\armenian{սիրէլու գըլլամ} \\
 Rule: Take the future converb, add the indicative present form of `to be'.
\ex \gll siɾ-e-l-u ɡ-əll-ɑ-m eʁ-eɾ \\
$\sqrt{}$-{\thgloss}-{\infgloss}-{\futcvb} {\ind}-be-{\thgloss}-1{\sg} be-{\eptcp} \\
\trans `I would like, apparently.' \\
\armenian{սիրէլու գըլլամ էղէր} \\
 Rule: Take the future converb, add the indicative present form of `to be', add the word /eʁ-eɾ/ `it apparently became' which provides an evidential meaning of uncertainty.
\ex \gll siɾ-e-l-u ɡ-əll-ɑ-m ɡoɾ \\
$\sqrt{}$-{\thgloss}-{\infgloss}-{\futcvb} {\ind}-be-{\thgloss}-1{\sg} {\prog} \\
\trans `I am going to like.' (with a meaning that I will (in the future) intend to like) (?) \\
\armenian{սիրէլու գըլլամ գօր} \\
 Rule: Take the future converb, add the indicative present form of `to be', add the progressive marker /ɡoɾ/. 
\ex \gll siɾ-e-l-u ɡ-əll-ɑ-m ɡoɾ eʁ-eɾ \\
$\sqrt{}$-{\thgloss}-{\infgloss}-{\futcvb} {\ind}-be-{\thgloss}-1{\sg} {\prog} be-{\eptcp} \\
\trans `I am going to like, apparently.' (?) \\
\armenian{սիրէլու գըլլամ գօր էղէր} \\
 Rule: Take the future converb, add the indicative present form of `to be', add the progressive marker /ɡoɾ/, add the word /eʁ-eɾ/ `it apparently became' which provides an evidential meaning of uncertainty.
\ex \gll siɾ-e-l-u əll-ɑ-m (ne) \\
$\sqrt{}$-{\thgloss}-{\infgloss}-{\futcvb} be-{\thgloss}-1{\sg} {\sbjv} \\
\trans `If I were to like.' (?) \\
\armenian{սիրէլու ըլլամ (նէ}) \\
 Rule: Take the future converb, add the subjuctive present form of `to be', add the subjunctive marker /ne/.
\ex \gll siɾ-e-l-u əll-ɑ-m (ne) eʁ-eɾ \\
$\sqrt{}$-{\thgloss}-{\infgloss}-{\futcvb} be-{\thgloss}-1{\sg} {\sbjv} be-{\eptcp} \\
\trans `If I were to like, apparently.' (?) \\
\armenian{սիրէլու ըլլամ (նէ) էղէր} \\
 Rule: Take the future converb, add the subjuctive present form of `to be', add the subjunctive marker /ne/, add the word /eʁ-eɾ/ `it apparently became' which provides an evidential meaning of uncertainty.
\ex \gll siɾ-e-l-u eʁ-ɑ-$\emptyset$ \\
$\sqrt{}$-{\thgloss}-{\infgloss}-{\futcvb} be-{\pst}-1{\sg} \\
\trans `I am to have been liked.' (?) (some sort of passive meaning) \\
\armenian{սիրէլու էղա} \\
 Rule: Take the future converb, add the past perfective form of `to be'. 
\ex \gll siɾ-e-l-u bidi əll-ɑ-m \\
$\sqrt{}$-{\thgloss}-{\infgloss}-{\futcvb} {\futcvb} be-{\thgloss}-1{\sg} \\
\trans `I will like.' \\
\armenian{սիրէլու բիդի ըլլամ} \\
 Rule: Take the future converb, add the future form of `to be' (made up of /pidi/ plus the verb form). 
\ex \gll siɾ-e-l-u bidi əll-ɑ-m eʁ-eɾ \\
$\sqrt{}$-{\thgloss}-{\infgloss}-{\futcvb} {\futcvb} be-{\thgloss}-1{\sg} be-{\eptcp} \\
\trans `I will like, apparently.' \\
\armenian{սիրէլու բիդի ըլլամ էղէր} \\
 Rule: Take the future converb, add the future form of `to be' (made up of /pidi/ plus the verb form), add the word /eʁ-eɾ/ `it apparently became' which provides an evidential meaning of uncertainty. 
\end{xlist}
 \end{exe}
 
 \begin{adjarianpage}\label{page:254}\end{adjarianpage}% should be 254



\translatorHD{In (\ref{sent:Istanbul:morpho:verb:complex:futcvb:pst}), Adjarian lists complex tenses build over the future converb with past tense-marking. }


\begin{exe}
 \ex Istanbul \label{sent:Istanbul:morpho:verb:complex:futcvb:pst}
 \begin{xlist}
 \ex \gll siɾ-e-l-u e-i-$\emptyset$ \\
$\sqrt{}$-{\thgloss}-{\infgloss}-{\futcvb} {aux}-{\pst}-1{\sg} \\
\trans `I was going to like.' \\
\armenian{սիրէլու էի} \\
 Rule: Take the future converb, add the past auxiliary.
\ex \gll siɾ-e-l-u e-i-$\emptyset$ eʁ-eɾ \\
$\sqrt{}$-{\thgloss}-{\infgloss}-{\futcvb} {aux}-{\pst}-1{\sg} be-{\eptcp} \\
\trans `I was going to like, apparently.' \\
\armenian{սիրէլու էի էղէր} \\
 Rule: Take the future converb, add the past auxiliary, add the word /eʁ-eɾ/ `it apparently became' which provides an evidential meaning of uncertainty.
\ex \gll siɾ-e-l-u ɡ-əll-ɑj-i-$\emptyset$ \\
$\sqrt{}$-{\thgloss}-{\infgloss}-{\futcvb} {\ind}-be-{\thgloss}-{\pst}-1{\sg} \\
\trans `I would had liked.' \\
\armenian{սիրէլու գըլլայի} \\
 Rule: Take the future converb, add the indicative past imperfective form of `to be'.
\ex \gll siɾ-e-l-u ɡ-əll-ɑj-i-$\emptyset$ eʁ-eɾ \\
$\sqrt{}$-{\thgloss}-{\infgloss}-{\futcvb} {\ind}-be-{\thgloss}-{\pst}-1{\sg} be-{\eptcp} \\
\trans `I would had liked, apparently.' \\
\armenian{սիրէլու գըլլայի էղէր} \\
 Rule: Take the future converb, add the indicative past imperfective form of `to be', add the word /eʁ-eɾ/ `it apparently became' which provides an evidential meaning of uncertainty.
\ex \gll siɾ-e-l-u ɡ-əll-ɑj-i-$\emptyset$ ɡoɾ \\
$\sqrt{}$-{\thgloss}-{\infgloss}-{\futcvb} {\ind}-be-{\thgloss}-{\pst}-1{\sg} {\prog} \\
\trans `I was going to be liking.' (with a meaning that I was (in the future) intending to like) (?) \\
\armenian{սիրէլու գըլլայի գօր} \\
 Rule: Take the future converb, add the indicative past imperfective form of `to be', add the progressive marker /ɡoɾ/. 
\ex \gll siɾ-e-l-u ɡ-əll-ɑj-i-$\emptyset$ ɡoɾ eʁ-eɾ \\
$\sqrt{}$-{\thgloss}-{\infgloss}-{\futcvb} {\ind}-be-{\thgloss}-{\pst}-1{\sg} {\prog} be-{\eptcp} \\
\trans `I was going to be liking, apparently.' (?) \\
\armenian{սիրէլու գըլլայի գօր էղէր} \\
 Rule: Take the future converb, add the indicative past imperfective form of `to be', add the progressive marker /ɡoɾ/, add the word /eʁ-eɾ/ `it apparently became' which provides an evidential meaning of uncertainty.
\ex \gll siɾ-e-l-u əll-ɑj-i-$\emptyset$ (ne) \\
$\sqrt{}$-{\thgloss}-{\infgloss}-{\futcvb} be-{\thgloss}-{\pst}-1{\sg} {\sbjv} \\
\trans `If I were going to like.' \\
\armenian{սիրէլու ըլլայի (նէ}) \\
 Rule: Take the future converb, add the subjuctive past imperfective form of `to be', add the subjunctive marker /ne/.
\ex \gll siɾ-e-l-u əll-ɑj-i-$\emptyset$ ne eʁ-eɾ \\
$\sqrt{}$-{\thgloss}-{\infgloss}-{\futcvb} be-{\thgloss}-{\pst}-1{\sg} {\sbjv} be-{\eptcp} \\
\trans `If I were going to like, apparently.' \\
\armenian{սիրէլու ըլլայի նէ էղէր} \\
 Rule: Take the future converb, add the subjuctive past imperfective form of `to be', add the subjunctive marker /ne/, add the word /eʁ-eɾ/ `it apparently became' which provides an evidential meaning of uncertainty.
\ex \gll siɾ-e-l-u bidi əll-ɑj-i-$\emptyset$ \\
$\sqrt{}$-{\thgloss}-{\infgloss}-{\futcvb} {\futcvb} be-{\thgloss}-{\pst}-1{\sg} \\
\trans `I was going to like.' \\
\armenian{սիրէլու բիդի ըլլայի} \\
 Rule: Take the future converb, add the future perfect form of `to be' (made up of /pidi/ plus the verb form). 
\ex \gll siɾ-e-l-u bidi əll-ɑj-i-$\emptyset$ eʁ-eɾ \\
$\sqrt{}$-{\thgloss}-{\infgloss}-{\futcvb} {\futcvb} be-{\thgloss}-{\pst}-1{\sg} be-{\eptcp} \\
\trans `I was going to like, apparently.' \\
\armenian{սիրէլու բիդի ըլլայի էղէր} \\
 Rule: Take the future converb, add the future perfect converb form of `to be' (with suffix /-u/), add the present auxiliary. 
\end{xlist}
 \end{exe}

\section{Literature}

As we said above, the Istanbul dialect is still not studied. The innumerable manuscripts that written in this dialect (newspapers, novels, fables, proverbs, folk songs, especially comedic writings and comedies) generally don't have the needed scientific accuracy. The latter condition can be satisfied by my collection of Istanbul-Armenian oral literature, from which only a part was published in the \citeauthor{AzgagrakanHandes}, volume 9 (\armenian{Թ}.), page 160-196. As a text sample, I place here the following real case, which is a letter of mine written with the scientific orthography.

\section{Text samples}

 
{\sampleoverview}

\armenian{Ան իրինգունը Բէօյիւք-Դէրէ գացէր էի, ձօվին քէնարը վէր վար փիյացա (պտոյտ) գընէի գոր։ Բանէ մը խաբար չունինք։ Մէյ մըն ա դէսնաս բաֆօրը էգավ, մէչէն խընջախընջ մարքթիքը դուրս թափէցան. ամմէնուն ըռէնգը նէդէր, բէնզըբէթը}... 

\begin{adjarianpage}\label{page:255}\end{adjarianpage}% should be 255

... \armenian{դէղնէր, իրարու հէդ խօսէլ-խօրաթէլ բիլէ չիքա (չկայ)։ Հրանդը դէմս էլավ. մօդէցա, ձառքս օմուզի դրի}, – \armenian{Ի՞նչ գա, ա՛խբար, ըսի։}

– \armenian{Սո՛ւս էղիր, ըսավ գամաց ձանօվ մը, բան չիքա, գամաց խօրաթէ, բադին վրա ջանջ գա (ծածկաբանութիւն որ կը նշանակէ «լսող օտար մարդ կայ»)։}

\armenian{Էրգուքնիս քօվ քօվի՝ բէրաննիս բաբընձաձ՝ իսգէլէէն անցանք փիյացան։ Օրթալըխը մարթ մարթասանք չմնաց. ամմէն մարթ դուն վազէց։ Նայէցա քի մէզի լսօղ չիքա, նօրէն դարցա Հրանդին։}

– \armenian{Է՛, Հրանդ, ըսէ նայի՛նք, լէզույիթ դագը բան մը գա ամա, գը բահէս գօր։}

– \armenian{Ցույց էղավ, ցո՛ւյց}… \armenian{ըսավ գամաց ձանօվ մը։}

– \armenian{Ձօ ի՛նչ գըսէս, ցո՜ւյց մի}…

– \armenian{Հա, Հնչագյաննէրը Բաբա-Ալին գօխէր էն, քանի մը ասգէր մէռցուցէր էն. Դաջիգնէրը դուրս թափէցան, զարգին, ջարթէցին, հազարէն էվէլ հայ մէռսցուցէր էն. անջախ խանութնիս գօցէցինք, քէօթրիւ փախանք, ինքըզինքնիս բաֆօր (շոգենաւ) նէդէցինք։}

– \armenian{Ձօ ի՛նչ գըսէս բէ՜։ Ադ ի՛նչ գէշ խաբար դուվիր ինձի։}

– \armenian{Ասիգա զաթը շադօնց ըլլար բիդի. ինգիլիզին զըռխլընէրը էգաձ շարվաձ էն Չանախ-Խալէին քէնարը. չըթի մը գը բէքլէէին գօր. աս ցույցը մա՛խսուսդան ըրին քի, Դաջիգնէրը էլլան հայէրը ջարթէն, ինգիլիզնէրն ա «Վա՛յ, դուն հայէրը գը ջարթէ՞ս գօր մի» ըսէլօվ՝ էլլան, յալլահ, Չանախ-Խալէն գօխէն, դի՛ւզ Սդամբօլին վրան}…\armenian{։}

\armenian{Սիրդս թըփըր թըփըր նէդէլ բաշլայէց. խնդա՞մ մի՝ լա՞մ մի. ան թաքքէյին գուզէի քի Հրանդին փաթդըվիմ՝ էրէսը, բէրանը բաքնէմ։}

– \armenian{Աս քիշէր ջամփա բիդի էլլան, ըսավ էյէր քի գէս քիշէրին թօփի-թիւֆէնգի ձանէր առնէս նէ, հիչ չվախնաս, ինգիլիզնէրն էն. մէզի ազադէլու բիդի գան։}

\armenian{Ասանգ խաբար մը գը բահվի՞. դուն վազէցի. նայէցա քի սուֆրան դրէր էն, հարըս, մարըս, Հէրսիլյան, Արմէնույին, Հայգը նստէր էր, ընձի գը բէքլէէն գօր։ Բանէ խաբար չունին։}

– \armenian{Ա՛սօր վարը գռիվ էղէր է. հայէրը Բաբա-Ալին առէր էն}... 



\begin{adjarianpage}\label{page:256}\end{adjarianpage}% should be 256


\armenian{հազարէն էվէլ հայ ջարթըվէր է. ամա վախնալու բան չիքա. աս քիշէր ինգիլիզին զըռխլընէրը Չանախ-Խալէն բիդի գօխէն, Բօլիսին վրա բիդի գան, քաղաքը բիդի առնէն, մէզի թաքավօրութին բիդի դան։}

\armenian{Ամմէնը դէղէրնուն վէր ցաթգէցին. ուրախութաննուն ի՛նչ ընէլնին չիյդէն. Հայգը բաշլայէց ձառքվընէրը իրարու զարնէլ. հարըս «Ա՚ֆէրիմ հայէր, ըսավ. տէսա՞ր մի, գնիգ, էս քէզի չէի՞ ըսէր գօր քի աս դարի մո՛ւթլախա ազադութին մը բիդի ըլլա»։ – «Է՜, ըսավ մարս ա, էս ա չէի՞ ըսէր գօր քի սա աշգըս քանի մը օր է գը խաղա գօր. բան մը բիդի ըլլա ամմա, ի՛նչ ըլլալիքը չիյդէի»։}

– \armenian{Թող ըլլա՛, թօղ ըլլա՛. աս ձէրութանս՝ բաց աշգօվ մէյ մը սա թաքավօրնիս դէսնամ դէ, բաշխա բան չէմ ուզէր Ասձուձմէ}… \armenian{հիչ բան մը չգըրնամ ընէր նէ՝՝ հիչ չէ նէ բիդի էրթամ ասգէրնէրուն համար փրինձ ըսդըգէլու։}

\armenian{Հացրէնիս գէրանք, էս իմ օդաս քաշվէցա, հարս, մարս, քուրէրս, ախբարս ալ իրէնց դէղէրը քաշվէցան բառգէցան։ Ամա վօրի՞ն քունը գը դանի։ Աշգըս բաց գը բէքլէէմ գօր քի հա հիմա գուքան ինգիլիզնէրը, հա հիմա։}

\armenian{Գէս քիշէրը անցէր էր. մէյ մըն ալ բո՜ւմբ}… \armenian{ձան մը էլավ. մգիգ ըրի. ձանը կըդրէցավ. աջաբա անդաջի՞ս էգավ գըսէմ. քիչ մը բէքլէէցի, դէսա քի չէ, բո՜ւմբ}… \armenian{ձան մը դահա էլավ. բո՜ւմ}… \armenian{գէնէ էդէվէն, գէնէ էդէվէն}…\armenian{։ Ալթըխ շիւփէ չմնաց։}

\armenian{Հարս անթիի օդայէն ձան դուվավ}.

– \armenian{Ձօ՛, Հրա՛չյա, արթո՞ւն էս}…\armenian{։}

– \armenian{Արթուն էմ, հա՛յրիգ}…\armenian{։}

– \armenian{Գը լսէ՞ս գօր, ի՞նչ է աս}…,

– \armenian{Գը լսէմ գօր, անօ՜նք էն}…\armenian{։}

\armenian{Ձանէրը էդէվէ էդէվ շադձան. դէօշէգնէրնուս ցաթգէցինք, փէնջիրէին առչէվը վազէցինք, ձօվին հէռունէրը գը նայինք գօր}… \armenian{խօրունգէն էգաձ ձան մըն էր, թամամ գրագին թօփին ձանը գըլմանէր}… \armenian{ձանէն յէթն ալ բարագ լուս մը գէլլար, ձօվին վրայէն խըզըլջըմի բէս գը զարնէր գասնէր գօր։}

– \armenian{Ինգիլիզին թօփէրն էն, ըսինք. Չանախ-Խալէն առին}…\armenian{։}

– \armenian{Է՜ Չանախ-Խալէէն մինչէվ հօս թօփի ձանա գուքա՞. քանի՞ սահաթվան ջամփա-յ-է։}





\begin{adjarianpage}\label{page:257}\end{adjarianpage}% should be 257


– \armenian{Օր մը գը քաշէ ամա, ասօր ինգիլիզ գըսէն}… \armenian{ինգիլիզին ինչ ըլլալը գիդէ՞ս}…

– \armenian{Ադօր խօ՞սք գա. բէլքիյ-ա քի (թերեւս եւ) Չանախ-Խալէն առէր էն դէ. Սիլիվրիին յա Չաթալջային բացէրն էն։}

– \armenian{Ադանգ ըլլալու է։}

\armenian{Խնդումնիս փօրէրնիս բահաձ, հէմ վախօվ, հէմ ուրախութինօվ սիրդէրնիս լէցունգ, ինչ ընէնք չիյգէնք. աշվընէրնիս ձօվին դնգէր գը բէքլէէնք գօր։ Ձանէրը էրթալօվ շադձան}… \armenian{Մէյ մըն ալ մէգ օրօդում մը, մէգ գիւրիւլթիւ-փաթըրդը մը, խըզըլջըմ, սաղանախի բէս արզէվ մը, արզէ՜վ մը քի դունէրնիս հէմէն դէղէն քշէ՝ դանի ձօվը լէցունէ բիդի։ Արզէվին խըզէն օլուխնէրը թէվէրու բէս նէդէլ բաշլայէցին, թավանին հին ու մին դէղէրէն ջուրը շառըլ-շուռուլ գը վազէ, ասդին գօցէնք՝ անթին գը վազէ, անթին գօցէնք՝ ասդին գը վազէ։}

– \armenian{Առի՞ր մի հիմա ինգիլիզը, ըսավ հարըս։}

\armenian{Հէռույէն էգաձ ձանէրը օրօդումի ձան է էղէր, լուսն ա խըզըլջըմին իլէն չըմչըրախին փառըլթըն}… \armenian{մէնք ա ինգիլիզը էգավ ըսէլէն՝ գը բէքլէէնք գօր։}

\armenian{Ինչ օր քէզի քրէցի նէ՝ մինագ մէզի չէղավ. ամմէնուն դունն ա ասանգ էղավ, հէմ հայ, հէմ դաջիգ։ Վարը (Վոսփորի հայերը այսպէս կը կոչեն բուն Պօլիսը) դահա խըյախ էղէր է. փիւթիւն դաջիգնէրը հայէրուն դունէրը լէցվէր էն, «Հիմա ինգիլիզը բիդի դա՝ փիւթիւն դաջիգնէրը ջարթէ բիդի, Ասձուձու սիրուն, աղբար դարվան դրացնութան սիրուն՝ մէզի ձագ մը խօթէցէք բահէցէք» ըսէլէն։}


\chapter{Rodosto}

\section{Background and literature}

\begin{adjarianpage}\label{page:258}\end{adjarianpage}% should be 258

In European Turkey, there is only one Armenian settlement that still preserves the Armenian language: the settlement of Rodosto and Malkara. The two are neighbors and are heavily-Armenian cities. Other places, such as Silivri, Çatalca, Çorlu, Gyumyurdjina, Edirne, are all entirely Turkish-speaking.


The Armenian dialect of this region is still not studied. There is not even a line written from the Rodosto language. There is only a folk prayer from Malkara, published in \citeauthor{Byurakn} 1898, page 756. 

In the summer of 1910, with the goal of studying Armenian dialects, I passed through Rodosto, where I prepared a study of the dialect by working with Armenoloist and philologist Tigran Efendi Paghtikian (\armenian{Մեծ. Տիգրան էֆ. Պաղտիկեան}; \translatorHD{SEA: /tiɡɾɑn pɑχtikjɑn/, SWA: /dikʰɾɑn bɑʁdiɡjɑn/}). I extract the following succinct sketch from this unpublished work of mine. 

\section{Phonology}
\subsection{Sound changes}
\subsubsection{Consonant changes}
\paragraph{Laryngeal changes}
The dialect of Rodosto does not differ much from the dialect of Istanbul. The sound system is already the same. The consonants here have only two degrees: voiced and voiceless aspirated. 

But in the dialects of Rodosto and Istanbul, there are many large differences. The Old Armenian voiceless unaspirated sounds have become voiced here, and the voiceless aspirates have stayed voiceless aspirates as in other dialects. But in contrast, the voiced consonants have become voiceless aspirates. 


This sound change, which is characteristic of also the Tigranakert and Malatya dialects, is very interesting from the point of view of the pronunciation of the literary Western language. As we know, the Classical voiced consonants are pronounced as voiceless aspirates in the Western literary language (/pʰ, kʰ, tʰ/ from <\armenian{բ, գ, դ}>, and so on); this is in contrast to even the vernacular language of Istanbul, where these same consonants... 



\begin{adjarianpage}\label{page:259}\end{adjarianpage}% should be 259

... become voiced sounds. This is such that the duality of pronunciation is the most common phenomenon for Western Armenians. When an Istanbul Armenian speaks at home, he pronounces as in (\ref{sent:Rodosto:phono:change:cons:voice:Istanbul}), while if he is talking to a literary person, he will say with the literary pronunciation as in (\ref{sent:Rodosto:phono:change:cons:voice:swa}).

\begin{exe}
 \ex \begin{xlist}
 \ex Istanbul \label{sent:Rodosto:phono:change:cons:voice:Istanbul}
 \begin{xlist}
 \ex \gll dur-ə bɑt͡sʰ \\
 door-{\defgloss} open.{\imp}.2{\sg} \\
 \trans `Open the door.'\\
 \armenian{դուռը բաց}
 \ex \gll dur-ə ɡot͡sʰ-e-$\emptyset$ \\
 door-{\defgloss} close-{\thgloss}-{\imp}.2{\sg} \\
 \trans `Close the door.'\\
 \armenian{դուռը գօցէ}
 \end{xlist}
 \ex cf. SWA as spoken by an Istanbul speaker \label{sent:Rodosto:phono:change:cons:voice:swa}
 \begin{xlist}
 \ex \gll tʰur-ə pʰɑt͡sʰ \\
 door-{\defgloss} open.{\imp}.2{\sg} \\
 \trans `Open the door.'\\
 \armenian{թուռը փաց} (standard: \armenian{դուռը բաց})
 \ex \gll tʰur-ə kʰot͡sʰ-e-$\emptyset$ \\
 door-{\defgloss} close-{\thgloss}-{\imp}.2{\sg} \\
 \trans `Close the door.'\\
 \armenian{թուռը քօցէ} (standard: \armenian{դուռը գոցէ})
 \end{xlist}
 \ex cf. SEA as a conservative descendant of Classical Armenian\label{sent:Rodosto:phono:change:cons:voice:sea}
 \end{xlist}
 
\end{exe}

\translatorHD{To clarify, the Istanbul form is more conservative with respect to the voicing values in Classical Armenian. The stops and affricates in the above Istanbul forms would be pronounced essentially the same as in Classical Armenian: /durən/ <\armenian{դուռն}> `door', /bɑt͡sʰ/ `open' <\armenian{բաց}>. }

\translatorHD{Note that this data suggests that during Adjarian's time, SWA had a phonemic contrast between trills /r/ and taps /ɾ/. Such a contrast is now lost for most Western Armenian speakers. }
 
The duality of this pronunciation has always been surprising for researchers. Every person has had the idea that literate Istanbul Armenian have created the aforementioned pronunciations using an artistic style. But the way of pronunciation for Rodosto, combined with Tigranakert and \armenian{ւ} Malatya, comes to finally remove this useless concept, and it proves that the literary pronunciation of Istanbul, is the work of Armenian migrants who came from these areas. The first literate people of Istanbul of course belong to this same migrant group, and they have also introduced their way of pronunciation, just as how now the Istanbul Armenians spread it across the provinces.

It remains to be asked how this pronunciation of voiceless aspirates originated in the dialects of Rodosto, Malatya, and Tigranakert. 


In my opinion, the path for this sound change is the voiced aspirated consonants. Rodosto, Malatya,\footnote{\translatorHD{The original page had <\armenian{Մալկարացիք}>. But on page 306, Adjarian provides an erratum that this should be <\armenian{Մալաթիացիք}>; I fixed it. }} and Tigranakert previously had voiced aspirated consonants, instead of the Old Armenian voiced consonants. The voiced aspirates, because of their contained breath /bʰ, ɡʰ, dʰ, d͡zʰ, d͡ʒʰ/ (<bh, gh, dh, jh>), still present a certain level of voiceless aspiration till today, such that an untrained ear would hear them as voiceless aspirates. In this, the French here have the same sounds as <p, k, t>. It is this breath which, by getting a bit stronger, the preceding element became voiceless, and this created the group of voiceless aspirated consonants. 

\subsubsection{Monophthong vowel changes}
\paragraph{Classical Armenian /e/ <\armenian{ե}> }

For the changes in vowels and diphthongs, we note that Classical /e/ <\armenian{ե}> became /e/ <\armenian{է}> (in all situations, except for the words in Table \ref{tab:Rodosto:phonology:change:vowel:e}). 



\begin{table}[H]
 	\centering 
 	\caption{Change from Classical Armenian /e/ <\armenian{ե}> to /je/ <\armenian{յէ}> in the Rodosto dialect}
 	\label{tab:Rodosto:phonology:change:vowel:e}
 	\begin{tabular}{|l | ll|ll| ll|}
 		\hline & \multicolumn{2}{l|}{Classical Armenian} &\multicolumn{2}{l|}{> Rodosto} & \multicolumn{2}{l|}{cf. SEA} \\ 
`I' &es & \armenian{ես} & jes & \armenian{յէս} & jes& \armenian{ես} \\
 \armenian{՝}when' & eɾb & \armenian{երբ} & jepʰ & \armenian{յէփ} & jeɾpʰ & \armenian{երբ} \\
 \armenian{՝}song' & eɾɡ & \armenian{երգ} & jeɾkʰ & \armenian{յէրք} & jeɾkʰ & \armenian{երգ} \\
\hline 
 	\end{tabular}
\end{table}


\paragraph{Classical Armenian /o/ <\armenian{ո}>} 

The Classical /o/ <\armenian{ո}> becomes /vo/ <\armenian{վօ}> at the beginning of monosyllabic words; it becomes /o/ <\armenian{օ}> everywhere else.


\subsubsection{Diphthongal vowel changes} 

The diphthongs changed as follows:\begin{itemize}
 \item CA /ɑi̯/ <\armenian{այ}> $\rightarrow$ /ɑ/ <\armenian{ա}> 
 \item CA /oi̯/ <\armenian{ոյ}> $\rightarrow$ /u/ <\armenian{ու}> 
 \item CA /iu̯/ <\armenian{իւ}> $\rightarrow$ /u/ <\armenian{ու}> 
\end{itemize}

\section{Morphology}
\subsection{Noun inflection or declension}


In the grammar, the declensions have no differences at all from Istanbul. For words with the CA ending /-utʰiu̯n/ <-\armenian{ութիւն}>, the ablative is only a repeated /n/ <\armenian{ն}> (Table \ref{tab:Rodosto:morpho:noun:utjun}). 

 


\begin{table}[H]
 	\centering 
 	\caption{Ablative marking of nominalizer suffix from Classical /-utʰiu̯n/ in the Rodosto dialect}
 	\label{tab:Rodosto:morpho:noun:utjun}
 	 
 \begin{tabular}{|l| ll|ll|}
 \hline & \multicolumn{2}{|l| }{`greatness'} &\multicolumn{2}{l| }{`greatness ({\abl}'} \\
 \hline 
 Classical Armenian & met͡sutʰiu̯n & \armenian{մեծութիւն} & met͡sutʰen-ē & \armenian{մեծութենէ} \\
 > Rodosto & & & mend͡zutʰen-ne & \armenian{մէնձութէննէ} \\
 cf. SWA & med͡zutʰʏn & \armenian{մեծութիւն} & med͡zutʰen-e & \armenian{մեծութենէ} \\
 cf. SEA & met͡sutʰjun & \armenian{մեծություն} & met͡sutʰjun-it͡sʰ & \armenian{մեծությունից} \\
 \hline \end{tabular}
 
\end{table}

\subsection{Numeral formation}

For the numeral adjectives, what is... 

\begin{adjarianpage}\label{page:260}\end{adjarianpage}% should be 260

interesting are the words in Table \ref{tab:Rodosto:morpho:noun:numeral}a, which are however present in Istanbul and other areas, where they require Armenian units during enumeration. In contrast, in Rodosto, the units are also Turkish, as in Table \ref{tab:Rodosto:morpho:noun:numeral}b; whereas the other decades (10-60) take Armenian units. 

\translatorHD{To clarify, he means that Rodosto has borrowed Turkish numerals to replace some Armenian numerals. }

\begin{table}[H]
 	\centering 
 	\caption{Borrowed numerals in the Rodosto dialect}
 	\label{tab:Rodosto:morpho:noun:numeral}
 	\begin{tabular}{|ll | ll| l|ll|}
 		\hline & &\multicolumn{2}{l|}{Rodosto} &\multicolumn{1}{l|}{Turkish} & \multicolumn{2}{l|}{cf. SEA} \\ 
a. & `70' & jetʰmiʃ & \armenian{յէթմիշ} & <yetmiş> & jotʰɑnɑsun & \armenian{յոթանասուն} \\
&`80' & sekʰsen & \armenian{սէքսէն} & <seksen> & utʰsun & \armenian{ութսուն} \\
& `90' & doχsɑn & \armenian{դօխսան} & <doksan> & innəsun & \armenian{իննսուն} \\
b. & `75' & jetʰmiʃ beʃ & \armenian{յէթմիշ բէշ} & <yetmiş beş> & jotʰɑnɑsun hiŋɡ & \armenian{յոթանասուն հինգ}\\
&`81' & sekʰsen biɾ & \armenian{սէքսէն բիր} & <seksen bir> & utʰsun mek & \armenian{ութսուն մեկ}\\
& `93' & doχsɑn ʏt͡ʃʰ & \armenian{դօխսան իւչ} & <doksan üç> & innəsun jeɾekʰ & \armenian{իննսուն երեք}\\
\hline 
 	\end{tabular}
\end{table}

\subsection{Pronoun inflection or declension}
The pronouns are also the same as in Istanbul. Here we find that the first person accusative is /jes/. For the others, notable forms are in Table \ref{tab:Rodosto:morpho:pronoun:dative}, which are either dative or accusative in Istanbul, but they are only dative in Rodosto. 


\begin{table}[H]
 	\centering 
 	\caption{Sample of dative pronouns in the Rodosto dialect}
 	\label{tab:Rodosto:morpho:pronoun:dative}
 	\begin{tabular}{|l ll| }
\hline 
dative 1SG `to me' & ind͡zi & \armenian{ինձի} \\
dative 1PL `to us' & mezi & \armenian{մէզի} \\
dative 2SG `to you' & kʰezi & \armenian{քէզի} \\
dative 2PL `to you' & t͡sʰezi & \armenian{ցէզի} \\
\hline 
 	\end{tabular}
\end{table}

The accusative forms are in Table \ref{tab:Rodosto:morpho:pronoun:accusative}. 



\begin{table}[H]
 	\centering 
 	\caption{Sample of accusative pronouns in the Rodosto dialect}
 	\label{tab:Rodosto:morpho:pronoun:accusative}
 	\begin{tabular}{|l ll| }
\hline 
accusative 1PL `to us' & mez & \armenian{մէզ} \\
accusative 2SG `to you' & kʰez & \armenian{քէզ} \\
accusative 2PL `to you' & t͡sʰez & \armenian{ցէզ} \\
\hline 
 	\end{tabular}
\end{table}



For the third person pronouns, we cite the words in Table \ref{tab:Rodosto:morpho:pronoun:dem} and so on. \translatorHD{Note that Adjarian calls them as just third person pronouns, but based on their SWA cognates, these pronouns act as third person demonstrative pronouns. }



\begin{table}[H]
 	\centering 
 	\caption{Sample of third person demonstrative pronouns in the Rodosto dialect}
 	\label{tab:Rodosto:morpho:pronoun:dem}
 	\begin{tabular}{|l l | ll| }
\hline 
\multicolumn{2}{|l|}{Singular `this'} & \multicolumn{2}{|l|}{Singular `these'} \\
ɑs & \armenian{աս} & ɑsonkʰ & \armenian{ասօնք} \\
ɑsiɡɑ & \armenian{ասիգա} & svonkʰ & \armenian{սվօնք}\\
ɑsiɡɑɡ & \armenian{ասիգագ} & & \\
ɑsiɡɑɡə & \armenian{ասիգագը} & & \\
səviɡɑ & \armenian{սըվիգա}& & \\
səviɡɑɡə & \armenian{սըվիգագը} & & \\
\hline 
 	\end{tabular}
\end{table}



\subsection{Verb inflection or conjugation}

\subsubsection{Theme vowel changes}
In conjugation, the Classical vowel /e/ <\armenian{ե}. becomes /i/ <\armenian{ի}> next to nasals. 

\translatorHD{To clarify, Adjarian provides the paradigm of the indicative present (Table \ref{tab:Rodosto:morpho:verb:paradigm:presentPastIndc}. In SWA, the theme vowel for a verb like `to like' stays a constant /e/. But in Rodosto, the theme vowel changes to /i/ before nasal suffixes. }


\begin{table}[H]
 	\centering 
 	\caption{Theme vowel changes in the indicative present <\armenian{ներկայ}> of the verb `to like' in the Rodosto dialect}
 	\label{tab:Rodosto:morpho:verb:paradigm:presentPastIndc}
 	 \begin{tabular}{|l| ll| ll|}
 		\hline & \multicolumn{2}{l|}{Rodosto } & \multicolumn{2}{l|}{cf. SWA} \\ \hline 
1SG & ɡə siɾ-i-m & \armenian{գը սիրիմ} & ɡə siɾ-e-m & \armenian{կը սիրեմ} \\
2SG & ɡə siɾ-e-s &\armenian{գը սիրէս} & ɡə siɾ-e-s & \armenian{կը սիրես} \\
3SG & ɡə siɾ-e-$\emptyset$ & \armenian{գը սիրէ} & ɡə siɾ-e-$\emptyset$ & \armenian{կը սիրէ} \\
1PL & ɡə siɾ-i-nkʰ & \armenian{գը սիրինք} & ɡə siɾ-e-ŋkʰ & \armenian{կը սիրենք} \\
2PL & ɡə siɾ-e-kʰ & \armenian{գը սիրէք} & ɡə siɾ-e-kʰ & \armenian{կը սիրէք} \\
3PL & ɡə siɾ-i-n& \armenian{գը սիրին} & ɡə siɾ-e-n & \armenian{կը սիրեն} \\
 		& \multicolumn{2}{l|}{{\ind} $\sqrt{}$-{\thgloss}-{\agr}} & \multicolumn{2}{l|}{{\ind} $\sqrt{}$-{\thgloss}-{\agr}} \\ 
 		\hline 
 		 
 	\end{tabular}
\end{table}

\subsubsection{Progressive marking with /ɡoɾ, ɡo, oɾ/ <\armenian{գօր, գօ, օր}> }

The progressive is formed with the formative /ɡoɾ/ <\armenian{գօր}>, against which we sometimes find /ɡo/ <\armenian{գօ}> or /oɾ/ <\armenian{օր}> (Table \ref{tab:Rodosto:morpho:verb:paradigm:prog}. 



\begin{table}[H]
 	\centering 
 	\caption{Variation in progressive marking in the Rodosto dialect}
 	\label{tab:Rodosto:morpho:verb:paradigm:prog}
 	 \begin{tabular}{|l| ll| ll|}
 		\hline & \multicolumn{2}{l|}{Rodosto } & \multicolumn{2}{l|}{cf. SWA} \\ \hline 
`I am eating' & ɡ-ud-i-m ɡoɾ & \armenian{գուդիմ գօր} & ɡ-ud-e-m ɡoɾ & \armenian{կ՚ուտեմ կոր} \\
& ɡ-ud-i-m ɡo & \armenian{գուդիմ գօ} & & \\
& ɡ-ud-i-m oɾ & \armenian{գուդիմ օր} & & \\
 		& \multicolumn{2}{l|}{{\ind} eat-{\thgloss}-{\agr} {\prog}} & \multicolumn{2}{l|}{{\ind} eat-{\thgloss}-{\agr} {\prog}} \\ 
 		\hline 
 		 
 	\end{tabular}
\end{table}

\translatorHD{To clarify, in SWA, the present progressive is formed by adding the progressive enclitic /ɡoɾ/ after the indicative present. In Rodosto, the shape of this progressive marker can vary. }


\subsubsection{Archaism in the past plural suffix}

The plural of the imperfective and perfective has the vowel /ɑ/ <\armenian{ա}>, similarly to the dialects of Sebastia (\translatorHD{\S\ref{sec:Sebastia:morphology:verb:archaicPlPst}}) and Akn (\translatorHD{\S\ref{sec:Akn:morphology:verb:archaicPstPl}}). 


\subsubsection{Future marking }

\translatorHD{In SWA, the future is marked by adding the proclitic /bidi/ before the finite verb. If the finite verb is the present form, then the construction marks the simple future; else if the finite verb is the past imperfective form, then the construction marks the future perfect. Rodosto presents some variation, as Adjarian describes. }


The future is built with the formative /bədə/ <\armenian{բըդը}>, which can be placed also after the verb (\ref{sent:Rodosto:morpho:verb:fut:move}). 

\begin{exe}
 \ex `I will like' \label{sent:Rodosto:morpho:verb:fut:move}
 \begin{xlist}
 \ex Rodosto \gll
 bədə siɾ-i-m \\
 {\fut} like-{\thgloss}-1{\sg} \\
 \trans \armenian{բըդը սիրիմ}
 \ex Rodosto \gll
 siɾ-i-m bədə\\
 like-{\thgloss}-1{\sg} {\fut} \\
 \trans \armenian{սիրիմ բըդը}
 \ex cf. SWA \gll
 bidi siɾ-e-m \\
 {\fut} like-{\thgloss}-1{\sg} \\
 \trans \armenian{պիտի սիրեմ}
 \end{xlist}
 
\end{exe}


It shortens to /bəd/ <\armenian{բըդ}> next to vowels (\ref{sent:Rodosto:morpho:verb:fut:short}).

 

\begin{exe}
 \ex `I will do' \label{sent:Rodosto:morpho:verb:fut:short}
 
 \begin{xlist}
 \ex Rodosto \gll
 bəd ɑn-i-m \\
 {\fut} do-{\thgloss}-1{\sg} \\
 \trans \armenian{բըդ անիմ}
 \ex cf. SWA \gll
 bidi ən-e-m \\
 {\fut} do-{\thgloss}-1{\sg} \\
 \trans \armenian{պիտի ընեմ}
 \end{xlist} 
\end{exe}

The old ones also have /bədəɾ/ <\armenian{բըդըր}>, which originates from the form: CA /piti oɾ/ <\armenian{պիտի որ}> `it is necessary that' (\ref{sent:Rodosto:morpho:verb:fut:long}). 

\begin{exe}
\ex \label{sent:Rodosto:morpho:verb:fut:long} \begin{xlist}
\ex `I will like'
\begin{xlist}
 \ex Rodosto \gll
 bədəɾ siɾ-i-m \\
 {\fut} like-{\thgloss}-1{\sg} \\
 \trans \armenian{բըդըր սիրիմ}
 \ex cf. SWA \gll
 bidi siɾ-e-m \\
 {\fut} like-{\thgloss}-1{\sg} \\
 \trans \armenian{պիտի սիրեմ}
\end{xlist}
\ex `I was going to like'
\begin{xlist}
 \ex Rodosto \gll
 bədəɾ siɾ-ej-i-$\emptyset$ \\
 {\fut} like-{\thgloss}-{\pst}-1{\sg} \\
 \trans \armenian{բըդըր սիրէյի}
 \ex cf. SWA \gll
 bidi siɾ-ej-i-$\emptyset$ \\
 {\fut} like-{\thgloss}-{\pst}-1{\sg} \\
 \trans \armenian{պիտի սիրէի}
\end{xlist}
\end{xlist}
\end{exe}

\section{Text samples}

{\sampleoverview}


Adjarian's source: These two articles were narrated especially to me by a large group of happy young people from Rodosto, and their chief was the grocer Mr. Hakop Malakian (\armenian{պր. Յակոբ Մալաքեան}; \translatorHD{SEA: /hɑkop mɑlɑkʰjɑn/, SWA:/hɑɡopʰ/}). I wrote this with the scientific orthography. 

\subsection{Sample 1}


\armenian{Դարի մը մէնք չէթէյով (խումբ) էլանք Իշէն քացանք. ջանփան ջուոջինանիս (մեր խեղկատակութիւն) շադ ղըյախ էր. արաբային էռչին ըռէյիզը նսդաձ էր. սահաթը քիշէրվան ալ օխդն ու գէսը գար. ըռէյիզը թէնջիրէ մը ղափուսխան ցառքը փռնաձ}... 

\begin{adjarianpage}\label{page:261}\end{adjarianpage}% should be 261

... \armenian{էր. ան քանի՛ արաբան գէրէա գօ նէ՝ իշդար մէչը յէղ գա նէ վրան քլօխը մէյ մը արավ։ Ինչ է նէ. վէ՛լասըլ (վերջապէս) Աղային ախփուրը հասանք։ Օնդօղօցը մէր արաբաջին իշգիւզարութին մ՚անէլ ուզէց, մանդանէրուն չըլբըրէն (սանձ) փռնէց, ի՛շդար ղուվէթ ունէր նէ՝ քաշէց էշօղլուն։ Գուզէ՞ս արաբան բաթմիշ ըլլա չամուոին մէչը Մանդանէրը բաթմիշ էղան, բօյունդուրըխը (լուծ) գօյրէցավ, մէնք ըսէս նէ՝ ամէննիս ա մէգիգ մէգիգ՝ փաչէրնիս սօթդաձ՝ չափուռ չուփուռ չուրէրէն թուս էլանք։ Հիմա բաշլէյէցանք Չիրիշին փօթուոին ղալայը գօխէլ (հայհոյել)։ Չիրիշ ըսաձս ա՝ ղաթըր մը գա նէ վէրը՝ ան հայվանն է։ Ի՛նչ է նէ. չէրգընցընինք, բէրէքէթ աս արաբաջիին աոխադաշը՝ Զըմբըռն ալ մէգդէղ էր։ Շը՛փդիյի (իսկոյն) անօր է էօքիւզնէրուն բոյունդուրուխնէրը աս մէրինին թախմիշ արանք. աս էշօղլու ալ գէնէ գօյրէցավ։ Յէթքը քացանք քօվի հարմանէն բօյունդուրուխ մը քէրվէցանք (ծծկ. գողնալ)։ Բէրէքէթ անօր վօր մէզ գէօլէն սէլամէթը հանէց։ Ալթըխ ջանփանիս ըռահաթ ըռահաթ քացանք։ Լաքին քիշ մ՚անթին գուզէ՞ս բայիրէն վար թախըր թուխըր արաբան թօնգօլէցավ, թէքիրլախին մէգը գօյրէցավ}…\armenian{։}

\subsection{Sample 2}

– \armenian{Ձէ Արթին, վո՞ւրգէ գուքաս գոր։ Շադօնց է քէզ դէսաձ չէյի։}

– \armenian{Հօ՞ս էյի վօր դէսնայիր։}

– \armenian{Հաբա վո՞ւր էիր։}

– \armenian{Չիյդէ՞ս}… \armenian{Բօլիս չէյի՞ յա՜։}

– \armenian{Յէ՞փ քացիր։}

– \armenian{Ջա՛նըմ. Ըսթանբօլէն բօսթանջի մը էգէր էր մալ առնէլու համար, յէս խանդըրմիշ արավ՝ ըսավ քի «աս դարի խարփուզը վէրը աղէգ գը փռնըվի գօր». շիյդագը հէ՛մէն հավդըցա։ Էրգու խայըխ փռնէցի, խարփուզնէրը լէցուցի, յալլահ Ըսթանբօլ։ Իրիգվան թէմ, սահաթը սանգ գէսի վրա էր, ջանփա էլանք։ Ինչվանք Էրէյիլի փացէրը ըռահաթ քացանք։ Է՛հ, իշդէ մութը աղէգ մը գօխաձ էր՛ մէյ մըն ա խըյախ լօդօս մը բաշլայէց փչէլ}… \armenian{Հիմա ի՞նչ անինք}… \armenian{Բէրէքէթ խափդաննիս իւշիւզար մարթ էր. Շաշըրմիշ չարավ. դիւմէնին քլօխը անցաձ՝ խայըխը աղէգ քշէց. անագ վօր գիւջբէլէգիւջ ինքըզինքնիս Սիրիվլի նէդէցինք։ Ան քիշէրը հօն լուսցուցանք. էրթէսի օրը ուզաձ հավանիս քդանք}, ...



\begin{adjarianpage}\label{page:262}\end{adjarianpage}% should be 262

... \armenian{քանի մը սահթըվան մէչ Ըսթանբօլ՝ Սանդըք-Բուրուն էրգաթ նէդէցինք։ Ինչ է նէ, էլանք, խարփուզնէրը բարբէցանք։ Էգու դէս քի, ի՞նչ փիյացա}… \armenian{չըսէ՞ս քի հարցունօղ էղա՞վ մի}… \armenian{մէէր իսէ աս դարի հիվընդութին գա ըսէլօվ՝ ժօղօվուրթը վախցուցէր ին, անմէն մարթ խարթուզ ուդէլը գը վախնա գօ. անագ վօր իրէք հարու հիսսուն խուռուշ զէնօվ ցէռվընուս դէֆ արինք քացավ։}

– \armenian{Է, հիմա ի՞նչ բըդ անէս։}

– \armenian{Ի՞նչ բըդ տնիմ}… \armenian{գէնէ յէս իմ թէռլիքջութանս նայիմ. ա՛խփար, զէնաաթէն աղէգ փան գա՞. «զէնաաթը էլմաս բիլէզիգ է» ըսէր ին նէ՝ բօշ դէղը չէ՛ յա՜։ Զաթէն էռչի վարբէդս ա յէս դէսաձին բէս՝ գէնէ քօվը գանչեց, յէթմիշ բէշ խուռուշ հաֆթալըխօվ։ Ի՞նչ մէխքըս բահիմ. նօրէն զէնաաթիս բըդը նսդիմ վէ՛սսէլամ։}

\subsection{Malkara}

Adjarian's source: See \citeauthor{Byurakn} 1898, page 756



\armenian{Հանսա երթանք Գալիլիա},

\armenian{Գալիլիա լեռ մը կայ},

\armenian{Լեռան մէջը ծով մը կայ},

\armenian{Ծովուն մէջը ծառ մը կայ},

\armenian{Ծառին վրայ բուն մը կայ},

\armenian{Բունին մէջը օձ մը կայ}

\armenian{Օձն օխտը պտուկ ունի}.

\armenian{Կթեցինք մակրդեցինք},

\armenian{Տիկ մը պանիր կոխեցինք}.

\armenian{Ով կերաւ՝ արմնցաւ},

\armenian{Ով չկերաւ՝ զարմացաւ}.

\armenian{Աչք տուողին աչքը ճաթի},

\armenian{Չար աչքը, չար պտողը ճաթի։}


\chapter{Crimea}\label{chapter:Crimea}
\section{Background}

\begin{adjarianpage}\label{page:263}\end{adjarianpage}% should be 263

This dialect was first spoken only in Crimea. In 1779, a large Armenian migration group left Crimea and migrated to Southern Russia, where they established the city of New Nakhichevan and its 5 surrounding Armenian villages. From here, the Armenians spread likewise to near and far Russian cities, such as Rostov, Stavropol, Maykop, Yekaterinodar, Yekaterinoslav, Taganrog, Dnipro, Nogaisk, Novocherkassk. The small Armenian settlements of these places speak the New Nakhichevan dialect. The Armenian-populated cities of Crimea are now Theodosia, Simferopol, Karasubazar, Bakhchisaray, and Yevpatoriya, which speak the same Armenian dialect. But Kerch, Yalta, and Sevastopol, as we say, represent more of a settlement from Trabzon. 


\section{Phonology}
\subsection{Segment inventory}
The dialect of Crimean is very close to the Istanbul dialect. Like the latter, it has the vowels /ɑ, e, ə, i, o, u, ʏ/ <\armenian{ա, է, ը, ի, օ, ու, իւ}>. The sound /ʏ/ <\armenian{իւ}> is used only in loanwords from Turkish and Tatar. But the sound /œ/ <\armenian{էօ}> is absent. This sound has changed in Turkish words into /e/ <\armenian{է}>. For example, Crimea /beɾek/ <\armenian{բէրէք}> from Turkish <börek> `burek'. 

There are no diphthongs. 

The consonants have only two degrees: voiced and voiceless aspirated. The Armenian voiced and voiceless unaspriated consonants became voiced, while the voiceless aspirated stayed the same. 

 
\subsection{Sound changes: Lenition of Classical /ɾ/ <\armenian{ր}> }
The use of the sound /ɾ/ <\armenian{ր}> in New Nakhichevan is very interesting. The old ones pronounce it as /ɾ/ <\armenian{ր}> in every condition. But in the new generation, the pronunciation is halted.\footnote{\translatorHD{For the word `halted', Adjarian uses the verb <\armenian{կաղայ}> which means `to limp' or `to halt'. I think Adjarian was trying to use a metaphorical meaning of this verb, but his exact intention is unclear to me. }} For them, the sound /ɾ/ <\armenian{ր}> is often very soft, almost close to the pronunciation of /ʒ/ <\armenian{ժ}>, which should of course have its own representation (\armenian{ր՜}). \translatorHD{It's unclear to me what Adjarian perceived as this weak rhotic symbol <\armenian{ր՜}>; I suspect he means the rhotic is lenited somehow. Note that he doesn't later use this symbol in his transcriptions anyway. }


This sound \armenian{ր՜} changes based on the previous and following sounds. Between the Classical sounds /i/ <\armenian{ի}> and /e/ <\armenian{ե}>, it becomes a simple... 


\begin{adjarianpage}\label{page:264}\end{adjarianpage}% should be 264

... /ʒ/ <\armenian{ժ}> (\ref{tab:Crimea:phonology:change:cons:r:z}). 

\translatorHD{Note that for the following data, it seems that Adjarian assumes that the Classical forms with an initial /e/ or /iu̯/ changed to */i/ in an intermediate hypothetical stage, and this */i/ then triggered the lenition of the rhotic to a fricative.}


\begin{table}[H]
 	\centering 
 	\caption{Change from Classical Armenian /ɾ/ <\armenian{ր}> to /ʒ/ <\armenian{ժ}> in the Crimea dialect}
 	\label{tab:Crimea:phonology:change:cons:r:z}
 	\begin{tabular}{|l | ll|ll| ll|}
 		\hline & \multicolumn{2}{l|}{Classical Armenian} &\multicolumn{2}{l|}{> Crimea} & \multicolumn{2}{l|}{cf. SEA} \\ 
 `three' &eɾekʰ & \armenian{երեք} & ʒekʰ & \armenian{ժէք} &jeɾekʰ, iɾekʰ & \armenian{երեք, իրեք} \\
 `they ({\nom})' &iu̯ɾe̯ɑnkʰ & \armenian{իւրեանք} & ʒenkʰ & \armenian{ժէնք} & iɾeŋkʰ & \armenian{իրենք} \\
 `their ({\gen})' &iu̯ɾe̯ɑnt͡sʰ & \armenian{իւրեանց} & ʒent͡sʰ & \armenian{ժէնց} & iɾent͡sʰ & \armenian{իրենց} \\
\hline 
 	\end{tabular}
\end{table}



Next to a dental voiceless aspirate /tʰ/ <\armenian{թ}>, it becomes /ʃ/ <\armenian{շ}> (Table \ref{tab:Crimea:phonology:change:cons:r:s}).


\begin{table}[H]
 	\centering 
 	\caption{Change from Classical Armenian /ɾ/ <\armenian{ր}> to /ʃ/ <\armenian{շ}> in the Crimea dialect}
 	\label{tab:Crimea:phonology:change:cons:r:s}
 	\begin{tabular}{|l | ll|ll| ll|}
 		\hline & \multicolumn{2}{l|}{Classical Armenian} &\multicolumn{2}{l|}{> Crimea} & \multicolumn{2}{l|}{cf. SEA} \\ 
 `to go' & eɾtʰɑl & \armenian{երթալ} & eʃtʰɑl & \armenian{էշթալ} & jeɾtʰɑl & \armenian{երթալ} \\
`man' &mɑɾd & \armenian{մարդ} & mɑʃtʰ & \armenian{մաշթ} &mɑɾtʰ & \armenian{մարդ} \\
`skin' &moɾd & \armenian{մորթ} & moʃtʰi & \armenian{մօշթի} &moɾtʰ & \armenian{մորթ} \\
\hline 
 	\end{tabular}
\end{table}


We are in the presence of the formation of a new phonetic rule, which is still not completely dominant. 

\section{Morphology}
\subsection{Pronoun inflection or declension}

Declension and conjugation are again very similar to the Istanbul dialect. Only that the accusative is the same as the dative, as in the /um/ <\armenian{ում}> branch. We place here the pronouns which show some differences from the Istanbul dialect (Table \ref{tab:Crimea:morpho:pronoun:personal}). 

 \begin{table}[H]
 	\centering 
 	\caption{Declension paradigm for personal pronouns in the Crimea dialect}
 	\label{tab:Crimea:morpho:pronoun:personal}
\begin{tabular}{|l|lll| lll|}
\hline & 1SG & 2SG & 3SG & 1PL & 2PL & 3PL 
\\
& `I' & `you' & `he' & `we' & `you' & `they' \\ 
\hline 
{\nom} & jes & dun & nɑ & minkʰ, menkʰ & dukʰ & nɑkʰɑ \\
 & \armenian{յէս} & \armenian{դուն} & \armenian{նա} & \armenian{մինք, մէնք} & \armenian{դուք} & \armenian{նաքա} \\\hline 
{\gen} & im & kʰu & nɑɾɑ & meɾ & d͡zeɾ & nɑt͡sʰɑ \\
 & \armenian{իմ} & \armenian{քու} & \armenian{նարա} & \armenian{մէր} & \armenian{ձէր} & \armenian{նացա} \\\hline 
{\dat}-{\acc} & ənd͡zi & kʰezi & nɑɾɑn & mezi & d͡zezi & nɑt͡sʰɑ \\
 & \armenian{ընձի} & \armenian{քէզի} & \armenian{նարան} & \armenian{մէզի} & \armenian{ձէզի} & \armenian{նացա} \\\hline 
{\abl} & ənd͡zi-men & kʰezi-men & nɑɾɑ-men & mezi-men & d͡zezi-men & nɑt͡sʰɑ-men \\
 & \armenian{ընձիմէն} & \armenian{քէզիմէն} & \armenian{նարամէն} & \armenian{մէզիմէն} & \armenian{ձէզիմէն} & \armenian{նացամէն} \\\hline 
{\ins} & ənd͡zi-mov & kʰezi-mov & nɑɾɑ-mov & mezi-mov & d͡zezi-mov & nɑt͡sʰɑ-mov \\
 & \armenian{ընձիմօվ} & \armenian{քէզիմօվ} & \armenian{նարամօվ} & \armenian{մէզիմօվ} & \armenian{ձէզիմօվ} & \armenian{նացամօվ} \\ \hline 
\end{tabular}\end{table}

\translatorHD{Adjarian lists various demonstrative pronouns that act as different forms for the proximal pronoun `this' (Table \ref{tab:Crimea:morpho:pronoun:dem}). }

 \begin{table}[H]
 	\centering 
 	\caption{Declension paradigm for the proximal demonstrative pronoun `this' and its various forms in the Crimea dialect}
 	\label{tab:Crimea:morpho:pronoun:dem}
\begin{tabular}{|l|ll ll|}
\hline 
{\nom} & isɑ & ɑs & ɑsviɡə & sɑ \\
 & \armenian{իսա} & \armenian{աս} & \armenian{ասվիգը} & \armenian{սա} \\
{\gen}-{\dat} & isəvoɾ & ɑsoɾ & ɑsəvoɾ & səvoɾ \\
 & \armenian{իսըվօր} & \armenian{ասօր} & \armenian{ասըվօր} & \armenian{սըվօր} \\
{\abl} & isəvoɾ-me & ɑsoɾ-me & ɑsəvoɾ-me & səvoɾ-me \\
 & \armenian{իսըվօրմէ} & \armenian{ասօրմէ} & \armenian{ասըվօրմէ} & \armenian{սըվօրմէ} \\
{\ins} & isəvoɾ-mov & ɑsoɾ-mov & ɑsəvoɾ-ov & səvoɾ-mov \\
 & \armenian{իսըվօրմօվ} & \armenian{ասօվ} & \armenian{ասըվօվ} & \armenian{սըվօրմօվ} \\ \hline 
\end{tabular}\end{table}

What is also said are the forms in Table \ref{tab:Crimea:morpho:pronoun:dem:other} which are declined in the same way.

\begin{table}[H]
 \centering
 \caption{Sample of other demonstrative pronouns in the Crimea dialect}
 \label{tab:Crimea:morpho:pronoun:dem:other}
 \begin{tabular}{|ll|ll|}
 \hline \multicolumn{2}{|l| }{Medial {\nom} {\sg} `that'} & \multicolumn{2}{|l| }{Distal {\nom} {\sg} `that yonder'} \\
 \hline 
 idɑ & \armenian{իդա}& inɑ & \armenian{ինա} \\
 ɑd & \armenian{ադ} & ɑn & \armenian{ան} \\
 ɑdəvoɾ & \armenian{ադըվօր} & ɑnəvoɾ & \armenian{անըվօր} \\
\hline 
\end{tabular}
\end{table}

\subsection{Numerals}
To form the ordinal numerals, the formative /-um/ <\armenian{ում}> is used (Table \ref{tab:Crimea:morpho:numeral:suffix}). \translatorHD{This is in contrast to CA and SEA/SWA which use the ordinal suffix /-(e)ɾoɾtʰ/}. 



\begin{table}[H]
 	\centering 
 	\caption{Ordinal numerals in the Crimea dialect}
 	\label{tab:Crimea:morpho:numeral:suffix}
 	\begin{tabular}{|l | ll|ll| ll|}
 		\hline & \multicolumn{2}{l|}{Classical Armenian} &\multicolumn{2}{l|}{> Crimea} & \multicolumn{2}{l|}{cf. SEA} \\ 
 \armenian{՝}two' & eɾku & \armenian{երկու} & & & jeɾku & \armenian{երկու} \\
 \armenian{՝}second' & eɾk-ɾoɾd & \armenian{երկրորդ} & eɾɡus-um& \armenian{էրգուսում} & jeɾk-ɾoɾtʰ & \armenian{երկրորդ} \\
 three' &eɾekʰ & \armenian{երեք} & & &jeɾekʰ & \armenian{երեք} \\
 \armenian{՝}third' & eɾ-ɾoɾd & \armenian{երրորդ} & ʒəkʰ-um & \armenian{ժէքում}& jeɾ-ɾoɾtʰ & \armenian{երրորդ} \\
`four' &t͡ʃʰoɾs & \armenian{չորս} & & & t͡ʃʰoɾs & \armenian{չորս} \\
 \armenian{՝}fourth' & t͡ʃʰoɾ-ɾoɾd & \armenian{չորրորդ} & t͡ʃʰoɾs-um & \armenian{չօրսում}& t͡ʃʰoɾ-ɾoɾtʰ & \armenian{չորրորդ} \\
\hline 
 	\end{tabular}
\end{table}


This formative /-um/ <\armenian{ում}> is from the Persian formative <-um>, with the same usage. \translatorHD{Note in that in his subsequent work, Adjarian later argued that this suffix was not borrowed from Persian but that it is a re-analysis of the Classical Armenian locative suffix /-m/ <\armenian{մ}> \citep[287ff]{Adjarian-1952-Liakatar4Book1AdjNumeral}. }

\subsection{Verb inflection or conjugation}

\subsubsection{Morphological properties}
\paragraph{Indicative marking}
In conjugation, we must note first the formatives /ɡ-, ɡə-, kʰə-/ <\armenian{գ, գը, քը}> of the present and imperfective. From these, the first is for vowel-initial verbs, the second for voiced consonant-initial verbs, and the third for voiceless aspirated-initial verbs (Table \ref{tab:Crimea:morpho:verb:change:indcMorpheme}). 

\translatorHD{To clarify, he means that the indicative morpheme is a prefix that displays allomorphy based on the type of verb-initial segment, including voicing assimilation. In contrast for the SWA cognates, we see a simpler type of allomorphy based on schwa epenthesis, without voicing assimilation.}


\begin{table}[H]
 	\centering 
 	\caption{Allomorphy of the indicative prefix in the Crimea dialect}
 	\label{tab:Crimea:morpho:verb:change:indcMorpheme}
 	\begin{tabular}{|l | ll| ll|}
 		\hline & \multicolumn{2}{l|}{Crimea} & \multicolumn{2}{l|}{cf. SWA} \\ 
`I go' & ɡ-eʃtʰ-ɑ-m & \armenian{գէշթամ} & ɡ-eɾtʰ-ɑ-m & \armenian{կ՚երթամ}\\ 
`I bring' & ɡə beɾ-i-m & \armenian{գը բէրիմ} & ɡə pʰeɾ-e-m & \armenian{կը բերեմ} \\
`I like' & kʰə siɾ-i-m & \armenian{քը սիրիմ} & ɡə siɾ-e-m & \armenian{կը սիրեմ} \\
& \multicolumn{2}{l| }{{\ind} $\sqrt{}$-{\thgloss}-1{\sg}}& \multicolumn{2}{l| }{{\ind} $\sqrt{}$-{\thgloss}-1{\sg}}\\
\hline 
 	\end{tabular}
\end{table}


\begin{adjarianpage}\label{page:265}\end{adjarianpage}% should be 265

 

\paragraph{Theme vowel changes }

The verbal ending /e/ <\armenian{ե}> becomes /i/ <\armenian{ի}> everywhere, except for the 3SG present. 

\translatorHD{To clarify, he means that for the E-Class, the theme vowel is /e/ in Classical Armenian and in SWA/SEA. But in Crimea, the reflex of this theme vowel is /i/, except in the present 3SG. We see examples of this change in \S\ref{sec:Crimea:morphology:verb:paradigm:subj}. }

\translatorHD{This change is likewise used in the present auxiliary (\S\ref{sec:Crimea:morphology:verb:paradigm:prespstperfect}). It does not occur for the theme vowel of the past imperfective (\S\ref{sec:Crimea:morphology:verb:paradigm:subj}) or past perfective (\S\ref{sec:Crimea:morphology:verb:paradigm:aor}). }

\paragraph{Class of the causative}

The causative verbs take the ending /t͡sʰnul/ <\armenian{ցնուլ}>, and they form the fourth conjugation class (Table \ref{tab:Crimea:morpho:verb:change:caus}). \translatorHD{To clarify, he means that causatives verb take the theme vowel /u/ in Crimea; in contrast, they take the theme vowel /e/ in SEA/SWA.} 


\begin{table}[H]
 	\centering 
 	\caption{Causative verbs in the Crimea dialect}
 	\label{tab:Crimea:morpho:verb:change:caus}
 	\begin{tabular}{|l | ll| ll|}
 		\hline & \multicolumn{2}{l|}{Crimea} & \multicolumn{2}{l|}{cf. SWA} \\ 
`to pass (trans.)' & ɑn-t͡sʰə-n-u-l & \armenian{անցընուլ} & ɑn-t͡sʰə-n-e-l & \armenian{անցընել}\\ 
`to ask' & hɑɾ-t͡sʰə-n-u-l & \armenian{հարցընուլ} & hɑɾ-t͡sʰə-n-e-l & \armenian{հարցնել}\\ 
`to melt (trans.)' & hɑl-e-t͡sʰ-n-u-l & \armenian{հալէցնուլ} & hɑl-e-t͡sʰ-n-e-l & \armenian{հալեցնուլ}\\ 
& \multicolumn{2}{l| }{$\sqrt{}$-({\thgloss})-{\caus}-{\thgloss}-{\infgloss}}& \multicolumn{2}{l| }{$\sqrt{}$-{\thgloss})-{\caus}-{\thgloss}-{\infgloss}}\\ 
\hline 
 	\end{tabular}
\end{table}


\subsubsection{General paradigm}

The following are important tenses of the verb `to like'. 

{\paradigmExplanation}

\paragraph{Subjunctive present and past imperfective}\label{sec:Crimea:morphology:verb:paradigm:subj}

\translatorHD{In SWA (Table \ref{tab:Crimea:morpho:verb:paradigm:subjPresent}), the subjunctive present is a finite verb form made up of the verb stem, plus a theme vowel, plus agreement suffixes. For a verb like `to like', the theme vowel is a constant /-e-/. The Crimea dialect uses a similar strategy with one difference: the theme vowel is /e/ in the 3SG, but /i/ elsewhere. }

\begin{table}[H]
 \centering
 \caption{Subjunctive present <\armenian{ստորադասական ներկայ}> of the verb `to like' in the Crimea dialect}
 \label{tab:Crimea:morpho:verb:paradigm:subjPresent}
 \begin{tabular}{|l|ll|ll|}
\hline & \multicolumn{2}{l|}{Crimea } & \multicolumn{2}{l|}{cf. SWA} \\
1SG & siɾ-i-m & \armenian{սիրիմ} & siɾ-e-m & \armenian{սիրեմ} \\
2SG & siɾ-i-s & \armenian{սիրիս} & siɾ-e-s & \armenian{սիրես} \\
3SG & siɾ-e-$\emptyset$ & \armenian{սիրէ} & siɾ-e-$\emptyset$ & \armenian{սիրէ} \\
1PL & siɾ-i-nkʰ & \armenian{սիրինք} &siɾ-e-ŋkʰ & \armenian{սիրենք} \\
2PL & siɾ-i-kʰ & \armenian{սիրիք} & siɾ-e-kʰ & \armenian{սիրէք} \\
3PL & siɾ-i-n & \armenian{սիրին} & siɾ-e-n & \armenian{սիրեն} \\
& \multicolumn{2}{l|}{$\sqrt{}$-{\thgloss}-{\agr}}& \multicolumn{2}{l|}{$\sqrt{}$-{\thgloss}-{\agr}}\\ 
\hline 
\end{tabular}
\end{table}

\translatorHD{In SWA, the subjunctive past imperfective (Table \ref{tab:Crimea:morpho:verb:paradigm:subjPast}) is formed by adding the past suffix /i/ and agreement suffixes after the theme vowel. The past suffix is absent in the 3SG. Crimea uses an identical strategy. Note how the theme vowel is /e/ in the past, but almost always /i/ in the present (Table \ref{tab:Crimea:morpho:verb:paradigm:subjPresent}). }



\begin{table}[H]
 \centering
 \caption{Subjunctive past <\armenian{ստորադասական անցեալ}> of the verb `to like' in the Crimea dialect}
 \label{tab:Crimea:morpho:verb:paradigm:subjPast}
 \begin{tabular}{|l|ll|ll|}
\hline & \multicolumn{2}{l|}{Crimea} & \multicolumn{2}{l|}{cf. SWA} \\
1SG & siɾ-e-i-$\emptyset$ & \armenian{սիրէի} & siɾ-ej-i-$\emptyset$ & \armenian{սիրէի} \\
2SG & siɾ-e-i-ɾ & \armenian{սիրէիր} & siɾ-ej-i-ɾ & \armenian{սիրէիր} \\
3SG & siɾ-e-$\emptyset$-ɾ & \armenian{սիրէր} & siɾ-e-$\emptyset$-ɾ & \armenian{սիրէր} \\
1PL & siɾ-e-i-nkʰʲ & \armenian{սիրէինք} & siɾ-ej-i-ŋkʰ & \armenian{սիրէինք} \\
2PL & siɾ-e-i-kʰʲ & \armenian{սիրէիք} & siɾ-ej-i-kʰ & \armenian{սիրէիք} \\
3PL & siɾ-e-i-n & \armenian{սիրէին} &siɾ-ej-i-n & \armenian{սիրէին} \\
& \multicolumn{2}{l|}{$\sqrt{}$-{\thgloss}-{\pst}-{\agr}}& \multicolumn{2}{l|}{$\sqrt{}$-{\thgloss}-{\pst}-{\agr}}\\ 

\hline 
\end{tabular}
\end{table}




\paragraph{Tenses built from the subjunctive: Indicative and future }

 \translatorHD{In Crimea and SWA, many other tenses seem to be built off of the subjunctive (Table \ref{tab:Crimea:morpho:verb:paradigm:complexSubjunctive}). The indicative present and past imperfective are built by adding the indicative prefix before the subjunctive present and subjunctive past; this prefix is /kʰə-/ for the verb `to like'. The future and future perfect are formed also by adding the proclitic /bidi/ before the appropriate subjunctive form. SWA behaves in essentially the same way, and I don't provide its paradigm. }

\begin{table}[H]
 \centering
 \caption{Forms that are built from the subjunctive forms for the verb `to like' in the Crimea dialect}
 \label{tab:Crimea:morpho:verb:paradigm:complexSubjunctive}
 \begin{tabular}{|l|ll|ll|}
\hline & 
\multicolumn{2}{l|}{Indicative present <\armenian{ներկայ}> } & \multicolumn{2}{l|}{Indicative past imperfective <\armenian{անկատար}>} \\
1SG & kʰə siɾ-i-m & \armenian{քը սիրիմ} & kʰə siɾ-e-i-$\emptyset$ & \armenian{քը սիրէի} \\
2SG & kʰə siɾ-i-s & \armenian{քը սիրիս} & kʰə siɾ-e-i-ɾ & \armenian{քը սիրէիր} \\
3SG & kʰə siɾ-e-$\emptyset$ & \armenian{քը սիրէ} & kʰə siɾ-e-$\emptyset$-ɾ & \armenian{քը սիրէր} \\
1PL & kʰə siɾ-i-nkʰ & \armenian{քը սիրինք} & kʰə siɾ-e-i-nkʰ & \armenian{քը սիրէինք} \\
2PL & kʰə siɾ-i-kʰ & \armenian{քը սիրիք} & kʰə siɾ-e-i-kʰ & \armenian{քը սիրէիք} \\
3PL & kʰə siɾ-i-n & \armenian{քը սիրին} & kʰə siɾ-e-i-n & \armenian{քը սիրէին} 
\\
& \multicolumn{2}{l|}{{\ind}-$\sqrt{}$-{\thgloss}-{\agr}}& \multicolumn{2}{l|}{{\ind}-$\sqrt{}$-{\thgloss}-{\pst}-{\agr}}
\\ \hline 
& \multicolumn{2}{l|}{Future <\armenian{ապառնի}>} & \multicolumn{2}{l|}{Future perfect <\armenian{անցեալ ապառնի}> } \\
1SG & bidi siɾ-i-m & \armenian{բիդի սիրիմ} & bidi siɾ-e-i-$\emptyset$ & \armenian{բիդի սիրէի} \\
2SG & bidi siɾ-i-s & \armenian{բիդի սիրիս} & bidi siɾ-e-i-ɾ & \armenian{բիդի սիրէիր} \\
3SG & bidi siɾ-e-$\emptyset$ & \armenian{բիդի սիրէ} & bidi siɾ-e-$\emptyset$-ɾ & \armenian{բիդի սիրէր} \\
1PL & bidi siɾ-i-nkʰ & \armenian{բիդի սիրինք} & bidi siɾ-e-i-nkʰ & \armenian{բիդի սիրէինք} \\
2PL & bidi siɾ-i-kʰ & \armenian{բիդի սիրիք} & bidi siɾ-e-i-kʰ & \armenian{բիդի սիրէիք} \\
3PL & bidi siɾ-i-n & \armenian{բիդի սիրին} & bidi siɾ-e-i-n & \armenian{բիդի սիրէին} 
\\
& \multicolumn{2}{l|}{{\fut} $\sqrt{}$-{\thgloss}-{\agr}}& \multicolumn{2}{l|}{{\fut} $\sqrt{}$-{\thgloss}-{\pst}-{\agr}}
\\\hline \end{tabular}
\end{table}

\paragraph{Present perfect and past perfect}\label{sec:Crimea:morphology:verb:paradigm:prespstperfect}

\translatorHD{In SWA, the present perfect (Table \ref{tab:Crimea:morpho:verb:paradigm:presentPerfect}) and past perfect (Table \ref{tab:Crimea:morpho:verb:paradigm:pastPerfect}) in are formed by combining a special non-finite form with the present/past auxiliary. For SWA, this non-finite verb can be either the resultative participle (verb with suffix /-ɑd͡z/) or the evidential participle (verb with suffix /-eɾ/). In SEA, this non-finite form is the perfective converb with the suffix /-il/\armenian{ղ} Crimea uses a similar system. Adjarian only provides a participle with the suffix /-il/ <\armenian{իլ}>. This suffix appears to be a cognate with the SEA perfective converb suffix /-il/, and I gloss it as such. }

\translatorHD{Note that in SWA, the present auxiliary has the form /e/, but Crimea has the form /i/ for the non-3SG. }

\begin{table}[H]
 \centering
 \caption{Present perfect <\armenian{յարակատար}> of the verb `to like' in the Crimea dialect}
 \label{tab:Crimea:morpho:verb:paradigm:presentPerfect}
 \begin{tabular}{|l|ll|ll|}
\hline & \multicolumn{2}{l|}{Crimea} & \multicolumn{2}{l|}{cf. SWA} \\
1SG & siɾ-il i-m & \armenian{սիրիլ իմ} & siɾ-ɑd͡z e-m & \armenian{սիրած եմ} \\
2SG &siɾ-il i-s & \armenian{սիրիլ իս} & siɾ-ɑd͡z e-s & \armenian{սիրած ես} \\
3SG & siɾ-il e-$\emptyset$ & \armenian{սիրիլ է} & siɾ-ɑd͡z e-$\emptyset$ & \armenian{սիրած է} \\
1PL & siɾ-il i-nkʰ & \armenian{սիրիլ ինք} & siɾ-ɑd͡z e-ŋkʰ & \armenian{սիրած ենք} \\
2PL & siɾ-il i-kʰ & \armenian{սիրիլ իք} & siɾ-ɑd͡z e-kʰ & \armenian{սիրած էք} \\
3PL & siɾ-il i-n & \armenian{սիրիլ ին} & siɾ-ɑd͡z e-n & \armenian{սիրած են} \\
& \multicolumn{2}{l|}{$\sqrt{}$-{\perfcvb} {\aux}-{\agr}}& \multicolumn{2}{l|}{$\sqrt{}$-{\rptcp} {\aux}-{\agr}}\\ 

\hline 
\end{tabular}
\end{table}


\begin{table}[H]
 \centering
 \caption{Past perfect <\armenian{գերակատար}> of the verb `to like' in the Crimea dialect}
 \label{tab:Crimea:morpho:verb:paradigm:pastPerfect}
 \begin{tabular}{|l|ll|ll| }
\hline & \multicolumn{2}{l|}{Crimea} & \multicolumn{2}{l|}{cf. SWA} \\
1SG & siɾ-il e-i-$\emptyset$ & \armenian{սիրիլ էի} & siɾ-ɑd͡z ej-i-$\emptyset$ & \armenian{սիրած էի} \\
2SG & siɾ-il e-i-ɾ & \armenian{սիրիլ էիր} & siɾ-ɑd͡z ej-i-ɾ & \armenian{սիրած էիր} \\
3SG & siɾ-il e-$\emptyset$-ɾ & \armenian{սիրիլ էր} & siɾ-ɑd͡z e-$\emptyset$-ɾ & \armenian{սիրած էր} \\
1PL & siɾ-il e-i-nkʰ & \armenian{սիրիլ էինք} & siɾ-ɑd͡z ej-i-ŋkʰ & \armenian{սիրած էինք} \\
2PL & siɾ-il e-i-kʰ & \armenian{սիրիլ էիք} & siɾ-ɑd͡z ej-i-kʰ & \armenian{սիրած էիք} \\
3PL & siɾ-il e-i-n & \armenian{սիրիլ էին} & siɾ-ɑd͡z ej-i-n & \armenian{սիրած էին} \\
& \multicolumn{2}{l|}{$\sqrt{}$-{\perfcvb} {\aux}-{\pst}-{\agr}}& \multicolumn{2}{l|}{$\sqrt{}$-{\rptcp} {\aux}-{\pst}-{\agr}}\\ 

\hline 
\end{tabular}
\end{table}


\paragraph{Past perfective or aorist}\label{sec:Crimea:morphology:verb:paradigm:aor}

\translatorHD{The past perfective (Table \ref{tab:Crimea:morpho:verb:paradigm:pastperfectiveAorist}) is also called the aorist. In SWA for /siɾ-e-l/ `to like', the past perfective is formed by taking the root and theme vowel, adding the aorist or perfective suffix /-t͡sʰ-/, and then adding the past suffix /-i/ and the appropriate agreement suffixes. The 3SG uses covert tense and agreement suffixes. The Crimea dialect behaves almost the same; the theme vowel is /e/ for the non-3SG, but /i/ for the 3SG. }


\begin{table}[H]
 \centering
 \caption{Past perfective or aorist <\armenian{կատարեալ}> of the verb `to like' in the Crimea dialect}
 \label{tab:Crimea:morpho:verb:paradigm:pastperfectiveAorist}
 \begin{tabular}{|l|ll|ll|}
\hline & \multicolumn{2}{l|}{Crimea} & \multicolumn{2}{l|}{cf. SWA} \\
1SG & siɾ-e-t͡sʰ-i-$\emptyset$ & \armenian{սիրէցի} & siɾ-e-t͡sʰ-i-$\emptyset$ & \armenian{սիրեցի} \\
2SG & siɾ-e-t͡sʰ-i-ɾ & \armenian{սիրէցիր} & siɾ-e-t͡sʰ-i-ɾ & \armenian{սիրեցիր} \\
3SG & siɾ-i-t͡sʰ-$\emptyset$-$\emptyset$ & \armenian{սիրից} & siɾ-e-t͡sʰ-$\emptyset$-$\emptyset$ & \armenian{սիրեց} \\
1PL & siɾ-e-t͡sʰ-i-nkʰ & \armenian{սիրէցինք} &siɾ-e-t͡sʰ-i-ŋkʰ & \armenian{սիրեցինք} \\
2PL & siɾ-e-t͡sʰ-i-kʰ & \armenian{սիրէցիք} & siɾ-e-t͡sʰ-i-kʰ & \armenian{սիրեցիք} \\
3PL & siɾ-e-t͡sʰ-i-n & \armenian{սիրէցին} &siɾuz-e-t͡sʰ-i-n & \armenian{սիրեցին} \\
& \multicolumn{2}{l|}{$\sqrt{}$-{\thgloss}-{\aor}-{\pst}-{\agr}}& \multicolumn{2}{l|}{$\sqrt{}$-{\thgloss}-{\aor}-{\pst}-{\agr}}\\ 

\hline 
\end{tabular}
\end{table}
\paragraph{Imperative and prohibitive}

\translatorHD{For the imperative 2SG, SWA adds a zero morph /-$\emptyset$/ after the theme vowel /e/ for a verb like `to like' (Table \ref{tab:Crimea:morpho:verb:paradigm:Imp}). For the 2PL, SWA adds the sequence /-e-t͡sʰ-ekʰ/ after the root such that /-e-t͡sʰ/ forms the aorist stem, while /-ekʰ/ is the agreement marker. Crimea does the exact same strategy. }


\begin{table}[H]
 \centering
 \caption{Imperative forms <\armenian{հրամայական}> for the verb `to like' in the Crimea dialect}
 \label{tab:Crimea:morpho:verb:paradigm:Imp}
 \begin{tabular}{|l|ll|ll |l|}
\hline & \multicolumn{2}{l|}{Crimea} & \multicolumn{2}{l|}{cf. SWA} & \\
2SG & siɾ-e-$\emptyset$ & \armenian{սիրէ} & siɾ-e-$\emptyset$ & \armenian{սիրէ} & $\sqrt{}$-{\thgloss}-{\imp}.2{\sg}
\\
2PL& siɾ-e-t͡sʰ-ekʰ& \armenian{սիրէցէք} & siɾ-e-t͡sʰ-ekʰ& \armenian{սիրեցէք} & $\sqrt{}$-{\thgloss}-{\aor}-{\imp}.2{\pl}
\\\hline \end{tabular}
\end{table}

\translatorHD{For the prohibitive or negative imperative (Table \ref{tab:Crimea:morpho:verb:paradigm:Proh}), SWA adds the prohibitive formative /mi/ before the verb. The prohibitive marker carries stress. The verb takes a suffix /-ɾ/ in the 2SG, and /-kʰ/ in the 2PL. In Crimea, the 2SG marker is /-l/, while the 2PL marker is /-kʰ/. Note that it's possible that this 2SG marker /-l/ is actually a non-finite form; I don't know how to gloss it. } 


\begin{table}[H]
 \centering
 \caption{Negative imperative or prohibitive forms for the verb `to like' in the Crimea dialect}
 \label{tab:Crimea:morpho:verb:paradigm:Proh}
 \begin{tabular}{|l|lll|ll l|}
\hline & \multicolumn{3}{l|}{Crimea} & \multicolumn{3}{l|}{cf. SWA} \\
2SG & m\'i siɾ-i-l & \armenian{մի՛ սիրիլ} & {\proh} $\sqrt{}$-{\thgloss}-?& m\'i siɾ-e-ɾ & \armenian{մի՛ սիրեր} & {\proh} $\sqrt{}$-{\thgloss}-2{\sg} \\
2PL & m\'i siɾ-i-kʰ& \armenian{մի՛ սիրիք} & {\proh} $\sqrt{}$-{\thgloss}-2{\pl}& m\'i siɾ-e-kʰ& \armenian{մի՛ սիրէք} & {\proh} $\sqrt{}$-{\thgloss}-2{\pl} \\
\hline \end{tabular}
\end{table}



\paragraph{Non-finite forms}

\translatorHD{Finally, Adjarian lists the following non-finite forms of this verb (participles or converbs) in Table \ref{tab:Crimea:morpho:verb:paradigm:participle}. Crimea and SWA/SEA have slightly different forms. Note that Adjarian uses the term `past participle' to mean multiple different types of non-finite forms: resultative participle with /-ɑt͡s/ in SEA, and the perfective converb /-il/ in SEA. }\footnote{\translatorHD{The original page had <\armenian{սիրէլ}> as the infinitive. But on page 306, Adjarian provides an erratum that it should be <\armenian{սիրիլ}>; I fixed it.}} 

\begin{table}[H]
 \centering
 \caption{Participles or converbs <\armenian{դերբայներ}> for the verb `to like' in the Crimea dialect}
 \label{tab:Crimea:morpho:verb:paradigm:participle}
 \begin{tabular}{|ll|ll |ll| l|}
\hline & & \multicolumn{2}{l|}{Crimea } & \multicolumn{2}{l|}{cf. SEA} & \\
 Infinitive & \armenian{անորոշ} & siɾ-i-l & \armenian{սիրիլ} & siɾ-e-l & \armenian{սիրել} & $\sqrt{}$-{\thgloss}-{\infgloss} \\
 Past & \armenian{անցեալ} & siɾ-il & \armenian{սիրիլ} & siɾ-el & \armenian{սիրել} & $\sqrt{}$-{\perfcvb} \\
& & siɾ-ɑd͡z & \armenian{սիրաձ}& siɾ-ɑt͡s & \armenian{սիրած}& $\sqrt{}$-{\rptcp} \\
 Future & \armenian{ապառնի} & siɾ-e-l-u & \armenian{սիրէլու} & siɾ-e-l-u & \armenian{սիրելու} & $\sqrt{}$-{\thgloss}-{\infgloss}-{\futcvb} \\
\hline \end{tabular}
\end{table}

\begin{adjarianpage}\label{page:266}\end{adjarianpage}% should be 266


\section{Literature}

There is no study on the Crimea dialect. But there are a few select manuscripts. 


{\litoverview}
 
\begin{itemize}
 \item Literature with the Crimea dialect
 \begin{itemize}
 \item К. Паткановъ. Матеріалы для изученія арм. нарѣчій, I. Говоръ Нахичеванскій 
\item \armenian{Ռ. Պատկանեանի Ընտիր երկասիրութիւնները, Ա եւ Բ. Պետերբ}. 1893. \armenian{մանաւանդ Գ. հտ. Ռոստով}, 1904
\item \armenian{Տիգրանեան Գ}. – \armenian{Առածք, ասացուածք եւ զրոյցք Նոր-Նախիջեւանի. Ռոստով}, 1892
\end{itemize}
\end{itemize}


There are also some fables in series of Armenian folk fables by Tigran Navasardian (\armenian{Տիգրան Նաւասարդեան}; \translatorHD{SEA: /tiɡɾɑn nɑvɑsɑɾdjɑn/, /nɑvɑsɑɾtʰjɑn/, SWA: /dikʰɾɑn nɑvɑsɑɾtʰjɑn/}) in his series of Armenian folk tales, and a small number of writings in the periodicals of New Nakhichevan: \citeauthor{NorGyank-NorNakhichevan}, \citeauthor{MerTsayn-NorNakhichevan}, and \citeauthor{Luys-NorNakhichevan} (published 1906-1911). 

\section{Text samples}

{\sampleoverview}

\subsection{New Nakhichevan}
Adjarian's source: See \citet[71-73]{Patkanian-1875-RussianDialects}. This presents the old language, with a pure /ɾ/ <\armenian{ր}> sound. After checking in New Nakhichevan, I have rendered it into scientific orthography.

\armenian{Ազիաթի հայէրուն դունը մէգ խուջուռ բան է. մէզի բէս ազբար (բակ) չունին. նացա դունէրէն թէփէն դէղինի բէս ղիւս է. սիրդէրը նէզանա նը՝ գէրթան, թօռղայնէրու բէսնագ՝ դունէրուն թէփէն գը նըսդին}. .

\armenian{Նացա հացի փուռն ալ – անունը գիդէի ամա՝ մօռցիլ իմ – խուջուռ բան է. ազբարին օրթան, շադէրուն ալ դունէրուն մէչը, մէքամ գուլօր փօս քը փօրին, մէչը սըվա գանին, շդէ դացա փուռը։ Հաց էփէլու լան նը՝ իդա փօսին մէչը նօմայ չօր փադ, խօռայ (չոր խռիւ) գուլուն, դագէն գը բըռընդէցնուն, յէթգէն գառնուն խումօրը աղլի փուռին բադէրուն գը ձէթին. շդէ իմացիր դացա հացին ֆասօնը}. – \armenian{ադէթօվ լաթ, մէղայասձու մէղա։ Ադ դահա հէչ։ Փուռին հացը գը ժօղօդին նը՝ քը հանին վրայի աղդօդ շարիգնէրը վարդիգնէրը, իդա փուռին մէչը քը թօթվին, օչիլնէրը քը թափթփին մէչը, ուռէլի (ոլոռն) բէսնագ չըթըռ չըթըռ գը բօհըրվին. աս ալ դացա լըվացքն է. թի՜ւֆ}… \armenian{շադ մունդառ հալխ ին իդա ազիաթի հայէրը}…

\armenian{Քաթքայի հայը ա՛սլը ջիւնաբէթ է. մէրէրէն շադ վար է. մէրէը բարէմ մարթու սըրա մարթ ին. նաքա՝ ինա մաջառ}-... 


\begin{adjarianpage}\label{page:267}\end{adjarianpage}% should be 267

... \armenian{արաբայով Խըրըմէն նօղայնէր քուքաննը՝ ադէթօվ նաքա ին. էրէսնէրը խաչ չի հանին նը՝ հայ էղաձնէրը բէլլի յալ չէ՛}… \armenian{Փամի՛լըյ, հէդնէրը թէմիզ հայնագ գալաջի գանիմ՝ չին հասքընալ. վսէռավնօ իսա բադը. գօռէլի բէս էրէսը գը նայէ «խաբար չիմ» իմիշ. գօյա գուզէ ասէլու քի՝ ասաձըդ չիմ հասքընար}… \armenian{գընա գըդիր նարա լիւզիւյօվ ի՞նչ ասէլ է «էձը ձօրը ընգէլ ա, գէլը էգէլ ա գէրէլ ա». գօյա գուզէ ասէլու քի աձը հէնդէքը ինգիլ է, գալը իգիլ է նարան գէրիլ է}… \armenian{էրէսդ խավարի. հայնագ ասա – դա. աձին ինչո՞ւ էձ գասիս – օր, թըլվադ բէրն. փօյա՛մը աձ ասա – դա. ձօրըս վօ՞րն է, հէնդէք ասիս նը չի՞լալ. «էգէլ ա գէէրէլ ա». ֆռանցուզնա՞գ գալաջի գանիս՝ ի՞նչ է. էգիլ է գէրիլ է ասիս նը՝ անգից աղէգ չէ՞ ինչ. բարէմ թէմիզ հայնագ է խօմ}… \armenian{Շդէ սիրէս խէնթ ին}… \armenian{Նաքա ֆօղ չին ասիլ՝ խօղ գասին. հավօղ չին ասիլ՝ խաղօղ գասին. խույի չին ասիլ՝ ջըրհօր գասին. չիչագ չին ասիլ՝ ձաղիգ գասին. շա՜շխըն, հէչ չիչագը ձաղէգ գըլա՞. ձաղէգը ան է, վօր էրաձ փաղէն յա թէզաքէն գը մընա. ան՝ բաշջային մէչի էլլաձը չիչագ է. վօրի՞ն գուզիս հարցուր. թէմիզ հայնագ է}…

\armenian{Էնջամը, շդէ ասրէս փօռթիթ արիլ ին իրէնց լիւզիւն. ամրէս օր՝ ամիսնէրօվ հէդէրը բիդի գէնաս, լիւզիւդ ալայ-մալայ բիդի ձըռմըռդըիս, վօր գուդօր-մուրդօր բան գըրնաս հասքըցնէլու}…

\armenian{Էրգու շափաթ Էրէվան գէցա, ասվաձային իր օրը հէդէրը ջէնգ գանէի, թէմիզ հայնագ քը սօրվէցնէի իդա մունդառ ազիաթնէրուն}…

\armenian{Ի՛նչբէս ախըռ փաթլամիշ չիլաս. մէմը իդա ալէվալէնէրուն նայէ. մարթ չին հավնիլ. մարթու վրա քը խընդան. իլլէքի մէր Նաշչուվանցինէրուս վրան՝ դա՛յմա քը խընդան. իմիշ՝ մէնք լիւզիւնէրըս փօռթիթ արիլ ինք, մօռցիլ ինք թէմիզ հայնագը}…

\subsection{Crimea}

Adjarian's source: See Tigran Navasardian's volume 7 on Armenian folk tales (\armenian{Տ. Նաւասարդեանի Հայ ժողովրդ. հէքիաթներ, Է}.), page 70-73.\footnote{I couldn't track down a more exact bibliographic description of this series, so I couldn't provide a bibliographic entry. }

\armenian{Ատենակով ժամանակով մէկ մ՚ կար, մէկ մ՚ չի կար՝ մէկ հատ թագաւոր կար։ Ադ թագաւորը ունէր մէկ հատ տեսօք աղջիկ։ Ադ աղջկանը անխատար մարդ կուզենայ էգիլ է, ամա մէկին տուած չէ։}

\begin{adjarianpage}\label{page:268}\end{adjarianpage}% should be 268

\armenian{Մէկ օր մը թագաւորը էլած ատենը չեօլին մէջ մէկ հատ ծեր մարդ կը տեսնէ նստիլ է՝ փատ կը ճղլտէ էղիլ է։ Թագաւորը կը մօտիկնայ քովը ու կը հարցունէ}.

– \armenian{Հոս ի՞նչ կանիս։}

– \armenian{Ի՜նչ անիմ, ասից ծերը, խսմէթ կը պաժնիմ։ Թագաւորը կը հարցունէ}.

– \armenian{Խսմէթ ի՞նչես կը պաժնիս։}

– \armenian{Տեսօքը չիրքինին կուտամ, ֆխարէն զէնկինին։ Թագաւորը կը հարցունէ}.

– \armenian{Իմիս աջկանս խսմէթը վո՞վ է։}

– \armenian{Քուկդդ աղջկան խսմէթը քու տունիդ խզմէթքէր Արաբն է։}

\armenian{Թագաւորին սիրտը կելլէ. կուգայ տուն, միտք կանէ թէ՝ ի՞նչես Արաբին հեռացունէ տէյին։ Վերջը մէկ գիր կը գրէ ու կուտայ Արաբին ու կասէ. «Տա՛ր իսա Ասծուն տուր»։ Նա եալ կառնէ կելլէ կերթայ։ Արաբը էրթցած ժամանակը մէկ հատ տուն կը տեսնէ. կը մտնէ նես կը տեսնէ, ու մէկ հատ կին մարդ նստիլ է ու ադ տունին թէփէէն ալ շո՜ռ շո՜ռ օսկիներ կը թափի։ Կին մարդը Արաբին կը հարցունէ}.

– \armenian{Վո՞ւր տեղ կերթաս, կասէ։}

– \armenian{Ասծու կերթամ, կասէ Արաբը։}

– \armenian{Ճա՛նըս, կասէ կին մարդը Արաբին. ասա՛ Ասծուն, կլայ ինձի ասխատար տուած օսկին, ամեն օր արապա-արապա մարդոց կը պաժնիմ՝ կէնէ շատ է։}

– \armenian{Աղէկ, կասէ Արաբը ու կելլէ կերթայ։ Գնացած ատենը կը տեսնէ ճամբին մէջ մէկ կուր մարդ նստած կեցիլ է։ Կուրը կասէ Արաբին}.

– \armenian{Վո՞ւր տեղ կերթաս։}

– \armenian{Ասծու կերթամ, կասէ Արաբը։}

– \armenian{Ճանըս, կասէ կուրը, ասա՛ Ասծուն, մինչուանքի ե՞րբ պիտի. նստիմ թոստըղանը (պղնձէ թաս) դիմացս։ – Արաբը կելնէ կերթայ ու գնացած ատենը կը տեսնէ մէկ մարդ թէք չամուռը պաթած կեցիլ է ու կը հարցունէ Արաբին}.

– \armenian{Վո՞ւր տեղ կերթաս։}

– \armenian{Ասծու կերթամ, կասէ Արաբը։}

– \armenian{Ճա՛նըս, ա՛խպարս, կասէ ադ մարդը. ասա՛ Ասծուն՝} ... 


\begin{adjarianpage}\label{page:269}\end{adjarianpage}% should be 269

... \armenian{մինչուանքի ե՞րբ աս տեղը պիտ կենամ. արդըխ քառսուն տարի է հոս մնացիլ իմ, ի՛նչ էլնել կը կրնամ՝ ի՛նչ մէջը մտնիլ։}

– \armenian{Աղէկ, կասէ Արաբը ու կելլէ կերթայ։ Մէյ մ՚ ալ տեսնիս տաղին (անտառ) մէջը մէկ հատ ծեր մարդ ռաստ կուգայ։}

– \armenian{Վո՞ւր տեղ կերթաս, կասէ ծերը։}

– \armenian{Ասծուն կերթամ, կասէ Արաբը։}

– \armenian{Ի՞նչ պիտի անիս Ասծուն, կասէ ծերը։}

– \armenian{Թագաւորս ինձի գիր տուից, պիտ Նարա տանիմ, կ՚ասէ Արաբը։}

– \armenian{Թուղթը ինձի տուր, կասէ ծերը ու ծեռքէն կառնէ։}

\armenian{Արաբը նարա կը պատմէ ճամբան ռաստ էկած կին մարդու, կուրի ու չամուռի մէջ պաթած մարդուն ասածները։ Ծերը կասէ Արաբին. «Դարձաձ ատենդ կ՚ասիս կին մարդուն.» – Երբ որ փառք Ասծու չ՚ասես նէ՝ ան ատենը օսկին թէփէէն թափելէն կը դադրէ։ Կուրին ալ կ՚ասիս, որ նա եալ քովի կետին թող փորէ՛, մէջէն ջուր կելնէ. ջուրը առնէ աչքերը թո՛ղ լուանայ՝ ան սհաթը աչքերը կը պացուին. հապա ան մարդուն ալ կասիս, որ քառսուն տարի տա՛հա թող կենայ չամուռին մէջը։}

\armenian{Արաբը ետ կը դառնայ, կերթայ ան կին մարդու քովը ու կասէ}. – \armenian{Ասուած ասից, որ երբ փառք Ասծու չասէ նէ, ան ատենը օսկին պիտ դադրի թափելէն։ Արաբը կ՚էլնէ կերթայ կուրին քովը ու կասէ}. – \armenian{Ասուած ասից, որ քովի գետինը թո՛ղ փորէ, ջուր կելնէ, աչքերը թո՛ղ լուանայ՝ կ՚աղեկնայ։ Կուրը դարձաւ ու Արաբին ասից}. – \armenian{դուն ինքդ փորէ՛։ Արաբը քիչ տեղ փորեց՝ էլած ջուրէն ձեռքերը ճերմակ էղան. քիչ մ՚ ալ փորից՝ ալա՛յ-մալա՛յ ճեպ-ճերմակ էղաւ, թէք մէկ հատ կօտիին տեղը սեւ մնաց։ Անկից Արաբը շիտակ թագաւորին կերթայ։}

\armenian{Թագաւորը Արաբին հիչ չի ճանչնայ, ամա նարա խիստ կը հաւնի, իրեն աղջկանը հետ կը պսակէ, քառսուն օր, քառսուն գիշեր հարսինք կ՚անէ։ – Ես ալ հոն էի. գինի խմեցի, պուիւէս վազեցաւ, պերանս չը գնաց։}


\chapter{Austro-Hungary }

\section{Background}


\begin{adjarianpage}\label{page:270}\end{adjarianpage}% should be 270

Big and small Armenian settlements are scattered across the many corners of Poland, Bukovina, Transylvania and Hungary; if they haven't forgotten the Armenian language, they speak a dialect which we thought it would be appropriate to call the Austria-Hungarian dialect as a general name. The Armenologist Hanusz (\armenian{Հանուշ}) has studied the Polish-Armenian vernacular in his two works (\textit{Sur la langue des Arméniens polonais, I. Mots recuillis à Kuti, Cracowi 1886}, and \textit{Beitrage zur Armenischen Dialectologie}).\footnote{\translatorHD{Based on Adjarian's prose, I have had difficulty finding the exact citations that Adjarian meant.}} I myself studied the dialect of Suceava in the \citeauthor{Bazmaveb} periodical (1899, page 112, 218, 326, 516, and 557), which is unfortunately half done. 

Because Suceava represents the most Armenian-speaking settlement of Austrian arguments, we must thus give a description of this dialect. 


\section{Phonology}
\subsection{Segment inventory}
\subsubsection{Monophthong vowels}

The Suceava dialect has the following vowels: /ɑ, e, ə, i, o, u/ <\armenian{ա, է, ը, ի, օ, ու}>.
\subsubsection{Diphthongal vowels}\label{sec:Crimea:phono:segment:dipth}

There are many diphthongs. While all the Armenian dialects have generally lost the diphthongs of Classical Armenian, in contrast the Suceava dialect has renewed them (Table \ref{tab:AustroHungary:phono:seg:vowel:diph}).

\begin{table}[H]
 \caption{Diphthongs in the Suceava dialect (Austro-Hungary)}
 \label{tab:AustroHungary:phono:seg:vowel:diph}
 \centering
 \begin{tabular}{|lll|}
 \hline Adjarian's transcription& Adjarian's explanation & IPA approximation\\
 \hline <\armenian{աւ}> & read as <\armenian{ա՛ու}> /\'ɑu/ & /ɑu̯/ \\
 <\armenian{իւ}> & read as <\armenian{ի՛ու}> /\'iu/ & /iu̯/ \\ 
 <\armenian{օւ}> & read as <\armenian{օ՛ու}> /\'ou/ & /ou̯/ \\
 <\armenian{ե}> & read as <\armenian{իէ՛}> /i\'e/ & /i̯e/ \\
 <\armenian{իեւ}> & read as <\armenian{իյէ՛ու}> /ij\'eu/ & /i̯eu̯/ \\
 <\armenian{իը}> & read as <\armenian{ի՛յը}> /\'ijə/ & /iə̯/ \\ \hline
 \end{tabular}

\end{table}

\translatorHD{Note that for the digraph <\armenian{իւ}>, Adjarian treated this as /iu̯/ for Classical Armenian, and as /ʏ/ for the previous dialect sections. But for the Austro-Hungary dialect, he treats <\armenian{իւ}> as /iu̯/. }


Among these, the <\armenian{աւ}> and <\armenian{իւ}> represent the Old Armenian <\armenian{աւ, իւ}> diphthongs: /ɑu̯, iu̯/ (Table \ref{tab:AustroHungary:phono:vowel:dipth:auiu}). 

 

\begin{table}[H]
 \centering
 \caption{Emergence of /ɑu̯/ <\armenian{աւ}> and /iu̯/ <\armenian{իւ}> in the Austro-Hungary dialect}
 \label{tab:AustroHungary:phono:vowel:dipth:auiu}
 \begin{tabular}{|l| ll|ll| ll|}
 \hline & \multicolumn{2}{l|}{Classical Armenian} &\multicolumn{2}{l|}{> Austro-Hungary} & \multicolumn{2}{l|}{cf. SEA} \\ 
 `pain' & t͡sʰɑu̯ & \armenian{ցաւ} & t͡sɑu̯ & \armenian{ցաւ} & t͡sʰɑv & \armenian{ցավ} \\ 
`honor' & pɑtiu̯& \armenian{պատիւ} & bɑdiu̯ & \armenian{բադիւ} & pɑtiv & \armenian{պատիվ} \\ 
\hline 
 \end{tabular}
\end{table}


The symbol <\armenian{ե}> represents the vowel <\armenian{ե}> /i̯e/, such as in the dialects of Mush and Van. But here the system is incomplete because the diphthong <\armenian{ո}> (read <\armenian{ուօ՛}> /u\'o/) (\translatorHD{meaning /u̯o/}) is missing. 

The Suceava sound <\armenian{իը}> /iə̯/ (Table \ref{tab:AustroHungary:phono:vowel:dipth:iə}), which originates from Classical Armenian /i/ <\armenian{ի}>, is close to the German sound <ie>, compare German <Bier>. 



\begin{table}[H]
 \centering
 \caption{Emergence of /iə̯/ <\armenian{իը}> in the Austro-Hungary dialect}
 \label{tab:AustroHungary:phono:vowel:dipth:iə}
 \begin{tabular}{|l| ll|ll| ll|}
 \hline & \multicolumn{2}{l|}{Classical Armenian} &\multicolumn{2}{l|}{> Austro-Hungary} & \multicolumn{2}{l|}{cf. SEA} \\ 
`heart' & siɾt & \armenian{սիրտ} & s\'iə̯ɾd & \armenian{սի՛ըրդ} & siɾt & \armenian{սիրտ} \\ 
\hline 
 \end{tabular}
\end{table}

\subsubsection{Consonants}
The consonants have three degrees: voiced, ... 

\begin{adjarianpage}\label{page:271}\end{adjarianpage}% should be 271

voiced aspirate, and voiceless aspirated. The Old Armenian voiced consonants have become voiced aspirates, the voiceless unaspirates became voiced, while the voiceless aspirated stay voiceless aspirated. 

\section{Morphology}
\subsection{Noun inflection or declension}

The plural marker is /-i̯eɾ, -ni̯eɾ/ <\armenian{եր, ներ}>, but there is also the formative /-sdɑn/ <\armenian{սդան}>, such as in the Karin dialect. The accusative always takes the preposition /z/ <\armenian{զ}>.\footnote{\translatorHD{Based on data from this dialect's pronouns, it seems that this preposition can vary between /z/ and /s/.}} The instrumental formative is /-ou̯/ <\armenian{օւ}> instead of the form /-ov/ <\armenian{ով}>. 


\subsection{Numerals}

The ordinal numerals are formed like in Nor Nakhichevan (Table \ref{tab:AustroHungary:morpho:numeral:suffix}). 

\begin{table}[H]
 	\centering 
 	\caption{Ordinal numerals in the Austro-Hungary dialect}
 	\label{tab:AustroHungary:morpho:numeral:suffix}
 	\begin{tabular}{|l | ll|ll| ll|}
 		\hline & \multicolumn{2}{l|}{Classical Armenian} &\multicolumn{2}{l|}{> Austro-Hungary } & \multicolumn{2}{l|}{cf. SEA} \\ 
 \armenian{՝}two' & eɾku & \armenian{երկու} & & & jeɾku & \armenian{երկու} \\
 \armenian{՝}second' & eɾk-ɾoɾd & \armenian{երկրորդ} & eɾɡus-um& \armenian{էրգուսում} & jeɾk-ɾoɾtʰ & \armenian{երկրորդ} \\
 three' &eɾekʰ & \armenian{երեք} & & &jeɾekʰ & \armenian{երեք} \\
 \armenian{՝}third' & eɾ-ɾoɾd & \armenian{երրորդ} & iɾekʰ-um & \armenian{իրէքում}& jeɾ-ɾoɾtʰ & \armenian{երրորդ} \\
\hline 
 	\end{tabular}
\end{table}

\subsection{Pronoun inflection or declension}


For the pronouns, we note the following.

\translatorHD{Table \ref{tab:AustroHungary:morpho:pron:not3} has personal pronouns that are not the third person. }

\begin{table}[H]
\centering
\caption{Declension paradigm of personal pronouns (not third person) in the Austro-Hungary dialect}
\label{tab:AustroHungary:morpho:pron:not3}
\begin{tabular}{|l|llll|}
\hline & 1SG & 2SG & 1PL & 2Pl \\
& `I' & `you' & `we' & `you' \\
\hline 
{\nom} & jes & dʰun & minkʰ & dʰukʰ \\
 & \armenian{յէս} & \armenian{դՙուն} & \armenian{մինք} & \armenian{դՙուք} \\\hline
{\gen} & zim & zkʰu & mi̯eɾ & d͡zʰi̯eɾ \\
 & \armenian{զիմ} & \armenian{զքու} & \armenian{մեր} & \armenian{ձՙեր} \\\hline
{\dat} & ind͡zi & kʰezi & mezi & d͡zezi \\
 & \armenian{ինձի} & \armenian{քէզի} & \armenian{մէզի} & \armenian{ձՙէզի} \\\hline
{\acc} & zis & skʰi̯es & smi̯ez & sd͡zʰi̯ez \\
 & \armenian{զիս} & \armenian{սքեզ} & \armenian{սմեզ} & \armenian{սձՙեզ} \\\hline
{\abl} & zim-me & zkʰu-me & meɾ-me & d͡zʰez-me \\
 & \armenian{զիմմէ} & \armenian{զքումէ} & \armenian{մէրմէ} & \armenian{ձՙէզմէ} \\ \hline
{\ins} & ind͡zi hed & kʰezi hed & mi̯eɾ hed & d͡zʰi̯ez hed \\
 & \armenian{ինձի հէդ} & \armenian{քէզի հէդ} & \armenian{մեր հէդ} & \armenian{ձՙեզ հէդ} \\\hline 
\end{tabular}
\end{table}


\translatorHD{Table \ref{tab:AustroHungary:morpho:pron:3} has personal pronouns that are for the logophoric third person. }

\begin{table}[H]
\centering
\caption{Declension paradigm of personal pronouns (third person logophoric) in the Austro-Hungary dialect}
\label{tab:AustroHungary:morpho:pron:3}
\begin{tabular}{|l|ll|ll|}
\hline & \multicolumn{2}{l|}{3SG `he'} & \multicolumn{2}{l|}{3PL `they'} \\\hline 
{\nom} & inkʰə & \armenian{ինքը} & iɾonkʰ & \armenian{իրօնք} \\
{\gen}-{\dat} & iɾi̯en & \armenian{իրեն} & iɾi̯ent͡sʰ & \armenian{իրենց} \\
{\acc} & zinkʰə & \armenian{զինքը} & ziɾonkʰ & \armenian{զիրօնք} \\
{\abl} & iɾenme & \armenian{իրէնմէ} & iɾent͡sʰme & \armenian{իրէնցմէ} \\
{\ins} & iɾi̯en hed & \armenian{իրեն հէդ} & iɾent͡sʰmou̯ & \armenian{իրէնցմօւ} \\ \hline
\end{tabular}
\end{table}

\translatorHD{Table \ref{tab:AustroHungary:morpho:pron:dem} has demonstrative medial pronouns `that'. }

\begin{table}[H]
\centering
\caption{Declension paradigm of demonstrative medial pronouns `that' in the Austro-Hungary dialect}
\label{tab:AustroHungary:morpho:pron:dem}
\begin{tabular}{|l|ll|ll|}
\hline & \multicolumn{2}{l|}{Singular `that'} & \multicolumn{2}{l|}{Plural `those'} \\\hline 
{\nom} & dʰɑ, ɑd & \armenian{դՙա, ադ} & ɑdonkʰ, dʰɑkʰɑ & \armenian{ադօնք, դՙաքա} \\
{\gen}-{\dat} & dʰɑɾɑ & \armenian{դՙարա} & dʰɑt͡sʰɑ & \armenian{դՙացա} \\
{\acc} & dʰɑɾɑ & \armenian{դՙարա} & dʰɑt͡sʰɑ & \armenian{դՙացա} \\
{\abl} & dʰɑɾɑ-me & \armenian{դՙարամէ} & dʰɑt͡sʰɑ-me & \armenian{դՙացամէ} \\
{\ins} & dʰɑɾɑ hed & \armenian{դՙարա հէդ} & dʰɑt͡sʰɑ hed & \armenian{դՙացա հէդ} 
\\ \hline 
\end{tabular}
\end{table}

\subsection{Verb inflection or conjugation}

{\paradigmExplanation}



\subsubsection{Indicative present and past imperfective}



\translatorHD{For the indicative present, SWA combines the indicative prefix /ɡ(ə)/ <\armenian{կը}> with a finite verb. This finite verb is the subjunctive form. For an E-Class verb like `to like' /siɾ-e-l/, the theme vowel is a constant /e/, and the 3SG marker is covert. In Austo-Hungary, Adjarian states that the indicative prefix is /ɡi/; the thmee vowel is /e/ for the 3SG, and /i/ elsewhere (Table \ref{tab:AustroHungary:morpho:verb:paradigm:presentPastIndc}). }

In conjugation, the present and imperfective formative is /ɡi/ <\armenian{գի}>. The vowel /e/ <\armenian{ե}> of verbal endings becomes /i/ <\armenian{ի}>. 


\begin{table}[H]
 	\centering 
 	\caption{Indicative present <\armenian{ներկայ}> of the verb `to like' in the Austro-Hungary dialect}
 	\label{tab:AustroHungary:morpho:verb:paradigm:presentPastIndc}
 	 \begin{tabular}{|l| ll| ll|}
 		\hline & \multicolumn{2}{l|}{Austro-Hungary } & \multicolumn{2}{l|}{cf. SWA} \\ \hline 
1SG & ɡi siɾ-i-m & \armenian{գի սիրիմ} & ɡə siɾ-e-m & \armenian{կը սիրեմ} \\
2SG & ɡi siɾ-i-s &\armenian{գի սիրիս} & ɡə siɾ-e-s & \armenian{կը սիրես} \\
3SG & ɡi siɾ-e-$\emptyset$ & \armenian{գի սիրէ} & ɡə siɾ-e-$\emptyset$ & \armenian{կը սիրէ} \\
1PL & ɡi siɾ-i-nkʰ & \armenian{գի սիրինք} & ɡə siɾ-e-ŋkʰ & \armenian{կը սիրենք} \\
2PL & ɡi siɾ-i-kʰ & \armenian{գի սիրիք} & ɡə siɾ-e-kʰ & \armenian{կը սիրէք} \\
3PL & ɡi siɾ-i-n& \armenian{գի սիրին} & ɡə siɾ-e-n & \armenian{կը սիրեն} \\
 		& \multicolumn{2}{l|}{{\ind} $\sqrt{}$-{\thgloss}-{\agr}} & \multicolumn{2}{l|}{{\ind} $\sqrt{}$-{\thgloss}-{\agr}} \\ 
 		\hline 
 		 
 	\end{tabular}
\end{table}



\translatorHD{For the indicative past imperfective, SWA combines the indicative prefix with a finite verb (the past imperfective). This finite form includes adds the past suffix /-i/ after the theme vowel, such as the past 2PL sequence /-i-kʰ/ (Table \ref{tab:AustroHungary:morpho:verb:paradigm:PastIndc}). This past suffix is however covert in the 3SG, along with a covert agreement suffix. In Austro-Hungary, we use essentially the same strategy. However, the 2SG suffix is /-s/ instead of /-ɾ/. Note that the theme vowel here is /e/ in the past, instead of /i/ as in the present (Table \ref{tab:AustroHungary:morpho:verb:paradigm:presentPastIndc}). }

The imperfective 2SG uses /-s/ <\armenian{ս}> in analogy to the present. This thing does not exist in any dialect. 


\begin{table}[H]
 \centering
 \caption{Indicative past imperfective <\armenian{անկատար}> of the verb `to like' in the Austro-Hungary dialect}
\label{tab:AustroHungary:morpho:verb:paradigm:PastIndc}
 \begin{tabular}{|l|ll|ll|}
\hline 
& \multicolumn{2}{l|}{Austro-Hungary} & \multicolumn{2}{l|}{cf. SWA} \\
1SG & ɡi siɾ-e-i-$\emptyset$ & \armenian{գի սիրէի} & ɡə siɾ-ej-i-$\emptyset$ & \armenian{կը սիրէի} \\
2SG & ɡi siɾ-e-i-s & \armenian{գի սիրէիս} & ɡə siɾ-ej-i-ɾ & \armenian{կը սիրէիր} \\
3SG &ɡi siɾ-e-$\emptyset$-ɾ & \armenian{գի սիրէր} & ɡə siɾ-e-$\emptyset$-ɾ & \armenian{կը սիրէր} \\
1PL & ɡi siɾ-e-i-nkʰ & \armenian{գի սիրէինք} & ɡə siɾ-ej-i-ŋkʰ & \armenian{կը սիրէինք} \\
2PL & ɡi siɾ-e-i-kʰ & \armenian{գի սիրէիք} & ɡə siɾ-ej-i-kʰ & \armenian{կը սիրէիք} \\
3PL & ɡi siɾ-e-i-n & \armenian{գի սիրէին} & ɡə siɾ-ej-i-n & \armenian{կը սիրէին} \\
& \multicolumn{2}{l|}{{\ind} $\sqrt{}$-{\thgloss}-{\pst}-{\agr} }& \multicolumn{2}{l|}{{\ind} $\sqrt{}$-{\thgloss}-{\pst}-{\agr} } \\
\hline 
\end{tabular}
\end{table}

 There are no progressive forms.

\subsubsection{Future marking}
\translatorHD{In SWA (Table \ref{tab:AustroHungary:morpho:verb:paradigm:fut}), the future is formed by adding the proclitic /bidi/ <\armenian{պիտի}> before the finite present-form of the verb. For Austro-Hungary, the form of the proclitic varies, as Adjarian describes. }

 The future is formed with the formative /bidoɾ/ <\armenian{բիդօր}>. But Hungarian Armenians use the form /bi/ <\armenian{բի}>, which is the shortening of the CA /piti/ <\armenian{պիտի}> `it is necessary', and it becomes /b/ <\armenian{բ}> next to vowels. 

 \begin{adjarianpage}\label{page:272}\end{adjarianpage}% should be 272


\begin{table}[H]
 	\centering 
 	\caption{Future <\armenian{ապառնի}> of the verb `to like' in the Austro-Hungary dialect}
 	\label{tab:AustroHungary:morpho:verb:paradigm:fut}
 	 \begin{tabular}{|l| ll| ll|ll|}
 		\hline & \multicolumn{4}{c|}{Austro-Hungary } & \multicolumn{2}{l|}{cf. SWA} \\ 
 		 & \multicolumn{2}{l|}{Suceava} & \multicolumn{2}{l|}{Hungary } & & \\ \hline 
1SG & bidoɾ siɾ-i-m & \armenian{բիդօր սիրիմ} & bi siɾ-i-m & \armenian{բի սիրիմ} & bidi siɾ-e-m & \armenian{պիտի սիրեմ} \\
2SG & bidoɾ siɾ-i-s &\armenian{բիդօր սիրիս} & bi siɾ-i-s &\armenian{բի սիրիս} & bidi siɾ-e-s & \armenian{պիտի սիրես} \\
3SG & bidoɾ siɾ-e-$\emptyset$ & \armenian{բիդօր սիրէ} & bi siɾ-e-$\emptyset$ & \armenian{բի սիրէ} & bidi siɾ-e-$\emptyset$ & \armenian{պիտի սիրէ} \\
1PL & bidoɾ siɾ-i-nkʰ & \armenian{բիդօր սիրինք} & bi siɾ-i-nkʰ & \armenian{բի սիրինք} & bidi siɾ-e-ŋkʰ & \armenian{պիտի սիրենք} \\
2PL & bidoɾ siɾ-i-kʰ & \armenian{բիդօր սիրիք} & bi siɾ-i-kʰ & \armenian{բի սիրիք} & bidi siɾ-e-kʰ & \armenian{պիտի սիրէք} \\
3PL & bidoɾ siɾ-i-n& \armenian{բիդօր սիրին} & bi siɾ-i-n& \armenian{բի սիրին} & bidi siɾ-e-n & \armenian{պիտի սիրեն} \\
 		& \multicolumn{2}{l|}{{\fut} $\sqrt{}$-{\thgloss}-{\agr}} & \multicolumn{2}{l|}{{\fut} $\sqrt{}$-{\thgloss}-{\agr}} & \multicolumn{2}{l|}{{\fut} $\sqrt{}$-{\thgloss}-{\agr}}\\ 
 		\hline 
 		 
 	\end{tabular}
\end{table}

\subsubsection{Replacing the past perfective with the present perfect}

\translatorHD{In SWA, the past perfective is marked in a synthetic manner by using the aorist stem. For example, for the verb `to like' /siɾ-e-l/, to express the past form `they liked', we use a synthetic form (\ref{sent:AustroHungary:morph:verb:past:swa:pstperfive}). Morphologically, we add the aorist suffix /t͡sʰ/ after the theme vowel, and then add the past and agreement suffixes. In contrast, a complex tense like the present perfect or past perfect is formed periphrastically by combining a non-finite form (such as the resultative participle) with a tensed auxiliary (\ref{sent:AustroHungary:morph:verb:past:swa:presperf}, \ref{sent:AustroHungary:morph:verb:past:swa:pstperf}). Such non-finite forms are often called past participles in the more traditional literature. }

\begin{exe}
 \ex SWA \label{sent:AustroHungary:morph:verb:past:swa}
 \begin{xlist}
 \ex Past perfective
 \gll siɾ-e-t͡sʰ-i-$\emptyset$, siɾ-e-t͡sʰ-i-n \\
 like-{\thgloss}-{\aor}-{\pst}-1{\sg}, like-{\thgloss}-{\aor}-{\pst}-3{\pl} \\
 \trans `I liked; they liked.' \label{sent:AustroHungary:morph:verb:past:swa:pstperfive}\\
 \armenian{սիրեցի, սիրեցին}
 \ex Present perfect 
 \gll siɾ-ɑd͡z e-m, siɾ-ɑd͡z e-n \\
 like-{\rptcp} {\aux}-1{\sg}, like-{\rptcp} {\aux}-3{\pl} \\
 \trans `I have liked; they have liked.' \label{sent:AustroHungary:morph:verb:past:swa:presperf}\\
 \armenian{սիրած եմ, սիրած են}
 \ex Past perfect 
 \gll siɾ-ɑd͡z ej-i-$\emptyset$, siɾ-ɑd͡z ej-i-n \\
 like-{\rptcp} {\aux}-{\pst}-1{\sg}, like-{\rptcp} {\aux}-{\pst}-3{\pl} \\
 \trans `I had liked; they had liked.' \label{sent:AustroHungary:morph:verb:past:swa:pstperf}\\
 \armenian{սիրած էի, սիրած էին}
 \end{xlist}
\end{exe}

\translatorHD{As Adjarian explains below, the Austro-Hungary dialect is innovative because it has lost the synthetic strategy to mark the past perfective. Instead, to capture the meaning of the past perfective, he reports that the Austro-Hungary dialect instead uses the cognate of the periphrastic present perfect from SWA. He is vague though as to how the meaning of the present perfect is marked, or the semantic role of the cognate of the SWA past perfect. }


The past participle is formed with the formative /-il/ <\armenian{իլ}>, which with are also formed the present perfect (\armenian{յարակատար}) and past perfect (\armenian{գերակատար}) forms. \translatorHD{Note that this /il/ forms seems a cognate of the SEA perfective converb /-el/, and I gloss it as such. }

But here, the Suceava dialect has a very interesting innovation. As is clear, many of the new European languages are losing the perfective in verbal tenses. For example, French forms <j'aimai, tu aimas, il aima, nous aimâmes, vois aimâtes, il aimèren> exist only in the literary language, while the the populace do not recognize such forms and instead use the present perfect (j'ai aimé, tu as aimé). In this way, thus the original meaning of the present perfect is lost, and it has moved to the place of the perfective. The same has happened in the Suceava dialect. This dialect has abandoned the use of the perfective tense (SWA /siɾ-e-t͡sʰ-i-$\emptyset$ `I liked' <\armenian{սիրեցի}>), and it uses the present perfect in its place, with the same meaning. Here are the conjugation of the two forms (Table \ref{tab:AustroHungary:morpho:verb:pastperf}). 

\begin{table}[H]
 \centering 
 \caption{Using periphrastic forms to mark the meaning the past perfective meaning for the verb `to like' in the Austro-Hungary dialect}
 \label{tab:AustroHungary:morpho:verb:pastperf}
\begin{tabular}{|l| ll| ll|}
\hline 
& \multicolumn{2}{l|}{Participle plus present auxiliary}& \multicolumn{2}{l|}{Participle plus past auxiliary} \\
& \multicolumn{2}{l|}{(cognate to SWA present perfect)}& \multicolumn{2}{l|}{(cognate to SWA past perfect)} 
\\ \hline
1SG & siɾ-il i-m & \armenian{սիրիլ իմ} & siɾ-il e-i-$\emptyset$ & \armenian{սիրիլ էի} \\
2SG & siɾ-il i-s & \armenian{սիրիլ իս} & siɾ-il e-i-s & \armenian{սիրիլ էիս} \\
3SG & siɾ-il e-$\emptyset$ & \armenian{սիրիլ է} & siɾ-il e-$\emptyset$-ɾ & \armenian{սիրիլ էր} \\
1PL & siɾ-il i-nkʰ & \armenian{սիրիլ ինք} & siɾ-il e-i-nkʰ & \armenian{սիրիլ էինք} \\
2PL & siɾ-il i-kʰ & \armenian{սիրիլ իք} & siɾ-il e-i-kʰ & \armenian{սիրիլ էիք} \\
3PL & siɾ-il i-n & \armenian{սիրիլ ին} & siɾ-il e-i-n & \armenian{սիրիլ էին} 
\\
& \multicolumn{2}{l|}{$\sqrt{}$-{\perfcvb} {\aux}-{\agr}}
& \multicolumn{2}{l|}{$\sqrt{}$-{\perfcvb} {\aux}-{\pst}-{\agr}}
\\\hline 

\end{tabular}
\end{table}

\section{Literature}

{\litoverview}


In the following works, we can find manuscripts that are written with the diverse branches of the Austro-Hungary dialect. 

 

 
\begin{itemize}
 \item Literature with the Austro-Hungary dialect
 \begin{itemize}
 \item \armenian{Հ. Գր. Գովրիկեան} 
 \begin{itemize}
 \item – \armenian{Դրանսիլուանիոյ հայոց մետրապոլիսը, Վեննա}. 1896
 \item – \armenian{Հայք յԵղիսաբեթուպոլիս, Վեննա}, 1893
 \end{itemize}
\item L Patrubány – Sprachwissenschaftliche Abhandlugnen, I and II
\end{itemize}\end{itemize}



\translatorHD{On page 279, Adjarian had a brief paragraph about the Romani language of Lomavren. I moved it here because it's more relevant here:}

\begin{quote}
 Here it should be mentioned also the Romani language of Lomavren (\armenian{հայ բոշայերէն}), whose lexicon is only Romani (\armenian{բոշայերէն}), while its grammar and phonology are Armenian and it belongs to the /kə/ <\armenian{Կը}> branch. On the Romani language, there are diverse statements, and the most complete summary is the one by the Armenologist \armenian{Ֆինք}: \citealt{Finck-1907-SpracheRomani}.\footnote{\translatorHD{He includes other bibliographic data, such as:\armenian{պտմ-փիլ ճիւշ}. VIII, N 5. I don't know how accurate this information is however, because I couldn't find a clear copy of this item online.}}
\end{quote}

 
\begin{adjarianpage}\label{page:273}\end{adjarianpage}% should be 273

\section{Text samples}

{\sampleoverview}

\subsection{Suceava dialect}

Adjarian's source: I prepared this with a priest from Suceava, Ter Karapet Kaynayian (\armenian{Տէր Կարապետ Կայնայեան}; \translatorHD{SEA: /teɾ kɑɾɑpet kɑjnɑjɑn/, SWA: /deɾ ɡɑɾɑbed ɡɑjnɑjɑn/}), with the scientific orthography. 

\subsubsection{Sample}

– \armenian{Բՙարի լուս. ի՞նչբէս էք։}

– \armenian{Շընորհագալ իմ. աղէգ։}

– \armenian{Ի՚նչբէս, հանգՙչի՞լ իս աս գՙըշէր։}

– \armenian{Զօր աղէգ բառգիլ իմ. նումայ բՙուրիջները ինձի բՙօգՙօյ չին դուվի. հիմբի (հիմայ) ո՞ւրախ (ուր) բիդօր էրթաս. ի՜նչ բիդօր անիս ադէս (այդպէս) գանուխ։}

– \armenian{Բէդգՙ է էրթամ բՙօշդՙան, վօր դՙէլէգրաֆ անիմ։}

– \armenian{Ի՞նչ դՙէլէգրաֆ։}

– \armenian{Նեբՙօ՛դՙըս գՙըրիլ է ինձի ՝ թէ աս օրերուն գուզէ մեր մօդ իգՙալու. ու գուզէ համ ըզհարսը բՙէրէ վօր ասդեղ փսագվէն։ Դէրդէրը ինձի ասիլ է գՙօ չի գարռնա զիրէնք փսագէլու, իլալլօւ վօր ասքէր (ազգական) ին. հիմբի գուզիմ դՙէլէգրաֆադՙ անիլու, վօր չիքա, զուրի (իզուր) խարջ անէ մըսքինը (խեղճ}). \armenian{գափսըսնա՜մ շադ վօր աթ խըդա՛րը խարջ արիլ է։ Բՙօլօր բՙանը հադի՛ըր էր. քըրչէրը հադի՛ըր էին. մուզիգՙանթները վարցաձ էին. րամէցէքի բիլէդՙնէրը խըրգաձ էին. գարջ ասիմ ամմէն բՙանը հադի՛ըր էր։}

– \armenian{Յէս գի մըդքիմ վօր բադրիարքարանը դՙէլէգրաֆադՙ անինք. յէւ խընդՙրինք վօր բՙօզվօլիդՙ անին. յէւ ամմէն բՙանը բադմինք դՙէլէգրաֆի մէչ. թէ բՙօլորը հադի՛ըր ին. բՙօզվօլիդՙ անէ քահանայուն վօր աս անգՙամը փսագէ։}

– \armenian{Աղէգ է. անինք. ի՞նչ խըդար ժամանագի մէչ գըլա բադասխանը իգՙալու օրընձեդՙ։}

– \armenian{Ասօր հինքշափթի յէ. ինչֆա՛նի շաբՙաթ օր գըլա բադասխանը իլա ի հօս։}

\armenian{Ուրբՙաթ իրգուն գը հանդըբին իրենք ա՛լվըշ։}

– \armenian{Բՙարիգուն. է՛, ի՛նչ է խաբարը. բադասխան գՙըդնըվի՞լ իք։}

\begin{adjarianpage}\label{page:274}\end{adjarianpage}% should be 274

– \armenian{Հա՛բա (այո՛}). \armenian{աղէգ է. բադրիարքը բօզվօլիդՙ արիլ է քահանայուն վօր փսագէ. հիմբի գէրթամ դՙէլէգրաֆադՙ անիմ վօր իգՙան։}

– \armenian{Ասա մէ ինձի. ձանուցումները յէ՞փ բիդօր անէ։}

– \armenian{Ադօր համար էղիլ իմ գՙըբՙիդՙընի՛ան, յէւ խընդՙրիլ իմ խօր բՙօզվօլիդՙ անէ մէգ դարբա (անգամ) իրէքի համար. յօ խօսդացիլ է թէ բՙօզվօլիդՙ գանէ, յօ քահանայուն գՙըրօւ գիմացնու. հիամ մէգալ վաղը (միւս օրը) գիրագի է. առվադուն քահանան ձանուցումները գանէ, յէւ գեսավօր յէդեւ փսագ։}

– \armenian{Զօր աղէգ է. հիմբի նայէ նումայ վօր օզգա բՙաները հադի՛ըր իլան. խաղալու սալօնը արանժա՞դՙ է. բՙօդիալները վըգՙսուի՞դՙ ին։}

– \armenian{Հա՛բա, բՙօլոր բՙանը հադի՛ըր ին. նումայ չիյդիմ ըզվօ՞ խըրգիմ իրենց դՙիմաց վաման վօր նըգՙըժիդՙ չանին զիրօնք։}

– \armenian{Խըրգէ ըզվերի Գՙօգՙօրը (Գրիգոր}). \armenian{իլլալու վօր ինքը ջանջՙֆօրներ ունէ։}

– \armenian{Աղէգ գասիս. զինքը գը խընդՙրիմ վօր էրթա։}

\armenian{Էրդուսում օրը՝ շաբՙաթ օր ա՛լվըշ գը հանդըբին։}

— \armenian{Բՙարի լուս։}

– \armenian{Բՙարի լուս. աչֆըները լուս. էգի՞լ ին հարսնավօրաքը։}

– \armenian{Էգիլ ին աս առվադու. հիմբի գըցիլ (սկսել) է բՙանը. գՙնա հօն՝ հօս- աս բէդգՙ է, ան բէդգՙ է. յօւ բՙօլօր բՙանը զիմ գՙըլխուս վրա է. չունիմ վօչ մէգ աժուդՙօր մը։}

– \armenian{Նումայ համբՙերություն, բՙա՛րեգամ, բՙօլօր բՙանը գի դՙառնա. թօխ գամաց. քանո՞ւմ սահաթն է փսագը։}

– \armenian{Վեցին։}

\armenian{Բՙախդՙը (ամոսւին) դուն գուգՙա յէգ գնգանը հէդ ադէս գը զուրուցէ}.

– \armenian{Է՛ Ռուժիգՙ, հադդըվի՛ըր}… \armenian{հարսնիքը մո՛ւզիգՙը ուժէ գի փչէ. բէդգՙ է էրթանք։}

– \armenian{Յէս հադի՛ըր իմ. դՙուն ալ չուսդՙ (շուտ) սեւ քըրչերըդՙ հաքի՛ըր. յէւ մընուշաները հանէ շուֆլադէն. յէւ դՙիըր գՙօնջուգը վօր չի մօռնաս։}

– \armenian{Ռո՛ւժիգՙ, դՙուն վօ՞ր բօդՙինները գի հաքնիս}… \armenian{հաքի՛ըր ջէրմագ գազու (մետաքս}). \armenian{իլալու վօր ջէրմագ գազու օրօգՙլան հաքիլ իս. չի մօռնաս բըրօշը ու բրանզօլէդՙան առնուս}… \armenian{օրօգՙլիդՙ}... 



\begin{adjarianpage}\label{page:275}\end{adjarianpage}% should be 275

\armenian{շլէբՙը շադ յէրգան է. բէդգՙ էր զինքը գՙռէօյդՙօրին դաս՝ վօր գըդրէ։}

– \armenian{Դՙո՛ւն ինչ գՙիդիս. ասբէս է մօդան. խրգէ չուսդՙ ֆիագՙրին դէվանց վօր իգՙա։}

– \armenian{Քալէ՛}… \armenian{ֆիագՙրը գի բՙօհէ (սպասել)։}

\armenian{Էրգուսն ա՛ գի նըսդին ֆիագՙրի մէչ, յէւ գէրթան հարսնիքին դունը։}

– \armenian{Բՙարօւ էգիլ իք։}

– \armenian{Գը շընաֆօրինք. Ասվաձ դա խըսմըթօւ ու դօլվըթօւ իլա։}

– \armenian{Շընօրհագալ ինք. դառօսը հէմ ձՙեր զավագացը գՙօլօջՙին}… \armenian{րամէցէք. խընդՙրիմ, րամէցէք ա՛լ վեր, խաշլումօրը քօւ։}

– \armenian{Դՙէռ չի՞ն էրթա փըսագ։}

– \armenian{Ա՛ս բաս (այս պահուս}). \armenian{նումայ քահանան իգՙա}… \armenian{հա՛, քահանան էգիլ է. է՛, րամէցէք, դՙըրըսուրաները գօ բՙօհին. առաչի դՙըրըսուրային մէչ զէրթա քահանան, յէւ բիքա իրիցգՙինը. էրգուսումին մէչ հարսը խաշլու մօրը հէդ. իրէքումին մէչ փէսան խաշլուին հէդ. յէդգՙը մէգալօնք. գի խընդՙրինք վօր բՙրէջՙէ դՙըրըրսուրաներօւ էրթան, զէ (զի) գան դիսդՙուլ։}

\armenian{Գամաց գամաց սէրէ դասը գամ դասնըհինգ դՙըրըսուրառընդօւ ժամ գէրթան, յէւ փսագը գի գըցէ. փսագէն ալվըշ հարսինը դունը գէրթան յէւ հարգըվէլուն էդէվանց գի գըցին սդՙօլները փռէլու. յէւ գի դՙըրվին մուսաֆիրները սդՙօլ. գի հասգըցվի վօր հարսը յէւ փէսան սդՙօլէն ջագադը յէւ իրենց դէվանց խաշլուն ու խաշլումարը. անօր դէվանց բՙրէջՙը ռընդօւ. հիմբի գի գըցին բՙարեգենացները։}

– \armenian{Զօր բիդօր անգՙընվի (ուշանաԼ) սդՙօլը. քանի՞ յէ սահադՙը։}

– \armenian{Ուժէ ինը անցիլ է։}

– \armenian{Բՙէդգՙ է խընդՙրինք գքահանան վօր ա՛լ չուսդՙ անէ. սըլիդՙ անէ բՙարեգենացներօվը, իլալու վօր դըղաները ուժէ չունին համբՙերություն։}

– \armenian{Ունին ժամանագ համ խաղալու դիսդՙուլ, ինչֆանի առվադու։}

– \armenian{Ա՛հ, աս է յէդգՙի բՙարեգենացը. գի լըմընցվի սդՙօլը։}

– \armenian{Շնօրհագալ ինք։}

– \armenian{Խնըդՙրիմ թօղություն։}


\begin{adjarianpage}\label{page:276}\end{adjarianpage}% should be 276



– \armenian{Հիմբի ի՛նչ գանինք. գէրթանք սալօնը. մո՛ւզիգՙը գըցիլ է փչելու}… \armenian{դիսդՙուլ մեձ է սալօնը. ու ռինդ լումինադՙ է, դառը զօր դաք է. դըղաները չին իմանա. գութֆի (կը թուի) թէ չին դիսնու ու չին լսի օզգա բՙան իքմընա (իմն ինչ)։ Ասբէս ին դղաները, գի բՙըռնի՞ս միդըՙ յէփ դՙուն ա դղա էիս. յէւ գի խաղան յէփ վալց, յէփ գՙադրիլ, յէփ հօ՛րա։ Հիմբի գուզին մազուր խաղալու. վօ՞ արանժադՙ գանէ։}

– \armenian{Ինձի գութֆի թէ աղաչա Օվանէսը զօր ռինդ մազուր արանժադՙ արիլ է. դարը ալվըշ վալց գուզին խաղալու. բէդգՙ է անգՙան իլա։}

– \armenian{Սահադՙը չօրս է. յէւ դՙրուսդըս (դուստր) չուզէ էրթալու. ամմէնը գի խնդՙրէ ա՛լա (է՛լի) քիչ մը, ա՛լա քիչ մը. ու զօր դՙրուդիդՙ իմ։}

– \armenian{Ադէս գանէ համ օրդՙիս. յէփ գասիմ իրեն քալէ դուն, դիսդՙուլ է, գի խընդՙրէ նումայ զաս գՙադրիլը ա՛լա. գՙօ անգաժադՙ իմ. գՙադրիլին դէվանց օզգա վալց ու օզգա՛լը յէւ սըվըրշիդՙ չուին. հա՛նա գի լուսանա}… \armenian{ուժէ ջիրախները գի փօխին. դարը հիմբի գէրթանք}… \armenian{նայեցէ՛ք դղաք, հաքնըվէցէք աղէգ. փաթըվէ՛ Հըռէփսիա}… \armenian{Գա՛րաբեդ, հաքի՛ըր ը}X\armenian{իբըրցի՛ըրրը, վօր չի բաղիս. քըրդընաձ էս}… \armenian{հիմբի առնունք բՙարօւ մնա դան մադՙիգՙօցը մօդէ}… \armenian{հա՛նա, հօս ին}… \armenian{բՙարի գՙըղէր բիքա Վարդՙէնիք, բՙարի գՙըշէր աղաչա Գյօրգէշ, աղաչա Լուսիգ։}

– \armenian{Բՙարի գՙըշէր ձՙեր հրամանօցը։}

– \armenian{Ասվաձ դա դօլվըթօւ ու խըսմըթօւ իլա։}

– \armenian{Շնօրհագալ ինք աշխադանքին։}

\subsubsection{Words}

\begin{enumerate}
 \item	\armenian{Զօր. թրք. զօր՝ շատ} 
 \item	\armenian{Նումայ. ռում}. numai\armenian{՝ միայն թէ} 
 \item	\armenian{Բՙուրիջՙ. ռմ} purici\armenian{՝ լու} 
 \item	\armenian{Բՙօգՙօյ. լեհ}. pokoi\armenian{՝ հանգստութիւն} 
 \item	\armenian{Բՙօշդՙա. ռմ}. posta\armenian{՝ նամակատուն} 
 \item	\armenian{Դՙէլէգրաֆ. ռմ}. telegraf\armenian{՝ հեռագիր} 
 \item	\armenian{Նեբՙօդՙ. ռմ}. nepot\armenian{՝ հօրեղբօրորդի} 
 \item	\armenian{Գՙօ. ռմ}. câ\armenian{՝ թէ, որ} 
 \item	\armenian{Հադի՛ըր. թրք}. hazər\armenian{՝ պատրաստ} 
\begin{adjarianpage}\label{page:277}\end{adjarianpage}% should be 277
 	 
 \item	 \armenian{Քուրջ՝ զգեստ} 
 \item	\armenian{Մուզիգՙանթ. ռմ}. musicant\armenian{՝ նուագածու} 
 \item	\armenian{Բիլէդՙ. ռմ}. bilet\armenian{՝ տոմսակ} 
 \item	\armenian{Բՙօզվօլիդՙ. լեհ. թոյլատրել} 
 \item	\armenian{Օրընձեդՙ՝ յետս} 
 \item	\armenian{Ալվըշ. ռմ}. earasi\armenian{՝ դարձեալ} 
 \item	\armenian{Գՙըբՙիդՙընիա. ռմ}. câpitania\armenian{՝ թաղապետութիւն} 
 \item	\armenian{Արանժադՙ. ռմ}. aranjat\armenian{՝ կարգաւորեալ} 
 \item	\armenian{Բՙօդիալ. ռմ}. podeal\armenian{՝ տախտակ} 
 \item	\armenian{Վըգՙսուիդՙ. ռմ}. vacsuit\armenian{՝ մոմած} 
 \item	\armenian{Վամա. ռմ}. vama\armenian{՝ մաքսատուն} 
 \item	\armenian{Նըգՙըժիդՙ. ռմ}. necajit\armenian{՝ նեղել} 
 \item	\armenian{Վերի. ռմ}. ver \armenian{հօրեղբօրորդի} 
 \item	\armenian{Աժուդՙօր. ռմ}. ajut0r\armenian{՝ օգնական} 
 \item	\armenian{Ռուժիգՙ. լեհ}. Ruza\armenian{՝ Վարդուհի} 
 \item	\armenian{Մուզիգՙ. գերմ}. Musik\armenian{՝ նուագ} 
 \item	\armenian{Ուժէ. լեհ}.uze\armenian{՝ արդէն} 
 \item	\armenian{Մնուշա. ռմ}. manusa\armenian{՝ ձեռնոց} 
 \item	\armenian{Շուֆլադ. ռմ}. sufladâ\armenian{՝ դարան} 
 \item	\armenian{Գՙօնջուգՙ՝ գրպան} 
 \item	 \armenian{Բօդՙին. ռմ}. botin\armenian{՝ կօշիկ} 
 \item	 \armenian{Շուֆլադ. ռմ}. rochie\armenian{՝ շրջազգեստ} 
 \item	 \armenian{Բրօշ. ռմ}. brosu\armenian{՝ մանեակ} 
 \item	 \armenian{Բրանզօլէդՙա. ռմ}. branzoleta\armenian{՝ ապարանջան} 
 \item	 \armenian{Շլէբՙ. գերմ}. Schleppe\armenian{՝ քղանցք} 
 \item	\armenian{Գՙռէօյդՙօր. ռմ}. croitoriu\armenian{՝ դերձակ} 
 \item	\armenian{Ֆիագՙր. գերմ}. fiacker\armenian{՝ կառք} 
 \item	\armenian{Գՙօլօջՙ՝ գլուխ} 
 \item	\armenian{Դՙրըսուրա. ռմ}. trasura\armenian{՝ կառք} 
 \item	\armenian{Բիքա՝ տիկին (լեհ. կամ հունգ}.) 
 \item	\armenian{Խաշլու՝ կնքահայր} 
 \item	\armenian{Խաշլումար՝ կնքամայր} 
 \item	\armenian{Բՙրէջՙէ՝ բոլոր} 
 \item	\armenian{Դիսդՙուլ. ռմ}. distul\armenian{՝ բաւական} 
 \item	\armenian{Ռընդ. ռմ}. rôndu\armenian{՝ կարգ} 
 \item	\armenian{Սդՙօլ. լեհ}. stol\armenian{՝ սեղան} 

\begin{adjarianpage}\label{page:278}\end{adjarianpage}% should be 278
 	 
 \item	\armenian{Սլիդՙ. ռմ}. salit\armenian{՝ շտապել} 
 \item	\armenian{Լումինադՙ. ռմ}. luminat\armenian{՝ լուսաւոր} 
 \item	\armenian{Դարը. ռմ}. dara\armenian{՝ բայց} 
 \item	\armenian{Դՙրուդիդՙ. ռմ}. truditu\armenian{՝ յոգնած} 
 \item	\armenian{Սվըրշիդՙ. ռմ}. sfarsitu\armenian{՝ վերջացած} 
 \item	\armenian{Իրըրցի՛ըր. գերմ}. überzieher\armenian{՝ վերարկու} 
\end{enumerate}

\subsection{Gherla or Armenopolis from Hungary}

Adjarian's source: See \armenian{Գովրիկեան, Դրանսիլուանիոյ Հայոց Մետրապոլիսը, էջ} 312

\armenian{Զարկիլ է ուժեմ կենացս վերջին սըհաթը – կասէ հոգեւարք հայրը էրկու որդուն։ Իմ էտէվանց միան դուք կի մնաք։ Ինչ որ բոլոր կենացս մէջ, հարկիւորութեամբ քաղիլ իմ նա՝ ձեզի կի թողում։ Ապրեցէք միամիտ, հանդարտ ու մէկտեղ։ Չի բաժնըվիք մէկը մէկալէն, գէրամ բաժնըված կարող չի պիլաք մեծ արուտուր անէլու։ Ըզձերը մի թողուք. ըզօզկայինը մի կամենաք։ Կանուխ ելեցէք, արաջը ժամ գընացէք, պատարագ լսելու. անոր էտէվանց բացէցէք պօլդ։ Թէ որ մէկ օրը տասը կրօշ վաստըկիլ իք նա, միայն ութը խարճեցէք։ Երբ շոգոտոլ (տօնավաճառ) երթաք, տարեցէք ձեր հետ ըզբադըրը (տէր ողորմեա) ու ամարը համ պունտա (մուշտակ)։ Ճամբօվը օտար մարդիկաց հըտ մի՛ բարեկըմվիք. ցանցառ մարդ ըլլա ում դիմաց ըզսըրտերդ բանաք։ Սիրեցէք զԱստված, բըրնեցէք ըզիրեն պատվիրանքները, եղեցէք ողորմասիրտ։ Պահեցէք ըզիմ անունը ու յիշատակը. ու տէրն մէրը, ում դիմաց հիմպիկ կերթամ, պի օրհնէ զձեզ։}

\armenian{Ճորով թաղիլ ին ըզմեռածը, մեծ աղբարը մորցըվիլ է վողորմած հոգի հոյրը խօսքերուն վրայէն ու վարիլ է տանէն ըզպիզտիկը։ Չի տուվի իրեն իքմըն ալ, միայն մէկ կով մը։ Քիչ ժամանակի վրա զան ալ ետ ուզիլ է։ Չի տուվի նա, դատըստընով արիլ է։ Հիմպիկ աղքատ աղբարը կառնու ըզաշխարհը ափը. ու ճամփա կելէ։ Կի երթա շատ ու քիչ։ Կի կաննի ու մէկ ծարի մը տակ կի հանգչի։}

\armenian{Կուգա էրկու ագրաւ ու ան ծարինը կի նըստին կի կըցին զուրուցէլու։}

– \armenian{Ի՞նչ նոր կա ձեր էրկիրը։}

– \armenian{Կի մերնին մարդիքը ծարվուն։}


\begin{adjarianpage}\label{page:279}\end{adjarianpage}% should be 279


– \armenian{Եշտ (հեշտ) պիլար ատորը աջողելու, թէ որ պազարը մէկ փըս մը փըրէին. ու մէկ ձիաւոր մարդ մը զան բոլորէր նա, ջուր պի էլէր։}

– \armenian{Ու ձեր մօտ չիգա՛ իքմըն ալ նոր բան։}

– \armenian{Թագաւորը զօր հիւանդ է ու չիտէ մարդ ա զինքը լաւցընէլու։}

– \armenian{Կա հարնին (ափոռ) սէմին տակը մէկ գորտ մը, թէ որ անոր եղովը քըսէին զինքը նա, պի լաւնար։}

\armenian{Աղքատը, վով ծարին տակէն ամէն խօսք աղէկ կի լսէր, ալ ինտան կերթա։ Կի հասնի ան քաղաքը ուրուխ ջրի պաքսութիւն ունացիլ ին։ Կանէ մէկ ջիղվըր (ջրհոր) մը, որին մէջ անխա ջուր քաղվիլ է որ դուս ալ վոթիլ է։ Ընդունած շատ պաշխըշովը, կերթա մայրաքաղաքը ուր թագաւորը բնակիլ է։ Կի լաւցընու զինքը։ Անխա գանծ կուտան իրեն, որ ճորով կըլայ տուն տանէլու։}

Footnote: \armenian{Այս առակը տպուած է նաեւ} Sprachwissenschaftliche Abhand. \armenian{թերթին մէջ}, 1 \armenian{էջ} 117-8, \armenian{եւրոպական տառադարձութեամբ. վերջինը թէեւ աւելի ստոյգ է, բայց դարձեալ բոլորովին ճիշտ չէ, ուստի հայերէն օրինակը անփոփոխ պահեցի։}


\translatorHD{Adjarian had a brief paragraph here about the Romani language of Lomavren. I moved it to the literature section. }



\begin{adjarianpage}\label{page:280}\end{adjarianpage}% should be 280

\part{The /el/ <\armenian{ել}> branch}

The /el/ <\armenian{ել}> branch has 3 dialects:

\begin{enumerate}
\item Dialect of Maragha
\item Dialect of Khoy
\item Dialect of Artvin
 
 
\end{enumerate}


\chapter{Maragha}\label{chapter:Maragha}
\section{Background}

\begin{adjarianpage}\label{page:281}\end{adjarianpage}% should be 281

The dialect of Maragha is spoken on the two sides of Lake Urmia. The northern side is found in the city of Maragha, while the western side is the city of Urmia, with its group of Armenian villages, a portion of which are Turkish-speaking. For this very interesting dialect, there is no published study or even a line from a published manuscript. During my time in Persia, I studied it, with two adult students from Maragha: Petros Hayrapetian (\armenian{ՊՊ. Պետրոս Հայրապետեան}; \translatorHD{SEA: /petɾos hɑɾɑpetjɑn/}) and Grigor Mnatsakanian (\armenian{Գրիգոր Մնացականեան}; \translatorHD{SEA: /ɡ(ə)ɾikʰoɾ mənɑt͡sʰɑkɑnjɑn/}). I present here a summary of my unpublished research. 

\section{Phonology}
\subsection{Segment inventory}

The sound system of the Maragha dialect is very rich in vowels and diphthongs (in total 14)\footnote{\translatorHD{The original page had 13. But on page 306, Adjarian provides an erratum that it should be 14. He likewise says that we should include the vowel <\armenian{է}>. I fixed it. }} in Table \ref{tab:Maragha:phono:seg:vowel}.

\begin{table}[H]
 \centering
 \caption{Vowel inventory of the Maragha dialect}
 \label{tab:Maragha:phono:seg:vowel}
 \begin{tabular}{llll}
/i/ <\armenian{ի}> & /ʏ/ <\armenian{իւ}> & /ui̯/ <\armenian{ու}ⁱ> & /u/ <\armenian{ու}> \\
/œ/ <\armenian{էօ}> & /e/ <\armenian{է}> & /o/ <\armenian{օ}> & /u̯e/ <\armenian{ուէ}>
\\
/əi̯/ <\armenian{ը}ⁱ> & /ə̟/ <\armenian{ըէ}> & /ə/ <\armenian{ը}> & /ɨ/ <\armenian{ը}̂> \\
 /æ/ <\armenian{ա}̈> & & & /ɑ/ <\armenian{ա}>
 \\
 \end{tabular}
 
\end{table}

\translatorHD{For the sounds <\armenian{ու}ⁱ> and <\armenian{ը}ⁱ>, Adjarian used a superscript <\armenian{ի}> /i/: <\armenian{ու}\textsuperscript{\armenian{ի}}, \armenian{ը}\textsuperscript{\armenian{ի}}. But because that could cause problems with my type-setter, I replaced them with a superscript <i>. }


The consonants are likewise rich with some new sounds in Table. 



\begin{table}[H]
 \centering
 \caption{Consonants of the Maragha dialect}
 \label{tab:Maragha:phono:segment:cons}
 \begin{tabular}{|l|lll|llll|lll|}
 \hline 
 & \multicolumn{3}{l|}{Labial}& \multicolumn{4}{l|}{Coronal}& \multicolumn{3}{l|}{Dorsal/Back}\\
 Stops& /b/ & /p/ & /pʰ/ & /d/ & /t/ & /tʰ/& & /ɡ/ & /k/ & /kʰ/ 
 \\
 & <\armenian{բ}> &<\armenian{պ}>& <\armenian{փ}> &<\armenian{դ}>& <\armenian{տ}> &<\armenian{թ}>&& <\armenian{գ}>& <\armenian{կ}>& <\armenian{ք}>\\
 & & & & & & && /ɡʲ/ & /kʲ/ & /kʰʲ/ \\
 & & & && & && <\armenian{գյ}>& <\armenian{կյ}>& <\armenian{քյ}>\\
 \hline 
 Affricates & && & /d͡z/ & /t͡s/ & /t͡sʰ/ & && & \\
 & && &<\armenian{ձ}>& <\armenian{ծ}>& <\armenian{ց}> & & & & \\
 & && & /d͡ʒ/ & /t͡ʃ/ & / t͡ʃʰ/ && & & \\
 & & & &<\armenian{ջ}>& <\armenian{ճ}>& <\armenian{չ}> & & && \\
 \hline 
 Fricatives& /f/&/v/& &/s/& /z/& /ʃ/& /ʒ/& /χ/ & /ʁ/ & /h/ \\
 & <\armenian{ֆ}>&<\armenian{վ}>& & <\armenian{ս}>& <\armenian{զ}>& <\armenian{շ}>& <\armenian{ժ}>& <\armenian{խ}> & <\armenian{ղ}> & <\armenian{հ}> \\
 && & & & & & & & & /hʲ/ \\
 && & & & & & & & & <\armenian{հյ}>
\\ \hline 
 Sonorants & /m/ & /n/& & /ɾ/ & /r/& /l/ & /j/ &/w/& & \\
& <\armenian{մ}> & <\armenian{ն}> && <\armenian{ր}>& <\armenian{ռ}>& <\armenian{լ}>& <\armenian{յ}> &<\armenian{ւ}>& & 
\\ \hline 
 \end{tabular}
\end{table}


For these vowels, it is worth giving a separate explanation for the following. The sound <\armenian{ը}̂> represents the Russian sound <ы>, meaning a /ə/ <\armenian{ը}> that is pronounced voiceless and closed. 

The sounds <\armenian{ը}ⁱ, \armenian{ու}ⁱ, \armenian{ըէ, ուէ}> represent approximately the sounds /əi, ui, əi, ue/ in fast pronunciation. \translatorHD{Based on this description, and to maintain consistency with previous uses of <\armenian{ըէ, ուէ}>, I use a diphthong notation with < ̯> for all but <\armenian{ըէ}>: /əi̯, ui̯, ə̟, u̯e/}

The consonants, as can be seen, have three degrees: voiced, voiceless unaspirated, voiceless aspirated). The dialect recognizes also the palatalized sounds /ɡʲ, kʲ, kʰʲ, hʲ/ <\armenian{գյ կյ քյ հյ}>, and the semi-sound /w/ <\armenian{ւ}> which is pronounced like the English letter <w>. 

\subsection{Sound changes}
\subsubsection{Monophthong vowel changes}
For the vowel changes, the following are notable.

\paragraph{Classical Armenian /ɑ/ <\armenian{ա}> }
Classical Armenian /ɑ/ <\armenian{ա}> became /ɑ/ <\armenian{ա}> or /æ/ <\armenian{ա}̈>.

\paragraph{Classical Armenian /e/ <\armenian{ե}> }
Classical Armenian /e/ <\armenian{ե}> became /je/ <\armenian{յէ}> (at the beginning of monosyllabic words), /e/ <\armenian{է}> (at the beginning of polysyllabic words), while word-medially it is /e, ə̟, i/ <\armenian{է, ըէ, ի}>.





\begin{adjarianpage}\label{page:282}\end{adjarianpage}% should be 282


\paragraph{Classical Armenian /i/ <\armenian{ի}> }
Classical Armenian /i/ <\armenian{ի}> became /i, əi̯, ə/ <\armenian{ի, ը}ⁱ, \armenian{ը}> (Table \ref{tab:Maragha:phonology:soundChange:monoph:i}).\footnote{\translatorHD{The original page had <\armenian{է}\textsuperscript{\armenian{ի}}> . But on page 306, Adjarian provided an erratum that this should be <\armenian{ըի}>, which I think was itself a typo for <\armenian{ը}\textsuperscript{\armenian{ի}}> or /\armenian{ը}ⁱ/. I fixed it. }}





\begin{table}[H]
 \centering
 \caption{Change from Classical Armenian /i/ <\armenian{ի}> to /i, əi̯, ə/ <\armenian{ի, ը}ⁱ, \armenian{ը}> in the Maragha dialect}
 \label{tab:Maragha:phonology:soundChange:monoph:i}
 \begin{tabular}{|l| ll|ll| ll|}
 \hline & \multicolumn{2}{l|}{Classical Armenian} &\multicolumn{2}{l|}{> Maragha} & \multicolumn{2}{l|}{cf. SEA} \\ 
 `barley' &ɡɑɾi & \armenian{գարի} &kʲæɾə & \armenian{կյա̈րը} &ɡɑɾi & \armenian{գարի} \\
 `scholar' &dəpiɾ & \armenian{դպիր} &təpəi̯ɾ & \armenian{տըպը}ⁱ\armenian{ր} &dəpiɾ & \armenian{դպիր} \\
\hline 
 \end{tabular}
\end{table}

\paragraph{Classical Armenian /o/ <\armenian{ո}> }

Classical Armenian /o/ <\armenian{ո}> became /vəi̯/ <\armenian{վը}ⁱ> word-initially (Table \ref{tab:Maragha:phonology:soundChange:monoph:oinit}). 



\begin{table}[H]
 \centering
 \caption{Change from Classical Armenian /o/ <\armenian{ո}> to /vəi̯/ <\armenian{վը}ⁱ> in the Maragha dialect}
 \label{tab:Maragha:phonology:soundChange:monoph:oinit}
 \begin{tabular}{|l| ll|ll| ll|}
 \hline & \multicolumn{2}{l|}{Classical Armenian} &\multicolumn{2}{l|}{> Maragha} & \multicolumn{2}{l|}{cf. SEA} \\ 
`lentil' & ospən & \armenian{ոսպն} & vəi̯sp & \armenian{վը}ⁱ\armenian{սպ} & vosp & \armenian{ոսպ} \\
 \armenian{՝}son' & oɾd\'i & \armenian{որդի} & vəi̯ɾtʰ\'ə & \armenian{վը}ⁱ\armenian{րթը՛} & voɾtʰ\'i & \armenian{որդի} \\
\hline 
 \end{tabular}
\end{table}

 
 In the body of the word, it becomes /o, œ, əi̯, u̯e, ui̯/ <\armenian{օ, էօ, ը}ⁱ, \armenian{ուէ, ու}ⁱ>, according to particular circumstances (Table \ref{tab:Maragha:phonology:soundChange:monoph:omed}). 



\begin{table}[H]
 \centering
 \caption{Change from Classical Armenian /o/ <\armenian{ո}> to /o, œ, əi̯, u̯e, ui̯/ <\armenian{օ, էօ, ը}ⁱ, \armenian{ուէ, ու}ⁱ> in the Maragha dialect}
 \label{tab:Maragha:phonology:soundChange:monoph:omed}
 \begin{tabular}{|l| ll|ll| ll|}
 \hline & \multicolumn{2}{l|}{Classical Armenian} &\multicolumn{2}{l|}{> Maragha} & \multicolumn{2}{l|}{cf. SEA} \\ 
 \armenian{՝}work' & ɡoɾt͡s & \armenian{գործ}& kui̯ɾt͡s & \armenian{կու}ⁱ\armenian{րծ} & ɡoɾt͡s & \armenian{գործ} \\
 \armenian{՝}grass' & χot & \armenian{խոտ}& χui̯t & \armenian{խու}ⁱ\armenian{տ} & χot & \armenian{խոտ} \\
 `earth' &hoɬ & \armenian{հող} & χu̯eʁ & \armenian{խուէղ} & hoʁ & \armenian{հող} \\
 `soul' & hoɡ\'i & \armenian{հոգի} & χokʰ\'ə & \armenian{խօքը՛} & hokʰ\'i & \armenian{հոգի} \\
 `to roll' & ɡəloɾel & \armenian{գլորել} & kʲʏllœɾel & \armenian{կյիւլլէօրէլ} & ɡəloɾel & \armenian{գլորել} \\
 `bishop' & episkopos & \armenian{եպիսկոպոս} & jəpəskɑpəi̯s & \armenian{յըպըսկապը}ⁱ\armenian{ս} & jepiskopos & \armenian{եպիսկոպոս} \\
\hline 
 \end{tabular}
\end{table}



\paragraph{Classical Armenian /u/ <\armenian{ու}> }

Classical Armenian /u/ <\armenian{ու}> became /u, ui̯, ʏ/ <\armenian{ու, ու}ⁱ, \armenian{իւ}> (Table \ref{tab:Maragha:phonology:soundChange:monoph:u}). 
 

\begin{table}[H]
 \centering
 \caption{Change from Classical Armenian /u/ <\armenian{ու}> to /u, ui̯, ʏ/ <\armenian{ու, ու}ⁱ, \armenian{իւ}> in the Maragha dialect}
 \label{tab:Maragha:phonology:soundChange:monoph:u}
 \begin{tabular}{|l| ll|ll| ll|}
 \hline & \multicolumn{2}{l|}{Classical Armenian} &\multicolumn{2}{l|}{> Maragha} & \multicolumn{2}{l|}{cf. SEA} \\ 
`house' &tun & \armenian{տուն} & tʏn & \armenian{տիւն} & tun & \armenian{տուն} \\ 
`bundle' & χuɾd͡z & \armenian{խուրձ} & χui̯ɾt͡sʰ & \armenian{խու}ⁱ\armenian{րց} & χuɾt͡sʰ & \armenian{խուրձ} \\
\hline 
 \end{tabular}
\end{table}

\subsubsection{Diphthongal vowel changes}


\paragraph{Classical Armenian /ɑi̯/ <\armenian{այ}> }

Classical Armenian /ɑi̯/ <\armenian{այ}> became /e/ <\armenian{է}> (Table \ref{tab:Maragha:phonology:soundChange:diph:ai}). 


\begin{table}[H]
 \centering
 \caption{Change from Classical Armenian /ɑi̯/ <\armenian{այ}> to /e/ <\armenian{է}> in the Maragha dialect}
 \label{tab:Maragha:phonology:soundChange:diph:ai} 
 \begin{tabular}{|l| ll|ll| ll|}
 \hline & \multicolumn{2}{l|}{Classical Armenian} &\multicolumn{2}{l|}{> Maragha} & \multicolumn{2}{l|}{cf. SEA} \\ 
`father' & hɑi̯ɾ & \armenian{հայր} & χeɾ & \armenian{խէր} & hɑjɾ & \armenian{հայր} \\ 
`sound' & d͡zɑi̯n & \armenian{ձայն} & t͡sen & \armenian{ծէն} & d͡zɑjn & \armenian{ձայն} \\ 
\hline 
 \end{tabular}
\end{table}
\paragraph{Classical Armenian /iu̯/ <\armenian{իւ}> }

Classical Armenian /iu̯/ <\armenian{իւ}> became /ʏ, i/ <\armenian{իւ, ի}> (Table \ref{tab:Maragha:phonology:soundChange:diph:iu}). 
 
\begin{table}[H]
 \centering
 \caption{Change from Classical Armenian /iu̯/ <\armenian{իւ}> to /ʏ, i/ <\armenian{իւ, ի}> in the Maragha dialect}
 \label{tab:Maragha:phonology:soundChange:diph:iu} 
 \begin{tabular}{|l| ll|ll| ll|}
 \hline & \multicolumn{2}{l|}{Classical Armenian} &\multicolumn{2}{l|}{> Maragha} & \multicolumn{2}{l|}{cf. SEA} \\ 
`hundred' & hɑɾiu̯ɾ & \armenian{հարիւր} & χæɾir & \armenian{խա̈րիր} & hɑɾjuɾ & \armenian{հարյուր} \\
 		\armenian{՝}snow' & d͡ziu̯n & \armenian{ձիւն}& t͡sʏn & \armenian{ծիւն} & d͡zjun & \armenian{ձյուն} \\ 
\hline 
 \end{tabular}
\end{table}


\paragraph{Classical Armenian /oi̯̯/ <\armenian{ոյ}> }

Classical Armenian /oi̯̯/ <\armenian{ոյ}> became /ʏ, ui̯/ <\armenian{իւ, ու}ⁱ> (Table \ref{tab:Maragha:phonology:soundChange:diph:oi}). \translatorHD{Adjarian provides the CA word `sleep' /kʰun/ <\armenian{քուն}>, but I think this is a mistake because it doesnt' have a diphthong /oi̯/. }

 
 
\begin{table}[H]
 \centering
 \caption{Change from Classical Armenian /oi̯̯/ <\armenian{ոյ}> became /ʏ, ui̯/ <\armenian{իւ, ու}ⁱ> in the Maragha dialect}
 \label{tab:Maragha:phonology:soundChange:diph:oi} 
 \begin{tabular}{|l| ll|ll| ll|}
 \hline & \multicolumn{2}{l|}{Classical Armenian} &\multicolumn{2}{l|}{> Maragha} & \multicolumn{2}{l|}{cf. SEA} \\ 
 \armenian{՝}light' & loi̯s & \armenian{լոյս}& lui̯s & \armenian{լու}ⁱ\armenian{ս} & lujs & \armenian{լույս} \\
 \armenian{՝}sleep' & kʰun & \armenian{քուն}& kʰʲʏn & \armenian{քյիւն} & kʰun & \armenian{քուն} \\
\hline 
 \end{tabular}
\end{table}

\subsubsection{Consonant changes}
The consonant changes are exactly the say as in the dialects of Van or Karabakh. The Classical sound /h/ <\armenian{հ}> is always /χ/ <\armenian{խ}>. 

\section{Morphology}
\subsection{Noun inflection or declension}

\subsubsection{Vowel harmony}
In the grammar, everything is established based on the rule of analogy. Nominal and verbal formatives and endings change their vowels according to the vowel that's contained in the root of the word. For example, the definite article becomes /-ɑ/ <\armenian{ա}> if the vowel of the word-final syllable is /ɑ/ <\armenian{ա}> or /u/ <\armenian{ու}>. But it becomes /-æ/ <\armenian{ա}̈> if that vowel is /æ, e, ʏ/ <\armenian{ա}̈, \armenian{է, իւ}>. The genitive formative is /-ə/ <\armenian{ը}> if the vowel of the word-final syllable is /ɑ/ <\armenian{ա}> or /ə/ <\armenian{ը}>. But that formative becomes /-ʏ/ <\armenian{իւ}> if the vowel is /ʏ/ <\armenian{իւ}> or /œ/ <\armenian{էօ}>. It also becomes /-u/ <\armenian{ու}> when in front the vowels /u, o/ <\armenian{ու, օ}>, and it becomes /-i/ <\armenian{ի}> in front the vowel /i/ <\armenian{ի}>. Even the copular verb is subject to these assimilatory changes. 

\subsubsection{Plural and case marking}

The plural formative is /-iɾ/ <\armenian{իր}> for monosyllabic words, /-niɾ/ <\armenian{նիր}> for vowel-final polysyllabic words, /-kʰiɾ/ <\armenian{քիր}> for consonant-final polysyllabic words. 

In declension, there is no loss or deletion of vowels (Table \ref{tab:Maragha:phonology:morpho:vowelRed}). 


\begin{table}[H]
 \centering
 \caption{No vowel reduction in the Maragha dialect}
 \label{tab:Maragha:phonology:morpho:vowelRed}
 \begin{tabular}{|l| ll|ll| ll|}
 \hline & \multicolumn{2}{l|}{Classical Armenian} &\multicolumn{2}{l|}{> Maragha} & \multicolumn{2}{l|}{cf. SEA} \\ 
`nose' &kʰitʰ & \armenian{քիթ} & & &kʰitʰ & \armenian{քիթ} \\
`nose-{\gen}' &kʰətʰ-i & \armenian{քթի} & kʰitʰ-i & \armenian{քիթի} &kʰətʰ-i & \armenian{քթի} \\
`meat' & mis & \armenian{միս} & & &mis & \armenian{միս} \\ 
`meat-{\gen}' & məs-i & \armenian{մսի} & mis-i &\armenian{միսի} &məs-i & \armenian{մսի} \\ 
`heart' & siɾt & \armenian{սիրտ} & & &siɾt & \armenian{սիրտ} \\ 
`heart-{\gen}' & səɾt-i & \armenian{սրտի} & siɾt-i & \armenian{սիրտի} & səɾt-i & \armenian{սրտի} \\ 
\hline 
 \end{tabular}
\end{table}

This dialect has the following cases: nominative, genitive-dative, accusative, ablative, and instrumental. There is no locative; the accusative is like the /um/ <\armenian{ում}> branch. While the ablative is formed with the formative /-en/ <\armenian{էն}>. 

\subsection{Verb inflection or conjugation}

\subsubsection{Overview of changes}\label{sec:Maragha:morpho:verb:overview}

As we said above, in the /el/ <\armenian{ել}> branch, the present stem is formed based on the verb's infinitive, by combining it or conjugating it with the auxiliary verb. For example, in the Maragha dialect, one says \armenian{իւզէլի իմ} or \armenian{իւզէլիմ} instead of saying \armenian{կուզեմ, ուզում եմ} As we said above, in the \armenian{ել} branch, the present stem is formed based on the verb's infinitive, by combining it or conjugating it with the auxiliary verb (\ref{sent:Maragha:morpho:verb:infconj}). 

\begin{exe}
 \ex `I want' \label{sent:Maragha:morpho:verb:infconj} \begin{xlist}
 \ex Maragha
 \begin{xlist}
 \ex \gll ʏz-e-l-i i-m \\
 want-{\thgloss}-{\infgloss}-{\impfcvb}? {\aux}-1{\sg} \\
 \trans \armenian{իւզէլի իմ} \label{sent:Maragha:morpho:verb:infconj:base} 
 \ex \gll ʏz-e-l-i-m \\
 want-{\thgloss}-{\infgloss}-{\aux}-1{\sg} \\
 \trans \armenian{իւզէլիմ} \label{sent:Maragha:morpho:verb:infconj:merged} 
 \end{xlist}
 \ex cf. SWA
 \gll ɡ-uz-e-m \\
 {\ind}-want-{\thgloss}-1{\sg} \\
 \trans \armenian{կ՚ուզեմ}
 \ex cf. SEA
 \gll uz-um e-m \\
 want-{\impfcvb} {\aux}-1{\sg} \\
 \trans \armenian{ուզում եմ}
 \end{xlist}
 
\end{exe}


\translatorHD{Note that in \ref{sent:Maragha:morpho:verb:infconj:base}, it seems that the verb ends in some vowel /i/ and then the auxiliary is added. It's unclear what is the morphological function of this final vowel. It could be glossed as a cognate of the irregular SEA imperfective converb suffix /-is/. But in (\ref{sent:Maragha:morpho:verb:infconj:merged}), it seems that this vowel is deleted and the auxiliary is cliticized or merged onto the verb. }
\begin{adjarianpage}\label{page:283}\end{adjarianpage}% should be 283


The formative /kə/ <\armenian{կը}> is used only in the future. Every past tense is formed from the present by adding the formative /eɾ/ <\armenian{էր}>, without differentiating for person or number. For the perfective, a new form has been created. 

\subsubsection{General paradigm}

The following is the complete conjugation of the verb `to like' (derived from CA /uz-e-l/ `to want' <\armenian{ուզել}>).

{\paradigmExplanation}

\paragraph{Indicative present and past imperfective}

\translatorHD{In SEA (Table \ref{tab:Maragha:morpho:verb:paradigm:presentIndc}), the indicative present is formed by taking combining a non-finite form of the verb (called the imperfective converb with the suffix /-um/) with the present auxiliary. In Maragha, we see a similar periphrastic approach. However, the non-finite form is based on the verb's infinitive. The auxiliary seems to then be cliticized onto the verb. Note how the two dialects diverge in the form of the auxiliary: /e/ for SWA, but /e, i/ for Maragha. }

\begin{table}[H]
 \centering
 \caption{Indicative present <\armenian{ներկայ}> in the Maragha dialect}
 \label{tab:Maragha:morpho:verb:paradigm:presentIndc}
 \begin{tabular}{|l|ll|ll|}
\hline & \multicolumn{2}{l|}{Maragha `to like'} & \multicolumn{2}{l|}{cf. SEA `to want'} \\
1SG & ʏz-e-l-i-m & \armenian{իւզէլիմ} & uz-um e-m & \armenian{ուզում եմ} \\
2SG & ʏz-e-l-i-s & \armenian{իւզէլիս} & uz-um e-s & \armenian{ուզում ես} \\
3SG & ʏz-e-l-i-$\emptyset$ & \armenian{իւզէլի} & uz-um e-$\emptyset$ & \armenian{ուզում է} \\
1PL & ʏz-e-l-i-nkʰʲ & \armenian{իւզէլինքյ} & uz-um e-ŋkʰ & \armenian{ուզում ենք} \\
2PL & ʏz-e-l-e-kʰʲ & \armenian{իւզէլէքյ} & uz-um e-kʰ & \armenian{ուզում եք} \\
3PL & ʏz-e-l-i-n & \armenian{իւզէլին} & uz-um e-n & \armenian{ուզում են} 
\\
& \multicolumn{2}{l|}{$\sqrt{}$-{\thgloss}-{\infgloss}-{\aux}-{\agr}}& \multicolumn{2}{l|}{$\sqrt{}$-{\impfcvb} {\aux}-{\agr}}\\\hline 
\end{tabular}
\end{table}

\translatorHD{For SEA, the indicative past imperfective uses the same imperfective converb as in the present (Table \ref{tab:Maragha:morpho:verb:paradigm:pastImpfIndc}). The difference is that auxiliary is now in the past tense. But in Maragha, we use a simpler strategy: the past-marking particle /eɾ/ is added after the present form. Note that this particle seems cliticized in the 3SG. }




\begin{table}[H]
 \centering
 \caption{Indicative pasti imperfective <\armenian{անկատար}> in the Maragha dialect}
 \label{tab:Maragha:morpho:verb:paradigm:pastImpfIndc}
 \begin{tabular}{|l|ll|ll|}
\hline & \multicolumn{2}{l|}{Maragha `to like'} & \multicolumn{2}{l|}{cf. SEA `to want'} \\
1SG & ʏz-e-l-i-m eɾ & \armenian{իւզէլիմ էր} & uz-um ej-i-$\emptyset$ & \armenian{ուզում էի} \\
2SG & ʏz-e-l-i-s eɾ & \armenian{իւզէլիս էր} & uz-um ej-i-ɾ & \armenian{ուզում էիր} \\
3SG & ʏz-e-l-$\emptyset$-$\emptyset$-eɾ & \armenian{իւզէլէր} & uz-um e-$\emptyset$-ɾ & \armenian{ուզում էր} \\
1PL & ʏz-e-l-i-nkʰʲ eɾ & \armenian{իւզէլինքյ էր} & uz-um ej-i-ŋkʰ & \armenian{ուզում էինք} \\
2PL & ʏz-e-l-e-kʰʲ eɾ & \armenian{իւզէլէքյ էր} & uz-um ej-i-kʰ & \armenian{ուզում էիք} \\
3PL & ʏz-e-l-i-n eɾ & \armenian{իւզէլին էր} & uz-um ej-i-n & \armenian{ուզում էին} 
\\
& \multicolumn{2}{l|}{$\sqrt{}$-{\thgloss}-{\infgloss}-{\aux}-{\agr} {\pst}}& \multicolumn{2}{l|}{$\sqrt{}$-{\impfcvb} {\aux}-{\pst}-{\agr}}\\\hline 
\end{tabular}
\end{table}

\paragraph{Present perfect and past perfect}

\translatorHD{The present perfect (Table \ref{tab:Maragha:morpho:verb:paradigm:presentPerfect}) and past perfect (Table \ref{tab:Maragha:morpho:verb:paradigm:pastPerfect}) in SEA are formed with periphrasis. The verb is in the form of the perfective converb with the suffix /-el/. The present tense auxiliary is added for the present perfect, while the past auxiliary for the past perfect.}

\translatorHD{Maragha likewise uses periphrasis but with two differences. First in Table \ref{tab:Maragha:morpho:verb:paradigm:presentPerfect}, the non-finite form can use either the suffix /-iɾ/ (cognate with the SEA perfective converb suffix /-el/), or the suffix /-ɑt͡s/ (cognate with the SEA resultative participle suffix /-ɑt͡s/). When the suffix /-ɑt͡s/ is used, the 3SG auxiliary is /ə/ instead of /i/. }

\begin{table}[H]
 \centering
 \caption{Present perfect <\armenian{յարակատար}> in the Maragha dialect}
 \label{tab:Maragha:morpho:verb:paradigm:presentPerfect}
 \begin{tabular}{|l|ll|ll|ll|}
\hline & \multicolumn{4}{l|}{Maragha `to like'} & \multicolumn{2}{l|}{cf. SEA `to want'} \\
& \multicolumn{2}{l|}{Form 1}& \multicolumn{2}{l|}{Form 2} && \\
1SG & ʏz-iɾ i-m & \armenian{իւզիր իմ} & ʏz-ɑt͡s i-m & \armenian{իւզած իմ} & uz-el e-m & \armenian{ուզել եմ} \\
2SG & ʏz-iɾ i-s & \armenian{իւզիր իս} & ʏz-ɑt͡s i-s & \armenian{իւզած իս} & uz-el e-s & \armenian{ուզել ես} \\
3SG & ʏz-iɾ i-$\emptyset$ & \armenian{իւզիր ի} & ʏz-ɑt͡s ə-$\emptyset$ & \armenian{իւզած ը} & uz-el e-$\emptyset$ & \armenian{ուզել է} \\
1PL & ʏz-iɾ i-nkʰʲ & \armenian{իւզիր ինքյ} & ʏz-ɑt͡s i-nkʰʲ & \armenian{իւզած ինքյ} & uz-el e-ŋkʰ & \armenian{ուզել ենք} \\
2PL & ʏz-iɾ e-kʰʲ & \armenian{իւզիր էքյ} & ʏz-ɑt͡s e-kʰʲ & \armenian{իւզած էքյ} & uz-el e-kʰ & \armenian{ուզել եք} \\
3PL & ʏz-iɾ i-n & \armenian{իւզիր ին} & ʏz-ɑt͡s i-n & \armenian{իւզած ին} & uz-el e-n & \armenian{ուզել են} 
\\
& \multicolumn{2}{l|}{$\sqrt{}$-{\perfcvb} {\aux}-{\agr}}& \multicolumn{2}{l|}{$\sqrt{}$-{\rptcp} {\aux}-{\agr}}& \multicolumn{2}{l|}{$\sqrt{}$-{\perfcvb} {\aux}-{\agr}}\\ 

\hline 
\end{tabular}
\end{table}

\translatorHD{In the past perfect, instead of using a special past auxiliary, we simply add the past particle /eɾ/ after the present auxiliary (Table \ref{tab:Maragha:morpho:verb:paradigm:pastPerfect}). Note that for the 3SG, the auxiliary is missing before the past particle /eɾ/. }


\begin{table}[H]
 \centering
 \caption{Past perfect <\armenian{գերակատար}> in the Maragha dialect}
 \label{tab:Maragha:morpho:verb:paradigm:pastPerfect}
 \begin{tabular}{|l|ll|ll|ll| }
\hline & \multicolumn{4}{l|}{Maragha `to like'} & \multicolumn{2}{l|}{cf. SEA `to want'} \\
& \multicolumn{2}{l|}{Form 1}& \multicolumn{2}{l|}{Form 2}& & \\
1SG & ʏz-iɾ i-m eɾ & \armenian{իւզիր իմ էր} & ʏz-ɑt͡s i-m eɾ & \armenian{իւզած իմ էր} & uz-el ej-i-$\emptyset$ & \armenian{ուզել էի} \\
2SG & ʏz-iɾ i-s eɾ & \armenian{իւզիր իս էր} & ʏz-ɑt͡s i-s eɾ & \armenian{իւզած իս էր} & uz-el ej-i-ɾ & \armenian{ուզել էիր} \\
3SG & ʏz-iɾ $\emptyset$-$\emptyset$ eɾ & \armenian{իւզիր էր} & ʏz-ɑt͡s $\emptyset$-$\emptyset$ eɾ & \armenian{իւզած էր} & uz-el e-$\emptyset$-ɾ & \armenian{ուզել էր} \\
1PL & ʏz-iɾ i-nkʰʲ eɾ & \armenian{իւզիր ինքյ էր} & ʏz-ɑt͡s i-nkʰʲ eɾ & \armenian{իւզած ինքյ էր} & uz-el ej-i-ŋkʰ & \armenian{ուզել էինք} \\
2PL & ʏz-iɾ e-kʰʲ eɾ & \armenian{իւզիր էքյ էր} & ʏz-ɑt͡s e-kʰ eɾʲ & \armenian{իւզած էքյ էր} & uz-el ej-i-kʰ & \armenian{ուզել էիք} \\
3PL & ʏz-iɾ i-n eɾ & \armenian{իւզիր ին էր} & ʏz-ɑt͡s i-n eɾ & \armenian{իւզած ին էր} & uz-el ej-i-n & \armenian{ուզել էին} 
\\
& \multicolumn{2}{l|}{$\sqrt{}$-{\perfcvb} {\aux}-{\agr} {\pst}}& \multicolumn{2}{l|}{$\sqrt{}$-{\rptcp} {\aux}-{\agr} {\pst}}& \multicolumn{2}{l|}{$\sqrt{}$-{\perfcvb} {\aux}-{\pst}-{\agr}}\\ 

\hline 
\end{tabular}
\end{table}

\paragraph{Complex future tense}
\translatorHD{Adjarian lists in Table \ref{tab:Maragha:morpho:verb:futComplex} a paradigm that he calls the complex future and its past form. The complex future is formed periphrastically by combining a non-finite form with the present auxiliary. Its past version (future perfect?) is formed by then adding the past particle /eɾ/. }

\translatorHD{Morphologically, the non-finite form seems to be built by adding the suffix /-u/ to the infinitive; the theme vowel becomes /o/. This non-finite form seems a cognate to the SEA future converb as in (\ref{sent:Maragha:morpho:verb:fut:sea}), and thus I gloss /-u/ as {\futcvb}. Unfortunately, Adjarian does not describe the semantic difference between his `complex future' and the simple `future'. The 3SG has a covert auxiliary }

\begin{exe}
\ex SEA \label{sent:Maragha:morpho:verb:fut:sea}
\begin{xlist}
\ex \gll uz-e-l-u e-m \\
want-{\thgloss}-{\infgloss}-{\futcvb} {\aux}-1{\sg}\\
\trans `I will want.'\\
\armenian{ուզելու եմ}
\end{xlist}
\end{exe}

\begin{table}[H]
\centering \caption{Complex future forms for the verb `to like' in the Maragha dialect }
\label{tab:Maragha:morpho:verb:futComplex}
\begin{tabular}{|l|ll|ll|}
\hline & \multicolumn{2}{l|}{Complex future <\armenian{բարդ ապառնի}> }& \multicolumn{2}{l|}{Past version <\armenian{անցեալ}> } \\ 
1SG & ʏz-o-l-u i-m & \armenian{իւզօլու իմ} & ʏz-ol-u i-m eɾ & \armenian{իւզօլու իմ էր} \\
2SG & ʏz-ol-u i-s & \armenian{իւզօլու իս} & ʏz-ol-u i-s eɾ & \armenian{իւզօլու իս էր} \\
3SG & ʏz-ol-u $\emptyset$-$\emptyset$ & \armenian{իւզօլու} & ʏz-ol-u $\emptyset$-$\emptyset$ eɾ & \armenian{իւզօլու էր} \\
1PL & ʏz-ol-u i-nkʰʲ & \armenian{իւզօլու ինքյ} & ʏz-ol-u i-nkʰʲ eɾ & \armenian{իւզօլու ինքյ էր} \\
2PL & ʏz-ol-u e-kʰʲ & \armenian{իւզօլու էքյ} & ʏz-ol-u e-kʰʲ eɾ & \armenian{իւզօլու էքյ էր} \\
3PL & ʏz-ol-u i-n & \armenian{իւզօլու ին} & ʏz-ol-u i-n eɾ & \armenian{իւզօլու ին էր} \\ 
& \multicolumn{2}{l|}{$\sqrt{}$-{\thgloss}-{\infgloss}-{\futcvb} {\aux}-{\agr}}& \multicolumn{2}{l|}{$\sqrt{}$-{\thgloss}-{\infgloss}-{\futcvb} {\aux}-{\agr} {\pst}}
\\ \hline 
\end{tabular}
\end{table}
\paragraph{Past perfective or aorist}

\translatorHD{The past perfective (Table \ref{tab:Maragha:morpho:verb:paradigm:pastperfectiveAorist}) is also called the aorist. In SEA for /uz-e-l/ `to want', the past perfective is formed by taking the root and theme vowel, adding the aorist or perfective suffix /-t͡sʰ-/, and then adding the past suffix /-i/ and the appropriate agreement suffixes. The 3SG uses covert tense and agreement suffixes. The Maragha dialect does a quite different strategy. Between the root and the agreement suffix, we see a single vowel /i/ or /u/. It seems this vowel acts as a past marker Yerevan dialect behaves the same. But the data is too limited to be sure. The two dialects seem to have incomparable morphology}


\begin{table}[H]
 \centering
 \caption{Past perfective or aorist <\armenian{կատարեալ}> in the Maragha dialect}
 \label{tab:Maragha:morpho:verb:paradigm:pastperfectiveAorist}
 \begin{tabular}{|l|ll|ll|}
\hline & \multicolumn{2}{l|}{Maragha `to like'} & \multicolumn{2}{l|}{cf. SEA `to want'} \\
1SG & ʏz-u-m & \armenian{իւզում} & uz-e-t͡sʰ-i-$\emptyset$ & \armenian{ուզեցի} \\
2SG & ʏz-i-ɾ & \armenian{իւզիր} & uz-e-t͡sʰ-i-ɾ & \armenian{ուզեցիր} \\
3SG & ʏz-i-t͡sʰ & \armenian{իւզից} & uz-e-t͡sʰ-$\emptyset$-$\emptyset$ & \armenian{ուզեց} \\
1PL & ʏz-u-nkʰ & \armenian{իւզունք} & uz-e-t͡sʰ-i-ŋkʰ & \armenian{ուզեցինք} \\
2PL &ʏz-u-kʰ & \armenian{իւզուք} & uz-e-t͡sʰ-i-kʰ & \armenian{ուզեցիք} \\
3PL &ʏz-u-n & \armenian{իւզուն} & uz-e-t͡sʰ-i-n & \armenian{ուզեցին} \\
& \multicolumn{2}{l|}{$\sqrt{}$-{\pst}?-{\agr}}& \multicolumn{2}{l|}{$\sqrt{}$-{\thgloss}-{\aor}-{\pst}-{\agr}}\\ 

\hline 
\end{tabular}
\end{table}
\paragraph{Subjunctive present and past imperfective } 

\translatorHD{In SEA, the subjunctive present (Table \ref{tab:Maragha:morpho:verb:paradigm:subjPresent}) is formed by adding agreement suffixes after the theme vowel /e/. These are the same agreement suffixes that are added onto the present auxiliary in the indicative present. For a verb like `to want', the 3SG involves changing the theme vowel /e/ to /i/ in the 3SG. The Maragha dialect is similar, but the theme vowel can vary between /ʏ, i, e/.} 


\begin{table}[H]
 \centering
 \caption{Subjunctive present <\armenian{ստորադասական ներկայ}> in the Maragha dialect}
 \label{tab:Maragha:morpho:verb:paradigm:subjPresent}
 \begin{tabular}{|l|ll|ll|}
\hline & \multicolumn{2}{l|}{Maragha `to like'} & \multicolumn{2}{l|}{cf. SEA `to want'} \\
1SG & ʏz-ʏ-m & \armenian{իւզիւմ} & uz-e-m & \armenian{ուզեմ} \\
2SG & ʏz-i-s & \armenian{իւզիս} & uz-e-s & \armenian{ուզես} \\
3SG & ʏz-ʏ-$\emptyset$ & \armenian{իւզիւ} & uz-i-$\emptyset$ & \armenian{ուզի} \\
1PL & ʏz-i-nkʰʲ & \armenian{իւզինքյ} & uz-e-ŋkʰ & \armenian{ուզենք} \\
2PL & ʏz-e-kʰʲ & \armenian{իւզէքյ} & uz-e-kʰ & \armenian{ուզեք} \\
3PL & ʏz-i-nʲ & \armenian{իւզին} & uz-e-n & \armenian{ուզեն} 
\\
& \multicolumn{2}{l|}{$\sqrt{}$-{\thgloss}-{\agr}}& \multicolumn{2}{l|}{$\sqrt{}$-{\thgloss}-{\agr}}\\ 
\hline 
\end{tabular}
\end{table}

\translatorHD{In SEA, the subjunctive past imperfective (Table \ref{tab:Maragha:morpho:verb:paradigm:subjPast}) is formed by adding the past suffix /i/ and agreement suffixes after the theme vowel. In Maragha, we instead add the past particle /eɾ/ after the verb. For the 3SG, this particle seems to cliticize to the verb and delete the verb's theme vowel. }



\begin{table}[H]
 \centering
 \caption{Subjunctive past <\armenian{ստորադասական անցեալ}> in the Maragha dialect}
 \label{tab:Maragha:morpho:verb:paradigm:subjPast}
 \begin{tabular}{|l|ll|ll|}
\hline & \multicolumn{2}{l|}{Maragha `to like'} & \multicolumn{2}{l|}{cf. SEA `to want'} \\
1SG & ʏz-ʏ-m eɾ & \armenian{իւզիւմ էր} & uz-ej-i-$\emptyset$ & \armenian{ուզեի} \\
2SG & ʏz-i-s eɾ & \armenian{իւզիս էր} & uz-ej-i-ɾ & \armenian{ուզեիր} \\
3SG & ʏz-$\emptyset$-$\emptyset$-eɾ & \armenian{իւզէր} & uz-e-$\emptyset$-ɾ & \armenian{ուզեր} \\
1PL & ʏz-i-nkʰʲ eɾ & \armenian{իւզինքյ էր} & uz-ej-i-ŋkʰ & \armenian{ուզեինք} \\
2PL & ʏz-e-kʰʲ eɾ & \armenian{իւզէքյ էր} & uz-ej-i-kʰ & \armenian{ուզեիք} \\
3PL & ʏz-i-nʲ eɾ & \armenian{իւզին էր} & uz-ej-i-n & \armenian{ուզեին} 
 \\
& \multicolumn{2}{l|}{$\sqrt{}$-{\thgloss}-{\agr} {\pst}}& \multicolumn{2}{l|}{$\sqrt{}$-{\thgloss}-{\pst}-{\agr}}\\ 

\hline 
\end{tabular}
\end{table}


 
\paragraph{Tenses built from the subjunctive: Future }
 
 
 \translatorHD{In Maragha, the future and future perfect are built off subjunctive (Table \ref{tab:Maragha:morpho:verb:paradigm:complexSubjunctive}). For the verb `to like', we simply add the future prefix /k-/. SEA behaves essentially the same and I don't provide its paradigm. }
 

\begin{table}[H]
 \centering
 \caption{Forms that are built from the subjunctive forms for the verb `to like' in the Maragha dialect}
 \label{tab:Maragha:morpho:verb:paradigm:complexSubjunctive}
 \begin{tabular}{|l|ll|ll|}
\hline & 
\multicolumn{2}{l|}{Future <\armenian{ապառնի}>} & \multicolumn{2}{l|}{Future perfect <\armenian{անցեալ ապառնի}> } \\
1SG & k-ʏz-ʏ-m & \armenian{կիւզիւմ} & k-ʏz-ʏ-m eɾ & \armenian{կիւզիւմ էր} \\
2SG & k-ʏz-i-s & \armenian{կիւզիս} & k-ʏz-i-s eɾ & \armenian{կիւզիս էր} \\
3SG & k-ʏz-ʏ-$\emptyset$ & \armenian{կիւզիւ} & k-ʏz-$\emptyset$-$\emptyset$-eɾ & \armenian{կիւզէր} \\
1PL & k-ʏz-i-nkʰʲ & \armenian{կիւզինքյ} & k-ʏz-i-nkʰʲ eɾ & \armenian{կիւզինքյ էր} \\
2PL & k-ʏz-e-kʰʲ & \armenian{կիւզէքյ} & k-ʏz-e-kʰʲ eɾ & \armenian{կիւզէքյ էր} \\
3PL & k-ʏz-i-nʲ & \armenian{կիւզին} & k-ʏz-i-nʲ eɾ & \armenian{կիւզին էր} \\
& \multicolumn{2}{l|}{{\fut} $\sqrt{}$-{\thgloss}-{\agr}}& \multicolumn{2}{l|}{{\fut} $\sqrt{}$-{\thgloss}-{\agr} {\pst}}
\\ \hline\end{tabular}
\end{table} 

\paragraph{Imperative and prohibitive}

\translatorHD{For the imperative 2SG, SEA adds the morph /-iɾ/ after the root for a verb like `to like' (Table \ref{tab:Maragha:morpho:verb:paradigm:Imp}). For the 2PL, archaic SEA adds the sequence /-e-t͡sʰ-ekʰ/ after the root such that /-e-t͡sʰ/ forms the aorist stem, while /-ekʰ/ is the agreement marker. More modern registers of SEA instead just add the sequence /-ekʰ/ directly after the root. Maragha is somewhat different. In the 2SG, we only see a suffix /-ʏ/ after the root. For the 2PL, we only see a suffix /-ekʰʲ/}


\begin{table}[H]
 \centering
 \caption{Imperative forms <\armenian{հրամայական}> in the Maragha dialect}
 \label{tab:Maragha:morpho:verb:paradigm:Imp}
 \begin{tabular}{|l|ll|ll|l|}
\hline & \multicolumn{2}{l|}{Maragha `to like'} & \multicolumn{2}{l|}{cf. SEA `to want'} & \\
2SG & ʏz-\'ʏ-$\emptyset$ & \armenian{իւզի՛ւ} & uz-$\emptyset$-\'iɾ & \armenian{ուզի՛ր} & $\sqrt{}$-{\thgloss}-{\imp}.2{\sg}
\\
2PL& & & uz-e-t͡sʰ-ekʰ& \armenian{ուզեցեք} & $\sqrt{}$-{\thgloss}-{\aor}-{\imp}.2{\pl}
\\
& ʏz-ekʰʲ&\armenian{իւզէքյ} & uz-ekʰ&\armenian{ուզեք}& $\sqrt{}$-{\imp}.2{\pl}
 
\\\hline \end{tabular}
\end{table}

\translatorHD{For the prohibitive or negative imperative (Table \ref{tab:Maragha:morpho:verb:paradigm:Proh}), SEA simply adds the prohibitive formative /mi/ before the imperative form. Maragha behaves the same } 


\begin{table}[H]
 \centering
 \caption{Negative imperative or prohibitive forms in the Maragha dialect}
 \label{tab:Maragha:morpho:verb:paradigm:Proh}
 \begin{tabular}{|l|ll|ll|l|}
\hline & \multicolumn{2}{l|}{Maragha `to like'} & \multicolumn{2}{l|}{cf. SEA `to want'} & \\
2SG & mi ʏz-ʏ-$\emptyset$ & \armenian{մի իւզիւ} & m\'i uz-$\emptyset$-iɾ & \armenian{մի՛ ուզիր} & {\proh} $\sqrt{}$-{\imp}.2{\sg} \\
2PL & mi ʏz-ekʰʲ& \armenian{մի իւզէքյ} & mi uz-ekʰ& \armenian{մի՛ ուզեք} & {\proh} $\sqrt{}$-{\imp}.2{\pl} \\
\hline \end{tabular}
\end{table}


\paragraph{Non-finite forms}

 
\translatorHD{On the original page, Adjarian didn't list any participles or non-finite forms. But on page 306, Adjarian provides an erratum with the following non-finite forms (Table \ref{tab:Maragha:morpho:verb:paradigm:participle}). }


\begin{table}[H]
 	\centering 
 	\caption{Participles or converbs <\armenian{դերբայներ}> in the Maragha dialect}
 	\label{tab:Maragha:morpho:verb:paradigm:participle}
 	\begin{tabular}{|ll|ll|ll|l|}
 		\hline && \multicolumn{2}{l|}{Maragha `to like'} & \multicolumn{2}{l|}{cf. SEA `to want'} & \\ 
 		Infinitive & \armenian{անորոշ} & ʏz-i-l & \armenian{իւզիլ} & uz-e-l & \armenian{ուզել} & $\sqrt{}$-{\thgloss}-{\infgloss} \\ 
 		Past & \armenian{անցեալ} & ʏz-ɑt͡s & \armenian{իւզած} & uz-ɑt͡s & \armenian{ուզած} & $\sqrt{}$-{\rptcp} \\ 
 		& &ʏz-iɾ & \armenian{իւզիր} & uz-el & \armenian{ուզել} & $\sqrt{}$-{\perfcvb} \\ 
 		Future & \armenian{ապառնի} & ʏz-o-l-u & \armenian{իւզօլու} & uz-e-l-u & \armenian{ուզելու} & $\sqrt{}$-{\thgloss}-{\infgloss}-{\futcvb} \\ 
 		\hline \end{tabular}
\end{table}






\begin{adjarianpage}\label{page:284}\end{adjarianpage}% should be 284

\section{Subdialects}
\subsection{Urmia}
The subdialect of Urmia is the same as Maragh. But from the subsequent text samples, it seems that there are some differences. 

\subsubsection{Morphological differencees}

For example, the plural formatives are /-eɾ, -neɾ/ <\armenian{էր, նէր}>, while in Maragha they are /-iɾ, -niɾ/ <\armenian{իր, նիր}>. 

The future is formed with the formative /tikʲi/ <\armenian{տիկյի}>, which is of course a form change from CA /piti/ <\armenian{պիտի}> `it is necessary'. 
\subsubsection{Object clitics}

The use of the possessive article in verbs is very interesting (\ref{sent:Maragha:subdialect:urmia:poss}).

\begin{exe}
 \ex Urmia (Maragha)
 \label{sent:Maragha:subdialect:urmia:poss}
 \begin{xlist}
 \ex \gll me t͡si prn-e-nkʰ-t \\
 one horse catch-{\thgloss}-1{\pl}-{\possSsg} \\
 \trans `(Let us) catch a horse for you.' \\
 \armenian{մէ ծի պռնէնքտ} 
 \ex \gll pʰtrt-e-s eɾ-d\\
 search-{\thgloss}-{\impfcvb}? {\pst}-{\possSsg} \\
 \trans `He was looking for you.' \\
 \armenian{փտռտես էրդ} 
 \ex \gll ɑrɑk-n ɑs-e-l-i, n\'ɑ ʃɑt kʰɑχt͡sʰɾ jel, ut-e-n-d, nɑ ʃɑt tʰɑr jel, tʰɑl-e-n-d \\
 proverb-{\defgloss} say-{\thgloss}-{\infgloss}-?, no? very sweet be?, eat-{\thgloss}-3{\pl}-{\possSsg}, no? very bitter be?, throw-{\thgloss}-3{\pl}-{\possSsg}\\
 \trans `The proverb says: Don't be too sweet, they'll eat you; don't be too bitter, they'll throw you away.'\\
 \armenian{առակն ասէլի} -- \armenian{նա՛ շատ քախցր յէլ՝ ուտէնդ, նա շատ թառ յէլ՝ թալէնդ}
 
 \end{xlist}
\end{exe}

This usage of the possessive article is borrowed from Persian, where one says for example <didem-et> `I saw you>, <binem-et> `I see you>. \translatorHD{He means that the Armenian possessive article here is acting as an object clitic. See \citet{Dolatian-prep-Definite} for similar data from other Iranian Armenian dialects. }

\section{Text samples}

{\sampleoverview}

\subsection{Maragha: \armenian{Խառնիս ինա̈ն նիշան}}



\armenian{Մէ օր Սօնան ինա̈ն Անդո՛ւնու կիւզիւն իւրիւս (իւրիւնց) տղային փսակին։ Սօնան կասը՛ Անդունին}.

– \armenian{Յար, էլչի օղօրկիհյ Հա̈րթիւնիւ ախչկան իւզօլու։}

– \armenian{Չէ՛, Սօ՛նա, մէ էօզգա̈՛նա̈ խիյալ ա̈րա}̈. \armenian{հա̈լբա̈թ նա̈րա̈ չուտուրուն. էն հարուս, յիս ախկատ։}

– \armenian{Չէ՛, Ա՛նդուն, իշքան խիյալ անէլիմ՛ նա̈րմէն աղէ՛կյա̈ չիմնա̈լի գյիննէլ. էսս (հէնց) յիս առէլիմ «Էթահյ նա̈րա̈ իւզիհյ. յա կըտան, յա չին տա̈»։}


\begin{adjarianpage}\label{page:285}\end{adjarianpage}% should be 285

– \armenian{Դէ մըկա քյի էտէնց ի, լաճիքյիրիս (մեր տղան) էլ իւզէլի, շապպաթ օ՛րա էլչի օղօրկու, թօղ էթա̈ն իւզիւն։}

\armenian{Շա̈պպա̈թ օ՛րա Սօնան շուտօվ կի զարթնի, սիմավա՛րա քի քիցի, չայի՛րա քը խըմին, ա՛ննա՛ն (յետոյ) Սօնան կըլը՛ կէթա̈ իւր բաջու տո՛ւնա, Միրվարիյին կասը՛ քյի՝}

– \armenian{Այ Միրվարը՛, ա՛խչի. յէրէկյ իրիկաս (ամուսնոյս) խէտա̈ մէ զա՛դ իհյ խիյալ արի. իւզէլիհյ Հա̈րթիւնիւ ախչկան առնիհյ միր Միսակին. կիւզիս յա̈ր կօսօրէն սօրա քյէլ էլչի, տըսնինքյ իշ կասին. կտան ա̈յա̈ր, կյա̈լ շ ա̈պպա̈թ էթա̈նքյ նիշա՛նա տինիհյ։}

– \armenian{Սօ՛նա, մըկա քյի էտէնց ի, յիս էլ շատ կուրախանամ քյի Միսակըին փսակէլէհյ։ Աշկիս վիրա̈ն, Սօ՛նա, կօսօրէն սօրա կէթամ քըչարչըրրվիմ, բա̈՛լքյա̈ առնիմ։}

\armenian{Կօսօրէն սօրա Միրվարին կըլը՛ կէթա̈ Հա̈րթիւնիւս տո՛ւնա, նա̈ր ախչկան իւզօլու։}

\armenian{Կէթա տո՛ւռա քըթըփը՛, կիկյա̈ն տո՛ւռա կըպա̈ացին, Միրվարին կըմըննը՛ նիս, Հա̈րթիւնիւ կնգան պա̈րօվ կըտա՛ Հա̈րթիւնիւ կնիկյ Նա̈րկիզն էլա̈ նա̈ր պա̈րօվա կառնը՛}

– \armenian{Փա̈հ, պա̈րօվ իս էկյի, Միրվարը՛ բաջի, էթ վա՞ր քամին ի քյէզի պէրի տա}̈. \armenian{աղէկյ ի, հա̈րտա̈ն մէ կյա̈լիս միր տո՛ւնա։}

– \armenian{Չէ՛, Նա̈՛րգյիզ բաջը՛, մկա էլ չի՛մ էր կյա}̈, \armenian{ամմա մէ խէյր պա̈նը՝ խամա յիմ էկյի։}

– \armenian{Ասա տսնիհյ, ի՞շ խէյր պա̈նի խամա՛ իս էկյի։}

– \armenian{Նա̈՛րգյիզ բաջի, աղէյ, թօղ ասիմ. տի՛ս, դիւզ ա̈ էկյիր իմ ծիր Նուբառին էլչի, կտաս՝ տու, չիս տա՝ մի՛ տու։}

– \armenian{Միրվարը՛ բաջը՛, յիս չիմ ասէլի չիմ տա, ամմա, իրիկյիս տո՛ւնա չի. թօ քիշի՛րա իրի՛կյա կյա}̈, \armenian{նա̈րմէն խա̈բա̈ր առնիմ, տսնիմ ի՞շ կասը՛։}

\armenian{Միրվարին կասը՛ Նա̈րգյիզին}.

– \armenian{Ամմա խայիշտ իմ ա̈նէլի քյի ասիս. բա̈լքա̈ կյա̈լ շա̈պպա̈թ նշա՛նա տինիհյ, բիյօլ (մի կերպ) սօրա-յէլ խառնի՛սա̈ անիհյ։}

– \armenian{Արխէին յիլ, Մի՛րվարը բաջը, յիս կասիմ։}

\armenian{Միրվարին յէլավ էկավ տուն։}

\armenian{Քիշիրվան Նարգյիզի մա՛րթա էկավ տուն. Նարգյի՛զա̈ ասաց իւր մարթուն}.

– \armenian{Միրվարին էկիր էր միր ախճկան էլչի. ի՞նչօխ իս ա̈նէլի. կըտաս ա̈յա̈ր, վա̈՛զա̈ կյա̈լուց ջուղաբ տամ։}


\begin{adjarianpage}\label{page:286}\end{adjarianpage}% should be 286

\armenian{Մա՛րթա ասաց}.

– \armenian{Ասված շիւնա̈խավիր ա̈նի, Միսա՛կա խէլքյօվ տղա՛-յը. կտամ. վա՛ղա Միլվարին կյա̈լուց ասա կտահյ։}

\armenian{Նա̈րվաղա Միլվարին էկավ Նա̈րգյիզի կը՛շտա, ասաց – Տալէ՞հյ։}

\armenian{Նա̈րգյիզն էլ ասաց}.

– \armenian{Կտահյ, հէ՞ր չիհյ տա։ Մա՛րթըս տիւն էթա̈լէն սօրա էկավ, ասը՛մ. էն էլ ասաց կտամ։}

\armenian{Կյիրա̈կյի օ՛րա Սօնան, Միլվարի ինա̈ն Անդո՛ւնա կինա̈ցը՛ն շիրինիյ խմօլու. շիրինի՛յա խմէլէն իրիքյ շա̈պպա̈թ սօրա հազըրվան խառնիսի թա̈դա̈րիհյ տըսնօլու։ Խառնիսի թա̈դա̈րի՛քյա̈ տսնէլէն սօրա, բաշլամըշը՛ն խառնի՛ստ. ըմմըին մարթնիրին կանչը՛ն, խառնիսի խա̈բա̈ր տուվը՛ն։ ա̈ռա̈չին քյիշի՛րա̈ խինա̈ տիրը՛ն, սօրավան քյիշի՛րն էլա̈ փսա՛կա կըռը՛ն։ Փսա՛կա կռօլուց խա̈լա̈թ ին էր քիցէլի խառսու կուլօ՛խա. բույօլում (յետոյ) ասէլին էր «Ասվաս շիւնա̈խավիր ա̈նի»։}

\armenian{Խառսուն ժամտունէն խանէլէն սրա Անդուն ինա̈ն Սօնան խաղալօվ խառսուն պէրը՛ն տուն։}

\subsection{Urmia subdialect}

Adjarian's source: Communicated by Mr. Kaloust Iskenderian (\armenian{պր. Գալուստ Իսքէնդէրեան}; \translatorHD{SEA: /ɡɑlust iskʰendeɾjɑn/}), a provincial inspector of Urmia schools. 
\subsubsection{Ikiaghach village}

 
\armenian{Մէ օր գնացիմ խասամ գետի յէ՛րզա. մէ պծառ կակուղ իմ էր քէլէլի (մի քիչ հանդարտ կերթայի}). \armenian{գե՛տա էնէնց ջօշմիշիր էր՝ յէրզէրէն թալէս էր ճի՛ւրա. իշկացիմ տըսամ մէ տէրտէր՝ ուր տէրօխնին, մէ կաշա (ասորի քահանա)՝ ուր տէրօխնին, մէ մալլա էլ ՝ ուր կնի՛կա. մէյն էլ մէ ծի կար կշտէ՛րա։ Նա՛ տէրտէ՛րա էյթիբար էր անէլի տանս մօ՛տա մնալ ուր կնի՛կա, նա կաշան, նա մալլան։ Մէ ծի կա տանց մօ՛տա, վէր տիկյի (պիտի) տարմօվ ճուխտ ճուխտ ըսնին մէկյէլ իրէ՛սա։ Մըկը (հիմակ) ի՞շխօ ա̈նինք վէր կնթնէ՛րա չը մնան օտար մարթու մօ՛տա։}

\subsubsection{Isalu village}

\armenian{Մալլա Նասրադի՛նա մէ օր իշէ՛րա խառիրէր տէ՛մա ՝ էթաս էր։ Կյըննաց ըլայ կյատիւկի (ձոր) վօ՛տա. մէ մարթ տար տէ}... 

\begin{adjarianpage}\label{page:287}\end{adjarianpage}% should be 287

... \armenian{առէց, ասաց. տիւն գյինաս յէս յէփ կմէռնէմ, ասա տըսնիմ։ Տէփ (յետոյ) էն իշկաց էտ մա՛րթա մի թահար մարթ ի, յէտնար – տէփ ասաց. է՛շա կյատիւկէն ըլէլիւն տիկյի օխտ տիր ա̈ռի. խէտ օխտ տիր տռէց ՝ տէփ էն վատին տիւն կմէռնէս։ Խա՛, տէփ էշէ՛րա կշէց կյատի՛ւկա. գլօ՛խա ըլէլիւն օխտ տիր է՛շա տռէց։ Մալլան ինկյավ պարզվավ, ասաց. յէս մէռամ։ Տէփ մնաց տա՛ղա. էշէ՛րա հա̈ր մէ՛կյա մէ թէխ կյընացին. մէ կյէլ ըկավ տաղ՝ էշէրէն մէ՛կյա կյէրավ. ասաց. Մալլա Նասրադի՛նա չմէռնէր՝ մըկը կյէ՛լա է՛շա չէր ուդը՛։ Տէփ մալլան ըլավ էկավ թէխ տո՛ւնա. իւր կնկյան ասաց. յէլ քէլ տուր տրկյէցին ասա վէր մալլան մէռիր ի, տանինք խօրենք իւրա։ Ըլավ կյըննաց տուր ու տըրկյէցին կանչէց. տարան տարա (զդա) խօրին։ Ասաց. մէ էրթիս թօղ, դա̈ն բա̈ դա̈ ընձի խաց պէ թալ. մնաց էտ մա՛րթա տաղ։ Տարմէն յէտի մէ կաթըրխանա էկա̈վ, տար կյէրէզմանը կուշտէն ընսնէսէն էր. էն օրն էլ տար կյընի՛կա. մա՛սալա, իւշ էր խաց պէրի. կլէօ՛խա էն ծակէն պա̈նցրացուց (յանի իշկաս էր հա՜) իշկամ խաց պէրիզ։ Կաթըրքյէ՛րա խռնան, պէռն էլ չինի աման էր. տա̈նհը ա՜մմէն տվին կօտռտին. տէփ էն կաթըրչինէ՛րա փառտին էն տէ՛ղա, ասին մէ իշկանք՝ տանք ինչէ՞ն խռնան. իշկացին մէ կյէրէզմանը վրա̈ մէ էրթըսը պէս ծայ կա̈։ Տէփ բա̈նա̈ արին (ուզեցին) մէ դէն մէ փէտ պարզէն. տէփ փէ՛տա պարզին, ը՜՜, տէփ էն դէն ծէն տըվից մալլան կյէրէզմանը՛ մէչէն. շատ մի՛ պարզէքյ, կը կը կըպնը աշկյիս. տէփ տանք ասին. հօ՛ հօ՛, կա չկա դէտ (այդ տեղէն) էն խռնէ (խրտներ}). \armenian{դէտ կլօխա խանիր ի}… \armenian{քակին տարա, խանին կաթըրչինէ՛րա. տէփ տարա բա̈նա̈ արին թըփէլ. շատ թըփը՛ն, էնղըդը թըփը՛ն ի՜ւր}…


\chapter{Khoy }

\section{Background}

\begin{adjarianpage}\label{page:288}\end{adjarianpage}% should be 288

The dialect of Khoy has an extensive distribution. It is found not only in the provinces of Khoy, Salmast, and Maku in Iran, but also in Russia in Igdir and Nakhichevan. During the large migration of Persian-Armenians in 1828, many Armenians from Salmast came and settled in Karabakh, where they founded the villages of Kori, Alighuli, Maghanjugh, Karashen in the province of Zangezur, and in villages of Alilu, Angeghakot, Kushchi-Tazakend, Uz, Mazra, Balak, Shaghat, Ltsen, Qara Klisa and Lower Qara Klisa in the province of Sisian. 

The dialect of Khoy has still not been studied. There are writings with this dialect in \citeauthor{Eminian} volume 2 (\armenian{Բ}.), page 300-304 and volume 4 (\armenian{Դ}), page 343-350. What's more important are N. Ter Avetikian's \armenian{«Ոտանաւոր աշխատութիւններ եւ Նշանագրութիւն Պարսկաստանից գաղթած Խոյեցւոց բարբառով» (Վաղարշապատ} 1900) and \armenian{«Բանաստեղծութիւններ եւ Կիրակոսի հարսանիքը» (Վաղարշապատ} 1903).\footnote{\translatorHD{Unfortunately, I couldn't track down these two bibliographic items, and thus couldn't add them to the bibliography. Furthermore, the page quality makes it unclear if the fourth word is \armenian{Նշանադրութիւն} or \armenian{Նշանագրութիւն}}}

By examining these excepts, it seems that the dialect of Khoy occupies a middle position between the dialects Maragha and Van. Its grammatical structure is the same as in the Maragha dialect, but its phonological rules are like the Van dialect. In other words, the Khoy dialect is closer to Classical Armenian than Maragha is. 

Because we think it's unneeded to further discuss these simplified phenomena, we direct the reader to the subsequent text samples. 

\begin{adjarianpage}\label{page:289}\end{adjarianpage}% should be 289

\section{Text samples}

{\sampleoverview}

Adjarian's source: See \armenian{Ն. Տէր-Աւետիքեանի, Ոտանաւոր աշխատութիւններ եւ նշանագրութիւն} (\translatorHD{\armenian{նշանադրութիւն}?}), \armenian{էջ} 46-49. 




– \armenian{Այ մառթ, տիւ գինաս որ խետ ախչի՛գյա ճոչացավ, մառթի էթալու խասավ. տո՛ւնա չի՛ սրփի, չի՛ ավըլի, աման-չամա՛նա չի լվա, տուռվէրքյա կեխտոտ կը թօղնի. շատ էլ որ խետը իյնես, ղաստէ կզարկի կը կոտռտի, յանի ինչի՞}. – \armenian{իմանան որ տանելու խասիր ի։ Էնէնց էլ տզան. հալա մէ յէլ, քէլ մտի փայան (գոմ)՝ տե՛ս, ի՞նչ կասնաս. էն հէյվան քյալե՛րա, գյամէմքե՛րա, կովե՛րա ընչիւկ վզե՛րա թաղվիր են կվի մէ՛չա. տիւ հէնց գինաս որ Կիրակո՛սա մէզի խմա պա՞ն ի անէլի մեր տո՛ւնա ավըրիր ի. վա՛յ վայի որ ասես «ա՛յ բալամ, էտէնց չեն անի», յէտ ի դառնալի խինգ խայիր քյաշում (յիշոց) ի տալի. ասէլի «Ալլահ վարա (Աստուած տայ) զըմէն էլ խատնեն». յանի ինչ ի, իմացէք որ յէս էլ փռայվէլու խասիր էմ. կօ էտէնց, ա՛յ մառթ. մկա տիւ գինաս։}

– \armenian{Աշկըս լո՜ս. մենք փսայվանք՝ մեր կլօ՛խա յեզոտով, թող էն էլ փսայվի, բալքի մեղրոտի. էն հալա յէրէյվան քյօրփան ի. մկավուստ սաբաբ ըլենք, մէ անծոտ պուճուճակ ախչիկ էլ դար խմա ուզենք, խա՛լխա մեզի ի՞նչ կասեն. չե՞ն ասի «յանի է՞տ ինչ ղայդա էր՝ մկավուստ մեխկի տոպրա՛կա կախին էն խեղճ տղայի վզէն». յէս ղալաթ կանեմ՝ դարա սաբաբ չեմ ըլի. դար պէրնէն կալա կաթի խոտ ի իկյալի։}

(\armenian{Կիրակոսը կ՚աղաչէ մօրը)։}

– \armenian{Նանա ջան. էնը խօքութ ղուրբա՛ն նանա. տիւ իմ աղէ՛կյա ասա բաբայիս կո՛ւշտա. տավարն էլ կպախեմ, տան զըմէն պա̈նի վրան էլ սիրտ կը ցավցուցեմ. հէնց էն ղըդայի որ՝ մէ ղայիմ կպնես բաբայիս յախան, որ մէ խա՛ ասի, բօլ ի. ամա էտ էլ քեզի ասեմ որ Ղուլիենց Շահբազի ախչիկ Նիգյարէն սավայի՝ որ վիզս կռէք՝ ճոկ մառթու ախչիկ չեմ առնի հա՜։}

(\armenian{Մայրը կը համոզէ ամուսինը, որ կը պատասխանէ}).

– \armenian{Այ՛ կնիկ, չունքի որ ասէս ես, թո՛ղ քյօ խաթրն էլ խօշ ըլի. բալքի սաղ չմնացինք մեռանք. սաղ իքյան Կիրակոսին փսակենք, յէս ինան տիւ էլ դհօլ զուռնայով մէ աղէկ քէֆ անենք},

\begin{adjarianpage}\label{page:290}\end{adjarianpage}% should be 290

\armenian{գիւլաշ կպնենք. ջահնա՛մա. դարմէն յէտ ի՛նչ կըլի՝ թող ըլի. ամա տիւ է՛տ ասա, վի՞ր ախչի՛կյա ուզենք, որ համ աբուռով ըլի, համ ղայրաթով. խօրորթի ըլի, որ մեր մատէն փուշ խանի։ (Ներս կը մտնէ Կիրակոսը)։ Կի՛րակոս, ա՛յ բալամ, նանատ ասէլի որ քեզի փսակենք. մկա տիւ ի՞նչ ես ասելի. ուզե՞ս ես թէ չէ. յա վի՞ր ախչիյն ես ուզելի. մէ աղվո՛րթա ասա ըշկամ. էլ ամչընալու վախտը չի։}

– \armenian{Յես չեմ ուզելի փսայվել. նանաս ի՝ որ կպիր ի յախաս, քշեր-ցերէկ ասէլի՝ «տկի (պիտի) քեզի փսակենք». մկա տիւ գինաս, նանաս. յես էթաս եմ փայան՝ տավարին յէմ տամ. համա, նանա, էն ասածս ի հա՜, Նիգյա՛րա։}

– \armenian{Չե՞սնալի, Ղուլիենց Շահբազի ախչկա խետն ի, ընձի ասիր ի. «կուզէք էն ի, չէք ուզի՝ կլեմ կլոխ կվերցեմ՝ տնէն կէթամ. իմ ուզա՛ծա Նիգյարն ի՜, Նիգյա՛րա։}

– \armenian{Ի՞շխօ մայար Շահբազին էնէնց խասած ախչի՞կ ունի։ Մենք ռաշպար մառթ ենք, մեր տան ջահէ՛լա տկի մէ պծառ (քիչ մը) էլ ծեռով-ոտով ըլի, պանի մէչ էփած ըլի, կանոխ մեր տան պա՛նա, տաշտի բժա՛րա (քազհան) տիւս կիկյա. էնէնց ըլի որ՝ ինկերէ, տրկեցէ յէտ չմնանք։ Շատ խարսներ տսիր եմ, որ իրկըվըկէն կլոխքե՛րա տնես են պա՛ռցա, ընչանք լոս խռալով մռփես են. չէ՛ն տսէլի, ախար սափորքե՛րա տարտակ ի, ճուր տկի պերենք, ավել տկի անենք, տո՛ւնա, քիւչան զբիլի ձեռէն ըլիր ի իշխօ փողո՛ցա. տսնա՞ս ես էն Յարթենէնց խա՛րսա, մառթու դար պէ՛սա մէ խարս ըլի, թող մէ աշկն էլ կոր ըլի։}


\chapter{Artvin}
\section{Background}

\begin{adjarianpage}\label{page:291}\end{adjarianpage}% should be 291


The city and province of Artvin are found south of Batumi. This province has two smaller provinces (\armenian{գաւառակ}): Ardanuç and\armenian{ւ} Şavşat-Imerkhevi. The city of Artvin has 1200 homes with Armenian residents, of which 230 are Apostolic and the remaining are Catholic. Artvin does not have an Armenian village in its surroundings. The town (\armenian{աւան}) of Ardanuç has only 200 homes of Catholic Armenians. The following villages are in the Ardanuç province (\armenian{գաւառակ}):
\begin{itemize}
 \item Tandzut, 110 Armenian houses and 5 Catholic houses
\item Norashen, 22 Armenian houses
\end{itemize}

The Armenian villages of the Şavşat-Imerkhevi province (\armenian{գաւառակ}) are:
\begin{itemize}
 \item Satlel (65 Catholic houses, 17 Armenian houses)
\item Mamanelis (12 Catholic houses)
\item Okrobakert (160 Armenian houses)
\item Pkhigur (25 Catholic houses)

\end{itemize}

East of Ardanuç, there is Ardahan; while Olti is to the south. 

The aforementioned area has its own dialect which belongs to the /el/ <\armenian{ել}> branch, and it occupies a midpoint between the dialects of Karin, Khoy, and Tbilisi. 

There is no published study on this dialect, nor a manuscript line, thus the following lines are the result of my own research, gathered from migrants from Artvin in Batumi. 

\section{Phonology}
\subsection{Overview}
The sound system of the Artvin dialect is like the dialect of Tbilisi. It has three degrees of consonants. 
\subsection{Sound changes}
\subsubsection{Classical Armenian /ɑi̯/ <\armenian{այ}>}

The Classical diphthong /ɑi̯/ <\armenian{այ}> becomes /e/ <\armenian{է}> (Table \ref{tab:Artvin:phonology:soundChange:diph:ai}). 


\begin{table}[H]
 \centering
 \caption{Change from Classical Armenian /ɑi̯/ <\armenian{այ}> becomes /e/ <\armenian{է}> in the Artvin dialect}
 \label{tab:Artvin:phonology:soundChange:diph:ai} 
 \begin{tabular}{|l| ll|ll| ll|}
 \hline & \multicolumn{2}{l|}{Classical Armenian} &\multicolumn{2}{l|}{> Artvin} & \multicolumn{2}{l|}{cf. SEA} \\ 
proximal `this' & ɑi̯s & \armenian{այս} & es & \armenian{էս} & ɑjs & \armenian{այս} \\ 
medial `that' & ɑi̯d & \armenian{այդ} & ed & \armenian{էդ} & ɑjd & \armenian{այդ} \\ 
distal `that' & ɑi̯n & \armenian{այն} & en & \armenian{էդ} & ɑjn & \armenian{այն} \\ 
`other' & ɑi̯l & \armenian{այլ} & el & \armenian{էլ} & ɑjl & \armenian{այլ} \\ 
`goat' & ɑi̯t͡s & \armenian{այծ} & et͡s & \armenian{էծ} & ɑjt͡s & \armenian{այծ} \\ 
 `vineyard' &ɑi̯ɡi& \armenian{այգի} & eɡi & \armenian{էգի} &ɑjɡi& \armenian{այգի} \\
\hline 
 \end{tabular}
\end{table}


\subsubsection{Loss of rhotic in some words}

The Classical word /hɑmɑɾ/ <\armenian{համար}> `for' has become /hɑmɑ/ <\armenian{համա}>, like in Tblisi. 

\subsubsection{Loss of initial /v/ in `on'}

An interesting phenomenon is the loss of the sound /v/ <\armenian{վ}> from the Classical word /veɾɑi̯/ <\armenian{վերայ}> (\translatorHD{cf. SWA: /vəɾɑ/ <\armenian{վրայ}>}), which has become /ɾɑ/ <\armenian{րա}> (\ref{sent:Artvin:phono:change:vra}). 

\begin{exe}
 \ex \label{sent:Artvin:phono:change:vra}
 \begin{xlist}
 \ex \glll 
 kʰɑɾ-i ɾɑ (Artvin) \\
 kʰɑɾ-i vəɾɑ (SWA) \\
 rock-{\gen} on \\
 \trans `on the/a rock' \\
 \armenian{քարի րա, քարի վրայ}
 \ex \glll 
 d͡zi-u ɾɑ nst-ɑ-$\emptyset$ (Artvin) \\
 t͡sʰij-u vəɾɑ nəst-ɑ-$\emptyset$ (SEA) \\
horse-{\gen} on sit-{\pst}-1{\sg} \\
 \trans `I sat on the/a horse.'\\
 \armenian{ձիու րա նստա, ձիու վրայ նստայ}
 \ex \glll t͡sɑr-i ɾɑ veɾ ɑnt͡sʰ-ɑ-v (Artvin) \\
 d͡zɑɾ-i vəɾɑ veɾ ɑnt͡sʰ-ɑ-v (SWA) \\ tree-{\gen} on up pass-{\pst}-3{\sg} \\
 \trans `he climbed up on the/a tree.' \\
 \armenian{ծառի րա վէր անցավ, ծառի վրայ վեր անցաւ}
 
 
 \end{xlist}
\end{exe}

\begin{adjarianpage}\label{page:292}\end{adjarianpage}% should be 292

\subsubsection{Retention of the sound /h/ <\armenian{հ}> }
The Classical sound /h/ <\armenian{հ}> does not become /χ/ <\armenian{խ}>, unlike the dialects of Maragha and Khoy.

\section{Morphology}
\subsection{Noun inflection or declension}
In declension, the ablative formative is /-men/ <\armenian{մէն}> (Table \ref{tab:Artvin:morpho:noun:abl}). 




\begin{table}[H]
 \centering
 \caption{Ablative marking in the Artvin dialect}
 \label{tab:Artvin:morpho:noun:abl}
 \begin{tabular}{|l| ll| ll|}
 \hline &\multicolumn{2}{l|}{Artvin} & \multicolumn{2}{l|}{cf. SWA} \\ 
 `from Artvin' & ɑɾtʰvinu-men & \armenian{Արթվինումէն} & ɑɾtʰvin-e & \armenian{Արթվինէ} \\ 
 ?-{\abl} & sɑvetʰu-men & \armenian{Սավէթումէն} & & \\ 
 ?-{\abl} & hetne-men & \armenian{հէտնէմէն} & & \\ 
\hline 
 \end{tabular}
\end{table}


As we know, this is one of the characteristics of the Tbilisi dialect. 

Similarly, the plural genitive is the form /-eɾ-u/ <\armenian{էրու}> (\ref{tab:Artvin:morpho:noun:pl}). 

\begin{table}[H]
 \centering
 \caption{Plural genitive marking in the Artvin dialect}
 \label{tab:Artvin:morpho:noun:pl}
 \begin{tabular}{|l| ll| ll|ll|}
 \hline &\multicolumn{2}{l|}{Artvin} & \multicolumn{2}{l|}{cf. SWA}& \multicolumn{2}{l|}{cf. SEA} \\ 
 `tree-{\pl}-{\gen}' & t͡sɑr-eɾ-u & \armenian{ծառէրու} & d͡zɑr-eɾ-u & \armenian{ծառերու}& t͡sɑr-eɾ-it͡sʰ & \armenian{ծառերից} \\ 
 `horse-{\pl}-{\gen}' & d͡zi-eɾ-u & \armenian{ձիէրու} & t͡sʰij-eɾ-u & \armenian{ձիերու}& d͡zij-eɾ-it͡sʰ &\armenian{ձիերից} \\ 
\hline 
 \end{tabular}
\end{table}

The locative is the usual form /-um/ <\armenian{ում}> (\ref{tab:Artvin:morpho:noun:loc}). 

\begin{table}[H]
 \centering
 \caption{Locative marking in the Artvin dialect}
 \label{tab:Artvin:morpho:noun:loc}
 \begin{tabular}{|l| ll| lll|}
 \hline &\multicolumn{2}{l|}{Artvin} & \multicolumn{2}{l|}{cf. SEA} \\ 
 ?-{\locgloss} & meʃ-um & \armenian{մէշում} & & \\
 `day-{\locgloss}' & oɾ-um&\armenian{օրում} & oɾ-um&\armenian{օրում} \\
\hline 
 \end{tabular}
\end{table} 


\subsection{Verb inflection or conjugation}

\subsubsection{Periphrasis in the indicative}
Verbal conjugation differs from Tbilisi. The present formative /-um/ <\armenian{ում}> absolutely does not exist. As in the Khoy dialect, this tense is formed with the form /-elis, -eli/ <\armenian{ելիս, ելի}>.

\translatorHD{I assume the segmentation is /-e-l-i(s)/ such that the /-i(s)/ is an imperfective converb added onto an infinitive. The rationale is that SEA also has this formative /-is/ as an irregular form of the regular imperfective converb suffix /-um/. Compare SEA against Artvin in (\ref{sent:Artvin:morpho:verb:impf}). }

\begin{exe}
 \ex \label{sent:Artvin:morpho:verb:impf}
 \begin{xlist}
 \ex Artvin 
 \gll χos-e-l-is e-m \\
 speak-{\thgloss}-{\infgloss}-{\impfcvb} {\aux}-1{\sg} \\
 \trans `I speak.'\\
 \armenian{խօսէլիս էմ}
 \ex cf. SEA 
 \gll χos-um e-m \\
 speak-{\impfcvb} {\aux}-1{\sg} \\
 \trans `I speak.'\\
 \armenian{խոսում եմ}
 \ex cf. SEA 
 \gll t-ɑ-l-is e-m \\
 give-{\thgloss}-{\infgloss}-{\impfcvb} {\aux}-1{\sg} \\
 \trans `I give.'\\
 \armenian{տալիս եմ}
 \end{xlist}
\end{exe}

\translatorHD{Adjarian provides further examples in (\ref{sent:Artvin:morpho:verb:impfMore})}

\begin{exe}
 \ex Artvin\label{sent:Artvin:morpho:verb:impfMore}
 \begin{xlist}
 \ex \gll pntr-e-l-is e-$\emptyset$ \\
 speak-{\thgloss}-{\infgloss}-{\impfcvb} {\aux}-3{\sg} \\
 \trans `He searches.'\\
 \armenian{փնտռէլիս է}
 \ex \gll el-n-e-l-i e-$\emptyset$ \\
 be-{\vx}-{\thgloss}-{\infgloss}-{\impfcvb} {\aux}-3{\sg} \\
 \trans `It is/becomes.'\\
 \armenian{ըլնէլի է}
 \ex \gll t͡ʃʰ-e-m kɑ/kɑji eɾtʰ-l-i \\
 {\neg}-{\aux}-1{\sg} can go-{\infgloss}-{\impfcvb} \\
 \trans `I cannot go.'\\ 
 \armenian{չէմ կա}, or \armenian{չէմ կայի էրթլի}
 \ex \gll voɾti e-s eɾtʰ-l-i \\
 where {\aux}-2{\sg} go-{\infgloss}-{\impfcvb} \\
 \trans `Where are you going?'\\
\armenian{վօ՞րտի էս էրթլի}
 \ex \gll t͡ʃʰ-e-m kɑ/kɑji χos-e-l-i \\
 {\neg}-{\aux}-1{\sg} can speak-{\thgloss}-{\infgloss}-{\impfcvb} \\
 \trans `I cannot speak.'\\ 
 \armenian{չէմ կայի խօսէլի}
 
 \end{xlist}
\end{exe}

\subsubsection{Future marking}

\translatorHD{In SEA, one way to form the future is to add the prefix /k(ə)-/ to the finite verb (\ref{sent:Artvin:morpho:verb:fut:SEA}). }

\begin{exe}
 \ex SEA \label{sent:Artvin:morpho:verb:fut:SEA}
 \gll kə-pʰəntɾ-e-n \\
 {\fut}-search-{\thgloss}-3{\pl} \\
 \trans `They will search.' \\
 \armenian{կփնտրեն}
\end{exe}

\translatorHD{It seems that Artvin uses a different prefix form. }

The future is formed with the formative /ku/ <\armenian{կու}> (\ref{sent:Artvin:morpho:verb:fut:Artvin}).

\begin{exe}
 \ex Artvin \label{sent:Artvin:morpho:verb:fut:Artvin}
 \begin{xlist}
 \ex \gll ku ɑʃ-i-n \\
 {\fut} ?-{\thgloss}-3{\pl} \\
 \trans I don't know what this verb is, but it would mean `they will X' \\
 \armenian{կու աշին}
 \ex \gll ku t-ɑ-s ɡɑ \\
 {\fut} give-{\thgloss}-2{\sg} ?\\
 \trans `You will give.' \\
 \armenian{կու դուս գա}
 \ex \gll ku pntr-i-n \\
 {\fut} search-{\thgloss}-3{\pl} \\
 \trans `They will search.' \\
 \armenian{կու փնտռին}
 \end{xlist}
\end{exe}




\subsubsection{Theme vowel changes}

In both the perfective and the future, the Classical theme vowel /e/ <\armenian{ե}> changes to /i/ <\armenian{ի}> (\ref{sent:Artvin:morpho:verb:theme}). 

\begin{exe}
 \ex Artvin \label{sent:Artvin:morpho:verb:theme}
 \begin{xlist}
 \ex \gll ɑʃ-i-t᷂͡sʰ \\
 ?-{\thgloss}-{\aor} \\
 \trans I don't know what this verb is, but it would be `He did X'\\
 \armenian{աշից}
 \ex \gll pntr-i-t᷂͡sʰ \\
 search-{\thgloss}-{\aor} \\
 \trans `He searched.' \label{sent:Artvin:morpho:verb:theme:search}\\
 \armenian{փնտռից}
 \ex \gll ku ɑʃ-i-n \\
 {\fut} ?-{\thgloss}-3{\pl} \\
 \trans I don't know what this verb is, but it would be `They were going to X'\\
 \armenian{կու աշին}. 
\end{xlist}
\end{exe}


\translatorHD{Unfortunately, Adjarian's data is too limited to make a more meaningful description or comparison with SEA/SWA. But essentially, what Adjarian describes is that the theme vowel /e/ is replaced by /i/ in some morphological contexts. Compare `he searched' from (\ref{sent:Artvin:morpho:verb:theme:search}) against SEA (\ref{sent:Artvin:morpho:verb:theme:search:SEA}). 
}

\begin{exe}
 \ex cf. SEA \label{sent:Artvin:morpho:verb:theme:search:SEA}
 \gll pəntɾ-e-t᷂͡sʰ \\
 search-{\thgloss}-{\aor} \\
 \trans `He searched.' \\
 \armenian{փնտրեց}
\end{exe}

\subsubsection{Imperative}

An especially interesting form is the second type of imperative (\ref{sent:Artvin:morpho:verb:imp:artvin}).

\begin{exe}
 \ex Artvin \label{sent:Artvin:morpho:verb:imp:artvin}
 \begin{xlist}
 \ex \gll ɡɾ-\'i-s ɑ \\
 write-{\thgloss}-2{\sg} ? \\
 \trans `Write!' \\
 \armenian{գրի՛ս ա}
 \ex \gll χos-\'i-s ɑ \\
 speak-{\thgloss}-2{\sg} ? \\
 \trans `Speak!' \\
 \armenian{խօսի՛ս ա}
 \end{xlist}
\end{exe}

These correspond to the Istanbul interrogative-like imperatives (\ref{sent:Artvin:morpho:verb:imp:istanbul}). 


\begin{exe}
 \ex Istanbul, when read as SWA words \label{sent:Artvin:morpho:verb:imp:istanbul}
 \begin{xlist}
 \ex \gll t͡ʃə-kʰəɾ-\'e-s \\
 {\neg}-write-{\thgloss}-2{\sg} ? \\
 \trans `Don't write!' \\
 \armenian{չգրե՞ս}
 \ex \gll t͡ʃə-χos-\'i-s \\
 {\neg}-speak-{\thgloss}-2{\sg} ? \\
 \trans `Don't speak!' \\
 \armenian{չխօսի՞ս} 
 \end{xlist}
\end{exe}



\translatorHD{I think what he means is that these Artvin imperative seem to be derived from subjunctive verbs; the Istanbul verbs seem subjunctive based on how they would be interpreted in SWA. }

\section{Text samples}

{\sampleoverview}

\armenian{Առաջ Արթվին շէնլիկը տասնըհինգ տուն է էղէ. բօլօրը մէշա. էն մէշումը Սավէթումէն կու աշին օ՛րը (որ) Արթվինումէն մուխ կու դուս գա. գուգան կու փնտռին խիտը (վրաց. կամուրջ}). \armenian{չէն կա (չեն կրնար) գտնի օ՛րա Ճօրօխը անցնին. էտէվ մէկ ավջին գէյիգի հէտնէմէն գալիս է օ՛րը զարնէ. առաչէվան կայբ է ըլնէլի. կայբ էղած վախտին փնտռրէ՛լիս է վօ՞րանց գնաց։ Աշից օրը խիտը գտավ. խիտն էլ փուրցէլը (վրց. մացառ թէ՞ բաղեղ) փաթըթած է. է՛նղադուր արավ օ՛րա խիտը անցավ էնթին։ Վէր անցավ օրմընումը, փնտռից ու շէնլիկի տէղը գտաւ։ Իշտէ էնդօր էտէվ, էֆէնդիմ, օրմանը կօտրէցին, էնդէղը քաղաք շինէցին, իշտէ էնդէղը էղավ Արթվին։} 

\armenian{Էն գտնօղ մարթու անունն էլ Արութէն է էղէ, էնդօր վրա դրէլ է Արթվին։}

\begin{adjarianpage}\label{page:293}\end{adjarianpage}% should be 293
 
 \translatorHD{Note that Adjarian had a note here about the Armenian diaspora. I moved it to \S\ref{sec:Branches:excluded}. } 


\part{Introduction to the translation}

\todo{add adjarian endmatter}


\todo{do turkish todo} 


\todo{do russian todo} 

\todo{do leftover todos} 

\todo{go through all case of ʁ to make sure its not ɬ}

\todo{go through every acse of stress and make sure entire line is stressed}

\todo{double check instances of ուէ ուի ուո to make sure did u.V}

\todo{fix names of locations and humans to transliteration-like thing,
}




\todo{double check against \url{https://en.wikipedia.org/wiki/Armenian_dialects}}

\todo{double check against the resources that manuk avedikyan recommended}

\begin{quote}
	As for geography, we have multiple sources we use. I saw the Adjarian list you sent, which is a more top level list - more cities and towns with some villages mentioned from the wider region. We use Robert Hewsen's Armenian Atlas, Haykakan teghanunneru bararan, Nisanyan's Index Anatolicus website (the best go-to), other Armenian maps in various publications, Tahir Sezen Osmanli yer adlari and Nuri Akbayar's similarly titled book, which helps with administrative units, timelines and Ottoman-era spellings of top level places, Koylerimiz 1968 official Turkish publication for only settlements classified as villages and other sources that may be more specific like Municipality websites and tourism and archeology sites of articles that give out names of creeks or fields, etc... Sarkis Karayan's book by Gomidas institute is the most comprehensive list of Armenian place names in the Ottoman Empire - it's not perfect but closest to perfect.
\end{quote}

\chapter{INTRO}


\section{Introduction}

Armenian is an Indo-European language. Its oldest attested form is Classical Armenian (CA). The modern language    is conventionally described as having two standardized variants: Standard Western Armenian (SWA or WA) and Standard Eastern Armenian (SEA or EA).  Alongside these two varieties, there are countless non-standard dialects, many of which were were made extinct because of the Armenian Genocide. 

This book is an English translation of a monograph originally written in Armenian by Hrachia Adjarian \citep{Adjarian-1911-DialectologyBook}: ``Հայ Բարբառագիտութիւն'' or ``Armenian dialectology''. The original monograph consisted of descriptions of 31 non-standard Armenian varieties, most of which are now extinct. Some descriptions are rather lengthy, while some are short. 

The present book is both a translation and commentary on this monograph. In the course of translating the original 300-paged book into English, I had to unpack a lot of implicit knowledge that Adjarian was using in order to describe the varieties. For example, Adjarian did not use morpheme boundaries, glosses, nor IPA symbols.  He would often just provide data points from a dialect, with a brief prosaic description and with cognates from Classical Armenian. That brief description relied on the reader's knowledge of Standard Armenian (and sometimes Classical Armenian) in order to deduce the linguistic structure of the non-standard dialect. In order to unpack this implicit information, the end result was a 500-paged translation, with glossing, translation, and morpheme segmentation.


The current introductory is written by myself, the translator. The rest of this book is my translation of Adjarian's writing. At times, I have to provide commentary and interrupt Adjarian's prose. To prevent ambiguity, I wrote my interruptions in the following format:

\begin{center}
	\translator{This is an interruption by the translator, Hossep Dolatian.}
\end{center}

To maximize the recoverability of information during the translation, I provide the page numbers from Adjarian's original monograph. 

The rest of this chapter provides basic information on Armenian, its dialectology, and my translation conventions.

\section{Armenian: Dialectology, state, and linguists}

\subsection{Basics of dialectology}

As stated, Classical Armenian is the oldest attested Armenian variety (circa the 4\textsuperscript{th} century CE). In contrast, Modern Armenian    is conventionally described as having two standardized variants: Standard Western Armenian (SWA or WA) and Standard Eastern Armenian (SEA or EA). These two variants are often also called just Western Armenian and Eastern Armenian. Besides the 

Alongside these two varieties, there are countless non-standard dialects, many of which are extinct or moribund. Some dialects were made extinct because of the Armenian Genocide; Smyrna is a potential example \citep{Vaux-2012-ArmenianSmyrna}. Some of these dialects survived the Genocide, but its speakers underwent language shift to one of the  standard varieties. For example, the Shamakhi dialect is undergoing such a shift to SEA \citep{Vlasyan-2019-Shamakhi}. 



\todo{go through bib to find anything relevant; im thinking there was an article soemwhere that talked about the effects of the armenian genocide on dialects. but i cant find it.}


Western Armenian developed as a koine language among  Armenians in the Ottoman Empire. It is conventionally treated as largely developing from the native dialect of Istanbul Armenian; this is because Istanbul was the center of Armenian culture in th Ottoman Empire. Similarly, Eastern Armenian developed as a koine language for Armenians in the Russian and Persian empires. Its center of development was Yerevan and Tbilisi  (called Tiflis among Armenians). This is a rather simple summary of the emergence of these two koines; see \citet{SayeedVaux-2017-EvolutionArmenian} for further discussion. 


Besides these two standardized koines, Armenians spoke a variety of non-standardized local dialects. For example, the Armenians of Smyrna (modern Izmir) had their own dialect of Armenian, as does Karabakh or Artsakh.  Just as there are two standard forms (Western Armenian and Eastern Armenians), dialects are conventionally divided into branches. Some dialects like Symrna belong to the Western branch (and are more similar to Western Armenian than to Eastern Armenian). While some dialects like Karabakh/Artsakh belong to the Eastern branch. 



Within Armenian linguistics, a pioneering researcher was Hratchia Adjarian (Armenian: Հրաչեայ Աճառեան, reformed spelling; also transliterated as Acharian). He was born in modern Istanbul in 1876, and undertook an education in linguistics in France. In French, his two most groundbreaking works was \citet{Adjarian-1899-ArmenianExplosives} where he developed an experimental procedure for Armenian consonant acoustics, and discovering an early form of voice onset time \citep{braun-2013-earlyCaseVOTAdjarian}. His second major was \citet{Adjarian-1909-ClassificationArmenianDialect} ``Classification des dialectes arméniens'' (``Classification of Armenian Dialects'') where he catalogued, described, and classified a set of Armenian varieties. This French monograph was then the basis for a larger work in Armenian   \citep{Adjarian-1911-DialectologyBook}: ``Հայ Բարբառագիտութիւն'' or ``Armenian dialectology''. 

The above is a basic conventional summary of how Armenian dialects work. But there are some points of nuance that we should clarify. The next section discusses these points. 

\subsection{Minor problems in dialectology}

\subsubsection{What is Standard Armenian?}
The term `standard' in the name `Standard Western Armenian' does not denote a literary variety that's limited to books and formal speech. For the majority of Western Armenians who speak some Armenian variety, that variety is Standard Western Armenian. Similarly, Standard Eastern Armenian is the native language of the majority of Eastern Armenians. The confusing use of the term `standard' is caused by patterns of language shift and dialect leveling among Armenians.

For example, in 1911, the year that the Adjarian monograph was published, the Armenians of Istanbul spoke a specific variety of Armenian called Istanbul Armenian. This variety is documented in this book in chapter \S\ref{chapter:istanbul}. This variety is not identical to SWA. For example, the word for `father' is [hɑjɾ] in SWA; but Adjarian describes the Istanbul form as [hɑɾ] (\S\ref{sec:Istanbul:phono:soundchange:diph}). 


Although Adjarian does not discuss the sociolinguistic situation of this community in depth, we suspect that this variety was acquired by Armenian children  at home. At school, they would learn Standard Western Armenian, so that they could read, write, and engage with other Armenian communities as a lingua franca. This dialect was further studied by Adjarian in a later monograph \citep{Adjarian-1941-IstanbulDialect}. 

Since 1911 however, the Istanbul community has shifted from using the Istanbul dialect to using SWA. As a member of the Istanbul Armenian community, Tabita Toparlak reports that the dialect described by Adjarian has died out. Instead, Armenian-speaking families   have shifted to using SWA at home. 




A similar situation is described for SEA and  Yerevan. In the time of Adjarian 1911, the dialect of Yerevan  had a distinctive feature of penultimate stress (\S\ref{sec:Yerevan:phono:stress}). In contrast, SEA has final stress. But in the decades since, the community in Yerevan has shifted towards using SEA and not the old Yerevan dialect.  A native of this city (Vahagn Petrosyan) reports the following:

\begin{quote}
	[The] ``Yerevan dialect'' is a historical label. Currently Yerevan speaks a colloquial version of Standard Eastern Armenian.
	
	I have heard the features described for this dialect in the speech of some Yerevan residents. For me the speech is associated with the [lower classes]. I don't know if these people are recent migrants from villages of the Ararat dialect areas or if they are the remnants of the original Yerevan dialect speakers. In any case, an average person certainly does not grow up speaking like that. I am from Yerevan and I speak normally. 
\end{quote}

Thus, the majority of the dialects in Adjarian 1911 are likely extinct, due to either genocide or language shift. For the standard varieties (SWA and SEA), these are not fossilized variants restricted to books. They are the native language of most Armenian-speaking children and adults.\footnote{This situation is similar to the development of Italian. The language developed from Tuscan, and was standardized. The language then became a state language, and spread throughout Italy. }

In contrast, it is relatively rare to find Armenian communities that still speak a non-standardized dialect. Of the non-standard varieties that still exist, they are sadly heavily endangered. The  exception is the community of Armenians in Tehran, Iran, who have developed and maintained their own dialect of Tehrani Iranian Armenian \citep{DolatianEtAl-prep-IranianGrammar}, which is surprisingly absent frm Adjarian's work.



Of course, like any spoken language, SWA has both informal and formal registers. The informal register is acquired at home, while the formal register is taught at schools or acquired via formal interactions.  The two registers have minor differences. Some traits of the informal register of SWA is found in various non-standard dialects. For example,  informal spoken SWA     uses a progressive marker [ɡoɾ], while formal written SWA bans this marker. The use of [ɡoɾ] is likewise found across various non-standard Western dialects such as Istanbul. Similarly, SEA has both formal and informal registers. The 3SG auxiliary `is' is pronounced [e] in formal SWA, but pronounced [ɑ] in informal SEA. The use of an auxiliary [ɑ] is again found in non-standard Eastern dialects like Yerevan (\S\ref{sec:Yerevan:morpho:verb:morphodetials:copula}). 

Thus the presence of the word `standard' in the names `Standard Eastern Armenian' or `Standard Western Armenian' does not indicate prescriptivism, but is due to the history of the emergence of these standardized varieties. Because of this history, names like `Standard Western Armenian' and   `Western Armenian' are interchangeable. In my experience, in Armenian linguistics, it is common to add the word `standard' in order to disambiguate terms like `Western Armenian' which could designate either a single variety (Standard Western Armenian) or an entire branch of dialects (Western dialects). 

In contrast, the term `Standard Western Armenian' can mean either `the formal register of Western Armenian' or `either the formal or informal register of Western Armenian.'  In my case,  I grew up in an Armenian speaking household in Lebanon. I acquired the informal register of (`Standard') Western Armenian. My idiolect does not correspond to any of the non-standard varieties like Smyrna, Trabzon, or Crimea.  I then learned the formal register `Standard' Western Armenian at school. 







\todo{add colloquial yerevan citatipns}

\subsubsection{What is a dialect?}

The original Armenian monograph from 1911 was called ``Armenian dialectology'', based on an expansion of a French monograph from 1909 called ``Classification of Armenian dialects.'' This section clarifies the term `dialect'. 

In linguistics, a common criterion for labeling a language variety as a dialect is mutual intelligiblity. Given two language varities A and B, if a speaker of A can easily understand a speaker of B, then the two varities are dialects of the same language. Based on this criterion, American English and British English are dialects of English. 

For Armenian however, the various varieties are conventionally called `dialects', but they are not mutually intelligible. For example, as a speaker of SWA, I have difficulty fully understanding a spoken SEA sentence due to my limited exposure to spoken SEA. Written SEA is however quite intelligible to me. Thus, I often hear among lay speakers of Armenian that the two dialects are mutually intelligible after significant exposure. 

For the non-standard dialects, mutual intelligibility is much weaker. For example, this monograph has text samples for each of the 31 non-standard dialects. I could translate any of them because I couldn't understand them. 


Thus based on the criterion of mutually intelligibility, Standard Armenian (SEA and SWA) and the various non-standard varieties (Julfa, Tigranakert, and so on) are mutually unintelligible and not  dialects of the same language. Because of mutually unintelligibility, some linguists go so far to use the term `Armenoid' to describe the different Armenian varieties, e.g., that Agulis is an Armenoid language \citep{Vaux-2008-ArmenianZok} and not an Armenian dialect (\S\ref{chapter:Agulis}). 

In Armenian studies however, all these language varieties are just called `dialects'. The term is used in a non-theoretical stance. I suspect that  because      all these Armenian varieties are spoken by ethnic Armenians, then the term `dialect' is used to indicate ethnic solidarity. To reduce confusion, I will also use the term `dialect' in this translation, simply because Adjarian himself was using the Armenian word for dialect: [bɑɾbɑr] <բարբառ>. In my own commentaries, I will often use the terms `dialect' and `variety' interchangeably. 





\subsubsection{What are the dialects?}

In Armenian dialectology, dialects are commonly divided into two branches: Western and Eastern. We   discuss some controversial choices that Adjarian made in his classification. 


As said before, Classical Armenian is the oldest attested Armenian variety. Modern Armenian varieties are conventionally divided into groups: Western dialects and Eastern dialects. In general, the division between dialects is based on geographic origin. The division (imperfectly) corresponds to the modern Turkey-Armenia border. Armenian varieties that developed west of this border (in the Ottoman Empire) belong to the Western branch, while varieties that developed to this east of this order (in the Russian/Persian empires) belong to the Eastern branch.  

For example, SWA developed in Istanbul, while SEA developed in Tbilisi and Yerevan. SWA and SEA are conventionally treated as standardized offshoots of Istanbul and Yerevan. This basic classification is summarized in Figure \ref{tree:dialect:basic}. 

\begin{figure}[H]
	\caption{Conventional and simplified family tree of Armenian}
	\label{tree:dialect:basic}
\centering
\begin{forest}
[{Classical Armenian}  [Western  [... ] [Trabzon ] [Crimea ] [Istanbul   [{Standard Western}] ] ]  [Eastern [Yerevan [{Standard Eastern}] ] [Julfa] [Agulis] [...]   ] ]
\end{forest}

	\end{figure}

  
The above simple classification correlates with an important islogloss in Armenian dialectology: the morphemes used to form the indicative present (Table \ref{intro:tab:ea wa ia differences: morpho}) \citep{Vaux-1995-ArmenianVerbDiachrony}. In Classical Armenian, the indicative present was formed by adding agreement suffixes directly onto to the verbal stem. The verb stem consists minimally of a root and theme vowel. But in modern SEA and SWA, this simple synthetic construction is instead used for the subjunctive present. To form the indicative present, SWA adds a prefix /ɡə/ կը before the synthetic form. In contrast, SEA uses a periphrastic or analytic construction. The verb is a non-finite form called the imperfective converb. The verb takes the non-inflecting suffix /-um/, while agreement is on an auxililary. 

\begin{table}[H]
	\caption{Morphemes used for the indicative present in CEA, SWA, and SEA}\label{intro:tab:ea wa ia differences: morpho}

%		\ex 
		\centering
		\begin{tabular}{|l| ll|}
			\hline   	& \multicolumn{2}{l| }{Indicative present `I like' } \\
		\hline 	CA & siɾ-e-m& like-{\thgloss}-1{\sg}\\
			SWA & ɡə-siɾ-e-m&{\ind}-like-{\thgloss}-1{\sg}\\
			SEA & siɾ-um e-m&   like-{\impfcvb} {\aux}-1{\sg}\\
			\hline 
		\end{tabular}

\end{table}

The above  parameters (geographical and morphological) are foundational to Armenian dialectology, but they have some problems. 

For the geographical parameter, the terms `Western' vs. `Eastern' refer to the ultimate geographic origin of some Armenian variety. But as Adjarian describes in \S\ref{sec:ThreeBranch:overview}, this geographic parameter can be confusing when we take into account migration patterns. For example, the dialect of Karin  (\S\ref{chapter:Karin}) is a Western dialect that historically developed in what is now modern Erzerum (in modern eastern Turkey). But during the 19\textsuperscript{th} and after the Armenian Genocide,   the Armenian community of Erzerum had migrated to what is now modern Armenia and Georgia. That is, this Western community moved from west of the Turkey-Armenia border to the east of this border. For Karin, it is thus somewhat call this dialect a Western dialect, even though it is geographically spoken on the east of the relevant geographic border. 

Because of the above geographical problems, Adjarian argued that descriptions like `Western' vs `Eastern' branches should be replaced with terms based on isoglosses. The most obvious isogloss for Adjarian was the morphology of the indicative present. He specifically argued that `Western' dialects (like SWA) belong to the /kə/ <կը> branch, while `Eastern' dialects (like SEA) belong to the /um/ <ում> branch.\footnote{Note that Adjarian used the term /kə/ branch instead of /ɡə/ branch. Although the   indicative prefix is /ɡə/ in SWA, this prefix is spelled <կը>. The ancestor of this prefix is pronounced as /kə/, such as in SEA where this prefix is used to mark conditional future.}

In addition to replacing geographic descriptions with morphological ones, Adjarian also argued that some dialects belong to a third separate branch. This branch is called the /el/ <էլ> branch and includes dialects like Maragha (\S\ref{chapter:Maragha}). For such dialects (\S\ref{sec:Maragha:morpho:verb:overview}), the indicative present   is formed by adding an auxiliary after the infinitive. The construction is periphrastic, and the surface [el] sequence is actually the theme vowel /-e-/ plus the infinitive suffix /-l/ (\ref{sent:Maragha:morpho:verb:infconj:merged}). 

\begin{exe}
		\ex Maragha (taken from \ref{sent:Maragha:morpho:verb:infconj:merged})
	
	 \gll ʏz-e-l-i-m \\
			want-{\thgloss}-{\infgloss}-{\aux}-1{\sg} \\
			\trans `I want.'\\
			իւզէլիմ \label{sent:Maragha:morpho:verb:infconj:merged REP in intro} 
			
	
\end{exe}

To summarize, Adjarian argues for a more detailed classification, as in Figure \ref{tree:dialect:adj}. 

\begin{figure}[H]
	\caption{Expanded family tree of Armenian based on Adjarian's classification}
	\label{tree:dialect:adj}
	\centering
	\begin{forest}
		[{Classical Armenian}  [/kə/ branch  [... ] [Trabzon ] [Crimea ] [Istanbul   [{Standard Western}] ] ]  [/um/ branch [Yerevan [{Standard Eastern}] ] [Julfa] [Agulis] [...]   ]  [/el/ branch [Maragha] [...] ] ]
	\end{forest}
	
\end{figure}

In contrast, in a simpler two-branch classification, the dialects of the /el/ branch  would be considered Eastern dialects. First, dialects like Maragha were formed in modern-day Iran and Russia; thus they are geographically east of Turkey. Second,  both /um/ branch and /el/ branch dialects utilize morphological periphrasis in forming the indiciative present. Thus, varieties like SEA and Maragha share a more abstract isogloss. Figure \ref{tree:dialect:compromise} shows a possible family tree by combining both geography and Adjarian's three branches. 

\begin{figure}[H]
	\caption{Family tree of Armenian based on geographic terms and Adjarian's /el/ branch}
	\label{tree:dialect:compromise}
	\centering
	\begin{forest}
		[{Classical Armenian}  [{Western (/kə/ branch)}  [... ] [Trabzon ] [Crimea ] [Istanbul   [{Standard Western}] ] ]  [Eastern  [{/um/ branch} [Yerevan [{Standard Eastern}] ] [Julfa] [Agulis] [...]   ]  [/el/ branch [Maragha] [...] ] ] ]
	\end{forest}
	
\end{figure}
 
This book maintains Adjarian's original three-way classification system. I do this so that the translation is faithful to Adjarian's original intentions. However, since 1911, it seems that most dialectological work in the West has not replaced geographic terms with isogloss-based terms.\footnote{I personally think that abandoning geographic terms is unconvincing. Although it is true that some Western dialects like Karin are now   spoken east of the Turkey-Armenia border, they still historically developed west of this border. What matters is a dialect's genetic relationships with other dialects, and geography is a major correlate of such genetic connections. } As for Soviet Armenia and the modern Republic, it seems that further dialectological work uncovered more and more sub-branches and groups, that's easier to summarize geographically \citep[\S 4]{Martirosyan-2019-Armeniandialects}. 



 \section{Phonological transcription}

Provide IPA to 

\subsection{Classical Armenian pronunciations}\label{sec:HossepIntro:phonotransc:CA}

Classical Armenian or CA is the oldest attested variety of Armenian. The earliest written records are from the fifth century. This section explains the phonemic inventory of Classical Armenian, and sets up an IPA transcription for Classical Armenian. I use IPA transcriptions so that we have a clearer idea on what sound changes occured from Classical Armenian to the modern dialects. 

\subsubsection{Difficulties in Classical Armenian}
Classical Armenian (CA) is an ancient dead language. We have no access to speakers, recordings, or phonetic analyses of CA. Thus, we will never know exactly what CA sounded like. Instead, we can approximate a probable CA phonology using the following pieces of information:
\begin{enumerate}
	\item orthography and transliteration conventions
		\item traditional pronunciation
\item post-Classical phonological changes
\end{enumerate}

To clarify the above points, Classical Armenian is written using the Armenian script. The script was invented in order to write Classical Armenian. It is thus likely that the orthography is close to the pronunciation of Classical Armenian.  The orthography is traditionally transliterated using the Hübschmann-Meillet-Benveniste transliteration system (HMB). Transliteration schemes can be found online, such as Wiktionary.\footnote{\url{https://en.wiktionary.org/wiki/Wiktionary:Armenian_transliteration}} The transliteration is not a phonological nor phonetic transcription, but it does help us determine approximate IPA symbols for CA. 

As for pronunciation, although CA is a dead language, there is a conventional system for how to read CA texts. This system is called `traditional pronunciation'. It was formulated sometime after the first written record of CA. An approximate date for this formulation is   between the 8\textsuperscript{th} and 12\textsuperscript{th} centuries  (\citealt[24]{Godel-1975-IntroClassicalArmenian}; \citealt[1039]{Macak-2017-PhonoClassicalArmenian}). The formulated conventions indicate a mix of phonological patterns that were attested in CA or that developed later in the post-Classical period.

For this book, I transcribe all CA forms using IPA. I do not use transliteration. The rationale is that transliteration systems by themselves do not unambiguously reflect the most likely phonological form of CA words. In order to understand how the various sound changes from CA to the modern dialects, it is more practical to transcribe both CA and modern Armenian in their phonological form, i.e, by using IPA symbols. 
\subsubsection{Monophthong vowel inventory}

Classical Armenian has seven basic monopthong vowels. These vowels are listed in Table \ref{tab:HossepIntr:classicalVowel}. I provide the native orthographic form, the HBM transliteration, and an approximate IPA symbol. 



\begin{table}[H]
	\centering
	\caption{Monophthong vowels of Classical Armenian}
	\label{tab:HossepIntr:classicalVowel}
	\begin{tabular}{|l|lllllll|}
		\hline 
		Orthography & ա & ե &  է & ը& ի & ո & ու\\
		HMB transliteration & a & e & ē & ə & i & o & u  \\
		IPA transcription & ɑ & e & ē  & ə & i & o & u  
		\\ \hline
	\end{tabular}
\end{table}



For the IPA transcription of Classical Armenian vowels, I adapt conventional transcriptions from the traditional pronunciation and from the modern standard dialects in the following way. 


For the grapheme <ա>, the modern standard dialects use a low back unrounded vowel /ɑ/. We don't know exactly what the ancient language used. For simplicity and illustration, we assume the Classical low vowel was likewise back. This seems to be an implicit assumption by Adjarian as well, because he later uses a different symbol <ա̈> to mark the low front vowel /æ/. 

For the front midvowel pair <ե, է>, we don't know the exact  phonetic difference in Classical Armenian. The two graphemes are often transliterated as <e> vs. <ē>, and they are argued to have a phonological contrast in terms of tenseness \cite[14]{Thomson-1989-IntroClassicalArmenian} or length  \citep[6]{Godel-1975-IntroClassicalArmenian}. I transcribe them as /e/ vs. /ē/. 


In the modern standard varieties, the word-medial reflex of these two letters is just /e/. The /e/ can be pronounced at essentially any height within the midvowel range: close-mid [e] or open-mid [ɛ]. For my own SWA ears, I cannot perceive the difference between [e, ɛ], suggesting that Armenian has a generic articulatory target for front midvowel.

For the midvowels <ե, ո>, the modern standard dialects can range between using low-mid /ɛ,ɔ/ and high-mid /e,o/. Such variation is free variation in my experience. For simplicity, I transcribe them high-mid /e,o/ instead of low-mid /ɛ,ɔ/, contra \citet[1039]{Macak-2017-PhonoClassicalArmenian}. 

The segments <ը, ի, ու> and transliterated and traditioally pronounced as  /ə, i, u/. 

\subsubsection{Diphthong vowel inventory}\label{sec:HossepIntro:phono:Classical:Diphthong}
In addition to monophthongal vowels, Classical  Armenian had  nine diphthongs (Table \ref{tab:HossepIntr:classicalDiphthong}). 


\begin{table}[H]
	\centering
	\caption{Diphthong vowels of Classical Armenian}
	\label{tab:HossepIntr:classicalDiphthong}
	\begin{tabular}{|l|lllllllll|}
		\hline 
		Orthography & այ &  աւ & եա &  եւ &  եայ&  եաւ&  իւ &  ոյ &  ուա\\
		HMB transliteration & ay &  aw & ea &  ew &  eay&  eaw&  iw &  oy &  ua\\
		IPA transcription & ɑi̯ &  ɑu̯ &e̯ɑ &  eu̯  &  e̯ɑi̯&  e̯ɑu̯ &  iu̯  &  oi̯ &  u̯ɑ
		\\ \hline
	\end{tabular}
\end{table}

Orthographically, Classical Armeinan diphthongs are made up of  a) a  vowel plus a glide symbol like <այ> <ay>,   b) two vowels like <եա> <ea>,  or c) a combination of vowels and glides like <եաւ> <eaw>. 

These orthographic sequences like <այ> <ay> were   pronounced and phonologically treated as diphthongs like [ɑi̯] and not as vowel-glide sequences like [ɑj]. The evidence is the following.  Philological and dialectological work uses the term `diphthong' (Armenian: [jeɾkbɑɾbɑr] <երկբարբառ>, literally `two-sounds' in Classical Armenian).  In the modern standard languages,   orthographic vowel-glide sequences like <ay> are pronouned as vowel-glides sequences /ɑj/, and philologists like Adjarian explicitly state that the standard dialects lack diphthongs (\S\ref{sec:AdjarianIntro:difference:soundChange:DiphthongLoss}). 

As for their IPA values, it is difficult to give a meaningful transcription for Classica diphthongs. I follow \citet{Macak-2017-PhonoClassicalArmenian} in placing an inverted breve under the less prominent member of the diphthong (= what would correspond to a high vowel). I note the following minor notatonal differences between my transcription and Macak. 

\begin{itemize}
		\item  For <եա> <ea>, \citet[1041,1043]{Macak-2017-PhonoClassicalArmenian} suggests /i̯ɑ/ but I opted for /e̯ɑ/ because it's more faithful to the orthography. 
		\item  For <իւ> <iw>, \citet[1041,1043]{Macak-2017-PhonoClassicalArmenian}  notes that this cluster can be pronounced as either /iu̯/ or /i̯u/ depending on phonological position. I opt for a uniform /iu̯/ because Adjarian does not notice such differences.
		\item For <ոյ>, the traditional pronunciation is /ui̯/ \citep[1039]{Macak-2017-PhonoClassicalArmenian}. But, the orthography suggests that this digraph was pronounced as /oi̯/. 
\end{itemize}

For <եայ>, I could not find a pre-established convention so I use /e̯ɑi̯/. 
	
There is some ambiguity when the an orthographic diphthong is  pre-vocalic like <այա> or <աւա>. The HMB translation is just <aja, awa>. Phonologically, I suspect the offlgide would have acted as a consonantal onset /ɑjɑ, ɑwɑ/ and not as a sequence of vowels /ɑi̯ɑ, ɑu̯ɑ/. I thus transcribe such pre-vocalic diphthongs as vowel-glide sequences. However, note that Adjarian seems to   phonologically treat these pre-vocalic forms as still phonologically diphthongs instead of vowel-glide sequences  (\ref{page:64}).  




There are other attested orthographic vowel-vowel sequences such as <ուէ> in <աղուէս> `fox' and <ուի> in <թուիլ> `to appear'. For these, the HMB transliteration would be <aɬuēs> and <tʿuil>. Their modern SEA pronunciations would use a /v/ in place of the <ու>: /ɑʁves, tʰəvil/. It's unclear if   historically such orthographic sequences were some type of diphthong too: /ɑɬu̯es, tʰu̯il/. But it is seems that the convention is to treat the digraph <ու> as a non-alternating /u/ \citep[15]{Thomson-1989-IntroClassicalArmenian}, and allow it to be part of vowel hiatus \citep[17]{Thomson-1989-IntroClassicalArmenian}. To be safe, I treat such sequences then as vowel hiatus as well: /ɑɬu.es, tʰu.il/.

Note that Classical grapheme sequence <աւ> /ɑu̯/  became SEA /o/, and this change encouraged the use of a new letter <օ> in its place. Adjarian often uses the letter <օ>    to refer the  ancient diphthong. When he does use the letter <օ> in these contexts (such as <մօր> `mother.{\gen}), I use the transliteration <ō> and the Classical pronunciation  /ɑu̯/: <mōr>, /mɑu̯ɾ/. I usually opt to use an alternative CA spelling with <աւ>: <մաւր> <mawɾ> /mɑu̯ɾ/.    I do this so that it's clearer what were the actual sound changes from Classical Armenian to the modern dialects. 


\subsubsection{Consonant inventory}



Classical Armenian had 30 consonants (Table \ref{tab:HossepIntr:classicalConsonant}).  

\begin{table}[H]
	\centering
	\caption{Consonants of Classical Armenian}
	\label{tab:HossepIntr:classicalConsonant}
	\begin{tabular}{|l|lllllllll|}
		\hline 
		Orthography & բ &պ& փ &դ& տ &թ& գ& կ& ք   \\
		HMB transliteration &  b &p& pʿ &d& t &tʿ& g& k& kʿ  \\
		IPA transcription & b &p& pʰ &d& t &tʰ& ɡ& k& kʰ  \\
		\hline 
		Orthography &ձ& ծ& ց &ջ& ճ& չ  & & &  \\
		HMB transliteration &j &c &cʿ& ǰ &č &čʿ & & &   \\
		IPA transcription & d͡z & t͡s & t͡sʰ & d͡ʒ & t͡ʃ & t͡ʃʰ & & & \\
		\hline 
		Orthography & վ & ս&  զ&  շ&  ժ&  խ & հ & & \\
		HMB transliteration & v & s&  z&  š&  ž&  x & h & & \\
		IPA transcription& v & s&  z&  ʃ&  ʒ&  χ & h & & 
		\\          
		\hline
		Orthography & մ &  ն &    ր&  ռ&  լ&  ղ &  ւ & յ &  \\
		HMB transliteration & m & n & r & ṙ&l &   ł & w & y &  \\
		IPA transcription & m & n & ɾ & r& l &  ł & w & j& 
		\\ \hline 
	\end{tabular}
\end{table}

For stops and affricates, Classical Armenian had a three-way laryngeal contrast. These contrast is conventionally treated as between voiced, voiceless unaspirated, and voiceless aspirated /b, p, pʰ/. 

For the fricatives, there are generally uncontroversial. 		 For the back fricative <խ>, the modern standard dialects show free variation between a velar /x/ vs. uvular /χ/ articulation. The uvular transcription is however more typical. I use the uvular form as the default transcription for Classical Armenian. 

For the nasals, there is no controversy. Note that the modern standard dialects have an allophonic velar nasal [ŋ] that's used when an alveolar nasal precedes a velar stop. The Armenian orthography does not mark this in any variety, including Classical Armenian. It is unknown if Classical Armenian likewise had nasal place assimilation before velar stops, but it is likely. To be safe, I don't use a velar nasal [ŋ] for Classical Armenian. 

For the rhotics    <ր,ռ> or <r, ṙ>, they are    pronounced as a flap vs. trill in the modern SEA /ɾ, r/. It's unclear if the <ր> was a flap or an approximant in the Classical language \citep[1040]{Macak-2017-PhonoClassicalArmenian}. I opt for a flap /ɾ/. Adjarian himself does not comment on the pronunciation of this rhotic. 


For the liquids, the symbol <լ> <l> is pronounced as a simple lateral /l/ in the traditional pronunciation and the modern standard dialects. The symbol <ղ> <ɬ> is pronounced as a voiced uvular fricative in the modern standard dialects, while it is generally treated as a dark or velar lateral /ɬ/ in   Classical Armenian (\S\ref{sec:AdjarianIntro:difference:soundChange:VelarGlide}, \citealt[ch2]{Macak-2016-StudiesClassicalModernArmenianPhono}).

For the sonorans <յ, ւ>, these are traditionally transliterated as <y,w>. These sounds are the glides /j,w/. However, it is difficult to know when such a letter was pronounced as a glide vs. part of a diphthong (\S\ref{sec:HossepIntro:phono:Classical:Diphthong}). 

\subsubsection{Schwa epenthesis}


Classical Armenian has a schwa symbol <ը> /ə/. This vowel is written in some words  like /əst/ <ըստ> <əst> `for'. However, it is likely that the sound /ə/ was pronounced in many words but was unwritten in the orthography. 

In the modern standard dialects, the orthography has long clusters of consonants (Table \ref{tab:IntroHossep:schwaEpen}). These clusters are broken up by schwas in pronunciation. A conventional analysis is to treat these schwas as epenthetic \citep{Vaux-1998-ArmenianPhono}. The rules for epenthesis are complicated but rule-governed \citep[cf.]{Dolatian-prep-Schwa}. It is likely that these epenthetic schwas present likewise in Classical Armenian. 



\begin{table}[H]
	\caption{Schwa epenthesis in Classical Armenian and the standard dialects} \label{tab:IntroHossep:schwaEpen}
	\centering 
	\begin{tabular}{|l| lll| l| }
		\hline `fire'	& CA & SEA  & SWA & \\
		<krak> &  k\textbf{ə}ɾɑk &ɡ\textbf{ə}ɾɑɡ &ɡ\textbf{ə}ɾɑɡ & կրակ
		\\\hline 
	\end{tabular}
\end{table}

There are various reasons to assume that Classical Armenian had the same unwritten schwa epentheiss rules as the modern standard dialects. In the traditional pronunciation,  it is  a convention is to pronounce unwritten schwas in exactly the same places as their modern forms (\citealt[16]{Godel-1975-IntroClassicalArmenian}; \citealt[116]{Thomson-1989-IntroClassicalArmenian};  \citealt[1043]{Macak-2017-PhonoClassicalArmenian}).   Diachronically, some of these unwritten schwas are reflexes of   Proto-Indo-European full vowels,  that got reduced in Proto-Armenian \citep[26]{Vaux-1998-ArmenianPhono}.  There is no synchronic  evidence of an unreduced vowel in the underlying form for these unwritten schwas.

Because of the above facts, I transcribe Classical Armenian with essentially the same epenthetic schwas that the modern standard dialects use. There are some situations where the traditional prononuciation of Classical Armenian sometimes     uses an epenthetic schwa while the standard dialects don't. For example,     the suffix /-kʰ/ <ք> is a nominalizer in   SEA and SWA. In the modern language, it does not use schwa epenthesis after stops or two consonants: [pɑɾt-kʰ] `debt' <պարտք>. But there are ambiguous and contradictory reports that the CA ancestor form (the plural  suffix \textit{-kʰ})  does use schwa epenthesis in more contexts than SEA/SWA. For example,  \citealt[18-19]{Godel-1975-IntroClassicalArmenian}'s prose suggests schwa epenthesis applies   after  a CC cluster [pɑɾt-əkʰ] or after a stop/affricate. In contrast,    \citealt[120]{Thomson-1989-IntroClassicalArmenian}'s prose suggests no schwa epenthesis after a CC cluster [pɑɾt-kʰ]. Thomson suggests that schwa epenthesis applies only if the /-kʰ/ follows a velar stop. For these limited cases where schwa epentheiss is unclear, I transcribe the CA forms with a question mark: [pɑɾt-(ə?)kʰ]. 




\subsubsection{Stress}


%
Thus, these epenthetic schwas are diachronically stable. They have been pronounced but unwritten for at least 1,000 years. 
There have been a few attempts at formalizing the rules for pronouncing these unwritten schwas for Classical Armenian \citep{Hammalian-1984-PhonoOldArmenian,Schwink-1994-ArmenianSchwaLexicalized,Pierce-2007-SchwaClassicalArmenian}. \citet{Pierce-2007-SchwaClassicalArmenian} has noted that as a spelling-pronunciation rule, essentially the same schwa epenthesis rules  are active for Classical Armenian and for Modern Armenian.



\subsection{Modern Standard pronunciations}
\subsection{Adjarian's notation in IPA}

\subsection{Transcription conventions}

ŋ
Classical transliteration: Hübschmann-Meillet-Benveniste

\subsection{General consonant and vowel inventories}

\subsection{Vowel inventory}


\subsection{Typical low vowel <ա>  /ɑ/}

For the letter <ա>, most traditional transliteration systems use a simpler transcription as <a>.  But phonetically, this letter represents a low back unrounded vowel /ɑ/ in modern Standard Western and Standard Eastern Armenian. Although we don't access to articulatory or acoustic data on the Armenian dialects, we suspect that Adjarian is using <ա> to denote a back unrounded vowel as well  for the following reasons. 


First, dialectological work often distinguishes a typical vowel <ա>  against an atypical fronted form like <ա̈> /æ/. This suggests that even in 1911, Adjarian perceived <ա> as contrasting against a front <ա̈> /æ/ by being back. 

Second, in the IPA, the letter <a> represents a front vowel too. Thus, if we use both <a, æ>, then it can create a false impression that there's a phonemic contrast between two front vowels <a, æ>, instead of between a front and back vowel <æ, ɑ>. 

Third, Adjarian was often sensitive in his perception of subtle acoustic differences. For example, in some dialects, he gives subtle judgments by saying that the low vowel <ա> had a more closed mouth in the dialect of \todo{get dialect example}. This suggests that he himself felt that <ա> represented a back \textit{unrounded} vowel. 

However,  in our own fieldwork, the dialect of Tehrani Iranian Armenian has a rounded back vowel /ɒ/ due to Persian contact \citep{DolatianEtAl-prep-IranianGrammar}. Perceiving this rounding is quite subtle. So it is possible that some of the dialects that Adjarian studied from Iran did in fact use a back rounded form /ɒ/ instead of a back unrounded form /ɑ/. It is impossible to know what was the case 100 years ago.

\subsection{Low front vowel /æ/}

Some dialects developed a low front vowel, corresponding to IPA /æ/. Adjarian used a special symbol to mark this sound: 
<\armeniang{ՠ}>. This letter was not used in future work in Armenian dialectology, but was replaced by the letter <ա̈>. Our text compiler had issues in using Adjarian's original letter <\armeniang{ՠ}> and it is essentially never used in modern works. So we opted to use the Armenian letter <ա̈>. 

\subsection{Mid vowels}


ե: i̯e

է: e

classical: է ē

classical diphthongs

ո: u̯o

\subsection{Other vowels}

<էօ, իւ, ու> /ø, ʏ

/u̯e/ <ուէ>

maragha
/ɨ/ <ը̂>  
/ui̯/ <ուⁱ>
/əi̯/ <ըⁱ>  


ը° : upside ը: a lowered ə as ə̞
\subsection{Uvular fricatives}
For the letters <խ,ղ>, modern Standard Western and Standard Eastern use uvular /χ, ʁ/. Though in the translator's experience, velar pronunciations are possible as free variation. Dialectological work does not distinguish velar vs. uvular pronunciations. So although we transcribe these letters consistently as uvular across the dialects, it is possible that they may be more velar in some dialects than others. 

\subsection{Rhotics}
For the rhotic letters <ռ,ր>, the modern Standard Eastern pronunciation is a trill-flap distinction /r-ɾ/. Modern dialects differ in whether the transcribed <ր>  letter is truly a flap /ɾ/ vs. an approximant /ɹ,ɻ/. For example, Standard Western and Eastern Armenian use a flap /ɾ/, while Tehrani Iranian Armenian uses an approximant /ɻ/.  Dialectologists donփt distinguish these flavors of /ɾ/. But to reduce confusion with the trill, we transcribe the trill as /r/, and the non-trill rhotic as /ɾ/. 

\subsection{Other consonants}
/ʕ, ħ, q/ /   <ՙ, հՙ, ղՙ/   and /lʲ/  <լՙ> 

՚: ħ  

յ̵   : ɦ

ղՙ /q/

he doesnt transcirbe velar nasals

ըէ

էօ իւ

shamakhi
էօօ
օօ

/œə̯/ <է\`օ>
classical schwa epenthesis before Ք godel page 18

/u̯ɑ/ <ուա>... 

<ւա> = /wɑ/ or /u̯ɑ/, <աւ> = /ɑu̯/, <օւ> = /ou̯/, <էւ> = /eu̯/, <եւ> = /i̯eu̯/, and so on. 

/ĕi̯     ,  ou̯, œu̯/  <էʲ,  օւ, էօւ>

oə̯, ei̯, ii̯, ɑi̯/ <օը, էյ, իʲ, աʲ>

inconstent in suceava         <իւ>   read as <ի՛ու> /\'iu/   /iu̯/ \\ 
<իեւ>   read as <իյէ՛ու> /ij\'eu/  /i̯eu̯/ \\
<իը>   read as  <ի՛յը> /\'ijə/  /iə̯/ \\ 


hes actually incosnstent in whehter a transcribed էյ should be just a vowel-glide sequence ej vs a diphthong ei̯ 


These are /ɑi̯,  ɑə̯, ɑu̯,  ĕi̯, u̯ɑ/  <աʲ,   աը, աւ, էʲ,  ուա>.


/e̞/ <է ̀ >

/ɑ̃/ <ա̄>
\section{Glossing and segmentation}
\subsection{Terminology}

գաղութ

emigre, migrant, settlement, 


New Armenian, Old Armenian


sound changes (ձայնաշրջութեամբ) 

provincial (գաւառական)


պայթուցիկ is literally stop, but he means it to refer to either stops or affricates. 





\subsection{Not Armenian}
Adjarian used Armenian script instead of Latin for Turkish. 


Wiktionary

Turkish: Tabita

Russian: Nikita

Georgian: David Erschler

Persian: Nazila Shafiei

me, sWA

\subsection{Names of people and locations}


Old Armenian, literary
\section{Dialectological parameters for variation}
bucket of citations to other people like martirosyan

The two branches show various parameters for dialectal variation. I focus on two major ones that distinguish SWA and SEA: laryngeal contrasts and the synthetic morphology. 


Phonologically, SEA has a three-way larnygeal contrast or voicing contrast among stops and affricates: voiced, voiceless unaspirated, and voiceless aspirated. Classical Armenian (CA) is argued to have had a similar three-way contrast as well. In contrast, SWA has a simpler two-way contrast: phonologically voiced and phonologically voiceless (Table \ref{tab:intro:ea wa differences: phono}). Conventionally, the SWA stops and affricates are treated as being voiced vs. voiceless aspirated.\footnote{The phonetic manifestation of the SWA voicing contrast is subject to geographical variation \citep{kellyKeshishian--2021-VoicingWesternArmenian,Tahtadjian-2021-PhoneticInterferenceProductionStopsWesternArmenianBilingual}.  For example, the SWA-speaking community in Turkey has a voiced vs. voiceless aspirate distinction: D vs. Tʰ. In contrast, the SWA-speaking community in Lebanon  instead has voiced vs. voiceless unaspirated: D vs. T.}

\begin{table}[H]
	\caption{Three-way laryngeal contrast in SEA  but not SWA}
	\label{tab:intro:ea wa differences: phono}
	\centering
	\begin{tabular}{|l|llll| lll|  }
		\hline   	& CA & SEA &   &  SWA & & & 
		\\
		/p/ & \textbf{p}ɑɾ& \textbf{p}ɑɾ  & `dance' & պար & & &  
		\\
		/pʰ/ & \textbf{pʰ}ɑk   & \textbf{pʰ}ɑk   & `closed' &փակ& \textbf{pʰ}ɑk & `closed' & փակ
		\\
		/b/ & \textbf{b}ɑd  & \textbf{b}ɑd  & `duck'  & բադ &  \textbf{b}ɑɾ & `dance'& պար \\ \hline
	\end{tabular}
\end{table}


The change from a three-way contrast in CA to a two-way contrast to SWA is a major topic in the diachronic phonology of Armenian. 
\subsection{Phonological parameters}

\subsubsection{Laryngeal changes or stop/affricate voicing}

\subsubsection{Palatalization of consonants}

\subsubsection{Word-initial diphthongization}

\subsubsection{Collections of individual sound changes}

\subsection{Morphophonological parameters}

\subsubsection{Vowel harmony}

\subsubsection{Affix mobility or prefix-suffix movement}

\subsubsection{Affix or morpheme repetition}

\subsubsection{Changes in theme vowels and the auxiliary}

\subsubsection{Voicing assimilation}


\subsection{Morphological parameters}

\subsubsection{Synthetic vs. periphrastic basic tenses}

\subsubsection{Archaism in the past plural ending}
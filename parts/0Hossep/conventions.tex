\part{Introduction to the translation}
\todo{fix errata from adjarian}
\todo{add adjarian endmatter}

\todo{go through all case of ʁ to make sure its not ɬ}

\todo{go through every acse of stress and make sure entire line is stressed}

\todo{double check instances of ուէ ուի ուո to make sure did u.V}

\todo{fix names of locations and humans, make sure no bolding in th names chapter
}

\todo{explain}
Տաճկահայաստան, Տաճկաստան , տաճկահայ -- ottoman turkey

Persian Armenia and Persarmenia (Armenian: Պարսկահայաստան –

added: Սըղնախ Signagi Sighnaghi

Վօլգա Վոլգա Volga

Մերվ Merv

Գութեր Kuter?\footnote{I couldn't track down}

Արշիպեղագոս

Մարմարա Marmara

Վոսփոր Bosporus Bosphorus

Եփրատ Euphrates

Միջերկրական Mediterranean

Սեւ ծով Black Sea

Սեւանայ լիճ Lake Sevan

Կասպից ծով Caspian Sea

Paytakaran (: Փայտակարան


Ermenikend[46] (Armenian: Արմենիքենդ; Azerbaijani: Ermənikənd -- original Էրմէնի քեանդ 

Արաքս : Araks, Aras

Ալիս  Alis, Kızılırmak river

  Նիկիա  Nicaea

  Կոստանդնուպօլիս Constantinople
'


Ոսկեղջիւր Golden Horn

\todo{double check against \url{https://en.wikipedia.org/wiki/Armenian_dialects}}


\chapter{INTRO}

\section{Armenian: Dialectology, state, and linguists}
\subsection{Brief history and dialectology  of Armenian}
\subsection{Adjarian as the found of Armenian linguistics}
\subsection{Goals of the translation}






\section{Phonological transcription}



\subsection{Classical Armenian pronunciations}

\subsection{Modern Standard pronunciations}
\subsection{Adjarian's notation in IPA}

\subsection{Transcription conventions}

Classical transliteration: Hübschmann-Meillet-Benveniste

\subsection{General consonant and vowel inventories}

\subsection{Vowel inventory}


\subsection{Typical low vowel <ա>  /ɑ/}

For the letter <ա>, most traditional transliteration systems use a simpler transcription as <a>.  But phonetically, this letter represents a low back unrounded vowel /ɑ/ in modern Standard Western and Standard Eastern Armenian. Although we don't access to articulatory or acoustic data on the Armenian dialects, we suspect that Adjarian is using <ա> to denote a back unrounded vowel as well  for the following reasons. 


First, dialectological work often distinguishes a typical vowel <ա>  against an atypical fronted form like <ա̈> /æ/. This suggests that even in 1911, Adjarian perceived <ա> as contrasting against a front <ա̈> /æ/ by being back. 

Second, in the IPA, the letter <a> represents a front vowel too. Thus, if we use both <a, æ>, then it can create a false impression that there's a phonemic contrast between two front vowels <a, æ>, instead of between a front and back vowel <æ, ɑ>. 

Third, Adjarian was often sensitive in his perception of subtle acoustic differences. For example, in some dialects, he gives subtle judgments by saying that the low vowel <ա> had a more closed mouth in the dialect of \todo{get dialect example}. This suggests that he himself felt that <ա> represented a back \textit{unrounded} vowel. 

However,  in our own fieldwork, the dialect of Tehrani Iranian Armenian has a rounded back vowel /ɒ/ due to Persian contact \citep{DolatianEtAl-prep-IranianGrammar}. Perceiving this rounding is quite subtle. So it is possible that some of the dialects that Adjarian studied from Iran did in fact use a back rounded form /ɒ/ instead of a back unrounded form /ɑ/. It is impossible to know what was the case 100 years ago.

\subsection{Low front vowel /æ/}

Some dialects developed a low front vowel, corresponding to IPA /æ/. Adjarian used a special symbol to mark this sound: 
<\armeniang{ՠ}>. This letter was not used in future work in Armenian dialectology, but was replaced by the letter <ա̈>. Our text compiler had issues in using Adjarian's original letter <\armeniang{ՠ}> and it is essentially never used in modern works. So we opted to use the Armenian letter <ա̈>. 

\subsection{Mid vowels}


ե: i̯e

է: e

classical: է ē

classical diphthongs

ո: u̯o

\subsection{Other vowels}

<էօ, իւ, ու> /ø, ʏ

 /u̯e/ <ուէ>

maragha
 /ɨ/ <ը̂>  
/ui̯/ <ուⁱ>
/əi̯/ <ըⁱ>  


ը° : upside ը: a lowered ə as ə̞
\subsection{Uvular fricatives}
For the letters <խ,ղ>, modern Standard Western and Standard Eastern use uvular /χ, ʁ/. Though in the translator's experience, velar pronunciations are possible as free variation. Dialectological work does not distinguish velar vs. uvular pronunciations. So although we transcribe these letters consistently as uvular across the dialects, it is possible that they may be more velar in some dialects than others. 

\subsection{Rhotics}
For the rhotic letters <ռ,ր>, the modern Standard Eastern pronunciation is a trill-flap distinction /r-ɾ/. Modern dialects differ in whether the transcribed <ր>  letter is truly a flap /ɾ/ vs. an approximant /ɹ,ɻ/. For example, Standard Western and Eastern Armenian use a flap /ɾ/, while Tehrani Iranian Armenian uses an approximant /ɻ/.  Dialectologists donփt distinguish these flavors of /ɾ/. But to reduce confusion with the trill, we transcribe the trill as /r/, and the non-trill rhotic as /ɾ/. 

\subsection{Other consonants}
/ʕ, ħ, q/ /   <ՙ, հՙ, ղՙ/   and /lʲ/  <լՙ> 

՚: ħ  

յ̵   : ɦ

ղՙ /q/

he doesnt transcirbe velar nasals

ըէ

էօ իւ

shamakhi
էօօ
օօ

/œə̯/ <է\`օ>
classical schwa epenthesis before Ք godel page 18

/u̯ɑ/ <ուա>... 

<ւա> = /wɑ/ or /u̯ɑ/, <աւ> = /ɑu̯/, <օւ> = /ou̯/, <էւ> = /eu̯/, <եւ> = /i̯eu̯/, and so on. 

 /ĕi̯     ,  ou̯, œu̯/  <էʲ,  օւ, էօւ>

 oə̯, ei̯, ii̯, ɑi̯/ <օը, էյ, իʲ, աʲ>

inconstent in suceava         <իւ>   read as <ի՛ու> /\'iu/   /iu̯/ \\ 
        <իեւ>   read as <իյէ՛ու> /ij\'eu/  /i̯eu̯/ \\
        <իը>   read as  <ի՛յը> /\'ijə/  /iə̯/ \\ 
        

 hes actually incosnstent in whehter a transcribed էյ should be just a vowel-glide sequence ej vs a diphthong ei̯ 


These are /ɑi̯,  ɑə̯, ɑu̯,  ĕi̯, u̯ɑ/  <աʲ,   աը, աւ, էʲ,  ուա>.


/e̞/ <է ̀ >

/ɑ̃/ <ա̄>
\section{Glossing and segmentation}
\subsection{Terminology}

գաղութ

emigre, migrant, settlement, 


New Armenian, Old Armenian


sound changes (ձայնաշրջութեամբ) 

provincial (գաւառական)


պայթուցիկ is literally stop, but he means it to refer to either stops or affricates. 





\subsection{Not Armenian}
Adjarian used Armenian script instead of Latin for Turkish. 


Wiktionary

Turkish: Tabita

Russian: Nikita

\subsection{Names of people and locations}

\section{Dialectological parameters for variation}
\subsection{Phonological parameters}

\subsubsection{Laryngeal changes or stop/affricate voicing}

\subsubsection{Palatalization of consonants}

\subsubsection{Word-initial diphthongization}

\subsubsection{Collections of individual sound changes}

\subsection{Morphophonological parameters}

\subsubsection{Vowel harmony}

\subsubsection{Affix mobility or prefix-suffix movement}

\subsubsection{Affix or morpheme repetition}

\subsubsection{Changes in theme vowels and the auxiliary}

\subsubsection{Voicing assimilation}


\subsection{Morphological parameters}

\subsubsection{Synthetic vs. periphrastic basic tenses}

\subsubsection{Archaism in the past plural ending}
\part{Introductory chapters by Adjarian}

 





\chapter{Introduction by Adjarian}




\section{History of Armenian dialectology}

\begin{adjarianpage}\label{page:1}\end{adjarianpage}% should be 1

The first person who was occupied with Armenian dialectology was the Dutch Armenologist Schröder (\armenian{Շրէօտէր}). In his work \citep{Shroder-1711-TheasauresArmenian}, he provides a separate section where he talks about various Armenian dialects in the Caucasus, and he provides a succinct sample of the Agulis and Julfa dialects. After him comes Jacques Chahan de Cirbied (\armenian{Շահան Ջրպետ}), an Armenologist from Evdokia. In his extensive grammar \citep{Cirbied-1823-GrammarArmenian}, he dedicates a part to Armenian dialects, about which he provides more information than his predecessor; and he provides a general sketch for a few of these dialects. Third place goes to the doctor Gevorg Akhverdian (\armenian{Գէորգ Ախվերդեան}; \translatorHD{SEA: /ɡevoɾkʰ ɑχveɾdjɑn/ or /ɑχveɾtʰjɑn/}). He had such a special love for Armenian dialectology that he first published the provincial songs of Sayat Nova in the Tbilisi dialect \citep{SayatNova}, providing an exhaustive introduction, where he studies the Tbilisi dialect with a very skilled and specialist pen. Akhverdian intended to study other dialects as well, but he died too soon, squashing his beautiful prospects. 

In 1866, the grammar of the Viennese monk Arsen Aydinian (\armenian{Հ. Ա. Այտընեան}; \translatorHD{SWA: /ɑɾsen ɑjdinjɑn/}) was published \citep{Aydnian-1867-GrammarArmenian}, wherein the author gave the first general classification of the Armenian dialects, though in very uncertain terms. Aydinian recognized four dialectal branches. 




\begin{adjarianpage}\label{page:2}\end{adjarianpage}% should be 2

\begin{enumerate}[noitemsep]
	\item Russia, Persia, and India
	\item Ottoman Armenia (\armenian{Տաճկահայաստան}) and Mesopotamia
	\item Asia Minor
	\item Austro-Hungary (Transylvania, Ardeal, Artyal,  or Artial: \armenian{Առտեալ})
	
\end{enumerate}

The author talks about each of the four branches separetly, provides a general description of each, focusing especially on the morphology (\armenian{ձեւաբանութիւն}). He does not consider the phonology (\armenian{ձայնաբանութիւն)}. 

In the same year, Kerovbe Patkanian (\armenian{Պատկանեան}; \translatorHD{SEA: /kʰeɾovbe pɑtkɑnjɑn/}) published a grammar in German on the Agulis dialect \citep{Patkanoff-1866-Agulis}. This was followed by the Armenologist Petermann's study on the Tbilisi dialect \citep{Petermann-1867-Tiflis}.

In 1869, Patkanian published his Russian work \citep{Patkanian-1869-RussianDialects}\footnote{\translatorHD{In the original source, Adjarian used the pre-revolution writing system and called the book: Изслҍдованіе о діалектахъ армянскаго языка.}}, where he gives a succinct description of eight Armenian dialects, based on a handful of written sources. In   1878,  the same author  published \citet{Patkanian-1875-RussianDialects}\footnote{\translatorHD{In the original source, Adjarian used the pre-revolution writing system and called the book: Матеріалы для изученія армянскихъ нарҍчій.}} in two volumes: the first volume on New Nakhichevan, and the second on Mush. Both works are extensive.

After Patkanian, there was no work on Armenian dialectology for a long time, until in 1883 when the Agulis linguist S. Sargsiants (\armenian{Ս. Սարգսեանց}; \translatorHD{SEA: /sɑɾkʰəsjɑnt͡s/}) published his detailed study on the Agulis dialect \citep{Sargiants-1883-Agulis}. His work exceeds all previous works, both in its extensiveness and its scientific accuracy. In 1886, the Polish Armenologist Jan Hanusz (\armenian{Յովհ. Հանուշ}) started a study on the Polish-Armenian dialect. He published two volumes, wherein he studies the lexicon and phonology (\armenian{ձայնաբանութիւն}) of Polish-Armenian.\footnote{\translatorHD{It's unclear to me what are the exact two volumes that Adjarian is referring to. But these two volumes are likely part of the citation to Hanusz by \citet{Martirosyan-2019-ArmenianDialectsBigVersionRussianJournal}.\label{footnote failed hanusz} (?)}} Because of his death, the remaining parts of his study of the dialect were unfinished. 

After Hanusz, the Russian Armenologist Alexander Thomson (Томсонъ; \translatorHD{modern Russian: Александр Томсон}) published a study on the Akhaltskha dialect in 1887 \citep{Thomson-1887-Karin} and the Tbilisi dialect in 1890 \citep{Thomson-1890-Tiflis}. 

Starting in 1896, Armenian dialectology gained   new momentum, and the number of studies grew day by day. In the same year, the \textit{Azgagrakan Handes} (\citetitle{AzgagrakanHandes}) was established, under the editorship of Lalayan (\armenian{Լալայեան}; \translatorHD{SEA: /lɑlɑjɑn/}). To this day, the journal continues to provide... 



\begin{adjarianpage}\label{page:3}\end{adjarianpage}% should be 3

... many samples on the provincial or vernacular language. Then in 1896, the Armenologist Melik S. Davit-Bek (\armenian{Մէլիք Ս. Դաւիթ-Բէգ}; \translatorHD{SWA: /melikʰ tʰɑvitʰ pʰekʰ/, SEA /dɑvitʰ beɡ/}) published a succinct study on the Marash dialect, first in Armenian (see \armenian{ՀԱ} 1986) and then in French.\footnote{\translatorHD{Adjarian spells this person's name in various ways. The Armenian record is \citet{DavitBek-1896-PhoneticMarash}, based on bibiliographic data from \citet{Martirosyan-2019-ArmenianDialectsBigVersionRussianJournal}. I couldn't find this person's work in French. (?)  \label{failed footnote for david bek marash}}} In 1897, Levon Mserian (\armenian{Լ. Մսերեան}; \translatorHD{SWA/SEA: /levon məseɾjɑn/, also called Mseriants \armenian{Մսերեանց} /məsəɾjɑnt͡sʰ/}) published a detailed study on the Mush dialect.\footnote{\translatorHD{It seems that Adjarian is referring to \citet{Mserianz-1899-Mush}.}} In 1898, 1899, and 1901,\footnote{\translatorHD{Adjarian is likely referring to this series: \citealt{Mseriants-1897-Part1,Mseriants-1901-Part2}.}} he published his dialectological series on the dialects of Aslanbeg, Suceava, Karabakh, and Van; the latter is in German. Then in 1898, the \citetitle{Byurakn} ethnographic periodical is established in Istanbul, which continued to exist for three years (1898-1900), providing very many samples on the Armenian dialects of various provinces, many of which were unheard of by then. If Lalayan's Ethnographic Journal had taken Byurakn's trajectory, then Armenian dialectology would have currently been in an envious position. The existence of Byurakn sadly did not last long, showing that we Armenians still do not have the capacity to keep scientific journals alive. 

In 1899, H. Kazandjian (\armenian{Յ}. \citealt{KazandjianBook}) published ``\textit{The provincial language of Evdokia}'' (\armenian{Եւդոկիոյ Հայոց գաւառաբարբառը}) in the journal \textit{Handes Amsorya} (\citetitle{HandesAmsorya}). Starting in 1900, the same journal starts publishing the extensive study of the Arapgir dialect by Melik S. Davit-Bek (\armenian{Մէլիք Ս. Դաւիթ-բէկ}). But unfortunately after a few years, the study halts. 

Special attention should be given to the editorship of the \textit{Eminian Azgagrakan Joghovatsou} (\citetitle{Eminian}), that was established in the Lazarian Institute in Moscow (\armenian{Մոսկուայի} \armenian{Լազարեան Ճեմարան}) by the will and testament of the deceased and skilled Armenologist M. Emin (\armenian{Մ. Էմին}). To this day, the journal has provided the most respectable volumes of all the dialectological works published so far. By now, it has published seven books, of which five are completely dedicated to Armenian dialects. 

Among the less known workers in Armenian dialectology is H. Nazariants (\armenian{Յ. Նազարեանց}; \translatorHD{SEA/SWA: /nɑzɑɾjɑnt͡sʰ/}). In the journal \textit{Ports} (year 5, number 2, page 150-164, \armenian{Փորձ հանդէս}; \translatorHD{SEA: /pʰoɾt͡sʰ hɑndes/}), he published an article called ``\textit{About the Armenian dialects}'' (\armenian{Հայոց բարբառների մասին}).\footnote{\translatorHD{I think this journal is the one listed on \href{https://hy.wikipedia.org/wiki/\armenian{Փորձ}_(\armenian{ամսագիր})}{Wikipedia}. Page 90 of this archived \href{https://arar.sci.am/dlibra/publication/286700/edition/263242/content}{index} lists the article by Nazariants.}} He proposed five sections, and called on Armenian folklorists to translate them to the dialect of each village and send them to him... 

\begin{adjarianpage}\label{page:4}\end{adjarianpage}% should be 4

... The man of course will of course be waiting for a long time for a hopeless endeavor. 

Let us remember lastly the German Armenologist Karst, who in his grammar on Cilician \citep{Karst-1901-MiddleArmenain} talks about the Middle period, specifically on Cilician Armenian. But whenever he explains the various words and forms, he always compares the data against the present Armenian dialects. 

\section{Shortcomings of dialectological studies}

As can be seen, Armenian dialectology is still not rich. The number of foundational and complete studies is small. Of the published sources, some are lacking in their phonology (\armenian{ձայնաբանութեամ)}, some in their morphology (\armenian{ձեւախօսական}) And when it comes to the transcription of words, they generally lack scientific accuracy. For example, see what Melik S. Davit-Bek (\armenian{Մէլիք Ս. Դաւիթ-Բէգ}) says in his Arapgir grammar (\armenian{ՀԱ} 1901, page 39): 

\begin{quote}
	By the term `usual pronunciation', we mean the Armenian pronunciation, whether it's from Yerevan, Tbilisi, Karabakh, or Van, Mush, Karin, Diyarbakır, or the Arapgir pronunciation. That is, we don't accept the so-called Western and Eastern pronunciations. Having studied a large number of the dialects of provinces of Van, Mush, Karin, Kharberd, Sebastia, and Diyarbakır, we have seen that there is no reason to accept such a decisive abyss. The main reason is that the actual populace is the only one that is entrusted with the provincial dialects and the pronunciation, whether in the Araratian provinces or in Lesser Armenia; there is one and only one pronunciation.''\footnote{\translatorHD{It's hard to translate the last few sentences because I couldn't make sense of the original; it was too convoluted. \label{footnote failed translation by Davit-Bek} (?)}}
\end{quote}


And this person is a linguist.

S. Sarsgian (\armenian{Ս. Սարգսեան}) was also a linguist, whose dialect of Agulis is a choice work. But see what he also says (Part B, page 111):

\begin{quote}
	``In the Agulis dialect, the sound /k/ <\armenian{կ}> is pronounced as hard (\armenian{կոշտ}) in nearly all positions, whether before a vowel or after. But if before the /k/ <\armenian{կ}> there is /i/ <\armenian{ի}> or /e/ <\armenian{է}> or such an /ɑ/ <\armenian{ա}> (\armenian{Ցղ. է})\footnote{\translatorHD{I don't understand the abbreviation that he uses. (?) \label{footnote failed translation abbr}}}, such that the literary form of the word uses /i/ <\armenian{ի}>, then that /k/ <\armenian{կ}> is pronounced as soft (\armenian{կակուղ}) /kʲ/ <\armenian{կյ}>. '' 
\end{quote}


But sometimes it happens that a person is confused on how to read such a form... 

\begin{adjarianpage}\label{page:5}\end{adjarianpage}% should be 5

... For example on page 39, line 6, it's written /mɑnɑk/ <\armenian{մանակ}> `alone'. On the contrary, the literary form of this word has two options: /minɑk/ <\armenian{մինակ}> or /menɑk/ <\armenian{մենակ}>. So how should we read this word: /mænæk/ <\armenian{մա̈նա̈կ}> or /mænækʲ/ <\armenian{մա̈նա̈կյ}>? Let's then be grateful for the French and English orthographies. 

It is apparent that in Armenian dialectological studies, even the best are considered incomplete and deficient.



\section{Program for dialectological studies}

In order to have a perfect study in Armenian dialectology, it should contain the following components, besides having brief geographical and statistical information on the studied dialect

\textit{Component A} - Phonetics (\armenian{Ձայնախօսութիւն}, German: \textit{Phonetik}): This section establishes the sound system of the dialect, meaning what sounds are found in the dialect, the way these sounds are articulated, their uses and number, their origins from either Armenian or from other sounds.

\textit{Component B} - Phonology (\armenian{Ձայնաբանութիւն}, German: \textit{Lautlehre}): This section provides all the rules for sounds in the dialect. One by one, it goes through the Armenian vowels, diphthongs, and consonants (\translatorHD{He means Classical Armenian sounds}); these are compared against the dialect. It establishes what Armenian letters or sounds underwent what sound changes in this dialect. Because phonology is the most important branch of linguistics, it is thus necessary that this chapter is detailed, accurate, and extensive. Each Armenian sound must be examined on its own in its position, meaning word-initially, word-medially, or word-finally, whether alone or next to a vowel or consonant. Furthermore, the provided examples must be complete, so that we can decide well the strength of the rule and the number of exceptions. 

\textit{Component C} - Morphology (\armenian{Ձեւախօսութիւն}, German: \textit{Morphologie}): This is the grammar in the conventional sense of the word. Or more accurately, this is called the etymology part of the grammar. In this section, it is necessary to give a detailed examination on the dialect's declensions, conjugations, pronouns, their form alternations, and so on. 

\textit{Component D} - Syntax (\armenian{Համաձայնութիւն}\footnote{\translatorHD{The Armenian term literally means `agreement'. But Adjarian is using it in the sense of `general syntax'.}}): This is... 

\begin{adjarianpage}\label{page:6}\end{adjarianpage}% should be 6


... an inseparable part of a grammar, which is necessary for every language. But because our dialects have not deviated from the usual agreement of literary language, it's not important to focus heavily on this part. 

At the end of every dialectological study, there must be an extensive text sample of the dialect. The text sample shows the syntactic and agreement rules of the dialect, and the use of the rules for the above components. It would be good if the manuscript used conversational data. With this, we can see how a verb is used in different tenses, numbers, and persons. 

Such are the required components for a dialectal study, so that a study is considered complete. It is also necessary to examine the circumstances under which the work can be scientifically established as accurate and complete. 

\section{Scientific alphabet}\label{sec:IntroAdjarian:scientificAlphabet}

For scientific accuracy, the first thing that we need is a scientific alphabet. This is an alphabet that we can use to show all the nuances of all the sounds of the studied dialect. For this goal, European linguistics have devised many and diverse letters for different purposes, such as symbols for lengthening or shortening, open or closed pronunciation, monophthong or diphthong, stressed or unstressed, simple or nasalized, voiced or voiceless, aspirated or unaspirated, and so on. It would be good of course if dialectology took these European symbols. But our nationalist zeal, the poverty of our publishers, and primarily the untrained eye don't allow the use of this desired point. Thus, we need a scientific alphabet that uses the Armenian letters.

Our focus is of course on a scientific alphabet, and it has no link at all with literary and current language orthographies. 

For a scientific alphabet, the required conditions are as follows:
\begin{enumerate}
	\item Each sound must be symbolized by only one symbol. 
	\item Each symbol must have only one sound. 
\end{enumerate}


For example, the sound /t͡ʃʰ/ <\armenian{չ}> is a single sound. Therefore, writing it with two or more letters (Eng. <ch>, French <tch>, ... 


\begin{adjarianpage}\label{page:7}\end{adjarianpage}% should be 7


... German <tsch>) is against the first condition. The sound /o/ <\armenian{օ}> is a single sound too. Writing this sound as <o>, <au>, or <eau> is against the first condition as well. As another example, consider the Armenian letter <\armenian{յ}>. Word-initially, this letter pronounced as /h/ <\armenian{հ}>, medially as /j/ <y>, and word-finally it's unspoken. This violates the second condition. The scientific alphabet requires that /t͡ʃʰ/ <\armenian{չ}> is written as one symbol (such as in Armenian or in the European scientific alphabet character <c>), the sound /o/ <\armenian{օ}> as just /o/ <\armenian{օ}>, and the <\armenian{յ}> letter has only one pronunciation (for us /j/), and so on. 

It speaks for itself that an un-read letter should not exist. 

Now, the 38 letters of Armenian cannot symbolize all dialectal sounds. This is especially because these letters have different pronunciations based on their position in the word, in both the Eastern and Western pronunciation systems. Thus, we need to establish once and for all what sounds they symbolize. This decision must be decisive and stable, used in all places and books. 

It is scientifically established that among the two literary languages, it is the Eastern form where the Armenian letters preserve the pronunciations of the fifth century, and the pronunciations agree with the transliteration or transcription of foreign words in Old Armenian. Because of this, our letters for stops and affricates should be based on the Eastern pronunciation (Table \ref{tab:intro:stopsaffr}).\footnote{\translatorHD{I suspect that in his transcriptions, Adjarian forgot to distinguish the two affricate series, and to distinguish aspiration. It could also have been a typo.}} 

\begin{table}[H]
	\centering
	\caption{Adjarian's transcriptions for Armenian sounds that vary between Standard Western and Standard Eastern Armenian}\label{tab:intro:stopsaffr} 
	\begin{tabular}{|lll|lll|lll|}
		\hline \multicolumn{3}{|l|}{Armenian letter} & \multicolumn{3}{l|}{Adjarian's transcription} & \multicolumn{3}{l|}{IPA letter} 
		\\ \hline 
		\armenian{բ} & \armenian{պ} & \armenian{փ} & b & p & p' & b & p & pʰ
		\\
		\armenian{գ} & \armenian{կ} & \armenian{ք} & g & k & k' & ɡ & k & kʰ
		\\
		\armenian{դ} & \armenian{տ} & \armenian{թ} & d & t & t' & d & t & tʰ
		\\
		\armenian{ձ} & \armenian{ծ} & \armenian{ց} & j & c & c & d͡z & t͡s & 
		t͡sʰ \\
		\armenian{ջ} &\armenian{ճ} &\armenian{չ} & j & c& c & d͡ʒ & t͡ʃ & 
		t͡ʃʰ
		\\ \hline 
	\end{tabular}
	
\end{table}

Among these, the first column is in the location of the second column for Western Armenian (\armenian{արեւմտեան}). The second column's sounds do not exist in Western Armenian.\footnote{\translatorHD{To clarify, he means that the letters from the first column are voiceless aspirated in Western Armenian; and that Western Armenian doesn't have phonemically voiceless unaspirated stops or affricates.}} It is hard to explain why. We should emphasize that the Western Armenian reader is not deceived by the equivalently transcribed sounds /p, k, t/  and analogous sounds. These European sounds are pronounced stronger than in Armenian, such that the Western Armenian perceives these sounds as /pʰ, kʰ, tʰ, t͡sʰ, t͡ʃʰ/ <\armenian{փ, ք, թ, ց, չ}>. However, it is not only strength... 

\begin{adjarianpage}\label{page:8}\end{adjarianpage}% should be 8

... that is perceived, but it is also the absence of voicing (the voicelessness of the sound), which is the equivalent to both the Armenian and the European. 

The following letters in Table \ref{tab:intro:othersound} are pronounced the same in Eastern and Western. Thus we don't need to give them special attention. 

\begin{table}[H]
	\centering
	\caption{Adjarian's transcriptions for Armenian sounds that don't vary between Standard Western and Standard Eastern Armenian}\label{tab:intro:othersound}
	\begin{tabular}{|lll ll|lllll|lllll|}
		\hline \multicolumn{5}{|l|}{Armenian letter} & \multicolumn{5}{l|}{Adjarian's transcription} & \multicolumn{5}{l|}{HD: IPA letter} 
		\\ \hline 
		\armenian{ա} & \armenian{է}& \armenian{ը}& \armenian{ի}& \armenian{օ} & 
		a & e & ə & i& o & 
		ɑ & e & ə & i& o 
		\\
		\armenian{ֆ} & \armenian{վ}& \armenian{ս}& \armenian{զ}& \armenian{շ} & 
		f & v & s & z & ṡ & 
		f & v& s& z& ʃ
		\\
		\armenian{ժ} & \armenian{խ} & \armenian{ղ} & \armenian{հ} & &
		ż & x & ġ & h & & 
		ʒ & χ & ʁ & h & 
		\\ 
		\armenian{լ} & \armenian{մ}& \armenian{ն}& \armenian{ր} & \armenian{ռ} & 
		l & m & n & r & ṙ & 
		l & m & n & ɾ & r
		\\ \hline 
	\end{tabular}
	
\end{table}

However for the letters <\armenian{ե, յ, ո, ւ}>, they have a complicated situation. The letter <\armenian{ե}> is pronounced as /je/ <ye> word-initially, as /e/ word-medially, and it is not found word-finally. But because /je/ is a mixture of sounds, we cannot use one symbol to symbolize it. Similarly, the sound /e/ is already symbolized by the letter <\armenian{է}.> We don't need to also represent /e/ by <\armenian{ե}>. Meaning that the letter <\armenian{ե}> is additional. Thus we should transcribe such vowels as in Table \ref{tab:frontMid}. 

\begin{table}[H]
	\centering
	\caption{Transcribing front mid vowels}\label{tab:frontMid}
	\begin{tabular}{|lll|l l|}
		\hline 
		& Trad. ortho. & Ref. ortho. & \multicolumn{2}{l|}{Transcription (SEA)}
		\\\hline 
		`yesterday' (standard) & \armenian{երէկ} & \armenian{երեկ} & \armenian{յէրէկ} & /jeɾek/ \\
		`yesterday' (dialectal) & \armenian{էրէկ} & \armenian{էրեկ} & \armenian{էրէկ} & /eɾek/ 
		\\ \hline 
	\end{tabular}
\end{table}


As we said, the letter <\armenian{յ}> is pronounced as /h/ word-initially, as /j/ word-medially, and is not pronounced word-finally. Such diversity is contrary to a scientific alphabet. For the sound /h/, we already have the letter <\armenian{հ}>. We don't need to use the letter <\armenian{յ}> for the same sound. Second, if some letter is unpronounced, then we don't need to write it. Once these two situations are removed, the <\armenian{յ}> letter ends up having only one sound /j/. And thus we read this letter in this way in both the beginning, middle, and end of words (Table \ref{tab:intr:scientificAlphabet:glideTranscribe}). The use of the letter <\armenian{յ}> for only this sound was also the situation in the fifth century. 


\begin{table}[H]
	\centering
	\caption{Transcribing the glide /j/}\label{tab:intr:scientificAlphabet:glideTranscribe}
	\begin{tabular}{|lll|l l|}
		\hline 
		& Trad. ortho. & Ref. ortho. & \multicolumn{2}{l|}{Transcription (SEA)}
		\\\hline 
		N/A & & & \armenian{յիս} & /jis/ \\
		`I' & \armenian{ես} & \armenian{ես} &\armenian{յէս} & /jes/ \\
		`Armenian' & \armenian{հայ} & \armenian{հայ}& \armenian{հայ} & /hɑj/ \\
		N/A& && \armenian{ալայ} & /ɑlɑj/
		\\ \hline 
	\end{tabular}
\end{table}


The letter <\armenian{ո}> is pronounced as /vo/ word-initially, /o/ word-medially, and is not found word-finally. Because /vo/ is a doubled sound, we shouldn't write it with one letter. For the sound /o/, we already have the letter <\armenian{օ}>, ... 
\begin{adjarianpage}\label{page:9}\end{adjarianpage}% should be 9

... thus the letter <\armenian{ո}> is excessive. The following words are written as in Table \ref{tab:intr:scientificAlphabet:vo}. 



\begin{table}[H]
	\centering
	\caption{Transcribing front mid vowels}\label{tab:intr:scientificAlphabet:vo}
	\begin{tabular}{|lll|l l|}
		\hline 
		& Trad. ortho. & Ref. ortho. & \multicolumn{2}{l|}{Transcription (SEA)}
		\\\hline 
		N/A & & & \armenian{օսկի} & /oski/ \\
		`gold' & \armenian{ոսկի} & \armenian{ոսկի} &\armenian{վօսկի} & /voski/ \\
		\hline 
	\end{tabular}
\end{table}


The letter <\armenian{ւ}> is read rightly as /v/. But the difference is that this this letter cannot start a word. We can't thus transcribe the word <\armenian{վրայ}> /vɾɑ/ `on' as <\armenian{ւրա}>. The use of the letter <\armenian{ւ}> is unneeded or excessive for the following reasons. We can't have one sound correspond to two symbols. We can't write the letter <\armenian{ւ}> word-initially. And the letter <\armenian{ւ}> is also used in the diphthongs \armenian{ու, իւ} /u, ʏ/.\footnote{\translatorHD{It is more accurate to use the term `digraph' instead of `diphthong' here, but the term `diphthong' is more faithful to Adjarian's original word \armenian{երկբարբառ} `diphthong'.}} Thus, we must write as in Table \ref{tab:intr:scientificAlphabet:v}, and not with the traditional orthography. 



\begin{table}[H]
	\centering
	\caption{Transcribing the letter <\armenian{ւ}>}\label{tab:intr:scientificAlphabet:v}
	\resizebox{\textwidth}{!}{%
	\begin{tabular}{|lll|l l|}
		\hline 
		& Trad. ortho. & Ref. ortho. & \multicolumn{2}{l|}{Transcription (SEA)}
		\\\hline 
		`bird (CA);  & \armenian{հաւ} &\armenian{հավ} & \armenian{հավ} & /hɑv/ \\
		chicken (SEA)'    & & & not <\armenian{հաւ}> & \\
		`king' & \armenian{թագաւոր} & \armenian{թագավոր} &\armenian{թաքավօր} & /tʰɑkʰɑvoɾ/ \\
		& & & not <\armenian{թագաւոր}> & \\
		`pain stone' & \armenian{ցաւաքար} & \armenian{ցավաքար} &\armenian{ցավաքար} & /t͡sʰɑvɑkʰɑɾ/ \\
		& & & not <\armenian{ցաւագար}> & \\
		\hline 
	\end{tabular}
}\end{table}




In this way, we establish the scientific value of the Armenian alphabet.

However, our dialects have sounds that the Armenian alphabet cannot explain, and for these sounds we need to create new symbols. 

When creating new symbols, we must consider two circumstances:
\begin{enumerate}
	\item Publication appropriateness, meaning we should create simple letters that aren't far off from the aesthetics and which are appropriate to the style of Armenian drawing. 
	\item The created letters should by themselves remind us what the sound is. In other words, we shouldn't create entirely novel forms, but the form should have some symbol or other formal marking that distinguishes it.
\end{enumerate}


Within Armenian dialects, the most commonly found sounds are the following.\footnote{\translatorHD{Adjarian does not use any special diacritics to denote diphthongs. Such markings are my own.}}

/æ/: This sound is between /ɑ/ <\armenian{ա}> and /e/ <\armenian{է}>, such as in the Karabakh word for /bɑn/ <\armenian{բան}> `thing'. This sound is transcribed by Sargsian (\armenian{Սարգսեան}) as an <\armenian{ա}> with two dots above it, by S. Melik Davit-Bek (\armenian{Ս. Մէլիք-Դաւիթ բէգ}) as <\armenian{ա}> with a circle on top. Both of these strategies are inappropriate. First, the fewer such markings are used, the better. Second, using this strategy makes it necessary to create a new letter. Third, experience has shown these these symbols are hard to keep on our letters; and because of their thinness, they break quickly. Fourth, when it's necessary to add stress on the sound, we end up using two or three markings next to each other. Because of these reasons, I consider the most appropriate strategy is to use a rotated <\armenian{ա}>... 

\begin{adjarianpage}\label{page:10}\end{adjarianpage}% should be 10

... This symbol was first thought of and used by the Protestant missionaries. The appropriateness of this letter is that doesn't have markings, we don't need to create a new letter, and we can add on it stress symbols.\footnote{\translatorHD{Ironically, it seems that Armenian dialectology prefers to mark the fronted vowel /æ/ using <\armenian{ա}> with two dots <\armenian{ա̈}>. This is the system that Adjarian argued against. And unfortunately, Adjarian's upside-down <\armenian{ա}> symbol was only recently given a Unicode symbol <\armeniang{{ՠ}}>, and a person has to actively download the right font so that they can even display this letter. Because of these issues, I have chosen to write Adjarian's upside-down <\armeniang{{ՠ}}> as <\armenian{ա̈}>, while I follow the IPA in using /æ/.
}}

/i̯e/: This sound is used by the villagers of Mush, Van, and Karin. We can consider this sound as a fast pronunciation of the sequence /ie/ <\armenian{իէ}>. This word is found for example in the words /mi̯eɾ/ <\armenian{մեր}> `our', /d͡zi̯eɾ/ <\armenian{ձեր}> `you.{\gen}.{\pl}'.\footnote{\translatorHD{Adjarian doesn't state the relevant dialect. (?) \label{footnote failed translation absent dialect}}} For this sound, we think it's appropriate to use the letter <\armenian{ե}>, because this sound's real source is represented and we won't need to invent a new sound.

/u̯o/: This sound is used in the same provinces. It is a fast pronunciation of a sequence /uo/ <\armenian{ուօ}>, such as in /su̯oχ/ \armenian{սոխ} `onion', /ɡu̯oʁ/ \armenian{գող} `thief'.\footnote{\translatorHD{Adjarian doesn't state the relevant dialect. (?) \label{footnote failed translation absent dialect 2}}} We represent this sound with the letter <\armenian{ո}> for the same reasons above. 

/bʰ, ɡʰ, dʰ, d͡zʰ, d͡ʒʰ/:\footnote{\translatorHD{Adjarian uses the European-based transcriptions <bh, gh, dh, jh, jh>.}} These sounds are found in many Armenian dialects as we shall later see more extensively. These are aspirated forms of the sounds /b, ɡ, d, d͡z, d͡ʒ/ <\armenian{բ,դ,գ,ձ,ջ}>. To represent these sounds, the most appropriate way is to add a reverse apostrophe: <\armenian{բՙ}, \armenian{դՙ}, \armenian{գՙ}, \armenian{ձՙ}, \armenian{ջՙ}>. 

/ɦ/:\footnote{\translatorHD{Adjarian uses an apostrophe-like symbol <\armenian{՚}>.}} This is a long glottal sound\footnote{\translatorHD{Adjarian uses the word \armenian{հագագ} which dictionaries translate as `uvular', but the definition of this word is more in line with a glottal articulation.}} The Armenians of Karin, Mush, Alashkert, and other places use this sound for the word-initial letter <\armenian{յ}>, such as in the pronunciation of the name /hɑkopʰ/ <\armenian{Յակոբ}> or /hɑɾutʰjun/ <\armenian{Յարութիւն}>. Because this sound is a type of /h/ <\armenian{յ}> sound, it is appropriate to use the symbol <\armenian{յ}̵> (the letter <\armenian{յ}> with a line through it). Although this is a new symbol, it does not need new molding because it looks like the Latin letter <f> but upside-down.

The only sound that we must inevitable mold is the small <\armenian{յ}> symbol. This has a wide use. It will be used to form the diphthongs /ɑi̯, ei̯, ii̯, oi̯/ <\armenian{այ, էյ, իյ, օյ}>, and to form palatalized sounds /ɡʲ, kʲ, kʰʲ, hʲ/ <\armenian{գյ, կյ, քյ, հյ}>.\footnote{\translatorHD{Adjarian implies that he wants this symbol <\armenian{յ}> to be a subscript. But the printed editions don't show a subscript form. It's possible that reprints of his work couldn't do a subscript <\armenian{յ}>. I don't do any subscript notation because I ultimately don't know what exact sounds he wants.}}

In a few dialects, the semivowel /w/ is found. Austrian Armenians have the diphthongs /ɑu̯, ou̯, eu̯/ and the triphthong /i̯eu̯/.\footnote{\translatorHD{Adjarian incorrectly calls /i͜e͜u/ a diphthong, perhaps because he didn't know of an Armenian translation for the word `triphthong'.}} To represent all of these, we must use the letter <\armenian{ւ}>, such that <\armenian{ւա}> = /wɑ/ or /u̯ɑ/, <\armenian{աւ}> = /ɑu̯/, <\armenian{օւ}> = /ou̯/, <\armenian{էւ}> = /eu̯/, <\armenian{եւ}> = /i̯eu̯/, and so on. 

The least inappropriate forms are <\armenian{էօ, իւ, ու}> /œ, ʏ, u/.\footnote{\translatorHD{Adjarian uses umlauted symbols <ö, ü>. But I use the conventional IPA symbols instead.}} For these sounds, we could have proposed united <\armenian{էօ, իւ, ու}> forms, ... 


\begin{adjarianpage}\label{page:11}\end{adjarianpage}% should be 11

... But because these would be impractical, we are forced to continue the old style for now.\footnote{\translatorHD{What Adjarian means is that we use a diagraph with two letters to represent some round vowels, like <\armenian{իւ}> /ʏ/. This violates   Adjarian's preference for using a  single letter, such as some sort of superscripted form like <\armenian{ի\textsuperscript{ւ}}>.}} 

Besides the above words, there are a few other rare sounds that we will see later. 

\section{Methods of studying the dialects}

There are four manners to study or investigate a dialect:
\begin{enumerate}
	\item The investigator is a local and thus knows the dialect as their mother tongue and then studies it. 
	\item The investigator is a foreigner and studies the dialect within the dialect's location. 
	\item The investigator studies the dialect but in a foreign location (not the dialect's location) by working with a person or persons who speak that dialect as a mother tongue. 
	\item The investigator does their research from written sources. 
	
\end{enumerate}

The first manner is the most desired manner. The second one is almost as good, the third one is less good, and the fourth one doesn't need anything, especially if the writer doesn't know about scientific orthographies.\footnote{\translatorHD{The original Armenian is <\armenian{իսկ չորրորդը բանի մը պէտք չէ, եթէ մանաւանդ գրողի գիտական ուղղագրութեան տեղեակ չէ}>. The first clause can mean either a compliment ``the fourth one doesn't need anything' where the phrase <\armenian{բանի մը}> `thing-{\gen} {\indf}'' is the direct object of the sentence. But the clause can also be an insult ``the fourth one is not needed by anything'' where <\armenian{պէտք}> `need' is the direct object. The following clause implies negativity, but it's unclear in total. (?) \label{footnote failed translation fourth}}}

However in every case, it's also necessary that the investigator is familiar with linguistic science and is experienced. 

How should we conduct a study of a dialect?

The most primary thing is the dictionary. Every dialect consists of the following three elements:
\begin{enumerate}
	\item Native words: These are words which descend from Old Armenian, such as <\armenian{ջուր}> /d͡ʒuɾ/ `water', <\armenian{հաց}> /hɑt͡sʰ/ `bread', <\armenian{գինի}> /ɡini/ `wine>.
	\item Provincial words: These are Armenian words that are absent from Classical Armenian, and are often newly formed words. For example <\armenian{ականակոյր}> /ɑkɑnɑkujɾ/ `very strong darkness', <\armenian{քաջքոտ}> /kʰɑd͡ʒkʰot/ `possessed by a demon', <\armenian{հրուկ}> /həɾuk/ `piece of soap', and so on. 
	\item Foreign words: These are words that were borrowed from many other languages, such as <\armenian{սամավար}> /sɑmɑvɑɾ/ `samovar', <\armenian{յօրղան}> /joɾʁɑn/ `blanket'.
\end{enumerate}

To study a dialect, the first group is the most important. By having their oldest forms in Old Armenian, there is a base line to discover the phonetic laws... \footnote{\translatorHD{It seems Adjarian is using some idiom <\armenian{հաստատուն եզր}>, literally `fixed or stable line/edge/border'. Robin Meyer suggested the translation of `base line.' (?)}} 



\begin{adjarianpage}\label{page:12}\end{adjarianpage}% should be 12

... The investigator must design a complete collection of these words. And to do so, the only way is to take an Armenian dictionary; and against each word, find the dialectal form, alongside its declension or conjugation system. In other words, we must design a dialectal Armenian dictionary. From the Armenian words, tens of thousands are lost in the dialect, so this work doesn't seem terrifying. However, we confess that this is nevertheless hard and burdensome. However, this is the only way. And the investigator will be comforted in knowing that when they are organizing the phonetics, phonology, and morphology, they will produce a complete work. This is because the investigator will be able to show us all the phonetic rules, all their examples and explanations, and also all the grammatical rules, their exceptions, and so on. 

Moving on from this general glance and program for dialectology, let us move on to detailing the present work. 

\section{Structure of the present work}

The goal of this present work is the general classification of Armenian dialects. We set the number of all the Armenian dialects as 31\footnote{\translatorHD{It's not completely clear if this number 31 is about the dialects that are studied in this book (which is 31), or the total number of dialects which Adjarian acknowledges as existing.}}, some of which have sub-dialects. We also considered it important to provide a sample text for every dialect and sub-dialect, to show the linguistic state of the dialect in practice. The samples that I personally collected are in the scientific orthography. As for the samples that I took from other sources, they generally don't have scientific accuracy. About this topic, I provide a note below each sample. 

Given the situation, when many of the 31 dialects are still unknown to science, or when they are only available from an insignificant manuscript, such a work is still premature. However, for this issue I have benefited from my own original works. 

In 1892, I started doing dialectological research for the first time. I organized first a succinct grammar of the Istanbul dialect. In 1898, I published... 
\begin{adjarianpage}\label{page:13}\end{adjarianpage}% should be 13


... a small volume on the Aslanbeg dialect, by working with a friend from Aslanbeg, Mr. Aleksan Nalpandian (\armenian{Ալէքսան Նալպանդեան}; \translatorHD{SEA: /ɑlekʰsɑn nɑlpɑndjɑn/, SWA /nɑlbɑntʰjɑn/}). This manuscript, as a first attempt, had its weaknesses when compared against the above points of our program. However, in the \citetitle{Byurakn} newspaper (1900, page 609-613), a certain Mushegh Varg (\armenian{Մուշեղ Վարգ}; \translatorHD{SEA: /muʃeʁ vɑɾɡ/ or /vɑɾkʰ/})  criticized that work, and found errors from page to page. I found it unnecessary to respond to his uncivil behavior, not only because I found his improper style unbecoming, but because he also  confuses phonetics with phonology, doesn't know what's an open /e/  closed /e/ <\armenian{է}>, and he seemed devoid of linguistic understanding. On this, I received a paper from Aslanbeg that said that a group of men were preparing to publicly condemn my study. However, being scared of its influential position, they are obliged to be satisfied with the same letter.\footnote{\translatorHD{I'm not completely sure what Adjarian means here. (?)}} 

After studying Aslanbeg, I started publishing a study on the Suceava dialect in the Venetian newspaper \citetitle{Bazmaveb} (\translatorHD{SEA: /bɑzmɑvep/, SWA: /pʰɑzmɑveb/}). I had prepared that study by working with a priest from Suceava named Father Karapet Kaynayian (\armenian{Տէր Կարապետ Կայնայեան}; \translatorHD{SEA: /teɾ kɑɾɑpet kɑjnɑjɑn/, SWA: /deɾ ɡɑɾɑbed ɡɑjnɑjɑn/}). However, the numerous typographical errors and the lack of printing caused the cancellation of my publication, and the work was left half-done.

My third study was on the Karabakh dialect, which I prepared with archimandrite Rev. Khachik Dadian (\armenian{Արժ. Խաչիկ Վրդ. Դադեան}; \translatorHD{SEA: /χɑt͡ʃʰik dɑdjɑn/, SWA: /tʰɑtʰjɑn/}), the honorable deacon of M. Babayian who was a deacon of Yesa, now archimandrite Zaven (\armenian{Շնորհ. Եսայի սակ. Մ. Բաբայեան (այժմ Զաւէն վրդ}.); \translatorHD{SEA: /zɑven/ and /bɑbɑjɑn/, SWA: /pʰɑpʰɑjɑn/}), and Mr.~Avetis Ter Harutyounian (\armenian{Պր. Աւետիս Տէր Յարութիւնեան}; \translatorHD{SEA: /ɑvetis teɾ hɑɾutʰjunjɑn/, SWA: /ɑvedis deɾ hɑɾutʰʏnjɑn/}). My work was organized based on the program that we set up above. 

Besides these, I also have many other unpublished studies. These studies are on the dialects of Agulis, Zeytun, Tbilisi, Kharberd, Karin, Hamshen, Maragha, Mush, New Nakhichevan, Vozim, Istanbul, Rodosto, Van, and Tigranakert. I have collected other information on many other dialects whether in person during my travels (Istanbul, Adapazar, Samsun, Trabzon, Baberd, Karin, Paris, Tbilisi, Etchmiadzin, Yerevan, Dilijan, Shushi, Tabriz, Baku, Batumi, New Bayazet, ... 


\begin{adjarianpage}\label{page:14}\end{adjarianpage}% should be 14

... New Nakhichevan, Rodosto) or through emigres. 

For a while, I've had the idea of creating a complete map of Armenian dialects, where every village would be categorized into a dialectal group. The French have just completed a linguistic atlas of French, which took them years to make. The whole thing forms a volume of 1750 maps, such that each word is marked in terms of what form is taken in every corner of France. We won't see such a grand undertaking even in our dreams. But it's possible to create a simple linguistic map.

With this goal, in July 1907 I started traveling. During the same time, I visited 31 Armenian villages in the New Bayazet province, except for the city where I stayed for a year. I decided the position of every village within the dialectological classification. And from each village I took a sample, as we shall see in my work. For the subsequent years, I set my mind to continue and complete my travels, as much as my life and abilities would permit. 

And thus these investigations happened, which allowed me to create the present volume, whose goal, as we said above, is the classification of Armenian dialects, their attested spread, their borders, their general characteristics, a general sketch of their phonetics, phonology, and morphology, and their characteristic borders with which a dialect differs from other dialects. Alongside my writings, there is a linguistic map of Armenian. There, I have marked only those cities and villages where Armenians exist. The language and dialect of those areas are decided or marked with colors and borders. We confess that there are many things missing that we need to fill, there many uncertain points that we might verify, and there are many errors to fix. Our book shows above all else what are the parts that need further study and where the attention of ethnographers should go. We expect in the future the completion of my work. 


\section{Differences between Old Armenian and New Armenian}

Before we go through my main work, I think that it is important... 



\begin{adjarianpage}\label{page:15}\end{adjarianpage}% should be 15

... that we explain in this introduction those differences that distinguish New Armenian from Old Armenian. Because these differences are common across almost all our dialects, then by discussing these differences, we save ourselves     extra work, and we won't need to repeat the same points for each dialect. 


The various differences between Old Armenian and New Armenian can be divided into four types: 
\begin{enumerate}
	\item phonetic differences
	\item lexical differences 
	\item morphological differences
	\item syntactic differences
\end{enumerate}


\subsection{Phonetic differences}\label{sec:IntroAdjarian:Differences:Phonetic}


\translatorHD{In this section, Adjarian first lists the various segments from Classical Armenian. He then describes how these segments underwent alternations to the modern dialects. However, Adjarian does not actually explain the phonological system of Classical Armenian. I explain Classical Armenian in \S\ref{sec:HossepIntro:phonotransc:CA}.}

\subsubsection{Segment inventory of Classical Armenian}
Old Armenian had the following 46 sounds. 
It had 7 vowels (Table \ref{tab:classicalVowel}). 

\begin{table}[H]
	\centering
	\caption{Monophthong vowels of Classical Armenian}
	\label{tab:classicalVowel}
	\begin{tabular}{|l|lllllll|}
		\hline 
		Orthography & \armenian{ա} & \armenian{ե} & \armenian{է} & \armenian{ը}& \armenian{ի} & \armenian{ո} & \armenian{ու}\\
		HMB transliteration & a & e & ē & ə & i & o & u \\
		IPA transcription & ɑ & e & ē & ə & i & o & u 
		\\ \hline
	\end{tabular}
\end{table}




Old Armenian had 9 diphthongs (Table \ref{tab:classicalDiphthong}). 


\begin{table}[H]
	\centering
	\caption{Diphthong vowels of Classical Armenian}
	\label{tab:classicalDiphthong}
	\begin{tabular}{|l|lllllllll|}
		\hline 
		Orthography & \armenian{այ} & \armenian{աւ} & \armenian{եա} & \armenian{եւ} & \armenian{եայ}& \armenian{եաւ}& \armenian{իւ} & \armenian{ոյ} & \armenian{ուա}\\
		HMB transliteration & ay & aw & ea & ew & eay& eaw& iw & oy & ua\\
		IPA transcription & ɑi̯ & ɑu̯ &e̯ɑ & eu̯ & e̯ɑi̯& e̯ɑu̯ & iu̯ & oi̯ & u̯ɑ
		\\ \hline
	\end{tabular}
\end{table}



Old Armenian had 30 consonants (Table \ref{tab:classicalConsonant}). 

\begin{table}[H]
	\centering
	\caption{Consonants of Classical Armenian}
	\label{tab:classicalConsonant}
	\begin{tabular}{|l|lllllllll|}
		\hline 
		Orthography & \armenian{բ} &\armenian{պ}& \armenian{փ} &\armenian{դ}& \armenian{տ} &\armenian{թ}& \armenian{գ}& \armenian{կ}& \armenian{ք} \\
		HMB transliteration & b &p& pʿ &d& t &tʿ& g& k& kʿ \\
		IPA transcription & b &p& pʰ &d& t &tʰ& ɡ& k& kʰ \\
		\hline 
		Orthography &\armenian{ձ}& \armenian{ծ}& \armenian{ց} &\armenian{ջ}& \armenian{ճ}& \armenian{չ} & & & \\
		HMB transliteration &j &c &cʿ& ǰ &č &čʿ & & & \\
		IPA transcription & d͡z & t͡s & t͡sʰ & d͡ʒ & t͡ʃ & t͡ʃʰ & & & \\
		\hline 
		Orthography & \armenian{վ} & \armenian{ս}& \armenian{զ}& \armenian{շ}& \armenian{ժ}& \armenian{խ} & \armenian{հ} & & \\
		HMB transliteration & v & s& z& š& ž& x & h & & \\
		IPA transcription& v & s& z& ʃ& ʒ& χ & h & & 
		\\ 
		\hline
		Orthography & \armenian{մ} & \armenian{ն} & \armenian{ր}& \armenian{ռ}& \armenian{լ}& \armenian{ղ} & \armenian{ւ} & \armenian{յ} & \\
		HMB transliteration & m & n & r & ṙ&l & ł & w & y & \\
		IPA transcription & m & n & ɾ & r& l & ł & w & j& 
		\\ \hline 
	\end{tabular}
\end{table}




\subsubsection{Sound changes from Classical to Modern Armenian}
In this phonetic system, New Armenian has introduced the following changes. 


\paragraph{Front mid vowels}\label{sec:IntroAdjarian:differences:phonetic:change:midvowelfront}

Old Armenian differentiated the Classical sounds /e,ē/ <\armenian{ե,է}> whose difference is however unclear to us. New Armenian has removed one of these two sounds, such that in many dialects (and also in the literary languages), these two sounds are rendered into one sound which we represent as /e/ <\armenian{է}>. Some of the dialects (such as Karin, Mush, Van, Suceava, etc.), differentiate the two types of /e/ <\armenian{է}> sounds in stressed syllables. They changed the Classical Armenian /ē/ <\armenian{է}> to /e/ <\armenian{է}>, while the Classical /e/ <\armenian{ե}> becomes a diphthong (\translatorHD{/i̯e/}).\footnote{\translatorHD{See \S\ref{sec:HossepIntro:translation:exp} for discussion how Adjarian uses his symbols for Classical vs. Modern Armenian.}} In unstressed syllables, both Classical /e, ē/ <\armenian{ե,է}> became /e/ <\armenian{է}>. Like some other dialects, the literary languages distinguish the reflexes of word-initial CA /e,ē/ <\armenian{ե,է}>, such that the reflex of initial CA /e/ <\armenian{ե}> is pronounced /je/ <\armenian{յէ}>, while the reflex of initial CA /ē/ <\armenian{է}> is pronounced as /e/ <\armenian{է}>. 





\begin{adjarianpage}\label{page:16}\end{adjarianpage}% should be 16

\paragraph{New vowels}


In the vowel series, some dialects have added a few new sounds. The chief among them are /æ, œ, ʏ/ <\armenian{ա̈, էօ, իւ}>. The literary language has not accepted these sounds. But the sounds /œ, ʏ/ <\armenian{էօ, իւ}> are used often in foreign words or in names (Table \ref{tab:frontRoundLiterary}).\footnote{\translatorHD{In my experience, such front round vowels only appear in borrowings for Standard Western, not Standard Eastern. Thus why Table \ref{tab:frontRoundLiterary} uses SWA instead of SEA.}} The sound /æ/ <\armenian{ա̈}> is not used in the literary languages.

\begin{table}[H]
	\centering
	\caption{Front round vowels in borrowings in literary Armenian (Standard Western Armenian)}
	\label{tab:frontRoundLiterary}
	\resizebox{\textwidth}{!}{%
	\begin{tabular}{|lll|ll|}
		\hline & &Trad. Ortho. & \multicolumn{2}{l|}{Transcription (SWA)} \\
		`Young Turk' & (French \textit{Jeune Turc}  & \armenian{Ժէօն Թիւրք} & \armenian{Ժէօն Թիւրք} & /ʒœn tʰʏɾkʰ/\\
		& or Turkish \textit{Jön Türk}) & && \\
		`Eugène Sue'& (French name) & \armenian{Էօժէն Սիւ} & \armenian{Էօժէն Սիւ} & /œʒen sʏ/
		\\ \hline 
	\end{tabular}
}\end{table}

\paragraph{Loss of Classical diphthongs}\label{sec:AdjarianIntro:difference:soundChange:DiphthongLoss}

New Armenian in contrast does not have diphthongs. The rich usage of diphthongs in Classical Armenian has wholly dissolved, becoming either vowels or vowel+consonant sequences. There are only a few dialects which have created new diphthongs. The literary language in contrast has preserved the form of the Old Armenian diphthongs, but they have been given a pronunciation which sometimes corresponds to Classical Armenian, sometimes to the present dialects, and sometimes to neither. The following is a summary of their form changes (Table \ref{tab:diphthongDiachrony}). 

\begin{table}[H]
	\centering
	\caption{Summary of diachronic changes from Classical Armenian diphthongs}
	\label{tab:diphthongDiachrony}
	
	\begin{tabular}{|ll|l l|l l|}
		\hline 
		\multicolumn{2}{|l|}{Classical Armenian} & \multicolumn{2}{l|}{Dialects} & \multicolumn{2}{l|}{Literary language} \\
		\hline 
		\armenian{այ}& ɑi̯ & ɑ, e & \armenian{ա, է} & ɑj & \armenian{այ} \\
		\armenian{աւ}& ɑu̯ & o, œ & \armenian{օ, էօ}& o & \armenian{օ} \\
		\armenian{եա}&e̯ɑ & e, i & \armenian{է, ի} & jɑ & \armenian{յա} \\
		\armenian{եւ}&eu̯ & ev, iv & \armenian{էվ, իվ} & ev & \armenian{էվ} \\
		\armenian{եայ}&e̯ɑi̯ & – & - & jɑ & \armenian{յա} \\
		\armenian{եաւ}& e̯ɑu̯ & ev, iv & \armenian{էվ, իվ} & ev & \armenian{էվ} \\
		\armenian{իւ}&iu̯ & u, ʏ & \armenian{ու, իւ} & ʏ, ju, ji & \armenian{իւ, յու, յի} \\
		\armenian{ոյ}&oi̯ & u, ʏ & \armenian{ու, իւ} & uj & \armenian{ույ} \\
		\armenian{ուա}&u̯ɑ &vɑ & \armenian{վա}&vɑ & \armenian{վա} \\\hline
	\end{tabular}
\end{table}

\paragraph{Change from CA /ɑu̯/ to modern /o/}\label{sec:IntroAdjarian:differences:phonetic:change:midvowelback}


Because the Classical diphthong /ɑu̯/ <\armenian{աւ}> became modern /o/ <\armenian{օ}>, the modern language created two types of vowels /o/ <\armenian{օ}>. One is /o/ <\armenian{օ}> from Classical /o/ <\armenian{ո}>, the other is /o/ <\armenian{օ}> from Classical /ɑu̯/ <\armenian{աւ}>. Of the dialects that distinguish the reflexes of CA /e,ē/ <\armenian{ե,է}>, they have also created a diphthong <\armenian{ո}> (read as /u̯o/ <\armenian{ուօ}>); in stressed syllables, they distinguish Classical /o/ that became modern /u̯o/ (\armenian{ո}>\armenian{ո}) from Classical /ɑu̯/ that became modern /o/ (\armenian{աւ}>\armenian{օ}).\footnote{\translatorHD{See \S\ref{sec:HossepIntro:translation:exp} for discussion how Adjarian uses his symbols for Classical vs. Modern Armenian.}} The literary language does not know of this distinction. For the literary language, the letters <\armenian{ո,օ}> have the same pronunciation /o/, and the diphthongal pronunciation of <\armenian{ո}> as /u͜o/ does not exist. The literary languages distinguishes only the word-initial letters <\armenian{ո,օ}> (just as for <\armenian{ե,է}>) with the former pronounced as /vo/ <\armenian{վօ}>, and the latter as /o/ <\armenian{օ}>. 




\begin{adjarianpage}\label{page:17}\end{adjarianpage}% should be 17

\paragraph{Laryngeal features of consonants}
Old Armenian distinguishes three degrees of plosive consonants: voiced (\armenian{թրթռուն}), voiceless unaspirated (\armenian{խուլ}), and voiceless aspirated (\armenian{թաւ}). \translatorHD{To clarify, he uses the word `plosive' here to mean a stop or affricate. }

 The voiceless aspirated series is preserved almost everywhere. But the voiced and voiceless unaspirated series have changed or exchanged places in many dialects. We will later see the details throughout my work, when each dialect is discussed in turn. Some of the dialects have introduced an entirely new series of plosives, which we can call voiced aspirated (\armenian{շնչաւոր թրթռուն}). These are the sounds /bʰ, ɡʰ, dʰ, d͡zʰ, d͡ʒʰ/ <\armenian{բՙ, գՙ, դՙ, ձՙ, ջՙ}>, which are represented in the European system as <bh, gh, dh> and so on. They originate from the Classical sounds /b, ɡ, d, d͡z, d͡ʒ/ <\armenian{բ, գ, դ, ձ, ջ}>. The literary Eastern language has in general preserved the old pronunciation of consonants. But the literary Western language has changed the voiced plosives into voiceless aspirated, while the voiceless unaspirated were changed to voiced (cf. my phonetic tables in \citealt{Adjarian-1899-ArmenianExplosives}; \translatorHD{translated in \S\ref{chapter:George}}). 

\paragraph{Changes for CA /j/ and CA /ɫ/}\label{sec:AdjarianIntro:difference:soundChange:VelarGlide}

For the other consonants, the most changes have happened to the CA sounds /j, ɫ/ <\armenian{յ, ղ}> whose pronunciations have entirely changed. The letter <\armenian{յ}> was pronounced as CA /j/ everywhere in the old language. But word-initially, almost every dialect has deleted this letter; some have turned it into /ɦ/ <\armenian{յ}̵> ; while the literary languages have turned it to /h/ <\armenian{հ}>. 

The letter <\armenian{ղ}> in the old language was some type of thick /l/ <\armenian{լ}>. But in all the dialects and in literary languages, this sound acquired its familiar guttural (\armenian{կոկորդային}) pronunciation without exception. 

\translatorHD{The letter <\armenian{ղ}>   was a velar lateral /ɫ/ in Classical Armenian.  It later became a dorsal fricative, such as the SEA and SWA /ʁ/.}

\paragraph{The sound /f/}

Old Armenian did not have the sound /f/ <\armenian{ֆ}>. The new dialects have created this sound, whether by borrowing foreign words or by native sound changes (\armenian{ձայնաշրջութեամբ}). The literary language uses this sound only in transcribing foreign words. 

\paragraph{Syncope of word medial CA /ɑ/}

In many of our dialects, especially the ones which are known as being in the Western branch, the reflex of the Classical sound /ɑ/ <\armenian{ա}> of polysyllabic words is deleted when it is not in the initial or final syllable. This sound change appears quite simply in the declension of words (Table \ref{tab:syncopeDataDecl}).\footnote{\translatorHD{For this section on syncope, Adjarian doesn't specify which modern variety of Armenian he's talking about. I assume he meant Standard Western Armenian. Note that in SWA, post-fricative stops deaspirate and there is obstruent voicing assimilation; thus a more correct transcription of /kʰɑʁkʰ-i/ is [kʰɑχk-i]. I don't modify Adjarian's original transcriptions.}}


\begin{table}[H]
	\centering
	\caption{Syncope of word medial CA /ɑ/ in declension} 
	\label{tab:syncopeDataDecl}
	\begin{tabular}{|ll| ll ll |}
		\hline & & `mouth'& & `city'& \\
		Classical & Nominative&beɾɑn & \armenian{բերան} &kʰɑɫɑkʰ & \armenian{քաղաք} 
		\\
		SWA& Nominative& pʰeɾɑn & \armenian{բերան} & kʰɑʁɑkʰ & \armenian{քաղաք} 
		\\
		& Genitive & pʰeɾn-i & \armenian{բերնի} & kʰɑχkʰ-i & \armenian{քաղքի} 
		\\
		& Intrumental & pʰeɾn-ov & \armenian{բերնով} &kʰɑχkʰ-ov & \armenian{քաղքով}
		\\ \hline 
	\end{tabular}
\end{table}

Such is the case also for the words in Table \ref{tab:syncopeDataOther}.


\begin{table}[H]
	\centering
	\caption{Syncope of word medial CA /ɑ/ in other words} 
	\label{tab:syncopeDataOther}
	\begin{tabular}{| l | ll ll |}
		\hline & `to waste' & & `wedding' & \\
		Classical & hɑtɑnil & \armenian{հատանիլ} & hɑrsɑnikʰ& \armenian{հարսանիք}
		\\
		SWA& hɑdnil & \armenian{հատնիլ} & hɑɾsɑnikʰ & \armenian{հարսանիք}
		\\
		& & & hɑɾsnikʰ & \armenian{հարսնիք}
		\\ \hline 
	\end{tabular}
\end{table}

Because of this, it often happens that two... 



\begin{adjarianpage}\label{page:18}\end{adjarianpage}% should be 18

... ... consonants become adjacent and this causes new sound changes to happen (Table \ref{tab:SyncopeFeedSoundChange}).

\begin{table}[H]
	\centering
	\caption{Medial syncope of CA /ɑ/ feeds other sound changes}
	\label{tab:SyncopeFeedSoundChange}
	\resizebox{\textwidth}{!}{%
	\begin{tabular}{| l | ll | ll | ll|}
		\hline &Classical & & > SWA && Other & \\
		`to pass' & ɑnt͡sʰɑnel & \armenian{անցանել} & ɑnt͡sʰnil & \armenian{անցնիլ} & ɑsnil & \armenian{ասնիլ}
		\\
		`to recognize' & t͡ʃɑnɑt͡ʃʰel & \armenian{ճանաչել} &d͡ʒɑnt͡ʃʰnɑl & \armenian{ճանչնալ} & t͡ʃɑʃnɑl & \armenian{ճաշնալ}
		\\
		`to button' & *kot͡ʃɑkel & *\armenian{կոճակել} &ɡod͡ʒɡel & \armenian{կոճկել} & koʒkel & \armenian{կոժկէլ} \\ \hline
	\end{tabular}
}
\end{table}

\paragraph{Rhotic metathesis}
Some words show the movement or metathesis of the Classical sound /ɾ/ <\armenian{ր}>, which is constant across all the dialects (Table \ref{tab:metathesis}).\footnote{\translatorHD{This seems like an overgeneralization. The modern word for `bridge' lacks metathesis is SWA /ɡɑmuɾt͡ʃ/ and SEA /kɑmuɾd͡ʒ/ <\armenian{կամուրջ}>.}} For this rule, \citet[241ff]{Grammont-Saussure}.



\begin{table}[H]
	\centering
	\caption{Diachronic rhotic metathesis}
	\label{tab:metathesis}
	\resizebox{\textwidth}{!}{%
	\begin{tabular}{|l|ll|lll|}
		\hline &Classical& &> Modern & & \\
		\hline `bridge' & kɑmuɾd͡ʒ & \armenian{կամուրջ} & kɑɾmund͡ʒ & \armenian{կարմունջ} & unspecified dialect 
		\\
		`carpet' & kɑpeɾt & \armenian{կապերտ} & kɑɾi̯pet & \armenian{կարպետ} & unspecified dialect 
		\\
		unclear gloss & pʰipʰeɾd & \armenian{փիփերդ} & pʰiɾpʰet & \armenian{փիրփէտ} & Karabakh 
		\\
		`clean' & *sesuɾb & *\armenian{սեսուրբ} & seɾsupʰ & \armenian{սէրսուփ} & Van 
		\\\hline 
		
	\end{tabular}
}
\end{table}

\paragraph{Nasal epenthesis}

In many places, in a word's final syllable, the nasal /n/ <\armenian{ն}> is inserted between a vowel and consonant (Table \ref{tab:nasalEpenthisis}).


\begin{table}[H]
	\centering
	\caption{Diachronic nasal epenthesis}
	\label{tab:nasalEpenthisis}
	\begin{tabular}{|l|ll|ll|}
		\hline &\multicolumn{2}{l|}{Classical Armenian}& \multicolumn{2}{l|}{> Unspecified modern variety} \\
		\hline `we' & mekʰ & \armenian{մեք} & menkʰ & \armenian{մենք}
		\\
		`green' & kɑnɑt͡ʃʰ & \armenian{կանաչ} & kɑnɑnt͡ʃʰ & \armenian{կանանչ} 
		\\
		`bridge' & kɑmuɾd͡ʒ & \armenian{կամուրջ} & kɑɾmund͡ʒ & \armenian{կարմունջ} 
		\\
		`recognition' & t͡ʃɑnɑt͡ʃʰ & \armenian{ճանաչ} & t͡ʃɑnɑnt͡ʃʰ & \armenian{ճանանչ} 
		\\
		`they' & *ɑnokʰ & *\armenian{անոք} & ɑnonkʰ & \armenian{անոնք} 
		\\\hline 
		
	\end{tabular}
\end{table} 




In these words, the insertion of the nasal /n/ <\armenian{ն}> is due to the influence of the preceding syllable's nasal /m,n/ <\armenian{մ,ն}>. In verbs, the 1{\pl} imperfective and perfective forms also show this insertion, via analogy to present verbs (Table \ref{tab:nasalEpenthisisVerb}). 


\begin{table}[H]
	\centering
	\caption{Diachronic nasal epenthesis in the past 1PL suffix with the example verb `to eat'}
	\label{tab:nasalEpenthisisVerb}
{%	\resizebox{\textwidth}{!}{%
	\begin{tabular}{|l|ll|ll|}
	\hline &\multicolumn{2}{l|}{Classical Armenian}& \multicolumn{2}{l|}{> SWA} \\
	\hline 		Indc.  Pres.  1PL & ut-e-mkʰ & \armenian{ուտեմք} & /ɡ-ud-e-nkʰ/& \armenian{կ՚ուտենք} 		   \\`we eat' & & &  [ɡ-ud-e-ŋkʰ] & \\ \hline
	Indc.   Past Impf.  1PL     & ut-ē-ɑ-kʰ & \armenian{ուտէաք} & /ɡ-ud-e-i-nkʰ/ &\armenian{կ՚ուտէինք}   
	\\`we were eating'& & &  [ɡ-ud-ej-i-ŋkʰ] &  		\\		\hline
	Past Pfv. 1PL    & keɾ-ɑ-kʰ & \armenian{կերաք}&/ɡeɾ-ɑ-nkʰ/&   \armenian{կերանք}	 \\`we ate' & &  & [ɡeɾ-ɑ-ŋkʰ]  &\\		
	\hline 
	
\end{tabular}
}\end{table} 


\translatorHD{What Adjarian means is that in Classical Armenian, the 1{\pl} suffix was /mkʰ/ for the present, but just /k/ʰ without a nasal for the past \citep[31,49]{Thomson-1989-IntroClassicalArmenian}. He argues the nasal spread via analogy.}

As for the word CA /kʰitʰ/ <\armenian{քիթ}> `nose' which became modern /kʰintʰ/ <\armenian{քինթ}>, and similar words, the insertion of the nasal /n/ is due to some unknown phonetic rule. 

\paragraph{Verbs of `to say' and `to do'}


Against the Classical words /ɑsel/ <\armenian{ասել}> `to say' and /ɑnel/ <\armenian{անել}> `to do', we often find in the new dialects words like /ɑsel/ <\armenian{ասել}> `to say' and /ɑnel/ <\armenian{անել}> `to do' (in the Eastern branch), while the Western branch has /əsel/ <\armenian{ըսել}> `to say' and /ənel/ <\armenian{ընել}> `to do'. And in this way, they have entered the literary language. 

\subsection{Lexical differences}

The lexicon of the new language has changed a lot. The largest portion of the words from Old Armenian have either been lost in the new dialects or have gained new meanings. The collection and study of this latter group of words is important for advancing the study of the history of their meanings.\footnote{\translatorHD{I think he means diachronic semantics or semantic change.}} The new dialects have also created many new words which are known under the name of provincial (\armenian{գաւառական}) words, and they do not exist in Classical Armenian. In my extensive provincial dictionary (unpublished), the number of these words is 30,000. The two literary languages have also created many new words, which are also absent from Classical Armenian. For example, SWA /ʃokʰenɑv/ \armenian{շոգենաւ} `steamboat' and /heɾɑχos/ \armenian{հեռախօս} `telephone'. The complete collection of these words is still lacking.


\begin{adjarianpage}\label{page:19}\end{adjarianpage}% should be 19

Those words that are common in both Classical Armenian and the new dialects often underwent certain sound changes which are not easy to explain with conventional phonetic laws. In many places we find words that have changed so much that it's quite hard to recognize their original form. For example, the Moks province has the word /χɑ/ <\armenian{խա}> instead of CA /het/ <\armenian{հետ}> `with, together'. The Zeytun dialect has the word /bɑju̯ob/ <\armenian{բայոբ}> instead of Classical /pɑrɑ̯u/ <\armenian{պառաւ}> `old woman'. The Hamshen dialect has the word /onluχkʰ/ <\armenian{օնլուխք}> instead of Classical /ɑnɑnuχ/ <\armenian{անանուխ}> `mint'. The number of such words is not large. 

Our dialects also have many foreign words which are borrowed from neighboring languages. The quality and quantity of these borrowings distinguishes the dialects based on their position. Among the lender languages (\translatorHD{meaning the languages which provide borrowings}), first place goes to Turkish which with its various branches (Ottoman, Azerbaijani Turkish, Tatar) has had a tremendous influence on our dialects without exception. The number of words in the Istanbul dialect that were borrowed from Turkish is 4200. For the dialects in Armenia proper, they have around only a half of this number. See \citet{Adjarian-1902-TUrkishWordsArmenian}. 

After Turkish, we have the languages of Kurdish, Georgian, Russia, and Italian. 

For words borrowed from Kurdish, the number of these words is still uncertain. These words are found in the dialects of Mush, Van, and Tigranakert. The words borrowed from Georgian are found in the dialects of Tbilisi and Artvin. The number of words borrowed from Russia is 600 in my (unpublished) collection, and they are found in in all the Russian-Armenian (\armenian{ռուսահայ}) dialects. In the New Nakhichevan dialect, these words reach the thousands. Italian borrowings are found only in the Istanbul dialect, and sometimes in the neighboring areas. There are also borrowings from Romanian, Polish, and Hungarian; these are found only in the Austro-Hungary dialect. 

The literary language does not have these lexical differences. The orthography of Old Armenian is restored almost everywhere (there are very few exceptions... 



\begin{adjarianpage}\label{page:20}\end{adjarianpage}% should be 20

... such as the words for `other': /ɑl, el/ <\armenian{ալ, էլ}> instead of /ɑjl/ <\armenian{այլ}>.) The provincial (\armenian{գաւառական}) words are in general left in the popular language (\armenian{ռամիկ լեզու}), and recently there is only a hope that they will enter the literary language.\footnote{\translatorHD{Adjarian is using flowery prose to imply that the provincial or dialectal words will likely not enter Standard Armenian.}} Foreign words are by principle excluded in our two literary languages. It is only Eastern Armenian where European scientific borrowings have some visibility.\footnote{Ter-Ghazarian \citep{DerGhazarian-DictionaryBorrowed} has collected these scientific borrowings. Their number is 1500 in that work.}

In this way, we can say that Old Armenian and the new literary languages do not have lexical differences. Our lexicon is entirely Classical, and it is significantly different from the   colloquial vernacular of the people. This is why the ordinary populace calls the literary language Classical Armenian. 


\subsection{Morphological differences}
In both the dialects and literary languages, there is a large number of morphological differences. The goal of these differences is linguistic simplification. Through the laws of analogy (\armenian{հանգիտութեան}), the most usual and regular forms of the language have been generalized, while secondary forms and exceptions have been erased.

%\translatorHD{In what follows, the subsection divisioning in this section is my own.}

\subsubsection{Declension}\label{section:introadjarian:differences:morpho:declension}


The declension of Classical Armenian, whose extreme complexity had by and large turned into a cause of difficulty, has been rendered into perfect simplicity in Modern Civil Armenian (\armenian{աշխարհաբար}). Of the many stems of Old Armenian, only one has been kept. The singular genitive-dative takes the suffix /-i/ <\armenian{ի}> and the ablative takes /-e/ <\armenian{է}> (These were unique to Classical Armenian /i/-stems <\armenian{ի}> and /ɑ/-stems <\armenian{ա}>). The instrumental takes /-ov/ <\armenian{ով}>, which was unique to the Classical /o/-stem <\armenian{ո}>. The plural has an entirely new construction. Classical Armenian formed its plurals with the suffixes /-kʰ, -t͡sʰ, -s/ <\armenian{ք, ց, ս}>, which vary based on the stem and declension. In contrast, New Armenian has two new plural suffixes which in all circumstances stay the same. These are /-eɾ/ <\armenian{եր}> for monosyllabic words, /-neɾ/ <\armenian{ներ}> for polysyllabic words. (For an explanation of these forms, see \citet[169]{Karst-1901-MiddleArmenain}, ... 



\begin{adjarianpage}\label{page:21}\end{adjarianpage}% should be 21

... \citet[456ff]{Pedersen1906Armenischundnachbarsprachen}.) The singular case markers are simply attached after these plural markers, without changing forms. It is only genitive-dative case suffix that takes the form /-u/ <\armenian{ու}> in the plural, whereas this suffix is restricted to a very small number of words in the singular. 

And this is the method of declension for the largest number of the new dialects and for the Western literary language. In a few other dialects and in the literary Eastern language, there are a few small differences. In these dialects, the ablative is formed by the new suffix /-it͡sʰ/ <-\armenian{ից}>. The plural genitive-dative case is formed the same way as in the singular, with the suffix /-i/ <\armenian{ի}>. And subsequently, there is more analogy than in the former dialects. 

It should be mentioned also that Classical prepositions /i-, j-, z-/ <\armenian{ի, յ, զ}> which were attached to various case declensions in Classical Armenian (accusative, ablative, locative (\armenian{ներգոյական}), prepositional accusative (\armenian{նախդրիւ հայցական}), narrative (\armenian{պատմական}), and circumlative (\armenian{պարառական})) have been lost in the new language.\footnote{\translatorHD{This is a slightly incorrect statement. The majority of dialects (and the standard dialects) no longer such inflectional prefixes. But a handful do, such as Mush (\S\ref{section:mush:morpho:noun:z},\ref{section:mush:morpho:noun:i}).}} In a few dialects and in the literary Eastern language, the locative is formed with the suffix /-um/ <-\armenian{ում}>. 



Table \ref{tab:declWAWEA} is a table of the declensions for the literary languages. 


\begin{table}[H]
	\caption{Declension system (plural + case) for Standard Western and Eastern Armenian}\label{tab:declWAWEA}
	
	\centering
	\resizebox{\textwidth}{!}{%
	\begin{tabular}{ |l|ll|ll|ll|ll|}
		\hline & \multicolumn{4}{l|}{The Western language, SWA} & \multicolumn{4}{l|}{The Eastern language, SEA} \\
		\hline 
		& \multicolumn{2}{l|}{Singular} & \multicolumn{2}{l|}{Plural} & \multicolumn{2}{l|}{Singular} & \multicolumn{2}{l|}{Plural} \\
		\hline 
		{\nom} & – & & /-eɾ/ & <-\armenian{եր}> & – & & /-eɾ/ & <-\armenian{եր}> \\
		& & & /-neɾ/ & <-\armenian{ներ}> & && /-neɾ/ & <-\armenian{ներ}>
		\\
		{\gen}/{\dat} & /-i/ & <-\armenian{ի}> & /-eɾ-u/ & <-\armenian{երու}> & /-i/ & <-\armenian{ի}> & /-eɾ-i/ & <-\armenian{երի}> \\
		& & & /-neɾ-u/ & <-\armenian{ներու}> & && /-neɾ-i/ & <-\armenian{ների}>
		\\
		{\acc} & \multicolumn{2}{l|}{(like {\nom})} & \multicolumn{2}{l|}{(like {\nom})} &\multicolumn{2}{l|}{(like {\nom} or {\dat})} &\multicolumn{2}{l|}{(like {\nom} or {\dat})} \\
		{\abl} & /-e/ & <-\armenian{է}> & /-eɾ-e/ & <-\armenian{երէ}> & /-it͡sʰ/ & <-\armenian{ից}> & /-eɾ-it͡sʰ/ & <-\armenian{երից}> \\ 
		& & & /-neɾ-e/ & <-\armenian{ներէ}> & && /-neɾ-it͡sʰ/ & <-\armenian{ներից}>
		\\
		
		{\ins} & /-ov/ & <-\armenian{ով}> & /-eɾ-ov/ & <-\armenian{երով}> & /-ov/ & <-\armenian{ով}> & /-eɾ-ov/ & <-\armenian{երով}> \\
		& & & /-neɾ-ov/ & <-\armenian{ներով}> & && /-neɾ-ov/ & <-\armenian{ներով}>
		\\ 
		{\locgloss} & \multicolumn{2}{l|}{(doesn't have)} & \multicolumn{2}{l|}{(doesn't have)} & /-um/ & <-\armenian{ում}> & /-eɾ-um/ & <-\armenian{երում}> \\
		& & & & & && /-neɾ-um/ & <-\armenian{ներում}>
		\\\hline
	\end{tabular}
}\end{table}

\subsubsection{Definite article}

Old Armenian had a definite article /-n/ <\armenian{ն}>,\footnote{\translatorHD{To clarify, the modern definite article is a reflex of the Classical distal demonstrative suffix. Classical Armenian did not treat the suffix /-n/ as a definite article \citep[29]{Thomson-1989-IntroClassicalArmenian}.}} but it did not have a general and regular usage. In the new language, phonetic developments created two forms: /-n/ <\armenian{ն}> which was specialized for vowel-final words, and /-ə/ <\armenian{ը}> for consonant-final words. Besides this, the language developed general and complete uses for the article, in the same way as do the new European languages (French, English, German, and so on). 




\begin{adjarianpage}\label{page:22}\end{adjarianpage}% should be 22

\subsubsection{Pronoun declension}
A few of the Old Armenian pronouns have been lost in the new language. Others have kept their old form. However, because the ablative, instrumental, and locative cases have distanced themselves from their previous state, these cases are formed in the way that nouns are, with suffixes: {\abl} /-e/ <-\armenian{է}>, {\abl} /-it͡sʰ/ <-\armenian{ից}>, {\ins} /-ov/ <-\armenian{ով}>, {\locgloss} /-um/ <-\armenian{ում}>. These suffixes are added not to the nominative form, but to the dative form.

\subsubsection{Adpositions}
All the prepositions have become postpositions. There are no prepositions in the new language.\footnote{\translatorHD{This is an incorrect overgeneralization. Modern Standard Armenian does have a handful of prepositions like SEA /ɑrɑnt͡sʰ/ <\armenian{առանց}> `without'.\label{footnote preposition in modern}}}


\subsubsection{Verb conjugations}
The morphological changes in verb conjugation are much larger. First and foremost, the fourth conjugation class (CA /-um/ <-\armenian{ում}>)\footnote{\translatorHD{He means that the theme vowel /u/ has been lost.}} has been erased, and New Armenian recognizes only three conjugations. Of the six verb forms from Old Armenian (present \armenian{ներկայ}, imperfective \armenian{անկատար}, perfective \armenian{կատարեալ}, Future <\armenian{ապառնի}>, imperative \armenian{հրամայական}, and subjunctive \armenian{ստորադասական}), only the perfective and imperative keep their old construction. The present and the imperfect have received three new constructions, which we will talk about later. The future has a composite shape and it is formed also in three new ways, in various dialects: with the formative /kə/ <\armenian{կը}>, with the formative /piti/ <\armenian{պիտի}>, or by combing the future participle (\armenian{դերբայ}) with the copular verb (\armenian{էական բայ}). The indicative present of Old Armenian has become the subjunctive present. 

In Classical Armenian, the formation of the passive was very complicated; and sometimes with the creation of simple verbs (\armenian{հասարակ բայեր}), the meanings can get confusing.\footnote{\translatorHD{I think what he means is that in Classical Armenian, verbs can  ambiguously be in the active voice or passive voice. (?)}} In place of these complications, New Armenian developed a very simple form /-vil/ <-\armenian{ուիլ}>, by which all passive verbs form one conjugation class.

In Classical Armenian, the negative (\armenian{բացասական}) had a very simple construction. And it should be thought that at least this construction has been free from general metamorphoses. But because the conjugation of verbs has entirely changed in its form, thus it is natural that the negative would follow these changes. 

The causative (\armenian{անցողական}) formative in Classical Armenian was /-et͡sʰut͡sʰɑnel/ <-\armenian{եցուցանել}>; because of its great length, it has shortened and become modern /-t͡sʰnel/ <\armenian{ցնել}>, /-t͡sʰut͡sʰel/ <\armenian{ցուցել}>, and so on. 

Let us also mention that New Armenian has created many new complex tenses, which did not exist in the old language.

\begin{adjarianpage}\label{page:23}\end{adjarianpage}% should be 23

\subsection{Syntactic differences}

In terms of syntax (\armenian{համաձայնական կողմէ}), the New Armenian dialects significantly differ from Classical Armenian. In many other cases, the literary language restored many things according to the old language; but in this case, the literary language completely follows the dialects; and the literary language rarely but sometimes diverges from the dialects, and that divergence is for  higher literary registers (\armenian{բարձր սեռերու}).%\footnote{\translatorHD{Subsection divisioning in this section is my own.}}

\subsubsection{Word order of verbs}

In Old Armenian, the verb was generally placed at the beginning of the sentence or before its arguments. In contrast, New Armenian works by putting the verb all the way at the end. Consider the following examples (\ref{sent:syntaxWordOrder}).

\begin{exe}
	\ex   \label{sent:syntaxWordOrder}
	\begin{xlist}
		
		\ex 		 `Noah and his sons entered the ark.' 
		
		\begin{xlist}
			\ex Classical Armenian \\ \glll V S ~ ~ O ~ \\
			emut noi̯ eu̯ oɾdi-kʰ noɾɑ i tɑpɑn-ən \\
			entered Noah and son-{\pl} his to ark-{\defgloss} \\
			\trans \armenian{Եմուտ Նոյ եւ որդիք նորա ի տապանն։} 
			\ex Modern Standard Western Armenian \\
			\glll S ~ ~ O V \\
			noj jev ɑnoɾ voɾtʰi-neɾ-ə dɑbɑn mədɑn \\ 
			Noah and his son-{\pl}-{\defgloss} ark entered \\
			\trans \armenian{Նոյ եւ անոր որդիները տապան մտան։}
		\end{xlist}
		
		\ex 		  `What should I do for my boy?'
		\begin{xlist}
			\ex Classical Armenian  \\ \glll O V ~ O ~ \\
			z-int͡ʃʰ ɑɾɑɾit͡sʰ vɑsən oɾdʰw-oi̯ imoi̯ \\
			{\acc}-what do for son-{\gen} my \\
			\trans \armenian{Զի՞նչ արարից վասն որդւոյ իմոյ։} 
			\ex Modern Standard Western Armenian \\ \glll O ~ O V \\
			dəʁ-u-s hɑmɑɾ int͡ʃʰ ənem \\ 
			boy-{\gen}-{\possFsg} for what do \\
			\trans \armenian{Տղուս համար ի՞նչ ընեմ։} 
		\end{xlist}
		\ex 		   `My boy, listen to your father's advice!' 
		\begin{xlist}
			\ex Classical Armenian \\ \glll V Voc O ~ ~ \\
			luɾ oɾde̯ɑk χəɾɑt-u hɑu̯ɾ kʰo \\
			listen son advice-{\dat} father.{\gen} your \\
			\trans \armenian{Լո՛ւր, որդեակ, խրատու հօր քո։} 
			\ex Modern Standard Western Armenian \\ \glll Voc ~ O ~ V \\
			dəʁɑ-s hoɾ-ətʰ χəɾɑd-ə mədiɡ əɾe \\ 
			boy-{\possFsg} father.{\gen}-{\possSsg} advice-{\defgloss} listen do \\
			\trans \armenian{Տղաս, հօրդ խրատը մտիկ ըրէ։} 
		\end{xlist}
	\end{xlist}
\end{exe}


\translatorHD{This is an overgeneralization for Classical Armenian. The default word order in Classical Armenian is SVO with free word order, while SEA/SWA is generally SOV \citep[20,33]{DumTragut-2002-WordOrderArmenian}. Though see \citet{SamvelianFaghiriKhurshudyan-2023-PersistenceSVOcaseModernEasternArmenian} on   word order problems in SEA.}

\translatorHD{Note that here and later in the translation, I gloss the Classical prefix /z-/ as an accusative marker. Though it has wide-ranging roles \citep[22]{Thomson-1989-IntroClassicalArmenian}.}


\subsubsection{Word order of genitive possessors}

In Old Armenian, the modifier word (\armenian{յատկացուցիչը}) was placed after the modified word (\armenian{յատկացեալը}). In New Armenian, the exact opposite occurs: the modifier is placed before the modified (\ref{sent:genPoss}). 

\begin{exe}
	
	\ex 		  `John's brother' 
	 \label{sent:genPoss}   \begin{xlist}
		\ex Classical Armenian \\ \glll N Poss \\
		z-eɫbɑi̯ɾ-ən jovhɑnn-u \\
		{\acc}-brother-{\dist} John-{\gen} \\
	\trans 	\armenian{Զեղբայրն Յովհաննու։}
		\ex Modern Standard Western Armenian \\ \glll Poss N \\
		ohɑnnes-i-n ɑχpɑɾ-ə \\ 
		John-{\gen}-{\defgloss} brother-{\defgloss} \\
	\trans 	\armenian{Օհաննէսին ախբարը}
	\end{xlist}
\end{exe}


\translatorHD{To clarify, Classical Armenian tends to place the gentiive possessor after the noun. But it is possible to place the possessor before the noun. See \citet[118]{DumTragut-2002-WordOrderArmenian}.}

\subsubsection{Word order of adjectives and nouns}

In Old Armenian, adjectives could be placed either before or after the noun. When the adjective is after the noun, the adjective agrees with the noun in number and case. When the adjective is before the noun, the adjective usually does not agree. Because the latter is the simplest structure, thus New Armenian always places its adjectives before the noun.

\translatorHD{For more information on adjective ordering in Classical Armenian, see \citet[75ff]{DumTragut-2002-WordOrderArmenian}.}



\subsubsection{Word order of demonstrative and possessive pronouns}

The demonstrative and possessive pronouns\footnote{\translatorHD{His Armenian term is more literally translated as `adjective', but the word `pronoun' is more technically correct.}} (\armenian{ցուցական եւ ստացական ածականները}), unlike the former (\translatorHD{meaning unlike adjectives)}, are usually placed after the noun and agree with the noun. In New Armenian, the opposite occurs: they are placed before the noun and don't agree (\ref{sent:demPoss}). 


\begin{exe}
	
	\ex  `my house, this man, to my father'  	\label{sent:demPoss}  \begin{xlist}
		
		\ex Classical Armenian \\ \glll N Poss, N Dem, N Poss\\ 
		tun im, ɑi̯ɾ-əs ɑi̯s, hɑu̯ɾ imum \\
		house my, man-{\prox} this, father.{\dat}/{\gen} my.{\dat} \\
		\trans \armenian{տուն իմ, այրս այս, հօր}
		\ex Modern Standard Western Armenian \\ \glll Poss N, Dem N, Poss N \\
		im dun-əs, ɑjs mɑɾtʰ-ə, im hoɾ-əs \\ 
		my house-{\possFsg}, this man-{\defgloss}, my father.{\dat}/{\gen}-{\possFsg} \\
		\trans  		\armenian{իմ տունս, այս մարդը, իմ հօրս}
	\end{xlist}
\end{exe} 

\translatorHD{For more information on the Classical Armenian word order, see \citet[93ff, 103ff]{DumTragut-2002-WordOrderArmenian}.}
\subsubsection{Word order of adpositions}

In Old Armenian, prepositions (\armenian{նախադրութիւններ}) were unconditionally placed before the noun. In the new language, the word `preposition' has no such meaning, because there are postpositions (\armenian{յետադրութիւն}).\footnote{\translatorHD{As said in footnote \ref{footnote preposition in modern}, this is an incorrect overgeneralization. The modern language has a handful of prepositions.}}For example (\ref{sent:Prep}). 


\begin{exe}
	
	\ex \label{sent:Prep}  \begin{xlist}
		
		\ex Classical Armenian \\ \glll P N Poss, P N ~ \\ 
		ɑrɑd͡ʒi hɑu̯ɾ imoi̯, ənd seɫɑn-ov kʰov \\ 
		front father.{\gen} my, under table-{\ins} your \\
		\trans `in front of my father, under your table' \\ 
		\armenian{առաջի հօր իմոյ, ընդ սեղանով քով}
		\ex Modern Standard Western Armenian \\ \glll N P, N P \\
		hoɾ-əs ɑɾt͡ʃev, seʁɑn-i-tʰ dɑɡ-ə \\ 
		father.{\gen}-{\possFsg} front, table-{\gen}-{\possSsg} under-{\defgloss} \\
		\trans `in front of my father, under your table' \\ 
		\armenian{հօրս առջեւ, սեղանիդ տակը}
	\end{xlist}
\end{exe} 


\translatorHD{For more information on the Classical Armenian word order, see \citet[128ff]{DumTragut-2002-WordOrderArmenian}.}

By individually taking these differences, they perhaps don't seem severe to us. But when we consider them entirely, and we compare the word order (\armenian{շարադասութիւն}) of the Modern Civil Armenian sentence... 



\begin{adjarianpage}\label{page:24}\end{adjarianpage}% should be 24

... to Classical Armenian, we shall be surprised by this great divergence that divides the two languages. 

And truly, while Old Armenian has free word order (\armenian{ազատ շարադասութեան}) like the syntax of old Indo-European languages, its analytical word order (\armenian{վերլուծական շարադասութեան}) completely follows the new European languages, such as French word order. In contrast, New Armenian lacks this syntactic freedom, and its words are placed in a stable order, just like in Turkish syntactic style, and unlike the European one. 

\translatorHD{For an overview of the differenes in word order between Classical Armenian and Modern Armenian, see \citet[\S2.3]{DumTragut-2002-WordOrderArmenian}.}


In (\ref{sent:IntroAdjarian:syntaxArmCaTuFrench}) are two sentences from Old and New Armenian, compared against French and Turkish.\footnote{
\translatorHD{For the following set of examples, I only segment the nominal inflection suffixes because those seem more important for explaining the syntax. I don't segment the rich verbal structures. For the  Turkish examples, Adjarian wrote them in the Armenian script. I converted his examples to modern Turkish.}}

\begin{exe}
	\ex\label{sent:IntroAdjarian:syntaxArmCaTuFrench}   \begin{xlist}
		
		\ex `I saw the bird that was singing on the tree.'\begin{xlist}
			\ex Classical Armenian\\ \gll 
			tesi əz-tʰərt͡ʃʰun-ən oɾ eɾɡēɾ i veɾɑi̯ t͡sɑr-oi̯-n \\
			saw {\acc}-bird-{\dist} that was.singing to on tree-{\gen}-{\dist} \\
			\trans \armenian{Տեսի զթռչունն որ երգէր ի վերայ ծառոյն։}
			\ex French \\\gll 
			J'ai vu l'oiseau qui chantait sur l'arbre \\
			I saw the.bird that was.singing   on the.tree \\
			\ex Modern Standard Western Armenian \\ \gll 
			d͡zɑɾ-i-n vəɾɑ jeɾkʰoʁ tʰəɾt͡ʃʰun-ə desɑ \\
			tree-{\gen}-{\defgloss} on singing bird-{\defgloss} saw \\
			\trans \armenian{Ծառին վրայ երգող թռչունը տեսայ}
			\ex (Ottoman) Turkish\\ \gll
			Ağac-ın üst-ün-de öten kuş-u gördüm \\
			tree-{\gen} top-{\poss}.3{\sg}-{\loc} singing bird-{\acc} saw \\

		\trans	\armenian{Աղաջըն իւսթիւնդէ էօթէօն քուշու գէօրդիւմ} \\
		Adjarian's transcription: [ɑʁɑd͡ʒən ʏstʰʏnde œtʰœn kʰuʃu ɡœɾdʏm]
		\end{xlist}
		\ex `The pages of the books of Leon, my neighbor's son' 
		\begin{xlist}
			\ex Classical Armenian \\\gll
			tʰeɾtʰ-(ə?)kʰ ɡəɾ-ot͡sʰ le{wo}n-i oɾdw-oi̯ dəɾɑt͡sʰw-oi̯ imoi̯\\
			page-{\pl} book-{\pl}.{\gen} Leon-{\gen} son-{\gen} neighbor-{\gen} my \\
			\trans \armenian{Թերթք գրոց Լեւոնի՝ որդւոյ դրացւոյ իմոյ}
			\ex French\\ \gll
			Les feuilles des livres de Leon fils de mon voisin \\
			the pages of books of Leon son of my neighbor \\
			\ex Modern Standard Western Armenian\\ \gll 
			tʰəɾɑt͡si-ji-s dəʁ-u-n levon-i-n kʰəɾkʰ-eɾ-u-n tʰeɾtʰ-eɾ-ə \\
			neighbor-{\gen}-{\possFsg} son-{\gen}-{\defgloss} Leon-{\gen}-{\defgloss} book-{\pl}-{\gen}-{\defgloss} page-{\pl}-{\defgloss}
			\\ \trans  \armenian{Դրացիիս տղուն Լեւոնին գրքերուն թերթերը}
			\ex (Ottoman) Turkish\\ \gll 
			Komşu-m-un oğl-u Levon'-un kitap-lar-ı-nın yaprak-lar-ı \\
			neighbor-{\poss}.1{\sg}-{\gen} son-{\poss}.3{\sg} Leon-{\gen} book-{\pl}-{\poss}.3{\sg}-{\gen} page-{\pl}-{\poss}.3{\sg}\\
		\trans 	\armenian{Քօնշումուն օղլու Լէօվոնըն քիթաբլարընըն յափրաքլարը} \\
			Adjarian's transcription: [kʰonʃumun oʁlu lœvonən kʰitʰɑblɑɾənən jɑpʰɾɑkʰlɑɾə]
		\end{xlist}
	\end{xlist}
\end{exe}

 


Everything is done in this way, such that you would think that New Armenian syntax is based on the Turkish template. On this investigatable issue, see Pedersen, KZ 32,472.\footnote{\translatorHD{Robin Meyer informs me that this article does not exist. Adjarian may have made a typo here.}} 




\chapter{Armenian residences}


\begin{adjarianpage}\label{page:25}\end{adjarianpage}% should be 25

 
\translatorHD{In this chapter, Adjarian provides population counts of Armenians in different regions from 1911. As Robin Meyer noticed, it's unclear where Adjarian got his numbers from. So I'm not sure how accurate or reliable they are.}

\translatorHD{I do not provided updated statistics from the 21st century. The Armenian Genocide has made it difficult to know the exact number of surviving speakers of non-standard dialects. Mass migrations have occurred after the genocide as well. For an overview, see \citet{Dekmejian-1997-ArmenianDiaspora} and the Wikipedia article on  \href{https://en.wikipedia.org/wiki/Armenian_population_by_country}{Armenian populations by country}.}


As we know, the homeland of the Armenians, Armenia, is divided today between three states. The largest portion is in the hand of the Ottomans; 7 out of 15 provinces of Old Armenia:

\begin{itemize}
	\item Upper Armenia (\armenian{Բարձր Հայք})
	\item Fourth Armenia or Sophene (\armenian{Չորրորդ Հայք})
	\item Aghdznik or Arzanene (\armenian{Աղձնիք})
	\item Turuberan (\armenian{Տուրուբերան})
	\item Moxoene or Mokk'     (\armenian{Մոկք})
	\item Korchayk or Corduene (\armenian{Կորճայք})
	\item Vaspurakan (\armenian{Վասպուրական})
\end{itemize}

A smaller portion is in the hands of the Russians: 
\begin{itemize}
	\item Artsakh (\armenian{Արցախ})
	\item Syunik (\armenian{Սիւնիք})
	\item Utik (\armenian{Ուտի})
	\item Gugark (\armenian{Գուգարք})
	\item Tayk (\armenian{Տայք})
	\item Ayrarat (\armenian{Այրարատ})
\end{itemize}

And the smallest part is in the hands of the Persians:\begin{itemize}
	\item Paytakaran (\armenian{Փայտակարան})
	\item Parskahayk or Persarmenia or Nor Shirakan (\armenian{Պարսկահայք})
\end{itemize} 

The largest portion of Armenians today are still found in their homeland. But outside of their homeland, Armenians have spread into many other countries in the following manner. 


\section{The northern migration line}
Armenian title: \armenian{Հիւսիսային գաղթնական գիծ}
\subsection{Georgia}

The city with the most Armenians is Tbilisi and its surrounding areas. But the Armenians are also scattered in other cities in Georgia, such as in the state of Tbilisi in Gori, Signagi, Telavi, Dusheti, Tianeti. In the Kutaisi province: Kutaisi, Poti, the two villages of Shorapani. In the Lechkhumi province, the village of Lailashi; in the Racha province: Oni village, Batumi, Artvin, Ardanuç, Şavşat, Sokhumi; in the Chernomorskaya province: Novorossiysk, Anapa, and the shores of the entire Black sea. The Armenian populace in this region is around 200,000. 
\subsection{Aghvank or Caucasian Albania}
The native population of this country was previously Armenian, while later a portion became  Muslim. In that way, today the native element of the country is Armenian or Turkish. The cities where Armenians live are Baku, Shamakhi (with 23 villages), Geokchay with 20 villages, Nukha (42 villages), Zagatala... 


\begin{adjarianpage}\label{page:26}\end{adjarianpage}% should be 26

... 12 villages, Agdash (6 villages), Quba (Khachmaz and Kilvar villages), and finally Derbent. The entire Armenian population of Aghvank is around 150,000 people.\footnote{\translatorHD{It's unfortunate that Armenians of modern Azerbaijan  are now either displaced, exiled, killed (ethnically cleansed), or oppressed.}}

\subsection{North Caucasus}

Here, the Armenians represent a mixture of migrants that came from different places. They live in the Dagestan area: Makhachkala, Temir-Khan-Shura, Chiri-Yurt, Ishkarty. In the Terek area: Kizlyar, Mozdok, Vladikavkaz. In the state of Stavropol: Stavropol, Machar or   Budyonnovsk (Surb Khach). In the Kuban area: Armavir, Yekaterinodar, Batalpashinsky Otdel, Yeysk, Caucasus,\footnote{\translatorHD{Adjarian must've meant some town called `Caucasus' in the Caucasus. He could've meant Kavkazskaya. (?)}} Labinsk, Maykop, Temryuk. In total, 28,835 people. 

\subsection{Tatarstan (from the Volga to the ocean)}

In this area, the Armenians are chiefly in the city of Astrakhan. But in recent years, they have spread also to farther places: Tsaritsyn, Saratov, Samara, Syzran, Simbirsky, Penza, Balashov, Uribeno, Durovka, Kamyshin, Krasnovodsk, Jibil, Chakichlar, Qızıl Arvad, Ashgabat, Artyk, Kaakhka, Dulak, Merv (Mary), Chardzhou, Petro-Aleksandrovsk, Samarkand, Bukhara, Ziyovuddin, Chernaevo, Golodnaya Steppe, Kattakurgan, Jizzakh, Khujand, Fergana, Kokand, Andijan, Osh, Namangan, Tashkent, Arys, Turkistan city, Petrovsk, and many Siberian stations. The entire number of Armenians in Tatarstan is 16,000.

\subsection{Crimea}

In its time, this place had a large Armenian population. But because of migrations in 1779, many people were scattered. Today, the Armenian-populated cities in this peninsula are Theodosia, Kerch, Alushta, Yalta, Sevastopol, Yevpatoriya, Perekop, Or or Armiansk, Simferopol, Bakhchisaray, Karasubazar, and Old Crimea. The migrants of Crimea are established in New Nakhichevan and its 5 villages, which they built. From here, they also spread to Rostov, Melitopol, Berdiansk, Azov, Novocherkassk, Nogaisk, Dnipro, Taganrog, Yekaterinoslav, and other places. The number of Armenians in this area is 35,000. 

\subsection{Russia}

Here, the Armenians are very few. The entire number is less than a thousand. A large portion are students... 
\begin{adjarianpage}\label{page:27}\end{adjarianpage}% should be 27

... and soldiers. The number of native and established people is small; they are found mainly in Moscow, Saint Petersburg, Kharkiv, Voronezh, and so on. 

\subsection{Poland}

At its time, Poland had a large Armenian populace, both in its Austrian and Russian parts. In the Russian part, there are no longer any Armenians. As for the Austrian part, the main Armenian-populated location is Kuty or Cuturi. Also, a very small number of Armenians is found in Lemberg and elsewhere. The Armenians of Kuty are around 100 houses.

\subsection{Romania}

The Armenian-populated cities are Focșani, Bucharest, Botoșani, Iași, Târgu Ocna, Galați, Brăila, Bacău, Roman, Constanța, Sulina, Tulcea, Babadag, Pitești, Giurgiu, Ploiești, and so on. The Armenian migrants consist of two specific groups. The old migrants or natives, and the new migrants who came from various corners of Ottoman Turkey after the massacres of the Ottoman Armenians. The total number of both groups is 14,000, of which 4000 people are the new migrants. 

\subsection{Bessarabia}

Here, very few Armenians are found in Chișinău, Akkerman, Khotyn, Balti, Bender, Ismail, and Hîncești, with whom we should include the Armenians of the Cherson province (Odesa and Grigoriopol).

\subsection{Austro-Hungary}

This is Bukovina, Transylvania, Hungary, and Austria proper. The Armenians of Bukovina primarily reside in the cities of Suceava, Chernivtsi, and Siret. The Armenians of Transylvania primarily live in the cities of Gherla or Armenopolis, Dumbrăveni  or Elisabethopolis, Gheorgheni, Sibviz, Brașov or Kronstadt. Small numbers of Armenians are scattered also in the various corners of Hungary, until Pest and Vienna. The total number of Armenians in this region is 15,000. 

\section{The southeastern migration line}

Armenian title: \armenian{Հարաւային-արեւելեան գաղթնական գիծ}

\subsection{Assyria}

There are Armenians only in Mosul, Kirkuk, Baghdad, Basra, and Suq al-Shuyukh. The total is 1400 people.

\subsection{Persia or Iran}


The Armenians of this country are divided into two separate  regions or  dioceses (\armenian{թեմերու}). Iranian Azerbaijan (Atropatene) and Persia proper. The Azerbaijan diocese has not only Khoy, Maku, Salmast, Urmia, and Karadagh, which are provinces of Armenia proper, ... 

\begin{adjarianpage}\label{page:28}\end{adjarianpage}% should be 28

... but also Tabriz, Mujumbar, Maragha, Kurdistan, and Ardabil. In Persia proper, the Armenian-populated cities are New Julfa (with its 80 villages), Tehran (with its 6 villages), Qazvin, Rasht, Anzali, Hamadan, Sheverin, Shiraz, Bushehr, and so on. The entire number of Persian Armenians is 66,000 of which 25,000 belong to Persian Armenia. 

\subsection{India}

It now has 700 Armenian residents who live in the cities of Kolkata, Madras, Bombay, and Dhaka. 

\subsection{Birmania or Myanmar}

The total number is 252 Armenians, of which 193 people live in Rangoon.

\subsection{Island of Java}

There are 170 Armenians, who live in Batavia (Jakarta), Surabaya, Singapore, Semarang, and so on. 

\section{The southwestern migration line}

Armenian title: \armenian{Հարաւային-արեւմտեան գաղթնական գիծ}


\subsection{Cilicia}

This has been Armenianized since the time of the Rubenid (\armenian{Ռուբինեան}) kingdom. Now, the main Armenian-populated cities are Sis, Hadjin, Zeytun, Adana, Tarsus, Mersin, Misis, Marash. They have a total of 190,000 Armenians.

\subsection{Cyprus}

There are now 562 Armenian residents, who are found mostly in the capital Nicosia. The others lives in Larnaca, Limassol, Paphos, Sourp Magar, Famagusta, and so on. 

\subsection{Syria and Lebanon}

The northern part, as bordering Cilicia, has a quite a lot of Armenians. But as we go south, the number of Armenians decreases. The total number of Armenians is 36,000 people, who live in the cities of Ayntap, Antioch, Aleppo, Beirut, Sham (Damascus), and Latakia. Ayntap has 6 villages, Antioch has 18 villages, Aleppo has 12 villages. Among these, the following villages are well-known. In Antioch: Svedia, Kessab, Aramo, and. In Aleppo: Kilis, Belen, and Jisr al-Shughur.

\subsection{Palestine}
There are 730 Armenians who live in Jerusalem, Jaffa, Bethlehem, and Ramla. 

\subsection{Egypt}

There are around 10,000 Armenians who live primarily in Alexandria and Cairo. 

\subsection{Other}

There are few Armenians who also live in Tripoli, Ethiopia, Cape Town, and Transvaal. 


\begin{adjarianpage}\label{page:29}\end{adjarianpage}% should be 29


\section{The western migration line}

Armenian title: \armenian{Արեւմտեան գաղթնական գիծ}

\subsection{Asia Minor or Anatolia}

This extends from the western borders of Armenia until the Archipelago (\armenian{Արշիպեղագոս}) and Marmara. It includes also     Lesser Armenia, which is a heavily Armenian-populated area. The main Armenian-populated cities in Asia Minor are, from east to west, Urfa, Malatya, Divriği, Akn, Arapgir, Şebinkarahisar, Gürün, Darende, Hisn-Mansur or Adıyaman, Trabzon, Gümüşhane, Giresun, Ordu, Sebastia, Evdokia, Amasia, Merzifon, Samsun, Kayseri, Yozgat, Ankara, Konya, Kastamonu, Kütahya, Afyonkarahisar, İzmir (Smyrna), Aydın, Manisa, Bursa, Bilecik, Balıkesir, Bandırma, Nicomedia, and Adapazar. The statistics of the area are still uncertain.

\subsection{Istanbul}

Taking together the villages that are on the two shores of the Bosporus or Bosphorus, there are 45 districts and 180,000 Armenians. Before the massacres, there were 250,000 Armenians, of which 60,000 were migrants. Because these people were deprived of their lands, the number of Istanbul Armenians significantly dropped. Now, it is rising again.\footnote{\translatorHD{It is ironic that the population in Istanbul was rising before the Armenian Genocide.}}

\subsection{Greece}

There are 200 Armenians who live primarily in Athens.

\subsection{Crete}


There are about the same number of Armenians   in Kandiye (Heraklion). 

\translatorHD{It's unclear what Adjarian means by this sentence. He could mean that in   Crete, there are about 200 Armenians (as in Athens). (?)}

\subsection{European Turkey}

The Armenian-populated cities are Adrianopolis, Rodosto, Malkara, Silivri, Çorlu, Gallipoli, and Thessaloniki.

\subsection{Bulgaria}

There are 15000 Armenians who live in the following cities: Varna, Ruse-Shumla, Silistra, Sofia, Tarnovo, Razgrad, Vidin, Dobrich, Teleorman, Filibe, Burgas, Tatar Pazardzhik, Sliven, Yambol, Eski Zagra, Haskovo, Aytos, Karnobat, and Straldzha. 

\subsection{France}

There are 1000 Armenians who live in Paris and Marseille, and a portion in Nancy, Montpellier, and so on.

\subsection{England}

Here, there aren't as many Armenians as in France. The Armenian-populated cities are London and Manchester. 

\subsection{The United States}

There are over 40,000 Armenians, who live primarily in Worcester, New York, Providence, Fresno, Boston, and many other cities.

\begin{adjarianpage}\label{page:30}\end{adjarianpage}% should be 30

\subsection{Other}

A small number of Armenians is also found in Italy, Switzerland, Belgium, Holland, and Germany, where there are still no migrant communities. And the resident Armenians there are only temporary immigrants.

\section{Summary}
The migrant community of Armenians is more than 1 million.

\chapter{Armenians who speak foreign languages}\label{chapter:NonSpeaking}

Although Armenian is the most widely spoken language among Armernians, there are many Armenians who have forgotten Armenian; because of the influence of the dominant languages, they have adopted foreign languages. The foreign-speaking Armenians are primarily found outside the borders of Armenia and Lesser Armenia, in various foreign countries. However, even in the extremities of Armenia, there are places where Armenian has been replaced by foreign languages. But in contrast, not all migrant Armenians have forgotten Armenian. There are many places like New Julfa, Astrakhan, Smyrna, Nicomedia, Istanbul, Suceava, and others where the Armenians speak more pure Armenian dialects than some Armenians do in Armenia proper. 

However we should emphasize the fact that anywhere where there is an Armenian (even if in Armenia proper), if the Armenian doesn't lose their mother tongue, then they know at least two languages: Armenian with either Turkish, Kurdish, Persian, or Russian. It is the female sex which falls behind in this regard and is generally more loyal to her mother tongue, than the male sex. This bilingualism of Armenians is caused by the foreign populations that coexist with the Armenians and that have an almost equal number of people as the Armenians. This bilingualism has had a significant effect on the Armenian language

The foreign languages that have been adopted by the Armenians are the following. 

%\translatorHD{Section divisioning is my own.}

\section{Turkish}

Turkish, with its two major dialects: Western Turkish or Ottoman, and Eastern Turkish or Azerbaijani. This language is spread across the following. 

\subsection{Western Asia Minor}

Almost all of Western Asia Minor, starting from around Kastamonu until Zile, south until Kayseri, and from southeast of Kayseri onto Sis and Ayntap until the Euphrates. From west of these borders until the beaches of Marmara, of the archipelago, and of ... 

\begin{adjarianpage}\label{page:31}\end{adjarianpage}% should be 31

... the Mediterranean, all the Armenians speak Turkish. Exceptions are only the Armenians in the regions of Istanbul, Nicomedia, and Smyrna, as well a number of the villages in Ankara and Aydın. These are specifically Stanoz (Yenikent), Nallıhan, Sivrihisar, Ödemiş, and Burdur. I have heard that a few of the villages in Yozgat are also Armenian-speaking, but their names aren't clear to me. 

\subsection{Asia Minor}
In Niksar, at the northeast side of Evdokia, there's an islet of Turkish-speaking Armenians due to the beastly barbarity of the many resident Turks.
\subsection{Island of Cyprus}

The old Armenian migrant community is Turkish-speaking, but the new migrant community is Armenian-speaking. 
\subsection{European Turkey, Bulgaria, and Eastern Rumelia}
Another region of Turkish-speaking Armenians is also European Turkey, Bulgaria, and Eastern Rumelia, starting from the other side of the Marmara. Exceptions are only Rodosto and Malkara. The other Armenian-populated cities, such as Gallipoli, Silivri, Çorlu, Ereğli, Çatalca, Adrianopolis, Dimetoka, Gyumyurdjina, and Dedeağaç, are Turkish-speaking. The old migrant communities of Bulgaria and Eastern Rumelia are entirely Turkish-speaking; but after the last Ottoman-Armenian massacres, the presence of a large number of asylum-seeking Armenians caused the restoration of the forgotten Armenian language, of course only in those cities where a large number of them were relocated, such as in Filibe, Burgas, Varna, Tarnovo, Ruse, and so on. The other cities remain Turkish-speaking, such as Silistra, Razgrad, Shumla, Sliven, Aytos, Karnobat, Yambol, Eski Zagra, and Haskovo. 
\subsection{Romania}
There are Romanian Armenian-populated cities that were previously settlements from Bulgaria, such as Babadag, Tulcea, Sulina. Here as well, the Armenians who fled the massacres have restored the Armenian language, such as in Galați, Ibraila, and Constanța. 

\subsection{Bessarabia}

Bessarabia is Turkish-speaking because it was previously part of Romania. Such as Ismail, Balti, Bender, Chișinău, Akkerman. Similarly the Armenian migrants of Bessarabia are Turkish-speaking, such as Grigoriopol, Odesa, and Cherson.

\subsection{Lazistan}
On the eastern side of Trabzon, there are Armenians found in Lazistan, who are scattered among the Turks and the Laz.

\subsection{West of Akhalkalak}

On the western side of Akhalkalak, there are four villages which are Bavra, ... 



\begin{adjarianpage}\label{page:32}\end{adjarianpage}% should be 32

... Khulgumo, Kartikami , and Turtskh; they are Turkish-speaking.

\subsection{Olti}
In the region of Olti, 45 verst (\translatorHD{almost 48km}) away from Olti, there is the Turkish-speaking village of Kalkos (25 houses).

\subsection{Urmia}
On the northern banks of Lake Urmia, especially in Savaj Bolagh and Miandoab, or in fewer words Persian Kurdistan, the small Armenian community is Turkish-speaking. 

\subsection{Summary}

As can be seen, the Turkish-speaking Armenians make up a significant number. But thankfully, this number decreases day by day. In all the major cities of Anatolia, such as in Bursa, Kayseri, and Yozgat, the new generation has become Armenian-speaking thanks to schools and because of immigration to Istanbul. A large portion of the Armenians in Cyprus, Eastern Rumelia, and Bulgaria have become Armenian-speaking thanks to the new migrants. The Ottoman government in its time used violent means or force to erase the Armenian language and to make Turkish   the dominant language (such as can be said for how the Pashas in Anatolia killed the language of Armenian-speaking Armenians), but currently it has no intention nor ability to do  so.\footnote{\translatorHD{It is quite sad that Adjarian's optimism was soon disproven by the Armenian Genocide.}} In Bessarabia, instead of Turkish, Russian is now wide-spread. The entire population already knows Russian, and we only need a short amount of time before Turkish is completely lost. 

\section{Georgian}

This language is spoken by almost all the Georgian-Armenians. Exceptions are Tbilisi and the cities on the shores of the Black Sea, such as Batumi, Poti, Sokhumi, and so on. The Armenians are Georgian-speaking in Sighnag, Telavi, Gori, Kutaisi, and the neighoring areas. Two of the villages of Akhalkalak are also Georgian-speaking: Vargavi and Khizabavra . The Armenians of Vladikavkaz are also Georgian-speaking, because a large portion of them have emigrated from Georgia. 

\section{Persian}

It is spoken on a very small border, between Mədrəsə (close to Shamakhi) and Kilvar (close to Quba) and in the villages of Khachmaz. Vardapet Makar Barkhudariants (\armenian{Մակար վրդ. Բարխուդարեանց}; \translatorHD{SEA: /mɑkɑɾ bɑɾχudɑɾjɑnt͡sʰ/, SWA: /mɑɡɑɾ pʰɑɾχutʰɑɾjɑnt͡sʰ/}) and bishop Mesrob Smbatian (\armenian{Մեսրոպ եպս. Սմբատեան}; \translatorHD{SEA: /mesɾop səmbɑtjɑn/, SWA: /mesɾob səmpʰɑtʰjɑn/}) have said in their topographies that the language of these villages is called Lahij (\armenian{լահճերէն}) and Tat (\armenian{թաթերէն}). But we shouldn't be confused by these names, because this language is a very clear and easy-to-understand dialect of Persian.


\begin{adjarianpage}\label{page:33}\end{adjarianpage}% should be 33



\section{Circassian}

Circassian (\armenian{չերքէզերէն}) is spoken only in Armavir, where there is an Armenian-populated village in the Kuban region. The Armenians of Armavir migrated from Circassia and founded this village in 1830.



\section{Kurdish}

In   Northern Armenia, Kurdish is a widely-spread language. But it has become the mother tongue at a small border. That is the villages of Hizan, the provinces of Ğarzan and Shirvan in Paghesh province (\armenian{կուստկալութիւն}); in the Tigranakert province (\armenian{կուստկալութիւն}), in the provinces of Meyafarikîn or Silvan, Beşiri, Bohtan; Samsat (formerly Samosata) in Cilicia. The total number is over 100 villages. 

\section{Arabic}\label{sec:Languages:Arabic}

It has become the mother tongue of the Armenians in Syria, Palestine, Mesopotamia, and Assyria. The Armenians in Aleppo, Damascus, Beirut, Mardin, Mosul, Kirkuk, and also Siirt in Armenia are Arabic-speaking. 

\section{Romanian}

This has become the mother tongue of the majority of Romanian-Armenian migrant community, and part of the Armenians in Bukovina. There are Turkish-speaking Armenians only at the eastern seashores of Romania until Galați; some of these people are Armenian-speaking thanks to the recent Armenian migrants. 

\section{Polish}

This is spread almost everywhere among the Polish-Armenians, except for Cuturi which is Armenian-speaking. The Armenians of Poland can be considered already  lost as a nation.\footnote{\translatorHD{Adjarian's original phrasing is \armenian{ազգովին կորսուած} which suggests that the Armenian community in Poland 1911 has assimilated to the local Polish population.}}

\section{Hungarian}

It is spoken as a mother tongue among the entirety of Armenians in Hungary and Transylvania. Except for the cities of Szamosújvár or Armenopolis and Gherla or Elisabethopolis, which are Armenian-speaking.


\section{English}

This is spoken in the Indian-Armenian migrant communities, whereas the Armenians of England are still a recent settlement so they're Armenian-speaking. 

\section{Summary}
The extent and borders of these languages are all accurately represented in the map that is placed at the end of this book. 




\chapter{The three branches of Armenian dialects}\label{chapter:Branches}

\section{Overview}\label{sec:ThreeBranch:overview}

\begin{adjarianpage}\label{page:34}\end{adjarianpage}% should be 34



In general among us, the Armenian dialects are divided into two branches: Eastern or Russian-Armenian dialects, and Western or Ottoman-Armenian dialects. For me, these terms are wrong and inappropriate, even though they are accepted and used everywhere. Calling the dialects Eastern or Western is wrong because many of the dialects that are called such are found at longitudinally equal degrees, yet when we compare them to each other, they don't fall either West or East. For example, the Van dialect and the Bayazit sub-dialect are both found longitudinally at the 44th degree, but the former is called Western while the latter Eastern. There are more surprising cases. For example, Artvin is much more west than Akhalkalak and Alexandropol (Gyumri); but based on the above division, Artvin is called Eastern, while Akhalkalak and Alexandropol are considered Western vernaculars. 

The names ``Russian-Armenian dialects'' and ``Ottoman-Armenian dialects'' are strange and in reality completely inappropriate. Many of the Armenians in Russia speak the Ottoman-Armenian dialects; just as there are Armenians in Ottoman Turkey that speak Russian-Armenian dialects. For example, in Russia,   Ottoman-Armenian dialects are spoken in New Nakhichevan, the Crimean peninsula, Sokhumi, Batumi,    Akhalkalak, Akhaltskha, Alexandropol, Kars, and the villages of New Bayazet. Similarly in Ottoman Turkey, Russian-Armenian dialects are spoken in Bayazit, Burdur, Ödemiş. Besides that, the migrant communities of Persian-Armenians, Bulgarian-Armenians, Romanian-Armenians, Egyptian-Armenians, and American-Armenians are ignored; and they are inappropriately called Russian-Armenians or Ottoman-Armenians. 

I propose here new terms which not only remove the aforementioned inconveniences, but they also have the benefit of incorporating the primary characteristic of the dialects that they describe. These terms are:


\begin{adjarianpage}\label{page:35}\end{adjarianpage}% should be 35

\begin{itemize}
	\item /um/ <\armenian{ում}> branch: With this name, we mean all the dialects that are called Eastern or Russian-Armenian.
	\item /kə/ <\armenian{կը}> branch: With this name, we mean all the dialects that are called Western or Ottoman-Armenian.
	
\end{itemize}

For the dialects of the /um/ <\armenian{ում}> branch, the locative case   (as well as the present and imperfective tenses) are made with the formative /-um/ <\armenian{ում}>. This is the main characteristic of these dialects; thus we give them this name. As for the dialects of the /kə/ <\armenian{կը}> branch, they do not have a locative case, they don't have a formative /-um/ <\armenian{ում}>, and the present and imperfective tenses are formed with the formative /kə/ <\armenian{կը}>. This is their primary characteristic, and thus they get this name. 

But besides these two, there is also a third branch which has dialects that have neither the /um/ <\armenian{ում}> nor /kə/ <\armenian{կը}> particles. They form the present and imperfective tenses using either the infinitive or some invisible means, and in combination with the /em/ <\armenian{եմ}> copular verb. Among our dialects, this branch is not generally distinguished and is it appended to the /um/ <\armenian{ում}> branch. We propose using the name /el/ <\armenian{ել}> branch. 

There is no confusion in our division, and the new terminology applies only to the dialects, and they don't have anything to do with the literary languages. For them, the term Eastern and Western, or Russian-Armenian and Ottoman-Armenian are still appropriate names, because the former language is centered in Tbilisi while the latter in Istanbul. 


\begin{adjarianpage}\label{page:36}\end{adjarianpage}% should be 36

\section{Terminology}
\translatorHD{This was originally a footnote on page 36. But it is quite important and stands out. So I made it its own section.}

Against the European word ``dialecte,''   we use the terms \armenian{բարբառ} `dialect', \armenian{գաւառաբարբար} `provincial dialect', and \armenian{գաւառական} `vernacular, provincial'. Because every word in the scientific language must be certain, we must thus decide on the  use of these words. The word \armenian{գաւառաբարբառ} `provincial dialect' is alien and the wrong word. It is alien because of its length; and because it already contains the word \armenian{բարբառ} `dialect', it doesn't add anything. It is wrong because a dialect has no connection to a province, and the dialect could be spoken not in the entire province but merely in a single village or city. For example, the Agulis dialect is not spoken in an entire province, but only in a small circle of villages. Similarly, the Istanbul dialect does not encompass an entire province, but only the city of Istanbul. Thus, it is preferable to use the word \armenian{բարբառ} `dialect'; it is shorter and more normal. 

A dialect can have some secondary branches that are slightly different from it; these are referred to by the European word ``sous-dialecte''. In this place, we use the Armenian word \armenian{ենթաբարբառ} `subdialect'. 

Subdialects also contain many groups, which are called in French ``parler''. For this, we use the word \armenian{գաւառական} `vernacular'. We also use this term in those situations where we cannot with certainty assign the spoken language of some place to a rank. We also use the term when we are enumerating dialects, subdialects, and vernaculars. In other words, the word \armenian{գաւառական} `vernacular' also has the general meaning of a non-literary language. 


\section{Excluded communities}\label{sec:Branches:excluded}

\translatorHD{This was originally a note on page \ref{page:293}. I moved it here because it's more relevant here.}

The Armenian settlements of Bulgaria, Rumania, Greece, France, England, Egypt, and America are newly formed, and are a mixture of Armenians from diverse places. They don't have a proper dialect, so they are not part of our present work. 



\part{The /el/ <\armenian{ել}> branch}

The /el/ <\armenian{ել}> branch has 3 dialects:

\begin{enumerate}
	\item Dialect of Maragha (\S\ref{chapter:Maragha})
	\item Dialect of Khoy (\S\ref{chapter:Khoy})
	\item Dialect of Artvin (\S\ref{chapter:Artvin})
	
	
\end{enumerate}


\chapter{Maragha}\label{chapter:Maragha}
\section{Background}

\begin{adjarianpage}\label{page:281}\end{adjarianpage}% should be 281

The dialect of Maragha is spoken on the two sides of Lake Urmia. The eastern side is found in the city of Maragha, while the western side is the city of Urmia, with its group of Armenian villages, a portion of which are Turkish-speaking. For this very interesting dialect, there is no published study or even a line from a published manuscript. During my time in Persia, I studied it, with two adult students from Maragha: Petros Hayrapetian (\armenian{ՊՊ. Պետրոս Հայրապետեան}; \translatorHD{SEA: /petɾos hɑɾɑpetjɑn/}) and Grigor Mnatsakanian (\armenian{Գրիգոր Մնացականեան}; \translatorHD{SEA: /ɡ(ə)ɾikʰoɾ mənɑt͡sʰɑkɑnjɑn/}). I present here a summary of my unpublished research. 

\translatorHD{For more recent work on this dialect, see \citet[221]{Martirosyan-2019-ArmenianDialectsBigVersionRussianJournal}.}

\section{Phonology}
\subsection{Segment inventory}

The sound system of the Maragha dialect is very rich in vowels and diphthongs (in total 14)\footnote{\translatorHD{The original page had 13. But on page 306, Adjarian provides an erratum that it should be 14. He likewise says that we should include the vowel <\armenian{է}>. I fixed it.}} in Table \ref{tab:Maragha:phono:seg:vowel}.

\begin{table}[H]
	\centering
	\caption{Vowel inventory of the Maragha dialect}
	\label{tab:Maragha:phono:seg:vowel}
	\begin{tabular}{|llll|}
\hline		/i/ <\armenian{ի}> & /ʏ/ <\armenian{իւ}> & /ui̯/ <\armenian{ուⁱ}> & /u/ <\armenian{ու}> \\
		/œ/ <\armenian{էօ}> & /e/ <\armenian{է}> & /o/ <\armenian{օ}> & /u̯e/ <\armenian{ուէ}>
		\\
		/əi̯/ <\armenian{ըⁱ}> & /ə̟/ <\armenian{ըէ}> & /ə/ <\armenian{ը}> & /ɨ/ <\armenian{ը̂}> \\
		/æ/ <\armenian{ա̈}> & & & /ɑ/ <\armenian{ա}>
		\\\hline
	\end{tabular}
	
\end{table}

\translatorHD{For the sounds <\armenian{ուⁱ}> and <\armenian{ըⁱ}>, Adjarian used a superscript <\armenian{ի}> /i/: <\armenian{ու}\textsuperscript{\armenian{ի}}, \armenian{ը}\textsuperscript{\armenian{ի}}. But because that could cause problems with my type-setter, I replaced them with a superscript <i>.}


The consonants are likewise rich with some new sounds in Table \ref{tab:Maragha:phono:segment:cons}. 



\begin{table}[H]
	\centering
	\caption{Consonants of the Maragha dialect}
	\label{tab:Maragha:phono:segment:cons}
	\resizebox{\textwidth}{!}{%
	\begin{tabular}{|l|lll|llll|lll|}
		\hline 
		& \multicolumn{3}{l|}{Labial}& \multicolumn{4}{l|}{Coronal}& \multicolumn{3}{l|}{Dorsal/Back}\\
		Stops& /b/ & /p/ & /pʰ/ & /d/ & /t/ & /tʰ/& & /ɡ/ & /k/ & /kʰ/ 
		\\
		& <\armenian{բ}> &<\armenian{պ}>& <\armenian{փ}> &<\armenian{դ}>& <\armenian{տ}> &<\armenian{թ}>&& <\armenian{գ}>& <\armenian{կ}>& <\armenian{ք}>\\
		& & & & & & && /ɡʲ/ & /kʲ/ & /kʰʲ/ \\
		& & & && & && <\armenian{գյ}>& <\armenian{կյ}>& <\armenian{քյ}>\\
		\hline 
		Affricates & && & /d͡z/ & /t͡s/ & /t͡sʰ/ & && & \\
		& && &<\armenian{ձ}>& <\armenian{ծ}>& <\armenian{ց}> & & & & \\
		& && & /d͡ʒ/ & /t͡ʃ/ & / t͡ʃʰ/ && & & \\
		& & & &<\armenian{ջ}>& <\armenian{ճ}>& <\armenian{չ}> & & && \\
		\hline 
		Fricatives& /f/&/v/& &/s/& /z/& /ʃ/& /ʒ/& /χ/ & /ʁ/ & /h/ \\
		& <\armenian{ֆ}>&<\armenian{վ}>& & <\armenian{ս}>& <\armenian{զ}>& <\armenian{շ}>& <\armenian{ժ}>& <\armenian{խ}> & <\armenian{ղ}> & <\armenian{հ}> \\
		&& & & & & & & & & /ç/ \\
		&& & & & & & & & & <\armenian{հյ}>
		\\ \hline 
		Sonorants & /m/ & /n/& & /ɾ/ & /r/& /l/ & /j/ &/w/& & \\
		& <\armenian{մ}> & <\armenian{ն}> && <\armenian{ր}>& <\armenian{ռ}>& <\armenian{լ}>& <\armenian{յ}> &<\armenian{ւ}>& & 
		\\ \hline 
	\end{tabular}
}
\end{table}


For these sounds, it is worth giving a separate explanation for the following. The sound <\armenian{ը̂}> represents the Russian sound <ы>, meaning a sound /ə/ <\armenian{ը}> that is pronounced voiceless and closed. 

The sounds <\armenian{ըⁱ}, \armenian{ուⁱ}, \armenian{ըէ, ուէ}> represent approximately the sounds /əi, ui, əi, ue/ in fast pronunciation. 

\translatorHD{Based on this description, and to maintain consistency with previous uses of <\armenian{ըէ, ուէ}>, I use a diphthong notation with < ̯> for all but <\armenian{ըէ}>: /əi̯, ui̯, ə̟, u̯e/.}

The consonants, as can be seen, have three degrees: voiced, voiceless unaspirated, voiceless aspirated. The dialect recognizes also the palatal sounds /ɡʲ, kʲ, kʰʲ, ç/ <\armenian{գյ, կյ, քյ, հյ}>, and the semi-sound /w/ <\armenian{ւ}> which is pronounced like the English letter <w>. 

\subsection{Sound changes}
\subsubsection{Monophthong vowel changes}
For the vowel changes, the following are notable.

\paragraph{Classical Armenian /ɑ/ <\armenian{ա}>}
Classical Armenian /ɑ/ <\armenian{ա}> remains /ɑ/ <\armenian{ա}> or /æ/ <\armenian{ա̈}>.

\paragraph{Classical Armenian /e/ <\armenian{ե}>}
Classical Armenian /e/ <\armenian{ե}> became /je/ <\armenian{յէ}> (at the beginning of monosyllabic words), /e/ <\armenian{է}> (at the beginning of polysyllabic words), while word-medially it is /e, ə̟, i/ <\armenian{է, ըէ, ի}>.





\begin{adjarianpage}\label{page:282}\end{adjarianpage}% should be 282


\paragraph{Classical Armenian /i/ <\armenian{ի}>}
Classical Armenian /i/ <\armenian{ի}> became /i, əi̯, ə/ <\armenian{ի, ըⁱ}, \armenian{ը}> (Table \ref{tab:Maragha:phonology:soundChange:monoph:i}).\footnote{\translatorHD{The original page had <\armenian{է}\textsuperscript{\armenian{ի}}>. But on page 306, Adjarian provided an erratum that this should be <\armenian{ըի}>, which I think was itself a typo for <\armenian{ը}\textsuperscript{\armenian{ի}}> or /\armenian{ըⁱ}/. I fixed it.}}





\begin{table}[H]
	\centering
	\caption{Change from Classical Armenian /i/ <\armenian{ի}> to /i, əi̯, ə/ <\armenian{ի, ըⁱ}, \armenian{ը}> in the Maragha dialect}
	\label{tab:Maragha:phonology:soundChange:monoph:i}
	\begin{tabular}{|l| ll|ll| ll|}
		\hline & \multicolumn{2}{l|}{Classical Armenian} &\multicolumn{2}{l|}{> Maragha} & \multicolumn{2}{l|}{cf. SEA} \\ 
		`barley' &ɡɑɾi & \armenian{գարի} &kʲæɾə & \armenian{կյա̈րը} &ɡɑɾi & \armenian{գարի} \\
		`scholar' &dəpiɾ & \armenian{դպիր} &təpəi̯ɾ & \armenian{տըպըⁱ}\armenian{ր} &dəpiɾ & \armenian{դպիր} \\
		\hline 
	\end{tabular}
\end{table}

\paragraph{Classical Armenian /o/ <\armenian{ո}>}

Classical Armenian /o/ <\armenian{ո}> became /vəi̯/ <\armenian{վըⁱ}> word-initially (Table \ref{tab:Maragha:phonology:soundChange:monoph:oinit}). 



\begin{table}[H]
	\centering
	\caption{Change from Classical Armenian /o/ <\armenian{ո}> to /vəi̯/ <\armenian{վըⁱ}> in the Maragha dialect}
	\label{tab:Maragha:phonology:soundChange:monoph:oinit}
	\begin{tabular}{|l| ll|ll| ll|}
		\hline & \multicolumn{2}{l|}{Classical Armenian} &\multicolumn{2}{l|}{> Maragha} & \multicolumn{2}{l|}{cf. SEA} \\ 
		`lentil' & ospən & \armenian{ոսպն} & vəi̯sp & \armenian{վըⁱ}\armenian{սպ} & vosp & \armenian{ոսպ} \\
		`son' & oɾd\'i & \armenian{որդի} & vəi̯ɾtʰ\'ə & \armenian{վըⁱ}\armenian{րթը՛} & voɾtʰ\'i & \armenian{որդի} \\
		\hline 
	\end{tabular}
\end{table}


In the body of the word, it becomes /o, œ, əi̯, u̯e, ui̯/ <\armenian{օ, էօ, ըⁱ}, \armenian{ուէ, ուⁱ}>, according to particular circumstances (Table \ref{tab:Maragha:phonology:soundChange:monoph:omed}). 



\begin{table}[H]
	\centering
	\caption{Change from Classical Armenian /o/ <\armenian{ո}> to /o, œ, əi̯, u̯e, ui̯/ <\armenian{օ, էօ, ըⁱ}, \armenian{ուէ, ուⁱ}> in the Maragha dialect}
	\label{tab:Maragha:phonology:soundChange:monoph:omed}
	\resizebox{\textwidth}{!}{%
	\begin{tabular}{|l| ll|ll| ll|}
		\hline & \multicolumn{2}{l|}{Classical Armenian} &\multicolumn{2}{l|}{> Maragha} & \multicolumn{2}{l|}{cf. SEA} \\ 
		`work' & ɡoɾt͡s & \armenian{գործ}& kui̯ɾt͡s & \armenian{կուⁱ}\armenian{րծ} & ɡoɾt͡s & \armenian{գործ} \\
		`grass' & χot & \armenian{խոտ}& χui̯t & \armenian{խուⁱ}\armenian{տ} & χot & \armenian{խոտ} \\
		`earth' &hoɫ & \armenian{հող} & χu̯eʁ & \armenian{խուէղ} & hoʁ & \armenian{հող} \\
		`soul' & hoɡ\'i & \armenian{հոգի} & χokʰ\'ə & \armenian{խօքը՛} & hokʰ\'i & \armenian{հոգի} \\
		`to roll' & ɡəloɾel & \armenian{գլորել} & kʲʏllœɾel & \armenian{կյիւլլէօրէլ} & ɡəloɾel & \armenian{գլորել} \\
		`bishop' & episkopos & \armenian{եպիսկոպոս} & jəpəskɑpəi̯s & \armenian{յըպըսկապըⁱ}\armenian{ս} & jepiskopos & \armenian{եպիսկոպոս} \\
		\hline 
	\end{tabular}
}
\end{table}



\paragraph{Classical Armenian /u/ <\armenian{ու}>}

Classical Armenian /u/ <\armenian{ու}> became /u, ui̯, ʏ/ <\armenian{ու, ուⁱ}, \armenian{իւ}> (Table \ref{tab:Maragha:phonology:soundChange:monoph:u}). 


\begin{table}[H]
	\centering
	\caption{Change from Classical Armenian /u/ <\armenian{ու}> to /u, ui̯, ʏ/ <\armenian{ու, ուⁱ}, \armenian{իւ}> in the Maragha dialect}
	\label{tab:Maragha:phonology:soundChange:monoph:u}
	\begin{tabular}{|l| ll|ll| ll|}
		\hline & \multicolumn{2}{l|}{Classical Armenian} &\multicolumn{2}{l|}{> Maragha} & \multicolumn{2}{l|}{cf. SEA} \\ 
			`water' &d͡ʒuɾ & \armenian{ջուր} & t͡ʃʏɾ & \armenian{ճիւր} & d͡ʒuɾ & \armenian{ջուր} \\ 
				`house' &tun & \armenian{տուն} & tʏn & \armenian{տիւն} & tun & \armenian{տուն} \\ 
		`bundle' & χuɾd͡z & \armenian{խուրձ} & χui̯ɾt͡sʰ & \armenian{խուⁱ}\armenian{րց} & χuɾt͡sʰ & \armenian{խուրձ} \\
		\hline 
	\end{tabular}
\end{table}

\subsubsection{Diphthongal vowel changes}


\paragraph{Classical Armenian /ɑi̯/ <\armenian{այ}>}

Classical Armenian /ɑi̯/ <\armenian{այ}> became /e/ <\armenian{է}> (Table \ref{tab:Maragha:phonology:soundChange:diph:ai}). 


\begin{table}[H]
	\centering
	\caption{Change from Classical Armenian /ɑi̯/ <\armenian{այ}> to /e/ <\armenian{է}> in the Maragha dialect}
	\label{tab:Maragha:phonology:soundChange:diph:ai} 
	\begin{tabular}{|l| ll|ll| ll|}
		\hline & \multicolumn{2}{l|}{Classical Armenian} &\multicolumn{2}{l|}{> Maragha} & \multicolumn{2}{l|}{cf. SEA} \\ 
		`father' & hɑi̯ɾ & \armenian{հայր} & χeɾ & \armenian{խէր} & hɑjɾ & \armenian{հայր} \\ 
		`sound' & d͡zɑi̯n & \armenian{ձայն} & t͡sen & \armenian{ծէն} & d͡zɑjn & \armenian{ձայն} \\ 
		\hline 
	\end{tabular}
\end{table}
\paragraph{Classical Armenian /iu̯/ <\armenian{իւ}>}

Classical Armenian /iu̯/ <\armenian{իւ}> became /ʏ, i/ <\armenian{իւ, ի}> (Table \ref{tab:Maragha:phonology:soundChange:diph:iu}). 

\begin{table}[H]
	\centering
	\caption{Change from Classical Armenian /iu̯/ <\armenian{իւ}> to /ʏ, i/ <\armenian{իւ, ի}> in the Maragha dialect}
	\label{tab:Maragha:phonology:soundChange:diph:iu} 
	\begin{tabular}{|l| ll|ll| ll|}
		\hline & \multicolumn{2}{l|}{Classical Armenian} &\multicolumn{2}{l|}{> Maragha} & \multicolumn{2}{l|}{cf. SEA} \\ 
		`hundred' & hɑɾiu̯ɾ & \armenian{հարիւր} & χæɾir & \armenian{խա̈րիր} & hɑɾjuɾ & \armenian{հարյուր} \\
		`snow' & d͡ziu̯n & \armenian{ձիւն}& t͡sʏn & \armenian{ծիւն} & d͡zjun & \armenian{ձյուն} \\ 
		\hline 
	\end{tabular}
\end{table}


\paragraph{Classical Armenian /oi̯/ <\armenian{ոյ}>}

Classical Armenian /oi̯/ <\armenian{ոյ}> became /ʏ, ui̯/ <\armenian{իւ, ուⁱ}> (Table \ref{tab:Maragha:phonology:soundChange:diph:oi}).

 \translatorHD{Adjarian provides the CA word `sleep' /kʰun/ <\armenian{քուն}>, but I think this is a mistake because it doesn't  have a diphthong /oi̯/. (?)}



\begin{table}[H]
	\centering
	\caption{Change from Classical Armenian /oi̯/ <\armenian{ոյ}> became /ʏ, ui̯/ <\armenian{իւ, ուⁱ}> in the Maragha dialect}
	\label{tab:Maragha:phonology:soundChange:diph:oi} 
	\begin{tabular}{|l| ll|ll| ll|}
		\hline & \multicolumn{2}{l|}{Classical Armenian} &\multicolumn{2}{l|}{> Maragha} & \multicolumn{2}{l|}{cf. SEA} \\ 
		`light' & loi̯s & \armenian{լոյս}& lui̯s & \armenian{լուⁱ}\armenian{ս} & lujs & \armenian{լույս} \\
		`sleep' & kʰun & \armenian{քուն}& kʰʲʏn & \armenian{քյիւն} & kʰun & \armenian{քուն} \\
		\hline 
	\end{tabular}
\end{table}

\subsubsection{Consonant changes}
The consonant changes are exactly the say as in the dialects of Van or Karabakh. The Classical sound /h/ <\armenian{հ}> is always /χ/ <\armenian{խ}>. 

\section{Morphology}
\subsection{Noun inflection or declension}

\subsubsection{Vowel harmony}
In the grammar, everything is established based on the rule of analogy. Nominal and verbal formatives and endings change their vowels according to the vowel that's contained in the root of the word. For example, the definite article becomes /-ɑ/ <\armenian{ա}> if the vowel of the word-final syllable is /ɑ/ <\armenian{ա}> or /u/ <\armenian{ու}>. But it becomes /-æ/ <\armenian{ա̈}> if that vowel is /æ, e, ʏ/ <\armenian{ա̈}, \armenian{է, իւ}>. The genitive formative is /-ə/ <\armenian{ը}> if the vowel of the word-final syllable is /ɑ/ <\armenian{ա}> or /ə/ <\armenian{ը}>. But that formative becomes /-ʏ/ <\armenian{իւ}> if the vowel is /ʏ/ <\armenian{իւ}> or /œ/ <\armenian{էօ}>. It also becomes /-u/ <\armenian{ու}> when in front the vowels /u, o/ <\armenian{ու, օ}>, and it becomes /-i/ <\armenian{ի}> in front the vowel /i/ <\armenian{ի}>. Even the copular verb is subject to these assimilatory changes. 

\subsubsection{Plural and case marking}

The plural formative is /-iɾ/ <\armenian{իր}> for monosyllabic words, /-niɾ/ <\armenian{նիր}> for vowel-final polysyllabic words, /-kʰiɾ/ <\armenian{քիր}> for consonant-final polysyllabic words. 

In declension, there is no loss or deletion of vowels (Table \ref{tab:Maragha:phonology:morpho:vowelRed}). 


\begin{table}[H]
	\centering
	\caption{No vowel reduction in the Maragha dialect}
	\label{tab:Maragha:phonology:morpho:vowelRed}
	\begin{tabular}{|l| ll|ll| ll|}
		\hline & \multicolumn{2}{l|}{Classical Armenian} &\multicolumn{2}{l|}{> Maragha} & \multicolumn{2}{l|}{cf. SEA} \\ 
		`nose' &kʰitʰ & \armenian{քիթ} & & &kʰitʰ & \armenian{քիթ} \\
		`nose-{\gen}' &kʰətʰ-i & \armenian{քթի} & kʰitʰ-i & \armenian{քիթի} &kʰətʰ-i & \armenian{քթի} \\
		`meat' & mis & \armenian{միս} & & &mis & \armenian{միս} \\ 
		`meat-{\gen}' & məs-i & \armenian{մսի} & mis-i &\armenian{միսի} &məs-i & \armenian{մսի} \\ 
		`heart' & siɾt & \armenian{սիրտ} & & &siɾt & \armenian{սիրտ} \\ 
		`heart-{\gen}' & səɾt-i & \armenian{սրտի} & siɾt-i & \armenian{սիրտի} & səɾt-i & \armenian{սրտի} \\ 
		\hline 
	\end{tabular}
\end{table}

This dialect has the following cases: nominative, genitive-dative, accusative, ablative, and instrumental. There is no locative; the accusative is like the /um/ <\armenian{ում}> branch. While the ablative is formed with the formative /-en/ <\armenian{էն}>. 

\subsection{Verb inflection or conjugation}

\subsubsection{Overview of changes}\label{sec:Maragha:morpho:verb:overview}

 

As we said above, in the \armenian{ել} branch, the present stem is formed based on the verb's infinitive, by combining it or conjugating it with the auxiliary verb (\ref{sent:Maragha:morpho:verb:infconj}). 

\begin{exe}
	\ex `I want' \label{sent:Maragha:morpho:verb:infconj} \begin{xlist}
		\ex Maragha
		\begin{xlist}
			\ex \gll ʏz-e-l-i i-m \\
			want-{\thgloss}-{\infgloss}-{\impfcvb}(?) {\aux}-1{\sg} \\
			\trans \armenian{իւզէլի իմ} \label{sent:Maragha:morpho:verb:infconj:base} 
			\ex \gll ʏz-e-l-i-m \\
			want-{\thgloss}-{\infgloss}-{\aux}-1{\sg} \\
			\trans \armenian{իւզէլիմ} \label{sent:Maragha:morpho:verb:infconj:merged} 
		\end{xlist}
		\ex cf. SWA \\
		\gll ɡ-uz-e-m \\
		{\ind}-want-{\thgloss}-1{\sg} \\
		\trans \armenian{կ՚ուզեմ}
		\ex cf. SEA \\
		\gll uz-um e-m \\
		want-{\impfcvb} {\aux}-1{\sg} \\
		\trans \armenian{ուզում եմ}
	\end{xlist}
	
\end{exe}


\translatorHD{Note that in \ref{sent:Maragha:morpho:verb:infconj:base}, it seems that the verb ends in some vowel /i/ and then the auxiliary is added. It's unclear what is the morphological function of this final vowel. It could be glossed as a cognate of the irregular SEA imperfective converb suffix /-is/. But in (\ref{sent:Maragha:morpho:verb:infconj:merged}), it seems that this vowel is deleted and the auxiliary is cliticized or merged onto the verb.(?)}
\begin{adjarianpage}\label{page:283}\end{adjarianpage}% should be 283


The formative /kə/ <\armenian{կը}> is used only in the future. Every past tense is formed from the present by adding the formative /eɾ/ <\armenian{էր}>, without differentiating for person or number. For the perfective, a new form has been created. 

\subsubsection{General paradigm}

The following is the complete conjugation of the verb `to like' (derived from CA /uz-e-l/ `to want' <\armenian{ուզել}>).

{\paradigmExplanation}

\paragraph{Indicative present and past imperfective}

\translatorHD{In SEA (Table \ref{tab:Maragha:morpho:verb:paradigm:presentIndc}), the indicative present is formed by taking combining a non-finite form of the verb (called the imperfective converb with the suffix /-um/) with the present auxiliary. In Maragha, we see a similar periphrastic approach. However, the non-finite form is based on the verb's infinitive. The auxiliary seems to then be cliticized onto the verb. Note how the two dialects diverge in the form of the auxiliary: /e/ for SEA, but /e, i/ for Maragha.}

\begin{table}[H]
	\centering
	\caption{Indicative present <\armenian{ներկայ}> in the Maragha dialect}
	\label{tab:Maragha:morpho:verb:paradigm:presentIndc}
	\begin{tabular}{|l|ll|ll|}
		\hline & \multicolumn{2}{l|}{Maragha `to like'} & \multicolumn{2}{l|}{cf. SEA `to want'} \\
		1SG & ʏz-e-l-i-m & \armenian{իւզէլիմ} & uz-um e-m & \armenian{ուզում եմ} \\
&  \multicolumn{2}{l|}{`I like'}  &  \multicolumn{2}{l|}{`I want'}  \\
		2SG & ʏz-e-l-i-s & \armenian{իւզէլիս} & uz-um e-s & \armenian{ուզում ես} \\
		3SG & ʏz-e-l-i-$\emptyset$ & \armenian{իւզէլի} & uz-um e-$\emptyset$ & \armenian{ուզում է} \\
		1PL & ʏz-e-l-i-nkʰʲ & \armenian{իւզէլինքյ} & uz-um e-ŋkʰ & \armenian{ուզում ենք} \\
		2PL & ʏz-e-l-e-kʰʲ & \armenian{իւզէլէքյ} & uz-um e-kʰ & \armenian{ուզում եք} \\
		3PL & ʏz-e-l-i-n & \armenian{իւզէլին} & uz-um e-n & \armenian{ուզում են} 
		\\
		& \multicolumn{2}{l|}{$\sqrt{}$-{\thgloss}-{\infgloss}-{\aux}-{\agr}}& \multicolumn{2}{l|}{$\sqrt{}$-{\impfcvb} {\aux}-{\agr}}\\\hline 
	\end{tabular}
\end{table}

\translatorHD{For SEA, the indicative past imperfective uses the same imperfective converb as in the present (Table \ref{tab:Maragha:morpho:verb:paradigm:pastImpfIndc}). The difference is that auxiliary is now in the past tense. But in Maragha, we use a simpler strategy: the past-marking particle /eɾ/ is added after the present form. Note that this particle seems cliticized in the 3SG.}




\begin{table}[H]
	\centering
	\caption{Indicative pasti imperfective <\armenian{անկատար}> in the Maragha dialect}
	\label{tab:Maragha:morpho:verb:paradigm:pastImpfIndc}
	\begin{tabular}{|l|ll|ll|}
		\hline & \multicolumn{2}{l|}{Maragha `to like'} & \multicolumn{2}{l|}{cf. SEA `to want'} \\
		1SG & ʏz-e-l-i-m eɾ & \armenian{իւզէլիմ էր} & uz-um ej-i-$\emptyset$ & \armenian{ուզում էի} \\
&  \multicolumn{2}{l|}{`I was liking'}  &  \multicolumn{2}{l|}{`I was wanting'}  \\
		2SG & ʏz-e-l-i-s eɾ & \armenian{իւզէլիս էր} & uz-um ej-i-ɾ & \armenian{ուզում էիր} \\
		3SG & ʏz-e-l-$\emptyset$-$\emptyset$-eɾ & \armenian{իւզէլէր} & uz-um e-$\emptyset$-ɾ & \armenian{ուզում էր} \\
		1PL & ʏz-e-l-i-nkʰʲ eɾ & \armenian{իւզէլինքյ էր} & uz-um ej-i-ŋkʰ & \armenian{ուզում էինք} \\
		2PL & ʏz-e-l-e-kʰʲ eɾ & \armenian{իւզէլէքյ էր} & uz-um ej-i-kʰ & \armenian{ուզում էիք} \\
		3PL & ʏz-e-l-i-n eɾ & \armenian{իւզէլին էր} & uz-um ej-i-n & \armenian{ուզում էին} 
		\\
		& \multicolumn{2}{l|}{$\sqrt{}$-{\thgloss}-{\infgloss}-{\aux}-{\agr} {\pst}}& \multicolumn{2}{l|}{$\sqrt{}$-{\impfcvb} {\aux}-{\pst}-{\agr}}\\\hline 
	\end{tabular}
\end{table}

\paragraph{Present perfect and past perfect}

\translatorHD{The present perfect (Table \ref{tab:Maragha:morpho:verb:paradigm:presentPerfect}) and past perfect (Table \ref{tab:Maragha:morpho:verb:paradigm:pastPerfect}) in SEA are formed with periphrasis. The verb is in the form of the perfective converb with the suffix /-el/. The present tense auxiliary is added for the present perfect, while the past auxiliary for the past perfect.}

\translatorHD{Maragha likewise uses periphrasis but with two differences. First in Table \ref{tab:Maragha:morpho:verb:paradigm:presentPerfect}, the non-finite form can use either the suffix /-iɾ/ (cognate with the SEA perfective converb suffix /-el/), or the suffix /-ɑt͡s/ (cognate with the SEA resultative participle suffix /-ɑt͡s/). When the suffix /-ɑt͡s/ is used, the 3SG auxiliary is /ə/ instead of /i/.}

\begin{table}[H]
	\centering
	\caption{Present perfect <\armenian{յարակատար}> in the Maragha dialect}
	\label{tab:Maragha:morpho:verb:paradigm:presentPerfect}
	{% \resizebox{\textwidth}{!}{%
			\begin{tabular}{|l|ll| ll|}
				\hline & \multicolumn{2}{l|}{Maragha `to like' (Form I} & \multicolumn{2}{l|}{cf. SEA `to want'} \\
				1SG & ʏz-iɾ i-m & \armenian{իւզիր իմ} &  uz-el e-m & \armenian{ուզել եմ} \\
				&  \multicolumn{2}{l|}{`I have liked'}    &  \multicolumn{2}{l|}{`I have wanted'}  \\
				2SG & ʏz-iɾ i-s & \armenian{իւզիր իս} &   uz-el e-s & \armenian{ուզել ես} \\
				3SG & ʏz-iɾ i-$\emptyset$ & \armenian{իւզիր ի} &   uz-el e-$\emptyset$ & \armenian{ուզել է} \\
				1PL & ʏz-iɾ i-nkʰʲ & \armenian{իւզիր ինքյ} & uz-el e-ŋkʰ & \armenian{ուզել ենք} \\
				2PL & ʏz-iɾ e-kʰʲ & \armenian{իւզիր էքյ} &  uz-el e-kʰ & \armenian{ուզել եք} \\
				3PL & ʏz-iɾ i-n & \armenian{իւզիր ին} &   uz-el e-n & \armenian{ուզել են} 
				\\
				& \multicolumn{2}{l|}{$\sqrt{}$-{\perfcvb} {\aux}-{\agr}}&  \multicolumn{2}{l|}{$\sqrt{}$-{\perfcvb} {\aux}-{\agr}}\\ 
				
				\hline 
				\hline & \multicolumn{2}{l|}{Maragha `to like' (Form II} & \multicolumn{2}{l|}{cf. SEA `to want'} \\
				1SG &  ʏz-ɑt͡s i-m & \armenian{իւզած իմ} & uz-el e-m & \armenian{ուզել եմ} \\
				&  \multicolumn{2}{l|}{`I have liked'}     &  \multicolumn{2}{l|}{`I have wanted'}  \\
				2SG &  ʏz-ɑt͡s i-s & \armenian{իւզած իս} & uz-el e-s & \armenian{ուզել ես} \\
				3SG &  ʏz-ɑt͡s ə-$\emptyset$ & \armenian{իւզած ը} & uz-el e-$\emptyset$ & \armenian{ուզել է} \\
				1PL &   ʏz-ɑt͡s i-nkʰʲ & \armenian{իւզած ինքյ} & uz-el e-ŋkʰ & \armenian{ուզել ենք} \\
				2PL & ʏz-ɑt͡s e-kʰʲ & \armenian{իւզած էքյ} & uz-el e-kʰ & \armenian{ուզել եք} \\
				3PL &   ʏz-ɑt͡s i-n & \armenian{իւզած ին} & uz-el e-n & \armenian{ուզել են} 
				\\
				&  \multicolumn{2}{l|}{$\sqrt{}$-{\rptcp} {\aux}-{\agr}}& \multicolumn{2}{l|}{$\sqrt{}$-{\perfcvb} {\aux}-{\agr}}\\ 
				
				\hline 
			\end{tabular}
		}
	\end{table}
	
	\translatorHD{In the past perfect, instead of using a special past auxiliary, we simply add the past particle /eɾ/ after the present auxiliary (Table \ref{tab:Maragha:morpho:verb:paradigm:pastPerfect}). Note that for the 3SG, the auxiliary is missing before the past particle /eɾ/.}
	
	
	\begin{table}[H]
		\centering
		\caption{Past perfect <\armenian{գերակատար}> in the Maragha dialect}
		\label{tab:Maragha:morpho:verb:paradigm:pastPerfect}
		{%	\resizebox{\textwidth}{!}{%
				\begin{tabular}{|l|ll|ll|}
					\hline & \multicolumn{2}{l|}{Maragha `to like' (Form I)} & \multicolumn{2}{l|}{cf. SEA `to want'} \\
					1SG & ʏz-iɾ i-m eɾ & \armenian{իւզիր իմ էր} & uz-el ej-i-$\emptyset$ & \armenian{ուզել էի} \\
					&  \multicolumn{2}{l|}{`I had liked'}  &    \multicolumn{2}{l|}{`I had wanted'}  \\
					2SG & ʏz-iɾ i-s eɾ & \armenian{իւզիր իս էր} &  uz-el ej-i-ɾ & \armenian{ուզել էիր} \\
					3SG & ʏz-iɾ $\emptyset$-$\emptyset$ eɾ & \armenian{իւզիր էր} &  uz-el e-$\emptyset$-ɾ & \armenian{ուզել էր} \\
					1PL & ʏz-iɾ i-nkʰʲ eɾ & \armenian{իւզիր ինքյ էր} &   uz-el ej-i-ŋkʰ & \armenian{ուզել էինք} \\
					2PL & ʏz-iɾ e-kʰʲ eɾ & \armenian{իւզիր էքյ էր} &   uz-el ej-i-kʰ & \armenian{ուզել էիք} \\
					3PL & ʏz-iɾ i-n eɾ & \armenian{իւզիր ին էր} &  uz-el ej-i-n & \armenian{ուզել էին} 
					\\
					& \multicolumn{2}{l|}{$\sqrt{}$-{\perfcvb} {\aux}-{\agr} {\pst}}&   \multicolumn{2}{l|}{$\sqrt{}$-{\perfcvb} {\aux}-{\pst}-{\agr}}\\ 
					
					\hline 
					\hline & \multicolumn{2}{l|}{Maragha `to like' (Form II)} & \multicolumn{2}{l|}{cf. SEA `to want'} \\
					1SG &  ʏz-ɑt͡s i-m eɾ & \armenian{իւզած իմ էր} & uz-el ej-i-$\emptyset$ & \armenian{ուզել էի} \\
					&  \multicolumn{2}{l|}{`I had liked'}  &  \multicolumn{2}{l|}{`I had wanted'}  \\
					2SG &  ʏz-ɑt͡s i-s eɾ & \armenian{իւզած իս էր} & uz-el ej-i-ɾ & \armenian{ուզել էիր} \\
					3SG &  ʏz-ɑt͡s $\emptyset$-$\emptyset$ eɾ & \armenian{իւզած էր} & uz-el e-$\emptyset$-ɾ & \armenian{ուզել էր} \\
					1PL &   ʏz-ɑt͡s i-nkʰʲ eɾ & \armenian{իւզած ինքյ էր} & uz-el ej-i-ŋkʰ & \armenian{ուզել էինք} \\
					2PL &  ʏz-ɑt͡s e-kʰ eɾʲ & \armenian{իւզած էքյ էր} & uz-el ej-i-kʰ & \armenian{ուզել էիք} \\
					3PL & ʏz-ɑt͡s i-n eɾ & \armenian{իւզած ին էր} & uz-el ej-i-n & \armenian{ուզել էին} 
					\\
					&   \multicolumn{2}{l|}{$\sqrt{}$-{\rptcp} {\aux}-{\agr} {\pst}}& \multicolumn{2}{l|}{$\sqrt{}$-{\perfcvb} {\aux}-{\pst}-{\agr}}\\ 
					
					\hline 
				\end{tabular}
			}
		\end{table}
		

\paragraph{Complex future tense}
\translatorHD{Adjarian lists in Table \ref{tab:Maragha:morpho:verb:futComplex} a paradigm that he calls the complex future and its past form. The complex future is formed periphrastically by combining a non-finite form with the present auxiliary. Its past version (past future (?)) is formed by then adding the past particle /eɾ/.}

\translatorHD{Morphologically, the non-finite form seems to be built by adding the suffix /-u/ to the infinitive; the theme vowel becomes /o/. This non-finite form seems a cognate to the SEA future converb as in (\ref{sent:Maragha:morpho:verb:fut:sea}), and thus I gloss /-u/ as {\futcvb}.  The 3SG has a covert auxiliary}

\begin{exe}
	\ex SEA \label{sent:Maragha:morpho:verb:fut:sea}
	\\
	 \gll uz-e-l-u e-m \\
		want-{\thgloss}-{\infgloss}-{\futcvb} {\aux}-1{\sg}\\
		\trans `I will want.'\\
		\armenian{ուզելու եմ}

\end{exe}

\begin{table}[H]
	\centering \caption{Complex future forms for the verb `to like' in the Maragha dialect}
	\label{tab:Maragha:morpho:verb:futComplex}
	\resizebox{\textwidth}{!}{%
	\begin{tabular}{|l|ll|ll|}
		\hline & \multicolumn{2}{l|}{Complex future <\armenian{բարդ ապառնի}>}& \multicolumn{2}{l|}{Past version <\armenian{անցեալ}>} \\ 
		1SG & ʏz-o-l-u i-m & \armenian{իւզօլու իմ} & ʏz-ol-u i-m eɾ & \armenian{իւզօլու իմ էր} \\
		2SG & ʏz-ol-u i-s & \armenian{իւզօլու իս} & ʏz-ol-u i-s eɾ & \armenian{իւզօլու իս էր} \\
		3SG & ʏz-ol-u $\emptyset$-$\emptyset$ & \armenian{իւզօլու} & ʏz-ol-u $\emptyset$-$\emptyset$ eɾ & \armenian{իւզօլու էր} \\
		1PL & ʏz-ol-u i-nkʰʲ & \armenian{իւզօլու ինքյ} & ʏz-ol-u i-nkʰʲ eɾ & \armenian{իւզօլու ինքյ էր} \\
		2PL & ʏz-ol-u e-kʰʲ & \armenian{իւզօլու էքյ} & ʏz-ol-u e-kʰʲ eɾ & \armenian{իւզօլու էքյ էր} \\
		3PL & ʏz-ol-u i-n & \armenian{իւզօլու ին} & ʏz-ol-u i-n eɾ & \armenian{իւզօլու ին էր} \\ 
		& \multicolumn{2}{l|}{$\sqrt{}$-{\thgloss}-{\infgloss}-{\futcvb} {\aux}-{\agr}}& \multicolumn{2}{l|}{$\sqrt{}$-{\thgloss}-{\infgloss}-{\futcvb} {\aux}-{\agr} {\pst}}
		\\ \hline 
	\end{tabular}
}\end{table}

\translatorHD{Unfortunately, Adjarian does not describe the semantic difference between his `complex future' and the simple `future'. The 1SG complex future could mean `I will like', while the past future could mean `I was going to like.'}
\paragraph{Past perfective or aorist}

\translatorHD{The past perfective (Table \ref{tab:Maragha:morpho:verb:paradigm:pastperfectiveAorist}) is also called the aorist. In SEA for /uz-e-l/ `to want', the past perfective is formed by taking the root and theme vowel, adding the aorist or perfective suffix /-t͡sʰ-/, and then adding the past suffix /-i/ and the appropriate agreement suffixes. The 3SG uses covert tense and agreement suffixes. The Maragha dialect does a quite different strategy. Between the root and the agreement suffix, we see a single vowel /i/ or /u/. It seems this vowel acts as a past marker. But the data is too limited to be sure. The two dialects seem to have incomparable morphology}


\begin{table}[H]
	\centering
	\caption{Past perfective or aorist <\armenian{կատարեալ}> in the Maragha dialect}
	\label{tab:Maragha:morpho:verb:paradigm:pastperfectiveAorist}
	\begin{tabular}{|l|ll|ll|}
		\hline & \multicolumn{2}{l|}{Maragha `to like'} & \multicolumn{2}{l|}{cf. SEA `to want'} \\
		1SG & ʏz-u-m & \armenian{իւզում} & uz-e-t͡sʰ-i-$\emptyset$ & \armenian{ուզեցի} \\
&  \multicolumn{2}{l|}{`I   liked'}  &      \multicolumn{2}{l|}{`I   wanted'}  \\
		2SG & ʏz-i-ɾ & \armenian{իւզիր} & uz-e-t͡sʰ-i-ɾ & \armenian{ուզեցիր} \\
		3SG & ʏz-i-t͡sʰ & \armenian{իւզից} & uz-e-t͡sʰ-$\emptyset$-$\emptyset$ & \armenian{ուզեց} \\
		1PL & ʏz-u-nkʰ & \armenian{իւզունք} & uz-e-t͡sʰ-i-ŋkʰ & \armenian{ուզեցինք} \\
		2PL &ʏz-u-kʰ & \armenian{իւզուք} & uz-e-t͡sʰ-i-kʰ & \armenian{ուզեցիք} \\
		3PL &ʏz-u-n & \armenian{իւզուն} & uz-e-t͡sʰ-i-n & \armenian{ուզեցին} \\
		& \multicolumn{2}{l|}{$\sqrt{}$-{\pst}(?)-{\agr}}& \multicolumn{2}{l|}{$\sqrt{}$-{\thgloss}-{\aor}-{\pst}-{\agr}}\\ 
		
		\hline 
	\end{tabular}
\end{table}
\paragraph{Subjunctive present and past} 

\translatorHD{In SEA, the subjunctive present (Table \ref{tab:Maragha:morpho:verb:paradigm:subjPresent}) is formed by adding agreement suffixes after the theme vowel /e/. These are the same agreement suffixes that are added onto the present auxiliary in the indicative present. For a verb like `to want', the 3SG involves changing the theme vowel /e/ to /i/ in the 3SG. The Maragha dialect is similar, but the theme vowel can vary between /ʏ, i, e/.} 


\begin{table}[H]
	\centering
	\caption{Subjunctive present <\armenian{ստորադասական ներկայ}> in the Maragha dialect}
	\label{tab:Maragha:morpho:verb:paradigm:subjPresent}
	\begin{tabular}{|l|ll|ll|}
		\hline & \multicolumn{2}{l|}{Maragha `to like'} & \multicolumn{2}{l|}{cf. SEA `to want'} \\
		1SG & ʏz-ʏ-m & \armenian{իւզիւմ} & uz-e-m & \armenian{ուզեմ} \\
&  \multicolumn{2}{l|}{`(if) I   like'}  &      \multicolumn{2}{l|}{`(if) I   want'}  \\
		2SG & ʏz-i-s & \armenian{իւզիս} & uz-e-s & \armenian{ուզես} \\
		3SG & ʏz-ʏ-$\emptyset$ & \armenian{իւզիւ} & uz-i-$\emptyset$ & \armenian{ուզի} \\
		1PL & ʏz-i-nkʰʲ & \armenian{իւզինքյ} & uz-e-ŋkʰ & \armenian{ուզենք} \\
		2PL & ʏz-e-kʰʲ & \armenian{իւզէքյ} & uz-e-kʰ & \armenian{ուզեք} \\
		3PL & ʏz-i-n & \armenian{իւզին} & uz-e-n & \armenian{ուզեն} 
		\\
		& \multicolumn{2}{l|}{$\sqrt{}$-{\thgloss}-{\agr}}& \multicolumn{2}{l|}{$\sqrt{}$-{\thgloss}-{\agr}}\\ 
		\hline 
	\end{tabular}
\end{table}

\translatorHD{In SEA, the subjunctive past (Table \ref{tab:Maragha:morpho:verb:paradigm:subjPast}) is formed by adding the past suffix /i/ and agreement suffixes after the theme vowel. In Maragha, we instead add the past particle /eɾ/ after the verb. For the 3SG, this particle seems to cliticize to the verb and delete the verb's theme vowel.}



\begin{table}[H]
	\centering
	\caption{Subjunctive past <\armenian{ստորադասական անցեալ}> in the Maragha dialect}
	\label{tab:Maragha:morpho:verb:paradigm:subjPast}
	\begin{tabular}{|l|ll|ll|}
		\hline & \multicolumn{2}{l|}{Maragha `to like'} & \multicolumn{2}{l|}{cf. SEA `to want'} \\
		1SG & ʏz-ʏ-m eɾ & \armenian{իւզիւմ էր} & uz-ej-i-$\emptyset$ & \armenian{ուզեի} \\
&  \multicolumn{2}{l|}{`(if) I   liked'}  &      \multicolumn{2}{l|}{`(if) I   wanted'}  \\
		2SG & ʏz-i-s eɾ & \armenian{իւզիս էր} & uz-ej-i-ɾ & \armenian{ուզեիր} \\
		3SG & ʏz-$\emptyset$-$\emptyset$-eɾ & \armenian{իւզէր} & uz-e-$\emptyset$-ɾ & \armenian{ուզեր} \\
		1PL & ʏz-i-nkʰʲ eɾ & \armenian{իւզինքյ էր} & uz-ej-i-ŋkʰ & \armenian{ուզեինք} \\
		2PL & ʏz-e-kʰʲ eɾ & \armenian{իւզէքյ էր} & uz-ej-i-kʰ & \armenian{ուզեիք} \\
		3PL & ʏz-i-n eɾ & \armenian{իւզին էր} & uz-ej-i-n & \armenian{ուզեին} 
		\\
		& \multicolumn{2}{l|}{$\sqrt{}$-{\thgloss}-{\agr} {\pst}}& \multicolumn{2}{l|}{$\sqrt{}$-{\thgloss}-{\pst}-{\agr}}\\ 
		
		\hline 
	\end{tabular}
\end{table}



\paragraph{Tenses built from the subjunctive: Future}


\translatorHD{In Maragha, the future and past future are built off subjunctive (Table \ref{tab:Maragha:morpho:verb:paradigm:complexSubjunctive}). For the verb `to like', we simply add the future prefix /k-/. SEA behaves essentially the same and I don't provide its paradigm.}


\begin{table}[H]
	\centering
	\caption{Forms that are built from the subjunctive forms for the verb `to like' in the Maragha dialect}
	\label{tab:Maragha:morpho:verb:paradigm:complexSubjunctive}
	\begin{tabular}{|l|ll|ll|}
		\hline & 
		\multicolumn{2}{l|}{Future <\armenian{ապառնի}>} & \multicolumn{2}{l|}{Past future <\armenian{անցեալ ապառնի}>} \\
		1SG & k-ʏz-ʏ-m & \armenian{կիւզիւմ} & k-ʏz-ʏ-m eɾ & \armenian{կիւզիւմ էր} \\
&  \multicolumn{2}{l|}{`I will like'}  &      \multicolumn{2}{l|}{`I was going to  like'}  \\
		2SG & k-ʏz-i-s & \armenian{կիւզիս} & k-ʏz-i-s eɾ & \armenian{կիւզիս էր} \\
		3SG & k-ʏz-ʏ-$\emptyset$ & \armenian{կիւզիւ} & k-ʏz-$\emptyset$-$\emptyset$-eɾ & \armenian{կիւզէր} \\
		1PL & k-ʏz-i-nkʰʲ & \armenian{կիւզինքյ} & k-ʏz-i-nkʰʲ eɾ & \armenian{կիւզինքյ էր} \\
		2PL & k-ʏz-e-kʰʲ & \armenian{կիւզէքյ} & k-ʏz-e-kʰʲ eɾ & \armenian{կիւզէքյ էր} \\
		3PL & k-ʏz-i-n & \armenian{կիւզին} & k-ʏz-i-n eɾ & \armenian{կիւզին էր} \\
		& \multicolumn{2}{l|}{{\fut} $\sqrt{}$-{\thgloss}-{\agr}}& \multicolumn{2}{l|}{{\fut} $\sqrt{}$-{\thgloss}-{\agr} {\pst}}
		\\ \hline\end{tabular}
\end{table} 

\paragraph{Imperative and prohibitive}

\translatorHD{For the imperative 2SG, SEA adds the morph /-iɾ/ after the root for a verb like `to like' (Table \ref{tab:Maragha:morpho:verb:paradigm:Imp}). For the 2PL, archaic SEA adds the sequence /-e-t͡sʰ-ekʰ/ after the root such that /-e-t͡sʰ/ forms the aorist stem, while /-ekʰ/ is the agreement marker. More modern registers of SEA instead just add the sequence /-ekʰ/ directly after the root. Maragha is somewhat different. In the 2SG, we only see a suffix /-ʏ/ after the root. For the 2PL, we only see a suffix /-ekʰʲ/. I suspect this suffix /-ʏ/ is a theme vowel based on the other paradigms.}


\begin{table}[H]
	\centering
	\caption{Imperative forms <\armenian{հրամայական}> in the Maragha dialect}
	\label{tab:Maragha:morpho:verb:paradigm:Imp}
	\begin{tabular}{|l|ll|ll|l|}
		\hline & \multicolumn{2}{l|}{Maragha `like!'} & \multicolumn{2}{l|}{cf. SEA `want!'} & \\
		2SG & ʏz-\'ʏ-$\emptyset$ & \armenian{իւզի՛ւ} & uz-$\emptyset$-\'iɾ & \armenian{ուզի՛ր} & $\sqrt{}$-{\thgloss}-{\imp}.2{\sg}
		\\
		2PL& & & uz-e-t͡sʰ-ekʰ& \armenian{ուզեցեք} & $\sqrt{}$-{\thgloss}-{\aor}-{\imp}.2{\pl}
		\\
		& ʏz-ekʰʲ&\armenian{իւզէքյ} & uz-ekʰ&\armenian{ուզեք}& $\sqrt{}$-{\imp}.2{\pl}
		
		\\\hline \end{tabular}
\end{table}

\translatorHD{For the prohibitive or negative imperative (Table \ref{tab:Maragha:morpho:verb:paradigm:Proh}), SEA simply adds the prohibitive formative /mi/ before the imperative form. Maragha behaves the same.} 


\begin{table}[H]
	\centering
	\caption{Negative imperative or prohibitive forms in the Maragha dialect}
	\label{tab:Maragha:morpho:verb:paradigm:Proh}
	\begin{tabular}{|l|ll|ll|l|}
		\hline & \multicolumn{2}{l|}{Maragha `do not like!'} & \multicolumn{2}{l|}{cf. SEA `do not want!'} & \\
		2SG & mi ʏz-ʏ-$\emptyset$ & \armenian{մի իւզիւ} & m\'i uz-$\emptyset$-iɾ & \armenian{մի՛ ուզիր} & {\proh} $\sqrt{}$-{\imp}.2{\sg} \\
		2PL & mi ʏz-ekʰʲ& \armenian{մի իւզէքյ} & mi uz-ekʰ& \armenian{մի՛ ուզեք} & {\proh} $\sqrt{}$-{\imp}.2{\pl} \\
		\hline \end{tabular}
\end{table}


\paragraph{Non-finite forms}


\translatorHD{On the original page, Adjarian didn't list any participles or non-finite forms. But on page 306, Adjarian provides an erratum with the following non-finite forms (Table \ref{tab:Maragha:morpho:verb:paradigm:participle}).}


\begin{table}[H]
	\centering 
	\caption{Participles or converbs <\armenian{դերբայներ}> in the Maragha dialect}
	\label{tab:Maragha:morpho:verb:paradigm:participle}
		\resizebox{\textwidth}{!}{%
		\begin{tabular}{|ll|ll|ll|l|}
		\hline && \multicolumn{2}{l|}{Maragha `to like'} & \multicolumn{2}{l|}{cf. SEA `to want'} & \\ 
		Infinitive & \armenian{անորոշ} & ʏz-i-l & \armenian{իւզիլ} & uz-e-l & \armenian{ուզել} & $\sqrt{}$-{\thgloss}-{\infgloss} \\ 
		Past & \armenian{անցեալ} & ʏz-ɑt͡s & \armenian{իւզած} & uz-ɑt͡s & \armenian{ուզած} & $\sqrt{}$-{\rptcp} \\ 
		& &ʏz-iɾ & \armenian{իւզիր} & uz-el & \armenian{ուզել} & $\sqrt{}$-{\perfcvb} \\ 
		Future & \armenian{ապառնի} & ʏz-o-l-u & \armenian{իւզօլու} & uz-e-l-u & \armenian{ուզելու} & $\sqrt{}$-{\thgloss}-{\infgloss}-{\futcvb} \\ 
		\hline \end{tabular}
}
\end{table}






\begin{adjarianpage}\label{page:284}\end{adjarianpage}% should be 284

\section{Subdialects}
\subsection{Urmia}
The subdialect of Urmia is the same as Maragha. But from the subsequent text samples, it seems that there are some differences. 

\subsubsection{Morphological differences}

For example, the plural formatives are /-eɾ, -neɾ/ <\armenian{էր, նէր}>, while in Maragha they are /-iɾ, -niɾ/ <\armenian{իր, նիր}>. 

The future is formed with the formative /tikʲi/ <\armenian{տիկյի}>, which is of course a form change from CA /piti/ <\armenian{պիտի}> `it is necessary'. 
\subsubsection{Object clitics}

The use of the possessive article in verbs is very interesting (\ref{sent:Maragha:subdialect:urmia:poss}).

\begin{exe}
	\ex Urmia (Maragha)
	\label{sent:Maragha:subdialect:urmia:poss}
	\begin{xlist}
		\ex \gll me t͡si prn-e-nkʰ-t \\
		one horse catch-{\thgloss}-1{\pl}-{\possSsg} \\
		\trans `(Let us) catch a horse for you.' \\
		\armenian{մէ ծի պռնէնքտ} 
		\ex \gll pʰtrt-e-s eɾ-d\\
		search-{\thgloss}-{\impfcvb}(?) {\pst}-{\possSsg} \\
		\trans `He was looking for you.' \\
		\armenian{փտռտես էրդ} 
		\ex \gll ɑrɑk-n ɑs-e-l-i, n\'ɑ ʃɑt kʰɑχt͡sʰɾ jel, ut-e-n-d, nɑ ʃɑt tʰɑr jel, tʰɑl-e-n-d \\
		proverb-{\defgloss} say-{\thgloss}-{\infgloss}-?, no? very sweet be?, eat-{\thgloss}-3{\pl}-{\possSsg}, no? very bitter be?, throw-{\thgloss}-3{\pl}-{\possSsg}\\
		\trans `The proverb says: Don't be too sweet, they'll eat you; don't be too bitter, they'll throw you away.'\\
		\armenian{առակն ասէլի} -- \armenian{նա՛ շատ քախցր յէլ՝ ուտէնդ, նա շատ թառ յէլ՝ թալէնդ}
		
	\end{xlist}
\end{exe}

This usage of the possessive article is borrowed from Persian, where one says for example <didem-et> `I saw you>, <binem-et> `I see you>. 

 

\translatorHD{He means that the Armenian possessive article here is acting as an object clitic. See \citet[\S7.1]{DolatianEtAl-prep-IranianGrammar} for similar data from other Iranian Armenian dialects.}

\section{Text samples}

{\sampleoverview}

\subsection{Maragha: \armenian{Խառնիս ինա̈ն նիշան}}



\armenian{Մէ օր Սօնան ինա̈ն Անդո՛ւնու կիւզիւն իւրիւս (իւրիւնց) տղային փսակին։ Սօնան կասը՛ Անդունին}.

– \armenian{Յար, էլչի օղօրկիհյ Հա̈րթիւնիւ ախչկան իւզօլու։}

– \armenian{Չէ՛, Սօ՛նա, մէ էօզգա̈՛նա̈ խիյալ ա̈րա̈}. \armenian{հա̈լբա̈թ նա̈րա̈ չուտուրուն. էն հարուս, յիս ախկատ։}

– \armenian{Չէ՛, Ա՛նդուն, իշքան խիյալ անէլիմ՛ նա̈րմէն աղէ՛կյա̈ չիմնա̈լի գյիննէլ. էսս (հէնց) յիս առէլիմ «Էթահյ նա̈րա̈ իւզիհյ. յա կըտան, յա չին տա̈»։}


\begin{adjarianpage}\label{page:285}\end{adjarianpage}% should be 285

– \armenian{Դէ մըկա քյի էտէնց ի, լաճիքյիրիս (մեր տղան) էլ իւզէլի, շապպաթ օ՛րա էլչի օղօրկու, թօղ էթա̈ն իւզիւն։}

\armenian{Շա̈պպա̈թ օ՛րա Սօնան շուտօվ կի զարթնի, սիմավա՛րա քի քիցի, չայի՛րա քը խըմին, ա՛ննա՛ն (յետոյ) Սօնան կըլը՛ կէթա̈ իւր բաջու տո՛ւնա, Միրվարիյին կասը՛ քյի՝}

– \armenian{Այ Միրվարը՛, ա՛խչի. յէրէկյ իրիկաս (ամուսնոյս) խէտա̈ մէ զա՛դ իհյ խիյալ արի. իւզէլիհյ Հա̈րթիւնիւ ախչկան առնիհյ միր Միսակին. կիւզիս յա̈ր կօսօրէն սօրա քյէլ էլչի, տըսնինքյ իշ կասին. կտան ա̈յա̈ր, կյա̈լ շ ա̈պպա̈թ էթա̈նքյ նիշա՛նա տինիհյ։}

– \armenian{Սօ՛նա, մըկա քյի էտէնց ի, յիս էլ շատ կուրախանամ քյի Միսակըին փսակէլէհյ։ Աշկիս վիրա̈ն, Սօ՛նա, կօսօրէն սօրա կէթամ քըչարչըրրվիմ, բա̈՛լքյա̈ առնիմ։}

\armenian{Կօսօրէն սօրա Միրվարին կըլը՛ կէթա̈ Հա̈րթիւնիւս տո՛ւնա, նա̈ր ախչկան իւզօլու։}

\armenian{Կէթա տո՛ւռա քըթըփը՛, կիկյա̈ն տո՛ւռա կըպա̈ացին, Միրվարին կըմըննը՛ նիս, Հա̈րթիւնիւ կնգան պա̈րօվ կըտա՛ Հա̈րթիւնիւ կնիկյ Նա̈րկիզն էլա̈ նա̈ր պա̈րօվա կառնը՛}

– \armenian{Փա̈հ, պա̈րօվ իս էկյի, Միրվարը՛ բաջի, էթ վա՞ր քամին ի քյէզի պէրի տա̈}. \armenian{աղէկյ ի, հա̈րտա̈ն մէ կյա̈լիս միր տո՛ւնա։}

– \armenian{Չէ՛, Նա̈՛րգյիզ բաջը՛, մկա էլ չի՛մ էր կյա̈}, \armenian{ամմա մէ խէյր պա̈նը՝ խամա յիմ էկյի։}

– \armenian{Ասա տսնիհյ, ի՞շ խէյր պա̈նի խամա՛ իս էկյի։}

– \armenian{Նա̈՛րգյիզ բաջի, աղէյ, թօղ ասիմ. տի՛ս, դիւզ ա̈ էկյիր իմ ծիր Նուբառին էլչի, կտաս՝ տու, չիս տա՝ մի՛ տու։}

– \armenian{Միրվարը՛ բաջը՛, յիս չիմ ասէլի չիմ տա, ամմա, իրիկյիս տո՛ւնա չի. թօ քիշի՛րա իրի՛կյա կյա̈}, \armenian{նա̈րմէն խա̈բա̈ր առնիմ, տսնիմ ի՞շ կասը՛։}

\armenian{Միրվարին կասը՛ Նա̈րգյիզին}.

– \armenian{Ամմա խայիշտ իմ ա̈նէլի քյի ասիս. բա̈լքա̈ կյա̈լ շա̈պպա̈թ նշա՛նա տինիհյ, բիյօլ (մի կերպ) սօրա-յէլ խառնի՛սա̈ անիհյ։}

– \armenian{Արխէին յիլ, Մի՛րվարը բաջը, յիս կասիմ։}

\armenian{Միրվարին յէլավ էկավ տուն։}

\armenian{Քիշիրվան Նարգյիզի մա՛րթա էկավ տուն. Նարգյի՛զա̈ ասաց իւր մարթուն}.

– \armenian{Միրվարին էկիր էր միր ախճկան էլչի. ի՞նչօխ իս ա̈նէլի. կըտաս ա̈յա̈ր, վա̈՛զա̈ կյա̈լուց ջուղաբ տամ։}


\begin{adjarianpage}\label{page:286}\end{adjarianpage}% should be 286

\armenian{Մա՛րթա ասաց}.

– \armenian{Ասված շիւնա̈խավիր ա̈նի, Միսա՛կա խէլքյօվ տղա՛-յը. կտամ. վա՛ղա Միլվարին կյա̈լուց ասա կտահյ։}

\armenian{Նա̈րվաղա Միլվարին էկավ Նա̈րգյիզի կը՛շտա, ասաց – Տալէ՞հյ։}

\armenian{Նա̈րգյիզն էլ ասաց}.

– \armenian{Կտահյ, հէ՞ր չիհյ տա։ Մա՛րթըս տիւն էթա̈լէն սօրա էկավ, ասը՛մ. էն էլ ասաց կտամ։}

\armenian{Կյիրա̈կյի օ՛րա Սօնան, Միլվարի ինա̈ն Անդո՛ւնա կինա̈ցը՛ն շիրինիյ խմօլու. շիրինի՛յա խմէլէն իրիքյ շա̈պպա̈թ սօրա հազըրվան խառնիսի թա̈դա̈րիհյ տըսնօլու։ Խառնիսի թա̈դա̈րի՛քյա̈ տսնէլէն սօրա, բաշլամըշը՛ն խառնի՛ստ. ըմմըին մարթնիրին կանչը՛ն, խառնիսի խա̈բա̈ր տուվը՛ն։ ա̈ռա̈չին քյիշի՛րա̈ խինա̈ տիրը՛ն, սօրավան քյիշի՛րն էլա̈ փսա՛կա կըռը՛ն։ Փսա՛կա կռօլուց խա̈լա̈թ ին էր քիցէլի խառսու կուլօ՛խա. բույօլում (յետոյ) ասէլին էր «Ասվաս շիւնա̈խավիր ա̈նի»։}

\armenian{Խառսուն ժամտունէն խանէլէն սրա Անդուն ինա̈ն Սօնան խաղալօվ խառսուն պէրը՛ն տուն։}

\subsection{Urmia subdialect}

Adjarian's source: Communicated by Mr. Kaloust Iskenderian (\armenian{պր. Գալուստ Իսքէնդէրեան}; \translatorHD{SEA: /ɡɑlust iskʰendeɾjɑn/}), a provincial inspector of Urmia schools. 
\subsubsection{Iki Aghaj village}


\armenian{Մէ օր գնացիմ խասամ գետի յէ՛րզա. մէ պծառ կակուղ իմ էր քէլէլի (մի քիչ հանդարտ կերթայի}). \armenian{գե՛տա էնէնց ջօշմիշիր էր՝ յէրզէրէն թալէս էր ճի՛ւրա. իշկացիմ տըսամ մէ տէրտէր՝ ուր տէրօխնին, մէ կաշա (ասորի քահանա)՝ ուր տէրօխնին, մէ մալլա էլ ՝ ուր կնի՛կա. մէյն էլ մէ ծի կար կշտէ՛րա։ Նա՛ տէրտէ՛րա էյթիբար էր անէլի տանս մօ՛տա մնալ ուր կնի՛կա, նա կաշան, նա մալլան։ Մէ ծի կա տանց մօ՛տա, վէր տիկյի (պիտի) տարմօվ ճուխտ ճուխտ ըսնին մէկյէլ իրէ՛սա։ Մըկը (հիմակ) ի՞շխօ ա̈նինք վէր կնթնէ՛րա չը մնան օտար մարթու մօ՛տա։}

\subsubsection{Isalu village}

\armenian{Մալլա Նասրադի՛նա մէ օր իշէ՛րա խառիրէր տէ՛մա ՝ էթաս էր։ Կյըննաց ըլայ կյատիւկի (ձոր) վօ՛տա. մէ մարթ տար տէ}... 

\begin{adjarianpage}\label{page:287}\end{adjarianpage}% should be 287

... \armenian{առէց, ասաց. տիւն գյինաս յէս յէփ կմէռնէմ, ասա տըսնիմ։ Տէփ (յետոյ) էն իշկաց էտ մա՛րթա մի թահար մարթ ի, յէտնար – տէփ ասաց. է՛շա կյատիւկէն ըլէլիւն տիկյի օխտ տիր ա̈ռի. խէտ օխտ տիր տռէց ՝ տէփ էն վատին տիւն կմէռնէս։ Խա՛, տէփ էշէ՛րա կշէց կյատի՛ւկա. գլօ՛խա ըլէլիւն օխտ տիր է՛շա տռէց։ Մալլան ինկյավ պարզվավ, ասաց. յէս մէռամ։ Տէփ մնաց տա՛ղա. էշէ՛րա հա̈ր մէ՛կյա մէ թէխ կյընացին. մէ կյէլ ըկավ տաղ՝ էշէրէն մէ՛կյա կյէրավ. ասաց. Մալլա Նասրադի՛նա չմէռնէր՝ մըկը կյէ՛լա է՛շա չէր ուդը՛։ Տէփ մալլան ըլավ էկավ թէխ տո՛ւնա. իւր կնկյան ասաց. յէլ քէլ տուր տրկյէցին ասա վէր մալլան մէռիր ի, տանինք խօրենք իւրա։ Ըլավ կյըննաց տուր ու տըրկյէցին կանչէց. տարան տարա (զդա) խօրին։ Ասաց. մէ էրթիս թօղ, դա̈ն բա̈ դա̈ ընձի խաց պէ թալ. մնաց էտ մա՛րթա տաղ։ Տարմէն յէտի մէ կաթըրխանա էկա̈վ, տար կյէրէզմանը կուշտէն ընսնէսէն էր. էն օրն էլ տար կյընի՛կա. մա՛սալա, իւշ էր խաց պէրի. կլէօ՛խա էն ծակէն պա̈նցրացուց (յանի իշկաս էր հա՜) իշկամ խաց պէրիզ։ Կաթըրքյէ՛րա խռնան, պէռն էլ չինի աման էր. տա̈նհը ա՜մմէն տվին կօտռտին. տէփ էն կաթըրչինէ՛րա փառտին էն տէ՛ղա, ասին մէ իշկանք՝ տանք ինչէ՞ն խռնան. իշկացին մէ կյէրէզմանը վրա̈ մէ էրթըսը պէս ծայ կա̈։ Տէփ բա̈նա̈ արին (ուզեցին) մէ դէն մէ փէտ պարզէն. տէփ փէ՛տա պարզին, ը՜՜, տէփ էն դէն ծէն տըվից մալլան կյէրէզմանը՛ մէչէն. շատ մի՛ պարզէքյ, կը կը կըպնը աշկյիս. տէփ տանք ասին. հօ՛ հօ՛, կա չկա դէտ (այդ տեղէն) էն խռնէ (խրտներ}). \armenian{դէտ կլօխա խանիր ի}… \armenian{քակին տարա, խանին կաթըրչինէ՛րա. տէփ տարա բա̈նա̈ արին թըփէլ. շատ թըփը՛ն, էնղըդը թըփը՛ն ի՜ւր}…


\chapter{Khoy}\label{chapter:Khoy}

\section{Background}

\begin{adjarianpage}\label{page:288}\end{adjarianpage}% should be 288

The dialect of Khoy has an extensive distribution. It is found not only in the provinces of Khoy, Salmast, and Maku in Iran, but also in Russia in Igdir and Nakhichevan. During the large migration of Persian-Armenians in 1828, many Armenians from Salmast came and settled in Karabakh, where they founded the villages of Kori, Alighuli, Maghanjugh, Karashen in the province of Zangezur, and in villages of Alilu, Angeghakot, Kushchi-Tazakend, Uz, Mazra, Balak, Shaghat, Ltsen, Qara Klisa and Lower Qara Klisa in the province of Sisian. 

The dialect of Khoy has still not been studied. There are writings with this dialect in \citetitle{Eminian} volume 2 (\armenian{Բ}.), page 300-304 and volume 4 (\armenian{Դ}), page 343-350. What's more important are N. Ter Avetikian's \armenian{«Ոտանաւոր աշխատութիւններ եւ Նշանագրութիւն Պարսկաստանից գաղթած Խոյեցւոց բարբառով» (Վաղարշապատ} 1900) and \armenian{«Բանաստեղծութիւններ եւ Կիրակոսի հարսանիքը» (Վաղարշապատ} 1903).\footnote{\translatorHD{Unfortunately, I couldn't track down these two bibliographic items, and thus couldn't add them to the bibliography. Furthermore, the page quality makes it unclear if the fourth word is \armenian{Նշանադրութիւն} or \armenian{Նշանագրութիւն}. (?)}}

By examining these excepts, it seems that the dialect of Khoy occupies a middle position between the dialects Maragha and Van. Its grammatical structure is the same as in the Maragha dialect, but its phonological rules are like the Van dialect. In other words, the Khoy dialect is closer to Classical Armenian than Maragha is. 

Because we think it's unneeded to further discuss these simplified phenomena, we direct the reader to the subsequent text samples. 

\translatorHD{For more recent work on this dialect, see \citet[221]{Martirosyan-2019-ArmenianDialectsBigVersionRussianJournal}.}
\begin{adjarianpage}\label{page:289}\end{adjarianpage}% should be 289

\section{Text samples}

{\sampleoverview}

Adjarian's source: See \armenian{Ն. Տէր-Աւետիքեանի, Ոտանաւոր աշխատութիւններ եւ նշանագրութիւն} (\translatorHD{\armenian{նշանադրութիւն}(?)}), \armenian{էջ} 46-49. 




– \armenian{Այ մառթ, տիւ գինաս որ խետ ախչի՛գյա ճոչացավ, մառթի էթալու խասավ. տո՛ւնա չի՛ սրփի, չի՛ ավըլի, աման-չամա՛նա չի լվա, տուռվէրքյա կեխտոտ կը թօղնի. շատ էլ որ խետը իյնես, ղաստէ կզարկի կը կոտռտի, յանի ինչի՞}. – \armenian{իմանան որ տանելու խասիր ի։ Էնէնց էլ տզան. հալա մէ յէլ, քէլ մտի փայան (գոմ)՝ տե՛ս, ի՞նչ կասնաս. էն հէյվան քյալե՛րա, գյամէմքե՛րա, կովե՛րա ընչիւկ վզե՛րա թաղվիր են կվի մէ՛չա. տիւ հէնց գինաս որ Կիրակո՛սա մէզի խմա պա՞ն ի անէլի մեր տո՛ւնա ավըրիր ի. վա՛յ վայի որ ասես «ա՛յ բալամ, էտէնց չեն անի», յէտ ի դառնալի խինգ խայիր քյաշում (յիշոց) ի տալի. ասէլի «Ալլահ վարա (Աստուած տայ) զըմէն էլ խատնեն». յանի ինչ ի, իմացէք որ յէս էլ փռայվէլու խասիր էմ. կօ էտէնց, ա՛յ մառթ. մկա տիւ գինաս։}

– \armenian{Աշկըս լո՜ս. մենք փսայվանք՝ մեր կլօ՛խա յեզոտով, թող էն էլ փսայվի, բալքի մեղրոտի. էն հալա յէրէյվան քյօրփան ի. մկավուստ սաբաբ ըլենք, մէ անծոտ պուճուճակ ախչիկ էլ դար խմա ուզենք, խա՛լխա մեզի ի՞նչ կասեն. չե՞ն ասի «յանի է՞տ ինչ ղայդա էր՝ մկավուստ մեխկի տոպրա՛կա կախին էն խեղճ տղայի վզէն». յէս ղալաթ կանեմ՝ դարա սաբաբ չեմ ըլի. դար պէրնէն կալա կաթի խոտ ի իկյալի։}

(\armenian{Կիրակոսը կ՚աղաչէ մօրը)։}

– \armenian{Նանա ջան. էնը խօքութ ղուրբա՛ն նանա. տիւ իմ աղէ՛կյա ասա բաբայիս կո՛ւշտա. տավարն էլ կպախեմ, տան զըմէն պա̈նի վրան էլ սիրտ կը ցավցուցեմ. հէնց էն ղըդայի որ՝ մէ ղայիմ կպնես բաբայիս յախան, որ մէ խա՛ ասի, բօլ ի. ամա էտ էլ քեզի ասեմ որ Ղուլիենց Շահբազի ախչիկ Նիգյարէն սավայի՝ որ վիզս կռէք՝ ճոկ մառթու ախչիկ չեմ առնի հա՜։}

(\armenian{Մայրը կը համոզէ ամուսինը, որ կը պատասխանէ}).

– \armenian{Այ՛ կնիկ, չունքի որ ասէս ես, թո՛ղ քյօ խաթրն էլ խօշ ըլի. բալքի սաղ չմնացինք մեռանք. սաղ իքյան Կիրակոսին փսակենք, յէս ինան տիւ էլ դհօլ զուռնայով մէ աղէկ քէֆ անենք},

\begin{adjarianpage}\label{page:290}\end{adjarianpage}% should be 290

\armenian{գիւլաշ կպնենք. ջահնա՛մա. դարմէն յէտ ի՛նչ կըլի՝ թող ըլի. ամա տիւ է՛տ ասա, վի՞ր ախչի՛կյա ուզենք, որ համ աբուռով ըլի, համ ղայրաթով. խօրորթի ըլի, որ մեր մատէն փուշ խանի։ (Ներս կը մտնէ Կիրակոսը)։ Կի՛րակոս, ա՛յ բալամ, նանատ ասէլի որ քեզի փսակենք. մկա տիւ ի՞նչ ես ասելի. ուզե՞ս ես թէ չէ. յա վի՞ր ախչիյն ես ուզելի. մէ աղվո՛րթա ասա ըշկամ. էլ ամչընալու վախտը չի։}

– \armenian{Յես չեմ ուզելի փսայվել. նանաս ի՝ որ կպիր ի յախաս, քշեր-ցերէկ ասէլի՝ «տկի (պիտի) քեզի փսակենք». մկա տիւ գինաս, նանաս. յես էթաս եմ փայան՝ տավարին յէմ տամ. համա, նանա, էն ասածս ի հա՜, Նիգյա՛րա։}

– \armenian{Չե՞սնալի, Ղուլիենց Շահբազի ախչկա խետն ի, ընձի ասիր ի. «կուզէք էն ի, չէք ուզի՝ կլեմ կլոխ կվերցեմ՝ տնէն կէթամ. իմ ուզա՛ծա Նիգյարն ի՜, Նիգյա՛րա։}

– \armenian{Ի՞շխօ մայար Շահբազին էնէնց խասած ախչի՞կ ունի։ Մենք ռաշպար մառթ ենք, մեր տան ջահէ՛լա տկի մէ պծառ (քիչ մը) էլ ծեռով-ոտով ըլի, պանի մէչ էփած ըլի, կանոխ մեր տան պա՛նա, տաշտի բժա՛րա (քազհան) տիւս կիկյա. էնէնց ըլի որ՝ ինկերէ, տրկեցէ յէտ չմնանք։ Շատ խարսներ տսիր եմ, որ իրկըվըկէն կլոխքե՛րա տնես են պա՛ռցա, ընչանք լոս խռալով մռփես են. չէ՛ն տսէլի, ախար սափորքե՛րա տարտակ ի, ճուր տկի պերենք, ավել տկի անենք, տո՛ւնա, քիւչան զբիլի ձեռէն ըլիր ի իշխօ փողո՛ցա. տսնա՞ս ես էն Յարթենէնց խա՛րսա, մառթու դար պէ՛սա մէ խարս ըլի, թող մէ աշկն էլ կոր ըլի։}


\chapter{Artvin} \label{chapter:Artvin}
\section{Background}

\begin{adjarianpage}\label{page:291}\end{adjarianpage}% should be 291


The city and province of Artvin are found south of Batumi. This province has two smaller provinces (\armenian{գաւառակ}): Ardanuç and  Şavşat-Imerkhevi. The city of Artvin has 1200 homes with Armenian residents, of which 230 are Apostolic and the remaining are Catholic. Artvin does not have an Armenian village in its surroundings. The town (\armenian{աւան}) of Ardanuç has only 200 homes of Catholic Armenians. The following villages are in the Ardanuç province (\armenian{գաւառակ}):
\begin{itemize}
	\item Tandzut, 110 Armenian houses and 5 Catholic houses
	\item Norashen, 22 Armenian houses
\end{itemize}

The Armenian villages of the Şavşat-Imerkhevi province (\armenian{գաւառակ}) are:
\begin{itemize}
	\item Satlel (65 Catholic houses, 17 Armenian houses)
	\item Mamanelis (12 Catholic houses)
	\item Okrobakert (160 Armenian houses)
	\item Pkhikur (25 Catholic houses)
	
\end{itemize}

East of Ardanuç, there is Ardahan; while Olti is to the south. 

The aforementioned area has its own dialect which belongs to the /el/ <\armenian{ել}> branch, and it occupies a midpoint between the dialects of Karin, Khoy, and Tbilisi. 

There is no published study on this dialect, nor a manuscript line, thus the following lines are the result of my own research, gathered from migrants from Artvin in Batumi. 

\translatorHD{For more recent work on this dialect, see \citet[227]{Martirosyan-2019-ArmenianDialectsBigVersionRussianJournal}. \citet[227]{Martirosyan-2019-ArmenianDialectsBigVersionRussianJournal} reports that some work treats Artvin as unified with the Tbilisi dialect.}
\section{Phonology}
\subsection{Overview}
The sound system of the Artvin dialect is like the dialect of Tbilisi. It has three degrees of consonants. 
\subsection{Sound changes}
\subsubsection{Classical Armenian /ɑi̯/ <\armenian{այ}>}

The Classical diphthong /ɑi̯/ <\armenian{այ}> becomes /e/ <\armenian{է}> (Table \ref{tab:Artvin:phonology:soundChange:diph:ai}). 


\begin{table}[H]
	\centering
	\caption{Change from Classical Armenian /ɑi̯/ <\armenian{այ}> becomes /e/ <\armenian{է}> in the Artvin dialect}
	\label{tab:Artvin:phonology:soundChange:diph:ai} 
	\begin{tabular}{|l| ll|ll| ll|}
		\hline & \multicolumn{2}{l|}{Classical Armenian} &\multicolumn{2}{l|}{> Artvin} & \multicolumn{2}{l|}{cf. SEA} \\ 
		proximal `this' & ɑi̯s & \armenian{այս} & es & \armenian{էս} & ɑjs & \armenian{այս} \\ 
		medial `that' & ɑi̯d & \armenian{այդ} & ed & \armenian{էդ} & ɑjd & \armenian{այդ} \\ 
		distal `that' & ɑi̯n & \armenian{այն} & en & \armenian{էն} & ɑjn & \armenian{այն} \\ 
		`other' & ɑi̯l & \armenian{այլ} & el & \armenian{էլ} & ɑjl & \armenian{այլ} \\ 
		`goat' & ɑi̯t͡s & \armenian{այծ} & et͡s & \armenian{էծ} & ɑjt͡s & \armenian{այծ} \\ 
		`vineyard' &ɑi̯ɡi& \armenian{այգի} & eɡi & \armenian{էգի} &ɑjɡi& \armenian{այգի} \\
		\hline 
	\end{tabular}
\end{table}


\subsubsection{Loss of rhotic in some words}

The Classical word /hɑmɑɾ/ <\armenian{համար}> `for' has become /hɑmɑ/ <\armenian{համա}>, like in Tbilisi. 

\subsubsection{Loss of initial /v/ in `on'}

An interesting phenomenon is the loss of the sound /v/ <\armenian{վ}> from the Classical word /veɾɑi̯/ <\armenian{վերայ}> (\translatorHD{cf. SWA: /vəɾɑ/ <\armenian{վրայ}>}), which has become /ɾɑ/ <\armenian{րա}> (\ref{sent:Artvin:phono:change:vra}). 

\begin{exe}
	\ex \label{sent:Artvin:phono:change:vra}
	\begin{xlist}
		\ex \glll 
		kʰɑɾ-i ɾɑ (Artvin) \\
		kʰɑɾ-i vəɾɑ (SWA) \\
		rock-{\gen} on \\
		\trans `on the/a rock' \\
		\armenian{քարի րա, քարի վրայ}
		\ex \glll 
		d͡zi-u ɾɑ nst-ɑ-$\emptyset$ (Artvin) \\
		t͡sʰij-u vəɾɑ nəst-ɑ-$\emptyset$ (SWA) \\
		horse-{\gen} on sit-{\pst}-1{\sg} \\
		\trans `I sat on the/a horse.'\\
		\armenian{ձիու րա նստա, ձիու վրայ նստայ}
		\ex \glll t͡sɑr-i ɾɑ veɾ ɑnt͡sʰ-ɑ-v (Artvin) \\
		d͡zɑr-i vəɾɑ veɾ ɑnt͡sʰ-ɑ-v (SWA) \\ tree-{\gen} on up pass-{\pst}-3{\sg} \\
		\trans `he climbed up on the/a tree.' \\
		\armenian{ծառի րա վէր անցավ, ծառի վրայ վեր անցաւ}
		
		
	\end{xlist}
\end{exe}

\begin{adjarianpage}\label{page:292}\end{adjarianpage}% should be 292

\subsubsection{Retention of the sound /h/ <\armenian{հ}>}
The Classical sound /h/ <\armenian{հ}> does not become /χ/ <\armenian{խ}>, unlike the dialects of Maragha and Khoy.

\section{Morphology}
\subsection{Noun inflection or declension}
In declension, the ablative formative is /-men/ <\armenian{մէն}> (Table \ref{tab:Artvin:morpho:noun:abl}). 




\begin{table}[H]
	\centering
	\caption{Ablative marking in the Artvin dialect}
	\label{tab:Artvin:morpho:noun:abl}
	\begin{tabular}{|l| ll| ll|}
		\hline &\multicolumn{2}{l|}{Artvin} & \multicolumn{2}{l|}{cf. SWA} \\ 
		`from Artvin' & ɑɾtʰvinu-men & \armenian{Արթվինումէն} & ɑɾtʰvin-e & \armenian{Արթվինէ} \\ 
		?-{\abl} & sɑvetʰu-men & \armenian{Սավէթումէն} & & \\ 
		?-{\abl} & hetne-men & \armenian{հէտնէմէն} & & \\ 
		\hline 
	\end{tabular}
\end{table}


As we know, this is one of the characteristics of the Tbilisi dialect. 

Similarly, the plural genitive is the form /-eɾ-u/ <\armenian{էրու}> (\ref{tab:Artvin:morpho:noun:pl}). 

\begin{table}[H]
	\centering
	\caption{Plural genitive marking in the Artvin dialect}
	\label{tab:Artvin:morpho:noun:pl}
	\resizebox{\textwidth}{!}{%
	\begin{tabular}{|l| ll| ll|ll|}
		\hline &\multicolumn{2}{l|}{Artvin} & \multicolumn{2}{l|}{cf. SWA}& \multicolumn{2}{l|}{cf. SEA} \\ 
		`tree-{\pl}-{\gen}' & t͡sɑr-eɾ-u & \armenian{ծառէրու} & d͡zɑr-eɾ-u & \armenian{ծառերու}& t͡sɑr-eɾ-it͡sʰ & \armenian{ծառերից} \\ 
		`horse-{\pl}-{\gen}' & d͡zi-eɾ-u & \armenian{ձիէրու} & t͡sʰij-eɾ-u & \armenian{ձիերու}& d͡zij-eɾ-it͡sʰ &\armenian{ձիերից} \\ 
		\hline 
	\end{tabular}
}
\end{table}

The locative is the usual form /-um/ <\armenian{ում}> (\ref{tab:Artvin:morpho:noun:loc}). 

\begin{table}[H]
	\centering
	\caption{Locative marking in the Artvin dialect}
	\label{tab:Artvin:morpho:noun:loc}
	\begin{tabular}{|l| ll| ll|}
		\hline &\multicolumn{2}{l|}{Artvin} & \multicolumn{2}{l|}{cf. SEA} \\ 
		?-{\locgloss} & meʃ-um & \armenian{մէշում} & & \\
		`day-{\locgloss}' & oɾ-um&\armenian{օրում} & oɾ-um&\armenian{օրում} \\
		\hline 
	\end{tabular}
\end{table} 


\subsection{Verb inflection or conjugation}

\subsubsection{Periphrasis in the indicative}
Verbal conjugation differs from Tbilisi. The present formative /-um/ <\armenian{ում}> absolutely does not exist. As in the Khoy dialect, this tense is formed with the form /-elis, -eli/ <\armenian{ելիս, ելի}>.

\translatorHD{I assume the segmentation is /-e-l-i(s)/ such that the /-i(s)/ is an imperfective converb added onto an infinitive. The rationale is that SEA also has this formative /-is/ as an irregular form of the regular imperfective converb suffix /-um/. Compare SEA against Artvin in (\ref{sent:Artvin:morpho:verb:impf}).}

\begin{exe}
	\ex \label{sent:Artvin:morpho:verb:impf}
	\begin{xlist}
		\ex Artvin  \\
		\gll χos-e-l-is e-m \\
		speak-{\thgloss}-{\infgloss}-{\impfcvb} {\aux}-1{\sg} \\
		\trans `I speak.'\\
		\armenian{խօսէլիս էմ}
		\ex cf. SEA  \\
		\gll χos-um e-m \\
		speak-{\impfcvb} {\aux}-1{\sg} \\
		\trans `I speak.'\\
		\armenian{խոսում եմ}
		\ex cf. SEA  \\
		\gll t-ɑ-l-is e-m \\
		give-{\thgloss}-{\infgloss}-{\impfcvb} {\aux}-1{\sg} \\
		\trans `I give.'\\
		\armenian{տալիս եմ}
	\end{xlist}
\end{exe}

\translatorHD{Adjarian provides further examples in (\ref{sent:Artvin:morpho:verb:impfMore}).}

\begin{exe}
	\ex Artvin\label{sent:Artvin:morpho:verb:impfMore}
	\begin{xlist}
		\ex \gll pntr-e-l-is e-$\emptyset$ \\
		speak-{\thgloss}-{\infgloss}-{\impfcvb} {\aux}-3{\sg} \\
		\trans `He searches.'\\
		\armenian{փնտռէլիս է}
		\ex \gll el-n-e-l-i e-$\emptyset$ \\
		be-{\vx}-{\thgloss}-{\infgloss}-{\impfcvb} {\aux}-3{\sg} \\
		\trans `It is/becomes.'\\
		\armenian{ըլնէլի է}
		\ex \gll t͡ʃʰ-e-m kɑ/kɑji eɾtʰ-l-i \\
		{\neggloss}-{\aux}-1{\sg} can go-{\infgloss}-{\impfcvb} \\
		\trans `I cannot go.'\\ 
		\armenian{չէմ կա}, or \armenian{չէմ կայի էրթլի}
		\ex \gll voɾt\'i e-s eɾtʰ-l-i \\
		where {\aux}-2{\sg} go-{\infgloss}-{\impfcvb} \\
		\trans `Where are you going?'\\
		\armenian{վօ՞րտի էս էրթլի}
		\ex \gll t͡ʃʰ-e-m kɑ/kɑji χos-e-l-i \\
		{\neggloss}-{\aux}-1{\sg} can speak-{\thgloss}-{\infgloss}-{\impfcvb} \\
		\trans `I cannot speak.'\\ 
		\armenian{չէմ կայի խօսէլի}
		
	\end{xlist}
\end{exe}

\subsubsection{Future marking}

The future is formed with the formative /ku/ <\armenian{կու}> (\ref{sent:Artvin:morpho:verb:fut:Artvin}).

\translatorHD{In SEA, one way to form the future is to add the prefix /k(ə)-/ to the finite verb. It seems that Artvin uses a different prefix form. The placement of this prefix can differ from SEA by attaching to material besides the verb.}


 



\translatorHD{Adjarian didn't explain what the following sentences meant. Nikita Bezrukov suggested the following translations based on his knowledge of relevant dialects. I thank him for the help. Though without an Artvin speaker to double check, we can't be fully sure.  (?)}

\begin{exe}
	\ex   \label{sent:Artvin:morpho:verb:fut:Artvin}
	\begin{xlist}
\ex \begin{xlist}
			\ex Artvin \\
		\gll ku ɑʃ-i-n \\
		{\fut} look-{\thgloss}-3{\pl} \\
		\trans  `They will look' \\
		\armenian{կու աշին}
	\ex SEA\footnote{\translatorHD{The verb `to look' in SEA doesn't start with a vowel. The Artvin datapoint shows how Artvin uses a single morph /ku/ before both consonants and vowels, while SEA uses [k-, kə-] based on the type of following segment.}} \\
\gll k-ɑbr-e-n \\
{\fut}-live-{\thgloss}-3{\pl} \\
\trans  `They will live.' \\
\armenian{կապրեն}
\end{xlist}
\ex  `He will go outside.'  
\begin{xlist}
		\ex  Artvin \\ 
		\gll ku dus ɡ-ɑ-$\emptyset$ \\
	{\fut} outside come-{\thgloss}-3{\sg}  \\
	\trans 		\armenian{կու դուս գա}
\ex  SEA \\ 
\gll duɾs kə-ɡ-ɑ-$\emptyset$ \\
outside {\fut}- come-{\thgloss}-3{\sg}  \\
\trans 		\armenian{դուրս կգա}
\end{xlist}	
		\ex `They will search.'
		\begin{xlist}
			\ex Artvin \\
			 \gll ku pntr-i-n \\
			{\fut} search-{\thgloss}-3{\pl} \\
			\trans  			\armenian{կու փնտռին}
			\ex SEA \\
			\gll kə-pʰəntɾ-e-n \\
			{\fut}-search-{\thgloss}-3{\pl} \\
			\trans `They will search.' \\
			\armenian{կփնտրեն}
			 
		\end{xlist}
	\end{xlist}
\end{exe}




\subsubsection{Theme vowel changes}

In both the perfective and the future, the Classical theme vowel /e/ <\armenian{ե}> changes to /i/ <\armenian{ի}> (\ref{sent:Artvin:morpho:verb:theme}). 

\begin{exe}
	\ex Artvin \label{sent:Artvin:morpho:verb:theme}
	\begin{xlist}
		\ex \gll ɑʃ-i-t͡sʰ \\
		look-{\thgloss}-{\aor} \\
		\trans  `He looked.'\footnote{\translatorHD{I thank Nikita Bezrukov for translating this verb.}}\\ 
		\armenian{աշից}
		\ex \gll pntr-i-t͡sʰ \\
		search-{\thgloss}-{\aor} \\
		\trans `He searched.' \label{sent:Artvin:morpho:verb:theme:search}\\
		\armenian{փնտռից}
		\ex \gll ku ɑʃ-i-n \\
		{\fut} look-{\thgloss}-3{\pl} \\
		\trans   `They were going to look.' \\
		\armenian{կու աշին}. 
	\end{xlist}
\end{exe}


\translatorHD{Unfortunately, Adjarian's data is too limited to make a more meaningful description or comparison with SEA/SWA. But essentially, what Adjarian describes is that the theme vowel /e/ is replaced by /i/ in some morphological contexts. Compare `he searched' from (\ref{sent:Artvin:morpho:verb:theme:search}) against SEA (\ref{sent:Artvin:morpho:verb:theme:search:SEA}). 
}

\begin{exe}
	\ex cf. SEA \label{sent:Artvin:morpho:verb:theme:search:SEA} \\
	\gll pəntɾ-e-t͡sʰ \\
	search-{\thgloss}-{\aor} \\
	\trans `He searched.' \\
	\armenian{փնտրեց}
\end{exe}

\subsubsection{Imperative}

An especially interesting form is the second type of imperative (\ref{sent:Artvin:morpho:verb:imp:artvin}).

\translatorHD{Note the unclear potential use of the auxiliary. (?)}

\begin{exe}
	\ex Artvin \label{sent:Artvin:morpho:verb:imp:artvin}
	\begin{xlist}
		\ex \gll ɡɾ-\'i-s ɑ \\
		write-{\thgloss}-2{\sg} ? \\
		\trans `Write!' \\
		\armenian{գրի՛ս ա}
		\ex \gll χos-\'i-s ɑ \\
		speak-{\thgloss}-2{\sg} ? \\
		\trans `Speak!' \\
		\armenian{խօսի՛ս ա}
	\end{xlist}
\end{exe}

These correspond to the Istanbul interrogative-like imperatives (\ref{sent:Artvin:morpho:verb:imp:istanbul}). 


\begin{exe}
	\ex Istanbul, when read as SWA words \label{sent:Artvin:morpho:verb:imp:istanbul}
	\begin{xlist}
		\ex \gll t͡ʃə-kʰəɾ-\'e-s \\
		{\neggloss}-write-{\thgloss}-2{\sg}   \\
		\trans `Don't write!' \\
		\armenian{չգրե՞ս}
		\ex \gll t͡ʃə-χos-\'i-s \\
		{\neggloss}-speak-{\thgloss}-2{\sg}   \\
		\trans `Don't speak!' \\
		\armenian{չխօսի՞ս} 
	\end{xlist}
\end{exe}



\translatorHD{I think what he means is that these Artvin imperative seem to be derived from subjunctive verbs; the Istanbul verbs seem subjunctive based on how they would be interpreted in SWA.}

\section{Text samples}

{\sampleoverview}

\armenian{Առաջ Արթվին շէնլիկը տասնըհինգ տուն է էղէ. բօլօրը մէշա. էն մէշումը Սավէթումէն կու աշին օ՛րը (որ) Արթվինումէն մուխ կու դուս գա. գուգան կու փնտռին խիտը (վրաց. կամուրջ}). \armenian{չէն կա (չեն կրնար) գտնի օ՛րա Ճօրօխը անցնին. էտէվ մէկ ավջին գէյիգի հէտնէմէն գալիս է օ՛րը զարնէ. առաչէվան կայբ է ըլնէլի. կայբ էղած վախտին փնտռրէ՛լիս է վօ՞րանց գնաց։ Աշից օրը խիտը գտավ. խիտն էլ փուրցէլը (վրց. մացառ թէ՞ բաղեղ) փաթըթած է. է՛նղադուր արավ օ՛րա խիտը անցավ էնթին։ Վէր անցավ օրմընումը, փնտռից ու շէնլիկի տէղը գտաւ։ Իշտէ էնդօր էտէվ, էֆէնդիմ, օրմանը կօտրէցին, էնդէղը քաղաք շինէցին, իշտէ էնդէղը էղավ Արթվին։} 

\armenian{Էն գտնօղ մարթու անունն էլ Արութէն է էղէ, էնդօր վրա դրէլ է Արթվին։}

\begin{adjarianpage}\label{page:293}\end{adjarianpage}% should be 293

\translatorHD{Note that Adjarian had a note here about the Armenian diaspora. I moved it to \S\ref{sec:Branches:excluded}.} 


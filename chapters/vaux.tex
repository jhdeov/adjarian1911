\section{Introduction}


Armenian culture is no different from any other in being intimately intertwined with the linguistic intricacies of its language; if, indeed, we can refer to a single Armenian language.\footnote{This paper developed from a paper delivered at the ``Armenian Linguistics in a Modern Perspective'' workshop held at the University of Leiden on March 26, 2003. I have followed the transliteration and transcription conventions of the Adjarian translation. I use IPA for my own renderings from my fieldwork. Thanks to Rivka Hyland, Ollie Sayeed, and Paige Wallace for comments on earlier drafts of this text.} When studying the language, we have to deal (at least) with Standard Western Armenian (SWA), Standard Eastern Armenian (SEA), Classical Armenian, and Middle Armenian.  The Armenian language encompasses more than just these literary varieties, however.  In addition to a number of arguably distinct languages,\footnote{If we employ the working definition that two linguistic varieties are separate languages if they are not mutually intelligible. \todo{ref}} such as   Zok or /zokeɾen/ <\armenian{զոկերէն}> (the language of Agulis \todo{refs}),  /kʲeʁɑt͡sʰǝneɾen/ <\armenian{Գյեղացըներէն}> (the language of Zeytun \todo{refs}),  /kʰesbǝnuokʰ/ <\armenian{Քեսպընուոք}> (the language of Kesab \todo{refs}),\footnote{  Also called in different sources /kʰ(ɾ)isdinuokʰ/ `Christian language' (the same name used by speakers of Svedia/Musaler dialect), and in Roman-script contexts ``Kesbenok,  Kes(s)aberen, or Kesspeneuts.''} and Lomavren, the language of the Armenian gypsies, there are more than 120 distinct dialects of Armenian documented in the literature.  Like the micro-cultures they represent, each of these dialects has its own unique archaisms and innovations.  For a sense of how much these dialects can differ from one another, consider the translations in \todo{(1)} of \todo{Hovhannēs T‘umanean’s} story ``The Liar,''  drawn from three geographic extremes of the Armenian-speaking world: Köprücü (northeast Turkey), Stepanakert (Karabagh), and Qaladuran (northwest Syria).

\begin{exe}
\ex  ``The Liar'',  /sutɑsɑnǝ/ <\armenian{Սուտասանը}>
\begin{xlist}
	\ex English: \\
ONCE UPON A TIME there lived a king. This king announced throughout the land:  
``I shall give half my kingdom to the man who can tell a lie that I admit to be a lie.''  
A shepherd came and said, ``Long May Your Majesty Reign! My father had a cudgel which he used to reach out from here and stir the stars in the sky.  
``That’s possible,'' answered the King. ``My grandfather had a pipe. He used to put one end of it in his mouth and stretch the other up to the sun to light it.''  
The man went out scratching his head.  
A tailor came and said, ``I am sorry, O King, to have come so late. I had intended to come earlier. But there was a heavy storm yesterday, and lightning rent the sky. I’ve been patching it up.'' 
``Oh, very commendable,'' said the King, ``but you haven’t patched it very well, for it rained a little again this morning.
This man also went away empty-handed.  
A peasant came in with a bag on his shoulder. ``What do you want, my good man?'' asked the King. ``You owe me a bag of gold. I have come for it.'' ``A bag of gold!'' exclaimed the King astonished. ``That s a lie. I do not owe you anything.''  
``All right. It is a lie. Then give me half your kingdom.''
``No, no. You’re quite right. It’s not a lie,'' the king tried to correct himself.   
``So I am telling the truth. Then give me the bag of gold.''

\ex Version 1: Köprücü,\footnote{  Narrated by Temel in Watertown, MA in 1995.}  a Hamshen subdialect, /χɑpʰoʁe/ \\

ɡonːɑ ɡu t͡ʃʰɡonːɑ ɡu mekʰ kʰǝɹɑl me ɡonːɑ ɡu.  ɑs kʰǝɹɑ́les uúne milːetʰin ɑnons ɡenɑ:  ‘vov u ɔ́jle biɹ χɑpʰɑ́ɑnɑ jes ɑsim χɑpʰelu t͡ʃʰɑ, im kʰǝɹɑlːouz ɡɛ́se ɡɑɹnu.’  kʰu kʰɑ hojiv me ɡɑsɑ, ‘kʰǝɹɑ́le sɑʁ elːi, im bɑbɑs hɑst meɡ pʰiɹ me uneɹ, ɑn pʰíɹe isti eɹɡent͡sʰenelov hɑvɑjin ɑstɑʁníje χɑɹne ɡuɹ.’  ‘elːi ɡɑɑ,’ kʰǝɹɑ́le d͡ʒevɑb ɡu dɑ.  ‘im bɑbs ɑ mekʰ pʰipʰo me uneɹ meɡ d͡zɑ́ɹe pʰiɑ́ne tʰene ɡuɹ meɡ ɑl d͡zɑ́ɹe ɑɑkʰɑɡɑn kʰole ɡuɹ.’  χɑpʰɔ́ʁe kʰelɔ́χe kʰijelov heɹunɑ ɡu.  tʰeɹzi me kʰu kʰɑ ɡɑsɑ, ‘ɑf ɑɑ́, kʰǝɹɑl, tʰez me kʰɑ ɑmɑ uʃɑt͡sʰɑ. ejekʰ ʃɑtʰ t͡ʃʰɑχ ejev ʃimʃekʰnɛ́ɹe ɡɑd͡zɡedet͡sʰin hɑvɑn bɑdeɹet͡sʰɑv ɡɑɹɡɑuʃ kʰenɑɑd͡z e.’  ‘hɑ, bedkʰ e ɑʁɑ́d͡zues,’ kʰǝɹɑ́le ɡɑsɑ, ‘ɑmɑ soj ɡɑɹɡɛ́de t͡ʃʰɑɹt͡sʰɑd͡z es.  ɑsɑkʰvɑn ɑ kʰit͡ʃʰ me t͡ʃʰɑχ ejev.’ tʰeɹzin ɑ tʰus kʰelːɑ.  ɑχkʰɑd mɑɹtʰ me met͡ʃʰnuz mednu ɡu tʰevin dɑ́ɡe kʰovɑ me.  kʰǝɹɑ́le hɑɹt͡sʰenɑ ɡu, ‘tʰun int͡ʃʰ kʰuzes tʰɑ?’  ‘ind͡zi meɡ kʰovɑ me ɑltʰun dɑlikʰ unes; donuʃ eɡɑd͡z im.’  ʃɑʃiɹmiʃ ɡenɑ kʰǝɹɑ́le, ‘meɡ kʰovɑ me?  χɑpʰes ɡu, jes kʰezi ɑltʰun dɑlikʰ t͡ʃʰunim.’  ‘mɑdem kʰi χɑpʰi ɡum nɑ, kʰǝɹɑlːǝʁǝtʰ ɡɛ́se du.’  ‘t͡ʃʰɑ t͡ʃʰɑ, ʃidɑɡ χɑɹbe ɡus,’ kʰǝɹɑ́le ɑstɑ́d͡ze tʰɑɹt͡sʰenɑ ɡu. ‘ɑstɑ́d͡zes ʃidɑɡ ɑ tʰɑ ɑ nɑ, du kʰovɑ me ɑltʰúne.’  t͡ʃʰɑɹesiz kʰǝɹɑ́le kʰovɑ me ɑltʰúne ɡu dɑ.

\todo{mention ɹ in intro from bert's hamshen work}

\ex Version 2:  Stepanakert  (Karabakh),\footnote{Narrated by Vladimir in Cambridge, MA on September 10, 1995.} /sot \'ɑsoʁǝ/

inúm ɑ t͡ʃʰinúm min tʰɑkʰɑvɔ́ɾ ɑ inúm.  es tʰɑkʰɑvɔ́ɾǝ ýæn jeɾkɾúm tɑɾɑ́t͡sum ɑ:  ‘hu min sot ɑsí, veɾ jes ɑ́sim sot ɑ, im tʰǝkʰɑvoɾutʰjɑ́n kɛ́sǝ yɾɑ́n kǝtɑ́m.’  min t͡ʃʰóbɑn kʲɑm ɑ ɑsúm, ‘tʰɑkʰɑvɛ́ɾǝ ɑ́pɾɑt͡s kenɑ́.  im hæɾ min kɔ́pɑl ɑ ynet͡sʰɑl veɾ estɛ́ʁit͡sʰ mɛ́knum eɾ jeɾkinkʰi ɑstʁɛ́ɾneɾ χɑ́rnum.’  ‘kǝpǝtɑ́hi,’ tʰɑkʰɑvɛ́ɾǝ ɑsum ɑ.  ‘im pɑpn el min t͡ʃʰúbuχ ɔ́neɾ min t͡sɛ́ɾǝ piɾɑnín eɾ tinúm en min t͡seɾn eɾ mɛ́knum ɑ ɑɾevít͡sʰ vɑ́ɾum.’  sot ɑ́soʁǝ kǝlóχǝ kɔ́ɾelɑv tys æ kʰʲinum.  min dɛ́ɾd͡zɑk ɑ kʲɑm ɑ́sum, ‘neɾʁutʰjún tʰǝkʰɑ́veɾ, jes ɛ́kʰyt͡sʰ pɪti kʲǽi uʃɑt͡sʰɑ.  jeɾɛ́k ʃɑt tʰoɾ jekɑ́v.  kǝt͡síkǝɾɑknéɾǝ tǝɾɑket͡sʰín jɛ́ɾkinkʰǝ t͡ʃǝʁvɛ́l eɾ kʲet͡sʰɑl eí kǝɾkǝtɛ́lu.’  ‘hɑ, lɑv ǝs ǝɾɑ́l,’  ɑsum ɑ tʰǝkʰɑ́veɾǝ, ‘bɑjt͡sʰ lɑv t͡ʃʰǝs kǝɾkɑ́tɑl.  es ǝrɑ́vut el mi χǝɾɛ́ɡʲ ɑnd͡zɾɛ́v jɛ́kɑ.’  es el ɑ tys kʲinúm.  min kʰʲɑ́sib ʃinɑ́t͡sʰi ɑ ni mǝtnúm, kɔ́tǝ kúrnǝ tɑ́kin.  ‘tu hint͡ʃʰ ǝs ɔ́zum ɑj mɑɾtʰ?’ tʰǝkʰɑ́veɾ hɑɾt͡sʰnúm ɑ.  ‘ind͡z min kot vǝ́ski ǝs pɑtkʰ.  jɛ́kɑl ǝm tɑním.’  ‘mi kot vǝski?’  zǝɾmɑ́num ɑ tʰǝkʰɑ́veɾǝ, ‘sot ǝs ɑsúm.  jes kʰez vǝ́ski t͡ʃʰem pɑtkʰ.’  ‘devɛ́ɾ sot ǝm ɑsúm, tʰǝkʰɑvoɾutʰjɑ́nǝtʰ kesǝ ind͡z to.’  ‘t͡ʃʰe t͡ʃʰe, dyz ǝs ɑsúm,’ χɔ́skʰǝ pʰɔ́χum ɑ tʰǝkʰáveɾǝ.  ‘devɛ́ɾ dyz um ɑsum, mi kot vǝskis tuɾ.’

\ex Version 3:    Qɑlɑduɾɑn  (Kesab),\footnote{Narrated by Mr Manjikian in Watertown, MA in January 1995.} /sot χusuǝʁǝ/

ɡit͡sʰiɾ i t͡ʃʰi ɡit͡sʰiɾ i tʰækʰævyɾ mǝ.  æt tʰækʰævyɾǝ ɡesi eɾ ihɑlen:  ‘uv ǝɾ mit͡s seœd mǝ χysæ, is esim seœd i, tʰekʰevyɾytʰynes ɡesǝ ɡudum eɾ.’  ɡu kʰu huvev mǝ ɡesi, ‘tʰekʰevyɾǝ sɑʁ ɡinœ.  im dyǝdǝ ɡuniǝɾ huest veɾyt͡sʰ mǝ, zæn ɡidnæn ɡǝ minɡieɾ tǝɾviɾ zeɾɡinkʰen zɑɾɑstɑʁ ɡinːiɾǝ ɡǝ χeɾinɡeɾ.’  ‘ɡɑɾnu ǝnːu,’  ɡesi tʰekʰevyɾy, ‘im duedǝ ɡunieɾ qelun mǝ mieɡ d͡zɑɾkʰǝ biɾunǝ ɡǝ tǝniǝɾ miæl d͡zɑɾkʰǝ ɡǝɾɡent͡sʰǝniǝɾ eɾiven ɡǝ vɑɾieɾ zǝ.’  sot χusuǝʁǝ ɡǝ hɑɾvǝno kʰǝloχǝ kʰǝɾiluǝn.  ɡu kʰu tʰeɾzej mǝ, ɡesi, ‘neɾoʁutʰjun tʰekʰevyɾ, is evilæ uχtuǝv bidæ ukʰeɾem himit͡sʰæ iɾiekʰ ʃuǝt end͡zǝɾiv iɡikʰ.  ɡǝd͡zɡǝdilien ɡɑɾdilien eɾɡenkʰǝ bɑdǝɾit͡sʰuǝv ɡet͡sʰuǝd͡z eɾem zǝ ɡeɾɡedilǝ.’  ‘ʃuǝt æʁyɾ iɾuǝd͡z is,’ ɡesi tʰekʰevyɾǝ, ‘ǝmːǝ æʁyɾ t͡ʃʰi ɡeɾɡeduǝd͡z eɾe zǝ. æs sǝbæχtæn kʰejt͡ʃʰ mǝ end͡zǝɾiv iɡikʰ.’  tʰeɾzejn ili tuɾit͡sʰi ɡǝnːi.  nikʰsi ɡǝ mǝnːæ æχkuǝt keʁet͡sʰej mǝ ɡulued͡zǝ tʰiven duǝkʰǝ.  ‘teon t͡ʃʰej hæ uzis ɑj moɾtʰ?’ ɡǝ hɑɾt͡sʰǝni tʰekʰevyɾǝ.  ‘ǝnd͡zi mieɡ ɡuluǝk isɡæ buoɾdkʰ is.  iɡuǝd͡z im zǝ denilǝ.’  ‘ɡuluǝm isɡæ?’  ɡǝ ʃɑʃmǝʃǝnːo tʰekʰevyɾǝ, ‘seœd hæ χusejs.  is kʰi isɡæ buoɾdkʰ t͡ʃʰim.’  ‘æn kʰi seœd hæ χusejm, tʰekʰevuɾutʰunetʰ ɡiese deo.’  ‘t͡ʃʰi t͡ʃʰi, uʁuɾtʰ hæ χusejs,’ χueskʰǝ hæ pʰeχi tʰekʰevyɾǝ.  ‘tʰi uʁuɾtʰ i im esuǝd͡zǝ, deo ind͡zi ɡuluǝmǝ isɡæ.’  umodǝ ɡǝdǝɾiluǝn tʰekʰevyɾǝ ɡu dued͡zeɾ ɡuluǝmǝ isɡæ.
\end{xlist}
	\end{exe}

  
The study of Armenian dialects ties together the present and the past as well as many different disciplines -- linguistics, history, folklore, anthropology, and music -- and geographical areas as diverse as Syria, Abkhazia, Iran, and India. Yet the field faces three challenges: the tools and methods of the field are still situated firmly in the 19th century;\footnote{Bernard Coulie observed in his presentation at the ``Armenian Linguistics in a Modern Perspective'' conference, ``Language and Text: The Material Basis of Our Research of Classical Armenian'' (subsequently published as Coulie 2014) that the situation is similar in Armenian manuscript studies and chronology. }\todo{refs} like many academic pursuits, Armenian dialectology is under threat as a result of economic developments in post-Soviet Armenia and Karabagh; and most of the non-standard dialects of Armenian are in imminent danger of disappearing altogether. 


These are the problems I address in this chapter. After surveying the present and past of the field, which revolve primarily around the activities of Hratchia Adjarian, I suggest some ways in which Armenian dialectology can make use of advances that have been made in general dialectology since Adjarian’s time, and consider what steps can be taken to document and perhaps even stabilize or revitalize the dialects before it is too late.




\section{The state of Armenian vs. American dialectology}


The central concern of work to date on Armenian dialects by scholars in Armenia, as well as by the majority of more recent Western scholars such as Andrzej Pisowicz and Georges Dumézil, has been the collection of word lists and short texts and the evaluation of their etymological significance relative to Classical Armenian. The manual for collection of Armenian dialect materials published by the Dialectological Institute (Muradyan et al. 1977 \todo{ref}) and the Institute’s activities since that time indicate that it intends to continue this course of action. 	

Such work is valuable and provides the groundwork for a wide range of illuminating research. However, the methodology employed by Muradyan et al. directly reflects the state of European linguistics and dialectology in the late nineteenth century, when the leading Armenian linguist, Hratchia Adjarian, studied in Paris under the Indo-Europeanist Antoine Meillet. Many important developments in linguistics and dialectology have occurred since that time, particularly due to the theoretical linguistic work of Noam Chomsky beginning in the 1950s and the sociolinguistic work of William Labov beginning in the 1960s. Since the appearance of the new concepts and methodologies developed by these scholars, linguistics and dialectology have advanced at great speed; it seems only logical, then, that dialectologists working on Armenian should benefit from these new methods. 	

The need for more comprehensive and contemporary methods of investigation is particularly imperative given the impending extinction of many (and perhaps most) of the remaining non-standard Armenian dialects. Soon it will no longer be possible to compensate for omissions in the work of earlier researchers by consulting native dialect speakers. In the remainder of this section I sketch the history of traditional work on Armenian dialects, contrast this with developments in Western dialectology in the twentieth century, and suggest some ways in which incorporation of the latter can enhance the former.


\subsection{Armenian dialectology}
	
\subsubsection{The past}
	
\subsubsubsection{Proto-dialectology}

\todo{paragraph?}

In a sense, the first known Armenian dialectologist was \todo{name}Mesrob’s student Koriwn\todo{name}. In his fifth-century biography of Mesrob, Koriwn mentions the existence of Armenian dialects in Siwnik‘\todo{name} and the land of the Medes (i.e. Agulis in this context). The fifth-century theologian \todo{name}Eznik\footnote{cf. Blanchard and Young 1998 for an English translation of Eznik’s text. \todo{ref}} observed  in turn that in his time there were two dialects of Armenian: one in the north and one in the south. Speaking of the Classical Armenian form /ɑi̯s/ <\armenian{այս}>, he says:

\begin{exe}
	\ex Classical Armenian \\
	\gll  Յորժամ մեք ասեմք թե սիք շնչե, ստորնեայք ասեն` այս շնչե \todo{check spelling} \\
	yoržam mek‘ asemk‘ t‘e sik‘ šnč‘e, storneayk‘ asen` ays šnč‘e	
	‘when we (i.e. Armenians of Koghb and northern Armenia) say sik‘ šnč‘e [for ‘a wind is blowing’], the southern [Armenians] say ays šnč‘e.’ \todo{fix all this}
	
\end{exe}


The form   /ɑi̯s/ <\armenian{այս}>  was in fact used in the north, but in the meaning `evil spirit' or `demon'; the southern Armenians preferred   /deu̯/ <\armenian{դեւ}>  for this function.	

Armenian dialects are next mentioned by the early Armenian grammarians; the sixth century translation of Dionysus Thrax refers to the dialect of \todo{Gordayk‘}, for example, and the eighth century commentary of \todo{Step‘anos of Siwnik‘} mentions eight dialects: \todo{Korchayk‘, Tayk‘, Khut‘ayk‘, Fourth Armenia, Sperk‘, Siwnik‘, Arts‘akh, and Central Armenia (Ararat).}


A significant amount of subsequent work, notably by Weitenberg and Jahukyan, has been devoted to identifying the nature and extent of dialect variation in the Classical Armenian period; this will be addressed in sections \todo{2.1.1.4 and 2.1.2.}

\subsubsubsection{The beginnings of Armenian dialectology: the 18th century}

\todo{paragraph?}

The first published analysis of significant amounts of Armenian dialect material was \todo{Rivola's} (1633) \textit{Dictionarium Armeno-Latinum}, which contained numerous lexical items from New Julfa and other dialects. Rivola did not distinguish between the dialects he employed, however, as \todo{Adjarian (1940)} points out.	

\todo{Schrœder's} \textit{Thesaurus Linguae Armenicae}, published in Amsterdam in 1711, provides extensive samples of Agulis, New Julfa, and other Armenian dialects, and documents contrasts between Civil Armenian (a form of proto-Modern Armenian; cf. \todo{Parnassian 1985}) and the classical language. 


\subsubsubsection{The heyday of Armenian dialectology: the nineteenth century}
	

\todo{paragraph?}

The nineteenth century saw an explosion of interest in the dialects of Armenian. \todo{Chahan Cirbied} (\armenian{Շահան Ջրպետեան}), a Tokat Armenian who early in his career taught a young \todo{Sunduk‘ean} in Tblisi/Tiflis and later was professor of Armenian in Paris, devoted an entire section of his \todo{1823} \textit{Grammaire de la Langue Arménienne} to some thirty nonliterary dialects of Armenian. Like Rivola, though, he generally does not specify the dialect from which he takes the individual forms he cites. In 1850, \todo{Shirmazanean} published his \textit{Patmut‘iwnk‘ i Lezu Erewanc‘oc‘} `Stories in the Language of the Yerevantsis,'  which discussed general features of the Yerevan dialect. In 1852, \todo{Hakhverdean’s Sayat‘-Nova} appeared in Moscow; the first portion of this landmark publication was devoted to the grammar of \todo{Hakhverdean and Sayat‘-Nova’s} native dialect of Tblisi. \todo{Aytĕnean’s} important \textit{K‘nnakan k‘erakanut‘iwn ashkharhabar kam ardi hayerēn lezui} [Critical Grammar of the Modern Armenian Language] appeared in Vienna in 1866. \todo{Aytĕnean} postulated the existence of Armenian dialects already in the fifth century, based on \todo{Koriwn’s} aforementioned reference to Armenian dialects in \todo{Siwnik‘} and the land of the Medes. He divided the modern dialects into four groups: Eastern Turkey, Western Turkey, Europe, and Russia/Persia/India. Petermann’s study of the Agulis dialect appeared in Berlin in the same year.	

Spurred on in part by the nationalism and romanticism sweeping through Europe, Armenian dialectology reached its zenith in the second half of the nineteenth century. Just as the brothers Grimm scoured the 19th-century German countryside in search of pure and ancient Germanic folklore, Armenians such as \todo{Garegin Sruandzteants‘} returned to their village roots in search of an epic narrative that could rival those being produced in neighboring countries. The discovery by \todo{Sruandzteants‘} of the Sasun epic (cf. his \textit{Mananay}, \todo{1876}) is one of the many fruits of this halcyon period of intellectual curiosity. Numerous works on the language and ethnography of Armenian villages followed in quick succession, documented in books such as \todo{Sedrakean’s} \textit{K‘nar Mshets‘wots‘ ew Vanets‘wots‘} (1874), \todo{Allahvertean’s} \textit{Ulnia kam Zēyt‘un} (1884), \todo{Sherents‘’s} \textit{Vana Saz} (1885, 1899), \todo{Haykuni’s} \textit{Ēminean Azgakrakan Zhoghovatsu} (1900-13), and \todo{Lalayean’s} \textit{Vaspurakan} (1914), and in journals such as \todo{Murch, Biwrakn, and Azgagrakan Handēs}.


The first Armenian dialectologist in the modern sense was \todo{Kerovbe Patkanov/Patkanean}, whose \todo{Izsledovanie o Dialektax’ Armjanskago Jazyka} [Treatise on the Dialects of the Armenian Language] appeared in Saint Petersburg in 1869. His monograph provided phonological and morphological sketches of the dialects of \todo{Astrakhan,} Yerevan, Tbilisi, Agulis, Karabakh, Khoy, New Julfa, Mush, Poland, and New Nakhichevan.\footnote{\todo{cit}He also produced longer separate treatments of the dialects of Agulis (1866) and New Nakhichevan and Mush (1875).} Soon thereafter, a veritable flood of dialect grammars began to pour out of Europe and Armenia; notable examples include \todo{Petermann’s} 1867 grammar of the Tblisi dialect, \todo{Sargseants‘’s} 1883 grammar of the Agulis dialect, \todo{Hanusz’s} 1886 dictionary and 1889 grammar of Polish Armenian, \todo{Tomson’s grammars of the Akhalts‘kha and Tbilisi} dialects (1887, 1890), \todo{Msereants‘’s} various works dealing with the Mush dialect (1897), and \todo{Gazanchean’s} monograph on the Tokat dialect (1899).


The high point of this period (and of all periods of Armenian dialectology) was the work of Adjarian, who studied with the French Indo-Europeanist Antoine Meillet in the 1890s, and was probably responsible for Meillet’s deep interest in Armenian. Adjarian was the first scholar to bring contemporary European linguistic tools to bear on the manifold intricacies of the Armenian dialects. Conversely, he made use of Armenian dialect materials to develop a phonetic concept that has occupied a central place in the Western study of phonetics since the 1960s, Voice Onset Time \todo{(Braun 2013; cf. Lisker and Abramson 1964)}. Adjarian was also unusually productive. Not only did he produce dozens of groundbreaking books on Armenian dialects and on the Armenian language in general; Adjarian also single-handedly founded the modern schools of Armenian linguistics and dialectology that still survive in Armenia today.

Adjarian established a method for collecting, presenting, and analysing linguistic materials that drew directly on Western linguistic work of the time. Though the work of the structuralists, Chomskyans, and variationists replaced it in the West in the twentieth century, armenologists today continue for the most part to use Adjarian’s method, particularly in Armenia. At the time Adjarian was learning linguistics in Paris in the late nineteenth century, European linguists were primarily interested in using linguistic data for historical purposes, especially the reconstruction of earlier stages of the language or language family under consideration. Dialect variation was understood to play a central role in this quest, insofar as it presented material for classification and subgrouping, thereby enabling triangulation to earlier historical stages, and as it contained archaisms that directly revealed aspects of the past, untainted (it was thought) by the vagaries of the modern world and literary languages. Because of this focus on historical questions in his intellectual milieu, Adjarian’s model for presenting dialect material revolved around the comparison of dialect forms to their antecedents in Classical Armenian, as shown in this sample from his grammar of the New Julfa dialect:

\begin{exe}
	\todo{fix}
	(3) 	§9 from Adjarian 1940 
	“Classical Armenian a normally remains a in all positions in the New Julfa dialect, e.g. ալիւր aliwr > ɑluɾ [flour], աղուէս aɫuēs > ɑʁvɛs [fox], առնել aṙnel > ɑniɛl [do], ասեղն aseɫn > ɑsuʁ [needle], աղանձ aɫanj > ɑʁɑndz [roast (n)], բարակ barak > bhɑɾɑk [thin], բարձր barjr > bhɑntshəɾ [high], բահ bah > bhɑχ [spade], ծիրան ciran > tsiɾɑn [apricot].”
	
\end{exe}

In order to facilitate comparisons of this type, and to make them consistent across dialects, Adjarian constructed a fixed word list that he employed for all of his field work and resultant dialect grammars. He would then extract a basic set of historical phonetic and morphological changes from this word list, append a few dialect texts with a handful of grammatical and lexical notes, and add an introduction briefly discussing the previous literature that he was able to find on and in the dialect. As was the case with his European teachers and colleagues, Adjarian was not overly concerned with identifying what individuals or texts each of his forms came from, nor did he consider the significant range of subdialectal and idiolectal variation that one finds in every dialect. For Adjarian, as for his contemporaries, dialects were for the most part idealized monoliths consisting of forms produced by older speakers that conformed to his historical expectations.


Linguistic work at the time also tended to neglect synchronic analysis of the grammatical system of the dialect at that point in time, and largely ignored acoustic phonetics, phonology (synchronic rules, rather than historical changes), syntax, and sociolinguistic nuances. The model that Adjarian developed for collecting and analysing Armenian dialect material was no exception to this. (\todo{Samvel Antʻosyan}, for example, divides his 1961 treatment of the Kayseri dialect into four sections: [historical] phonology, morphology, lexicon, and text samples.) Adjarian’s methods, as still practiced in Armenia today, provide an interesting window into the state of linguistic and dialectological research in Europe in the late nineteenth century.

\subsubsubsection{The fall of Armenian dialects and dialectology: the 20th century}
\todo{paragraph}

At the beginning of the twentieth century, Armenian dialectology was still at its peak. Meillet’s students Adjarian, \todo{Dawit‘ Bēk, Maxudianz,} and Benveniste produced an abundance of excellent dialectological studies. The school that Adjarian began in Armenia would ultimately generate such productive dialectologists as \todo{Jahukyan, Gharibyan, Grigoryan, the two Muradyans, and Adjarian’s niece, Amalya Khatchatrian.}


Despite the widespread interest in Armenian dialectology and folklore at the turn of the century, there were far more Armenian villages, dialects, songs, stories, and so on than could be collected by the army of amateur and professional armenologists. This was less true after 1915, when the Armenian populations of most of these villages were eradicated. Dialectological work has continued since 1915, with notable products such as \todo{Malkhaseants‘’s Hayeren Bats‘adrakan Baṛaran} (1944), the dialect grammars and dialect survey produced by the Institute of Dialectology in Yerevan (most notably Adjarian’s \todo{1954} treatment of the Van dialect and \todo{Davt‘yan’s 1966 atlas of Karabagh dialects}), \todo{Dumézil’s} studies of the Hamshen and Musaler dialects, \todo{Pisowicz’ grammar of the P‘arpi} dialect, and the memorial volumes produced by the various compatriotic organizations in the United States (e.g. \todo{Galustean 1934} for Marash). Popular interest in the language and culture of village Armenia has waned, though, a fact reflected in the evanescence of almost all of the compatriotic societies.

\subsubsection{The present}

At the present time, little is being published on Armenian dialectology. The Institute of Dialectology in Yerevan is still technically active, and engaged in the collection of materials for its dialect atlas. In practice, however, these plans appear to have been put on hold indefinitely, as Armenia and Karabkgh have been attending to more pressing matters. (One notable exception is Armen Sargsyan at Karabagh State University, who as of my visit in 2001 was still actively collecting dialect materials from Karabakh villages via his students from those villages.)

The bright light of current Armenian dialectology was until 2012 the University of Leiden in the Netherlands, which for some time featured three talented Armenian dialectologists, Jos Weitenberg, Uwe Blaesing, and Hrach Martirosyan. Weitenberg was primarily concerned with reconstructing the chronology of linguistic developments between Proto- and Modern Armenian, using evidence from manuscript `errors' and variation within the classical and medieval languages and the modern dialects. Of particular interest is his 2001 analysis of the chronological development of penultimate and final stress in the Armenian world, using sophisticated arguments from relative chronology. Blaesing is a turkologist by trade, but his wife is a Hemşinli, and he has consequently done a great deal of useful work on the Armenian lexical material that survives in the Turkish dialect of the Hemşin region of northeastern Turkey.

Weitenberg initiated in the 1990s a collaborative effort with the Dialect Institute in Yerevan, designed to facilitate the generation of isogloss maps from data collected via the \todo{1977} Muradyan manual from approximately 500 village varieties of Armenian. It remains to be seen whether the Institute’s database chronicles actual usage or a historical dialectologist’s conception of what a ``pure'' form of the dialect should have looked like at some point in the past, but in either case the maps resulting from this project should be quite helpful. We will turn to the challenge of constructing proper dialect maps in section \todo{3.}


\subsubsection{The future: the state of the Armenian dialects}

Many dialects, such as \todo{Nicomedia, Kirzan, and Shamakhi,} appear to be already dead; many more are alive but have only a handful of speakers remaining, such as Marash, Urfa, and Van. Zok, the Armenian dialect of Agulis, appears to have no more than two remaining speakers; Jerusalem lost its last fluent speaker less than twenty years ago. A fair number of dialects still have communities in which everyone speaks the dialect, but in some of these, such as \todo{Zeytun and Kesab,} the communities do not have a permanent location and will likely disappear in the near future. Those which do have permanent locations, such as Hamshen, \todo{Anjar, Diarbekir,} and the various dialects in Armenia proper and Karabakh (and perhaps \todo{Javakhk‘ and Abkhazia}), stand a better chance of surviving but are already beset by the pressures of the official languages of the countries in which they are spoken.

The key to understanding the future of these dialects is to look not at the number of people who speak them, but rather at the number of children who are learning them. Thus, for example, the number of Zeytun speakers is fairly large, but of the Zeytuntsis I worked with in Boston from 1990-2003 none had children who speak the dialect. By this measure, in fact, even SWA is in trouble. To the best of my knowledge, the only forms of Modern Armenian that are relatively secure for the immediate future are Eastern Armenian and Iranian Armenian.

Another problem is that many (and perhaps most) of the surviving dialects have yet to receive proper study. This includes many dialects in Iran (\todo{Tabriz, Chaharmahal, and many smaller villages, as well as the regional standard of Teheran), Iraq (Baghdad, Basra, Mosul, and the northern villages such as Zakho and Tell Kibār} that were inhabited by Armenians until the American invasion in 2003), \todo{Nakhichevan (speakers of several of these dialects currently reside in the south of Armenia), Jerusalem, western Turkey (Bolu , Smyrna, Adapazar, etc.}, and even Ethiopia and Turkmenistan. (It is not clear that the latter two Armenian communities actually possess distinctive dialects, as they have not been closely studied.) The mixed language of the Boshas, Lomavren, whose grammar is taken from the \todo{Erzerum} dialect of Armenian, also remains to receive extensive systematic study.

\subsection{Documentation}
Although a good deal remains to be done in the documentation of Armenian dialects, much work exists in both published and unpublished form. In this section I survey some of the more important work carried out to date, and compare the state of publication on Armenian dialects to that of English dialects.

\subsubsection{ Documentation of Armenian dialects}

\subsubsubsection{Syntheses}

A number of synthetic works on Armenian dialects and Armenian dialectology have appeared to date, most notably the aforementioned \todo{1869 monograph by Patkanov}, Adjarian’s doctoral thesis (published in French in \todo{1909 and in slightly expanded form in Armenian in 1911}), \todo{Gharibyan 1953 and 1958, Grigoryan 1957, Jahukyan 1972, Asatryan 1985, and Greppin and Khatchatrian 1986}. The last of these merely summarizes briefly the phonological rudiments of a few of the better-known dialects, but has the advantage of being the only description available in English (though the text samples for each dialect are not translated). Adjarian \todo{1909} is older, but is significantly more comprehensive and should be preferred by those who are unable to read Armenian. \todo{Adjarian’s \textit{Complete Grammar of the Armenian Language} (1952-1971)} contains a wealth of dialect information, but this material is not systematically presented, being interspersed sporadically throughout all seven volumes, and therefore is not as useful as most of the other synthetic works.

\todo{Jahukyan 1972} differs from the other works just mentioned in that it does not contain actual descriptions of dialects or even scattered dialect expressions or words, but instead contains a wealth of phonological, morphological, geographical, and bibliographical information on 124 different varieties of modern, medieval, and ancient Armenian. It moreover contains an enormous and unparalleled bibliography of books, articles, and unpublished manuscripts dealing with Armenian dialects, most of which are almost impossible to procure outside of Yerevan.  I have found this to be the dialectological treatise that I most often consult. The analytical component of the book, however, is somewhat problematic, as I discuss in section \todo{2.3.1.2.}

The above-mentioned studies are most useful and reliable in the sense that they were produced by linguists, but naturally suffer from their age, which prevents them from incorporating the host of more recent documentation, such as \todo{Kostandyan 1979 (Kharberd-Erznka), Lusents‘ 1982 (Aresh), Muradyan 1982 (Moks), Haneyan 1982 (Edesia), Madat‘yan 1985 (Alashkert), Ch‘olak‘ean 1986 (Kesab), Hapēshean 1986 (Musadagh), Markosyan 1989 (Ararat), Mezhunts‘ 1989 (Shamshadin-Dilijan), Hananyan 1995 (Khdrbek), and Adjarian 2003 (Cilicia; written much earlier, but unavailable until his manuscript was finally published in Yerevan in 2003}). Jahukyan makes use of these more recent works in his \todo{1997} monograph on dialectal elements in Armenian colophons, but the scope of this work is greatly restricted compared to that of the synthetic works discussed above.

Though they are synthetic in a larger but not dialectological sense, one should include here the \todo{Soviet Armenian Encyclopedia (Hambarts‘umian et al. 1974-1986), Petrosyan et al. 1975, and Petrosyan 1987}, which have brief but useful entries for dozens of dialects, including many whose grammars are not readily available outside of Yerevan.

\subsubsubsection{Non-linguistic collections}

Armenia has also produced a wealth of non-linguistic materials that happen to be of great use for dialectological purposes, though their transcriptions are often unreliable, omitting phonetic nuances in order to increase readability and/or because the transcriber, not being a linguist, did not notice them. Still, these collections can be useful for morphological, syntactic, lexicographic, and sometimes even phonological purposes. An incredible resource that remains largely untapped by dialectologists is the folklore collections \todo{(Tēr Aghēk‘sandrean 1886, Lalayean 1892, 1899, 1901, 1914, Haykuni 1900-1913, Abeghyan 1944-, Malkhaseantsʻ 1958, Orbeli et al. 1959-, Aṛak‘elyan 1970-, Grigoryan-Spandaryan 1971, etc.),} which in most cases have the advantage of providing the name, age, and specific village of origin for each storyteller. The various ethnographic works by \todo{Sruandzteantsʻ (Mananay (1876), T‘oros Aghbar (1884), Hamov Hotov (1884))} also contain many dialect texts, though their exact provenance is not always as thoroughly detailed as in the other works. The journal \todo{Biwrakn}, produced in Constantinople at the turn of the twentieth century, features dozens of dialect texts lacking the same sorts of information as those published by \todo{Sruandzteantsʻ}. Since virtually none of these sources  provide translations of their dialect texts into a western language or even into Standard Armenian, it is also appropriate to mention here \todo{Charles Dowsett’s} \textit{Armenian folk-tales and fables}, published in 1972 under the pseudonym of Charles Downing. The book contains 63 folk tales and fables, as well as ten pages of proverbs, translated from a wide range of Armenian dialects into English. The source for each story is clearly documented, so that one can easily track down the original Armenian (or Russian) versions.

Primary literary sources written in non-standard dialects also contain a wealth of dialect material, such as for the Tblisi dialect \todo{Sayatʻ-Nova (Hakhverdean 1852) and Sundukyan (1863 et seqq.; see his collected works, 1951-1961), Shirvanzade in Shamakhi dialect (1885, 1905, 1911, 1920, etc.; see his collected works, 1958-1962), Zargarean in Agulis dialect (1912), Ayvaz in Istanbul dialect (2001), and so on.}

Certain collections of riddles, sayings, games, and other ethnographic materials are also rich sources of dialect, notably \todo{Abeghyan’s collection of games (1940), Ghanalanyan’s collection of sayings (1960), Harutʻyunyan’s collection of riddles (1965), and Sargsyan’s collection of riddles from Karabagh (2002).} Though the dialect materials contained in these books are almost all taken from previously published sources, they remain valuable thanks to their thematic organization and the fact that many of their sources are not available outside of Yerevan. One problem with these collections is that when a given saying, riddle, etc. occurs in several dialects, they only provide a single, somewhat standardized rendition of it, leaving out information about the particular form it takes in any of the individual dialects. For instance, \todo{Harutʻyunyan 1965 states that riddle #951b (Q. ɛɾku ɔtʰɑʁ, min sjunɑni ‘two rooms, one pillar’; A: kʰitʰ ‘nose’) is used in Karabagh, Nakhichevan, and Borchʻalu,} but does not say which of these dialects (if any) the particular pronunciation he employs is from. \todo{Sargsyan 2002} has a more serious version of the same problem; whereas in many cases one can recover the original sources (and hence the informants) from \todo{Harutʻyunyan’s work}, because he provides references for his riddles, Sargsyan’s riddles are taken directly from uncredited native speakers, and it is therefore impossible to know exactly where they are from. This is an important lacuna in a work on Karabakh, since the region contains numerous distinct dialect groups \todo{(q.v. Davtʻyan 1966).} Sargsyan (personal communication) states that his informants are from \todo{Martuni and Hadrutʻ}, but this is still not sufficient information to place his rich materials in their proper dialectological context.

\subsubsubsection{Dictionaries}

We have a number of excellent dialect dictionaries, each of which possesses certain drawbacks. \todo{Adjarian 1913} features several thousand words and expressions, each of which is identified by its dialect of origin, but he regularizes their phonology, merges many dialects, and lacks data for the numerous dialects studied after the publication of his book. \todo{Amatuni 1912} has a host of data not to be found elsewhere and covers many dialects, but Amatuni was not a linguist and hence his renditions of the forms are unreliable. Also, like \todo{Achaṛean 1913,} he does not include many dialects and merges forms for disparate dialects. 

\todo{Achaṛean’s Armenian etymological dictionary} (published in handwritten form in \todo{1926, and then in typeset form in 1971-1979)} is for most purposes the most useful dialectal (and historical) dictionary, because it provides fairly reliable transcriptions and is organized by Classical Armenian headword, so it is relatively easy to compare forms in different dialects. However, an inevitable consequence of its early publication date is that many dialects are not included. In addition, often the Classical headword for a given dialect form is not obvious, and words not derived from classical forms are missing. Moreover, non-etymologically related words are not included in a given entry; for instance, the entry for Classical eɾ \todo{‘why’ (originally the genitive singular form of i ‘thing, what’)} provides the dialectal forms \todo{Davrēzh heɾ, Maragha heɾ ~ hejɾ, and Astrakhan neɾ (volume 2, p. 119)}, but does not mention other common dialectal forms for \todo{ ‘why’, such as intʃʰu, uɾ, əndeɾ, χi, χɑ, χɑs, hɔɾi, him, and zme (cf. Suk‘iasyan 1967:234). Only in the entry for intʃʰ ‘what’ (volume 2, p. 245) does one find mention of parallel forms intʃʰu, New Julfa tʃʰum(ɑɾ), Alashkert ɦəntʃʰi, and (in unspecified dialects) əntʃʰu, intʃʰi, and hintʃʰi embedded in Adjarian’s discussion of derivatives of intʃʰ.}


The excellent four-volume explanatory dictionary of Modern Armenian published by \todo{Stepʻan Malkhasyantsʻ in 1944} is replete with dialectal and other forms not found in conventional dictionaries, and is what I most often use when parsing both standard and dialect texts. However, \todo{Malkhasyantsʻ} does not cite the dialects or sources from which he takes each entry, and he conveys few nuances of non-standard pronunciation; his work is therefore more consistently useful for translation than it is for dialectological analysis.

\todo{Sukʻiasyan’s 1965} synonym dictionary is useful for identifying dialect variants, as shown by the entries for \todo{gɑzɑɾ ‘carrot’ (p. 133) and həndkɑhɑv ‘turkey’ (p. 393) in (4) and (5) respectively:}

\begin{exe}
	\todo{ref}
	(4)	gɑzɑɾ, g[ɔjɑkɑn ‘noun’]. 1. stɛpʁin: 2. stɛpʁin vɑjɾi, dɔks, tʃɑnduk, ɑʃmunisɑ, kʰɛʃiɾ:
	
	(5)	hndkɑhɑv, g[ɔjɑkɑn ‘noun’]. (kend[ɑ]b[ɑnɑkɑn ‘zoological].) hnduhɑv, hndkɑkʰɑʁɑʁ, (b[ɑ]ɾb[ɑɾɑjin ‘dialectal’].) tʃuɾtʃuɾ, tʃuɾtʃuɾ, gɔɾɛl, kɔɾɛl, tʃɔlɔk, kulkul, kuɾkuɾ, msɾɑhɑv, kʰɛlkʰɛl, tʃuluχ.
	
\end{exe}

Like \todo{Malkhasyantsʻ 1944, though, Sukʻiasyan} does not provide references for specific forms. One cannot infer from his book, for example, that \todo{չուլուխ tʃʰuluχ is used for ‘turkey’ in Hamshen (Acharean 1947:262; cf. Turkish çulluk ‘woodcock’ (Scolopax rusticola)), or even that ստեբղին əsdɛpʰʁin is the form for ‘carrot’ in SWA}. Key variants moreover are not included, such as the `turkey' form hindig in Abkhazian Hamshen (Avik Topchyan, personal communication), or the `carrot' forms\todo{ Baberd pʰəɾtʃʰulig, Tomarza pʰyrtʃʰykʰly, Gyumri pʰuɾtʃʰuluʁ, Kharberd and Kesab pʰɾtʃʰækʰli (for all of these cf. Turkish dialectal purçuluk), and Marash hɑvutʃʰ ~ havundʒ (cf. Turkish havuç). Oddly, Sukʻiasyan’s entry for the regional ‘carrot’ variant stepʁin (related to Greek staphulinos) includes forms not mentioned in the entry for gɑzɑɾ: tɑpʁin (which is used in Van, though Sukʻiasyan does not mention this) and teʁpind.}

\todo{Artem Sargsyan} and his team in Yerevan completed their seven-volume \textit{Dialect Dictionary of the Armenian Language} in 2012, but it is missing key requirements for a modern scholarly tool. It has the advantage of covering many more dialects than the other sources mentioned above, but is surprisingly limited in its coverage given the resources available in Yerevan; most of the dialect forms that I have looked for in it have not been there, and well-known dialects such as the Muslim Hamshen varieties described by \todo{Dumézil (1963 et seqq.)} are not incorporated. One also finds curious omissions of well-known dialect forms; for example, \todo{New Julfa լապստակ lɑpstɑk is missing from the entry for ‘hare’}. This same entry reveals another problem: it is unclear why the authors have separated from one another variant forms within a single dialect, e.g. \todo{լափստրայ lɑpʰstrɑj and լափստրակ lɑpʰstrɑk for Urmia.}

Along similar lines, when there is more than one reference for a specific dialect in an entry, the authors do not make clear which one a given form is taken from, nor where in the source text to find it. The citation of variant forms is inconsistent, as well; for instance, the entry for \todo{esōr} cites the variant form \todo{əsōr, but not the variant sōr.} The entry for \todo{էր ɛɾ ‘why’ (volume 2, p. 32)} mentions that it is used in Ararat, but does not provide a link to the aforementioned related forms \todo{Tabriz հէր hɛɾ, Maragha հէր hɛɾ ~ հէյր hɛjɾ, and Astrakhan նէր nɛɾ cited by Acharean} in his etymological dictionary.

The dictionary also is often not specific about dialect forms and references, as can be seen in the following example:

\begin{exe}
	\todo{f}
	
	(6) 	sample entry from Sargsyan et al. volume 2, p. 30
	ԷՍՏՈՒՐ, Ար. Ն. ԸՍՏՈՒՐ, Ղրբ. Սրա: Էստուր համար եմ ասում, որ լավ տղա է (ՎՏՍ): Փուղ պիտի վուր էստուրը դիմանա (ԳՍ): Ըստուր յաղի մին:
	
	ɛstuɾ, Ar. N əstuɾ, Ghrb. Sra. ɛstuɾ hɑmɑɾ ɛm ɑsum, vɔɾ lɑv tʁɑ ɛ (VTS). pʰuʁ piti vuɾ ɛstuɾə dimɑnɑ (GS). əstuɾ jɑʁi min.
	
\end{exe}

This entry provides no sources for the \todo{Ar [Ararat], N [Nakhichevan?], and Ghrb [Karabagh]} forms, nor does it give precise references that would enable one to locate the exact quotes in \todo{VTS [= Vagharshak Ter-Suk‘iasyan] and GS [=Gabriel Sundukyan]},\footnote{It is odd to provide an example in Tbilisi dialect here, given that the entry does not mention Tbilisi as one of the dialects that the headword appears in.} nor does it furnish a citation for the source from which \todo{əstuɾ jɑʁi min} is taken. Moreover, there is no translation of the example sentences into a western language or even into Standard Armenian, and there is no translation of headwords into a western language. Without translations, it is often unclear which of several possibilities a word means in a specific context. This is particularly problematic with synonyms; for instance, \todo{ɑlɑkɔʃkɔtʃ means ‘bat’ in some dialects, and ‘butterfly’ in others (cf. Suk‘iasyan 1965:17). Since Sargsyan et al.} do not provide glosses for their entries, it is impossible to tell which of these two is being referred to in a given instance. In order for the field of Armenian linguistics to move forward, the rich dialectal materials contained in this dictionary need to be made accessible to the linguistic world outside of Armenia, which requires providing the basic but essential scholarly tools just mentioned.

It is also worth mentioning \todo{Gabikean’s 1968} dictionary of Armenian plant names, which is a treasure trove of dialect material often unavailable elsewhere, either because the available materials on that dialect do not mention it \todo{(e.g. Hamshen ɑlʒi/ɑɾʒi (p. 12), zimbilɑk (p. 57)),} or there are no materials at all available on that dialect \todo{(e.g. Yozgat, Pʻinkean, Mashkert, Brgnikʻ, Zaṛa). Gabikean} does not provide sources for his dialect terms, and it is not possible to find all of the forms from a given dialect in one place, but it remains an important source of primary dialect data.

\subsubsubsection{Treatments of individual dialects}

I have already mentioned many grammars of individual dialects, yet too many books and articles of this type still remain to list here. Most of these works\footnote{With a few exceptions such as the works on Hamshen subdialects by \todo{Dumézil and Blaesing.}} are designed for Armenian-speaking dialectologists, and hence lack translations, explanations, and glosses of the dialect source material, except for the occasional translation of an easier word. Moreover, the dialect material is not rendered in the IPA or any other transcription intelligible to outsiders or general linguists. As a result, it is likely that the rich store of Armenian dialect material will never reach an audience broader than the dialectologists who specialize in it.\footnote{At a 2001 meeting in Stepanakert with a group of western linguists, representatives from the Dialect Institute claimed that they continue to transcribe Armenian dialects using the Armenian script because this script is better able to represent the nuances of dialect pronunciation. In fact, the IPA is able to render all of these distinctions and more, and do so in a way that is consistent across the world’s languages and intelligible to the entire world linguistic community.}  Nonetheless, the dialect grammars and article-length treatments produced in the nineteenth and twentieth centuries remain a rich repository of material that, if it were available, would be a source of great interest to both dialectologists and theoretical linguists worldwide.


Particularly noteworthy are the grammars that attempt to place their respective dialects in the context of the larger sphere of Armenian dialects, such as \todo{Baghramyan 1960 (Dersim) and 1964 (Shamakhi), Poghosyan 1965 (Hädrut‘), Davt‘yan 1966 (Karabagh), and Blæsing 2003 (Hamshen)}. A typical sample from these works is \todo{Baghramyan’s }map of the forms of the present tense in the Dersim region, which I have recast here:


\begin{exe}
	\todo{picture}
	
	(7) formation of the present and imperfect indicative in Armenian dialects of the Dersim region (based on Baghramyan 1960, map 2)
\end{exe}

Baghramyan does not attempt to relate the forms in (7) to what we find in other Armenian dialects, but his map provides a level of village-by-village detail that is unusual in Armenian dialectology. (Baghramyan does not cite his sources for the village forms represented in his maps.)

\subsubsubsection{Unpublished and ongoing work}

In addition to the published work described in the preceding sections, a great deal of Armenian dialect material has been collected and/or analysed without being published. This includes not only a large percentage of the theses, dialect glossaries, and articles listed in \todo{Jahukyan 1972, such as Varzhapetyan’s 1962 thesis on the Siwnik‘ dialect or H. Hovhannisyan’s} manuscript dictionary of the Agulis dialect, but also extensive collections of folk texts held in various archives in Yerevan \todo{(mentioned in Abeghyan 1940, Ghanalanyan 1960, and Harutyunyan 1965)}, videotaped and/or transcribed narratives by genocide survivors held at the Armenian Library and Museum of America (in Watertown, MA) and elsewhere,  and older manuscripts written in dialects. The latter class of materials is typically branded as ``written in corrupt Armenian'' in manuscript catalogs, and (due presumably to its non-historical, non-religious, and non-illustrated content) generally ignored by Armenologists. Typical examples include the sixteenth-century word list that appears to be in the contemporary Armenian dialect of Ankara and the seventeenth-century Armenian hexaglot dictionary held in the Oriental Library in Oxford, both of which are described in \todo{Baronian and Conybeare 1918.}


Ideally, these materials would be made available to the entire academic community, preferably via the internet, though something like the microfiche format of \todo{Weitenberg’s \textit{Armenia: Selected Sources}} would also be acceptable. Making these materials available outside of Armenia would benefit the Armenian economy as well as armenologists, provided that external funding sources are employed to contract typists and internet experts within the Republic.

Perhaps the most important unpublished resource of all is the data collected by the Dialect Institute in Yerevan. According to the members of the Institute,\footnote{\todo{Or the “Hrachja Atcharian Institute of Language”, according to http://inf.sci.am/about/research/41-lang.html; which institute is actually carrying out this work is not clear to me}} they  have elicited and transcribed answers to all of the more than 700 questions in their dialect manual \todo{(Muradyan et al. 1977)} for some 500 villages that have or had Armenian-speaking populations. Their website\footnote{\todo{The original site accessed in 2004, http://inf.sci.am/about/research/41-lang.html, exists as of June 2016 only in cache form (http://goo.gl/mpQorg).}}  asserts (as of October 22, 1998) that “during the last two decades the most outstanding achievement in linguistics has been the creation of [a] dialectological atlas - a collection of maps, each volume of w[h]ich will show the phonetic, lexical, [and] grammatical phenomena (isoglosses), and it will present fully all dialectal areas of the historical Armenia with further displacements and changes. Under the sup[er]vision of [the] International Association of Armenological Researches a collaboration contract is signed with Leiden University (Holland) and joint projects are being conducted concerning the following themes: ‘The Armenian Dialectological Atlas’ [and] ‘Historical Dialectology’.” As of 2016 neither of these projects has surfaced, nor have the data on the 500 villages been made available to scholars; one can only hope that these materials will one day appear. However, they were not collected using methods considered reliable by modern social scientists; in my experience (based partially on study of published materials and partially on observation of members of the Dialect Institute collecting data from informants in Karabakh in the summer of 2001) the fieldworkers note only the most archaic forms that they expect to find in a given dialect, rather than what speakers actually use, and they ignore intracommunity variation along age, gender, class, and idiolectal lines.

The fieldwork that has been carried out in Armenia on a sporadic basis by Anaïd Donabédian (of INALCO, Paris) and her students over the past fifteen years or so gives more cause for hope. Though the materials they are collecting are fairly traditional—texts, words, and so on—she and her students are well trained in modern linguistics and aware of the fundamentals of scientific field work.

\subsubsection{Documentation of English dialects}


As might be expected, the varieties of English have been documented more extensively and thoroughly than have their Armenian counterparts. The most recent large-scale surveys of English dialects are the Survey of [British] English Dialects (Orton 1962a, 1962b, 1978), the Dictionary of American Regional English, or DARE (Cassidy and Hall 1985, 1991, 1996, 2002), and most recently the Harvard Dialect Survey, conducted online by myself and Scott Golder in 2002-2003 (http://hcs.harvard.edu/~golder/dialect). The first two of these focus on older individuals, but pay close attention to phonetic detail, are careful not to put words into the mouths of their informants, allow for variation, carefully document the personal histories of each individual studied, clearly explain each dialect form, and provide numerous excellent maps. They therefore provide positive models for the future conduct of Armenian dialect research.

Consider for example DARE, which is based on interviews carried out in all fifty United States between 1965 and 1970, and on a comprehensive collection of written materials covering the entire range of U.S. history. Each entry has the basic form in (8):

\begin{exe}
	\todo{pic}
	
	(8) sample entry from DARE, volume 2
	idropped egg n Also iidrop egg [iiiProb from Scots dial; cf SND drap v. 5. (2) (b) 1824] ivchiefly NEng See Map vsomewhat old-fash 
	viA poached egg. 
	
	vii1884viiiHarper's New Mth. Mag. 69.306/1 ixMA, Martha was . . eating her toast and a dropped egg. 1896 (c1973) Farmer Orig. Cook Book 93, Dropped Eggs (Poached). 1933 Hanley Disks neMA, Dropped egg--take and put a pan of milk on the stove and boil and drop the egg in and let it cook. 1941 LANE Map 295 (Poached Eggs), throughout NEng, Dropped eggs. . .1 inf, ceVT, Drop eggs. 1948 Peattie Berkshires 323 wMA, In Berkshire . . you could not get a poached egg, but you could get a "dropped" egg, which was the same thing. 1965 PADS 43.24 seMA, 6 [infs] poached eggs, 4 [infs] dropped eggs, 1 [inf] dropped egg on toast. 1965-70 DARE (Qu. H35, xWhen eggs are taken out of the shell and cooked in boiling water, you call them ______ eggs) 40 Infs, xichiefly NEng, Dropped; xiiNH15, Dropped egg on toast. xiii[33 of 41 Infs old]  1975 Gould ME Lingo 82, Dropped egg--Maine for poached egg, usually on toast. 1977 Yankee Jan 73 Isleboro ME, The people on Isleboro eat dropped eggs instead of poached. 
	
\end{exe}



As can be seen in (8), each dialect term is provided as a headword (i), accompanied by its equivalent in standard English (vi) and by variant forms (ii). An etymology is provided if known, as well as the source for this etymology (iii). If any regional generalizations governing the distribution of the term can be identified, these are stated (iv), as is the degree to which the form is actually used (v). Specific appearances of the form in literary sources (viii) and DARE survey responses (x) are explicitly listed in chronological order, together with the year (vii) and state (ix) in which they appeared, and regional (xi) and other (xiii) tendencies (concerning age, gender, race, and the like) in these survey responses (xi). The specific location of each informant who produced the form in question is given (xii) and can easily be tracked down in the introductory matter at the beginning of the first volume; for example, NH15 (xii) represents a white female homemaker from Berlin, New Hampshire born in 1922.

The DARE survey on which much of the information in the dictionary is based is impressively large (1847 questions), compared to some 700+ questions in Muradyan et al. 1977. Administering such a lengthy survey is quite difficult, but not impossible, and the set of questions in Muradyan’s survey could easily be augmented.

The most obvious problem with carrying out a large survey (or any survey, for that matter) is getting data from a large number of speakers, which is essential if one wants to reflect accurately the speech patterns of a community. (Existing Armenian dialect studies typically represent the speech of only one or a handful of older individuals; to date there has been no interest in mapping the preferences of entire communities, a research program started in the West by Labov in the 1960s.) The Harvard Dialect Survey addresses this problem by administering its questions online, which makes it possible for English speakers worldwide to answer the questions quickly and easily, and in significant numbers (more than 50,000 individuals completed the survey between September 2002 and May 2003, making it the largest dialect survey conducted up to that time). The price for this advantage is that the survey must be limited to questions that can be understood and reliably answered by individuals with no linguistic background; questions concerning lexical choice and certain pronunciation distinctions work well, but questions involving subtle phonetic, syntactic, or semantic distinctions or requiring knowledge of linguistic concepts usually do not.

Questions on the survey are also more effective if they involve dialect differences that are well-known and salient, such as the variation for ‘sweetened carbonated beverage’ depicted in (9); obscure items that most speakers no longer know, such as parts of a plow, tend not to generate useful results and often dissuade survey takers from continuing to the rest of the questions.

\begin{exe}
	\todo{pic}

(9) map of the three principal terms for sweetened carbonated beverages in the Continental United States (http://www.popvssoda.com/countystats/total-county.html)



\end{exe}

I have found that the best way to persuade large numbers of individuals to complete a survey is to provide maps of the responses received up to that point. For the Harvard Dialect Survey we generated rough and ready automated maps of the sort in (10), which represents the distribution of the three main terms for drinking fountain in the continental United States for the first 10,656 respondents to the survey.

\begin{exe}
	\todo{pic}
	
	(10) Q103. What do you call the thing from which you might drink water in a school?
	
	bubbler (red; 3.84\% of responses)
	drinking fountain (green; 33.16\% of responses)
	water fountain (purple; 60.97 of responses)
	
	
\end{exe}

A significant advantage of electronically-administered surveys is that their responses are instantaneously available in digital form, and therefore can easily be mapped and statistically analyzed in a variety of ways. Figure (11) presents a map generated in Arcview + Adobe Photoshop for the Harvard Dialect Survey question involving the second-person plural subject pronoun; yellow dots depict areas where “y’all” predominates, and green dots are areas where “you guys” is the preferred form.

\begin{exe}
	\todo{theres pictures}
	
	
	(11)	Q50. What word(s) do you use to address a group of two or more people?
	
	
\end{exe}

One can object that the responses to the survey will not be reliable, since they are elicited from voluntary respondents of a self-selecting subset of the population, namely those individuals who have the resources, the interest, and the leisure time to complete a lengthy dialect survey on the internet. This stands in contrast to most scientific surveys, which attempt to target a random but representative sampling of the population. In my opinion, though, the sheer scale of the Harvard Dialect Survey (HDS) more than compensates for the weaknesses in its sampling methods: surveying 50,000 Americans makes it quite likely that one will accurately capture regional linguistic tendencies, perhaps more so than a survey that uses representative random sampling but only collects data from 300 Americans. Let us say, for example, that a given regional form only occurs in Michigan, but even there it is only employed by 30\% of Michiganders. A survey of 300 Americans will presumably contain no more than 6 Michiganders, whereas the Harvard Dialect Survey elicited data from one thousand eighty-nine of them. It should be clear from this quick comparison that the chances of discovering the usage of this regional form are fairly slim in the survey based on the random sample, whereas they are quite good in the HDS. A survey on the scale of the HDS moreover greatly improves one’s chances of delineating the exact geographic boundaries of linguistic isoglosses. The price for this accuracy is that one must constantly remain aware of the limitations of the survey; the HDS, for example, may well not provide an accurate reflection of the speech patterns of African Americans in the United States. This would of course be less of a problem for a survey of Armenian dialect speakers, since they show significantly less ethnic and socio-economic diversity than do speakers of American English.

In fact, much of the online survey work that has been done for English can be done with equal ease for Armenian. The one problem that comes to mind with the Armenian case is that most speakers of nonstandard varieties of Armenian, being older, likely do not use the internet. However, this can be compensated for to some extent by making the notional Armenian dialect survey available online in a form that younger Armenians can administer to their dialect-speaking parents and grandparents. With suggestions for how to record their dialect-speaking relatives on audio and videotape, together with specific recommendations for what to record, these younger Armenians might contribute valuable data to the field. The ideal Armenian dialect survey website would then make it possible for these individuals to upload their recordings to the site for processing and dissemination by the linguists managing the site.

A significant amount of excellent Armenian dialect material has been collected over the past three hundred years, but we must intensify our efforts to make these materials available to the scholarly community, especially to those scholars in the West who could contribute to our understanding of the material and help integrate it into the realm of scholarship on general linguistics, if only the materials were available in a language with which they were familiar. We must also expand our efforts to collect new materials, using the methods developed by sociolinguistics over the past fifty years. These efforts can be significantly facilitated by employing computer and internet resources, particularly in the collection and dissemination of surveys and recordings.

\subsection{Theory and Method}

In the previous section I made a number of recommendations for the collection of Armenian dialect data, based on advances over the past few decades in the research methodology employed for English dialects. In this section I turn to parallel advances made in theoretical linguistics and sociolinguistics, again with an eye towards ways in which these can be of use in Armenian dialectology. Armenian dialectology would benefit from expanding beyond its current focus on historical and prescriptive issues to include consideration of synchronic grammatical and sociolinguistic variation.

\subsubsection{Synchronic}

I mentioned earlier that Armenian linguistics essentially preserves unchanged the state of European linguistics in the 1890s, when Adjarian studied in Paris with Meillet. Since Adjarian’s departure from Europe there have been four pivotal theoretical innovations in Western linguistics that are relevant for our purposes: 

\begin{exe}
(12) innovations in Western linguistics
1. 	Saussure’s separation of synchrony and diachrony;
2. 	Saussure’s distinction between langue and parole;
3.	Labov’s identification of the importance of linguistic variation, both intra-community and intra-individual;
4.	Labov’s synthesis of synchrony, diachrony, and variation.

\end{exe}

Saussure demonstrated that there are important differences between the \textit{diachronic} (or historical) and the \textit{synchronic} axes of language. Whereas most prior linguists had thought of languages much as we think about humans, with their essence being most evident (and perhaps only evident) when looking over their entire life span, Saussure showed that the structure of a language is in fact best viewed synchronically, that is to say, at a single point in time. There are two main reasons for this: (i) each individual forms a linguistic system in his head based on the primary linguistic data to which he is exposed (typically as a child), and it is this system, in which everything coheres, that makes the most logical object of study; (ii) normal humans do not know the history of their language, so the linguistic system they construct in their heads does not contain historical derivations of words, sounds, syntactic constructions, and the like, and it is therefore misleading for the linguist to construct an analysis that conflates different stages in the history of the language. This conflation is precisely what we still find in most Armenian dialectology, though; there is no effort to elucidate the synchronic structure of grammars inside the heads of individual speakers at a specific point in time.

Saussure also argued convincingly that it is important for linguists to distinguish between \textit{langue} and \textit{parole}, or the internal and external manifestations of language respectively. (Chomsky and most contemporary linguists now refer to these as \textit{competence} and \textit{performance}.) It is well and good to study linguistic performance such as recorded conversations, text corpora, and the like, but Saussure and Chomsky point out that at best this reveals only a small part of the linguistic knowledge we have stored inside our heads, and at worst it reveals nothing or misleads us, as in the case of speech errors, abortive utterances, or drunken speech. To take a simple example, we do not want to say that a stutterer stores the word <tar> in his head as \todo{/ththththɑɾ/}, even if he tends to pronounce it that way.  If we are interested in studying the synchronic linguistic systems inside the heads of individual speakers, as Saussure and Chomsky suggest we should be, then we must employ certain methods of data collection and not be content with passive reception of tokens of linguistic performance.


Labov first demonstrated in the 1960s, however, that careful elicitation and study of linguistic performance can reveal important aspects of synchronic grammars as well. On a very simple but essential level, Labov showed (as we will see in greater detail below) that every component of the grammar—phonetic, phonological, morphological, syntactic, semantic—typically displays systematic variation within a speech community, conditioned by social factors such as gender, age, ethnicity, race, and socio-economic status. The speech of individuals varies systematically as well, depending on who one is speaking with, who one wants to establish solidarity with, and so on. The fact that this variation is so \textit{extensive} requires that we recognize and capture it in our study of a language or dialect, and the fact that the variation is so \textit{systematic} suggests that we should capture it in our model of the synchronic grammar. At present neither of these is done in Armenian dialectology, which treats dialects as undifferentiated monoliths unaffected by gender, social status, speech register, and so on.


In his more recent work Labov has gone one step further, linking the synchronic variation just discussed to historical change. More specifically, he suggests that the roots of historical change can be found in synchronic variation, and conversely that historical changes can create synchronic variation. For Labov, diachronic linguistics is a form of synchronic linguistics, because the history of a language is a succession of synchronic grammars mediated by the language acquisition process and by social factors. In order to understand the syncrhonic grammar, however, we must understand what sorts of linguistic variation the individuals in a speech community are exposed to, what value judgements they assign to these variants, how much of the time each variant is used, and so on. Again, this sort of information is absent in studies of Armenian dialects.

Sociolinguistics and dialectology as currently practiced in the United States employ the four theoretical advances just discussed, which makes them essentially synchronic in focus. Armenian dialectology, on the other hand, is purely historical, with very few exceptions. American dialectology is a branch of sociology and linguistic theory, whereas Armenian dialectology as practiced up to this point is a branch of philology.

The American tradition has two primary foci: (i) nuanced examination of the present synchronic system of the speech variety under consideration, (ii) emphasis on the speech community rather than idealized/isolated individual, especially in the Labovian (as opposed to Chomskyan) tradition. Labov and his followers maintain that the nature of a language only emerges when one looks at the linguistic behavior of a sufficiently large and representative number of its speakers, and hence is statistically oriented. The Chomskyan camp, on the other hand, maintains that valid generalizations best emerge from inspection of individual grammars (idiolects), and statistical conflation of individual grammars may obscure the actual nature of the language. Consider a simple idealized linguistic community, half of which have idiolects that lead them to use the word ələkoʃkoʃ for ‘bat’ 100\% of the time and for ‘butterfly’ 0\% of the time; in other words, ələkoʃkoʃ is their only word for ‘bat’, and means only ‘bat’. The other half of the community uses ələkoʃkoʃ only for ‘butterfly’, and never for ‘bat’. If the linguist averages over this community, it will appear that the language of the community has two meanings for ələkoʃkoʃ, ‘bat’ and ‘butterfly’, each used with equal frequency, when in fact there is not a single member of the community whose grammar contains this system.

Fortunately there is a relatively good resolution to this tension between the Labovian and Chomskyan models. It is important to study individual idiolects separately, rather than conflating their outputs, so that one knows what the range of linguistic systems within the speech community is. Once this has been done, though, one can conduct certain statistical computations over the set of idiolects to determine what generalizations emerge. In the case of the idealized community mentioned earlier, for example, the right sort of statistical analysis would reveal two peaks in the distribution, corresponding to the ‘butterfly’ and ‘bat’ subdialects.

In addition to controlling for idiolectal and subdialectal variation of the sort just discussed, it is also important to control for variation along other axes, such as age, gender, class, and (if working on a published corpus) literary genre. It is dangerous, for example, to draw conclusions from generalizations such as in (13):

\begin{exe}
	(13) “In Classical Armenian, all three ‘flat’ constituents preferably and predominantly precede the noun [Adj+N 75.17\%, Num+N 86.69\%, Qnt+N 91.10\%, Dem+N 80.79\%], though it is a prepositional language. The preferred and predominant position of the GEN is postnominal [N+Gen88.66\%].” (Dum-Tragut 2002:292)
\end{exe}








The literature of a given century typically contains a wide range of genres, regional origins, and manuscript transmission histories, each of which has significant effects on the linguistic content of the individual texts (see Coulie 2014 for discussion of the linguistic effects of manuscript transmission, for example). Without identifying and controlling for these variables,  generalizations such as (13) are at best spurious and at worst misleading.

Diachronic conflations of the sort just discussed also fail to identify synchronic phonological processes at work in the minds of individual speakers. Isolated scholars such as Pisowicz, T‘okhmakhyan, Khach‘atryan, Weitenberg, and Achaṛean occasionally identify synchronic processes in their work, but most linguistic generalizations in the Armenian dialectological literature contain only historical generalizations of the sort in (3). For these linguists (as for traditional historical linguists in the West) languages are viewed as sets of words, rather than sets of rules and constraints operating on a lexicon; in this model, historical changes must occur at one specific point in time, and cannot remain active over time, since there are no rules to be passed on from one generation to the next. There is clear evidence that linguistic rules can and do remain active from one generation to the next, though—witness the alternations produced by the vowel shift in Zok (e.g. tsɔɾ ‘tree’ ~ tsɑ́ɾɑɾ ‘trees’) that were still active in Acharean’s time (he published his Agulis grammar in 1935), though they had already first taken place by Schroeder’s time (1711). A theory that ignores linguistic rules of this sort misses much of what is interesting and important about a dialect.

Armenian dialectologists typically do not distinguish between phonetics and phonology either, which is probably due to the fact that this distinction had not yet been clearly drawn in European linguistics by the time Acharean finished his training in Paris.

Instrumental phonetic studies are another important lacuna in the field. Useful work has been carried out by Acharean 1890, Allen 1950, Khach‘atryan and T‘okhmakhyan 1988, Ladefoged and Maddieson 1996, and Hacopian 2003, but much of the basic documentation remains incomplete, especially on non-standard dialects. (Even for the literary dialects many basic questions have not been addressed, such as whether the SWA “voiced” stops are short-lag or voice-lead, and when the high and mid vowels are tense vs. lax.) The equipment needed to make high-quality recordings and carry out sophisticated phonetic analysis is now readily available and affordable, so it would now be easier to carry out this basic work on the surviving dialects.

A final general synchronic problem is that work within Soviet and Republican Armenia has focused on \textit{prescription}—what a dialect should look like, according to the linguist—rather than on \textit{description}—what a dialect actually does look like. I once witnessed a group of linguists from the Dialect Institute working with an old man from Ashan in Karabagh, which began innocently enough with them asking him to produce his forms for a few lexical items. Up to that point in our afternoon with him he had been speaking Eastern Armenian with a Karabagh accent (as most Karabagh Armenians are able to do), so his responses naturally contained a mixture of Standard and dialect forms. (He was framing his answers in Karabagh dialect for the most part, but the particular lexical items he produced were sometimes variants of the Standard form with Karabagh phonology applied, rather than the completely distinct forms the linguists were expecting.) The head linguist then chided the old man for not knowing the ‘correct’ forms for his village, and tried asking him a few more questions. By this time he was so flustered at being told by a professor from Yerevan that he was not speaking properly that he switched entirely into SEA. The linguists from the Dialect Institute then abandoned him in frustration, saying as they left that he didn’t speak the dialect. (I went up to him a few minutes later with an anthropological colleague who speaks a Karabagh subdialect similar to his, and the two of them immediately started chattering in dialect.)

It should be obvious from anecdotes like this that prescriptive attitudes can make it difficult to collect useful dialect material. Prescriptivism is also anti-scientific—compare the case of the physicist: he studies how objects actually \textit{do} fall, rather than wasting his time telling the objects the rate at which they \textit{should} fall. For dialectology to be scientific, we need to focus on what real people actually say, not what they “should” say.

\subsubsubsection{Parameters of variation}

What dialect speakers actually say is conditioned by a range of social and linguistic factors, as we saw earlier. Pre-Labovian dialectology (and Armenian dialectology, as already mentioned) recognize the existence of linguistic variation conditioned by region and time, but few other variables. Dialectological work in the West since the mid 1960s, by contrast, has identified significant variation along numerous other axes, including class, age, gender, and register. Consider the following isogloss map depicting the distribution of terms for ‘food trough in a cow-house’ in southeastern England:

\begin{exe}
	\todo{pic}
	(14) ‘food trough in a cow-house’ (based on a map in Crystal 1995)
\end{exe}


One can see in (14) that the distribution of \textit{manger} in this region shows clear geographical conditioning: it extends from its original base around London down along the three principal motorways in the region, which stretch from London to Portsmouth, Brighton, and Folkestone. What one can infer from this map is that trough was the original term in the southeast, and is being supplanted by the London term, \textit{manger}, in areas where London commuters are moving outwards along the motorways.

Linguistic variation can be conditioned by register as well. Perhaps the best-known case is the use of r-deletion in non-prevocalic position in New York City English (Labov 1966, 1972). Labov and his assistants elicited the phrase \textit{fourth floor} from more than two hundred sales staff at three New York department stores—Saks, Macy’s, and Klein’s—by locating in advance a product sold on the fourth floor of that store (say, socks) and asking the worker where the socks were located. The linguist then covertly noted whether the worker pronounced the two r’s in \textit{fou\underline{r}th} and \textit{floo\underline{r}} respectively. The workers’ responses to this question were held to be representative of a “casual speech” register. In order to elicit “careful speech” forms of the same phrase, Labov and company then indicated that they hadn’t heard the first time, and asked the same representative to repeat what they had just said.

What Labov found was that, contrary to popular belief, it was not the case that New Yorkers dropped all their r’s. Instead, all of the speakers surveyed pronounced the coda r’s some of the time, with the percentage of use depending on both the department store (presumably reflecting variation in social class) and the speech register. All of the speakers interviewed pronounced the coda r’s about twice as often in their emphatic/careful speech register as in their casual speech.

In formal terms, the New York City variety of American English contains a rule that deletes r when it is in a syllable Coda (i.e. when it is not followed by a vowel), and this rule is suppressed to a greater extent in more careful speech registers.

The same process may be conditioned by class as well. Wolfram found in his 1969 study of  the variety of English spoken by the African-American community in Detroit, for example, that as one moves down the socio-economic ladder from the upper middle class to the lower working class (LWC), the application of r-deletion increases significantly, from 20\% in the upper middle class to more than 70\% in the lower working class.

In general terms, linguistic variables can be (and typically are) conditioned by more than just region. When working on Armenian dialects we therefore should be careful not to focus solely on regional variation while ignoring variation by age, gender, and the like.

\subsubsubsection{The construction of dialect}

Western work on dialectology in the last few decades has also begun to develop a more nuanced and realistic picture of what dialects are and how they are constructed by their speech communities and by linguists. Martin 1954 and Trudgill 1972, 1978 first showed that individuals’ perceptions of their own speech are not always accurate: females (especially in the middle class) tend to \textit{over-report} their use of features of the standard dialect, whereas men (especially in the working class) tend to \underline{under-report} their usage of these features. In other words, women tend to feel they are speaking more “properly” than they actually are, and men feel the opposite. This finding has a number of important consequences: (i) we should not rely on native speakers’ reports of how they or other parts of the speech community work; (ii) field workers should be careful to control the contexts in which they conduct their interviewers to minimize the social anxiety their informants may be feeling.

Similarly, field workers should be aware that individuals speak differently depending on the person to whom they are speaking. The dialectologist, who is typically well educated and seen as an authority figure, often (unconsciously) prompts informants to speak in a relatively formal/careful register closer to the standard language than the dialectologist would prefer. I have found that an excellent way to circumvent this problem is to bring with me to the interview one or more friends or colleagues who are native speakers of the dialect, and to have them ask questions of a non-linguistic nature in dialect. This facilitates establishing a rapport with the informant, and makes it clear to them that they can speak in their dialect without any dire consequences. Involving more than one native speaker in the session also creates an environment in which natural unguarded conversation is possible, making it possible to collect connected speech to a degree that is impossible when one is asking a single native speaker disjointed questions about individual vocabulary words in their dialect and so on.

Another component involved in building a linguistic picture of a speech community is the use of statistics. Until the Labovian revolution in the 1960s, dialectologists in all countries generally focused their attention on older rural speakers, as these were felt to preserve the purest or oldest form of the speech variety in their region. As mentioned previously, this remains the focus in Armenian dialectology today. Labov shifted the emphasis in American dialectology to building an accurate picture of the speech community as a whole, not just that of old male farmers. This goal requires identifying and surveying a statistically representative sample of speakers in the community, cutting across genders, races, ethnicities, occupations, ages, and the like.

\subsubsection{Diachronic}

Though the diachronic axis of Armenian dialect work is relatively well developed, it too can be augmented by advances made in the West over the past century, particularly in the areas of subgrouping and chronology of sound changes. Our findings in these areas are of particular importance outside of dialectology, because they can help us clarify the complex pattern of Armenian migrations over the past two millennia and address the question of when Armenian first began to split into separate dialects.

\subsubsubsection{Subgrouping}

Perhaps the primary concern of Armenian dialectologists to date has been the classification and historical subgrouping of the various dialects. This preoccupation is clear from the title of the first systematic survey of the Armenian dialects, Achaṛean’s \textit{Classification of the Armenian Dialects} (1909), and of the most recent major synthesis, Jahukyan’s \textit{Introduction to Armenian Dialectology} (1972). The methodology employed in these works is in certain ways not entirely satisfactory by modern standards, however, and has produced results that are somewhat problematic, as we will see.


\subsubsubsubsection{ Traditional classifications}
The best known subgrouping of Armenian dialects is into Eastern and Western branches, based on features such as the ones in (15).

\begin{exe}
\ex (15) traditional Armenian subgrouping criteria
a.	nominal morphology (Eastern dialects have a distinct locative in -um (Jahukyan 1972’s feature \#64); Western dialects use the accusative or genitive/dative case for locative functions)
b.	\textbf{verbal morphology} (Eastern dialects use -um to form the present tense; Western dialects use some form of gu-)
c.	\textbf{agreement} (Eastern dialects only allow one agreement marker per noun phrase (im tun-ə ~ tun-əs, but *im tun-əs); Western dialects require agreement marking on the head noun (im dun-əs)
d.	\textbf{consonant system} (Eastern dialects have voiced {b d g dz dʒ} and voiceless {p t k ts tʃ}; Western dialects have the opposite)
\end{exe}

(By ‘Western’ dialects I mean dialects 1-72 in Jahukyan 1972, and by ‘Eastern’ dialects I mean Jahukyan’s dialects 73-120.) 

Adjarian 1909 first showed systematically that the set of present tense formations in the Armenian world was more complicated; Gharibyan 1958 and Jahukyan 1972 added further nuances to this scheme.

\begin{exe} \ex table \todo{table}
(16) modern present formations
source	underlying form	surface form	gloss
SEA	gu pher-e-m	gəpheɾem	I carry
SEA	ber-ume-m	beɾumem	I carry
Meghri	mn-ɑ-lis i-m	mn-ɑ-lis i-m	I stay
Kesab	hɑ pæn-e-m	hɑ pænem	I work

\end{exe}

The consonant system was also found by linguists to be more complicated than the picture in (16) suggests (cf. Furke 1959, Garibyan 1959, 1962, Agayan 1960, Georgiev 1960, Jaukyan 1960, Benvenist 1961, Feydi 1961, Fogt 1961, Leman 1961, Makaev 1961, Otrembskij 1961, Zabrockij 1961, Ivanov 1962, Zhirmunskij 1962, Pisowicz 1976, 1998, Garrett 1991). Scholars now typically divide the Armenian dialects into seven groups based on the outcomes of the three original Indo-European stop series in word-initial position (17). 

\begin{exe} \ex \todo{table}
(17)	correspondences in initial position
1	d	dh	t	Indo-European
2	d	dh	th	Sebastia
3	t	dh	th	Yerevan
4	d	d	th	Istanbul
5	d	t	th	Sasun, Middle Armenian
6	d	th	th	Malatia, SWA
7	t	d	th	Classical Armenian, Agulis, SEA
8	t	t	th	Van
\end{exe} 

Representative words for each series are given in (18).

\begin{exe} \ex \todo{table}
(18)		*D	*Dh	*T
Indo-European	dekjm̩t `ten’	bheremi `I carry’	`eight’ okjtō
Sebastia	dasə	bherem	uthə
Yerevan	tassə	bherem	uth
Istanbul	dasə	berem	úthu
Sasun	das	perəm	uth
SWA	dasə	pherem	uthə
Classical	tasn	berem	uth
Van	tas	pirem	uth
\end{exe}

\subsubsubsubsection{Innovations and the Wave Theory}

Each of the linguistic features just discussed is interesting on its own merits, but does not tell us much about the relations between and subgrouping of the Armenian dialects. We need to focus on non-trivial linguistic innovations (such as the development of the կու gu-present in many Western dialects (Jahukyan 1972’s feature \\#78), or the development of a new -ից -itsh ablative in many Eastern dialects, Jahukyan 1972’s feature \#59), rather than archaisms (such as the preservation of the Classical Armenian stop series in group 6 dialects and of the Classical -է -ē ablative in Western Armenian), because all dialects are equally likely to preserve a given feature of their linguistic ancestor, whereas the probability that two dialects would independently develop the same innovation is significantly lower than the probability that one dialect innovated and passed that innovation on to two descendants.

However, some innovations can develop independently in isolated speech varieties and therefore are not good diagnostics. Examples of this type include the development of word-final devoicing (which occurred independently in languages as disparate as German, Turkish, Sanskrit, and Russian), or (in the case of Armenian dialects) borrowing the Turkish ordinal suffix –indʒi (Jahukyan 1972’s feature 69, attested in Rodosto, Istanbul, Nicomedia, Eudokia, Trebizonde, Kharberd-Erznka, Syria, most of Karabagh, some of Tigranakert, Mush, and Astrakhan), or developing vowel harmony under Turkish influence (Jahukyan 1972’s feature 38: some Cilician and Syrian dialects, Meghri, Agulis, Karabagh, Havarik, Shamakhi, Khoy, and Maragha).

The idea that every linguistic innovation is independent of every other is central to the Wave Theory (Schmidt 1872), variants of which sociolinguists today generally prefer to the Tree Theory (Stammbaumstheorie; Schleicher 1853) that remains popular among philologists. (See Garrett 2006 for a critique of applying the Tree Theory to the Indo-European family.) The basic idea of the Wave Theory is that a linguistic innovation starts from an individual and gradually propagates outward, sometimes even crossing language boundaries (creating Sprachbund phenomena).  This propagation can move from major urban centers to increasingly less-populated areas, as with the spread of uvular r (IPA [ʁ]) in Europe (Trudgill 1974).

When there is close cultural and linguistic contact between dialect groups over an extended period of time, it is often the case that several linguistic innovations propagate over the same geographic expanse, as with the numerous overlapping isoglosses that Kurath 1949 correlated with the three major dialect regions in the United States: the North, South, and Midlands. A typical example in the Armenian world is the dialect subgroup that contains Karabagh, Khoy, and Maragha, which share a number of non-trivial innovations, including their consonant shifts, development of penultimate stress and a present tense formation in -lis, change of ɾ > h in pronominal forms like sɾɑnkh > s[ə]hɑnkh, and placement of negative elements after the verb.

Because of their grounding in historical and cultural contact, isogloss clusters of this type inform us about the historical relations between the dialect communities involved. They can help us reconstruct the historical movements and subgroupings of dialect communities and establish times before which certain innovations must have occurred; see section 2.3.2.2 for exemplification and discussion.

The standard interpretation of the Wave Theory (see Petyt 1980) also provides a means of defining dialects in synchronic terms: dialects are linguistic areas characterized by the overlap of a number of isoglosses. Important dialect groupings are defined in the same manner. Jahukyan 1972 implements a classification of the Armenian dialects based on this principle, as we will now see.

\subsubsubsubsection{Jahukyan 1972}

The best attempt to collate Armenian isoglosses is Jahukyan’s \textit{Introduction to Armenian dialectology} (1972), which employs what he refers to as a ‘multi-featured classification’ procedure. This procedure involved the selection of one hundred phonological and morphological features (for example ‘devoicing of original plain voiced stops’, ‘present tense employing the particle \textit{ku’}) and one hundred twenty dialects for which these features are known. Jahukyan then assigned the values +,  , ±, or ∓ to each feature in each dialect, producing a 100 x 120 grid. The basic data Jahukyan presents in this way are extremely useful, particularly in affording us a glimpse of certain dialects whose grammars are as yet unpublished and in giving a general overview of numerous important phonological and morphological isoglosses. 

The problems lie in what Jahukyan does with the data. His method of classification consists of adding the pluses and minuses for each feature (the former are assigned a value of 1 and the latter Ø), yielding a tally between 0 and 100 for each dialect. Jahukyan then classifies dialects into groups based on their numerical scores, with varieties within 22 points of each other counting as subdialects, 22.5-44 points counting as dialects, and 44.5 points or more counting as dialect groups (p. 127). In this scheme Classical Armenian differs from Middle Armenian by 24.5 points, from SEA by 25 points, and from SWA by 25.5 points. Western Armenian differs from Middle Armenian by 14.5 points, and from Eastern Armenian by 23.5 points (p. 199).

This method suffers from a number of problems. First of all, blindly grouping features will not produce useful results: for example, two dialects with scores of 30 could be grouped together and yet possess no common features because dialect 1 possessed features 1-30 while dialect 2 possessed features 71-100. Second, Jahukyan’s ‘features’ are often grab bags of unrelated phenomena. For example, feature 25 is assigned a positive value if the dialect in question has any of the following (1972:48):

\begin{exe} \ex \todo{table}
(19) Jahukyan’s feature 25:
a.	Change of original {dʒ tʃ tʃh ʒ ʃ} to {dz ts tsh z s}
b.	Loss of preconsonantal initial s
c.	Change of original intervocalic s to h
d.	Change of original rs sequences to ʃ
e.	Change of r to h in demonstrative pronouns
f.	Development of w from original v
\end{exe}

It is unclear why Jahukyan would combine such disparate features into a single item except to make the final tallies correspond to his preconceived notions of what groups the dialects should fall into. (Note that the groupings produced by his method do not differ significantly from prior classifications, though we should expect them to, given the random nature of his procedure.) Third, many of his features involve trivial changes that are unlikely to be optimal criteria for subgrouping (discussed below), for example the development of the front vowels æ œ y (feature 34). 

Fourth, a number of Jahukyan’s features are archaisms rather than innovations, for example the preservation of the Common Armenian simple present tense formation (feature 100.1). It is strange that Jahukyan and other members of the linguistic school established by Achaṛean in Armenia so commonly employ archaisms as criteria for subgrouping, given that Achaṛean’s teacher, Meillet, was well aware that innovations alone can be used in this way. There are other reasons why it is very useful to know which dialects preserve archaic features, but we should not use this information for the purpose of subgrouping. 

Another problem with Jahukyan’s work is his failure to provide specific references for his data. Many of the dialects he describes have been analyzed by several different authors, who often disagree in their descriptions (cf. the formation of the present tense in Khotorjur, which according to some sources preserves the Classical form and according to others uses gu with a distribution similar to Hamshen and Erzerum). Jahukyan’s chart of pluses and minuses does not tell us whose analysis he has chosen in a particular case, nor why that one is better than others.

Jahukyan also uses glottochronology to assess the relationships between the modern dialects and Classical Armenian (pp. 219-245), using lists of 100, 200, and 215 words. Glottochronology was already discredited by the time Jahukyan’s book came out in 1972 (cf. Gudschinsky 1956, Sjoberg and Sjoberg 1956, Taylor 1961, Dyen 1964, Blust 2000, Matisoff 2000, McMahon and McMahon 2000), which raises the question of why this section of the book was written. The basic fallacy in glottochronology is the a priori assumption that \textit{all} languages change \textit{at the same rate all the time}. This is simply not true, not only regarding \textit{different} languages, but even within a single language. It is well known that individual word types do not change at the same rate; for example, numbers are more resistant to change than other lexical categories. A language’s lexical retention rate may also be affected by external factors such as borrowing, taboo, having a strong/conservative literary tradition, ethnic or national pride, and the like. Since these factors obviously act differently on different cultures and languages, we in fact expect languages to change at different rates. This prediction is borne out in comparing English and German, for example, which share 75 cognates in the Swadesh list and therefore by his formula separated 954 years ago, i.e. in the 11th century AD. In reality, we know that English and German separated by the fifth century AD, six hundred years earlier than the glottochronological model dictates. 

\subsubsubsubsection{ Archaisms}
We saw in the previous sections that it is important for purposes of subgrouping to focus on non-trivial linguistic innovations. Linguistic archaisms in dialects can be of great use as well, just not for subgrouping. Non-linguists tend to think of archaism as an overarching property of one language or dialect vs. another; compare the popular debate among (lay) Armenians as to whether Eastern Armenian is more conservative than Western Armenian, or vice versa. In fact each component of a language has its own history (this is the central idea of the Wave Theory discussed earlier), and every variety of Armenian (or any other language) contains both archaisms and innovations. To take a simple example, Western Armenian is conservative with regard to orthography and the ablative singular ending –է -ɛ, while Eastern Armenian is conservative with respect to the consonant system and preserving a distinct form for the locative singular. Since all forms of Armenian are likely to contain a host of different archaisms, it is important to look closely at all of the dialects if one is interested in elucidating the earliest stages of Armenian.

Some interesting archaisms that surface in Armenian dialects include:

\begin{exe} \ex
(20) archaisms in modern Armenian dialects
a.	the խ χ in Zok կախց kɑχtsh ‘milk’ (cf. կաթ kɑth/gɑth in all other forms of Armenian, including Classical), which may be the reflex of the original l that can be seen in Greek galakt- ‘milk’;
b.	the voiced aspirates in group 1 and 2 dialects, which some scholars believe to directly preserve the original Indo-European voiced aspirates (cf. Garrett 1991 for discussion);
c.	the Classical present tense formation (բերեմ bɛɾɛm, as opposed to բերում եմ bɛɾum ɛm, etc.) in Aramo and some dialects in Iraq;\footnote{ According to Garibian 1962:4 and Jahukyan 1972; Gharibyan states elsewhere, however, (Gharibyan 1958) that Aramo forms the present with the prefix \textit{ha}.}
d.	the Karabagh interrogative hu ‘who’, which according to Adjarian preserves the original v-less form found in Classical Armenian ո ɔ (all other dialects have added a -v);
e.	the u-conjugation in Homshetsma and several other dialects (it was lost in both literary dialects; (Jahukyan 1972’s feature \#77));
f.	χendɑtshnuʃ ‘make someone rejoice’ in Homshetsma, preserving the original semantics of Classical χndɑl ‘rejoice’ (in all other forms of modern Armenian it now means ‘laugh’);
g.	the preservation of hɑw in the meaning ‘bird’ in the dialects of Mush, Tbilisi, and Van (in all other dialects, including SEA and SWA, its meaning has become limited to ‘chicken’);
h.	the two series of mid vowels in New Julfa, Mush, and several other dialects (the literary dialects have merged the two series);
i.	the medial -ɑ- in penultimate stress dialects such as Karabagh and Agulis (Middle Armenian and the modern final stress dialects normally deleted all medial ɑ’s; contrast Agulis hɾsænikh with Classical hɑrsɑnikʰ and SWA hɑɾsnikh);
j.	the z- definite/accusative in Mush, Kesab, and several other dialects, preserved more or less intact from its Classical predecessor but lost in the literary and most other dialects (Jahukyan 1972’s feature \#72);
k.	the Classical -kh plural, preserved with vowel-final stems in New Julfa and a few other dialects.
}

\subsubsubsection{ Chronology of historical changes}
With an appropriate set of linguistic innovations and archaisms in hand, one can begin constructing a historical dialectology. Jos Weitenberg has carried out pioneering work on this topic (cf. Weitenberg 2001 for instance), so here I will just outline some of the most basic facts.

When developing a chronology of linguistic changes, it is to avoid \textit{argumenta ex silentio} and to use only positive evidence. As philosophers and scientists often put it, absence of evidence is not evidence of absence; the fact that we do not have evidence for a given phenomenon at a given point in time and space does not mean that it did not exist there and then. To take a simple Armenian example, the fact that we do not have any materials written in Armenian before the fifth century AD does not entail that the language did not exist before that time. Similarly, the texts and inscriptions that we \textit{do} have for the subsequent 1500 years are likely only a small fragment of what was produced during that time, the rest having been lost or destroyed, and the totality of written material is in turn an infinitesimal part of the set of Armenian utterances produced during that time, if one includes the spoken language. For this reason it is important to avoid arguments such as “we have no evidence of Armenian dialects before the fifth century, therefore there were none”, or “my Hamshen informants use hɑv for ‘chicken’, not ‘bird’, therefore the semantic change of ‘bird’ to ‘chicken’ happened before the development of the Hamshen dialect”. Instead we must build our arguments strictly on positive evidence, such as “we can infer with a reasonable degree of confidence that the change of original Armenian w to v happened by 953, because an inscription at T‘alin from that year contains a <v> erroneously written for <w>”.

With this principle in mind, we can begin to build a fairly reliable picture of the historical development of Armenian and its dialects. We know from the rendition of place names in the early eighth-century text \textit{La Narratio de Rebus Armeniae} (Garritte 1952) that some of the “western” consonant shifts had already taken place by that time, for instance; Weitenberg 1983 identifies more such dialect changes in the Autun Glossary of c. 800 AD. Scribal “errors” in dated early manuscripts sometimes reveal dialect innovations; manuscript colophons are equally instructive (q.v. Jahukyan 1997). A particularly good repository of such errors, both for its age and for its abundance of dialectisms, is the Moscow Gospel (887 AD). In it one finds for example the modern 1st plural verbal suffix -nkh (compare Classical  mkh), and the common dialectal monophthongization of ay > a in final position in polysyllabic words (which we find in SWA as well). 

Inscriptions are also useful for historical dialectology (q.v. Arakelyan et al 1960-1982). For instance, the change of word-final այ ɑj > ա ɑ occurs in an inscription at Gndevank‘ from 931, not longer after the Moscow Gospel was produced. The monophthongization of preconsonantal աւ ɑw > օ ɔ and post-nasal voicing occur on a khach‘k‘ar at Gaṛni from 879, which reads ի բարէխոսութի ինծ i bɑɾeχɔsutʰi ints for ի բարէխաւսութի[ւն] ինձ i bɑɾeχɑwsutʰi[wn] indz. (The final -i for -iwn is a scribal convention; the form ինծ for ինձ probably represents a pronunciation [indz] rather than [ints], because Armenian dialects generally do not devoice stops after nasals, but instead suggests that in the dialect of the inscription writer the pronunciation of ts had become voiced and that of dz had become voiceless, as in Western Armenian.) Some more examples: the change of preconsonantal յ ɑj > է ɛ found in many modern (primarily eastern) dialects appears in the aforementioned 953 T‘alin inscription; the modern 1st singular aorist \textit{tvi} ‘I gave’ (compare Classical etu) surfaces in an inscription at Bagnayr in 1042; the modern -ɛɾ plural (compare Classical -kh) appears at Vanevan in 903.

 

Even when one does not have good dates for linguistic changes, it is sometimes possible to infer roughly when they happened from their relationship to other linguistic changes, as we saw earlier with stress shift and medial a-deletion. One can employ external evidence as well: Achaṛyan 1952 for example dated the sound law named after him to between the seventh and eleventh centuries, based on the fact that Arabic loans (seventh century) undergo the rule but Turkish loans (eleventh century and following) do not.

Another helpful type of external evidence is dated population movements: dialects with well-documented migration histories can be used to determine termini ante quos for certain linguistic innovations. Thus, for example, we know exactly when the New Julfa community was forcibly moved from Julfa in Nakhichevan to Isfahan in Iran (1605-6), and similarly for Burdur (moved from Karabagh in 1610), Akhalts‘kha  and Gyumri (moved from Erzerum in 1828), and New Nakhichevan (moved from the Crimea in 1779). With this information in hand, if we see an innovation shared by one of these transplanted dialects and its former, but not current, neighbors we can say with some certainty that the innovation occurred sometime before the dialect moved.\footnote{ Though we must also bear in mind the possibility that the transplanted dialect and its original neighbors may actually preserve an archaic feature that all other dialects have lost, in which case the \textit{terminus ante quem} would be meaningless. This possibility can normally be discounted, however.} For example, Burdur employs the characteristic Eastern present formation of a locative participle plus forms of the copula, indicating that this construction had arisen by the time the Burdur community left Karabagh.\footnote{ Mkrtch‘yan 1971:21 lists a number of additional \textit{-um} dialects in western Turkey: Antalya, Bolu, Denizli, Diyner, Dovrek, Duzce, Elmali, Ereyli, Gasaba, Isparta, Kirk-Aghach, Nazilli, Ödemiş, Punikia, and Zonguldak.} The same reasoning can be used for the consonant shifts, vowel harmony system, and penultimate stress system that Burdur shares with Karabagh. Note that it is not so safe to draw conclusions from features that Burdur does \textit{not} share with Karabagh. For example, Burdur differs from Karabagh in not having the -ɑkɑn future, not assigning its nominal plurals to the -i- conjugation, and lacking the characteristic Karabagh vowel ǝe (which in the Karabagh varieties I have listened to is a lax [ɛ] preceded by a velarized consonant). Recall from our earlier discussion the maxim that only non-trivial innovations can be used for historical dialectology; these differences between Karabagh and Burdur may be \textit{losses} that Burdur developed as a result of contact with western dialects, rather than \textit{innovations} that Karabagh developed after the Burdur community left.

Given the almost total lack of dialect information predating the mid-nineteenth century, chronologies established in this way can be invaluable. A particularly interesting case involves Adjarian’s Law, which shows up primarily in the southeastern extremes of the Armenian world (Agulis, Karabagh, Karchevan, Kṛzen, Maragha, Meghri, Salmast, Shamakhi, Shatakh, Syria, Van, Varhavar, and Khoy), but also surfaces farther west, in the non-contiguous areas villages of Musaler and Malatia (Vaux 1998). Since Adjarian’s Law is such a non-trivial innovation—it involves fronting vowels after original voiced obstruents—one suspects that these two dialects did not develop it independently, but rather inherited it from a linguistic ancestor shared with Karabagh, Van, and the other Adjarian’s Law dialects. Though we do not have much information on the origins of the Armenian community in Malatia, there is some evidence that the Musaler community was founded in part by immigrants from the Karabagh region (Andreasyan 1967), which dovetails nicely with the fact that the peculiar vowel shift they underwent appears to be related to the ones found in Karabagh and Agulis (Vaux 1998).

\subsection{Methodology and data collection}

Despite the positive leads just discussed, our understanding of the history of Armenian and its dialects is far from complete. Most historical materials are not yet available in electronic form, and we need more extensive documentation of the surviving dialects. Given the paucity and advanced age of dialect speakers, data collection should in my opinion be the most immediate concern of Armenian dialectologists. For those who accept this call, there are two basic questions to be addressed: what subset of the large store of information to target and how to elicit this information efficiently and accurately. 

We already saw in the case of the dialectologists converging on the elderly man from Ashan that there are problems in the ways in which Armenian dialect fieldwork is currently conducted, and I suggested that this reflects the nineteenth-century predilection for finding pure dialect forms and misses out on advances in interviewing technique made since the 1960s. One can alleviate this to a certain degree by designing elicitation materials that limit the ways in which the fieldworker using it can go astray, and that make it maximally simple for informants to provide useful and reliable responses. 

The only such item currently available is Muradyan et al.’s excellent \textit{Program for the Collection of Materials for an Armenian Dialectological Atlas} (1977). I have found this manual to be extremely useful in conducting fieldwork; for instance, the words are well chosen in the sense that speakers generally know many of them and find them interesting, and they yield a high percentage of dialect forms (whereas words elicited at random typically end up being no different than in the standard language). The forms that the authors provide as typical answers are drawn from a wide range of dialects, so that speakers typically recognize one or more of them. (As the authors correctly identified, it is sometimes necessary to present dialect speakers with a list of possibilities when they are drawing a blank on the meaning in question.) The vocabulary section and the T‘umanyan text at the end of the manual are especially well chosen; the phonological and morphological parts are harder to collect successfully from speakers, though this is largely not the fault of the authors.

Typical entries in the lexical component of the manual look like this:


\begin{exe} \ex 
	(21)	Questions 14, 496, and 778 from Muradyan et al. 1977
	a. 14. Ստվեր – շուք, շօք, շօխկ – շըվաք – հօվ…
	Ծերունիները նստած էին պատի ստվերում։
	Ծառի ստվերը երկարեց։
	14. 	stvɛɾ - ʃukʰ, ʃɔkʰ, ʃɔχk - ʃǝvɑkʰ - hɔv…
	
	b. 496. Տղա – 1. Արու զավակ։ 2. Երեխա։…
	496. təʁɑ 1. ɑɾu zɑvɑk   2. jɛɾɛχɑ  
	
	Bert: ‘male child’,2. ‘child [no gender specified]’
	c. 778. կրեայ (կրիա, կուրիա, կիրիրա…)
	kǝɾjɑ (kɾiɑ, kuɾiɑ, kiɾiɾɑ...)  
	bert: ‘turtle’
	
\end{exe}


(21a) is drawn from the section of meanings expressed by different words across dialects. It presents the SEA word for an item (e.g. stvɛɾ ‘shadow’), followed by a set of common dialect equivalents (ʃukh, hɔv, etc.). There are many ways to elicit the forms in this section, but I generally find it most efficient to simply ask the speaker what the word for ‘shadow’ is in their dialect, and so on. It is normally better to prompt informants in another language (be it Russian, English, or any other language they are comfortable with) rather than Armenian, because the Armenian form may influence their recall of the relevant dialect forms. Each headword in this section of the manual is followed by one or two sentences designed to illustrate how the word is used, which is a good idea in theory but less than perfect in practice, because the sentences constructed by the authors are often too short and general to convey to the informant what the keyword means.

(21b) represents the section on semantic variation, wherein the authors provide words that vary in meaning across dialects. In the example here, the word tʁɑ means ‘male child’ in some dialects, and ‘child’ (unspecified for gender) in others; informants are supposed to state which of the meanings (or both, or neither) they use in their dialect.

(21c) comes from the section in the manual featuring words whose phonology differs from dialect to dialect. In the entry provided here, the word for ‘turtle’ various in the details of its pronunciation, from [kǝɾjɑ] in SEA to [kuɾiɑ] in New Julfa, and so on.

All of these sections work quite well with most informants. However, it would be a simple matter to enhance the manual in ways that would make the elicitation process easier and even more successful. For example, the current manual is presented entirely in SEA written in Armenian script. Many dialect speakers do not know the Armenian script, though, and many more do not understand SEA. One of the largest communities of this type is the Hemshinli along the Black Sea in northeastern Turkey; in my fieldwork I have had to convert the questions into Turkish before presenting them to my informants. In order to cover the range of linguistic abilities possessed by Armenian dialect speakers, we should augment the current manual with translations into Russian, English, French, Turkish, and Arabic. It is also extremely helpful to include pictures for every entry, to the extent that this is possible (with some abstract concepts it would be difficult, but the current manual does not contain many such items). A revised manual would ideally also contain sections that elicit syntactic constructions and free/casual connected speech.

As discussed above, Armenian dialectologists have thus far concentrated on the elicitation and collection of word lists and texts, comparing these to Classical Armenian in the hopes of discovering archaic lexical items and constructions. This course of research reflects the preoccupation of nineteenth-century philologists with linguistic history, notably the Neogrammarians’ search for pure dialect forms to confirm their belief in the exceptionlessness of phonological rules. The Neogrammarians believed that a given sound change operated over a specific area, affecting every word whose phonological structure met the structural description of the change. They felt that the regularity of sound change would be clearer in nonstandard dialects than in a standard language, which in their view was more likely to have been subject to a variety of external influences. 

Though I agree with the fundamental insight of the Neogrammarian doctrine, I feel that the program of investigation that arose from this philosophy has two serious shortcomings. First of all, it pays little heed to the syntactic component of language, which has become a central concern of linguists in the last sixty years (more on this below). Secondly, it fails to recognize the language of an individual dialect speaker as an autonomous grammar with internal structure. The latter point has two important implications.

First of all, structuralists beginning with Saussure realized that the internal structure of a linguistic system plays a role in its synchronic and diachronic behavior. This idea descends from Plato’s postulation that the value of elements in a group derives from their relationships to other elements in the group: a person, for example, is good only by virtue of comparison to some other or ideal person. Saussure extended this idea to language, asserting that linguistic units derive their value from their oppositions to other linguistic units in a given linguistic system. In this view, the roundness of a vowel such as o has significance only by virtue of the existence in the same system of non-round vowels such as e. Following this line of reasoning, any change in a linguistic system necessarily changes the network of oppositions and thereby affects the fundamental status of each element within that system.

The Platonic theory of oppositions has occupied a central role in linguistic theory since the time of Saussure, continuing with the structuralists of the middle of this century and the generativists of recent decades. Nevertheless, there is little or no attempt to identify such structures in existing studies of Armenian dialects, carried out for the most part by disciples of Achaṛean, who as we saw earlier trained with Meillet in Paris two decades before the publication of Saussure’s revolutionary \textit{Cours de Linguistique Générale}. Even such rudimentary elements of linguistic structure as phonemic contrasts and allophonic distribution are almost universally ignored, with a few notable exceptions (e.g. Pisowicz 1969, Haneyan 1978, and Khach‘atryan 1988). In order to establish these structures properly, we need access to the whole range of a speaker’s competence, including the significant proportion of borrowings from neighboring languages, not merely the 1500 or so descendants of Classical Armenian words typically found in the dialect grammars. We also need to survey the possible consonant and vowel sequences, the inventories of stressed and unstressed vowels, the placement of epenthetic vowels, and so on. If a dialect has vowel harmony, for example, the field worker should establish how this system actually operates synchronically. Existing studies of vowel harmony dialects such as Agulis and Karchevan content themselves with establishing harmonic vowel sets, not considering questions of neutral vowels and consonants, the domain and direction of application, and other issues of vital interest to contemporary linguists. It is only with great difficulty that this type of information can be inferred from existing grammars.

The second implication is that it is important to ascertain the personal histories of the speakers studied: what dialects they have been in contact with, where their ancestors lived, and so on. Achaṛean regularly provided the ages and names of his principal informants but apparently did not inquire beyond this point. It is not clear how much information of this type will be elicited in the Institute’s new project. Without such personal information, it is significantly more difficult to establish what forms and constructions have been borrowed from or influenced by particular dialects and languages. Fieldworkers should also record the social status and profession of all informants, as well as the social context in which the interview takes place. A well-known dialectological study in America (Labov 1972) demonstrated that speakers typically control a range of linguistic registers and that they make a choice according to their assessment of the social status of the interlocutor. A particularly unfortunate manifestation of this phenomenon regularly occurs with Armenian informants, who, if they judge the interviewer to be an intellectual or a foreigner, tend to speak in standard Armenian, even when they are native speakers of nonstandard dialects. Noted phonetician Amalya  Khach‘aturian, a niece of Achaṛean, once told me that Achaṛean’s favorite method of eliciting dialect data from informants was to pretend to be a speaker of the dialect in whatever village he was visiting at the time. Of course this normally is not a feasible plan for most field workers. Therefore, it is important to have some idea of informants’ perception of their status relative to the interviewer so that we may have some idea of the linguistic register they have chosen.

One last note on the elicitation of data. My first linguistics professor once stated that linguistics is not concerned with what does \textit{not} occur, only with what does. The strong interpretation of this view, characteristic of pre-structuralist linguists and of Armenian dialectologists as well, not only runs counter to current wisdom but is also detrimental to our knowledge of Armenian dialects. In the domain of syntax, for example, it is vital to know what variations of a sentence are possible, impossible, or somewhere between these extremes. It is not sufficient to know that ‘I ate apples yesterday’ is an acceptable English sentence; we must know also how the permutations ‘yesterday I ate apples’, ‘I yesterday ate apples’, ‘I ate yesterday apples’, and so on are judged and what makes them (un)acceptable to a given speaker.

One of the reasons this type of information has been absent from work on Armenian dialects (in addition to the amount of work it entails) is that many Armenian linguists believe it ‘obvious’ in view of the supposedly parallel behavior of standard Armenian. This suffers from two problems: not only does the failure to be explicit often conceal subtle differences, but also the inevitability of linguistic change virtually guarantees that parallels in the standard language which are ‘obvious’ now will be inscrutable and unrecoverable at some point in the future. For example, we are already unsure what norms Achaṛean was referring to when he labeled certain dialectal constructions similar to the standard language; the Standard Armenian of 2016 differs significantly from the Armenian acquired by Achaṛean in the late nineteenth century, a period when the language was still taking in large amounts of vocabulary and grammar from non-standard dialects (consider for example the influence on the standard language of such writers as Abovyan (K‘anak‘eṛ dialect) and Sundukyan (T‘iflis dialect). 

Let us turn now to the problem of how to elicit data from dialect speakers efficiently and accurately. Due to the limited number of surviving dialect speakers, Armenian dialectologists do not face the same sociological and statistical tasks that preoccupy most current Western dialectologists. We saw earlier that when one is studying the varieties of English spoken in New York City, for example, sociolinguists aim to survey representative samples of men and women of different ages, ethnic groups, social classes, and so on in order to ensure an unbiased survey. When dealing with Armenian dialects, however, the number of available speakers is normally so small that one cannot pick and choose in this way. For the same reason, the problem of selecting a method of data collection is not of concern to Armenian dialectologists, as it is to many others. Linguists working with dialects with large numbers of speakers with access to telephones and the internet must weigh the relative merits of options such as \textit{postal surveys}, which provide extensive and simultaneous coverage, but sacrifice quality control and limit the amount of data received from each informant; \textit{telephone interviews}, which provide less coverage and convenience but greater quality control and phonetic detail; \textit{internet questionnaires}, which allow quicker and more extensive coverage but tend to attract primarily university-age respondents; and \textit{direct interviews}, which allow for the most detailed and controlled data collection, but sacrifice coverage and convenience, and do not necessarily provide a picture of the dialect in question at a single point in time, since it may take many years to complete all of the interviews. In the case of most Armenian dialect work, however, the only feasible option is the direct interview.

Once an interview has been recorded, linguists should aim to meet certain minimal requirements in transcribing it. Utterances should be transcribed in a scientific manner that represents all details that might be of interest; a good example of Armenian transcription can be found in Allen’s 1950 description of New Julfa dialect. Existing transcriptions of Armenian dialects typically render all forms in either standard Armenian orthography, thereby losing most details of phonetic interest, or a modified phonemic system based on standard Armenian orthography, in which features such as the palatalization and (de)voicing of consonants or the umlauting of vowels is represented, but placement of stress and schwas are left for readers to extrapolate based on their knowledge of standard Armenian. 

The first alternative is unsatisfactory in many respects, as has been recognized since Achaṛean’s time. But the second too is flawed, for two reasons. First, it restricts the appeal of Armenian dialectology to those who can extrapolate key information from standard Armenian, thereby excluding all but a very small number of linguists. In light of this problem, it is perhaps little wonder that Armenian has rarely been studied by theoretical linguists, though neighboring languages such as Turkish and Arabic have played central roles in linguistic theory over the past few decades. For example Kenstowicz 1994, the standard summary of contemporary work in phonological theory, devotes seven pages to Turkish and almost 100 pages to Arabic, but does not mention Armenian. 

Second, the problem with not representing schwas and stress is that these are not nearly as predictable as those who omit them seem to think. The rules governing schwa placement are not simple or obvious, and the position of epenthetic vowels can be variable. Thus, for example, a word such as խօսեցնել χɔsɛtsʰnɛl ‘cause to speak’ may be pronounced [χɔsɛtsʰnɛl] by one speaker and [χɔsɛtsʰənɛl] by another (or the same speaker may produce both forms on separate occasions); representing such a form in standard orthography as խօսեցնել fails to convey which pronunciation (if either) the informant has chosen. The same argument holds for the representation of stress contours: for example, one speaker may pronounce ‘especially’ as [mɑ́nɑvɑnd] whereas another says [mɑnɑvɑ́nd]. This type of information is important to the linguist, yet is rarely conveyed in studies of Armenian dialects.

Existing works on Armenian dialects also do not normally translate the texts they provide or to give meanings of words that occur elsewhere in the book or article. This is often justified with the claim that the meanings of the forms and texts are obvious, but inspection of any dialect text quickly reveals that this is not the case. During the course of translating Achaṛean’s grammar of the New Julfa dialect, for example, I encountered scores of words and phrases that defied interpretation. In addition, the appeal to ‘obviousness’ commonly conceals an inability on the part of the author in question to understand the form or passage at hand. Even when the significance of a form seems obvious, one should bear in mind that when we try to make explicit that which seems obvious, we often find that interesting subtleties emerge.


\section{What Can (or Should) Be Done?}

With these methodological and theoretical preliminaries out of the way, let us now consider some of the more practical aspects of how to implement the desiderata we have outlined. First, it is desirable to develop an online dialect archive accessible to all interested individuals, containing analyzed audio and video, bibliography, an accessible database of vocabulary, contact information for fieldwork projects, online versions of rare dialect books and texts, and so on. (Anaïd Donabédian at INALCO in Paris has the beginnings of such a project.) Making resources available in this way can increase the pool of talented people willing and able to analyze the data, which is too much for the current set of scholars to handle on their own. Dialect forms should be transcribed in the IPA; in addition to conveying all the relevant nuances of dialect pronunciation, as discussed earlier, this makes the Armenian data comparable to those of other languages, and hence of use and interest to the linguistic community as a whole. Last, but not least, we should strive to save as many Armenian dialects as possible.

One may ask whether it is worth saving the endangered dialects of Armenian. I have already argued that language is an essential part of culture; without the Hamshen language, for example, we cannot fully appreciate the culture of the Hamshen Armenians. It is also important to know something about these dialects in order to appreciate Armenian literature. Literary Armenian was not created ex nihilo, nor does it exist in a vacuum; it draws its strength from the dialectal sources from which it was derived. Consider for example the influence on the standard language of such writers as Sundukyan, who wrote in Tbilisi dialect, Patkanean (Nor Nakhichevan), or Shirvanzade (Shamakhi). Finally, by preserving these varieties of Armenian, we also preserve the oral literature, songs, games, and traditions that are the backbone of Armenian culture.

So, what can we do to save the endangered dialects of Armenian? Two possibilities come immediately to mind. The first would be to establish child care centers where the elders, who are the repository of our culture’s language and traditions, care for the children, who can acquire languages effortlessly. The second would be to incentivize the study of the Armenian language. At present, students do not perceive Armenology to be a viable course of study, because there are so few jobs available. By the same token, universities are unlikely to create such jobs when there are no students. The Turkish government has seen the way to break this vicious circle, by enticing universities to create professorships in Turkish Studies. The only way that universities will offer Armenian courses is if they are provided with funding for positions by the Armenian community. The only way that students will take these courses in significant numbers is if they feel that they have a good chance of getting a job; we therefore should aim to create a critical mass of positions. Establishing college minors in Armenian Studies should also help greatly.

Supporting Armenian studies can have other benefits as well. Consider, for example, the aforementioned fact that Armenian has rarely been studied by theoretical linguists, though neighboring languages such as Turkish and Arabic have played central roles in linguistic theory over the past few decades. Why is this? The perceived unviability of Armenian studies that I mentioned earlier is largely responsible. Without a critical mass of students entering the field of Armenology, Armenian will continue to be ignored in the fields to which it can and should contribute. Once that critical mass is reached, however, the field can mushroom again like it did in the time of Meillet, and it is only a matter of time before the next Adjarian, who had previously been considering law school, decides instead to major in Armenology.

